
\documentclass[version=last,
    paper=a4, % paper size
    10pt, % default font size
    usenames,
    dvipsnames,
    oneside, % ONLINE
    headings=openany, % open chapters on odd and even pages
    %toc=chapterentrywithdots, % Table of Contents style
    %BCOR=7mm, % PRINT Binding Correction
    %DIV=13, % typearea 161.54 mm x 228.46 mm, top margin 22.85 mm, inner margin 16.15 mm
    %DIV=14, % 165.00 233.36 21.21 15.00
    DIV=15 % 168.00 237.60 19.80 14.00
]{scrbook}
\usepackage{typearea}
\usepackage[automark,headsepline,footsepline]{scrlayer-scrpage} % Headers and footers

%%
%% Fonts, encoding, spacing, indentation
%%
\usepackage{txfonts}
\renewcommand{\familydefault}{\sfdefault} % Default to Sans Serif font
\usepackage[english]{babel}
\usepackage[T1]{fontenc}
\usepackage[utf8]{inputenc}

% Paragraph spacing
%\usepackage{parskip}    % Paragraph spacing
%\setlength{\parindent}{0em} % Don't indent paragraphs - ONLINE
%\setlength{\parskip}{1.3 ex plus 0.5ex minus 0.3ex} % 1-1.8 ex vertical space between paragraphs - ONLINE

% Spacing of headings
%\RedeclareSectionCommand[afterskip=3pt]{section} % less space after section
%\RedeclareSectionCommand[beforeskip=0cm]{subsection} % less space between HRule and project name
%\RedeclareSectionCommand[afterskip=0.1\baselineskip]{subsubsection} % less space after progressreport subheadings

% Table font size
\usepackage{etoolbox}
\AtBeginEnvironment{longtabu}{\footnotesize}{}{}

%%
%% Tables, columns, layout
%%
\usepackage{multicol}   % 2 col publications
\usepackage{pdflscape}  % Landscape pages
\usepackage{pdfpages}   % Include PDFs
\usepackage{hanging}    % hanging paragraphs for publications
%\usepackage{titletoc}   % Required for manipulating the table of contents
\setcounter{tocdepth}{2} % TOC list down to section
\usepackage{enumerate}  % Enumerations
\usepackage{enumitem}   % Enumerations
\usepackage{longtable}  % Multipage table
\usepackage{tabu}       %
\setlength{\tabulinesep}{1.5mm} % Consistent vertical spacing in tabu

%%
%% Graphics, images, colours
%%
\usepackage{graphicx} % embedded images
\usepackage{eso-pic} %
\usepackage{colortbl} % define custom named colours
\definecolor{RedFire}{RGB}{146,25,28}
\definecolor{ParksWildlife}{RGB}{0,85,144}
\definecolor{successbg}{RGB}{223,240,216}
\definecolor{errorbg}{RGB}{242,222,222}
\definecolor{warningbg}{RGB}{252,248,227}
\definecolor{infobg}{RGB}{217,237,247}
\definecolor{muted}{RGB}{153,153,153}
\definecolor{success}{RGB}{70,136,71}
\definecolor{error}{RGB}{185,74,72}
\definecolor{warning}{RGB}{192,152,83}
\definecolor{info}{RGB}{58,135,173}

\definecolor{required}{RGB}{192,152,83}
\definecolor{requiredbg}{RGB}{252,248,227}
\definecolor{denied}{RGB}{185,74,72}
\definecolor{deniedbg}{RGB}{242,222,222}
\definecolor{granted}{RGB}{70,136,71}
\definecolor{grantedbg}{RGB}{223,240,216}
\definecolor{not reqiured}{RGB}{153,153,153}
\definecolor{not requiredbg}{RGB}{255,255,255}

\usepackage{tikz} % Drawing
\usetikzlibrary{arrows,shapes,positioning,shadows,trees}

%%
%% Links, URLs
%%
\usepackage[
    linktoc=all,
    %colorlinks=false,  %PRINT
    colorlinks=true, % ONLINE
    linkcolor=RedFire, % ONLINE
    urlcolor=ParksWildlife, % ONLINE
    pdftitle=Progress Report STP 2016-001 (FY 2014-2015)
]{hyperref}

% Black magic to linebreak URLs
\usepackage{url}
\makeatletter
\g@addto@macro{\UrlBreaks}{\UrlOrds}
\makeatother

%%
%% Custom macros
%%
% Thick Horizontal rule
\newcommand{\HRule}{\vspace{8mm}\\\noindent\rule{\linewidth}{0.1pt}}

% Custom Tikz node for SDS diagram
\newcommand\mynode[6][]{
    \node[#1] (#2){
        \parbox{#3\relax}{
            \begin{center}
            \textbf{#4}\\
            #5\\
            \footnotesize{#6}
            \end{center}}};}



\usepackage[automark,headsepline,footsepline,plainfootsepline]{scrlayer-scrpage}
\automark*[section]{}
\addtokomafont{pageheadfoot}{\normalfont\footnotesize\sffamily} % Don't italicise
\renewcommand{\chaptermark}[1]{\markleft{#1}{}}     % Chapter: suppress numbering
\renewcommand{\sectionmark}[1]{\markright{#1}{}}    % Section: suppress numbering

% Header (inner, center, outer)
\ihead{\href{http://sdis.dpaw.wa.gov.au/documents/studentreport/1521/}{Progress Report STP 2016-001 (FY 2014-2015)}}
%\chead{\href{http://sdis.dpaw.wa.gov.au}{Science Directorate Information System}}
\ohead{\href{https://www.dpaw.wa.gov.au/about-us/science-and-research}{\includegraphics[height=6mm, keepaspectratio]{/mnt/projects/sdis/staticfiles/img/logo-dpaw.png}}}

% Footer (inner, center, outer)
\ifoot{\textbf{Printed}~Mon, 10 Jul 2017 17:22:33 +0800} % inner/left footer
\cfoot{}
\ofoot[\bfseries\thepage]{\bfseries\thepage}        % Page number (also [plain])


\pagestyle{scrheadings}
\setkomafont{pageheadfoot}{\normalfont}

%-----------------------------------------------------------------------------%
\begin{document}
\raggedbottom

%-----------------------------------------------------------------------------%
% Title page
\subject{Progress Report STP 2016-001 (FY 2014-2015)
}
\title{Vegetation responses to Noongar land management practices in old and
young landscapes
}
\subtitle{Plant Science and Herbarium
}
\author{}
\publishers{\small
    \subsection*{Project Core Team}
\begin{tabu} {X X}
\textbf{Supervising Scientist} & Margaret Byrne
\\
\textbf{Data Custodian} & Margaret Byrne
\\
\textbf{Site Custodian} & 
\\
\end{tabu}


    \subsection*{Project status as of July 10, 2017, 5:22 p.m.}
\begin{tabu} {X X}
& Approved and active
\\
\end{tabu}

    
\subsection*{Document endorsements and approvals as of July 10, 2017, 5:22 p.m.}
\begin{tabu} {X X}

%\rowcolor{grantedbg}
    \textbf{Project Team} & 
    \textcolor{granted}{ granted}\\

%\rowcolor{grantedbg}
    \textbf{Program Leader} & 
    \textcolor{granted}{ granted}\\

%\rowcolor{grantedbg}
    \textbf{Directorate} & 
    \textcolor{granted}{ granted}\\

\end{tabu}



}
\uppertitleback{}
\lowertitleback{}
\date{}

%-----------------------------------------------------------------------------%
% Front matter
\frontmatter
\maketitle
%-----------------------------------------------------------------------------%
% Main matter
\mainmatter

\section*{Vegetation responses to Noongar land management practices in old and
young landscapes
}

M Byrne



\section*{Progress Report}

This project aims to improve our understanding of the nexus between
Noongar culture and contemporary floristic distributions, ultimately
informing our understanding of South-west Australian Floristic Region
(SWAFR) and connection of Noongar people with country.~ Through
collaborative research with southern coastal Noongar Elders and
archaeologists, the project explores how specific cultural activities
may have influenced biodiversity within the SWAFR through a series of
investigations examining vegetation patterns and processes.~ Specific
areas of focus include testing for patterns in Noongar plant resource
concentration at cultural nodes and botanical differentiation between
nodes of varying historic cultural use; phylogeographical investigation
of culturally important taxa in the genus \emph{Platysace};
investigations of resource concentration effects of
edible\emph{Platysace} tuber harvest; and an analysis of Noongar land
use patterns in relation to landscape age and propensity to disturbance.

The phylogeographical work is testing the hypothesis that humans are a
historical vector influencing the distribution of the staple Noongar
food species, \emph{P. deflexa} and \emph{P.~trachymeniodes}.~ To date,
a total of 793 individuals of \emph{P. deflexa},\emph{P.
trachymenioides}, \emph{P. effusa}, \emph{P. maxwellii}, \emph{P.
compressa} and \emph{P. juncea} have been sampled at 109 sites located
in inland areas southwest of a line from Northampton to the Cape Arid
National Park. DNA extraction from leaf matter is almost complete, and
analysis has commenced of three noncoding chloroplast DNA regions,
comprising trnS-trnG52S, trnV-ndhc and psbD-trnT. DNA extraction,
polymerase chain reaction and sequencing has been carried out on
approximately half of the selected sampled individuals, with planning
underway to complete the remaining samples in July 2017.



%-----------------------------------------------------------------------------%
% Back matter
%\backmatter
\end{document}
%-----------------------------------------------------------------------------%

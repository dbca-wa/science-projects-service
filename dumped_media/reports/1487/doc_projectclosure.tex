
\documentclass[version=last, paper=a4, DIV=18, usenames, dvipsnames]{scrartcl}
\usepackage{txfonts}
\usepackage{pdflscape}
\usepackage{pdfpages}
\usepackage[english]{babel} % English language/hyphenation
%%% Bootstrap colors
\definecolor{RedFire}{RGB}{146,25,28}
\definecolor{ParksWildlife}{RGB}{0,85,144}
\definecolor{successbg}{RGB}{223,240,216}
\definecolor{errorbg}{RGB}{242,222,222}
\definecolor{warningbg}{RGB}{252,248,227}
\definecolor{infobg}{RGB}{217,237,247}
\definecolor{muted}{RGB}{153,153,153}
\definecolor{success}{RGB}{70,136,71}
\definecolor{error}{RGB}{185,74,72}
\definecolor{warning}{RGB}{192,152,83}
\definecolor{info}{RGB}{58,135,173}

\definecolor{required}{HTML}{D9534F}
\definecolor{denied}{HTML}{D9534F}
\definecolor{granted}{HTML}{47A447}
\definecolor{not required}{RGB}{200, 200, 200}

\usepackage[colorlinks=true,pdftitle=doc\_projectclosure.pdf
,linktoc=all,linkcolor=RedFire,urlcolor=ParksWildlife]{hyperref}
\usepackage{colortbl}
\usepackage{longtable}
\usepackage{tabu}
\setlength{\tabulinesep}{1.5mm}
\usepackage{enumerate}
\usepackage{enumitem}
\usepackage{fancyhdr}
\usepackage{lastpage}
\usepackage{graphicx}
\usepackage{eso-pic}
\usepackage{hyphenat}
\renewcommand{\familydefault}{\sfdefault}



\newcommand{\HRule}{\rule{\linewidth}{0.1pt}}

\newcommand{\placetextbox}[3]{% \placetextbox{<horizontal pos>}{<vertical pos>}{<stuff>}
  \setbox0=\hbox{#3}% Put <stuff> in a box
  \AddToShipoutPictureFG*{% Add <stuff> to current page foreground
    \put(\LenToUnit{#1\paperwidth},\LenToUnit{#2\paperheight}){\vtop{{\null}\makebox[0pt][c]{#3}}}%
  }%
}%




%-----------------------------------------------------------------------------%
% Headers and footers
%
\fancypagestyle{plain}{
  \fancyhf{}
  \setlength\headheight{60pt} % push page content below header
  \renewcommand{\headrulewidth}{0.1pt}
  \renewcommand{\footrulewidth}{0.1pt}
  
  
  \fancyhead[L]{ 
    \href{http://sdis.dpaw.wa.gov.au}{
    \includegraphics[scale=0.6]{/mnt/projects/sdis/staticfiles/img/logo-dpaw.png}}
  }
  \fancyhead[R]{ 
      \hfill
      \href{http://sdis.dpaw.wa.gov.au}{Science Directorate Information System} 
      \newline 
      \href{http://sdis.dpaw.wa.gov.au/documents/projectclosure/1487/}{Project Closure 1998-6} 
  }
  
  
  
  
  \fancyfoot[L]{ \leftmark\newline\textbf{Printed}\textit{ June 30, 2015, 2:54 p.m. }}
  \fancyfoot[R]{  \, \newline Page \thepage\ of \pageref{LastPage} }
  
  
}
\pagestyle{plain}
%
% end Headers
%-----------------------------------------------------------------------------%

\begin{document}

%-----------------------------------------------------------------------------%
% Title page
%

%
% end title page
%-----------------------------------------------------------------------------%





\section*{Key publications and documents}
Robinson, R.M. 2003. Short-term impact of thinning and fertilizer
application on Armillaria root disease in regrowth karri
(\emph{Eucalyptus diversicolor} F. Muell.) in Western Australia. Forest
Ecology and Management \textbf{176}: 417-426.

~

Robinson, R.M. and R.H. Smith. 2001. Fumigation of karri regrowth stumps
with metham-sodium to control \emph{Armillaria luteobubalina}.
Australian Forestry \textbf{64}: 209-215.

~

Robinson, R.M., Williams, M. and Smith, R.H. 2003. Incidence of
Armillaria root disease in karri regrowth forest is underestimated by
surveys of above ground symptoms. Australian Forestry \textbf{66}:
273-278.

~

Robinson, R.M. 2005. Volume loss in thinned karri regrowth infected by
Armillaria luteobubalina in Western Australia. In Mańka, M and Łokomy,
P. (Eds). Proceedings of the 11th IUFRO International Conference on Root
and Butt Rots of Forest Trees. Poznań and Białowieża, Poland, August
16-22, 2004. The August Cieszkowski Agricultural University, Poznań,
Poland: 296-303



\section*{Knowledge Transfer}
Robinson, R.M. and Rayner, M. 1998. \emph{Armillaria luteobubalina} in
regrowth karri forests. A report on the state of knowledge of Armillaria
root disease in karri regrowth forests in the southwest of Western
Australia and recommendations for future research. Internal Report.
CALM, Science and Information Division, Como, WA. 22p.

~

Robinson, R.M. 2007. \emph{Armillaria luteobubalina} in regrowth karri
stands in Western Australia. Borers and Rots conference. Institute of
Foresters of Australia.

~

Parks and Wildlife 2014. Silviculture guideline for karri forest. Forest
and Ecosystem Management Division Guideline No. 2.

~

Sub Committee on National Forest Health Annual Reports on the status of
pests, diseases and quarantine. Prepared for the Plant Health Committee.

~

Field days and briefings for FEM personnel and the Independent
Silvicultural Review Panel (2011)





\clearpage



\end{document}

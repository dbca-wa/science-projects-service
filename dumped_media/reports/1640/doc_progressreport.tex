
\documentclass[version=last,
    paper=a4, % paper size
    10pt, % default font size
    usenames,
    dvipsnames,
    oneside, % ONLINE
    headings=openany, % open chapters on odd and even pages
    %toc=chapterentrywithdots, % Table of Contents style
    %BCOR=7mm, % PRINT Binding Correction
    %DIV=13, % typearea 161.54 mm x 228.46 mm, top margin 22.85 mm, inner margin 16.15 mm
    %DIV=14, % 165.00 233.36 21.21 15.00
    DIV=15 % 168.00 237.60 19.80 14.00
]{scrbook}
\usepackage{typearea}
\usepackage[automark,headsepline,footsepline]{scrlayer-scrpage} % Headers and footers

%%
%% Fonts, encoding, spacing, indentation
%%
\usepackage{txfonts}
\renewcommand{\familydefault}{\sfdefault} % Default to Sans Serif font
\usepackage[english]{babel}
\usepackage[T1]{fontenc}
\usepackage[utf8]{inputenc}

% Paragraph spacing
%\usepackage{parskip}    % Paragraph spacing
%\setlength{\parindent}{0em} % Don't indent paragraphs - ONLINE
%\setlength{\parskip}{1.3 ex plus 0.5ex minus 0.3ex} % 1-1.8 ex vertical space between paragraphs - ONLINE

% Spacing of headings
%\RedeclareSectionCommand[afterskip=3pt]{section} % less space after section
%\RedeclareSectionCommand[beforeskip=0cm]{subsection} % less space between HRule and project name
%\RedeclareSectionCommand[afterskip=0.1\baselineskip]{subsubsection} % less space after progressreport subheadings

% Table font size
\usepackage{etoolbox}
\AtBeginEnvironment{longtabu}{\footnotesize}{}{}

%%
%% Tables, columns, layout
%%
\usepackage{multicol}   % 2 col publications
\usepackage{pdflscape}  % Landscape pages
\usepackage{pdfpages}   % Include PDFs
\usepackage{hanging}    % hanging paragraphs for publications
%\usepackage{titletoc}   % Required for manipulating the table of contents
\setcounter{tocdepth}{2} % TOC list down to section
\usepackage{enumerate}  % Enumerations
\usepackage{enumitem}   % Enumerations
\usepackage{longtable}  % Multipage table
\usepackage{tabu}       %
\setlength{\tabulinesep}{1.5mm} % Consistent vertical spacing in tabu

%%
%% Graphics, images, colours
%%
\usepackage{graphicx} % embedded images
\usepackage{eso-pic} %
\usepackage{colortbl} % define custom named colours
\definecolor{RedFire}{RGB}{146,25,28}
\definecolor{ParksWildlife}{RGB}{0,85,144}
\definecolor{successbg}{RGB}{223,240,216}
\definecolor{errorbg}{RGB}{242,222,222}
\definecolor{warningbg}{RGB}{252,248,227}
\definecolor{infobg}{RGB}{217,237,247}
\definecolor{muted}{RGB}{153,153,153}
\definecolor{success}{RGB}{70,136,71}
\definecolor{error}{RGB}{185,74,72}
\definecolor{warning}{RGB}{192,152,83}
\definecolor{info}{RGB}{58,135,173}

\definecolor{required}{RGB}{192,152,83}
\definecolor{requiredbg}{RGB}{252,248,227}
\definecolor{denied}{RGB}{185,74,72}
\definecolor{deniedbg}{RGB}{242,222,222}
\definecolor{granted}{RGB}{70,136,71}
\definecolor{grantedbg}{RGB}{223,240,216}
\definecolor{not reqiured}{RGB}{153,153,153}
\definecolor{not requiredbg}{RGB}{255,255,255}

\usepackage{tikz} % Drawing
\usetikzlibrary{arrows,shapes,positioning,shadows,trees}

%%
%% Links, URLs
%%
\usepackage[
    linktoc=all,
    %colorlinks=false,  %PRINT
    colorlinks=true, % ONLINE
    linkcolor=RedFire, % ONLINE
    urlcolor=ParksWildlife, % ONLINE
    pdftitle=Progress Report SP 2011-005 (FY 2015-2016)
]{hyperref}

% Black magic to linebreak URLs
\usepackage{url}
\makeatletter
\g@addto@macro{\UrlBreaks}{\UrlOrds}
\makeatother

%%
%% Custom macros
%%
% Thick Horizontal rule
\newcommand{\HRule}{\vspace{8mm}\\\noindent\rule{\linewidth}{0.1pt}}

% Custom Tikz node for SDS diagram
\newcommand\mynode[6][]{
    \node[#1] (#2){
        \parbox{#3\relax}{
            \begin{center}
            \textbf{#4}\\
            #5\\
            \footnotesize{#6}
            \end{center}}};}



%-----------------------------------------------------------------------------%
% Headers and Footers
\automark{section}
\ohead{\href{http://sdis.dpaw.wa.gov.au/documents/progressreport/1640/}{Progress Report SP 2011-005
}}
\chead{\href{http://sdis.dpaw.wa.gov.au}{SDIS}} % center header ONLINE
\ihead{\href{http://sdis.dpaw.wa.gov.au}{
        \includegraphics[scale=0.4]{/mnt/projects/sdis/staticfiles/img/logo-dpaw.png}}}
\ifoot{\textbf{Printed}~Tue, 5 Jul 2016 13:25:08 +0800} % inner/left footer
\cfoot{} % center footer
\ofoot{\pagemark} % outer/right footer
\pagestyle{scrheadings}
\setkomafont{pageheadfoot}{\normalfont}

%-----------------------------------------------------------------------------%
\begin{document}
\raggedbottom

%-----------------------------------------------------------------------------%
% Title page
\subject{Progress Report SP 2011-005
}
\title{Ecology and management of the northern quoll in the Pilbara
}
\subtitle{Animal Science
}
\author{}
\publishers{\small
    \subsection*{Project Core Team}
\begin{tabu} {X X}
\textbf{Supervising Scientist} & Judy Dunlop
\\
\textbf{Data Custodian} & Judy Dunlop
\\
\textbf{Site Custodian} & Judy Dunlop
\\
\end{tabu}


    \subsection*{Project status as of July 5, 2016, 1:25 p.m.}
\begin{tabu} {X X}
& Approved and active
\\
\end{tabu}

    
\subsection*{Document endorsements and approvals as of July 5, 2016, 1:25 p.m.}
\begin{tabu} {X X}

%\rowcolor{grantedbg}
    \textbf{Project Team} & 
    \textcolor{granted}{ granted}\\

%\rowcolor{grantedbg}
    \textbf{Program Leader} & 
    \textcolor{granted}{ granted}\\

%\rowcolor{grantedbg}
    \textbf{Directorate} & 
    \textcolor{granted}{ granted}\\

\end{tabu}



}
\uppertitleback{}
\lowertitleback{}
\date{}

%-----------------------------------------------------------------------------%
% Front matter
\frontmatter
\maketitle
%-----------------------------------------------------------------------------%
% Main matter
\mainmatter

\section*{Ecology and management of the northern quoll in the Pilbara
}

A Dunlop, K Rayner


\section*{Context}
The northern quoll \emph{Dasyurus hallucatus} is listed as an threatened
species under the Commonwealth's \emph{Environment Protection and
Biodiversity Conservation Act} 1999. Funding from mining offset
conditions are being used to gain a better understanding of quoll
distribution, ecology, demographics and management requirements in the
Pilbara region on a landscape scale. There are two major components of
the project: regional monitoring and disturbance area monitoring.
Regional survey and monitoring of Pilbara northern quoll populations
over 10+ years will provide a regional context for understanding
population dynamics.~Other ecological~



\section*{Aims}
\begin{itemize}
\itemsep1pt\parskip0pt\parsep0pt
\item
  Improve understanding of northern quoll population distribution,
  ecology and demography in the Pilbara.
\item
  Provide information to resource development companies that will allow
  appropriate management of mining sites to ensure the persistence of
  resident northern quoll populations.
\item
  Plan, establish and implement a regional northern quoll monitoring
  program in the Pilbara.
\item
  Develop an understanding of quoll habitat requirements and model
  predicted distribution in the Pilbara.
\end{itemize}



\section*{Progress}
\begin{itemize}
\itemsep1pt\parskip0pt\parsep0pt
\item
  The third~season of regional northern quoll monitoring at ten sites
  commenced in May 2016.
\item
  Predictive species distribution modelling was undertaken in
  collaboration with Edith Cowan University. Distribution maps have been
  produced, and expanded to include scenarios of climate change and cane
  toad invasion.
\item
  Northern quoll distribution model paper has been produced and
  submitted for publication.
\item
  Research priorities for the Pilbara northern quoll (as determined by
  2013 workshop) were published in Australian Mammalogy (Cramer et. al,
  2016).
\item
  A third northern quoll workshop was hosted by Parks and Wildlife in
  May 2016 with support from Roy Hill.
\item
  Quoll distributional data is continually added to the Pilbara
  Threatened Species portal in NatureMap.
\item
  Dietary analysis was undertaken on 500 northern quoll scats from
  throughout the Pilbara.~
\item
  Manuscript for dietary analysis produced, and submitted for
  publication.
\item
  Northern quoll spatial use and home range estimates were generated
  from an Honours project in association with Edith Cowan University.
\item
  Novel GPS collars for northern quolls were tested in a field setting,
  and provided information on spatial use and interactions with
  infrastructure.
\end{itemize}



\section*{Management implications}
\begin{itemize}
\itemsep1pt\parskip0pt\parsep0pt
\item
  Enhanced distributional data is publicly availability in an online
  repository for decision-making relating to northern quolls in the
  Pilbara. Future monitoring of northern quolls can be aligned with the
  methods of the regional program, to enable regional comparisons of
  population trends and change.
\item
  Sophisticated northern quoll population distribution maps can be used
  to predict the likelihood of occurrence, and inform management
  decisions. Areas without data collection have been identified as
  priorities for ground-truthing, and key populations likely to be
  impacted by future threatening processes have been determined.
\item
  Results from GPS tracking of northern quoll suggests that impacts can
  be limited if~known quoll habitat is~not fragmented or destroyed by
  infrastructure developments.
\item
  Modelling the changes in mortality of different cohorts of northern
  quolls has enabled best-practise baiting regimes to be implemented for
  feral cats in the Pilbara.
\end{itemize}



\section*{Future directions}
\begin{itemize}
\itemsep1pt\parskip0pt\parsep0pt
\item
  Outcomes of the 2016 workshop will be made publically available and
  assist in setting future directions.
\item
  Regional monitoring will continue, including collection of additional
  presence records.
\item
  Population genetics for Pilbara northern quolls will be assessed~with
  a further 500 DNA~samples to be~analysed. This will reveal information
  about the important northern quoll conservation units, genetic
  diversity within the region and effective home range size.
\item
  Paternal genetics of northern quoll offspring will be examined, to
  inform on relatedness and paternity of litter-mates.
\item
  Investigation into the interactions between northern quolls and
  introduced species (including predators; feral cat, red fox, wild dog,
  and the invasive cane toad) will continue.
\item
  Characterisation of northern quoll denning requirements, with the view
  to protecting these key~habitat features, or recreating them with
  artificial habitat.
\end{itemize}



%-----------------------------------------------------------------------------%
% Back matter
%\backmatter
\end{document}
%-----------------------------------------------------------------------------%


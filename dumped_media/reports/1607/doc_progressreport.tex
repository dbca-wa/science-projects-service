
\documentclass[version=last,
    paper=a4, % paper size
    10pt, % default font size
    usenames,
    dvipsnames,
    oneside, % ONLINE
    headings=openany, % open chapters on odd and even pages
    %toc=chapterentrywithdots, % Table of Contents style
    %BCOR=7mm, % PRINT Binding Correction
    %DIV=13, % typearea 161.54 mm x 228.46 mm, top margin 22.85 mm, inner margin 16.15 mm
    %DIV=14, % 165.00 233.36 21.21 15.00
    DIV=15 % 168.00 237.60 19.80 14.00
]{scrbook}
\usepackage{typearea}
\usepackage[automark,headsepline,footsepline]{scrlayer-scrpage} % Headers and footers

%%
%% Fonts, encoding, spacing, indentation
%%
\usepackage{txfonts}
\renewcommand{\familydefault}{\sfdefault} % Default to Sans Serif font
\usepackage[english]{babel}
\usepackage[T1]{fontenc}
\usepackage[utf8]{inputenc}

% Paragraph spacing
%\usepackage{parskip}    % Paragraph spacing
%\setlength{\parindent}{0em} % Don't indent paragraphs - ONLINE
%\setlength{\parskip}{1.3 ex plus 0.5ex minus 0.3ex} % 1-1.8 ex vertical space between paragraphs - ONLINE

% Spacing of headings
%\RedeclareSectionCommand[afterskip=3pt]{section} % less space after section
%\RedeclareSectionCommand[beforeskip=0cm]{subsection} % less space between HRule and project name
%\RedeclareSectionCommand[afterskip=0.1\baselineskip]{subsubsection} % less space after progressreport subheadings

% Table font size
\usepackage{etoolbox}
\AtBeginEnvironment{longtabu}{\footnotesize}{}{}

%%
%% Tables, columns, layout
%%
\usepackage{multicol}   % 2 col publications
\usepackage{pdflscape}  % Landscape pages
\usepackage{pdfpages}   % Include PDFs
\usepackage{hanging}    % hanging paragraphs for publications
%\usepackage{titletoc}   % Required for manipulating the table of contents
\setcounter{tocdepth}{2} % TOC list down to section
\usepackage{enumerate}  % Enumerations
\usepackage{enumitem}   % Enumerations
\usepackage{longtable}  % Multipage table
\usepackage{tabu}       %
\setlength{\tabulinesep}{1.5mm} % Consistent vertical spacing in tabu

%%
%% Graphics, images, colours
%%
\usepackage{graphicx} % embedded images
\usepackage{eso-pic} %
\usepackage{colortbl} % define custom named colours
\definecolor{RedFire}{RGB}{146,25,28}
\definecolor{ParksWildlife}{RGB}{0,85,144}
\definecolor{successbg}{RGB}{223,240,216}
\definecolor{errorbg}{RGB}{242,222,222}
\definecolor{warningbg}{RGB}{252,248,227}
\definecolor{infobg}{RGB}{217,237,247}
\definecolor{muted}{RGB}{153,153,153}
\definecolor{success}{RGB}{70,136,71}
\definecolor{error}{RGB}{185,74,72}
\definecolor{warning}{RGB}{192,152,83}
\definecolor{info}{RGB}{58,135,173}

\definecolor{required}{RGB}{192,152,83}
\definecolor{requiredbg}{RGB}{252,248,227}
\definecolor{denied}{RGB}{185,74,72}
\definecolor{deniedbg}{RGB}{242,222,222}
\definecolor{granted}{RGB}{70,136,71}
\definecolor{grantedbg}{RGB}{223,240,216}
\definecolor{not reqiured}{RGB}{153,153,153}
\definecolor{not requiredbg}{RGB}{255,255,255}

\usepackage{tikz} % Drawing
\usetikzlibrary{arrows,shapes,positioning,shadows,trees}

%%
%% Links, URLs
%%
\usepackage[
    linktoc=all,
    %colorlinks=false,  %PRINT
    colorlinks=true, % ONLINE
    linkcolor=RedFire, % ONLINE
    urlcolor=ParksWildlife, % ONLINE
    pdftitle=Progress Report SP 2013-005 (FY 2015-2016)
]{hyperref}

% Black magic to linebreak URLs
\usepackage{url}
\makeatletter
\g@addto@macro{\UrlBreaks}{\UrlOrds}
\makeatother

%%
%% Custom macros
%%
% Thick Horizontal rule
\newcommand{\HRule}{\vspace{8mm}\\\noindent\rule{\linewidth}{0.1pt}}

% Custom Tikz node for SDS diagram
\newcommand\mynode[6][]{
    \node[#1] (#2){
        \parbox{#3\relax}{
            \begin{center}
            \textbf{#4}\\
            #5\\
            \footnotesize{#6}
            \end{center}}};}



%-----------------------------------------------------------------------------%
% Headers and Footers
\automark{section}
\ohead{\href{http://sdis.dpaw.wa.gov.au/documents/progressreport/1607/}{Progress Report SP 2013-005
}}
\chead{\href{http://sdis.dpaw.wa.gov.au}{SDIS}} % center header ONLINE
\ihead{\href{http://sdis.dpaw.wa.gov.au}{
        \includegraphics[scale=0.4]{/mnt/projects/sdis/staticfiles/img/logo-dpaw.png}}}
\ifoot{\textbf{Printed}~Tue, 5 Jul 2016 10:45:15 +0800} % inner/left footer
\cfoot{} % center footer
\ofoot{\pagemark} % outer/right footer
\pagestyle{scrheadings}
\setkomafont{pageheadfoot}{\normalfont}

%-----------------------------------------------------------------------------%
\begin{document}
\raggedbottom

%-----------------------------------------------------------------------------%
% Title page
\subject{Progress Report SP 2013-005
}
\title{Improving the use of remote cameras as a survey and monitoring tool
}
\subtitle{Animal Science
}
\author{}
\publishers{\small
    \subsection*{Project Core Team}
\begin{tabu} {X X}
\textbf{Supervising Scientist} & Neil Thomas
\\
\textbf{Data Custodian} & 
\\
\textbf{Site Custodian} & 
\\
\end{tabu}


    \subsection*{Project status as of July 5, 2016, 10:45 a.m.}
\begin{tabu} {X X}
& Approved and active
\\
\end{tabu}

    
\subsection*{Document endorsements and approvals as of July 5, 2016, 10:45 a.m.}
\begin{tabu} {X X}

%\rowcolor{grantedbg}
    \textbf{Project Team} & 
    \textcolor{granted}{ granted}\\

%\rowcolor{grantedbg}
    \textbf{Program Leader} & 
    \textcolor{granted}{ granted}\\

%\rowcolor{grantedbg}
    \textbf{Directorate} & 
    \textcolor{granted}{ granted}\\

\end{tabu}



}
\uppertitleback{}
\lowertitleback{}
\date{}

%-----------------------------------------------------------------------------%
% Front matter
\frontmatter
\maketitle
%-----------------------------------------------------------------------------%
% Main matter
\mainmatter

\section*{Improving the use of remote cameras as a survey and monitoring tool
}

N Thomas, M Cowan, B MacMahon, S Garretson


\section*{Context}
The use of camera traps is often regarded as an effective tool for fauna
survey and monitoring with the assumption that they provide high
quality, cost effective data. However, our understanding of appropriate
methods for general survey and species detection, particularly in the
small to medium sized range of mammals, remains poorly understood.
Within Parks and Wildlife use of camera traps to date has usually been
restricted to simple species inventories or behavioural studies and
beyond this there has been little assessment of deployment methods or
appropriate analytical techniques. This has sometimes led to erroneous
conclusions being derived from captured images. Camera traps have the
potential to offer a comparatively reliable and relatively unbiased
method for monitoring medium to large native and introduced mammal
species throughout the state, including a number of significant cryptic
species that are currently not incorporated under the Western Shield
fauna monitoring program. However, research is required to validate and
test different survey designs (temporal and spatial components) and
methods of deploying camera traps, and to interpret the results in a
meaningful way. In particular, work is needed to determine how best to
use remote cameras to provide rigorous data on species detectability,
and species richness and density.



\section*{Aims}
\begin{itemize}
\itemsep1pt\parskip0pt\parsep0pt
\item
  Establish suitable methodology for use of camera traps to estimate the
  presence and relative abundances of native and introduced mammals
  species in the south-west of Western Australia.
\item
  Investigate the effectiveness of baited (active) and un-baited
  (passive) cameras sets to inventory targeted species.
\item
  Investigate and assess the most appropriate methods of image analysis
  and data storage.
\end{itemize}



\section*{Progress}
\begin{itemize}
\itemsep1pt\parskip0pt\parsep0pt
\item
  Continue to provide advice on camera trap survey~methodology to other
  sections within Parks and Wildlife, tertiary institutions, industry
  and NGO's.
\item
  The project has been incorporated, as a major field component, into
  the South West Fauna Recovery Project (Dryandra)~with much of the day
  to day running of the field program being undertaken by the Great
  Southern District.
\item
  A Science and Conservation Information Sheet on assessing camera traps
  to census mammals has been prepared.
\item
  There have been ongoing developments with the open source Access
  database, CPW Photo Warehouse, as modifications are requested from
  departmental users to meet their specific requirement for a database
  to manage camera trap images.
\item
  Reported to South West Fauna Recovery Project team the detection rates
  and spatial patterns for all critical weight range mammals and larger
  species within the primary Dryandra block for 2014. Data from 2015 to
  present is currently being assessed.
\item
  Deliver presentation on using camera traps to the Kwongan Foundation's
  annual conference and to the UWA Environmental Studies Group.
\end{itemize}



\section*{Management implications}
\begin{itemize}
\itemsep1pt\parskip0pt\parsep0pt
\item
  Camera traps are an effective tool for detecting a suite of species
  currently not adequately monitored by the Western Shield program.
  However, there is growing evidence that they are not the covert tool
  that they were first thought to be and may have biases into detecting
  some species.~
\item
  Careful consideration is required where lures or baits are used as
  there is a considerable risk of introducing bias in detections.
\item
  Managers need to give special consideration to the specific species
  being targeted, questions being addressed, type of camera trap, survey
  design and set-up before undertaking any camera trap survey.
\item
  Reconyx camera traps (models HC600 and PC900) continue to be the most
  effective commercially available~camera traps for departmental
  requirements and remain recommended for use.
\end{itemize}



\section*{Future directions}
\begin{itemize}
\itemsep1pt\parskip0pt\parsep0pt
\item
  Ongoing assessment of on track/off track use of camera traps to
  quantitatively monitor foxes and feral cats in Dryandra.
\item
  Investigate potential biases in detections when using camera traps to
  monitor fauna and pest species such as trap avoidance/behavioural
  changes, and how to mitigate these biases.
\item
  Integrate Dryandra camera trap work as a monitoring technique for cat
  bait effectiveness trials.
\item
  Continue work on reviewing~and/or modifying open source Access Camera
  trap databases as they become available so that their~functionality
  better suits the Department's needs.
\item
  Continue to undertake desktop reviews~of new camera traps
  (particularly cameras with video capability) as they become available
  to determine~if any new models are better suited to the Department's
  needs.
\end{itemize}



%-----------------------------------------------------------------------------%
% Back matter
%\backmatter
\end{document}
%-----------------------------------------------------------------------------%


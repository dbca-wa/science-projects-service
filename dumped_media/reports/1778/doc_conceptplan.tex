
\documentclass[version=last,
    paper=a4,                               % paper size
    10pt,                                   % default font size
    dvipsnames,
    % twoside,                                % PRINT Binding Correction
    oneside,                              % ONLINE
    headings=openany,                       % open chapters on odd and even pages
    open=any,
    BCOR=7mm,                               % PRINT Binding Correction
    %DIV=13,    % typearea 161.54mm x 228.46mm, top 22.85mm, inner 16.15mm
    %DIV=14,    % 165.00 233.36 21.21 15.00
    DIV=15,     % 168.00 237.60 19.80 14.00
    % toc=chapterentrywithdots              % Table of Contents style
]{scrbook}
\usepackage{typearea}


%------------------------------------------------------------------------------%
% Headers and footers
%------------------------------------------------------------------------------%
\usepackage[automark,headsepline,footsepline,plainfootsepline]{scrlayer-scrpage}
\automark*[section]{}
\addtokomafont{pageheadfoot}{\normalfont\footnotesize\sffamily} % Don't italicise
\renewcommand{\chaptermark}[1]{\markleft{#1}{}}     % Chapter: suppress numbering
\renewcommand{\sectionmark}[1]{\markright{#1}{}}    % Section: suppress numbering

% Header (inner, center, outer)
% \ihead{\href{http://sdis.dpaw.wa.gov.au}{\textbf{Concept Plan SP 2016-068}}}
%\chead{\href{http://sdis.dpaw.wa.gov.au}{Science Directorate Information System}}
% \ohead{\href{https://www.dpaw.wa.gov.au/about-us/science-and-research}{\includegraphics[height=8mm, keepaspectratio]{/mnt/projects/sdis/staticfiles/img/logo-dpaw.png}}}

% Footer (inner, center, outer)
% \ifoot{\RaggedRight\leftmark}                       % Chapter
% \cfoot{\RaggedLeft\rightmark}                       % Section
% \ofoot[\bfseries\thepage]{\bfseries\thepage}        % Page number (also [plain])


%------------------------------------------------------------------------------%
% Fonts, encoding
%------------------------------------------------------------------------------%
%\usepackage{avant}             % Use the Avantgarde font for headings
\usepackage{txfonts}
\usepackage{mathptmx}
\usepackage{gensymb}            % provides \textdegree
\renewcommand{\familydefault}{\sfdefault} % Default to Sans Serif font
\usepackage{microtype}          % Slightly tweak font spacing for aesthetics
\usepackage[english]{babel}
\usepackage[utf8]{inputenc}  % Allow letters with accents
\usepackage[utf8]{luainputenc}  % Allow letters with accents
\usepackage[T1]{fontenc}        % Use 8-bit encoding that has 256 glyphs
\usepackage{textcomp}
\usepackage[explicit]{titlesec}           % Customise of titles
%\DeclareUnicodeCharacter{0080}{\textregistered}
\DeclareUnicodeCharacter{00B0}{\textdegree}

%------------------------------------------------------------------------------%
% Tables, columns, layout
%------------------------------------------------------------------------------%
\usepackage{etoolbox}
\AtBeginEnvironment{longtabu}{\footnotesize}{}{}  % Table font size
\usepackage{booktabs}           % Required for nicer horizontal rules in tables
\usepackage{multicol}           % 2 col publications
\usepackage{pdflscape}          % Landscape pages
\usepackage{pdfpages}           % Include PDFs
\usepackage{hanging}            % hanging paragraphs for publications
%\usepackage{titletoc}          % Manipulate the table of contents
\setcounter{tocdepth}{2}        % TOC list down to section
\usepackage{enumerate}          % Enumerations
\usepackage{enumitem}           % Enumerations
\usepackage{longtable}          % Multipage table
\usepackage{tabu}               %
\setlength{\tabulinesep}{1.5mm} % Consistent vertical spacing in tabu
\newcommand{\HRule}{\vspace{8mm}\noindent\rule{\linewidth}{0.1pt}}
\usepackage[export]{adjustbox}  % minipage, image frame


%------------------------------------------------------------------------------%
% Graphics, images, colours
%------------------------------------------------------------------------------%
\usepackage{graphicx} % embedded images
\usepackage{wrapfig}  % wrap figures in text
\usepackage{caption}  % allow unnumbered captions
\usepackage{eso-pic} % Required for specifying an image background in the title page
\usepackage{colortbl} % define custom named colours
\usepackage{xstring} % Conditionals
\usepackage{transparent} % Allow transparent images

\definecolor{RedFire}{RGB}{146,25,28}
% Following PICA branding guidelines
% https://dpaw.sharepoint.com/Divisions/pica/Documents/Branding%20guidelines.pdf
\definecolor{dpawblue}{RGB}{35,97,146}          % Pantone 647
\definecolor{dpaworange}{RGB}{237,139,0}        % Pantone 144
\definecolor{dpawgreen}{RGB}{116,170,80}        % Pantone 7489
\definecolor{dpawred}{RGB}{124,46,44}           % Paul's suggestion

% bootstrap colours
\definecolor{successbg}{RGB}{223,240,216}
\definecolor{errorbg}{RGB}{242,222,222}
\definecolor{warningbg}{RGB}{252,248,227}
\definecolor{infobg}{RGB}{217,237,247}
\definecolor{muted}{RGB}{153,153,153}
\definecolor{success}{RGB}{70,136,71}
\definecolor{error}{RGB}{185,74,72}
\definecolor{warning}{RGB}{192,152,83}
\definecolor{info}{RGB}{58,135,173}

% SDIS approval colours
\definecolor{required}{RGB}{192,152,83}
\definecolor{requiredbg}{RGB}{252,248,227}
\definecolor{denied}{RGB}{185,74,72}
\definecolor{deniedbg}{RGB}{242,222,222}
\definecolor{granted}{RGB}{70,136,71}
\definecolor{grantedbg}{RGB}{223,240,216}
\definecolor{notrequired}{RGB}{153,153,153}
\definecolor{notrequiredbg}{RGB}{255,255,255}

\usepackage{tikz} % Drawing
\usetikzlibrary{arrows,shapes,positioning,shadows,trees}


%------------------------------------------------------------------------------%
% Hyperlinks
%------------------------------------------------------------------------------%
\usepackage[open=true]{bookmark}
\usepackage{nameref}
\usepackage{ifxetex,ifluatex}
\ifxetex
  \usepackage[
    setpagesize=false,        % page size defined by xetex
    unicode=false,            % unicode breaks when used with xetex
    xetex]{hyperref}
\else
  \usepackage[unicode=true]{hyperref}
\fi

\hypersetup{
  backref=true,
  pagebackref=true,
  hyperindex=true,
  breaklinks=true,
  urlcolor=dpawblue,
  bookmarks=true,
  bookmarksopen=false,
  pdfauthor={Science and Conservation Division, Dept Parks and Wildlife, WA},
  pdftitle=Concept Plan SP 2016-068
,
  colorlinks=true,
  linkcolor=dpawblue,
  pdfborder={0 0 0}}

\urlstyle{same}                         % don't use monospace font for urlstyle


%------------------------------------------------------------------------------%
% Black magic to linebreak URLs
%------------------------------------------------------------------------------%
\usepackage{url}
\makeatletter\g@addto@macro{\UrlBreaks}{\UrlOrds}\makeatother
\Urlmuskip=0mu plus 1mu


%------------------------------------------------------------------------------%
% Fix latex errors
%------------------------------------------------------------------------------%
\providecommand{\tightlist}{\setlength{\itemsep}{0pt}\setlength{\parskip}{0pt}}

% copy-pasted HTML <span> in SDIS fields becomes \text{} in tex source
\providecommand{\text}{}


%------------------------------------------------------------------------------%
% Custom Tikz node for SDS diagram
%------------------------------------------------------------------------------%
\newcommand\mynode[6][]{
  \node[#1] (#2){
    \parbox{#3\relax}{
      \begin{center}
      \textbf{#4}\\
      #5\\
      \footnotesize{#6}
      \end{center}
    }};}


%------------------------------------------------------------------------------%
% Custom no-pagebreaks-environment
%------------------------------------------------------------------------------%
\newenvironment{absolutelynopagebreak}
  {\par\nobreak\vfil\penalty0\vfilneg\vtop\bgroup}
  {\par\xdef\tpd{\the\prevdepth}\egroup\prevdepth=\tpd}


%------------------------------------------------------------------------------%
% Remove the header from odd empty pages at the end of chapters
%------------------------------------------------------------------------------%
\makeatletter
\renewcommand{\cleardoublepage}{
\clearpage\ifodd\c@page\else
\hbox{}
\vspace*{\fill}
\thispagestyle{empty}
\newpage
\fi}


%----------------------------------------------------------------------------------------
%  Page flow control
%----------------------------------------------------------------------------------------
%\widowpenalty=10000
%\clubpenalty=10000
%\vbadness=1200
%\hbadness=11000


%----------------------------------------------------------------------------------------
%   CHAPTER HEADINGS
%----------------------------------------------------------------------------------------
\newcommand{\thechapterimage}{}
\newcommand{\chapterimage}[1]{\renewcommand{\thechapterimage}{#1}}

% Numbered chapters with mini tableofcontents
\def\thechapter{\arabic{chapter}}
\def\@makechapterhead#1{
%\thispagestyle{plain}
{\centering \normalfont\sffamily
\ifnum \c@secnumdepth >\m@ne
\if@mainmatter
\startcontents
\begin{tikzpicture}[remember picture,overlay]
\node at (current page.north west)
{\begin{tikzpicture}[remember picture,overlay]
\node[anchor=north west,inner sep=0pt] at (0,0) {
\includegraphics[width=\paperwidth,height=0.5\paperwidth]{\thechapterimage}};
%------------------------------------------------------------------------------%
% Small contents box in the chapter heading
% Mini TOC background box
%\fill[color=dpawblue!10!white,opacity=.2] (1cm,0) rectangle (
%  3.5cm, % Mini TOC box width
%  -3.5cm % Mini TOC box height
%);
% Mini TOC text content
%\node[anchor=north west] at (1.1cm,.35cm) {
%  \parbox[t][8cm][t]{6.5cm}{
%    \huge\bfseries\flushleft
%    \printcontents{l}{1}{
%    \setcounter{tocdepth}{1}                   % Mini TOC level depth
%    }
% }
%};
%------------------------------------------------------------------------------%
% Chapter heading
\draw[anchor=west] (5cm,-9cm) node [
rounded corners=20pt,
fill=dpawblue!10!white,
text opacity=1,
draw=dpawblue,
draw opacity=1,
line width=1.5pt,
fill opacity=.2,
inner sep=12pt]{
    \huge\sffamily\bfseries\textcolor{black}{
      \thechapter. #1\strut\makebox[22cm]{}
    }
};
\end{tikzpicture}};
\end{tikzpicture}}
\par\vspace*{240\p@}                            % Push text below chapter image
\fi
\fi}

%------------------------------------------------------------------------------%
% Unnumbered chapters without mini tableofcontents
%------------------------------------------------------------------------------%
\def\@makeschapterhead#1{
%\thispagestyle{plain}
{\centering \normalfont\sffamily
\ifnum \c@secnumdepth >\m@ne
\if@mainmatter
\begin{tikzpicture}[remember picture,overlay]
\node at (current page.north west)
{\begin{tikzpicture}[remember picture,overlay]
\node[anchor=north west,inner sep=0pt] at (0,0) {
  \includegraphics[width=\paperwidth,height=0.5\paperwidth]{\thechapterimage}};
% Mini TOC background box
%\fill[color=dpawblue!10!white,opacity=.2] (1cm,0) rectangle (
%  3.5cm,                                       % Mini TOC box width
%  -3.5cm                                       % Mini TOC box height
%);
% Mini TOC text content
%\node[anchor=north west] at (1.1cm,.35cm) {
%  \parbox[t][8cm][t]{6.5cm}{
%    \huge\bfseries\flushleft
%    \printcontents{l}{1}{
%    \setcounter{tocdepth}{1} % Mini TOC level depth
%    }
%}
%};
\draw[anchor=west] (5cm,-9cm) node [rounded corners=20pt,
  fill=dpawblue!10!white,fill opacity=.6,inner sep=12pt,text opacity=1,
  draw=dpawblue,draw opacity=1,line width=1.5pt]{
  \huge\sffamily\bfseries\textcolor{black}{#1\strut\makebox[22cm]{}}};
\end{tikzpicture}};
\end{tikzpicture}}
\par\vspace*{240\p@}
\fi
\fi
}
\makeatother



\usepackage[automark,headsepline,footsepline,plainfootsepline]{scrlayer-scrpage}
\automark*[section]{}
\addtokomafont{pageheadfoot}{\normalfont\footnotesize\sffamily} % Don't italicise
\renewcommand{\chaptermark}[1]{\markleft{#1}{}}     % Chapter: suppress numbering
\renewcommand{\sectionmark}[1]{\markright{#1}{}}    % Section: suppress numbering

% Header (inner, center, outer)
\ihead{\href{http://sdis.dpaw.wa.gov.au/documents/conceptplan/1778/}{Concept Plan SP 2016-068}}
%\chead{\href{http://sdis.dpaw.wa.gov.au}{Science Directorate Information System}}
\ohead{\href{https://www.dpaw.wa.gov.au/about-us/science-and-research}{\includegraphics[height=6mm, keepaspectratio]{/mnt/projects/sdis/staticfiles/img/logo-dpaw.png}}}

% Footer (inner, center, outer)
\ifoot{\textbf{Printed}~Tue, 12 Dec 2017 16:48:20 +0800} % inner/left footer
\cfoot{}
\ofoot[\bfseries\thepage]{\bfseries\thepage}        % Page number (also [plain])


\pagestyle{scrheadings}
\setkomafont{pageheadfoot}{\normalfont}

%-----------------------------------------------------------------------------%
\begin{document}
\raggedbottom

%-----------------------------------------------------------------------------%
% Title page
\subject{Concept Plan SP 2016-068
}
\title{South West Threatened Fauna Recovery Project: Southern Jarrah Forest
}
\subtitle{Animal Science
}
\author{}
\publishers{\small
    \subsection*{Project Core Team}
\begin{tabu} {X X}
\textbf{Supervising Scientist} & Adrian Wayne
\\
\textbf{Data Custodian} & Adrian Wayne
\\
\textbf{Site Custodian} & 
\\
\end{tabu}


    \subsection*{Project status as of Dec. 12, 2017, 4:48 p.m.}
\begin{tabu} {X X}
& Pending project plan approval
\\
\end{tabu}

    
\subsection*{Document endorsements and approvals as of Dec. 12, 2017, 4:48 p.m.}
\begin{tabu} {X X}

%\rowcolor{grantedbg}
    \textbf{Project Team} & 
    \textcolor{granted}{ granted}\\

%\rowcolor{grantedbg}
    \textbf{Program Leader} & 
    \textcolor{granted}{ granted}\\

%\rowcolor{grantedbg}
    \textbf{Directorate} & 
    \textcolor{granted}{ granted}\\

\end{tabu}



}
\uppertitleback{}
\lowertitleback{}
\date{}

%-----------------------------------------------------------------------------%
% Front matter
\frontmatter
\maketitle
%-----------------------------------------------------------------------------%
% Main matter
\mainmatter


\section*{South West Threatened Fauna Recovery Project: Southern Jarrah Forest
}



\subsection*{Science and Conservation Division Program}

Animal Science




\subsection*{Parks and Wildlife Service}

Service 2: Conserving Habitats, Species and Ecological Communities




\subsection*{Aims}

The purpose of the ``South West Threatened Fauna Recovery Project is to
integrate the new \emph{Eradicat®} feral cat bait with current fox
baiting in four different Western Australian environments. Funded by the
Commonwealth and Western Australian State Government, sites have been
specifically selected to direct the funding to the improved conservation
of species identified in the Commonwealth's Threatened Species Strategy,
and to align with fauna recovery programs underway or planned by Parks
and Wildlife in the south-west of WA. The four locations represent four
different climatic regions of the south-west of Western Australia;
northern wheatbelt (Kalbarri N.P.), central wheatbelt (Dryandra
Woodland) southern jarrah forest (Upper Warren region), and heathland
(southcoast reserves). The primary goal of the project is to protect and
contribute to the recovery of key threatened mammal and bird species
found at each of these sites.

The goals for the Upper Warren component are to recover wild populations
of western ringtail possums, woylies and numbats in the Perup/Upper
Warren area, through:

\begin{itemize}
\itemsep1pt\parskip0pt\parsep0pt
\item
  effective integration of feral cat control with existing fox control
\item
  effective neighbour engagement; and
\item
  successful establishment of an insurance population of western
  ringtail possums in the Perup Sanctuary.
\end{itemize}

This concept plan is focussed on the first of these goals, i.e. to
recover wild populations of western ringtail possums, woylies and
numbats in the Perup/Upper Warren area through effective integration of
feral cat control with existing fox control.

\textbf{OBJECTIVE}: Develop an effective cat and fox control program
using \emph{Eradicat\textsuperscript{®}} and Probait baits that reduces
their impact sufficiently on vulnerable threatened native fauna to
support viable and secure populations.

\textbf{STRATEGIES}:

\begin{enumerate}
\itemsep1pt\parskip0pt\parsep0pt
\item
  Determine when and where to deploy \emph{Eradicat\textsuperscript{®}}
  baits for effective feral cat control.
\item
  Experimentally conduct toxic \emph{Eradicat\textsuperscript{®}}
  baiting within a limited area of the Upper Warren to assess the
  effectiveness on target species (cats and foxes) and effects on
  non-target species (with a focus on potentially vulnerable and
  priority beneficiaries), having successfully completed and been
  informed by Strategy 1. This is to limit and verify the risks to
  populations of non-target species.
\item
  Operationally integrate \emph{Eradicat\textsuperscript{®}} and Probait
  into the introduced predator control and monitoring program across the
  Upper Warren region, having successfully completed and been informed
  by Strategies 1 and 2.
\item
  Consider the merits of rolling out the improved introduced predator
  control and monitoring program across other forested landscapes in
  south-western Australia based on the outcomes of Strategies 1, 2 and 3
  above.
\end{enumerate}




\subsection*{Expected outcome}

\begin{itemize}
\itemsep1pt\parskip0pt\parsep0pt
\item
  Determining when and where \emph{Eradicat®} baiting under existing
  protocols may be most effective in targeting feral cats and minimising
  the potential risk to non-target native mammals in the southern jarrah
  forests of WA.
\item
  Quantifying the effectiveness of \emph{Eradicat\textsuperscript{®}}
  baiting on cat and fox densities and potential risks to non-target
  native mammals in the southern jarrah forests of WA.
\end{itemize}




\subsection*{Strategic context}

\begin{itemize}
\itemsep1pt\parskip0pt\parsep0pt
\item
  Department of Parks and Wildlife strategic goal (Wildlife): Conserve,
  protect and manage the State's native plants and animals, and achieve
  habitat, ecosystem and landscape=scale conservation and protection
  based on best practice science.
\item
  Science and Conservation Strategic Plan 2014-2017: Recovery of key
  animal species, understand and manage threats from pest.
\item
  Directly relates to the fauna recovery programs underway or planned by
  Parks and Wildlife, in particular directly addressing one of the key
  threats (cats) to many native mammals in SW WA.
\end{itemize}




\subsection*{Expected collaborations}

\begin{itemize}
\itemsep1pt\parskip0pt\parsep0pt
\item
  The other three project components within the South West Threatened
  Fauna Recovery Project.
\item
  International student projects including French intern Lena Baraud
  (February -- June 2016) and CIEE student cohorts for two weeks in
  autumn and spring of each year.
\item
  Warren regional staff.
\end{itemize}


\subsection*{Proposed period of the project}
Feb. 1, 2016 -- Dec. 31, 2018



\subsection*{Staff time allocation }



\begin{longtabu} to \linewidth { |  X | X | X | X | }
\hline
\rowcolor{infobg}
Role & Year 1 & Year 2 & Year 3\\
\hline
\endhead



Scientist & 0.50 & 0.50 & 0.50\\



Technical & 1.80 & 1.80 & 1.80\\



Volunteer & 1.0 & 1.0 & 1.0\\



Collaborator &  &  & \\


\hline
\end{longtabu}



\subsection*{Indicative operating budget }



\begin{longtabu} to \linewidth { |  X | X | X | X | }
\hline
\rowcolor{infobg}
Source & Year 1 & Year 2 & Year 3\\
\hline
\endhead



Consolidated Funds (DPaW) & 165,500 & 402,100 & 415,750\\



External Funding & 102,000 & 57,000 & 312,000\\


\hline
\end{longtabu}






%-----------------------------------------------------------------------------%
% Back matter
%\backmatter
\end{document}
%-----------------------------------------------------------------------------%

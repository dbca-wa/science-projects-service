
\documentclass[version=last,
    paper=a4, % paper size
    10pt, % default font size
    usenames,
    dvipsnames,
    oneside, % ONLINE
    headings=openany, % open chapters on odd and even pages
    %toc=chapterentrywithdots, % Table of Contents style
    %BCOR=7mm, % PRINT Binding Correction
    %DIV=13, % typearea 161.54 mm x 228.46 mm, top margin 22.85 mm, inner margin 16.15 mm
    %DIV=14, % 165.00 233.36 21.21 15.00
    DIV=15 % 168.00 237.60 19.80 14.00
]{scrbook}
\usepackage{typearea}
\usepackage[automark,headsepline,footsepline]{scrlayer-scrpage} % Headers and footers

%%
%% Fonts, encoding, spacing, indentation
%%
\usepackage{txfonts}
\renewcommand{\familydefault}{\sfdefault} % Default to Sans Serif font
\usepackage[english]{babel}
\usepackage[T1]{fontenc}
\usepackage[utf8]{inputenc}

% Paragraph spacing
%\usepackage{parskip}    % Paragraph spacing
%\setlength{\parindent}{0em} % Don't indent paragraphs - ONLINE
%\setlength{\parskip}{1.3 ex plus 0.5ex minus 0.3ex} % 1-1.8 ex vertical space between paragraphs - ONLINE

% Spacing of headings
%\RedeclareSectionCommand[afterskip=3pt]{section} % less space after section
%\RedeclareSectionCommand[beforeskip=0cm]{subsection} % less space between HRule and project name
%\RedeclareSectionCommand[afterskip=0.1\baselineskip]{subsubsection} % less space after progressreport subheadings

% Table font size
\usepackage{etoolbox}
\AtBeginEnvironment{longtabu}{\footnotesize}{}{}

%%
%% Tables, columns, layout
%%
\usepackage{multicol}   % 2 col publications
\usepackage{pdflscape}  % Landscape pages
\usepackage{pdfpages}   % Include PDFs
\usepackage{hanging}    % hanging paragraphs for publications
%\usepackage{titletoc}   % Required for manipulating the table of contents
\setcounter{tocdepth}{2} % TOC list down to section
\usepackage{enumerate}  % Enumerations
\usepackage{enumitem}   % Enumerations
\usepackage{longtable}  % Multipage table
\usepackage{tabu}       %
\setlength{\tabulinesep}{1.5mm} % Consistent vertical spacing in tabu

%%
%% Graphics, images, colours
%%
\usepackage{graphicx} % embedded images
\usepackage{eso-pic} %
\usepackage{colortbl} % define custom named colours
\definecolor{RedFire}{RGB}{146,25,28}
\definecolor{ParksWildlife}{RGB}{0,85,144}
\definecolor{successbg}{RGB}{223,240,216}
\definecolor{errorbg}{RGB}{242,222,222}
\definecolor{warningbg}{RGB}{252,248,227}
\definecolor{infobg}{RGB}{217,237,247}
\definecolor{muted}{RGB}{153,153,153}
\definecolor{success}{RGB}{70,136,71}
\definecolor{error}{RGB}{185,74,72}
\definecolor{warning}{RGB}{192,152,83}
\definecolor{info}{RGB}{58,135,173}

\definecolor{required}{RGB}{192,152,83}
\definecolor{requiredbg}{RGB}{252,248,227}
\definecolor{denied}{RGB}{185,74,72}
\definecolor{deniedbg}{RGB}{242,222,222}
\definecolor{granted}{RGB}{70,136,71}
\definecolor{grantedbg}{RGB}{223,240,216}
\definecolor{not reqiured}{RGB}{153,153,153}
\definecolor{not requiredbg}{RGB}{255,255,255}

\usepackage{tikz} % Drawing
\usetikzlibrary{arrows,shapes,positioning,shadows,trees}

%%
%% Links, URLs
%%
\usepackage[
    linktoc=all,
    %colorlinks=false,  %PRINT
    colorlinks=true, % ONLINE
    linkcolor=RedFire, % ONLINE
    urlcolor=ParksWildlife, % ONLINE
    pdftitle=Progress Report SP 2001-004 (FY 2015-2016)
]{hyperref}

% Black magic to linebreak URLs
\usepackage{url}
\makeatletter
\g@addto@macro{\UrlBreaks}{\UrlOrds}
\makeatother

%%
%% Custom macros
%%
% Thick Horizontal rule
\newcommand{\HRule}{\vspace{8mm}\\\noindent\rule{\linewidth}{0.1pt}}

% Custom Tikz node for SDS diagram
\newcommand\mynode[6][]{
    \node[#1] (#2){
        \parbox{#3\relax}{
            \begin{center}
            \textbf{#4}\\
            #5\\
            \footnotesize{#6}
            \end{center}}};}



%-----------------------------------------------------------------------------%
% Headers and Footers
\automark{section}
\ohead{\href{http://sdis.dpaw.wa.gov.au/documents/progressreport/1671/}{Progress Report SP 2001-004
}}
\chead{\href{http://sdis.dpaw.wa.gov.au}{SDIS}} % center header ONLINE
\ihead{\href{http://sdis.dpaw.wa.gov.au}{
        \includegraphics[scale=0.4]{/mnt/projects/sdis/staticfiles/img/logo-dpaw.png}}}
\ifoot{\textbf{Printed}~Mon, 4 Jul 2016 16:18:17 +0800} % inner/left footer
\cfoot{} % center footer
\ofoot{\pagemark} % outer/right footer
\pagestyle{scrheadings}
\setkomafont{pageheadfoot}{\normalfont}

%-----------------------------------------------------------------------------%
\begin{document}
\raggedbottom

%-----------------------------------------------------------------------------%
% Title page
\subject{Progress Report SP 2001-004
}
\title{Translocation of critically endangered plants
}
\subtitle{Plant Science and Herbarium
}
\author{}
\publishers{\small
    \subsection*{Project Core Team}
\begin{tabu} {X X}
\textbf{Supervising Scientist} & Leonie Monks
\\
\textbf{Data Custodian} & Leonie Monks
\\
\textbf{Site Custodian} & Leonie Monks
\\
\end{tabu}


    \subsection*{Project status as of July 4, 2016, 4:18 p.m.}
\begin{tabu} {X X}
& Approved and active
\\
\end{tabu}

    
\subsection*{Document endorsements and approvals as of July 4, 2016, 4:18 p.m.}
\begin{tabu} {X X}

%\rowcolor{grantedbg}
    \textbf{Project Team} & 
    \textcolor{granted}{ granted}\\

%\rowcolor{grantedbg}
    \textbf{Program Leader} & 
    \textcolor{granted}{ granted}\\

%\rowcolor{grantedbg}
    \textbf{Directorate} & 
    \textcolor{granted}{ granted}\\

\end{tabu}



}
\uppertitleback{}
\lowertitleback{}
\date{}

%-----------------------------------------------------------------------------%
% Front matter
\frontmatter
\maketitle
%-----------------------------------------------------------------------------%
% Main matter
\mainmatter

\section*{Translocation of critically endangered plants
}

L Monks, R Dillon, D Coates


\section*{Context}
The contribution of translocations (augmentation, re-introductions,
introductions) of threatened flora to the successful recovery of species
requires development of best-practice techniques and a clear
understanding of how to assess and predict translocation success.



\section*{Aims}
\begin{itemize}
\itemsep1pt\parskip0pt\parsep0pt
\item
  Develop appropriate translocation techniques for a range of Critically
  Endangered and other Threatened flora considered a priority for
  translocation.
\item
  Develop detailed protocols for assessing and predicting translocation
  success.
\item
  Establish a translocation database for all threatened plant
  translocations in Western Australia.
\end{itemize}



\section*{Progress}
\begin{itemize}
\itemsep1pt\parskip0pt\parsep0pt
\item
  Infill planting was completed for translocations of two Critically
  Endangered plant species at two sites.
\item
  Monitoring was undertaken for 44 sites of 31 taxa translocated in
  previous years.
\item
  Detailed demographic monitoring was undertaken for \emph{Acacia
  cochlocarpa} subsp. \emph{cochlocarpa} at translocation and natural
  sites, and demographic monitoring will be used to develop a Population
  Viability Analysis (PVA) model for this subspecies.
\item
  Drafting of two~publications on flora translocation methods and
  success criteria continued.
\item
  Assisted Departmental District and Regional staff plan and implement a
  range of flora translocations.
\item
  Provided advice to Departmental staff on flora translocations
  proposals submitted for approval.
\item
  Seed collections for mating system studies were completed for two
  translocated populations of \emph{Banksia brownii}.
\item
  A National Environmental Science Program (NESP) Threatened Species Hub
  project on plant translocations was established with a significant
  component of the research based in Western Australia with the
  Department.~
\end{itemize}



\section*{Management implications}
\begin{itemize}
\itemsep1pt\parskip0pt\parsep0pt
\item
  Translocations lead to the improved conservation status for threatened
  flora, particularly Critically Endangered plant taxa.
\item
  The improved awareness of best-practice translocation methods for
  Parks and Wildlife staff and community members undertaking such work,
  leads to greater translocation success.
\item
  Further development of success criteria and methods for analysing
  long-term success, such as the use of PVA, mating system analysis and
  genetic variability analysis, will ensure completion criteria are
  adequately addressed and that resources can be confidently
  re-allocated to new translocation projects.
\item
  Ongoing monitoring of translocations is providing information on the
  success of methods used and the probability of long-term success.
  Close collaboration with District and Regional staff enables this
  information to be used immediately to inform other flora translocation
  projects.
\end{itemize}



\section*{Future directions}
\begin{itemize}
\itemsep1pt\parskip0pt\parsep0pt
\item
  Continue the planting of experimental translocations of Critically
  Endangered and other Threatened flora where further translocations are
  deemed necessary.
\item
  Continued monitoring of flora translocations and further development
  of criteria for evaluating success, such as PVA, mating system and
  genetic variability analysis.
\item
  Complete a review on translocation outcomes and success in Western
  Australia.
\item
  Publish paper on translocation methods.
\item
  Publish paper \emph{Lambertia orbifolia} PVA study
\item
  Continue data collection for development of a PVA model for
  translocated and natural populations of \emph{A. cochlocarpa} subsp.
  \emph{cochlocarpa}.
\item
  Continue collaboration with the NESP Threatened Species Hub on
  threatened flora translocation research.
\end{itemize}



%-----------------------------------------------------------------------------%
% Back matter
%\backmatter
\end{document}
%-----------------------------------------------------------------------------%



\documentclass[version=last, paper=a4, DIV=18, usenames, dvipsnames]{scrartcl}
\usepackage{txfonts}
\usepackage{pdflscape}
\usepackage{pdfpages}
\usepackage[english]{babel} % English language/hyphenation
%%% Bootstrap colors
\definecolor{RedFire}{RGB}{146,25,28}
\definecolor{ParksWildlife}{RGB}{0,85,144}
\definecolor{successbg}{RGB}{223,240,216}
\definecolor{errorbg}{RGB}{242,222,222}
\definecolor{warningbg}{RGB}{252,248,227}
\definecolor{infobg}{RGB}{217,237,247}
\definecolor{muted}{RGB}{153,153,153}
\definecolor{success}{RGB}{70,136,71}
\definecolor{error}{RGB}{185,74,72}
\definecolor{warning}{RGB}{192,152,83}
\definecolor{info}{RGB}{58,135,173}

\definecolor{required}{HTML}{D9534F}
\definecolor{denied}{HTML}{D9534F}
\definecolor{granted}{HTML}{47A447}
\definecolor{not required}{RGB}{200, 200, 200}

\usepackage[colorlinks=true,pdftitle=doc\_progressreport.pdf
,linktoc=all,linkcolor=RedFire,urlcolor=ParksWildlife]{hyperref}
\usepackage{colortbl}
\usepackage{longtable}
\usepackage{tabu}
\setlength{\tabulinesep}{1.5mm}
\usepackage{enumerate}
\usepackage{enumitem}
\usepackage{fancyhdr}
\usepackage{lastpage}
\usepackage{graphicx}
\usepackage{eso-pic}
\usepackage{hyphenat}
\renewcommand{\familydefault}{\sfdefault}



\newcommand{\HRule}{\rule{\linewidth}{0.1pt}}

\newcommand{\placetextbox}[3]{% \placetextbox{<horizontal pos>}{<vertical pos>}{<stuff>}
  \setbox0=\hbox{#3}% Put <stuff> in a box
  \AddToShipoutPictureFG*{% Add <stuff> to current page foreground
    \put(\LenToUnit{#1\paperwidth},\LenToUnit{#2\paperheight}){\vtop{{\null}\makebox[0pt][c]{#3}}}%
  }%
}%




%-----------------------------------------------------------------------------%
% Headers and footers
%
\fancypagestyle{plain}{
  \fancyhf{}
  \setlength\headheight{60pt} % push page content below header
  \renewcommand{\headrulewidth}{0.1pt}
  \renewcommand{\footrulewidth}{0.1pt}
  
  
  \fancyhead[L]{ 
    \href{http://sdis.dpaw.wa.gov.au}{
    \includegraphics[scale=0.6]{/mnt/projects/sdis/staticfiles/img/logo-dpaw.png}}
  }
  \fancyhead[R]{ 
      \hfill
      \href{http://sdis.dpaw.wa.gov.au}{Science Directorate Information System} 
      \newline 
      \href{http://sdis.dpaw.wa.gov.au/documents/progressreport/1341/}{Progress Report 2012-25 (FY 2014-2015)} 
  }
  
  
  
  
  \fancyfoot[L]{ \leftmark\newline\textbf{Printed}\textit{ June 29, 2015, 9:48 a.m. }}
  \fancyfoot[R]{  \, \newline Page \thepage\ of \pageref{LastPage} }
  
  
}
\pagestyle{plain}
%
% end Headers
%-----------------------------------------------------------------------------%

\begin{document}

%-----------------------------------------------------------------------------%
% Title page
%

%
% end title page
%-----------------------------------------------------------------------------%




\section*{Context Summary}
Barrow Island nature reserve is one of Australia's most important
conservation reserves, particularly for mammal and marine turtle
conservation. It has also been the site of a producing oil field since
1964. In 2003 the WA Government approved the development of the Gorgon
gas field off the north west of Barrow Island, and associated LNG plant
on Barrow Island, subject to several environmental offset conditions.
One of these was the threatened and priority fauna translocation program
that provides for the translocation of selected Barrow Island fauna
species to other secure island and mainland sites. This will assist in
improving the conservation status of these species, and also allow the
reconstruction of the fauna in some areas. It was also an opportunity to
examine the factors affecting translocation success, and improve these
where necessary. Targetted species are the golden bandicoot, brushtail
possum, spectacled hare-wallaby, boodie, water rat, black and white
fairy-wren, and spinifexbird.



\section*{Aims Summary}
\begin{itemize}
\itemsep1pt\parskip0pt\parsep0pt
\item
  Successfully translocate selected mammal and bird species from Barrow
  Island to other secure island and mainland sites.
\item
  Reconstruct the fauna in areas where these species have become locally
  extinct.
\item
  Ensure ongoing appropriate management at the translocation sites,
  particularly introduced predators.
\item
  Develop and refine protocols for fauna translocation and monitoring.
\end{itemize}



\section*{Progress}
\begin{itemize}
\itemsep1pt\parskip0pt\parsep0pt
\item
  Monitored translocated species on the Montebello Islands, Doole Island
  and Lorna Glen. All translocated species at these sites have
  established and increased in abundance.
\item
  No further translocations have occurred since 2012 due to lack of
  sites where feral cats and foxes are adequately under control.
\item
  Taxonomic work on the water rat has shown that the Barrow Island form
  is sufficiently different from the south-west form to warrant
  subspecific status. This has implications for sourcing founder water
  rats for the proposed reintroduction to the Montebello Islands.
\item
  Draft Barrow Island fauna translocation strategy developed.
\item
  Draft business case for an expanded fenced enclosure at Lorna Glen
  prepared.
\item
  Annual report on progress provided to Chevron.
\item
  Project advisory group established.
\end{itemize}



\section*{Management implications}
Arid zone rangelands fauna reconstruction and conservation techniques
developed by this project will have broad state and national
application. The outcomes of the project will contribute to the
management of Parks and Wildlife rangeland properties and provide
guidance for future fauna reconstruction, e.g. Dirk Hartog Island. It
will also contribute to an improvement in the conservation status of
several threatened fauna taxa.



\section*{Future directions}
\begin{itemize}
\itemsep1pt\parskip0pt\parsep0pt
\item
  Research into effectiveness of feral cat baiting at Cape Range to be
  undertaken in August 2014 so that an integrated fox/cat baiting regime
  can be developed for more effective reduction in fox and feral cat
  abundances. This will contribute significantly to this site becoming
  another fauna reconstruction site.
\item
  Ongoing monitoring of the translocated mammals and birds at all the
  release sites. As part of this, a spectacled hare-wallaby monitoring
  program on Hermite Island using remote cameras will be developed.
\item
  Barrow Island fauna translocation strategy to be finalised.
\item
  Continue to plan for an expanded fenced enclosure at Lorna Glen in the
  context of an Indigenous Protected Area agreement with traditional
  owners.
\end{itemize}




\clearpage



\end{document}


\documentclass[version=last,
    paper=a4, % paper size
    10pt, % default font size
    usenames,
    dvipsnames,
    oneside, % ONLINE
    headings=openany, % open chapters on odd and even pages
    %toc=chapterentrywithdots, % Table of Contents style
    %BCOR=7mm, % PRINT Binding Correction
    %DIV=13, % typearea 161.54 mm x 228.46 mm, top margin 22.85 mm, inner margin 16.15 mm
    %DIV=14, % 165.00 233.36 21.21 15.00
    DIV=15 % 168.00 237.60 19.80 14.00
]{scrbook}
\usepackage{typearea}
\usepackage[automark,headsepline,footsepline]{scrlayer-scrpage} % Headers and footers

%%
%% Fonts, encoding, spacing, indentation
%%
\usepackage{txfonts}
\renewcommand{\familydefault}{\sfdefault} % Default to Sans Serif font
\usepackage[english]{babel}
\usepackage[T1]{fontenc}
\usepackage[utf8]{inputenc}

% Paragraph spacing
%\usepackage{parskip}    % Paragraph spacing
%\setlength{\parindent}{0em} % Don't indent paragraphs - ONLINE
%\setlength{\parskip}{1.3 ex plus 0.5ex minus 0.3ex} % 1-1.8 ex vertical space between paragraphs - ONLINE

% Spacing of headings
%\RedeclareSectionCommand[afterskip=3pt]{section} % less space after section
%\RedeclareSectionCommand[beforeskip=0cm]{subsection} % less space between HRule and project name
%\RedeclareSectionCommand[afterskip=0.1\baselineskip]{subsubsection} % less space after progressreport subheadings

% Table font size
\usepackage{etoolbox}
\AtBeginEnvironment{longtabu}{\footnotesize}{}{}

%%
%% Tables, columns, layout
%%
\usepackage{multicol}   % 2 col publications
\usepackage{pdflscape}  % Landscape pages
\usepackage{pdfpages}   % Include PDFs
\usepackage{hanging}    % hanging paragraphs for publications
%\usepackage{titletoc}   % Required for manipulating the table of contents
\setcounter{tocdepth}{2} % TOC list down to section
\usepackage{enumerate}  % Enumerations
\usepackage{enumitem}   % Enumerations
\usepackage{longtable}  % Multipage table
\usepackage{tabu}       %
\setlength{\tabulinesep}{1.5mm} % Consistent vertical spacing in tabu

%%
%% Graphics, images, colours
%%
\usepackage{graphicx} % embedded images
\usepackage{eso-pic} %
\usepackage{colortbl} % define custom named colours
\definecolor{RedFire}{RGB}{146,25,28}
\definecolor{ParksWildlife}{RGB}{0,85,144}
\definecolor{successbg}{RGB}{223,240,216}
\definecolor{errorbg}{RGB}{242,222,222}
\definecolor{warningbg}{RGB}{252,248,227}
\definecolor{infobg}{RGB}{217,237,247}
\definecolor{muted}{RGB}{153,153,153}
\definecolor{success}{RGB}{70,136,71}
\definecolor{error}{RGB}{185,74,72}
\definecolor{warning}{RGB}{192,152,83}
\definecolor{info}{RGB}{58,135,173}

\definecolor{required}{RGB}{192,152,83}
\definecolor{requiredbg}{RGB}{252,248,227}
\definecolor{denied}{RGB}{185,74,72}
\definecolor{deniedbg}{RGB}{242,222,222}
\definecolor{granted}{RGB}{70,136,71}
\definecolor{grantedbg}{RGB}{223,240,216}
\definecolor{not reqiured}{RGB}{153,153,153}
\definecolor{not requiredbg}{RGB}{255,255,255}

\usepackage{tikz} % Drawing
\usetikzlibrary{arrows,shapes,positioning,shadows,trees}

%%
%% Links, URLs
%%
\usepackage[
    linktoc=all,
    %colorlinks=false,  %PRINT
    colorlinks=true, % ONLINE
    linkcolor=RedFire, % ONLINE
    urlcolor=ParksWildlife, % ONLINE
    pdftitle=Progress Report SP 2011-021 (FY 2015-2016)
]{hyperref}

% Black magic to linebreak URLs
\usepackage{url}
\makeatletter
\g@addto@macro{\UrlBreaks}{\UrlOrds}
\makeatother

%%
%% Custom macros
%%
% Thick Horizontal rule
\newcommand{\HRule}{\vspace{8mm}\\\noindent\rule{\linewidth}{0.1pt}}

% Custom Tikz node for SDS diagram
\newcommand\mynode[6][]{
    \node[#1] (#2){
        \parbox{#3\relax}{
            \begin{center}
            \textbf{#4}\\
            #5\\
            \footnotesize{#6}
            \end{center}}};}



%-----------------------------------------------------------------------------%
% Headers and Footers
\automark{section}
\ohead{\href{http://sdis.dpaw.wa.gov.au/documents/progressreport/1631/}{Progress Report SP 2011-021
}}
\chead{\href{http://sdis.dpaw.wa.gov.au}{SDIS}} % center header ONLINE
\ihead{\href{http://sdis.dpaw.wa.gov.au}{
        \includegraphics[scale=0.4]{/mnt/projects/sdis/staticfiles/img/logo-dpaw.png}}}
\ifoot{\textbf{Printed}~Mon, 4 Jul 2016 15:57:14 +0800} % inner/left footer
\cfoot{} % center footer
\ofoot{\pagemark} % outer/right footer
\pagestyle{scrheadings}
\setkomafont{pageheadfoot}{\normalfont}

%-----------------------------------------------------------------------------%
\begin{document}
\raggedbottom

%-----------------------------------------------------------------------------%
% Title page
\subject{Progress Report SP 2011-021
}
\title{Western Australian terrestrial fauna surveys
}
\subtitle{Biogeography
}
\author{}
\publishers{\small
    \subsection*{Project Core Team}
\begin{tabu} {X X}
\textbf{Supervising Scientist} & Lesley Gibson
\\
\textbf{Data Custodian} & Lesley Gibson
\\
\textbf{Site Custodian} & Lesley Gibson
\\
\end{tabu}


    \subsection*{Project status as of July 4, 2016, 3:57 p.m.}
\begin{tabu} {X X}
& Approved and active
\\
\end{tabu}

    
\subsection*{Document endorsements and approvals as of July 4, 2016, 3:57 p.m.}
\begin{tabu} {X X}

%\rowcolor{grantedbg}
    \textbf{Project Team} & 
    \textcolor{granted}{ granted}\\

%\rowcolor{grantedbg}
    \textbf{Program Leader} & 
    \textcolor{granted}{ granted}\\

%\rowcolor{grantedbg}
    \textbf{Directorate} & 
    \textcolor{granted}{ granted}\\

\end{tabu}



}
\uppertitleback{}
\lowertitleback{}
\date{}

%-----------------------------------------------------------------------------%
% Front matter
\frontmatter
\maketitle
%-----------------------------------------------------------------------------%
% Main matter
\mainmatter

\section*{Western Australian terrestrial fauna surveys
}

M Cowan, L Gibson, AH Burbidge, D Pearson


\section*{Context}
The Department with the assistance of the Western Australian Museum has
a long-standing commitment to undertaking regional biogeographic surveys
of the State. These surveys have underpinned the selection of areas for
the conservation reserve system, provided information to determine the
conservation status of species and filled significant gaps in
biodiversity knowledge. While these large scale surveys provide analyses
of biodiversity patterning for regional-scale conservation planning,
sites are usually too sparse and often lack detail at finer scales. The
more localised surveys undertaken will fill spatial and/or habitat gaps
in the larger regional surveys, extend geographic coverage, assist in
resolving taxonomic issues, increase ecological understanding, provide
information on fine-scale biodiversity patterns and in many cases
complement regional surveys.



\section*{Aims}
\begin{itemize}
\itemsep1pt\parskip0pt\parsep0pt
\item
  Provide understanding of landscape-scale terrestrial fauna
  biodiversity and concomitant patterning in terrestrial fauna to inform
  conservation planning and as a baseline for future monitoring.
\item
  Collect, manage and interpret data on the distribution, ecological
  tolerances and conservation status of terrestrial fauna species and
  communities.
\end{itemize}



\section*{Progress}
\begin{itemize}
\itemsep1pt\parskip0pt\parsep0pt
\item
  In September 2015, a survey in a remote part of the northern
  Nullarbor~was undertaken (Colville 1:100,000 map sheet) in
  collaboration with the Goldfields Region and a report on the results
  of the vertebrate survey was produced.
\item
  In September 2015 a~ survey of the area around~Kiwirrkurra in the
  esatern Gibson Desert was undertaken. This ~was ~a collaboration
  between the ~Traditional Owners, the Western Australian Museum~and~the
  Australian Government's BushBlitz Program. The work focused in an area
  with little~terrestrial vertebrate information, and along with
  extensive DNA collections,~populations of ~several rare species~were
  identified. A report on this work has been completed and all data have
  been submitted to the BushBlitz Team. Genetic samples and voucher
  specimens have been lodged with the Western Australian Museum.
\item
  Survey and monitoring guidelines~for the sandhill dunnart
  (\emph{Sminthopsis psammophila}), and a baseline survey design in the
  Great Victoria Desert,~were completed.
\end{itemize}

~

~

~



\section*{Management implications}
The immediate aims of the individual projects vary depending on the
needs of the funding source, but usually contribute to improved species
distributional and ecological understanding, prioritisation of
conservation actions by local managers, and/or the assessment of
potential environmental impacts of land use development proposals.
Individual survey projects assist regional conservation and land
managers to understand local biodiversity patterning and its underlying
drivers, and permit the use of this information to assess environmental
impacts, prioritise conservation actions, set biodiversity management
targets, establish baselines for monitoring and monitor change. The
combination of surveys enables improved understanding of species
distributions and habitat requirements at a State level, thus
contributing to bioregional analyses, assessment of the design of the
conservation estate, reviews of species conservation status and analyses
of the relationships between species and broad-scale gradients and
threats such as climate change.



\section*{Future directions}
\begin{itemize}
\itemsep1pt\parskip0pt\parsep0pt
\item
  Colville map sheet~data uploaded to NatureMap.
\item
  Peterswald map sheet vertebrate report completed.
\item
  The development of collaborative arrangements to facilitate future
  surveys is ongoing.
\end{itemize}

~



%-----------------------------------------------------------------------------%
% Back matter
%\backmatter
\end{document}
%-----------------------------------------------------------------------------%


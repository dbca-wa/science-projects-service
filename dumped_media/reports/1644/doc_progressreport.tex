
\documentclass[version=last,
    paper=a4, % paper size
    10pt, % default font size
    usenames,
    dvipsnames,
    oneside, % ONLINE
    headings=openany, % open chapters on odd and even pages
    %toc=chapterentrywithdots, % Table of Contents style
    %BCOR=7mm, % PRINT Binding Correction
    %DIV=13, % typearea 161.54 mm x 228.46 mm, top margin 22.85 mm, inner margin 16.15 mm
    %DIV=14, % 165.00 233.36 21.21 15.00
    DIV=15 % 168.00 237.60 19.80 14.00
]{scrbook}
\usepackage{typearea}
\usepackage[automark,headsepline,footsepline]{scrlayer-scrpage} % Headers and footers

%%
%% Fonts, encoding, spacing, indentation
%%
\usepackage{txfonts}
\renewcommand{\familydefault}{\sfdefault} % Default to Sans Serif font
\usepackage[english]{babel}
\usepackage[T1]{fontenc}
\usepackage[utf8]{inputenc}

% Paragraph spacing
%\usepackage{parskip}    % Paragraph spacing
%\setlength{\parindent}{0em} % Don't indent paragraphs - ONLINE
%\setlength{\parskip}{1.3 ex plus 0.5ex minus 0.3ex} % 1-1.8 ex vertical space between paragraphs - ONLINE

% Spacing of headings
%\RedeclareSectionCommand[afterskip=3pt]{section} % less space after section
%\RedeclareSectionCommand[beforeskip=0cm]{subsection} % less space between HRule and project name
%\RedeclareSectionCommand[afterskip=0.1\baselineskip]{subsubsection} % less space after progressreport subheadings

% Table font size
\usepackage{etoolbox}
\AtBeginEnvironment{longtabu}{\footnotesize}{}{}

%%
%% Tables, columns, layout
%%
\usepackage{multicol}   % 2 col publications
\usepackage{pdflscape}  % Landscape pages
\usepackage{pdfpages}   % Include PDFs
\usepackage{hanging}    % hanging paragraphs for publications
%\usepackage{titletoc}   % Required for manipulating the table of contents
\setcounter{tocdepth}{2} % TOC list down to section
\usepackage{enumerate}  % Enumerations
\usepackage{enumitem}   % Enumerations
\usepackage{longtable}  % Multipage table
\usepackage{tabu}       %
\setlength{\tabulinesep}{1.5mm} % Consistent vertical spacing in tabu

%%
%% Graphics, images, colours
%%
\usepackage{graphicx} % embedded images
\usepackage{eso-pic} %
\usepackage{colortbl} % define custom named colours
\definecolor{RedFire}{RGB}{146,25,28}
\definecolor{ParksWildlife}{RGB}{0,85,144}
\definecolor{successbg}{RGB}{223,240,216}
\definecolor{errorbg}{RGB}{242,222,222}
\definecolor{warningbg}{RGB}{252,248,227}
\definecolor{infobg}{RGB}{217,237,247}
\definecolor{muted}{RGB}{153,153,153}
\definecolor{success}{RGB}{70,136,71}
\definecolor{error}{RGB}{185,74,72}
\definecolor{warning}{RGB}{192,152,83}
\definecolor{info}{RGB}{58,135,173}

\definecolor{required}{RGB}{192,152,83}
\definecolor{requiredbg}{RGB}{252,248,227}
\definecolor{denied}{RGB}{185,74,72}
\definecolor{deniedbg}{RGB}{242,222,222}
\definecolor{granted}{RGB}{70,136,71}
\definecolor{grantedbg}{RGB}{223,240,216}
\definecolor{not reqiured}{RGB}{153,153,153}
\definecolor{not requiredbg}{RGB}{255,255,255}

\usepackage{tikz} % Drawing
\usetikzlibrary{arrows,shapes,positioning,shadows,trees}

%%
%% Links, URLs
%%
\usepackage[
    linktoc=all,
    %colorlinks=false,  %PRINT
    colorlinks=true, % ONLINE
    linkcolor=RedFire, % ONLINE
    urlcolor=ParksWildlife, % ONLINE
    pdftitle=Progress Report SP 2010-008 (FY 2015-2016)
]{hyperref}

% Black magic to linebreak URLs
\usepackage{url}
\makeatletter
\g@addto@macro{\UrlBreaks}{\UrlOrds}
\makeatother

%%
%% Custom macros
%%
% Thick Horizontal rule
\newcommand{\HRule}{\vspace{8mm}\\\noindent\rule{\linewidth}{0.1pt}}

% Custom Tikz node for SDS diagram
\newcommand\mynode[6][]{
    \node[#1] (#2){
        \parbox{#3\relax}{
            \begin{center}
            \textbf{#4}\\
            #5\\
            \footnotesize{#6}
            \end{center}}};}



%-----------------------------------------------------------------------------%
% Headers and Footers
\automark{section}
\ohead{\href{http://sdis.dpaw.wa.gov.au/documents/progressreport/1644/}{Progress Report SP 2010-008
}}
\chead{\href{http://sdis.dpaw.wa.gov.au}{SDIS}} % center header ONLINE
\ihead{\href{http://sdis.dpaw.wa.gov.au}{
        \includegraphics[scale=0.4]{/mnt/projects/sdis/staticfiles/img/logo-dpaw.png}}}
\ifoot{\textbf{Printed}~Mon, 11 Jul 2016 13:52:48 +0800} % inner/left footer
\cfoot{} % center footer
\ofoot{\pagemark} % outer/right footer
\pagestyle{scrheadings}
\setkomafont{pageheadfoot}{\normalfont}

%-----------------------------------------------------------------------------%
\begin{document}
\raggedbottom

%-----------------------------------------------------------------------------%
% Title page
\subject{Progress Report SP 2010-008
}
\title{Effects of the Gorgon Project dredging program on the marine
biodiversity of the Montebello/Barrow Islands marine protected areas
}
\subtitle{Marine Science
}
\author{}
\publishers{\small
    \subsection*{Project Core Team}
\begin{tabu} {X X}
\textbf{Supervising Scientist} & Alan Kendrick
\\
\textbf{Data Custodian} & Stuart Field
\\
\textbf{Site Custodian} & Stuart Field
\\
\end{tabu}


    \subsection*{Project status as of July 11, 2016, 1:52 p.m.}
\begin{tabu} {X X}
& Approved and active
\\
\end{tabu}

    
\subsection*{Document endorsements and approvals as of July 11, 2016, 1:52 p.m.}
\begin{tabu} {X X}

%\rowcolor{grantedbg}
    \textbf{Project Team} & 
    \textcolor{granted}{ granted}\\

%\rowcolor{grantedbg}
    \textbf{Program Leader} & 
    \textcolor{granted}{ granted}\\

%\rowcolor{grantedbg}
    \textbf{Directorate} & 
    \textcolor{granted}{ granted}\\

\end{tabu}



}
\uppertitleback{}
\lowertitleback{}
\date{}

%-----------------------------------------------------------------------------%
% Front matter
\frontmatter
\maketitle
%-----------------------------------------------------------------------------%
% Main matter
\mainmatter

\section*{Effects of the Gorgon Project dredging program on the marine
biodiversity of the Montebello/Barrow Islands marine protected areas
}

S Field, RD Evans, K Friedman, G Shedrawi


\section*{Context}
The Gorgon Project (GP), which is based on Barrow Island, is one of the
world's largest natural gas projects and the largest single-resource
natural gas project in Australia's history. The plant will include three
5-million-tonne-per-annum LNG trains, with domestic gas piped to the
mainland, and a four-kilometre-long loading jetty for international
shipping.

The GP includes a dredging program that involves the removal and dumping
of approximately 7.6 M tonnes of marine sediment over a period of
approximately 18 months. The Gorgon Dredging Offset Monitoring
Evaluation and Reporting Project (Gorgon MER) will investigate the
potential impacts of the dredging and dumping activities on selected
marine communities within the Montebello/Barrow Islands marine protected
areas (MBIMPA). This monitoring will also help inform future
environmental impact assessments by improving predictions of the spatial
scale and nature of the likely impacts of dredging and dumping
activities on sensitive marine communities. Additionally, this project
will increase the knowledge base of the MBIMPA.



\section*{Aims}
\begin{itemize}
\itemsep1pt\parskip0pt\parsep0pt
\item
  Assess the nature and extent of potential impacts of the Gorgon
  dredging program on the condition of coral, fish and other important
  ecological communities of the MBIMPA.
\item
  Determine the cause/s of any changes in the condition of the above
  communities, with particular focus on dredging, dumping and
  resuspension of spoil.
\item
  Assess the effects of potential confounding natural (e.g. cyclones,
  disease, predation, bleaching) and other anthropogenic (e.g. fishing)
  pressures on the condition of coral communities of the MBIMPA.
\item
  Assess the nature and extent of the impacts from the Gorgon dredging
  program on the social assets of the MBIMPA.
\end{itemize}



\section*{Progress}
\begin{itemize}
\itemsep1pt\parskip0pt\parsep0pt
\item
  Progress has continued on writing the final Gorgon dredging program
  report, which describes potential impacts of marine construction on
  bio-physical assets. All chapters have now been internally reviewed.
\item
  A pilot study to examine the utility of identifying coral disease from
  digital images was carried out.
\end{itemize}



\section*{Management implications}
\begin{itemize}
\itemsep1pt\parskip0pt\parsep0pt
\item
  Phase One of the Gorgon MER project provides Department managers and
  scientists with a relatively intensive baseline for assessing
  potential impacts on, and recovery of, coral communities within the
  MBIMPA, with a particular focus on potential impacts related to the
  dredging program for the Gorgon Project. Information outputs include
  temporal condition and related pressure measures for biophysical
  assets (e.g. coral, finfish and macro-invertebrate communities) that
  aligns with the Department's marine monitoring program for the MBIMPA.
\item
  The data generated from this monitoring program will also complement
  Offset E of the Pluto LNG program aimed at improving the capacity of
  government and industry to manage the impacts of dredging on tropical
  coral reef communities. The Gorgon MER project also strategically
  assists the planning for future environmental impact assessments by
  improving predictions of the spatial scale and nature of the likely
  impacts of dredging and dumping activities on sensitive marine
  communities.
\end{itemize}



\section*{Future directions}
\begin{itemize}
\itemsep1pt\parskip0pt\parsep0pt
\item
  Finalisation and publication of the Gorgon MER Phase One final project
  report.
\item
  Initiation of fieldwork for Gorgon MER Phase Two (longer-term
  strategic monitoring) that is closely linked to the activity and
  reporting of the Western Australian marine Monitoring Program.
\item
  Completion of peer reviewed publications and archiving of all data
  collected.
\end{itemize}



%-----------------------------------------------------------------------------%
% Back matter
%\backmatter
\end{document}
%-----------------------------------------------------------------------------%


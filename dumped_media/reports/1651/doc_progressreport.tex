
\documentclass[version=last,
    paper=a4, % paper size
    10pt, % default font size
    usenames,
    dvipsnames,
    oneside, % ONLINE
    headings=openany, % open chapters on odd and even pages
    %toc=chapterentrywithdots, % Table of Contents style
    %BCOR=7mm, % PRINT Binding Correction
    %DIV=13, % typearea 161.54 mm x 228.46 mm, top margin 22.85 mm, inner margin 16.15 mm
    %DIV=14, % 165.00 233.36 21.21 15.00
    DIV=15 % 168.00 237.60 19.80 14.00
]{scrbook}
\usepackage{typearea}
\usepackage[automark,headsepline,footsepline]{scrlayer-scrpage} % Headers and footers

%%
%% Fonts, encoding, spacing, indentation
%%
\usepackage{txfonts}
\renewcommand{\familydefault}{\sfdefault} % Default to Sans Serif font
\usepackage[english]{babel}
\usepackage[T1]{fontenc}
\usepackage[utf8]{inputenc}

% Paragraph spacing
%\usepackage{parskip}    % Paragraph spacing
%\setlength{\parindent}{0em} % Don't indent paragraphs - ONLINE
%\setlength{\parskip}{1.3 ex plus 0.5ex minus 0.3ex} % 1-1.8 ex vertical space between paragraphs - ONLINE

% Spacing of headings
%\RedeclareSectionCommand[afterskip=3pt]{section} % less space after section
%\RedeclareSectionCommand[beforeskip=0cm]{subsection} % less space between HRule and project name
%\RedeclareSectionCommand[afterskip=0.1\baselineskip]{subsubsection} % less space after progressreport subheadings

% Table font size
\usepackage{etoolbox}
\AtBeginEnvironment{longtabu}{\footnotesize}{}{}

%%
%% Tables, columns, layout
%%
\usepackage{multicol}   % 2 col publications
\usepackage{pdflscape}  % Landscape pages
\usepackage{pdfpages}   % Include PDFs
\usepackage{hanging}    % hanging paragraphs for publications
%\usepackage{titletoc}   % Required for manipulating the table of contents
\setcounter{tocdepth}{2} % TOC list down to section
\usepackage{enumerate}  % Enumerations
\usepackage{enumitem}   % Enumerations
\usepackage{longtable}  % Multipage table
\usepackage{tabu}       %
\setlength{\tabulinesep}{1.5mm} % Consistent vertical spacing in tabu

%%
%% Graphics, images, colours
%%
\usepackage{graphicx} % embedded images
\usepackage{eso-pic} %
\usepackage{colortbl} % define custom named colours
\definecolor{RedFire}{RGB}{146,25,28}
\definecolor{ParksWildlife}{RGB}{0,85,144}
\definecolor{successbg}{RGB}{223,240,216}
\definecolor{errorbg}{RGB}{242,222,222}
\definecolor{warningbg}{RGB}{252,248,227}
\definecolor{infobg}{RGB}{217,237,247}
\definecolor{muted}{RGB}{153,153,153}
\definecolor{success}{RGB}{70,136,71}
\definecolor{error}{RGB}{185,74,72}
\definecolor{warning}{RGB}{192,152,83}
\definecolor{info}{RGB}{58,135,173}

\definecolor{required}{RGB}{192,152,83}
\definecolor{requiredbg}{RGB}{252,248,227}
\definecolor{denied}{RGB}{185,74,72}
\definecolor{deniedbg}{RGB}{242,222,222}
\definecolor{granted}{RGB}{70,136,71}
\definecolor{grantedbg}{RGB}{223,240,216}
\definecolor{not reqiured}{RGB}{153,153,153}
\definecolor{not requiredbg}{RGB}{255,255,255}

\usepackage{tikz} % Drawing
\usetikzlibrary{arrows,shapes,positioning,shadows,trees}

%%
%% Links, URLs
%%
\usepackage[
    linktoc=all,
    %colorlinks=false,  %PRINT
    colorlinks=true, % ONLINE
    linkcolor=RedFire, % ONLINE
    urlcolor=ParksWildlife, % ONLINE
    pdftitle=Progress Report SP 2009-006 (FY 2015-2016)
]{hyperref}

% Black magic to linebreak URLs
\usepackage{url}
\makeatletter
\g@addto@macro{\UrlBreaks}{\UrlOrds}
\makeatother

%%
%% Custom macros
%%
% Thick Horizontal rule
\newcommand{\HRule}{\vspace{8mm}\\\noindent\rule{\linewidth}{0.1pt}}

% Custom Tikz node for SDS diagram
\newcommand\mynode[6][]{
    \node[#1] (#2){
        \parbox{#3\relax}{
            \begin{center}
            \textbf{#4}\\
            #5\\
            \footnotesize{#6}
            \end{center}}};}



%-----------------------------------------------------------------------------%
% Headers and Footers
\automark{section}
\ohead{\href{http://sdis.dpaw.wa.gov.au/documents/progressreport/1651/}{Progress Report SP 2009-006
}}
\chead{\href{http://sdis.dpaw.wa.gov.au}{SDIS}} % center header ONLINE
\ihead{\href{http://sdis.dpaw.wa.gov.au}{
        \includegraphics[scale=0.4]{/mnt/projects/sdis/staticfiles/img/logo-dpaw.png}}}
\ifoot{\textbf{Printed}~Fri, 14 Oct 2016 17:23:25 +0800} % inner/left footer
\cfoot{} % center footer
\ofoot{\pagemark} % outer/right footer
\pagestyle{scrheadings}
\setkomafont{pageheadfoot}{\normalfont}

%-----------------------------------------------------------------------------%
\begin{document}
\raggedbottom

%-----------------------------------------------------------------------------%
% Title page
\subject{Progress Report SP 2009-006
}
\title{Taxonomic resolution and description of new plant species, particularly
Priority Flora from those areas subject to mining in Western Australia
}
\subtitle{Plant Science and Herbarium
}
\author{}
\publishers{\small
    \subsection*{Project Core Team}
\begin{tabu} {X X}
\textbf{Supervising Scientist} & Juliet Wege
\\
\textbf{Data Custodian} & Juliet Wege
\\
\textbf{Site Custodian} & Juliet Wege
\\
\end{tabu}


    \subsection*{Project status as of Oct. 14, 2016, 5:23 p.m.}
\begin{tabu} {X X}
& Approved and active
\\
\end{tabu}

    
\subsection*{Document endorsements and approvals as of Oct. 14, 2016, 5:23 p.m.}
\begin{tabu} {X X}

%\rowcolor{grantedbg}
    \textbf{Project Team} & 
    \textcolor{granted}{ granted}\\

%\rowcolor{grantedbg}
    \textbf{Program Leader} & 
    \textcolor{granted}{ granted}\\

%\rowcolor{grantedbg}
    \textbf{Directorate} & 
    \textcolor{granted}{ granted}\\

\end{tabu}



}
\uppertitleback{}
\lowertitleback{}
\date{}

%-----------------------------------------------------------------------------%
% Front matter
\frontmatter
\maketitle
%-----------------------------------------------------------------------------%
% Main matter
\mainmatter

\section*{Taxonomic resolution and description of new plant species, particularly
Priority Flora from those areas subject to mining in Western Australia
}

J Wege, KA Shepherd, R Butcher, B Rye, T Macfarlane, M Hislop, S Dillon,
A Perkins, R Davis


\section*{Context}
Western Australia has a rich flora that is far from fully known. New
species continue to be discovered through the taxonomic assessment of
herbarium collections, floristic surveys and the botanical assessment of
mineral leases. There are \emph{c.} 1350 putatively new and undescribed
taxa currently recorded in Western Australia, a significant proportion
of which are poorly known, geographically restricted and/or under threat
(i.e. Threatened or Priority Flora). The lack of detailed information on
these taxa makes accurate identification problematic and inevitably
delays the Department's ability to survey and accurately assess their
conservation status.



\section*{Aims}
Resolve the taxonomy and expedite the description of manuscript or
phrase-named plant taxa, particularly Threatened and Priority Flora and
those taxa vulnerable to future mining activities.



\section*{Progress}
\begin{itemize}
\itemsep1pt\parskip0pt\parsep0pt
\item
  A Threatened \emph{Atriplex} and a rare \emph{Tricoryne} were
  published in \emph{Australian Systematic Botany}.
\item
  21 new conservation-listed taxa in the genera \emph{Babingtonia},
  \emph{Hibbertia}, \emph{Lasiopetalum}, \emph{Leucopogon},
  \emph{Malleostemon}, \emph{Ptilotus} and \emph{Stylidium} were
  published in \emph{Nuytsia}.
\item
  A further 50 new species (of which 41 are conservation-listed) were
  published in a special issue of \emph{Nuytsia} as a result of a
  collaboration between the Botanic Gardens and Parks Authority and
  staff and research associates of the Western Australian Herbarium.
\item
  Nine phrase-named taxa under \emph{Dampiera}, \emph{Goodenia},
  \emph{Leucopogon}, \emph{Philotheca}, \emph{Scaevola}, \emph{Senna},
  \emph{Stylidium} and \emph{Vigna} were found to be synonymous with
  known species. Papers recommending their removal from WACensus (and in
  several cases the \emph{Threatened and Priority Flora list} \emph{for
  Western Australia}) were published in \emph{Nuytsia}.
\item
  \emph{Ricinocarpos} Eastern Goldfields (A. Williams 3), a putative new
  species from mineral leases in the Coolgardie region, was added to
  WACensus and the \emph{Threatened and Priority Flora list} \emph{for
  Western Australia}.
\item
  Papers describing new, conservation-listed species of
  \emph{Allocasuarina}, \emph{Cochorus}, \emph{Dysphania},
  \emph{Lasiopetalum}, \emph{Leucopogon} and \emph{Tetratheca} and
  \emph{Trachymene} have been progressed.
\end{itemize}



\section*{Management implications}
\begin{itemize}
\itemsep1pt\parskip0pt\parsep0pt
\item
  The provision of names, scientific descriptions, illustrations and
  associated data will enhance the capacity of conservation and industry
  practioners to identify new species, thereby improving species
  management, conservation assessments and land use planning.
\end{itemize}



\section*{Future directions}
\begin{itemize}
\itemsep1pt\parskip0pt\parsep0pt
\item
  Identify and formally describe new taxa of conservation significance.
\end{itemize}



%-----------------------------------------------------------------------------%
% Back matter
%\backmatter
\end{document}
%-----------------------------------------------------------------------------%


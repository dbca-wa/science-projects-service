
\documentclass[version=last,
    paper=a4, % paper size
    10pt, % default font size
    usenames,
    dvipsnames,
    oneside, % ONLINE
    headings=openany, % open chapters on odd and even pages
    %toc=chapterentrywithdots, % Table of Contents style
    %BCOR=7mm, % PRINT Binding Correction
    %DIV=13, % typearea 161.54 mm x 228.46 mm, top margin 22.85 mm, inner margin 16.15 mm
    %DIV=14, % 165.00 233.36 21.21 15.00
    DIV=15 % 168.00 237.60 19.80 14.00
]{scrbook}
\usepackage{typearea}
\usepackage[automark,headsepline,footsepline]{scrlayer-scrpage} % Headers and footers

%%
%% Fonts, encoding, spacing, indentation
%%
\usepackage{txfonts}
\renewcommand{\familydefault}{\sfdefault} % Default to Sans Serif font
\usepackage[english]{babel}
\usepackage[T1]{fontenc}
\usepackage[utf8]{inputenc}

% Paragraph spacing
%\usepackage{parskip}    % Paragraph spacing
%\setlength{\parindent}{0em} % Don't indent paragraphs - ONLINE
%\setlength{\parskip}{1.3 ex plus 0.5ex minus 0.3ex} % 1-1.8 ex vertical space between paragraphs - ONLINE

% Spacing of headings
%\RedeclareSectionCommand[afterskip=3pt]{section} % less space after section
%\RedeclareSectionCommand[beforeskip=0cm]{subsection} % less space between HRule and project name
%\RedeclareSectionCommand[afterskip=0.1\baselineskip]{subsubsection} % less space after progressreport subheadings

% Table font size
\usepackage{etoolbox}
\AtBeginEnvironment{longtabu}{\footnotesize}{}{}

%%
%% Tables, columns, layout
%%
\usepackage{multicol}   % 2 col publications
\usepackage{pdflscape}  % Landscape pages
\usepackage{pdfpages}   % Include PDFs
\usepackage{hanging}    % hanging paragraphs for publications
%\usepackage{titletoc}   % Required for manipulating the table of contents
\setcounter{tocdepth}{2} % TOC list down to section
\usepackage{enumerate}  % Enumerations
\usepackage{enumitem}   % Enumerations
\usepackage{longtable}  % Multipage table
\usepackage{tabu}       %
\setlength{\tabulinesep}{1.5mm} % Consistent vertical spacing in tabu

%%
%% Graphics, images, colours
%%
\usepackage{graphicx} % embedded images
\usepackage{eso-pic} %
\usepackage{colortbl} % define custom named colours
\definecolor{RedFire}{RGB}{146,25,28}
\definecolor{ParksWildlife}{RGB}{0,85,144}
\definecolor{successbg}{RGB}{223,240,216}
\definecolor{errorbg}{RGB}{242,222,222}
\definecolor{warningbg}{RGB}{252,248,227}
\definecolor{infobg}{RGB}{217,237,247}
\definecolor{muted}{RGB}{153,153,153}
\definecolor{success}{RGB}{70,136,71}
\definecolor{error}{RGB}{185,74,72}
\definecolor{warning}{RGB}{192,152,83}
\definecolor{info}{RGB}{58,135,173}

\definecolor{required}{RGB}{192,152,83}
\definecolor{requiredbg}{RGB}{252,248,227}
\definecolor{denied}{RGB}{185,74,72}
\definecolor{deniedbg}{RGB}{242,222,222}
\definecolor{granted}{RGB}{70,136,71}
\definecolor{grantedbg}{RGB}{223,240,216}
\definecolor{not reqiured}{RGB}{153,153,153}
\definecolor{not requiredbg}{RGB}{255,255,255}

\usepackage{tikz} % Drawing
\usetikzlibrary{arrows,shapes,positioning,shadows,trees}

%%
%% Links, URLs
%%
\usepackage[
    linktoc=all,
    %colorlinks=false,  %PRINT
    colorlinks=true, % ONLINE
    linkcolor=RedFire, % ONLINE
    urlcolor=ParksWildlife, % ONLINE
    pdftitle=Progress Report SP 2009-002 (FY 2015-2016)
]{hyperref}

% Black magic to linebreak URLs
\usepackage{url}
\makeatletter
\g@addto@macro{\UrlBreaks}{\UrlOrds}
\makeatother

%%
%% Custom macros
%%
% Thick Horizontal rule
\newcommand{\HRule}{\vspace{8mm}\\\noindent\rule{\linewidth}{0.1pt}}

% Custom Tikz node for SDS diagram
\newcommand\mynode[6][]{
    \node[#1] (#2){
        \parbox{#3\relax}{
            \begin{center}
            \textbf{#4}\\
            #5\\
            \footnotesize{#6}
            \end{center}}};}



%-----------------------------------------------------------------------------%
% Headers and Footers
\automark{section}
\ohead{\href{http://sdis.dpaw.wa.gov.au/documents/progressreport/1653/}{Progress Report SP 2009-002
}}
\chead{\href{http://sdis.dpaw.wa.gov.au}{SDIS}} % center header ONLINE
\ihead{\href{http://sdis.dpaw.wa.gov.au}{
        \includegraphics[scale=0.4]{/mnt/projects/sdis/staticfiles/img/logo-dpaw.png}}}
\ifoot{\textbf{Printed}~Mon, 11 Jul 2016 16:09:54 +0800} % inner/left footer
\cfoot{} % center footer
\ofoot{\pagemark} % outer/right footer
\pagestyle{scrheadings}
\setkomafont{pageheadfoot}{\normalfont}

%-----------------------------------------------------------------------------%
\begin{document}
\raggedbottom

%-----------------------------------------------------------------------------%
% Title page
\subject{Progress Report SP 2009-002
}
\title{Spatial and temporal patterns in the structure of intertidal reef
communities in the marine parks of south-western Australia
}
\subtitle{Marine Science
}
\author{}
\publishers{\small
    \subsection*{Project Core Team}
\begin{tabu} {X X}
\textbf{Supervising Scientist} & Alan Kendrick
\\
\textbf{Data Custodian} & 
\\
\textbf{Site Custodian} & 
\\
\end{tabu}


    \subsection*{Project status as of July 11, 2016, 4:09 p.m.}
\begin{tabu} {X X}
& Approved and active
\\
\end{tabu}

    
\subsection*{Document endorsements and approvals as of July 11, 2016, 4:09 p.m.}
\begin{tabu} {X X}

%\rowcolor{grantedbg}
    \textbf{Project Team} & 
    \textcolor{granted}{ granted}\\

%\rowcolor{grantedbg}
    \textbf{Program Leader} & 
    \textcolor{granted}{ granted}\\

%\rowcolor{grantedbg}
    \textbf{Directorate} & 
    \textcolor{granted}{ granted}\\

\end{tabu}



}
\uppertitleback{}
\lowertitleback{}
\date{}

%-----------------------------------------------------------------------------%
% Front matter
\frontmatter
\maketitle
%-----------------------------------------------------------------------------%
% Main matter
\mainmatter

\section*{Spatial and temporal patterns in the structure of intertidal reef
communities in the marine parks of south-western Australia
}

M Rule, A Kendrick, J Huisman


\section*{Context}
The Marmion Marine Park (MMP) and Shoalwater Islands Marine Park (SIMP)
are located on the north and south Perth metropolitan coast,
respectively, while Ngari Capes Marine Park (NCMP) is in WA's
south-west. These marine parks support a diverse range of marine
conservation values ranging from various marine habitats to threatened
marine fauna, and are dominated by sub-tidal and emergent limestone
reefs and shallow sandy embayments. The marine parks are subject to high
levels of recreational and commercial human activity due to their
proximity to population centres. Significant areas of intertidal reef
platform occur in both mainland and island shores and as isolated
offshore patch reefs. While a number of local studies of intertidal
communities provide a significant regional knowledge base, the broad
spatial patterns of intertidal biodiversity across MMP, SIMP and NCMP
are not adequately understood. Particular gaps exist in our knowledge of
the intertidal communities of offshore platform reefs. This study will
determine spatial and temporal patterns in the distribution of
intertidal reef organisms in WA's temperate marine reserves.
Relationships between the composition of these communities and the
physical structure and location of the reefs will also be examined.



\section*{Aims}
\begin{itemize}
\itemsep1pt\parskip0pt\parsep0pt
\item
  Determine the spatial and temporal patterns in the composition of
  intertidal reef communities in the MMP, SIMP and NCMP, including the
  proposed northern extension to the SIMP.
\item
  Determine if the intertidal reef communities in management zones
  protected from extractive activities differ from the intertidal reef
  communities of otherwise comparable reefs.
\item
  Assist in the development of methods for long-term monitoring of
  intertidal communities in temperate marine reserves.
\end{itemize}



\section*{Progress}
\begin{itemize}
\itemsep1pt\parskip0pt\parsep0pt
\item
  A manuscript describing differences in community structure on
  intertidal granite and limestone substrates at NCNP was initiated.
\item
  Preliminary intertidal surveys were commenced at six new sites at
  Jurien Bay Marine Park in conjunction with regional staff. A total of
  37 species were recorded during these surveys, 13 of which had not
  been recorded in previous surveys of temperate marine reserves.
\item
  A collaboration to investigate intertidal ecology with researchers
  from CSIRO, the Western Australian Museum and Southern Cross
  University has been scoped and in ready for implementation.
\end{itemize}



\section*{Management implications}
\begin{itemize}
\itemsep1pt\parskip0pt\parsep0pt
\item
  As the first comprehensive spatial and temporal study of the
  biological communities associated with intertidal reefs of Western
  Australia's temperate marine parks and reserves, this work provide a
  baseline understanding of intertidal reef condition in the marine
  reserves in relation to natural processes and possible anthropogenic
  impacts. The results will assist to ascertain the conservation
  significance of these habitats and assist the implementation of
  long-term intertidal reef monitoring and management in marine parks
  and reserves.
\end{itemize}

~



\section*{Future directions}
\begin{itemize}
\itemsep1pt\parskip0pt\parsep0pt
\item
  Ecological papers from data collected in MMP, SIMP and NCMP will be
  published.
\item
  A species identification guide to assist future intertidal reef
  monitoring in MMP and SIMP will be completed.
\item
  Intertidal reef surveys at NCMP will continue.
\item
  A pilot survey~examining broad-scale, latitudinal patterns in
  diversity with researchers from CSIRO, the Western Australian Museum
  and Southern Cross University will be conducted in the MMP~and SIMP
  during the summer of 2016/17
\end{itemize}



%-----------------------------------------------------------------------------%
% Back matter
%\backmatter
\end{document}
%-----------------------------------------------------------------------------%


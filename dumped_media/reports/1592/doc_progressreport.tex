
\documentclass[version=last,
    paper=a4, % paper size
    10pt, % default font size
    usenames,
    dvipsnames,
    oneside, % ONLINE
    headings=openany, % open chapters on odd and even pages
    %toc=chapterentrywithdots, % Table of Contents style
    %BCOR=7mm, % PRINT Binding Correction
    %DIV=13, % typearea 161.54 mm x 228.46 mm, top margin 22.85 mm, inner margin 16.15 mm
    %DIV=14, % 165.00 233.36 21.21 15.00
    DIV=15 % 168.00 237.60 19.80 14.00
]{scrbook}
\usepackage{typearea}
\usepackage[automark,headsepline,footsepline]{scrlayer-scrpage} % Headers and footers

%%
%% Fonts, encoding, spacing, indentation
%%
\usepackage{txfonts}
\renewcommand{\familydefault}{\sfdefault} % Default to Sans Serif font
\usepackage[english]{babel}
\usepackage[T1]{fontenc}
\usepackage[utf8]{inputenc}

% Paragraph spacing
%\usepackage{parskip}    % Paragraph spacing
%\setlength{\parindent}{0em} % Don't indent paragraphs - ONLINE
%\setlength{\parskip}{1.3 ex plus 0.5ex minus 0.3ex} % 1-1.8 ex vertical space between paragraphs - ONLINE

% Spacing of headings
%\RedeclareSectionCommand[afterskip=3pt]{section} % less space after section
%\RedeclareSectionCommand[beforeskip=0cm]{subsection} % less space between HRule and project name
%\RedeclareSectionCommand[afterskip=0.1\baselineskip]{subsubsection} % less space after progressreport subheadings

% Table font size
\usepackage{etoolbox}
\AtBeginEnvironment{longtabu}{\footnotesize}{}{}

%%
%% Tables, columns, layout
%%
\usepackage{multicol}   % 2 col publications
\usepackage{pdflscape}  % Landscape pages
\usepackage{pdfpages}   % Include PDFs
\usepackage{hanging}    % hanging paragraphs for publications
%\usepackage{titletoc}   % Required for manipulating the table of contents
\setcounter{tocdepth}{2} % TOC list down to section
\usepackage{enumerate}  % Enumerations
\usepackage{enumitem}   % Enumerations
\usepackage{longtable}  % Multipage table
\usepackage{tabu}       %
\setlength{\tabulinesep}{1.5mm} % Consistent vertical spacing in tabu

%%
%% Graphics, images, colours
%%
\usepackage{graphicx} % embedded images
\usepackage{eso-pic} %
\usepackage{colortbl} % define custom named colours
\definecolor{RedFire}{RGB}{146,25,28}
\definecolor{ParksWildlife}{RGB}{0,85,144}
\definecolor{successbg}{RGB}{223,240,216}
\definecolor{errorbg}{RGB}{242,222,222}
\definecolor{warningbg}{RGB}{252,248,227}
\definecolor{infobg}{RGB}{217,237,247}
\definecolor{muted}{RGB}{153,153,153}
\definecolor{success}{RGB}{70,136,71}
\definecolor{error}{RGB}{185,74,72}
\definecolor{warning}{RGB}{192,152,83}
\definecolor{info}{RGB}{58,135,173}

\definecolor{required}{RGB}{192,152,83}
\definecolor{requiredbg}{RGB}{252,248,227}
\definecolor{denied}{RGB}{185,74,72}
\definecolor{deniedbg}{RGB}{242,222,222}
\definecolor{granted}{RGB}{70,136,71}
\definecolor{grantedbg}{RGB}{223,240,216}
\definecolor{not reqiured}{RGB}{153,153,153}
\definecolor{not requiredbg}{RGB}{255,255,255}

\usepackage{tikz} % Drawing
\usetikzlibrary{arrows,shapes,positioning,shadows,trees}

%%
%% Links, URLs
%%
\usepackage[
    linktoc=all,
    %colorlinks=false,  %PRINT
    colorlinks=true, % ONLINE
    linkcolor=RedFire, % ONLINE
    urlcolor=ParksWildlife, % ONLINE
    pdftitle=Progress Report SP 2015-016 (FY 2015-2016)
]{hyperref}

% Black magic to linebreak URLs
\usepackage{url}
\makeatletter
\g@addto@macro{\UrlBreaks}{\UrlOrds}
\makeatother

%%
%% Custom macros
%%
% Thick Horizontal rule
\newcommand{\HRule}{\vspace{8mm}\\\noindent\rule{\linewidth}{0.1pt}}

% Custom Tikz node for SDS diagram
\newcommand\mynode[6][]{
    \node[#1] (#2){
        \parbox{#3\relax}{
            \begin{center}
            \textbf{#4}\\
            #5\\
            \footnotesize{#6}
            \end{center}}};}



%-----------------------------------------------------------------------------%
% Headers and Footers
\automark{section}
\ohead{\href{http://sdis.dpaw.wa.gov.au/documents/progressreport/1592/}{Progress Report SP 2015-016
}}
\chead{\href{http://sdis.dpaw.wa.gov.au}{SDIS}} % center header ONLINE
\ihead{\href{http://sdis.dpaw.wa.gov.au}{
        \includegraphics[scale=0.4]{/mnt/projects/sdis/staticfiles/img/logo-dpaw.png}}}
\ifoot{\textbf{Printed}~Tue, 5 Jul 2016 10:24:46 +0800} % inner/left footer
\cfoot{} % center footer
\ofoot{\pagemark} % outer/right footer
\pagestyle{scrheadings}
\setkomafont{pageheadfoot}{\normalfont}

%-----------------------------------------------------------------------------%
\begin{document}
\raggedbottom

%-----------------------------------------------------------------------------%
% Title page
\subject{Progress Report SP 2015-016
}
\title{Improved fauna recovery in the Pilbara -- Assessing the uptake of feral
cat baits by northern quolls, and their associated survivorship
}
\subtitle{Animal Science
}
\author{}
\publishers{\small
    \subsection*{Project Core Team}
\begin{tabu} {X X}
\textbf{Supervising Scientist} & Mark Cowan
\\
\textbf{Data Custodian} & Mark Cowan
\\
\textbf{Site Custodian} & 
\\
\end{tabu}


    \subsection*{Project status as of July 5, 2016, 10:24 a.m.}
\begin{tabu} {X X}
& Approved and active
\\
\end{tabu}

    
\subsection*{Document endorsements and approvals as of July 5, 2016, 10:24 a.m.}
\begin{tabu} {X X}

%\rowcolor{grantedbg}
    \textbf{Project Team} & 
    \textcolor{granted}{ granted}\\

%\rowcolor{grantedbg}
    \textbf{Program Leader} & 
    \textcolor{granted}{ granted}\\

%\rowcolor{grantedbg}
    \textbf{Directorate} & 
    \textcolor{granted}{ granted}\\

\end{tabu}



}
\uppertitleback{}
\lowertitleback{}
\date{}

%-----------------------------------------------------------------------------%
% Front matter
\frontmatter
\maketitle
%-----------------------------------------------------------------------------%
% Main matter
\mainmatter

\section*{Improved fauna recovery in the Pilbara -- Assessing the uptake of feral
cat baits by northern quolls, and their associated survivorship
}

K Morris, M Cowan, J Angus, S Garretson, H Anderson, K Rayner


\section*{Context}
The northern quoll (\emph{Dasyurus hallucatus}) is one of seven
terrestrial mammal species that has declined in the Pilbara over the
last 100 years.~ Predation by feral cats is regarded as one of the most
significant threatening processes for this Vulnerable listed species.
The recent development of the \emph{Eradicat\textsuperscript{®}} bait
provides an opportunity to~control feral cats at a landscape scale in
the Pilbara. However knowledge of the diet and laboratory trials suggest
that northern quolls may be at some risk from ingestion of toxic feral
cat baits. This risk needs to be examined in a field situation where
alternative prey items for quolls may reduce the risk from toxic bait
ingestion. This project is funded from a Rio Tinto \emph{EPBC Act}
offset condition.



\section*{Aims}
\begin{itemize}
\itemsep1pt\parskip0pt\parsep0pt
\item
  To assess the field uptake of \emph{Eradicat\textsuperscript{®}} feral
  cat baits by northern quolls and impact on survivorship~in the
  Pilbara.
\item
  To develop an effective cat control strategy that will benefit the
  northern quoll and other threatened species in the Pilbara.
\end{itemize}



\section*{Progress}
\begin{itemize}
\itemsep1pt\parskip0pt\parsep0pt
\item
  Radio collars were attached to~ 21 northern quolls at Yarraloola to
  enable tracking and determination of fate through a trial
  \emph{Eradicat\textsuperscript{®}}~baiting. As a control,~ 20
  individuals were also collared~ at Red Hill.
\item
  \emph{Eradicat\textsuperscript{®}} feral cat baiting of a 20,000 ha
  area at Yarraloola was undertaken in July 2015.
\item
  Tracking of all animals was carried out prior to baiting through to
  the last collar removal in October 2015. During this time there were
  five predation events~at Yarraloola and seven at Red Hill with the
  majority of these attributable to feral cats.
\item
  Of the radiocollared quolls, there were no~deaths as a result of
  \emph{Eradicat\textsuperscript{®}}bait injestion.
\item
  There was no evidence of any reproductive impacts from
  \emph{Eradicat\textsuperscript{®}}~baits on the monitored female
  northern quolls~(number of pouch young were~higher at Yarraloola than
  at Red Hill).
\item
  Details on movement patterns of the radio collared animals at both
  Yarraloola and Red Hill were collated showing significant variation
  between the sexes.
\item
  12 trapping sites were established at both Yarraloola and Red Hill to
  provide~baseline monitoring data. Captures from these sites were
  ~assessed to determine if this number of sites would provide
  sufficient power to detect population change. Subsequently to review
  by the biometrician the number of monitoring sites will be increased
  to 18.
\item
  Monthly progress reports were submitted to Rio Tinto throughout the
  field program.
\item
  Final reports on the bait uptake and the monitoring trials were
  submitted to Rio Tinto at the end of 2015, and used in an application
  to use \emph{Eradicat\textsuperscript{®}} baits over
  \textgreater{}100,000 ha in 2016.
\end{itemize}



\section*{Management implications}
\begin{itemize}
\itemsep1pt\parskip0pt\parsep0pt
\item
  As no detrimental impacts to northern quolls were observed from this
  trial, an operational cat baiting campaign over a much larger area of
  Yarraloola (\textgreater{}100,000ha) will be undertaken in July 2016,
  subject to approval by the Australian Pesticides and Veterinary
  Medicines Authority~(APVMA)~.
\item
  Positive results from this trial have broader implications for feral
  cat management in areas where northern quolls are known to occur.
\item
  A sound trapping~methodology for monitoring northern quoll numbers in
  areas of relatively low abundance has been developed.
\end{itemize}



\section*{Future directions}
\begin{itemize}
\itemsep1pt\parskip0pt\parsep0pt
\item
  Subject to satisfactory progress in 2016, implement large
  scale~\emph{Eradicat\textsuperscript{®}} feral cat~ baiting trials
  over Yarraloola on an annual basis until at least 2019.
\item
  Use remote camera traps~in a before-after-control-impact design to
  monitor the effect of \emph{Eradicat\textsuperscript{®}}baiting on
  feral cats at Yarraloola.
\item
  Continue monitoring northern quolls using established~trapping sites
  at both Yarraloola and Red Hill to detect changes in populations size
  as a response to on-ground management actions.
\item
  Investigate the use of~genetic tools to supplement trap based
  monitoring data.
\item
  Pursue registration~of \emph{Eradicat\textsuperscript{®}} feral cat
  baits for~operational use in areas where northern quolls are present.
\end{itemize}



%-----------------------------------------------------------------------------%
% Back matter
%\backmatter
\end{document}
%-----------------------------------------------------------------------------%



\documentclass[version=last,
    paper=a4,                               % paper size
    10pt,                                   % default font size
    dvipsnames,
    % twoside,                                % PRINT Binding Correction
    oneside,                              % ONLINE
    headings=openany,                       % open chapters on odd and even pages
    open=any,
    BCOR=7mm,                               % PRINT Binding Correction
    %DIV=13,    % typearea 161.54mm x 228.46mm, top 22.85mm, inner 16.15mm
    %DIV=14,    % 165.00 233.36 21.21 15.00
    DIV=15,     % 168.00 237.60 19.80 14.00
    % toc=chapterentrywithdots              % Table of Contents style
]{scrbook}
\usepackage{typearea}


%------------------------------------------------------------------------------%
% Headers and footers
%------------------------------------------------------------------------------%
\usepackage[automark,headsepline,footsepline,plainfootsepline]{scrlayer-scrpage}
\automark*[section]{}
\addtokomafont{pageheadfoot}{\normalfont\footnotesize\sffamily} % Don't italicise
\renewcommand{\chaptermark}[1]{\markleft{#1}{}}     % Chapter: suppress numbering
\renewcommand{\sectionmark}[1]{\markright{#1}{}}    % Section: suppress numbering

% Header (inner, center, outer)
% \ihead{\href{http://sdis.dpaw.wa.gov.au}{\textbf{Project Closure SP 2007-002}}}
%\chead{\href{http://sdis.dpaw.wa.gov.au}{Science Directorate Information System}}
% \ohead{\href{https://www.dbca.wa.gov.au/science/10-biodiversity-and-conservation-science}{
% \includegraphics[height=8mm, keepaspectratio]{/mnt/projects/sdis/staticfiles/img/logo-dbca-bcs.jpg}}}

% Footer (inner, center, outer)
% \ifoot{\RaggedRight\leftmark}                       % Chapter
% \cfoot{\RaggedLeft\rightmark}                       % Section
% \ofoot[\bfseries\thepage]{\bfseries\thepage}        % Page number (also [plain])


%------------------------------------------------------------------------------%
% Fonts, encoding
%------------------------------------------------------------------------------%
%\usepackage{avant}             % Use the Avantgarde font for headings
\usepackage{txfonts}
\usepackage{mathptmx}
\usepackage{gensymb}            % provides \textdegree
\renewcommand{\familydefault}{\sfdefault} % Default to Sans Serif font
\usepackage{microtype}          % Slightly tweak font spacing for aesthetics
\usepackage[english]{babel}
\usepackage[utf8]{inputenc}  % Allow letters with accents
\usepackage[utf8]{luainputenc}  % Allow letters with accents
\usepackage[T1]{fontenc}        % Use 8-bit encoding that has 256 glyphs
\usepackage{textcomp}
\usepackage[explicit]{titlesec}           % Customise of titles
%\DeclareUnicodeCharacter{0080}{\textregistered}
\DeclareUnicodeCharacter{00B0}{\textdegree}

%------------------------------------------------------------------------------%
% Tables, columns, layout
%------------------------------------------------------------------------------%
\usepackage{etoolbox}
\AtBeginEnvironment{longtabu}{\footnotesize}{}{}  % Table font size
\usepackage{booktabs}           % Required for nicer horizontal rules in tables
\usepackage{multicol}           % 2 col publications
\usepackage{pdflscape}          % Landscape pages
\usepackage{pdfpages}           % Include PDFs
\usepackage{hanging}            % hanging paragraphs for publications
%\usepackage{titletoc}          % Manipulate the table of contents
\setcounter{tocdepth}{2}        % TOC list down to section
\usepackage{enumerate}          % Enumerations
\usepackage{enumitem}           % Enumerations
\usepackage{longtable}          % Multipage table
\usepackage{tabu}               %
\setlength{\tabulinesep}{1.5mm} % Consistent vertical spacing in tabu
\newcommand{\HRule}{\vspace{8mm}\noindent\rule{\linewidth}{0.1pt}}
\usepackage[export]{adjustbox}  % minipage, image frame


%------------------------------------------------------------------------------%
% Graphics, images, colours
%------------------------------------------------------------------------------%
\usepackage{graphicx} % embedded images
\usepackage{wrapfig}  % wrap figures in text
\usepackage{caption}  % allow unnumbered captions
\usepackage{eso-pic} % Required for specifying an image background in the title page
\usepackage{colortbl} % define custom named colours
\usepackage{xstring} % Conditionals
\usepackage{transparent} % Allow transparent images

\definecolor{RedFire}{RGB}{146,25,28}
% Following PICA branding guidelines
% https://dpaw.sharepoint.com/Divisions/pica/Documents/Branding%20guidelines.pdf
\definecolor{dpawblue}{RGB}{35,97,146}          % Pantone 647
\definecolor{dpaworange}{RGB}{237,139,0}        % Pantone 144
\definecolor{dpawgreen}{RGB}{116,170,80}        % Pantone 7489
\definecolor{dpawred}{RGB}{124,46,44}           % Paul's suggestion

% bootstrap colours
\definecolor{successbg}{RGB}{223,240,216}
\definecolor{errorbg}{RGB}{242,222,222}
\definecolor{warningbg}{RGB}{252,248,227}
\definecolor{infobg}{RGB}{217,237,247}
\definecolor{muted}{RGB}{153,153,153}
\definecolor{success}{RGB}{70,136,71}
\definecolor{error}{RGB}{185,74,72}
\definecolor{warning}{RGB}{192,152,83}
\definecolor{info}{RGB}{58,135,173}

% SDIS approval colours
\definecolor{required}{RGB}{192,152,83}
\definecolor{requiredbg}{RGB}{252,248,227}
\definecolor{denied}{RGB}{185,74,72}
\definecolor{deniedbg}{RGB}{242,222,222}
\definecolor{granted}{RGB}{70,136,71}
\definecolor{grantedbg}{RGB}{223,240,216}
\definecolor{notrequired}{RGB}{153,153,153}
\definecolor{notrequiredbg}{RGB}{255,255,255}

\usepackage{tikz} % Drawing
\usetikzlibrary{arrows,shapes,positioning,shadows,trees}


%------------------------------------------------------------------------------%
% Hyperlinks
%------------------------------------------------------------------------------%
\usepackage[open=true]{bookmark}
\usepackage{nameref}
\usepackage{ifxetex,ifluatex}
\ifxetex
  \usepackage[
    setpagesize=false,        % page size defined by xetex
    unicode=false,            % unicode breaks when used with xetex
    xetex]{hyperref}
\else
  \usepackage[unicode=true]{hyperref}
\fi

\hypersetup{
  backref=true,
  pagebackref=true,
  hyperindex=true,
  breaklinks=true,
  urlcolor=dpawblue,
  bookmarks=true,
  bookmarksopen=false,
  pdfauthor={Biodiversity and Conservation Science, Department of Biodiversity, Conservation and Attractions, WA},
  pdftitle=Project Closure SP 2007-002
,
  colorlinks=true,
  linkcolor=dpawblue,
  pdfborder={0 0 0}}

\urlstyle{same}                         % don't use monospace font for urlstyle


%------------------------------------------------------------------------------%
% Black magic to linebreak URLs
%------------------------------------------------------------------------------%
\usepackage{url}
\makeatletter\g@addto@macro{\UrlBreaks}{\UrlOrds}\makeatother
\Urlmuskip=0mu plus 1mu


%------------------------------------------------------------------------------%
% Fix latex errors
%------------------------------------------------------------------------------%
\providecommand{\tightlist}{\setlength{\itemsep}{0pt}\setlength{\parskip}{0pt}}

% copy-pasted HTML <span> in SDIS fields becomes \text{} in tex source
\providecommand{\text}{}


%------------------------------------------------------------------------------%
% Custom Tikz node for SDS diagram
%------------------------------------------------------------------------------%
\newcommand\mynode[6][]{
  \node[#1] (#2){
    \parbox{#3\relax}{
      \begin{center}
      \textbf{#4}\\
      #5\\
      \footnotesize{#6}
      \end{center}
    }};}


%------------------------------------------------------------------------------%
% Custom no-pagebreaks-environment
%------------------------------------------------------------------------------%
\newenvironment{absolutelynopagebreak}
  {\par\nobreak\vfil\penalty0\vfilneg\vtop\bgroup}
  {\par\xdef\tpd{\the\prevdepth}\egroup\prevdepth=\tpd}


%------------------------------------------------------------------------------%
% Remove the header from odd empty pages at the end of chapters
%------------------------------------------------------------------------------%
\makeatletter
\renewcommand{\cleardoublepage}{
\clearpage\ifodd\c@page\else
\hbox{}
\vspace*{\fill}
\thispagestyle{empty}
\newpage
\fi}


%----------------------------------------------------------------------------------------
%  Page flow control
%----------------------------------------------------------------------------------------
%\widowpenalty=10000
%\clubpenalty=10000
%\vbadness=1200
%\hbadness=11000


%----------------------------------------------------------------------------------------
%   CHAPTER HEADINGS
%----------------------------------------------------------------------------------------
\newcommand{\thechapterimage}{}
\newcommand{\chapterimage}[1]{\renewcommand{\thechapterimage}{#1}}

% Numbered chapters with mini tableofcontents
\def\thechapter{\arabic{chapter}}
\def\@makechapterhead#1{
%\thispagestyle{plain}
{\centering \normalfont\sffamily
\ifnum \c@secnumdepth >\m@ne
\if@mainmatter
\startcontents
\begin{tikzpicture}[remember picture,overlay]
\node at (current page.north west)
{\begin{tikzpicture}[remember picture,overlay]
\node[anchor=north west,inner sep=0pt] at (0,0) {
\includegraphics[width=\paperwidth,height=0.5\paperwidth]{\thechapterimage}};
%------------------------------------------------------------------------------%
% Small contents box in the chapter heading
% Mini TOC background box
%\fill[color=dpawblue!10!white,opacity=.2] (1cm,0) rectangle (
%  3.5cm, % Mini TOC box width
%  -3.5cm % Mini TOC box height
%);
% Mini TOC text content
%\node[anchor=north west] at (1.1cm,.35cm) {
%  \parbox[t][8cm][t]{6.5cm}{
%    \huge\bfseries\flushleft
%    \printcontents{l}{1}{
%    \setcounter{tocdepth}{1}                   % Mini TOC level depth
%    }
% }
%};
%------------------------------------------------------------------------------%
% Chapter heading
\draw[anchor=west] (5cm,-9cm) node [
rounded corners=20pt,
fill=dpawblue!10!white,
text opacity=1,
draw=dpawblue,
draw opacity=1,
line width=1.5pt,
fill opacity=.2,
inner sep=12pt]{
    \huge\sffamily\bfseries\textcolor{black}{
      \thechapter. #1\strut\makebox[22cm]{}
    }
};
\end{tikzpicture}};
\end{tikzpicture}}
\par\vspace*{240\p@}                            % Push text below chapter image
\fi
\fi}

%------------------------------------------------------------------------------%
% Unnumbered chapters without mini tableofcontents
%------------------------------------------------------------------------------%
\def\@makeschapterhead#1{
%\thispagestyle{plain}
{\centering \normalfont\sffamily
\ifnum \c@secnumdepth >\m@ne
\if@mainmatter
\begin{tikzpicture}[remember picture,overlay]
\node at (current page.north west)
{\begin{tikzpicture}[remember picture,overlay]
\node[anchor=north west,inner sep=0pt] at (0,0) {
  \includegraphics[width=\paperwidth,height=0.5\paperwidth]{\thechapterimage}};
% Mini TOC background box
%\fill[color=dpawblue!10!white,opacity=.2] (1cm,0) rectangle (
%  3.5cm,                                       % Mini TOC box width
%  -3.5cm                                       % Mini TOC box height
%);
% Mini TOC text content
%\node[anchor=north west] at (1.1cm,.35cm) {
%  \parbox[t][8cm][t]{6.5cm}{
%    \huge\bfseries\flushleft
%    \printcontents{l}{1}{
%    \setcounter{tocdepth}{1} % Mini TOC level depth
%    }
%}
%};
\draw[anchor=west] (5cm,-9cm) node [rounded corners=20pt,
  fill=dpawblue!10!white,fill opacity=.6,inner sep=12pt,text opacity=1,
  draw=dpawblue,draw opacity=1,line width=1.5pt]{
  \huge\sffamily\bfseries\textcolor{black}{#1\strut\makebox[22cm]{}}};
\end{tikzpicture}};
\end{tikzpicture}}
\par\vspace*{240\p@}
\fi
\fi
}
\makeatother



\usepackage[automark,headsepline,footsepline,plainfootsepline]{scrlayer-scrpage}
\automark*[section]{}
\addtokomafont{pageheadfoot}{\normalfont\footnotesize\sffamily} % Don't italicise
\renewcommand{\chaptermark}[1]{\markleft{#1}{}}     % Chapter: suppress numbering
\renewcommand{\sectionmark}[1]{\markright{#1}{}}    % Section: suppress numbering

% Header (inner, center, outer)
\ihead{\href{http://sdis.dpaw.wa.gov.au/documents/projectclosure/1787/}{Project Closure SP 2007-002}}
%\chead{\href{http://sdis.dpaw.wa.gov.au}{Science Directorate Information System}}
\ohead{\href{https://www.dbca.wa.gov.au/science/10-biodiversity-and-conservation-science}{
\includegraphics[height=6mm, keepaspectratio]{/mnt/projects/sdis/staticfiles/img/logo-dbca-bcs.jpg}}}
% Footer (inner, center, outer)
\ifoot{\textbf{Printed}~Mon, 4 Feb 2019 17:01:27 +0800} % inner/left footer
\cfoot{}
\ofoot[\bfseries\thepage]{\bfseries\thepage}        % Page number (also [plain])


\pagestyle{scrheadings}
\setkomafont{pageheadfoot}{\normalfont}

%-----------------------------------------------------------------------------%
\begin{document}
\raggedbottom

%-----------------------------------------------------------------------------%
% Title page
\subject{Project Closure SP 2007-002
}
\title{Identifying the cause(s) of the recent declines of woylies in south-west
Western Australia
}
\subtitle{Animal Science
}
\author{}
\publishers{\small
    \subsection*{Project Core Team}
\begin{tabu} {X X}
\textbf{Supervising Scientist} & Adrian Wayne
\\
\textbf{Data Custodian} & 
\\
\textbf{Site Custodian} & 
\\
\end{tabu}


    \subsection*{Project status as of Feb. 4, 2019, 5:01 p.m.}
\begin{tabu} {X X}
& Terminated and closed
\\
\end{tabu}

    
\subsection*{Document endorsements and approvals as of Feb. 4, 2019, 5:01 p.m.}
\begin{tabu} {X X}

%\rowcolor{grantedbg}
    \textbf{Project Team} & 
    \textcolor{granted}{ granted}\\

%\rowcolor{grantedbg}
    \textbf{Program Leader} & 
    \textcolor{granted}{ granted}\\

%\rowcolor{grantedbg}
    \textbf{Directorate} & 
    \textcolor{granted}{ granted}\\

\end{tabu}



}
\uppertitleback{}
\lowertitleback{}
\date{}

%-----------------------------------------------------------------------------%
% Front matter
\frontmatter
\maketitle
%-----------------------------------------------------------------------------%
% Main matter
\mainmatter

\section*{Identifying the cause(s) of the recent declines of woylies in south-west
Western Australia
}



\section*{Key publications and documents}

\subsection{PEER-REVIEWED SCIENTIFIC PUBLICATIONS}

Averis, S., R. C. A. Thompson, A. J. Lymbery, A. F. Wayne, K. D. Morris,
and A. Smith. 2009. The diversity, distribution and host-parasite
associations of trypanosomes in Western Australian Wildlife.
\emph{Parasitology} \textbf{136}1269-1279.

Bennett, M. D., A. Reiss, H. Stevens, E. Heylen, M. Van Ranst, A. F.
Wayne, M. Slaven, J. N. Mills, K. S. Warren, A. J. O'Hara, and P. K.
Nicholls. 2010. The first complete papillomavirus genome characterized
from a marsupial host: a novel isolate from \emph{Bettongia
penicillata}. \emph{Journal of Virology} \textbf{84}:5448-5453.

Botero, A., Thompson, C.K., Peacock, C.S., Clode, P.L., Nicholls, P.K.,
Wayne, A.F., Lymbery, A.J., Thompson, R.C.A., 2013. Trypanosomes genetic
diversity, polyparasitism and the population decline of the critically
endangered Australian marsupial, the brush tailed bettong or woylie
(\emph{Bettongia penicillata}). International Journal for Parasitology:
Parasites and Wildlife 2, 77-89.

Glen, A., A. F. Wayne, M. Maxwell, and J. Cruz. 2010. Comparative diets
of the chuditch, a threatened marsupial carnivore, in the northern and
southern jarrah forests, Western Australia. \emph{Journal of Zoology
(London)}\textbf{282}:276-283.

Groom, C. 2010. Justification for continued conservation efforts
following the delisting of a threatened species: A case study of the
woylie, \emph{Bettongia penicillata ogilbyi}(Marsupialia: Potoroidae).
\emph{Wildlife Research}\emph{.} \textbf{37}: 183-193.

Kaewmongkol, G., Kaewmongkol, S., Burmej, H., Bennett, M. D., Fleming,
P., Adams, P. J., Wayne, A. F., Ryan, U., Irwin, P. and Fenwick, S. G.
2011. Diversity of \emph{Bartonella} species detected in arthropod
vectors from animals in Australia. \emph{Comparative Immunology,
Microbiology \& Infectious Diseases} \textbf{34(5)}, 411-417.

Northover, A.S., Godfrey, S.S., Lymbery, A.J., Morris, K.D., Wayne,
A.F., Thompson, R.C.A., 2015. Evaluating the effects of Ivermectin
treatment on communities of gastrointestinal parasites in translocated
woylies (\emph{Bettongia penicillata}). EcoHealth. DOI:
10.1007/s10393-015-1088-2

Pacioni, C. and Spencer, P. 2010. Capturing genetic information using
non-target species markers in a species that has undergone a population
crash. \emph{Australian Mammalogy} \textbf{32}: 33-38.

Pacioni, C.,~ Wayne, A. F. and Spencer, P. 2011. Effects of habitat
fragmentation on population structure and long distance gene flow in an
endangered marsupial: the woylie. \emph{Journal of Zoology}
\textbf{283}, 98-107.

Pacioni, C., Wayne, A., Spencer, P.B.S., 2013. Genetic outcomes from the
translocations of the critically endangered woylie. Current Zoology
59(3): 294-310.

Pacioni, C.,~ Robertson, I.,~ Maxwell, M.,~ Van Weenan, J. and Wayne, A.
F. 2013. Hematologic characteristics of the woylie (\emph{Bettongia
penicillata ogilbyi}). \emph{Journal of Wildlife Diseases.}
\textbf{49(4)}: 816-830.

Pacioni, C.,~ Johansen, C. A.,~ Mahony, T. J.,~ O'Dea, M.,~ Robertson,
I.,~ Wayne, A. F. and Ellis, T. 2014. A virological investigation into
declining woylie populations \emph{Australian Journal of Zoology}.
\textbf{61}: 446-453.

Pacioni, C., Eden, P., Reiss, A., Ellis, T., Knowles, G., Wayne, A.F.,
2015. Disease hazard identification and assessment associated with
wildlife population declines. Ecological Management \& Restoration 16,
142-152.

Pacioni, C., Hunt, H., Allentoft, M., Vaughan, T., Wayne, A., Baynes,
A., Haouchar, D., Dortch, J., Bunce, M., 2015. Genetic diversity loss in
a biodiversity hotspot: ancient DNA quantifies genetic decline and
former connectivity in a critically endangered marsupial. Molecular
Ecology 24, 5813-5828.

Pan, S., Thompson, R.C.A., Grigg, M.E., Sundar, N., Smith, A., Lymbery,
A.J., 2012. Western Australian marsupials are multiply co-infected with
genetically diverse strains of \emph{Toxoplasma gondii}. International
Journal for Parasitology 7(9).

Parameswaran, N., O'Handley, R. M., Grigg, M. E., Wayne, A. F. and
Thompson, R. C. A. 2009. Vertical transmission of \emph{Toxoplasma
gondii} in Australian Marsupials.
\emph{Parasitology}\textbf{136}:939-44.

Parameswaran, N., A. Thompson, N. Sundar, S. Pan, and M. Grigg. 2010.
Nonarchetypal Type II-like and atypical strains of \emph{Toxoplasma
gondii} infecting marsupials of Australia. \emph{International Journal
for Parasitology}. \textbf{40}: 635-640.

Rong, J.,~ Bunce, M.,~ Wayne, A.,~ Pacioni, C.,~ Ryan, U. and Irwin, P.
J. 2012. High prevalence of \emph{Theileria penicillata} in woylies
(\emph{Bettongia penicillata}). \emph{Experimental Parasitology}
\textbf{131(2)}, 157-161.

Smith, A., P. Clark, S. Averis, A. J. Lymbery, A. F. Wayne, K. D.
Morris, and R. C. A. Thompson. 2008. Trypanosomes in a declining species
of threatened Australian marsupial, the brush-tailed bettong
\emph{Bettongia penicillata} (Marsupalia: Potoroidae).
\emph{Parasitology} \textbf{135}:1-7.

Thompson, C.K., Botero, A., Wayne, A.F., Godfrey, S.S., Lymbery, A.J.,
Thompson, R.C.A., 2013. Morphological polymorphism of \emph{Trypanosoma
copemani} and description of the genetically diverse \emph{T. vegrandis
sp. nov.}from the critically endangered Australian potoroid, the
brush-tailed bettong (\emph{Bettongia penicillata} (Gray, 1837)).
\emph{Parasites \& Vectors} \textbf{6:} 121-133.

Thompson, C. K.,~ Wayne, A. F.,~ Godfrey, S. S. and Thompson, R. C. A.
2014. Temporal and spatial dynamics of trypanosomes infecting the
brush-tailed bettong (\emph{Bettongia penicillata}): a cautionary note
of disease-induced population decline. \emph{Parasites and Vectors}
\textbf{7}:169.

Thompson, C.K., Wayne, A.F., Godfrey, S.S., Thompson, R.C.A., 2015.
Survival, age estimation and sexual maturity of pouch young of the
brush-tailed bettong (\emph{Bettongia penicillata}) in captivity.
Australian Mammalogy 37, 29-38.

Thompson, RCA,~ A. Smith , A. J. Lymbery, S. Averis, K. D. Morris A. F.
Wayne 2010. \emph{Giardia}in Western Australian wildlife.
\emph{Veterinary Parasitology} \textbf{170}: 207-211.

Wayne, A.F., Maxwell, M.A., Ward, C., Vellios, C., Ward, B., Liddelow,
G.L., Wilson, I., Wayne, J.C., Williams, M.R., 2013a. The importance of
getting the numbers right: quantifying the rapid and substantial decline
of an abundant marsupial, \emph{Bettongia penicillata}. \emph{Wildlife
Research} 40, 169-183.

Wayne, A.F., Maxwell, M., Ward, C.G., Vellios, C.V., Wilson, I., Wayne,
J.C., Williams, M.R., 2015. Sudden and rapid decline of the abundant
marsupial, \emph{Bettongia penicillata} in Australia. Oryx 49(1),
175-185.

Yeatman, G.J., Wayne, A.F., 2015. Home range and habitat use of a
critically endangered marsupial (\emph{Bettongia penicillata ogilbyi})
inside and outside a predator proof sanctuary. Australian Mammalogy 37,
157-163.

Zosky, K., K. Bryant, M. Calver, and A. F. Wayne. 2010. Do preservation
methods affect the identification of dietary components from faecal
samples? A case study using a mycophagous marsupial. \emph{Australian
Mammalogy} \textbf{32}:173-176.

~

\subsection{POPULAR PUBLICATIONS/ARTICLES}

Ballast, B., Wayne, A.F., 2012. Perup Sanctuary Information Brochure,
pp. 1-4. Dept. Environment and Conservation.

Mitchell, S., and A. F. Wayne. 2008. Down but not out: solving the
mystery of the woylie population crash. \emph{Landscope}
\textbf{25}:10-15.

Richards, J.,~ Gardner, T. and Copley, M. 2009. Bringing back the
animals. \emph{Landscope} \textbf{24(3)}, 54 - 61.

Smith, A.,~ Lymbery, A. and Thompson, A. 2009. Meet the parasites.
\emph{Landscope} \textbf{24(4)}, 24 - 29.

Tolfree, S., Wayne, A.F., 2012. Help Save Woylies (Youtube video)

Tolfree, S., Wayne, A.F., 2012. Woylie Handling Video Manual (DVD).
Adrian Wayne, DEC, Australia.

Wayne, A. F. 2009. `Woylie declines: what are the causes? Science
Division Information Sheet 7/2009.' Department of Environment and
Conservation, Perth, Western Australia.

Wayne, A. F. 2010. Woylie Conservation Research (Fact Sheet). South
Coast Natural Resource Management, Albany, Western Australia.

Wayne, A. and Moore, J. 2011. The jewel in the crown: Perup and the
Upper Warren. \emph{Landscope} \textbf{26(3)}, 10-17.

Wayne, A.F., 2014. Perup Sanctuary: insuring woylies against extinction.
\emph{Landscope} \textbf{29(3)}, 12-17.

Wayne, A. F. 2014. Perup Sanctuary: Insuring woylies against extinction.
\emph{Wildlife Rescue Magazine} \textbf{10}, 47-57.

~

\subsection{HONOURS AND POSTGRADUATE STUDENT THESES}

Abdad, M.Y. (2011). An epidemiological and serological study of
\emph{Rickettsia} in Western Australia. PhD Thesis, Murdoch University.

Bateman, A., 2014. The short-term response of the woylie \emph{Bettongia
penicillata} \emph{ogilbyi} to habitat expansion via underpasses in a
fenced reserve at Whiteman Park. Honours thesis. University of Western
Australia, Perth.

Botero Gomez, L.A., 2014. Genotypic and phenotypic diversity of
trypanosomes infecting Australian marsupials and their association with
the population decline of the brush-tailed bettong or woylie
(\emph{Bettongia penicillata}). In, School of Veterinary and Life
Sciences. PhD thesis. Murdoch University, Western Australia.

Cherriman, (2007). Simon C. Territory size and diet throughout the year
of the Wedge-tailed Eagle \emph{Aquila audax}in the Perth region,
Western Australia. Honours thesis. Curtin University. {[}not directly
involved in WCRP -- Upper Warren but is with Karakamia and
Dryandra{]}{]}

Eikelboom, T. H. (2010). A field comparison of survey methods for
estimating the population density of woylies (\emph{Bettongia
penicillata}) at Karakamia Wildlife Sanctuary. Honours thesis.
University of Western Australia.

Hide, A. (2006). Survival and dispersal of the threatened woylie
\emph{Bettongia penicillata} after translocation. Honours thesis,
University of Western Australia.

Hunt, H. (2010). Molecular evidence of a genetic bottleneck using
temporal genetic data in an endangered marsupial, (\emph{Bettongia
penicillata ogilbyi}). School of Biological Sciences and Biotechnology.
Honours thesis. Murdoch University, Murdoch, Western Australia.

Kaewmongkol, G.(2012). Detection and characterization of
\emph{Bartonella} species in Western Australia. PhD Thesis, Murdoch
University, Murdoch, Western Australia.

Madon, E. (2006). Mating systems and reproductive anatomy in marsupials:
explaining the evolution of the anterior vaginal expansion of
\emph{Bettongia penicillata}. Page 66. School of Animal Biology. Honours
thesis. University of Western Australia, Perth.

O'Brien, R. Christopher, B.A., M.F.S. (2008) Forensic Animal Necrophagy
in the South-west of Western Australia: Species, feeding patterns, and
taphonomic effects. Doctor of Philosophy thesis, University of Western
Australia.

Pacioni, C. (2010). The population and epidemiological dynamics
associated with recent decline of woylies (\emph{Bettongia penicillata})
in Australia. PhD thesis, Murdoch University.

Parameswaran, N. (2008). \emph{Toxoplasma gondii}in Australian
marsupials. PhD Thesis, Murdoch University.

Rogers, P. (2009). Predator profiling as a tool for the conservation of
the woylie (\emph{Bettongia penicillata}). Honours thesis, University of
Western Australia.

Rong Ho, J. (2009) Biological and Molecular Studies of \emph{Theileria
penicillata}infecting Woylies (\emph{Bettongia penicillata
ogilbyi)}Honours thesis, Murdoch University.

Yeatman, G. (2010). Demographic changes of a woylie (\emph{Bettongia
penicillata}) population: a response to increasing density and climate
drying? Honours thesis, University of Western Australia.

Zosky, K. (2011). Food resources and the decline of woylies
\emph{Bettongia penicillata ogilbyi} in southwestern Australia. PhD
thesis, Murdoch University, Western Australia.

~

\subsection{STUDENT REPORTS}

Basille, S. (2010). The epidemiology of piroplasm infection in the
woylie \emph{(Bettongia penicillata ogilbyi).}Undergraduate Project
(Independent Study Contract). Murdoch University, Western Australia.

Bennett, K. (2012). `An evaluation of two methods for assessing
population densities of introduced predators in southwest Western
Australia. Undergraduate independent study report.' Biology SIT Study
Abroad Program, Cairns, Australia.

Harradine, E., White, L., Madsen, A., McMahon, S., Borkowski, K., Pinto,
E. (2012). Perup -- Nature's Guesthouse: Strategic destination
management plan (2012-2016). Tourism Management course unit (TOU303)
group project, Murdoch University.

Horton, C. (2013). `An Analysis of the Woylie (Bettongia penicillata
ogilbyi) Population Following Translocation from Perup Sanctuary to
Yendicup in the Upper Warren Region of Western Australia.' CIEE Student
project report.

Jaimes, S. (2010) Where's the Woylie? Possible factors affecting
post-decline Woylie (\emph{Bettongia penicillata ogilbyi}) abundance in
the Upper Warren Region of South West Australia. Undergraduate
independent study report. Pacific Lutheran University, Biology SIT Study
Abroad Program, Cairns, Australia.

Jenkins, Gina. (2013). Sampling in 2013 of \emph{Bettongia
penicillata}(Woylie) and \emph{Trichosurus vulpecula}(Brush-tail Possum)
populations at Warrup and Keninup Sites in southwestern, Western
Australia.

Jenkins, Gina. (2013). Changes in female \emph{Bettongia
penicillata}(Woylie) populations from 2010 to 2013 within the Perup
Sanctuary, Warrup and Keninup sites in southwestern, Western Australia

McCalmont, J. (2010). Evaluation of conservation measures for a specific
endangered species,\emph{Bettongia penicillata.} 3rd year BSc (Hons)
project. Institute of Biological, Environmental and Rural Science,
University of Wales, Aberystwyth.

McGuirk, E. (2013). 'Assessing the success of translocation and
effectiveness of surveillance tactics in the conservation of Woylie
(\emph{Bettongia penicillata}) species in Yendicup of Upper Warren,
Western Australia.' CIEE Student project report.

Rowells, Jessica. (2013). Comparing \emph{Bettongia penicillata
ogilbyi}Population between Warrup and Keninup

Rowells, Jessica. (2013). Comparing male to female \emph{Bettongia
penicillata ogilby}populations inside and outside of the Perup sanctuary
over time.

Shah, Pooja. (2013). Woylie population distribution in Warrup and
Keninup

Shah, Pooja. (2013). Recurrence of woylies in Perup Sanctuary compared
to Warrup and Keninup; where do they survive better.

Wallace, Samantha. (2013). Woylie monitoring: Population numbers in
Warrup and Keninup of the Upper Warren.

Wallace, Samantha. (2013). Woylie Monitoring: Health of Woylies from
2010-2013 at the Perup Sanctuary, Warrup and Keninup.

~

\subsection{REPORTS}

\textbf{2006}

Wayne, A.F. (2006). Science Project Plan, Department of Environment and
Conservation, Woylie conservation research project: An investigation
into the declines of woylies (Bettongia penicillata) in south-western
Australia.

Bougher, N. L.(2006). Identity and taxonomy of truffle fungi from an
initial survey at Karakamia Wildlife Sanctuary. Department of
Environment and Conservation.

Wayne, A. F. (2006). Interim assessment of the evidence for a recent
decline in woylie abundance in south-western Australia. Pages 1-28.
CALM, Perth, Western Australia.

Wayne, A. F., I. Wilson, J. Northin, B. Barton, J. Gillard, K. Morris,
P. Orell, and J. Richardson. (2006). Situation report and project
proposal: identifying the cause(s) for the recent declines of woylies in
south-western Australia. A report to the Department of Conservation and
Land Management Corporate Executive. Pages 1-28. CALM, Perth, Western
Australia.

\textbf{2007}

Robinson, R., J. Fielder, M. Maxwell, N. L. Bougher, W. Sicard, and A.
Wayne. (2007). Woylie conservation research project: Preliminary survey
of hypogeous fungi in the Upper Warren region. Department of Environment
and Conservation.

\textbf{2008}

Eden, Paul and Wayne, Adrian (2008). Metabolic bone disease of woylies.

Pacioni, C., Spencer, P.B.S. and Wayne, A.F. (July 2008). ``Conservation
Conundrum: The population dynamics associated with recent decline of
woylies (\emph{Bettongia penicillata)}in Australia. Final report to the
Australian Academy of Sciences.'

Pacioni, C., Spencer, P.B.S. and Wayne, A.F. (Dec 2008). ``Conservation
Conundrum: The population dynamics associated with recent decline of
woylies (\emph{Bettongia penicillata)}in Australia. Interim report to
the South Coast Natural Resource Management Inc.'

DEC (2008). A. Wayne (Ed.). Diagnosis of recent woylie (\emph{Bettongia
penicillata ogilbyi}) declines in southwestern Australia. Progress
report of the Woylie Conservation Research Project. Perth, Western
Australia, Western Australian Government Department of Environment and
Conservation: 1-314.

DEC (2008). I. Wilson (Ed.). Woylie Conservation Project Operations
Handbook. Perth, Western Australia, Western Australian Government
Department of Environment and Conservation: 1-158.

Freegard, C. (2008). `\emph{Bettongia penicillata ogilbyi --} nomination
of a Western Australian species for listing as threatened, change of
status or delisting. `Dept. Conservation and Land Management, Perth,
Western Australia.

Mawson, P., Atkins, K. and Goldberg, J. (2008). `\emph{Bettongia
penicillata obilbyi --}woylie: Species Information Sheet.' Department of
the Environment and Water Resources.

\textbf{2009}

DEWHA (2009). Department of the Environment, Water, Heritage and the
Arts (2009). Bettongia penicillata ogilbyi in Species Profile and
Threats Database, Department of the Environment, Water, Heritage and the
Arts, Canberra. Available from:
\href{http://www.environment.gov.au/sprat}{http://www.environment.gov.au/sprat}.

Orell, P. (2009). `Current Status of the woylie: based on an update of
the species information sheet submitted to the Department of Environment
and Water Resources in 2008.' Department of Environment and
Conservation, Perth, Western Australia.

Threatened Species Scientific Committee (TSSC) (2009). Commonwealth
Listing Advice on Bettongia penicillata ogilbyi (Woylie). {[}Online{]}.
Department of the Environment, Water, Heritage and the Arts. Available
from:
\href{http://www.environment.gov.au/biodiversity/threatened/species/pubs/66844-listing-advice.pdf}{http://www.environment.gov.au/biodiversity/threatened/species/pubs/66844-listing-advice.pdf}.

Pacioni, C.,~ Spencer, P. B. S. and Wayne, A. F. (2009). 'Conservation
Conundrum: The population dynamics associated with recent decline of
woylies (\emph{Bettongia penicillata}) in Australia. Final report to the
South Coast Natural Resource Managment Inc.'

Wayne, A. F.,~ Friend, T.,~ Burbidge, A.,~ Morris, K. and van Weenen, J.
(2008). Bettongia penicillata. In: IUCN 2009. IUCN Red List of
Threatened Species. Version 2009.2.
\textless{}www.iucnredlist.org\textgreater{}.

Wayne, A. F. and Maxwell, M. (2009). `Examples of woylie skin and fur
conditions associated with declines in the Upper Warren populations.'
Deptartment of Environment and Conservation.

\textbf{2010}

Eden, P.,~ Reiss, A.,~ Nicholls, P. and Wayne, A. (2010). `Investigation
of skin disease observed in woylies (Bettongia penicillata ogilvyi) from
the Upper Warren region, WA.'

Hunt, H. (2010). A temporal assessment investigating the effects of
population declines on genetic diversity, in the critically endangered
woylie\emph{(Bettongia penicillata ogilbyi):}A Summary to the Woylie
Recovery Team. Murdoch University, Perth, Western Australia

Wayne, A. (2010). `Woylie Conservation Research Project - annual report
to Wildlife Conservation Action.' Manjimup, Western Australia.

\textbf{2011}

Hamilton, N. and Rolfe, J. (2011). Assessment of introduced predator
presence within the Perup Sanctuary, Western Australia. Unpublished
report. Department of Environment and Conservation, Woodvale, Western
Australia.

Wayne, A.F., Maxwell, M., Nicholls, P., Pacioni, C., Reiss, A., Smith,
A., Thompson, R.C.A., Vellios, C., Ward, C., Wayne, J.C., Wilson, I.,
Williams, M.R., (2011). The woylie conservation research project:
investigating the cause(s) of woylie declines in the Upper Warren
region. Progress Report, p. 73. Dept. Environment and Conservation,
Manjimup, Western Australia.

\textbf{2012}

Pacioni, C. (2012). Integrating genetic data in the development of the
management plan for the woylie population at Whiteman Park.

Edwards, S., (Oct 2012). Balban predator control program. Wildthings
Animal Control Solutions.

\textbf{2013}

Edwards, S., (March 2013). Boyicup predator control program. Wildthings
Animal Control Solutions.

Wayne, A. F.,~ Maxwell, M. A.,~ Ward, C. G.,~ Vellios, C. V.,~ Wilson,
I. J. and Dawson, K. E. (2013c). `Woylie Conservation and Research
Project: Progress Report 2010-2013.' Department of Parks and Wildlife,
Perth, Western Australia.

Yeatman, G.J., Wayne, A.F., Mills, H., 2013. Terrestrial vertebrate
assemblage and species level patterns between major habitat types inside
and outside a fenced exclosure in south-western Australia, p. 22.
University of Western Australia, Perth.

\textbf{2014}

Pacioni, C. (2014). 'Modelling woylie (\emph{Bettongia penicillata})
populaition genetics to inform management strategies.' WWF-Australia,
Perth, Western Australia.

Pacioni, C., 2014. Population modelling for the development of genetic
managment guidelines in translocation programs. Perth, Western
Australia, p. 27.

~

\textbf{2015}

Skerratt, L.F., 2015. Outbreak Investigation for the Critically
Endangered Woylie (\emph{Bettongia penicillata}). Final Report to
WWF-Australia on agreement contract number 0230-007. In. James Cook
University, Townsville.

\subsection{CONFERENCE PROCEEDINGS}

\textbf{2006}

Wayne, A. F. (2006). The Post-Recovery Collapse of the Woylie:
Diagnosing their recent declines in south-western Australia. Australian
Mammal Society Scientific Meeting and Macropod Symposium. Australian
Mammal Society, Melbourne, Victoria.

\textbf{2007}

Abdad, Y., P. Adams, L. Pallant, A. Li, A. F. Wayne, and S. Fenwick.
(2007). Potential bacterial pathogens in Woylies from SW WA (ABSTRACT).
Page 50. Wildlife Disease Association (Australasian Section), Dryandra
Woodland.

Burmej, H, Smith, A, Lymbery, A, Wayne, A, Morris, K, Fenwick, S,
Thompson, RCA. (2007). The biodiversity and host-specificity of fleas in
the woylie (Bettongia penicillata). ASP \& ARC/NHMRC Research Network
for Parasitology Annual Conference, Canberra, ACT, 8-11 July, 2007.

Burmej, H., A. Smith, A. J. Lymbery, A. F. Wayne, K. D. Morris, S.
Fenwick, and A. Thompson. (2007). The biodiversity, ecology and
importance of ectoparasites in the woylie (Bettongia penicillata) and
sympatric species (ABSTRACT). Page 51. Wildlife Disease Association
(Australasian Section), Dryandra Woodland.

Clark, P., A. Thompson, A. F. Wayne, and S. Averis. (2007). Trypanosomes
in relation to the diagnosis of woylie declines: prevalence and
molecular characterisation (ABSTRACT). Page 47. Wildlife Disease
Association (Australasian Section), Dryandra Woodland.

Eden, P., K. Payne, R. Vaughan, S. Vitali, and A. F. Wayne. (2007).
Clinical aspects of disease investigations in population declines of
woylies (Betongia penicillata ogilbyi) (ABSTRACT). Page 44. Wildlife
Disease Association (Australasian Section), Dryandra Woodland.

Knowles, G., P. Eden, and A. F. Wayne. (2007). Results of woylies
(Bettongia penicillata ogilbyi) submitted for necropsy to Murdoch
University, September 2005 to May 2007 (ABSTRACT). Page 45. Wildlife
Disease Association (Australasian Section), Dryandra Woodland.

Pacioni, C., P. B. S. Spencer, and A. F. Wayne. (2007). Woylie
(Bettongia penicillata ogilbyi) conservation genetics project
(ABSTRACT). Page 46. Wildlife Disease Association (Australasian
Section), Dryandra Woodland.

Parameswaran, N, O'Handley, R, Thompson, RCA. (2007). Development of an
ELISA for the detection of Toxoplasma gondii antibodies in macropod
marsupials. ASP \& ARC/NHMRC Research Network for Parasitology Annual
Conference, Canberra, ACT, 8-11 July, 2007.

Parameswaran, N., R. O'Handley, A. F. Wayne, A. Smith, P. Eden, A. J.
Lymbery, K. D. Morris, and A. Thompson. (2007). Toxoplasma in woylies
(ABSTRACT). Page 49. Wildlife Disease Association (Australasian
Section), Dryandra Woodland.

Smith, A., A. J. Lymbery, A. Elliot, U. Parkar, A. F. Wayne, and A.
Thompson. (2007). Prevalence and diversity of woylie endoparasites: from
individuals to populations and sympatric species (ABSTRACT). Page 48.
Wildlife Disease Association (Australasian Section), Dryandra Woodland.

Wayne, A. F., A. Thompson, G. Knowles, P. Eden, M. Swinburn, and K.
Morris. (2007). Diagnosing recent woylie declines in south-western
Australia: Situation assessment, research approach and disease
investigations (Paper). Wildlife Disease Symposium, Perth Zoo, Western
Australia.

Wayne, A. F., C. Ward, M. Maxwell, C. V. Vellios, I. Wilson, J. Wayne,
M. Swinburn, B. Ward, K. Morris, A. Thompson, G. Knowles, P. Eden, A.
Dugand, J. Williams, and T. Gardiner. (2007). Diagnosing the recent
woylie (Bettongia penicillata) collapse in south-western Australia.
Ecological Society of Australia Annual Conference. Ecological Society of
Australia, Perth, Western Australia.

Wayne, A. F., C. Ward, M. Maxwell, C. V. Vellios, I. Wilson, J. Wayne,
M. Swinburn, B. Ward, K. Morris, A. Thompson, G. Knowles, P. Eden, A.
Dugand, J. Williams, and T. Gardiner. (2007). Diagnosing the recent
woylie declines in south-western Australia (ABSTRACT). Page 43. Wildlife
Disease Association (Australasian Section), Dryandra Woodland, Western
Australia.

\textbf{2008}

Averis, S. (2008). Characterisation of Trypanosomes from Australian
Marsupials. (ABSTRACT). In `Proceedings of the European Multicolloquium
of Parasitology'. Paris, France.

Burmej, H. I. (2008). Describing ectoparasite biodiversity in threatened
Western Australian mammals: new methods and challenges. (Abstract). In
`Proceedings of the Australasian Wildlife Management Society, 21st
Annual Conference'. Fremantle, Western Australia.

Pacioni, C.,~ Spencer, P. and Wayne, A. F. (2008). Understanding woylie
decline: a molecular perspective. (Abstract). In `Proceedings of the
Australasian Wildlife Management Society, 21st Annual Conference'.
Fremantle, Western Australia.

Pacioni, C.,~ Spencer, P. B. S. and Wayne, A. F. (2008). Unravelling the
contribution of genetics in the decline of the woylie (ABSTRACT). In
`Proceedings of the Wildlife Disease Association (Australasian
section)'. Kioloa, New South Wales.

Parameswaran N., Grigg, M., O'Handley, R., Thompson, RCA. (2008).
Molecular characterisation of \emph{Toxoplasma gondii} DNA in wild
Australian marsupials. ASP \& ARC/NHMRC Research Network for
Parasitology Annual Conference, Glenelg South Australia, 6-9 July, 2008.

Parameswaran, N.,~ Pan, S.,~ Lymbery, A.,~ Smith, A.,~ Wayne, A. F.,~
Morris, K.,~ Grigg, A. H. and Thompson, R. C. A. (2008). Toxoplasma in
Australian wildlife --~ food for thought? (Abstract). In `Proceedings of
the Australasian Wildlife Management Society, 21st Annual Conference'.
Fremantle, Western Australia.

Parkar, U.,~ Traub, R.,~ Vitali, S.,~ Wayne, A. F.,~ Morris, K. and
Thompson, R. C. A. (2008). Characterisation of \emph{Blastocystis}
isolates from zoo animals and native wildlife. (Abstract). In
`Proceedings of the Australasian Wildlife Management Society, 21st
Annual Conference'. Fremantle, Western Australia.

Reiss, A.,~ Eden, P.,~ Wayne, A. F.,~ Nicholls, P. and Thompson, R. C.
A. (2008). Veterinary investigation of population declines of the woylie
(\emph{Bettongia penicillata}) in south-west Western Australia
(Abstract). In `Proceedings of the Australasian Wildlife Management
Society, 21st Annual Conference'. Fremantle, Western Australia.

Rose, K.,~ McCallum, H.,~ Wayne, A. F. and Speare, R. (2008). Wildlife
Health and Biosecurity in Australia, Federal Parliamentary Forum on
Biosecurity. Canberra.

Siah, W. S.,~ Woods, R.,~ Rose, K.,~ Skerratt, L.,~ Wayne, A. F.,~
Mckenzie, J. and Field, H. (2008). Protecting Wildlife Health in
Australia: how will we do it better in a `One Health' setting? In
`Proceedings of the Wildlife Disease Association Conference'. Kioloa,
New South Wales, Australia.

Smith, A.,~ Clark, P.,~ Averis, S.,~ Lymbery, A.,~ Wayne, A. F.,~
Morris, K. and Thompson, A. (2008). A novel \emph{Trypanosoma} sp. and
its role in the decline of a threatened species of Australian marsupial,
the brush-tailed bettong (\emph{Bettongia penicillata}) (ABSTRACT). In
`Proceedings of the European Multicolloquium of Parasitology'. Paris,
France.

Smith, A.,~ Clark, P.,~ Averis, S.,~ Lymbery, A.,~ Wayne, A. F.,~
Morris, K. and Thompson, R. C. A. (2008). The role of trypanosomes in
the decline of a threatened species of Australian marsupial, the
brush-tailed betong (\emph{Bettongia penicillata}) (Abstract). In
`Proceedings of the Australasian Wildlife Management Society, 21st
Annual Conference'. Fremantle, Western Australia.

Thompson, R.C.A. (2008). Transmission and life cycle patterns of
\emph{Toxoplasma}. Joint International Tropical Medicine Meeting,
Bangkok, Thailand, 13-14 October, 2008.

Wayne, A. F.,~ Ward, C.,~ Maxwell, M.,~ Vellios, C.,~ Wilson, I.,~
Wayne, J. C.,~ Thompson, R. C. A.,~ Reiss, A.,~ Eden, P. and Richards,
J. (2008). Diagnosing the recent woylie (\emph{Bettongia penicillata})
collapse in south-western Australia. (Abstract). In `Proceedings of the
Australasian Wildlife Management Society, 21st Annual Conference'.
Fremantle, Western Australia.

\textbf{2009}

Armstrong, T., Best, W., Charman, S., Don, R., Laverty, C., Luna G.,
Scaffidi, A., Sims, C., Thompson, R.C.A., White, K. (2009). A novel and
highly potent class of compounds for the treatment of trypanosomiasis.
Keystone Symposia on Molecular and Cellular Biology, Drug Discovery for
Protozoan Parasites, Breckenridge, Colorado, USA. 22-26 March, 2009.

Pan, S.,~ Thompson, A.,~ Lymbery, A.,~ Smith, A.,~ Morris, K.,~ Wayne,
A. F. and Grigg, M. E. (2009). Atypical \emph{Toxoplasma gondii}
genotypes in Western Australian wildlife species. In `Proceedings of the
World Association for the Advancement of Veterinary Parasitology'.
Calgary Alberta, Canada.

Parkar, U., Traub, R., Wayne, A., Morris, K., Thompson, R.C.A. (2009).~
The prevalence and molecular characterization of \emph{Blastocystis}
isolates from wild Australian native fauna. ASP/NHMRC Research Network
for Parasitology Annual Conference, University of Sydney, New South
Wales, 12-15 July, 2009.

Smith, A., Lymbery, A., Averis, S., Wayne, A., Morris, K., and Thompson,
R.C.A. (2009). Low host specificity in trypanosomes associated with
native Australian wildlife. In ``Proceedings of the World Association
for the advancement of Veterinary Parasitology'. Calgary Alberta,
Canada. 9 -- 13 August, 2009.

Thompson, R.C.A. (2009). The molecular ecology of \emph{Toxoplasma} in
wildlife. The Royal Golden Jubilee PhD Congress X, Jomtien Palm Beach
Resort, Pattaya, Chonburi, Thailand. 3-5 April, 2009.

Reiss, A., Vitali, S., Eden, P. and Wayne A (2009). Perth Zoo Vet
Involvement In Conservation Programs -- Recent Developments. In
Proceedings of the Wildlife Disease Association- Australasian Section
Annual Conference, Catlins, New Zealand

Wayne, A. F. (2009). Where have all the woylies gone? Is predation, food
or disease the primary agent of population collapse? In `Proceedings of
the Dwellingup Seminar Series, Science Division, Department of
Environment and Conservation'. Dwellingup, Western Australia.

Wayne, A. F.,~ Thompson, A.,~ Reiss, A.,~ Eden, P.,~ Pacioni, C.,~
Smith, A.,~ Nicholls, P.,~ Page, M.,~ Van Weenen, J.,~ Morris, K.,~
Marlow, N. and Orell, P. (2009). Investigation of woylie (brush-tailed
bettongs) population declines -- a critical and challenging problem. In
`Proceedings of the Australian College of Veterinary Scientists,
Scientific Meeting'. Surfers Paradise, Queensland.

Wayne, A. F.,~ Thompson, A.,~ Reiss, A.,~ Page, M.,~ Van Weenen, J.,~
Maxwell, M.,~ Ward, C. and Vellios, C. (2009). A wildlife pandemic? Are
introduced predators and disease the causes of the collapse of woylie
(brush-tailed bettong) populations? In `Proceedings of the Australian
Mammal Society, Scientific Meeting'. Perth, Western Australia.

\textbf{2010}

Pan, S., Thompson, R.C.A., Lymbery, A.J., Smith, A., Morris, K., Wayne,
A.F. and Grigg, M.E. Non-archetypal \emph{Toxoplasma gondii} genotypes
in Western Australian wildlife species. XIIth International Congress of
Parasitology (ICOPA) Melbourne, Aug 2010

Paparini, A., K. Warren, P. DeTores, P. Irwin, U. Ryan. 2010.
Identification of new species of haemoprotozoa in marsupials. ICOPA
Conference, Melbourne, Aug 2010

Parkar, U., R. J. Traub, N. Hijjawi, A. Wayne, K. Morris, and R. C. A.
Thompson. 2010. The development of a molecular-based multiplex method to
detect \emph{Blastocystis, Cryptosporidium} sp. and \emph{Giardia
duodenalis}. XIIth International Congress of Parasitology (ICOPA)
Melbourne, Aug 2010

Reiss, A., A. F. Wayne, C. Pacioni, P. Nicholls, P. Eden, A. Thompson,
and A. Smith. 2010. The decline of the woylie (Bettongia penicillata) in
Western Australia- an update on the investigation. (ABSTRACT).
Australian College of Veterinary Scientists, Surfers Paradise,
Queensland.

Thompson, C., Smith, A., Wayne, A. F., Thompson, A. (2010). Who's biting
the woylie and what are they transmitting? (ABSTRACT) Ecological Society
of Australia 2010 Conference, Canberra, 6-10 December

Wayne, A.,~ Marlow, N. J. and de Tores, P. (2010). Woylie
(\emph{Bettongia penicillata ogilbyi}): research for conservation
(ABSTRACT). In `Proceedings of the Threatened Species Research Forum:
Western Australian Ecology Centre, 9th July 2010: a Review of WA
Government Research into Threatened Species'. p. 30. (Conference
Organising Committee, Perth)

Yeatman, G., H. Mills, M. Page, and A. Wayne. (2010). Demographic
changes of a woylie (Bettongia penicillata) population with increasing
density in a fenced reserve. (ABSTRACT). Australian Wildlife Management
Society Annual Conference, Torquay, Victoria

\textbf{2011}

Botero, A., Christopher S. Peacock, Adrian F. Wayne, RC Andrew Thompson
(2011). Biological and genetic characterisation of trypanosomes in
Western Australian marsupials. 10-13 July 2011. Cairns, Australia. The
Australian Society for Parasitology annual conference.

Thompson, C.,~ Smith, A.,~ Wayne, A. and Thompson, A. (2011).
Transmission dynamics of trypanosomes in declining, stable and enclosed
populations of Brush-tailed Bettongs (ABSTRACT). In `Proceedings of the
Australian Society for Parasitology Annual Conference'. 10-13 July 2011.
Cairns, Australia.

Shuting, Pan., Andrew Thompson, Alan Lymbery, Andrew Smith, Keith
Morris, Adrian Wayne and Michael Grigg. (2011). \emph{Toxoplasma gondii}
infection and genotypes in Western Australian wildlife species. 2011
Australian Society for Parasitology Annual Conference, 10-13 July, The
Pullman Reef Casino Hotel, Cairns, Australia.

Worth, A.R., Lymbery A., Wayne A., Thompson RCA. (2011). Do atypical
Australian strains of \emph{Toxoplasma gondii} affect the behaviour of
hosts? Australian Society for Parasitology Annual Conference 10-13 July
2011, Cairns, Australia.

\textbf{2012}

Botero, Adriana Luz Gomez, Craig Thompson, Andrew Smith, Philip
Nicholls, Christopher Peacock, Adrian Wayne, R.C. Andrew Thompson.
(2012). The potential of mixed infections between different species of
\emph{trypanosomes}as a contributory factor in the population decline of
the threatened Australian marsupial, the Brush-tailed Bettong
\emph{Bettongia penicillata.} 2012 ASP Annual Conference 2-5 July
Launceston, Australia.

Botero Gomez, Adriana Luz, Craig Thompson, Andrew Smith, Philip
Nicholls, Christopher Peacock, Adrian Wayne, R.C. Andrew Thompson.
(2012). Genetic diversity of \emph{trypanosomes} in Western Australian
marsupials and the potential of mixed infections as a contributory
factor in the population decline of brush-tailed bettong
(\emph{Bettongia penicillata).Joint 61\textsuperscript{st}
WDA/10\textsuperscript{th} Biennial EWDA Conference --}Convergence in
Wildlife Health, Lyon 23-27 July 2012

Thompson, C. Botero Gomez LA., Smith A., Wayne A., Thompson RCA. (2012).
\emph{Trypanosome polyparasitism}and the decline of the critically
endangered Australian potoroid, the Brush-Tailed Bettong,
\emph{Bettongia penicillata}(Gray, 1837) European Multicolloquium of
Parasitology, Cluj-Napoca, Romania 25-29 July, 2012

~

\subsection{POSTERS}

Botero, A., Christopher S. Peacock, Adrian F. Wayne, RC Andrew Thompson
(2011). Biological and genetic characterisation of trypanosomes in
Western Australian marsupials. (Poster) Cairns, Australia. The
Australian Society for Parasitology annual conference.

Botero, A., Christopher Peacock, Andrew Smith, Alan Lymbery, Peta Clode,
RC Andrew Thompson. (2011). Genetic diversity and biological behaviour
of trypanosomes infecting Western Australian marsupials. (Poster) School
of Veterinary and Biomedical Sciences, Murdoch University, Perth, WA,
Australia.

Botero, A., Christopher Peacock, Peta Clode, Adrian F. Wayne, RC Andrew
Thompson. (2012). \emph{Trypanosoma copemani}infection as a contributory
factor in the population decline of the endangered Australian marsupial:
the woylie. (Poster) School of Veterinary and Biomedical Sciences,
Murdoch University, Perth, WA, Australia

Burmej, H., A. Smith, A. J. Lymbery, A. F. Wayne, K. D. Morris and A.
Thompson. (2007). The Biodiversity and Host-specificity of Fleas in the
Woylie (\emph{Bettongia penicillata}) (Poster) Australian Society of
Parasitology, Canberra, Australian Capital Territory

Burmej, H. I.,~ Wayne, A. F.,~ Abdad, Y.,~ Lymbery, A.,~ Morris, K. and
Thompson, R. C. A. (2009). Ectoparasites and the woylie (\emph{Bettongia
penicillata ogilbyi}): biodiversity and ectoparasite burden. (Poster).
In `Proceedings of the Australian Mammal Society'. Perth, Western
Australia.

Jones, K.,~ Thompson, R. C. A.,~ Wayne, A. F. and Godfrey, S. (2014).
Parasite transmission in the critically endangered woylie. (Poster). In
`Proceedings of the 2014 Australian Society for Parasitology 50th
Anniversary Conference'. Canberra, ACT Australia.

Northover, A.,~ Thompson, R. C. A.,~ Lymbery, A.,~ Wayne, A. F. and
Godfrey, S. (2014). Investigating the incidence of polyparasitism in
translocated woylies (\emph{Bettongia penicillata}): Impact on host
fitness and translocation success. (Poster) In `Proceedings of the 2014
Australian Society for Parasitology 50th Anniversary Conference'.
Canberra, ACT Australia.

Pacioni, C. (2006). A conservation conundrum: the population and
epidemiological dynamics associated with recent decline of woylies
(\emph{Bettongia penicillata}) in Australia (Poster). Post graduate
poster day. Department of Environment and Conservation, Murdoch
University, Perth, Western Australia.

Pacioni, C., Spencer, P., Wayne, A. (2007). Assessing the genetic costs
of recurring population crashes in the woylie (\emph{Bettongia
penicillata}). (Poster). Post graduate poster day. Murdoch University,
Perth, Western Australia

Pacioni, C., Spencer, P., Wayne, A. (2008). Modelling real-time
migration rates during a population decline (POSTER). Post graduate
poster day. Murdoch University, Perth, Western Australia

Wayne, A. F.,~ Ward, C.,~ Maxwell, M.,~ Vellios, C.,~ Wilson, I. and
Wayne, J. (2008). Where have all the woylies gone? (Poster) ~Department
of Conservation and Land Management, Woylie Conservation Symposium and
Workshop.

Worth, A , Lymbery A, Wayne A, Fleming,~ Thompson R.C.A. (2011). Do
Australian strains of \emph{Toxoplasma gondii} affect the behaviour of
hosts? (Poster) Australian Society for Parasitology Annual Conference
10-13 July 2011, Cairns, Australia.

Zosky, K., Wayne, A., Bryant, K. and Calver, M. (2009). Variation in the
diet of the woylie, Bettongia penicillata. (Poster). In `Proceedings of
the Australian Mammal Society'. Perth, Western Australia.

~

\subsection{PRIZES}

Botero A., Best post graduate student poster -- Understanding disease.
2012. (Sponsored by Interpath services). Post-graduate poster day,
Murdoch University.

Pacioni, C., Best post graduate introductory student poster, first year
(2006). (Sponsored by Oz Biotech). Post-graduate poster day, Murdoch
University

Caccianaga, Rian. EJG Pitman Prize, Australian Statistical Conference
(2008). Analysis of upper warren woylie data.

Pacioni, C. (2008). Best student presentation. Wildlife Disease
Association (Australasian section). Kioloa, New South Wales.

Australian Wildlife Conservancy, Biodiversity Conservation Award,
Western Australian Environment Awards (2009).

Wayne, A.F., Maxwell, M., Vellios, C., Ward, C. (2012). Science Division
Award for Contribution as a Team to Conservation Science. May 2012.
Department of Environment and Conservation.




\section*{Knowledge Transfer}

Hard copy records held at~~~~~~~ Manjimup

Electronic data stored on:~~~~~~~ Fauna File (Manjimup)

Backup copy stored at:~~~~~~~~~~~ Fauna File (Kensington)~ ~

Study sites are on the Science Plot Register (Manjimup), which provides
the information to the GIS Section.




\section*{Dataset links}






\section*{Hardcopy location}

Wildlife Science Library and Manjimup




\section*{Backup location}

None



%-----------------------------------------------------------------------------%
% Back matter
%\backmatter
\end{document}
%-----------------------------------------------------------------------------%

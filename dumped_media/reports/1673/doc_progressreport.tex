
\documentclass[version=last,
    paper=a4, % paper size
    10pt, % default font size
    usenames,
    dvipsnames,
    oneside, % ONLINE
    headings=openany, % open chapters on odd and even pages
    %toc=chapterentrywithdots, % Table of Contents style
    %BCOR=7mm, % PRINT Binding Correction
    %DIV=13, % typearea 161.54 mm x 228.46 mm, top margin 22.85 mm, inner margin 16.15 mm
    %DIV=14, % 165.00 233.36 21.21 15.00
    DIV=15 % 168.00 237.60 19.80 14.00
]{scrbook}
\usepackage{typearea}
\usepackage[automark,headsepline,footsepline]{scrlayer-scrpage} % Headers and footers

%%
%% Fonts, encoding, spacing, indentation
%%
\usepackage{txfonts}
\renewcommand{\familydefault}{\sfdefault} % Default to Sans Serif font
\usepackage[english]{babel}
\usepackage[T1]{fontenc}
\usepackage[utf8]{inputenc}

% Paragraph spacing
%\usepackage{parskip}    % Paragraph spacing
%\setlength{\parindent}{0em} % Don't indent paragraphs - ONLINE
%\setlength{\parskip}{1.3 ex plus 0.5ex minus 0.3ex} % 1-1.8 ex vertical space between paragraphs - ONLINE

% Spacing of headings
%\RedeclareSectionCommand[afterskip=3pt]{section} % less space after section
%\RedeclareSectionCommand[beforeskip=0cm]{subsection} % less space between HRule and project name
%\RedeclareSectionCommand[afterskip=0.1\baselineskip]{subsubsection} % less space after progressreport subheadings

% Table font size
\usepackage{etoolbox}
\AtBeginEnvironment{longtabu}{\footnotesize}{}{}

%%
%% Tables, columns, layout
%%
\usepackage{multicol}   % 2 col publications
\usepackage{pdflscape}  % Landscape pages
\usepackage{pdfpages}   % Include PDFs
\usepackage{hanging}    % hanging paragraphs for publications
%\usepackage{titletoc}   % Required for manipulating the table of contents
\setcounter{tocdepth}{2} % TOC list down to section
\usepackage{enumerate}  % Enumerations
\usepackage{enumitem}   % Enumerations
\usepackage{longtable}  % Multipage table
\usepackage{tabu}       %
\setlength{\tabulinesep}{1.5mm} % Consistent vertical spacing in tabu

%%
%% Graphics, images, colours
%%
\usepackage{graphicx} % embedded images
\usepackage{eso-pic} %
\usepackage{colortbl} % define custom named colours
\definecolor{RedFire}{RGB}{146,25,28}
\definecolor{ParksWildlife}{RGB}{0,85,144}
\definecolor{successbg}{RGB}{223,240,216}
\definecolor{errorbg}{RGB}{242,222,222}
\definecolor{warningbg}{RGB}{252,248,227}
\definecolor{infobg}{RGB}{217,237,247}
\definecolor{muted}{RGB}{153,153,153}
\definecolor{success}{RGB}{70,136,71}
\definecolor{error}{RGB}{185,74,72}
\definecolor{warning}{RGB}{192,152,83}
\definecolor{info}{RGB}{58,135,173}

\definecolor{required}{RGB}{192,152,83}
\definecolor{requiredbg}{RGB}{252,248,227}
\definecolor{denied}{RGB}{185,74,72}
\definecolor{deniedbg}{RGB}{242,222,222}
\definecolor{granted}{RGB}{70,136,71}
\definecolor{grantedbg}{RGB}{223,240,216}
\definecolor{not reqiured}{RGB}{153,153,153}
\definecolor{not requiredbg}{RGB}{255,255,255}

\usepackage{tikz} % Drawing
\usetikzlibrary{arrows,shapes,positioning,shadows,trees}

%%
%% Links, URLs
%%
\usepackage[
    linktoc=all,
    %colorlinks=false,  %PRINT
    colorlinks=true, % ONLINE
    linkcolor=RedFire, % ONLINE
    urlcolor=ParksWildlife, % ONLINE
    pdftitle=Progress Report SP 2000-015 (FY 2015-2016)
]{hyperref}

% Black magic to linebreak URLs
\usepackage{url}
\makeatletter
\g@addto@macro{\UrlBreaks}{\UrlOrds}
\makeatother

%%
%% Custom macros
%%
% Thick Horizontal rule
\newcommand{\HRule}{\vspace{8mm}\\\noindent\rule{\linewidth}{0.1pt}}

% Custom Tikz node for SDS diagram
\newcommand\mynode[6][]{
    \node[#1] (#2){
        \parbox{#3\relax}{
            \begin{center}
            \textbf{#4}\\
            #5\\
            \footnotesize{#6}
            \end{center}}};}



%-----------------------------------------------------------------------------%
% Headers and Footers
\automark{section}
\ohead{\href{http://sdis.dpaw.wa.gov.au/documents/progressreport/1673/}{Progress Report SP 2000-015
}}
\chead{\href{http://sdis.dpaw.wa.gov.au}{SDIS}} % center header ONLINE
\ihead{\href{http://sdis.dpaw.wa.gov.au}{
        \includegraphics[scale=0.4]{/mnt/projects/sdis/staticfiles/img/logo-dpaw.png}}}
\ifoot{\textbf{Printed}~Mon, 4 Jul 2016 16:18:21 +0800} % inner/left footer
\cfoot{} % center footer
\ofoot{\pagemark} % outer/right footer
\pagestyle{scrheadings}
\setkomafont{pageheadfoot}{\normalfont}

%-----------------------------------------------------------------------------%
\begin{document}
\raggedbottom

%-----------------------------------------------------------------------------%
% Title page
\subject{Progress Report SP 2000-015
}
\title{The population ecology of critically endangered flora
}
\subtitle{Plant Science and Herbarium
}
\author{}
\publishers{\small
    \subsection*{Project Core Team}
\begin{tabu} {X X}
\textbf{Supervising Scientist} & Colin Yates
\\
\textbf{Data Custodian} & Colin Yates
\\
\textbf{Site Custodian} & Colin Yates
\\
\end{tabu}


    \subsection*{Project status as of July 4, 2016, 4:18 p.m.}
\begin{tabu} {X X}
& Approved and active
\\
\end{tabu}

    
\subsection*{Document endorsements and approvals as of July 4, 2016, 4:18 p.m.}
\begin{tabu} {X X}

%\rowcolor{grantedbg}
    \textbf{Project Team} & 
    \textcolor{granted}{ granted}\\

%\rowcolor{grantedbg}
    \textbf{Program Leader} & 
    \textcolor{granted}{ granted}\\

%\rowcolor{grantedbg}
    \textbf{Directorate} & 
    \textcolor{granted}{ granted}\\

\end{tabu}



}
\uppertitleback{}
\lowertitleback{}
\date{}

%-----------------------------------------------------------------------------%
% Front matter
\frontmatter
\maketitle
%-----------------------------------------------------------------------------%
% Main matter
\mainmatter

\section*{The population ecology of critically endangered flora
}

C Yates, D Coates, N Gibson, C Ramalho


\section*{Context}
South-west Western Australia is a global hotspot of plant diversity.
Determining the relative importance of multiple threatening processes,
including the interactions between fragmentation and small population
processes, fire regimes, weed invasion and grazing regimes, is critical
for conservation and management of threatened flora (Declared Rare
Flora) and Threatened Ecological Communities.



\section*{Aims}
Determine the critical biological factors and the relative importance of
contemporary ecological interactions and processes that limit population
viability and persistence of threatened flora, particularly Critically
Endangered species and other key plant species occurring in Threatened
Ecological Communities.



\section*{Progress}
\begin{itemize}
\itemsep1pt\parskip0pt\parsep0pt
\item
  Continued monitoring the demography of the Critically Endangered
  \emph{Verticordia} \emph{staminosa} subsp.~\emph{staminosa} in
  relation to a drying climate in south-west Western Australia
\item
  Established micro-climate sensor array in the Ravensthorpe Range to
  develop climatic layers at appropriate scales for modelling the
  distribution of short range endemics under the influence of a
  projected warmer and drier climate.
\item
  Microclimate data collection from Ravensthorpe Range completed.
\item
  Data compiliation of Ravenshorpe Range microclimate data completed.
\item
  A paper on fire response of threatened flora~in the south west has
  been finalised and will be submitted to Biodiversity and Conservation.
\item
  Undertaken analysis of range size in~Threatened and Priority shrub
  species in the south west to better understand possible extinction
  debt.
\end{itemize}



\section*{Management implications}
The long-term monitoring of the eastern Stirling Range Montane Heath and
Thicket community and comparison with historical sources has
demonstrated dramatic changes in the community as a consequence of
\emph{Phytophora cinnamomi} and recent fire regimes. Using the
International Union for Conservation of Nature (IUCN) Ecosystem Risk
Assessment criteria this community is ranked as Critically Endangered.
Continued management of \emph{P. cinnamomi} through phosphite
application and managing the fire return interval will be critical to
conserve the remaining values of the thicket, together with an
\emph{ex-situ} conservation program for the most threatened species.

A review and analysis of the fire response of threatened flora and the
development of fields in the threatened and Priority Flora database will
assist in the design an delivery of~improved fire management protocols
for threatened flora.

The analysis of extinction debt in the highly fragmented south west
landscape will further assist in the development of protocols for
prioritising threatened flora for management intervention and recovery
actions.



\section*{Future directions}
\begin{itemize}
\itemsep1pt\parskip0pt\parsep0pt
\item
  Continue to write up and publish research on the eastern Stirling
  Range Montane Heath and Thicket Community.
\item
  Continue monitoring \emph{V. staminosa} subsp. \emph{staminosa} and
  begin analysis of long term monitoring data-set investigating the
  effects of declining rainfall on the recent dynamics of the
  population.
\item
  Analyse data from micro-climate sensor array in the Ravensthorpe
  Range.
\item
  Publish paper on fire responses of threatened flora
\item
  Continue analysis of~Threatened and Priority Flora database records to
  estimate the level of extinction debt for threatened plants in the
  highly fragmented south west landscape and prepare a draft manuscript.
\end{itemize}



%-----------------------------------------------------------------------------%
% Back matter
%\backmatter
\end{document}
%-----------------------------------------------------------------------------%



\documentclass[version=last,
    paper=a4, % paper size
    10pt, % default font size
    usenames,
    dvipsnames,
    oneside, % ONLINE
    headings=openany, % open chapters on odd and even pages
    %toc=chapterentrywithdots, % Table of Contents style
    %BCOR=7mm, % PRINT Binding Correction
    %DIV=13, % typearea 161.54 mm x 228.46 mm, top margin 22.85 mm, inner margin 16.15 mm
    %DIV=14, % 165.00 233.36 21.21 15.00
    DIV=15 % 168.00 237.60 19.80 14.00
]{scrbook}
\usepackage{typearea}
\usepackage[automark,headsepline,footsepline]{scrlayer-scrpage} % Headers and footers

%%
%% Fonts, encoding, spacing, indentation
%%
\usepackage{txfonts}
\renewcommand{\familydefault}{\sfdefault} % Default to Sans Serif font
\usepackage[english]{babel}
\usepackage[T1]{fontenc}
\usepackage[utf8]{inputenc}

% Paragraph spacing
%\usepackage{parskip}    % Paragraph spacing
%\setlength{\parindent}{0em} % Don't indent paragraphs - ONLINE
%\setlength{\parskip}{1.3 ex plus 0.5ex minus 0.3ex} % 1-1.8 ex vertical space between paragraphs - ONLINE

% Spacing of headings
%\RedeclareSectionCommand[afterskip=3pt]{section} % less space after section
%\RedeclareSectionCommand[beforeskip=0cm]{subsection} % less space between HRule and project name
%\RedeclareSectionCommand[afterskip=0.1\baselineskip]{subsubsection} % less space after progressreport subheadings

% Table font size
\usepackage{etoolbox}
\AtBeginEnvironment{longtabu}{\footnotesize}{}{}

%%
%% Tables, columns, layout
%%
\usepackage{multicol}   % 2 col publications
\usepackage{pdflscape}  % Landscape pages
\usepackage{pdfpages}   % Include PDFs
\usepackage{hanging}    % hanging paragraphs for publications
%\usepackage{titletoc}   % Required for manipulating the table of contents
\setcounter{tocdepth}{2} % TOC list down to section
\usepackage{enumerate}  % Enumerations
\usepackage{enumitem}   % Enumerations
\usepackage{longtable}  % Multipage table
\usepackage{tabu}       %
\setlength{\tabulinesep}{1.5mm} % Consistent vertical spacing in tabu

%%
%% Graphics, images, colours
%%
\usepackage{graphicx} % embedded images
\usepackage{eso-pic} %
\usepackage{colortbl} % define custom named colours
\definecolor{RedFire}{RGB}{146,25,28}
\definecolor{ParksWildlife}{RGB}{0,85,144}
\definecolor{successbg}{RGB}{223,240,216}
\definecolor{errorbg}{RGB}{242,222,222}
\definecolor{warningbg}{RGB}{252,248,227}
\definecolor{infobg}{RGB}{217,237,247}
\definecolor{muted}{RGB}{153,153,153}
\definecolor{success}{RGB}{70,136,71}
\definecolor{error}{RGB}{185,74,72}
\definecolor{warning}{RGB}{192,152,83}
\definecolor{info}{RGB}{58,135,173}

\definecolor{required}{RGB}{192,152,83}
\definecolor{requiredbg}{RGB}{252,248,227}
\definecolor{denied}{RGB}{185,74,72}
\definecolor{deniedbg}{RGB}{242,222,222}
\definecolor{granted}{RGB}{70,136,71}
\definecolor{grantedbg}{RGB}{223,240,216}
\definecolor{not reqiured}{RGB}{153,153,153}
\definecolor{not requiredbg}{RGB}{255,255,255}

\usepackage{tikz} % Drawing
\usetikzlibrary{arrows,shapes,positioning,shadows,trees}

%%
%% Links, URLs
%%
\usepackage[
    linktoc=all,
    %colorlinks=false,  %PRINT
    colorlinks=true, % ONLINE
    linkcolor=RedFire, % ONLINE
    urlcolor=ParksWildlife, % ONLINE
    pdftitle=Progress Report CF 2011-117 (FY 2015-2016)
]{hyperref}

% Black magic to linebreak URLs
\usepackage{url}
\makeatletter
\g@addto@macro{\UrlBreaks}{\UrlOrds}
\makeatother

%%
%% Custom macros
%%
% Thick Horizontal rule
\newcommand{\HRule}{\vspace{8mm}\\\noindent\rule{\linewidth}{0.1pt}}

% Custom Tikz node for SDS diagram
\newcommand\mynode[6][]{
    \node[#1] (#2){
        \parbox{#3\relax}{
            \begin{center}
            \textbf{#4}\\
            #5\\
            \footnotesize{#6}
            \end{center}}};}



%-----------------------------------------------------------------------------%
% Headers and Footers
\automark{section}
\ohead{\href{http://sdis.dpaw.wa.gov.au/documents/progressreport/1688/}{Progress Report CF 2011-117
}}
\chead{\href{http://sdis.dpaw.wa.gov.au}{SDIS}} % center header ONLINE
\ihead{\href{http://sdis.dpaw.wa.gov.au}{
        \includegraphics[scale=0.4]{/mnt/projects/sdis/staticfiles/img/logo-dpaw.png}}}
\ifoot{\textbf{Printed}~Mon, 11 Jul 2016 13:42:32 +0800} % inner/left footer
\cfoot{} % center footer
\ofoot{\pagemark} % outer/right footer
\pagestyle{scrheadings}
\setkomafont{pageheadfoot}{\normalfont}

%-----------------------------------------------------------------------------%
\begin{document}
\raggedbottom

%-----------------------------------------------------------------------------%
% Title page
\subject{Progress Report CF 2011-117
}
\title{WAMSI 2: Kimberley Marine Research Program
}
\subtitle{Marine Science
}
\author{}
\publishers{\small
    \subsection*{Project Core Team}
\begin{tabu} {X X}
\textbf{Supervising Scientist} & Kelly Waples
\\
\textbf{Data Custodian} & Corrine Severin
\\
\textbf{Site Custodian} & 
\\
\end{tabu}


    \subsection*{Project status as of July 11, 2016, 1:42 p.m.}
\begin{tabu} {X X}
& Approved and active
\\
\end{tabu}

    
\subsection*{Document endorsements and approvals as of July 11, 2016, 1:42 p.m.}
\begin{tabu} {X X}

%\rowcolor{grantedbg}
    \textbf{Project Team} & 
    \textcolor{granted}{ granted}\\

%\rowcolor{grantedbg}
    \textbf{Program Leader} & 
    \textcolor{granted}{ granted}\\

%\rowcolor{grantedbg}
    \textbf{Directorate} & 
    \textcolor{granted}{ granted}\\

\end{tabu}



}
\uppertitleback{}
\lowertitleback{}
\date{}

%-----------------------------------------------------------------------------%
% Front matter
\frontmatter
\maketitle
%-----------------------------------------------------------------------------%
% Main matter
\mainmatter

\section*{WAMSI 2: Kimberley Marine Research Program
}

K Waples, S Field


\section*{Context}
The Kimberley Marine Research Program (KMRP) is~undertaking a program of
marine research to support the management of the proposed Great
Kimberley Marine Park (which will include state marine parks at Camden
Sound, Horizontal Falls, North Kimberley, Roebuck Bay and Eighty Mile
Beach) and the coastal waters outside of these proposed marine parks.
The KMRP is being developed and implemented through the Western
Australian Marine Science Institution (WAMSI), with Parks and Wildlife
as lead agency responsible for the direction, coordination and
administration of the research program.

A Science Plan for the KMRP was developed to address priority research
and information needs to support the management of ecological and social
values in the Kimberley region through joint management of the Kimberley
marine park network. The plan comprises a suite of multidisciplinary
research projects focussed around two themes: (1) biophysical and social
characterisation, to provide the foundational datasets required for
marine park and marine resource management, as well as better
understanding and management of current human impacts; and (2)
understanding key ecosystem processes, to provide the scientific
understanding of ecosystem functioning and response to a range of
potential human impacts that are likely to arise in the future,
including climate change.

The research program will be underway between 2012 and 2017 and will
involve up to 80 scientists from eight research or management
institutions in Western Australia working collaboratively on 25 research
projects. Aboriginal involvement is a key component to the success of
the research program and all projects are engaging with Aboriginal
people and developing partnerships with the relevant traditional owners
to include their participation and to ensure the research outcomes
benefit local communities.



\section*{Aims}
\begin{itemize}
\itemsep1pt\parskip0pt\parsep0pt
\item
  Ensure the KMRP research projects are developed and delivered in line
  with the State's priority needs, and to meet Parks and Wildlife and
  joint manager management strategies for the newly-formed and proposed
  marine parks and reserves in the Kimberley.
\item
  Ensure integration of research projects within the KMRP, both in terms
  of field logistics and science findings, so that the program as a
  whole produces a clear understanding of Kimberley marine ecosystems
  and the interactions between them that is useful to management.
\item
  Ensure that the KMRP is undertaken in a culturally appropriate way in
  partnership with local Aboriginal people and delivered in a way that
  will help their longer-term aspirations.
\item
  Ensure that knowledge transfer and uptake occurs between scientists,
  joint managers and decision makers.
\end{itemize}



\section*{Progress}
\begin{itemize}
\itemsep1pt\parskip0pt\parsep0pt
\item
  Field research has been completed for thirteen~of the 21 projects with
  a field component while~the remaining eight will be conducting field
  work in 16/17. Thirteen of these 21 projects have included
  participation by traditional owners on field work.~
\item
  Final reports have been received and reviewed for three projects and
  milestone reports received, reviewed and approved for an additional 13
  projects.~
\item
  The final Project Agreement for the Indigenous Knowledge project was
  completed and signed off and the project work has commenced. This
  project includes a Saltwater Country Working Group with
  representatives from seven indigenous~communities, all participating
  in this project on integrating indigenous knowledge with western
  science for healthy country management.~~
\item
  Two workshops were held with the Saltwater Country Working Group that
  is leading the Indigenous Knowledge project to assist in developing
  and implementing this project and also to develop a forum for these
  saltwater communities to come together to share knowledge and interact
  with a regional perspective.~
\item
  Engagement with relevant traditional owner groups is ongoing and
  formal research agreements are in place with three indigenous groups.
  Relationships have been established and fostered with the IPA
  Coordinators for Dambimangari and Bardi Jawi and the Healthy Country
  officer of Wunambal Gaamberra to assist with indigenous engagement and
  the practicalities of working on country with sea ranger groups.
\item
  The Knowledge Transfer and Uptake Plan is being implemented through a
  range of activities including:

  \begin{itemize}
  \itemsep1pt\parskip0pt\parsep0pt
  \item
    The KMRP Advisory Committee has been formed and includes
    representatives from major stakeholders including Parks and Wildlife
    planning and regional management staff, and Department of Fisheries
    and indigenous representatives. The Advisory Committee has met four
    times over the past year.
  \item
    Topic-specific workshops have continued with the focus during
    2015/16 on data management and sharing.
  \item
    Two Node Leader memos have been issued to address cross-node issues
    including reporting, communication, data management and indigenous
    engagement.
  \item
    The Node Leadership has met regularly with the WAMSI CEO to discuss
    project issues, program outcomes and knowledge exchange.
  \item
    Two Science Review sessions have been held, evaluating eight KMRP
    projects. All actions arising from these reviews have been
    addressed.
  \item
    Legacy opportunities are being developed through ongoing discussions
    with the WAMSI CEO, Kimberley Institute, the Saltwater Country
    Working Group, the science community as well as a range of central
    and regional Parks and Wildlife staff.
  \item
    The Communication Strategy is being implemented through a range of
    activities including:

    \begin{itemize}
    \itemsep1pt\parskip0pt\parsep0pt
    \item
      Stories released in the WAMSI Bulletin and Kimberley Tides
      Newsletter;
    \item
      ABC radio and video communications on at least five projects;
    \item
      Production of project profiles describing research objectives,
      findings and management implications for eight projects, which are
      available on-line.;and
    \item
      The KMRP has provided financial assistance to the Science in
      Roebuck Bay Seminar Series for 2016. This has included three
      project presentations to the~community so far, with an additional
      two planned later in 2016. ~ ~ ~ ~ ~ ~ ~ ~ ~ ~ ~ ~~
    \end{itemize}
  \end{itemize}
\end{itemize}



\section*{Management implications}
The KMRP outputs will increase the capacity to manage human impacts in
the Kimberley marine reserves and improve understanding of the
ecological and socio-cultural significance of the biodiversity assets of
the Kimberley for joint managers, industry and the community. The
program also enhances the capacity of indigenous rangers and working
relationships with indigenous~communities, thereby increasing the
opportunity for more productive and bipartisan joint management in the
future. ~Findings ta dater are being used in the development of a
monitoring program for key biodiversity assets across the Kimberley.



\section*{Future directions}
\begin{itemize}
\itemsep1pt\parskip0pt\parsep0pt
\item
  Ensure that research progresses as planned and is completed in a
  timely manner.
\item
  Review final reports and develop management implications and tools
  jointly with lead scientists and managers.
\item
  Ensure that key research findings are shared with stakeholders in a
  format that allows for direct discussion and feedback on management
  needs, including products and tools that~will support~ongoing
  collaboration and management.~
\item
  Refine and develop a synthesis plan for the KMRP to provide a regional
  perspective on marine biodiversity, ecosystem function and the
  pressures on~these values which can be mitigated by management.
\item
  Continue with the implementation of knowledge exchange activities
  through the Advisory Committee and other avenues.
\end{itemize}



%-----------------------------------------------------------------------------%
% Back matter
%\backmatter
\end{document}
%-----------------------------------------------------------------------------%


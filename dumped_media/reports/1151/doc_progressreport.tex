\documentclass[version=last, paper=a4, DIV=18, usenames, dvipsnames]{scrartcl}
\usepackage{txfonts}
\usepackage{pdflscape}
\usepackage{pdfpages}
\usepackage[english]{babel} % English language/hyphenation
%%% Bootstrap colors
\definecolor{RedFire}{RGB}{146,25,28}
\definecolor{ParksWildlife}{RGB}{0,85,144}
\definecolor{successbg}{RGB}{223,240,216}
\definecolor{errorbg}{RGB}{242,222,222}
\definecolor{warningbg}{RGB}{252,248,227}
\definecolor{infobg}{RGB}{217,237,247}
\definecolor{muted}{RGB}{153,153,153}
\definecolor{success}{RGB}{70,136,71}
\definecolor{error}{RGB}{185,74,72}
\definecolor{warning}{RGB}{192,152,83}
\definecolor{info}{RGB}{58,135,173}
\usepackage[colorlinks=true,pdftitle=doc\_progressreport.pdf,linktoc=all,linkcolor=RedFire,urlcolor=ParksWildlife]{hyperref}
\usepackage{colortbl}
\usepackage{longtable}
\usepackage{tabu}
\setlength{\tabulinesep}{1.5mm}
\usepackage{enumerate}
\usepackage{enumitem}
\usepackage{fancyhdr}
\usepackage{lastpage}
\usepackage{graphicx}
\usepackage{eso-pic}
\usepackage{hyphenat}



%%% Custom headers/footers (fancyhdr package)
\fancypagestyle{plain}{
\fancyhf{}
\setlength\headheight{40pt}
\renewcommand{\headrulewidth}{0.1pt}
\renewcommand{\footrulewidth}{0.1pt}



    \fancyhead[L]{ \href{http://sdis.dpaw.wa.gov.au/documents/progressreport/1151/download/tex/}{} \newline }
\fancyhead[R]{ \hfill\href{http://www.dpaw.wa.gov.au}{Department of Parks and Wildlife}\newline\href{http://sdis.dpaw.wa.gov.au}{Pythia}}




\fancyfoot[L]{ \leftmark\newline\textbf{Last Modified}\textit{ }\quad\textbf{Printed}\textit{ July 11, 2014, 9:06 a.m. } }
\fancyfoot[R]{  \, \newline Page \thepage\ of \pageref{LastPage} } % Pagenumbering


}
\pagestyle{plain}


\newcommand{\HRule}{\rule{\linewidth}{0.1pt}}

\newcommand{\placetextbox}[3]{% \placetextbox{<horizontal pos>}{<vertical pos>}{<stuff>}
  \setbox0=\hbox{#3}% Put <stuff> in a box
  \AddToShipoutPictureFG*{% Add <stuff> to current page foreground
    \put(\LenToUnit{#1\paperwidth},\LenToUnit{#2\paperheight}){\vtop{{\null}\makebox[0pt][c]{#3}}}%
  }%
}%

\begin{document}

\setcounter{secnumdepth}{-1}


\begin{titlepage}
\begin{center}
% Upper part of the page
\begin{minipage}[t]{0.28\textwidth}
\begin{flushleft}
\href{http://www.dpaw.wa.gov.au}{\includegraphics[scale=0.6]{/var/www/sdis_8271/staticfiles/img/logo-dpaw.png}}
\end{flushleft}
\end{minipage}
\begin{minipage}[b]{0.7\textwidth}
\begin{flushright}
    \href{http://sdis.dpaw.wa.gov.au/documents/progressreport/1151/download/tex/}{}) \\
\end{flushright}
\end{minipage}
\HRule \\[0.4cm]
\vfill
\textsc{\Huge Science project 2012-27 North Kimberley Landscape Conservation Initiative: monitoring and evaluation \newline }
\vfill
\textsc{\Huge Progress Report}

\vfill\vfill\vfill\vfill
title and summary

\vfill\vfill\vfill\vfill\vfill\vfill\vfill\vfill

\textbf{Version created on} July 11, 2014, 9:06 a.m.
\vfill
\textbf{Last Modified on}  by 
\vfill\vfill
\textbf{Report Status}\\\,
\begin{tabu} to \linewidth { | X[l] | X | }
\hline
\rowcolor{infobg}
Status & Last Updated \\
\hline
\textbf{Planning - } \\
\hline
\end{tabu}
\vfill
\textbf{Science Project Overview}\\\,
\begin{tabu} to \linewidth { | X[l] | X | }
\hline
\rowcolor{infobg}
Part & Checklist Last Updated \\
\hline
\textbf{Part A - Summary \& Approval} & bla \\
\hline
\end{tabu}

\end{center}
\end{titlepage}

\setcounter{tocdepth}{2}
\tableofcontents
\clearpage






\section{Context Summary}



This project is a biodiversity monitoring and evaluation program to inform adaptive management of fire and cattle in the north Kimberley. The adaptive management program that forms the Landscape Conservation Initiative (LCI) of the Kimberley Science and Conservation Strategy commenced in 2011 in response to perceived threats by cattle and fire to biodiversity conservation in the north Kimberley. This initiative is based on the hypothesis that large numbers of introduced herbivores and the impacts of current fire regimes are associated with declines of critical-weight-range mammals, contraction and degradation of rainforest patches, and degradation of vegetation structure and habitat condition in savannas. This monitoring and evaluation program will provide a report card on performance of landscape management initiatives in the north Kimberley, particularly prescribed burning and cattle culling, in maintaining and improving biodiversity status.






\section{Aims Summary}



\begin{itemize}

  \item Inform management of biodiversity status in representative areas after prescribed burning and cattle control programs have been applied.

  \item Provide warning when landscape ecological thresholds have been reached, for example decline of mammals to below 2\% capture rate, or decline of mean shrub projected ground cover to <2\%.

  \item Compare biodiversity outcomes in intensively managed and unmanaged areas so that the effectiveness of DEC management interventions in maintaining and improving conservation values can be evaluated.

\end{itemize}






\section{Progress}



\begin{itemize}

  \item In 2012, LCI monitoring and evaluation sites in the King Leopold and Mitchell River national parks were re-sampled, and new sites established at two locations in the Prince Regent National Park, at Mount Trafalgar and Cascade Creek. Plans in 2013 are to expand the LCI monitoring and evaluation network into Drysdale River National Park and the Bachsten Creek/Mount Elizabeth area.

  \item DEC continues to develop monitoring partnerships with indigenous traditional owner groups including the Wunambal Gammbera, the Ballangara and the Willingan Aboriginal Corporations. Partnerships with other groups undertaking biodiversity monitoring activities in the region are being developed.

  \item Mammal abundance and richness in the North Kimberley is stable or improving relative to historical survey records from 1981-1982, 2003-2004 and 2007-2010. The diversity and abundance of mammals was much lower in more inland areas of the Mitchell River National Park, where cattle density is higher and the incidence and frequency of fire since 2000 has been greater.

  \item Populations of brush-tailed rabbit rat (\emph{Conilurus penicillatus)} and golden-backed tree rat (\emph{Mesembriomys macrurus}) have increased in areas where cattle have been culled over the past four years. This is strong evidence of negative effects of cattle on these species. Resurgence of \emph{C. penicillatus} appears to be restricted to areas with older growth woodlands with high tree-hollow density.

  \item No obvious relationship was found between vegetation ground cover, time since fire and critical weight range mammal diversity or abundance, despite theoretical predictions. This suggests that site-specific vegetation attributes have little direct influence on mammal abundance. Larger-scale landscape attributes, for instance the pattern of unburnt patches over tens of kilometres, may have greater influence on site specific abundance.

  \item Dingoes were common throughout the King Leopold and Mitchell River national parks, irrespective of mammal abundance. This suggests that dingoes do not have a strong negative impact on small mammal populations, and in fact may have a net positive effect through suppression of cat predation. Few cats were recorded in either park, although they were seen more often at sites in the southern part of Mitchell River National Park where mammal abundance and diversity was lower.

\end{itemize}






\section{Management implications}



\begin{itemize}

  \item There is strong evidence that cattle have negative influences on critical weight range mammals, including threatened species such as \emph{Conilurus penicillatus. _Culling programs should therefore be maintained and expanded in important conservation reserves}._

  \item There is now statistical evidence that the LCI has shifted the fire regime in the north Kimberley from dominance by late dry season bushfires to a situation where equal proportions of the country are burnt during the early and later periods of the dry season. Monitoring and evaluation data suggest that this is benefitting threatened mammal assemblages, or at least is not detrimental to them, and provides evidence that current fire management practices in the North Kimberley should be continued to maintain conservation values in the region.

  \item Lower mammal abundance and diversity at inland sites in conjunction with higher cattle and fire frequency indicates that prescribed burning and cattle culling initiatives should be expanded into these areas as a matter of priority.

  \item Dingoes can contribute to the conservation of small mammals by suppressing cats, and therefore dingo baiting should be avoided in conservation reserves.

\end{itemize}






\section{Future directions}



\begin{itemize}

  \item Monitoring and evaluation will be continued so that the effectiveness of management interventions can be evaluated.

  \item Collaborative monitoring programs will be expanded to incorporate adjoining areas on pastoral lease and Indigenous-owned land to provide comparative data on mammal populations and vegetation condition where cattle populations remain high and fire regimes are not managed.

\end{itemize}






\clearpage



\end{document}


\documentclass[version=last,
    paper=a4,                               % paper size
    10pt,                                   % default font size
    dvipsnames,
    % twoside,                                % PRINT Binding Correction
    oneside,                              % ONLINE
    headings=openany,                       % open chapters on odd and even pages
    open=any,
    BCOR=7mm,                               % PRINT Binding Correction
    %DIV=13,    % typearea 161.54mm x 228.46mm, top 22.85mm, inner 16.15mm
    %DIV=14,    % 165.00 233.36 21.21 15.00
    DIV=15,     % 168.00 237.60 19.80 14.00
    % toc=chapterentrywithdots              % Table of Contents style
]{scrbook}
\usepackage{typearea}


%------------------------------------------------------------------------------%
% Headers and footers
%------------------------------------------------------------------------------%
\usepackage[automark,headsepline,footsepline,plainfootsepline]{scrlayer-scrpage}
\automark*[section]{}
\addtokomafont{pageheadfoot}{\normalfont\footnotesize\sffamily} % Don't italicise
\renewcommand{\chaptermark}[1]{\markleft{#1}{}}     % Chapter: suppress numbering
\renewcommand{\sectionmark}[1]{\markright{#1}{}}    % Section: suppress numbering

% Header (inner, center, outer)
% \ihead{\href{http://sdis.dpaw.wa.gov.au}{\textbf{Project Plan SP 2014-003}}}
%\chead{\href{http://sdis.dpaw.wa.gov.au}{Science Directorate Information System}}
% \ohead{\href{https://www.dpaw.wa.gov.au/about-us/science-and-research}{\includegraphics[height=8mm, keepaspectratio]{/mnt/projects/sdis/staticfiles/img/logo-dpaw.png}}}

% Footer (inner, center, outer)
% \ifoot{\RaggedRight\leftmark}                       % Chapter
% \cfoot{\RaggedLeft\rightmark}                       % Section
% \ofoot[\bfseries\thepage]{\bfseries\thepage}        % Page number (also [plain])


%------------------------------------------------------------------------------%
% Fonts, encoding
%------------------------------------------------------------------------------%
%\usepackage{avant}             % Use the Avantgarde font for headings
\usepackage{txfonts}
\usepackage{mathptmx}
\usepackage{gensymb}            % provides \textdegree
\renewcommand{\familydefault}{\sfdefault} % Default to Sans Serif font
\usepackage{microtype}          % Slightly tweak font spacing for aesthetics
\usepackage[english]{babel}
\usepackage[utf8]{inputenc}  % Allow letters with accents
\usepackage[utf8]{luainputenc}  % Allow letters with accents
\usepackage[T1]{fontenc}        % Use 8-bit encoding that has 256 glyphs
\usepackage{textcomp}
\usepackage[explicit]{titlesec}           % Customise of titles
%\DeclareUnicodeCharacter{0080}{\textregistered}
\DeclareUnicodeCharacter{00B0}{\textdegree}

%------------------------------------------------------------------------------%
% Tables, columns, layout
%------------------------------------------------------------------------------%
\usepackage{etoolbox}
\AtBeginEnvironment{longtabu}{\footnotesize}{}{}  % Table font size
\usepackage{booktabs}           % Required for nicer horizontal rules in tables
\usepackage{multicol}           % 2 col publications
\usepackage{pdflscape}          % Landscape pages
\usepackage{pdfpages}           % Include PDFs
\usepackage{hanging}            % hanging paragraphs for publications
%\usepackage{titletoc}          % Manipulate the table of contents
\setcounter{tocdepth}{2}        % TOC list down to section
\usepackage{enumerate}          % Enumerations
\usepackage{enumitem}           % Enumerations
\usepackage{longtable}          % Multipage table
\usepackage{tabu}               %
\setlength{\tabulinesep}{1.5mm} % Consistent vertical spacing in tabu
\newcommand{\HRule}{\vspace{8mm}\noindent\rule{\linewidth}{0.1pt}}
\usepackage[export]{adjustbox}  % minipage, image frame


%------------------------------------------------------------------------------%
% Graphics, images, colours
%------------------------------------------------------------------------------%
\usepackage{graphicx} % embedded images
\usepackage{wrapfig}  % wrap figures in text
\usepackage{caption}  % allow unnumbered captions
\usepackage{eso-pic} % Required for specifying an image background in the title page
\usepackage{colortbl} % define custom named colours
\usepackage{xstring} % Conditionals
\usepackage{transparent} % Allow transparent images

\definecolor{RedFire}{RGB}{146,25,28}
% Following PICA branding guidelines
% https://dpaw.sharepoint.com/Divisions/pica/Documents/Branding%20guidelines.pdf
\definecolor{dpawblue}{RGB}{35,97,146}          % Pantone 647
\definecolor{dpaworange}{RGB}{237,139,0}        % Pantone 144
\definecolor{dpawgreen}{RGB}{116,170,80}        % Pantone 7489
\definecolor{dpawred}{RGB}{124,46,44}           % Paul's suggestion

% bootstrap colours
\definecolor{successbg}{RGB}{223,240,216}
\definecolor{errorbg}{RGB}{242,222,222}
\definecolor{warningbg}{RGB}{252,248,227}
\definecolor{infobg}{RGB}{217,237,247}
\definecolor{muted}{RGB}{153,153,153}
\definecolor{success}{RGB}{70,136,71}
\definecolor{error}{RGB}{185,74,72}
\definecolor{warning}{RGB}{192,152,83}
\definecolor{info}{RGB}{58,135,173}

% SDIS approval colours
\definecolor{required}{RGB}{192,152,83}
\definecolor{requiredbg}{RGB}{252,248,227}
\definecolor{denied}{RGB}{185,74,72}
\definecolor{deniedbg}{RGB}{242,222,222}
\definecolor{granted}{RGB}{70,136,71}
\definecolor{grantedbg}{RGB}{223,240,216}
\definecolor{notrequired}{RGB}{153,153,153}
\definecolor{notrequiredbg}{RGB}{255,255,255}

\usepackage{tikz} % Drawing
\usetikzlibrary{arrows,shapes,positioning,shadows,trees}


%------------------------------------------------------------------------------%
% Hyperlinks
%------------------------------------------------------------------------------%
\usepackage[open=true]{bookmark}
\usepackage{nameref}
\usepackage{ifxetex,ifluatex}
\ifxetex
  \usepackage[
    setpagesize=false,        % page size defined by xetex
    unicode=false,            % unicode breaks when used with xetex
    xetex]{hyperref}
\else
  \usepackage[unicode=true]{hyperref}
\fi

\hypersetup{
  backref=true,
  pagebackref=true,
  hyperindex=true,
  breaklinks=true,
  urlcolor=dpawblue,
  bookmarks=true,
  bookmarksopen=false,
  pdfauthor={Science and Conservation Division, Dept Parks and Wildlife, WA},
  pdftitle=Project Plan SP 2014-003
,
  colorlinks=true,
  linkcolor=dpawblue,
  pdfborder={0 0 0}}

\urlstyle{same}                         % don't use monospace font for urlstyle


%------------------------------------------------------------------------------%
% Black magic to linebreak URLs
%------------------------------------------------------------------------------%
\usepackage{url}
\makeatletter\g@addto@macro{\UrlBreaks}{\UrlOrds}\makeatother
\Urlmuskip=0mu plus 1mu


%------------------------------------------------------------------------------%
% Fix latex errors
%------------------------------------------------------------------------------%
\providecommand{\tightlist}{\setlength{\itemsep}{0pt}\setlength{\parskip}{0pt}}

% copy-pasted HTML <span> in SDIS fields becomes \text{} in tex source
\providecommand{\text}{}


%------------------------------------------------------------------------------%
% Custom Tikz node for SDS diagram
%------------------------------------------------------------------------------%
\newcommand\mynode[6][]{
  \node[#1] (#2){
    \parbox{#3\relax}{
      \begin{center}
      \textbf{#4}\\
      #5\\
      \footnotesize{#6}
      \end{center}
    }};}


%------------------------------------------------------------------------------%
% Custom no-pagebreaks-environment
%------------------------------------------------------------------------------%
\newenvironment{absolutelynopagebreak}
  {\par\nobreak\vfil\penalty0\vfilneg\vtop\bgroup}
  {\par\xdef\tpd{\the\prevdepth}\egroup\prevdepth=\tpd}


%------------------------------------------------------------------------------%
% Remove the header from odd empty pages at the end of chapters
%------------------------------------------------------------------------------%
\makeatletter
\renewcommand{\cleardoublepage}{
\clearpage\ifodd\c@page\else
\hbox{}
\vspace*{\fill}
\thispagestyle{empty}
\newpage
\fi}


%----------------------------------------------------------------------------------------
%  Page flow control
%----------------------------------------------------------------------------------------
%\widowpenalty=10000
%\clubpenalty=10000
%\vbadness=1200
%\hbadness=11000


%----------------------------------------------------------------------------------------
%   CHAPTER HEADINGS
%----------------------------------------------------------------------------------------
\newcommand{\thechapterimage}{}
\newcommand{\chapterimage}[1]{\renewcommand{\thechapterimage}{#1}}

% Numbered chapters with mini tableofcontents
\def\thechapter{\arabic{chapter}}
\def\@makechapterhead#1{
%\thispagestyle{plain}
{\centering \normalfont\sffamily
\ifnum \c@secnumdepth >\m@ne
\if@mainmatter
\startcontents
\begin{tikzpicture}[remember picture,overlay]
\node at (current page.north west)
{\begin{tikzpicture}[remember picture,overlay]
\node[anchor=north west,inner sep=0pt] at (0,0) {
\includegraphics[width=\paperwidth,height=0.5\paperwidth]{\thechapterimage}};
%------------------------------------------------------------------------------%
% Small contents box in the chapter heading
% Mini TOC background box
%\fill[color=dpawblue!10!white,opacity=.2] (1cm,0) rectangle (
%  3.5cm, % Mini TOC box width
%  -3.5cm % Mini TOC box height
%);
% Mini TOC text content
%\node[anchor=north west] at (1.1cm,.35cm) {
%  \parbox[t][8cm][t]{6.5cm}{
%    \huge\bfseries\flushleft
%    \printcontents{l}{1}{
%    \setcounter{tocdepth}{1}                   % Mini TOC level depth
%    }
% }
%};
%------------------------------------------------------------------------------%
% Chapter heading
\draw[anchor=west] (5cm,-9cm) node [
rounded corners=20pt,
fill=dpawblue!10!white,
text opacity=1,
draw=dpawblue,
draw opacity=1,
line width=1.5pt,
fill opacity=.2,
inner sep=12pt]{
    \huge\sffamily\bfseries\textcolor{black}{
      \thechapter. #1\strut\makebox[22cm]{}
    }
};
\end{tikzpicture}};
\end{tikzpicture}}
\par\vspace*{240\p@}                            % Push text below chapter image
\fi
\fi}

%------------------------------------------------------------------------------%
% Unnumbered chapters without mini tableofcontents
%------------------------------------------------------------------------------%
\def\@makeschapterhead#1{
%\thispagestyle{plain}
{\centering \normalfont\sffamily
\ifnum \c@secnumdepth >\m@ne
\if@mainmatter
\begin{tikzpicture}[remember picture,overlay]
\node at (current page.north west)
{\begin{tikzpicture}[remember picture,overlay]
\node[anchor=north west,inner sep=0pt] at (0,0) {
  \includegraphics[width=\paperwidth,height=0.5\paperwidth]{\thechapterimage}};
% Mini TOC background box
%\fill[color=dpawblue!10!white,opacity=.2] (1cm,0) rectangle (
%  3.5cm,                                       % Mini TOC box width
%  -3.5cm                                       % Mini TOC box height
%);
% Mini TOC text content
%\node[anchor=north west] at (1.1cm,.35cm) {
%  \parbox[t][8cm][t]{6.5cm}{
%    \huge\bfseries\flushleft
%    \printcontents{l}{1}{
%    \setcounter{tocdepth}{1} % Mini TOC level depth
%    }
%}
%};
\draw[anchor=west] (5cm,-9cm) node [rounded corners=20pt,
  fill=dpawblue!10!white,fill opacity=.6,inner sep=12pt,text opacity=1,
  draw=dpawblue,draw opacity=1,line width=1.5pt]{
  \huge\sffamily\bfseries\textcolor{black}{#1\strut\makebox[22cm]{}}};
\end{tikzpicture}};
\end{tikzpicture}}
\par\vspace*{240\p@}
\fi
\fi
}
\makeatother



\usepackage[automark,headsepline,footsepline,plainfootsepline]{scrlayer-scrpage}
\automark*[section]{}
\addtokomafont{pageheadfoot}{\normalfont\footnotesize\sffamily} % Don't italicise
\renewcommand{\chaptermark}[1]{\markleft{#1}{}}     % Chapter: suppress numbering
\renewcommand{\sectionmark}[1]{\markright{#1}{}}    % Section: suppress numbering

% Header (inner, center, outer)
\ihead{\href{http://sdis.dpaw.wa.gov.au/documents/projectplan/1261/}{Project Plan SP 2014-003}}
%\chead{\href{http://sdis.dpaw.wa.gov.au}{Science Directorate Information System}}
\ohead{\href{https://www.dpaw.wa.gov.au/about-us/science-and-research}{\includegraphics[height=6mm, keepaspectratio]{/mnt/projects/sdis/staticfiles/img/logo-dpaw.png}}}

% Footer (inner, center, outer)
\ifoot{\textbf{Printed}~Tue, 12 Dec 2017 16:46:17 +0800} % inner/left footer
\cfoot{}
\ofoot[\bfseries\thepage]{\bfseries\thepage}        % Page number (also [plain])


\pagestyle{scrheadings}
\setkomafont{pageheadfoot}{\normalfont}

%-----------------------------------------------------------------------------%
\begin{document}
\raggedbottom

%-----------------------------------------------------------------------------%
% Title page
\subject{Project Plan SP 2014-003
}
\title{Cat Eradication on Dirk Hartog Island
}
\subtitle{Animal Science
}
\author{}
\publishers{\small
    \subsection*{Project Core Team}
\begin{tabu} {X X}
\textbf{Supervising Scientist} & Dave Algar
\\
\textbf{Data Custodian} & M Johnston
\\
\textbf{Site Custodian} & Dave Algar
\\
\end{tabu}


    \subsection*{Project status as of Dec. 12, 2017, 4:46 p.m.}
\begin{tabu} {X X}
& Approved and active
\\
\end{tabu}

    
\subsection*{Document endorsements and approvals as of Dec. 12, 2017, 4:46 p.m.}
\begin{tabu} {X X}

%\rowcolor{grantedbg}
    \textbf{Project Team} & 
    \textcolor{granted}{ granted}\\

%\rowcolor{grantedbg}
    \textbf{Program Leader} & 
    \textcolor{granted}{ granted}\\

%\rowcolor{grantedbg}
    \textbf{Directorate} & 
    \textcolor{granted}{ granted}\\

%\rowcolor{grantedbg}
    \textbf{Biometrician} & 
    \textcolor{granted}{ granted}\\

%\rowcolor{not requiredbg}
    \textbf{Herbarium Curator} & 
    \textcolor{not required}{ not required}\\

%\rowcolor{not requiredbg}
    \textbf{Animal Ethics Committee} & 
    \textcolor{not required}{ not required}\\

\end{tabu}



}
\uppertitleback{}
\lowertitleback{}
\date{}

%-----------------------------------------------------------------------------%
% Front matter
\frontmatter
\maketitle
%-----------------------------------------------------------------------------%
% Main matter
\mainmatter



\section*{Cat Eradication on Dirk Hartog Island
}



\subsection*{Science and Conservation Division Program}

Animal Science




\subsection*{Parks and Wildlife Service}

Service 2: Conserving Habitats, Species and Ecological Communities


\subsection*{Project Staff}
\begin{tabu} {X X X}
%\rowcolor{infobg}
\textbf{Role} & \textbf{Person} & \textbf{Time allocation (FTE)}\\

Supervising Scientist & Dave Algar & 0.5\\

Technical Officer & Gary Desmond & 1.0\\

Technical Officer & Jason Fletcher & 1.0\\

Technical Officer & Neil Hamilton & 0.5\\

Research Scientist & M Johnston & 1.0\\

Technical Officer & Mike Onus & 1.0\\

Technical Officer & Cameron Tiller & 1.0\\

\end{tabu}




\subsection*{Related Science Projects}

2003-005 Development of effective broad-scale aerial baiting strategies
for the control of feral cats


\subsection*{Proposed period of the project}
Jan. 2, 2014 -- Dec. 31, 2018



\section*{Relevance and Outcomes}


\subsection*{Background}

There is extensive evidence that domestic cats (Felis catus) introduced
to offshore and oceanic islands around the world have had deleterious
impacts on endemic land vertebrates and breeding bird populations (eg.
van Aarde 1980; Moors and Atkinson 1984; King 1985; Veitch 1985; Bloomer
and Bester 1992; Bester et al. 2002; Keitt et al. 2002; Pontier et al.
2002; Blackburn et al. 2004; Martinez-Gomez and Jacobsen 2004; Nogales
et al. 2004; Ratcliffe et al. 2009; Bonnaud et al. 2010). Feral cats
have been known to drive numerous extinctions of endemic species on
islands and have contributed to at least 14\% of all 238 vertebrate
extinctions recorded globally by the IUCN (Nogales et al., 2013). In
addition, predation by feral cats currently threatens 8\% of the 464
species listed as critically endangered (Medina et al. 2011; Nogales et
al. 2013). Island faunas that have evolved for long periods in the
absence of predators are particularly susceptible to cat predation
(Dickman, 1992). Dirk Hartog Island--once a high biodiversity island--is
no exception. On Dirk Hartog Island (620km2), which is the largest
island off the Western Australian coast (Abbott and Burbidge 1995), 10
of the 13 species of native terrestrial mammals once present are now
locally extinct (Baynes 1990; McKenzie et al. 2000) probably due to
predation by cats (Burbidge 2001; Burbidge and Manly 2002). The
extirpated species of mainly medium-sized mammals include: boodie
(Bettongia lesueur),woylie (Bettongia penicillata), western barred
bandicoot (Perameles bougainville), chuditch (Dasyurus geoffroii),
mulgara (Dasycercus cristicauda), dibbler (Parantechinus apicalis),
greater stick-nest rat (Leporillus conditor), desert mouse (Pseudomys
desertor), Shark Bay mouse (Pseudomys fieldi), and heath mouse
(Pseudomys shortridgei). Only smaller species still inhabit the island:
ash-grey mouse (Pseudomys albocinereus), sandy inland mouse (Pseudomys
hermannsburgensis), and the little long-tailed dunnart (Sminthopsis
dolichura). It is possible that the banded hare-wallaby (Lagostrophus
fasciatus) and rufous hare-wallaby (Lagorchestes hirsutus) were also on
the island as they are both on nearby Bernier and Dorre Islands, and
were once on the adjacent mainland. The island also contains threatened
bird species including: Dirk Hartog Island white-winged fairy wren
(Malurus leucopterus leucopterus), Dirk Hartog Island southern emu-wren
(Stipiturus malachurus hartogi), and the Dirk Hartog Island rufous
fieldwren (Calamanthus campestris hartogi). A population of the western
spiny-tailed skink (Egernia stokesii badia) found on the island is also
listed as threatened. Since the 1860s, Dirk Hartog Island has been
managed as a pastoral lease grazed by sheep (Ovis aries) and goats
(Capra hircus). More recently, tourism has been the main commercial
activity on the island. Cats were probably introduced by early
pastoralists and became feral during the late 19th century (Burbidge
2001). The island was established as a National Park in November 2009,
which now provides the opportunity to reconstruct the native mammal
fauna (Algar et al. 2011). Dirk Hartog Island could potentially support
one of the most diverse mammal assemblages in Australia and contribute
significantly to the long-term conservation of several threatened
species. Successful eradication of feral cats would be a necessary
precursor to any mammal reintroductions.




\subsection*{Aims}

This project aims to eradicate feral cats on, and improve the
biodiversity values of Dirk Hartog Island.




\subsection*{Expected outcome}

This project aligns with the Corporate Plan and Science Division
Strategic Plan for Biodiversity Conservation Research as outlined below.

Corporate Plan

1. Conserving biodiversity

1.5 Protect diversity from threatening processes, agents and activities
including pest animals

Expand and enhance the Western Shield wildlife recovery program
incorporating introduced predator control

Expand programs for the control of pest animals

Implement integrated management strategies to control pests and diseases

Give special attention to the protection of internationally recognised
natural values of World Heritage sites

Science Division Strategic Plan for Biodiversity Conservation Research

G2 Understand the threats to biodiversity and develop evidence-based
management options to ameliorate threats

Threatened species and communities

2.7 Participate in active adaptive management programs that will lead to
improved conservation status of threatened arid zone medium-sized
mammals (links with 2.2), a group that has declined significantly since
European settlement. Adaptive management plans have been developed for
Dirk Hartog Island.

Threatening processes

2.20 Complete research into sustained, effective control of feral cats
across a range of biomes.

2.34 Develop safe and effective control technologies for feral cats,
camels, goats and pigs on DPaW-managed lands.

G6 Promote and facilitate the uptake of research findings and
communicate the contribution of science to biodiversity conservation and
natural resource management.

This process has already commenced with a number of manuscripts
published on preliminary work (see below).




\subsection*{Knowledge transfer}

This will be among the largest islands in the world where feral cat
eradication has been attempted and there will be global interest in the
outcome of this project and the techniques used. Knowledge and
technology transfer to other agencies contemplating cat eradications on
islands will be through publication of manuscripts in scientific
journals and presentations at various conferences.




\subsection*{Tasks and Milestones}

\begin{itemize}
\itemsep1pt\parskip0pt\parsep0pt
\item
  2013-14 Construction of infrastructure southern site, cat barrier
  fence and monitoring grid access. Baiting southern section late
  autumn/early winter.
\item
  2014-15 Monitoring and trapping programs southern site. Construction
  of infrastructure northern site and monitoring grid. Baiting northern
  section late autumn/early winter. Use of detector dogs in southern
  section late winter.
\item
  2015-16 Monitoring and trapping programs northern site. Use of
  detector dogs in northern section late winter.
\item
  2016-17 Surveillance monitoring.
\item
  2017-18 Surveillance monitoring.
\end{itemize}




\subsection*{References}

Aarde van, R.J. (1980). The diet and feeding behaviour of feral cats,
Felis catus, on Marion Island. South African Journal of Wildlife
Research 10, 123-128.

Abbott, I. and Burbidge, A.A. (1995). The occurrence of mammal species
on the islands of Australia: a summary of existing knowledge.
CALMScience 1(3), 259-324.

Algar, D., Hilmer, S., Onus, M., Hamilton, N. and Moore, J. (2011). New
national park to be cat-free. LANDSCOPE 26(3), 39-45.

Baynes, A. (1990). The mammals of Shark Bay, Western Australia. In:
Research in Shark Bay--Report of the France-Australe Bicentenary
Expedition Committee, (eds. P.F. Berry, S.D. Bradshaw and B.R. Wilson)
pp 313-325. Western Australian Museum, Perth, WA.

Bester, M.N.; Bloomer, J.P.; van Aarde, R.J.; Erasmus, B.H.; van
Rensburg, P.J.J; Skinner, J.D.; Howell, P.G. and Naude, T.W. (2002). A
review of the successful eradication of feral cats from sub-Antarctic
Marion Island, Southern Indian Ocean. South African Journal of Wildlife
Research 32(1), 65-73.

Blackburn, T.M., Cassey, P., Duncan, R.P., Evans, K.L. and Gaston, K.J.
(2004). Avian extinction and mammalian introductions on oceanic islands.
Science 305, 1955-1958.

Bloomer, J.P. and Bester, M.N. (1992). Control of feral cats on
sub-Antarctic Marion Island, Indian Ocean. Biological Conservation 60,
211-219.

Bonnaud, E., Zarzosa-Lacoste, D., Bourgeois, K., Ruffino, L., Legrand,
J. and Vidal, E. (2010). Top-predator control on islands boosts endemic
prey but not mesopredators. Animal Conservation, 13(6): 556-567.

Burbidge, A. (2001). Our largest island. LANDSCOPE 17(2), 16-22.

Burbidge, A.A. and Manly, B.F.J. (2002). Mammal extinctions on
Australian islands: causes and conservation implications. Journal of
Biogeography 29, 465-473.

Dickman, C.R. (1992). Conservation of mammals in the Australasian
region: the importance of islands. In: Australia and the Global
Environmental Crisis, (eds. J.N. Coles and J.M. Drew) pp 175-214.
(Academy Press, Canberra.)

Keitt, B.S.; Wilcox, C.; Tershy, B.R.; Croll, D.A. and Donlan, C.J.
(2002). The effect of feral cats on the population viability of
Black-vented Shearwaters (Puffinus opisthomelas) on Natividad Island,
Mexico. Animal Conservation 5, 217-223.

King, W.B. (1985). Island birds: will the future repeat the past? In:
Conservation of Island Birds, (ed. P.J. Moors) pp 3-15. ICBP Technical
Publication No. 3.

Martinez-Gomez, J.E. and Jacobsen, J.K. (2004). The conservation status
of Townsend's shearwater Puffinus auricularis auricularis. Biological
Conservation 116, 35-47.

McKenzie, N.L., Hall, N. and Muir, W.P. (2000). Non-volant mammals of
the southern Carnarvon Basin, Western Australia. Records of the Western
Australian Museum Supplement No. 6, 479-510.

Medina, F.M., Bonnaud, E., Vidal, E., Tershy, B.R., Zavaleta, E.S.,
Donlan, C.J., Keitt, B.S., Le Corre, M., Horwath, S.V. and Nogales, M.
(2011). A global review of the impacts of invasive cats on island
endangered vertebrates. Global Change Biology 17, 3503-3510.

Moors, P.J. and Atkinson, I.A.E. (1984). Predation on seabirds by
introduced animals, and factors affecting its severity. In: Status and
Conservation of the World's Seabirds, (eds. J.P. Croxall, P.J.H. Evans
and R.W. Schreiber). pp 667-90. ICBP Technical Publication No. 2.

Nogales, M., Martin, A., Tershy, B.R., Donlan, C.J., Veitch, D., Puerta,
N., Wood, B. and Alonso, J. (2004). A review of feral domestic cat
eradication on islands. Conservation Biology 18(2), 310-319.

Nogales, M., E. Vidal, E., Medina, F.M., Bonnaud, E., Tershy, B.R.,
Campbell, K.J. and Zavaleta, E.S. (2013). Feral cats and biodiversity
conservation: the urgent prioritization of island management. BioScience
63(10), 804-810.

Pontier, D., Say, L., Debias, F., Bried, J., Thioulouse, J., Micol, T.
and Natoli, E. (2002). The diet of cats (Felis catus L.) at five sites
on the Grande Terre, Kerguelen archipelago. Polar Biology 25, 833-837.

Ratcliffe, N., Bell, M., Pelembe, T., Boyle, D., White, R.B.R., Godley,
B., Stevenson, J. and Sanders, S. (2009). The eradication of feral cats
from Ascension Island and its subsequent recolonization by seabirds.
Oryx 44(1), 20-29.

Veitch, C.R. (1985). Methods of eradicating feral cats from offshore
islands in New Zealand. In: Conservation of Island Birds, (ed. P.J.
Moors), pp 125-141. ICBP Technical Publication No. 3.



\section*{Study design}


\subsection*{Methodology}

Site Description

Dirk Hartog Island is an area of 620km2 (25°50'S 113°0.5'E) and lies
within the Shark Bay World Heritage Property of Western Australia. The
island is approximately 79km long and a maximum of 11km wide with its
long axis in a south-east to north-west direction. Vegetation on the
island is generally sparse, low and open and comprises spinifex
(Triodia) hummock grassland with an overstorey of Acacia coriacea,
Pittosporum phylliraeoides over Acacia ligulata, Diplolaena dampieri,
Exocarpus sparteus shrubs over Triodia sp., Acanthocarpus preissii and
Atriplex bunburyana hummock grasses, chenopods or shrubs (Beard 1976).
Adjacent to the western coastline is mixed open chenopod shrubland of
Atriplex sp., Olearia oxillaris and Frankenia sp. and slightly inland in
more protected sites, Triodia plurinervata, Triodia sp., Melaleuca
huegelii, Thryptomene baeckeacea and Atriplex sp.. There are patches of
bare sand and several birridas (salt pans). On the east coast there are
patches of mixed open heath of Diplolaena dampieri, Myoporum sp. and
Conostylis sp. shrubs (Beard 1976).

The climate of the region is `semi-desert Mediterranean' (Beard 1976;
Payne et al. 1987). The mean annual rainfall for Denham (recording
station 006044, located 37km to the east of Dirk Hartog Island) is 224mm
(Bureau of Meteorology 2013; long-term records 1893-2013). The wettest
month is June with an average of 55mm. February is the hottest month
with a mean daily maximum of 31.8°C while July is the coolest month with
mean daily maximum of 21.7°C (Ibid.). Prevailing winds are southerly in
the morning swinging to the south-west in the afternoon with the sea
breeze (Bureau of Meteorology 2013).

Dirk Hartog Island is classified as a `Coastal Dune' geomorphic district
(Payne et al. 1987) and consists of coastal dunes and undulating plains
of shallow calcareous sand over limestone or calcrete. Five land systems
occur on the island, three of which (Coast, Edel and Inscription
collectively form 99\% of the island; Payne et al. 1987) and are
described below: -

Coast--occurs along the entire western side of the island and consists
of large long-walled parabolic dunes and narrow swales, unstable
blow-out areas and bare mobile dunes, minor limestone hills and rises
and steep sea cliffs (41.9\%);

Edel--occurs in eastern and south-eastern parts of the island and
consists of undulating sandy plains with minor low dunes, limestone
rises and saline flats (32.5\%);

Inscription--is found in the north-east and central-east of the island.
It consists of gently undulating sandy plains over limestone (24.3\%);

The two remaining land systems are Birrida (0.7\%) and Littoral (0.6\%).

Since the 1860s, Dirk Hartog Island has been managed as a pastoral lease
grazed by sheep (Ovis aries) and goats (Capra hircus). More recently,
tourism has been the main commercial activity on the island. Cats were
probably introduced by early pastoralists and became feral during the
late 19th century (Burbidge 2001). The island was established as a
National Park in November 2009, which now provides the opportunity to
reconstruct the native mammal fauna (Algar et al. 2011). Dirk Hartog
Island could potentially support one of the most diverse mammal
assemblages in Australia and contribute significantly to the long-term
conservation of several threatened species. Successful eradication of
feral cats would be a necessary precursor to any mammal reintroductions.

Cat Eradication

Control of feral cats is recognised as one of the most important fauna
conservation issues in Australia today and as a result, a national
`Threat Abatement Plan (TAP) for Predation by Feral Cats' has been
developed (EA 1999; DEWHA 2008). The TAP seeks to protect affected
native species and ecological communities, and to prevent further
species and ecological communities from becoming threatened. In
particular, the first objective of the TAP is to: -

\begin{itemize}
\itemsep1pt\parskip0pt\parsep0pt
\item
  Prevent feral cats from occupying new areas in Australia and eradicate
  feral cats from high-conservation-value `islands'
\end{itemize}

There are a number of obligate rules that must be met for an
island-based species eradication program to be successful (Parkes 1990;
Bomford and O'Brien 1995; Myers et al. 2000).

\begin{enumerate}
\itemsep1pt\parskip0pt\parsep0pt
\item
  community support for the program;
\item
  all target species are at risk;
\item
  the population can be killed faster than replacement;
\item
  reinvasion must be prevented. Eradication will only be temporary if
  the influx of individuals continues;
\item
  the target species can be detected at relatively low densities. Easy
  detection allows residual pockets of individuals to be identified and
  targeted for treatment;
\item
  the cost can be justified and resources must be sufficient to fund the
  program to its conclusion.
\end{enumerate}

In addition to these rules, an additional requirement for successful
eradication must be that the lines of authority must be clear and must
allow the individual or lead agency to take all necessary actions (Myers
et al. 2000).

The eradication of feral cats proposed for Dirk Hartog Island follows a
similar course of action (phases) outlined by Ramsey et al. (2011).
Their first phase involves a succession of removal events that reduce
the pest population to such low levels that further efforts often do not
find and remove any more animals. Their second phase attempts to
validate or assess whether in fact this lack of detection means
eradication may have been achieved. Assuming no more pests are found,
their third phase is one of surveillance to confirm the assessment and
it may continue until a decision is made to stop and declare the
eradication a success. Detecting survivors and interpreting the lack of
such detections to set stop rules are critical elements of this strategy
(Ibid.). Their process collects spatially explicit data on the numbers
of animals removed and on the effort to do this as it proceeds.

The size of Dirk Hartog Island, in particular its length, pose
logistical constraints on conducting an eradication campaign across the
entire island simultaneously. It is not practical or feasible to monitor
for cat activity over such a large area and as such, the eradication
campaign will be conducted in stages. Each of these stages is outlined
briefly below and then each of the techniques to be used are detailed
further later in this document.

\begin{itemize}
\item
  Stage 1 (January-April 2014) will be dedicated to establishment of
  infrastructure in the southern section (Herald Bay) including
  accommodation and equipment storage, installation of the southern
  monitoring track network and construction of the barrier fence (see
  below for details of barrier fence-line). Infrastructure construction
  will be transportable to provide flexibility in its use and options
  for utilization elsewhere at the completion of the project. Transport
  of people, equipment and supplies will follow the biosecurity
  protocols developed for Dirk Hartog Island to ensure that this project
  does not introduce or spread additional invasive species on Dirk
  Hartog Island. Time restrictions due to delays in delivery of the
  barge have meant that only infrastructure south of the barrier fence
  can be established within this time period. Infrastructure north of
  the fence (Sandy Bay accommodation site) and installation of the
  northern monitoring track network will need to be established when
  time permits later in 2014.
\item
  Stage 2 (May/June 2014-May/June 2015) {[}Phase 1{]} a baiting campaign
  will be conducted May/June 2014 south of the cat barrier fence, an
  area of approximately 220km2. An intensive monitoring program will be
  adopted following the baiting campaign to locate any cat activity.
  Where warranted, ground-baiting and trapping will be implemented to
  remove any cats that remain. {[}Phase 2{]} At the completion of the
  monitoring/trapping program, a team of detector dogs and their
  handlers will be contracted to independently verify eradication.
\item
  Stage 3 (May/June 2015-May/June 2016) {[}Phase 1{]} a baiting campaign
  will be conducted May/June 2015 north of the cat barrier fence, an
  area of approximately 420km2. As above, an intensive monitoring
  program will be adopted following the baiting campaign to locate any
  cat activity. Where warranted, ground-baiting and trapping will be
  implemented to remove any cats that remain. {[}Phase 2{]} At the
  completion of the monitoring/trapping program, a team of detector dogs
  and their handlers will be contracted to independently verify
  eradication.
\item
  Stage 4 June 2016-June 2018) {[}Phase 3{]} a two year surveillance
  monitoring program will be instigated for a further two years prior to
  any native species reintroductions.
\end{itemize}

Note that delays in getting a barge built and delivered have resulted in
a number of setbacks to the start date for this campaign. It is now
anticipated that the program will commence at the beginning of 2014 and
will be conducted over a five-year period (2014-2018). The barge is a
critical component to the conduct of the project and will be used for to
transport infrastructure materials, vehicles, equipment and personnel to
the island.

Cat Barrier Fence

A temporary electrified cat barrier fence is being constructed east-west
across the island at Herald Bay (13km in length) effectively dividing
the island into two allowing a concentration of control efforts in the
two discrete sections of the island (see location Fig. 4). Use of an
interior fence has been demonstrated to reduce the cost and increase the
overall likelihood of successful eradication on the island (Bode et al.
2013).

The fence design is based on those described in Moseby and Read (2006)
and Robley et al. (2006) and will consist of an 1800mm high `rabbit
wire' fence with a 600mm `floppy' overhang and a 300mm `foot' buried
into the ground (see example shown at Fig. 1). Due to the strong
prevailing southerly winds (right angles to the fence) the support poles
(star pickets) will be positioned every 6m with galvanised strainer
posts every 100m. Four high tensile galvanised wires will be
installed--one at ground level, one at 750mm high, another at 1500mm and
the highest at 1800mm. The fence mesh will consist of 40mm diameter
hexagonal mesh `rabbit netting'. Figure 1 shows the lower parts of the
fence as 30mm mesh; however this is not required for this fence as the
design shown at Figure 1 was to prevent infant rabbits from passing
through (rabbits are not present on Dirk Hartog Island). The mesh is to
be fixed to the northern side of the fence, with overhang and foot
facing to the north. The overhang is to be shaped using 1200mm lengths
of heavy gauge galvanised wire that is woven through to the top 60mm of
the mesh (i.e. 1200mm to 1800mm height) and then continued through the
entirety of the overhang. The 300mm foot is to be bent in the lower part
of the mesh and buried into the ground at a slight angle so that the end
of the foot is approximately 100mm deep (see Fig. 2).

Figure 1. Example of 1800mm fence with 600mm floppy top and 300mm foot,
taken from Robley et al. (2006), fence design 1

Whilst erosion is not envisaged to be a problem because the fence is at
right angles to the prevailing winds (Oceanica 2013), the fence will be
monitored on a very regular basis enabling inspection of any points
where erosion of the foot is possible (eg dunes) and can be modified as
necessary. Where the fence travels over any limestone outcropping,
especially at the western and eastern ends, the foot of the fence will
be secured to the limestone in such a fashion that there are no gaps
between the foot and the immediate substrate. A galvanised gate is to be
installed at the intersection of the main track that runs north along
the eastern coast to permit vehicle movement. The same mesh material and
design, including overhang, is to be used on the gate as is on the
fence. The gate will be positioned so that it swings 30mm above the
level of the track. A concrete mesh barrier will be placed under the
gate.

The fence will be equipped with two active electric wires, one at 1500mm
high and the other at approximately 1800mm high (different to that in
Figure 1). Wires will be positioned 70mm from the fence (anything over
80mm has been found generally ineffective for cats). Where the fence
crosses the main track the electrification wires will pass underground
in conduit. The gate will not be electrified, as it is anticipated this
could present problems arising from tourist activities. The powering
(energizer) system will be of sufficient power to energise the active
wiring to between 5-7kV along the length of the fence. The energiser
will be powered by a deep cycle or marine battery in combination with a
solar panel, the combination of which will be capable of supplying the
required power 24 hours a day, 365 days a year.

Figure 2. Fence design showing the foot bent into the lower netting and
wire overhang support

The fence is to end within 2m of the limestone cliff at the west end of
the island and within 1m of the eastern end of the island. The design
comprises a swinging gate at the end of the fence that has light chain
hanging from its base and dangling over the cliff edge. For maintenance
purposes, the locking pins of the gate can be lifted and it can be swung
landward. The electrified wiring along the fence can be extended onto
the gate. The gate will have a floppy top the same as that on the fence.
Motion-activated sirens and strobe lights will be used to deter cats
from approaching the fence-end. To monitor possible incursions a number
of motion detector cameras will be installed at both ends of the fence
and a network of traps installed to capture any invading cats. Baits
will also be deployed 5km past the barrier fence-line to reduce
incursion pressure by providing a buffer zone into which cats would
initially disperse.

Figure 3. Fence-end design

Baits and Baiting Application

Baiting is recognised as the most effective method for controlling feral
cats on mainland Australia (Short et al. 1997; EA 1999; Algar and
Burrows 2004; Algar et al. 2007; Algar et al. 2013a), and has been used
as the primary technique for eradicating cats on islands (Algar et al.
2002; Algar et al. 2010). World-wide, cat eradications have been
attempted on a number of islands with 82 successful campaigns that range
in size from 5-29,000ha (Campbell et al. 2011). There have also been
eradication attempts on a further 15 islands that have failed (Ibid.).
All successful campaigns on islands \textgreater{}2,500ha utilised
primary poisoning with toxic baits, with the exception of Santa Catalina
(3,020ha). Interestingly, seven failed campaigns on the five largest
islands (all \textgreater{}400ha) did not use toxicants (Campbell et al.
2011).

The bait designed and developed by DPaW researchers for the control of
feral cats is known as Eradicat®. These baits are manufactured at the
DPaW Bait Factory at Harvey. The Eradicat® bait is described in detail
in Algar and Burrows (2004) and Algar et al. (2007). Eradicat® baits
contain 4.5mg of directly injected toxin `1080' (sodium
monofluoroacetate). Frozen baits will be transported to Denham in the
dedicated Western Shield bait truck. Prior to deployment, baits will be
distributed on established bait racks on Peron Peninsula so they may
thaw and `sweat'. This process causes the oils and lipid-soluble digest
material to exude from the surface of the bait making the bait more
attractive to feral cats.

To optimise baiting efficacy, it is essential that baiting campaigns are
conducted prior to the onset of late autumn/winter rainfall, which
long-term weather records suggest for the Shark Bay area often begins in
late May/early June (Bureau of Meteorology 2013). A dedicated baiting
aircraft is used to deploy the baits at previously designated bait drop
points. The baiting aircraft flies at a nominal speed of 160kt and 500ft
(Above Ground Level) and a GPS point is recorded on the flight plan each
time bait leaves the aircraft. A bag of 50 baits is loaded into the `cat
bait carousel', through a funnel, for each drop. The carousel is rotated
by a DC motor speed controller which sets the carousel rotation speed
(and therefore the bait distribution distance). The carousel has been
designed with five segments separated by vanes so that, when loaded, the
baits are distributed reasonably evenly around the segments of the
carousel. The baits are loaded prior to the `aerial baiting computer
system' initiating the bait drop. When triggered by the `aerial baiting
computer system', the carousel rotates one complete revolution
distributing the cat baits over the distance set by the carousel motor
speed controller. The bombardier releases a bag of 50 baits into each
1km map grid, along flight transects 1 km apart, to achieve an
application rate of 50 baits km-2. The ground spread of 50 baits is
approximately 200 x 40m (Algar et al. 2013b).

A 25,000ha, pilot study was conducted on Dirk Hartog Island in March-May
2009 to assess the efficacy of the current aerial baiting strategy, the
primary control technique to be used in the proposed eradication
campaign (Algar et al. 2011). Prior to the baiting program, a number of
cats were fitted with GPS data-logger radio-collars, to assess baiting
efficacy and also to provide detailed information on cat activity
patterns. Two independent methods were used to monitor baiting efficacy.
Baiting efficacy was firstly determined from the percentage of
radio-collared cats found dead following the baiting program. The second
method involved surveys of cat activity at sand plots and along
continuous track transects to derive indices of activity. The difference
in indices pre- and post-baiting was then used as a measure of baiting
efficacy.

The pilot study achieved very positive results with radio-collar returns
(12 of 15 radio-collared cats ate at least one toxic bait) and indices
of activity indicating that 80+\% of the feral cat population died
following bait consumption (Algar et al. 2011). These results
demonstrated that a baiting program, with the Eradicat® bait as the
primary control technique, would be highly effective in an eradication
campaign on Dirk Hartog Island. High baiting efficacy was achieved
despite what appeared to be a plentiful prey resource on the island with
an abundant rodent population present following significant rainfall
events over the previous two years. Several radio-collared feral cats
were also implicated in predation of Loggerhead turtle (Caretta caretta)
hatchlings (Hilmer et al. 2010). It is possible that an even greater
baiting efficacy could have been achieved when the prey resource was
less abundant as optimal rates of bait consumption by feral cats are
achieved during periods of food stress (Short et al. 1997; Algar et al.
2007; Algar et al.2013a).

Consumption of baits is not only a function of bait attractiveness and
palatability but also bait encounter (Algar et al. 2007). All cats in
this study should have had some opportunity to encounter baits given the
baiting intensity and pattern flown by the aircraft. Despite being
opportunistic predators, cats will only consume a food item if they are
hungry (Bradshaw 1992); if a bait is encountered when the animal is not
hungry it may not be consumed regardless of the attractiveness of the
bait. Therefore baiting intensity and distribution pattern as well as
bait longevity are critical components of successful baiting campaigns.
Analysis of daily cat movement patterns on the island and encounter
rates for various transect spacings (see Monitoring) suggest that
reducing flight path widths to 500m may result in increased bait
encounter, particularly in the short-term and may further improve
baiting efficacy (Algar et al. 2011). However, increasing baiting
intensity beyond 50 baits km-2 along 1.0km flight paths will not
necessarily improve baiting efficacy (Algar and Burrows 2004). The home
ranges inhabited by several cats in this study were almost centrally
located between aerial bait transects and as a result these animals had
less opportunities to encounter a bait. These cats would have
experienced a greater bait encounter rate if the flight lines were at
0.5km intervals rather than 1.0km.

All three radio-collared cats that survived the baiting campaign were in
excellent body condition and were obviously not food stressed. Two of
these animals occupied/patrolled beaches while the remaining cat was
utilising other food sources as it was not thought to be accessing
beaches where turtle hatchlings were available. All three animals
frequented one or more `Bait Exclusion Zones' but also spent time where
bait encounter was likely. The Western Australian guidelines for use of
1080 baits provides for `Bait Exclusion Zones' of 500m radius at and
around sites subject to high human visitation. The eradication plan will
seek exemption from the requirement to establish `Bait Exclusion Zones',
as these may provide a bait-free refuge for cats, particularly those
with small home ranges such as sub-adults.

The information gained from this pilot study has been used in the
planning of flight transects to maximise the likelihood of feral cats
encountering a bait within the shortest possible time, rather than
arbitrarily assigning transect spacing. This will optimise baiting
efficacy and provide a more cost-effective baiting campaign. The
modifications proposed to the current baiting regime will maximise the
likelihood of the entire cat population encountering a bait(s) when
hungry. Modifying the bait pattern to provide baits in more complex
topography such as that fringing the coast is also proposed. Unlike
baiting campaigns on mainland sites where baits are required to be laid
annually to control cat numbers, a single-site baiting leading to
eradication of cats on Dirk Hartog Island will provide extremely
cost-effective control.

In summary, the results obtained in the pilot trial demonstrated that a
baiting program, with the Eradicat® bait as the primary control
technique, will be highly effective in an eradication campaign on Dirk
Hartog Island. The effectiveness of this management tool on the island
could be further optimised by integrating the following recommendations:
-

\begin{itemize}
\itemsep1pt\parskip0pt\parsep0pt
\item
  Baiting intensity should be increased from 50 to 100 baits/km2. This
  would maximise the likelihood of the entire cat population
  encountering a bait (s) when hungry;
\item
  Modifying the bait pattern to provide baits in narrower transects as
  well as more baits in complex topography such as dunes and/or swales.
  Cats would experience a greater bait encounter rate if the flight
  lines were at 500m intervals rather than 1000m;
\item
  Considering whether `Bait Exclusion Zones' are necessary on the island
  as they would provide a bait free refuge for cats particularly those
  with small home ranges, such as sub-adults;
\item
  Timing of bait application could be delayed to allow for cooler
  weather (but not rain) and completion of turtle hatching.
\end{itemize}

It is unlikely that an aerial baiting program alone will result in cat
eradication. Following the monitoring survey for cat activity
post-baiting (see Monitoring), a ground baiting program will be
undertaken with baits either laid by hand or strung at `Bait Suspension
Devices' (see Algar and Brazell 2008; Algar et al. in press) in the
vicinity of cats that are still alive. This will then be followed by the
intensive monitoring and trapping campaign to remove cats that survive
baiting outlined below.

Monitoring

Monitoring is to be conducted within an adaptive management framework to
inform decision making, whereby actions will be taken if/when certain
events occur. The monitoring program will provide information on where
control effort is required and whether additional measures and/or
resources are needed. Decisions must be made while there is opportunity
to act; delaying decisions will remove these chances and risk the
success of the eradication campaign. This framework will allow project
management to change tactics as results dictate; flexibility in planning
is therefore essential. A key component of this eradication program is
to employ monitoring methods that will provide quantitative estimates of
the effectiveness of control operations; the techniques must also be
capable of detecting animals at low density populations. Of necessity,
the monitoring of feral cat activity must be conducted across the entire
island and requires the provision of suitable access (see below; Survey
tracks). Monitoring of cat activity on the island will serve four
purposes: -

\begin{enumerate}
\itemsep1pt\parskip0pt\parsep0pt
\item
  Firstly, monitoring surveys conducted pre- and post-baiting will be
  used in conjunction with radio-collar returns to assess baiting
  efficacy;
\item
  Secondly, following the baiting campaigns an intensive monitoring
  program will be adopted to locate any cat activity and where
  warranted, ground-baiting and a trapping program will be implemented
  to remove any cats that remain;
\item
  Thirdly, following the completion of each of the two intensive
  monitoring programs (southern {[}winter 2015{]} and northern sections
  {[}winter 2016{]}) detector dogs and their handlers will be contracted
  to verify the absence of cats and to independently corroborate that
  eradication has been successfully achieved;
\item
  Finally, surveillance monitoring for cat activity will then be
  conducted on a seasonal basis for a following two years prior to the
  native species reintroductions, as an insurance policy that no cats
  have been overlooked or reintroduced.
\end{enumerate}

Survey tracks

Much of the former pastoral road network has regenerated, with many
roads and fence lines shown on the pastoral plan being impassable and in
many instances difficult to identify. The monitoring program is to be
conducted from ATVs which will need to traverse the entire island in a
safe and efficient manner. Prior to implementing the monitoring program,
it will be necessary to construct a grid network of survey tracks to
allow monitoring of cat activity across the island. The spacing of these
tracks must of necessity be of width that will permit detection of any
cat during the survey period (i.e. two weeks each month) and therefore
provide confidence in the survey technique. Information obtained from
the GPS data-logger radio-collars during the pilot study (Algar et al.
2011) was used to determine the likelihood of detection and to optimise
the proposed spacing of the survey tracks for the eradication program.
Analysis was performed in R2.9.0. (R Development Core Team 2009). Data
from all cats alive immediately prior to baiting were utilised but only
locations with an HDOP6 are less precise and are more likely to have
shown a cat crossing a track line which it did not actually cross. For
each simulation, four sets of track lines were located at random
starting points and spaced at intervals of 500, 1,000, 1,500 and 2,000m
respectively. Track lines were parallel to the long axis of the island
and the orientation of the dune system. This is the preferred course for
survey tracks for logistic reasons and to minimise disturbance and
erosion to dunes. For each set of track lines, the time from initial
collaring of each cat to when it would have first crossed the track line
was determined. This process was repeated 5,000 times with different
random starting locations for the track lines each time. For each
spacing (500, 1,000, 1,500 and 2,000m), the 95th percentile of the time
to cross a track line for each cat was interpreted as the time it would
take to be 95\% sure of detecting that cat during survey. Analysis of
daily movement patterns, pooled for all cats, indicated that the time
(mean + s.e.) to encounter track lines spaced at 500, 1000, 1500 and
2000m was 1.0 + 0.2, 1.8 + 0.5, 4.6 + 1.1 and 12.2 + 3.2 days
respectively. Cat movement data suggest that placement of monitoring
tracks at a width of approximately 2.0km across the full length of the
island will be sufficient to enable detection of these animals within
each survey period. Choice of this spacing for the monitoring tracks and
separation of camera traps (see later) is further strengthened by data
collected on home ranges (100\% MCP) of the radio-collared cats in the
pilot study which were 12.7 km2 for males and 7.8km2 for females
(Johnston et al. 2010). Thus, every cat has at least some probability of
its sign being observed or the animal being photographed (i.e., one
camera trap within each animal's home range).

Placement of monitoring tracks will strike a balance between limiting
vegetation disturbance and erosion and optimizing cat encounters during
survey periods. The location of the access route (tracks) has also taken
into account factors such as logistics and efficiency of servicing. The
track network linking the monitoring plots utilises existing tracks and
fence lines where possible and has considered visual impacts and erosion
potential. There is a total of 184km of secondary tracks on the island
of which 103km will be used to access some of the monitoring plots and
62km will need to be maintained to provide ATV access. Approximately 147
km of new track will be created across the island. Any new tracks will
be established using a rubber tracked skid steer loader (Positrack)
fitted with a front mounted mulcher that will permit vegetation
rehabilitation of the survey tracks at the completion of the program.
The new tracks will be the width of the Positrack (less than 2m wide)
and the vegetation mulched as the machine moves forward along the
proposed alignment, leaving the mulched vegetation laid on the track as
it progresses. Natural openings in the vegetation will be used in
preference to clearing vegetation. The new tracks were initially located
by DPaW staff (primarily P. Rampant; Remote Sensing Officer and J.
Asher; Project Manager `Dirk Hartog Island Restoration') using a number
of software packages and datasets (listed by Oceanica 2013). Several
minor changes were made to the location of these tracks following
consultation with Oceanica who have endorsed track location and are
considered appropriate to minimise the risk of erosion (Oceanica 2013).

The following guidelines were used to help to determine the track
position (Oceanica 2013):

1. Utilise existing tracks and fence lines;

2. Relocate monitoring points to existing tracks if point is within a
few hundred metres;

3. Avoid placing monitoring points on birridas or on active dunes;

4. Avoid areas of floristic, Aboriginal heritage or European heritage
importance;

5. Avoid unstable and sensitive vegetation;

6. Avoid crossing dunes;

7. Minimise slope on proposed tracks;

8. Position tracks at low points in the terrain;

9. Minimise track visibility when approaching the island via boat;

10. Minimise visibility down the proposed tracks from main access roads;

11. Minimise wind tunnel effect from prevailing southerly wind by
avoiding long straight north-south tracks.

The creation of the combined monitoring track network and in some cases
the repositioning of the monitoring plots was an interactive 2D/3D
process. The track placement decisions were primarily determined by the
above guidelines and were applied in approximately the above order.
Tracks and plots if necessary, were positioned in a 2D environment using
ArcMap (ESRI) and were viewed and refined in the 3D environment of
ArcScene (a component of ESRI's 3D Analyst extension).

The ATV tracks and fence line will be monitored for signs of potential
erosion on a daily basis for two weeks of each month during the cat
monitoring phase. Any early signs of erosion will be immediately
ameliorated with brushing and/or matting. Tracks not required following
completion of the cat eradication project will be allowed to
rehabilitate.

1. Monitoring baiting efficacy

Two independent methods for monitoring baiting efficacy will be
implemented: 1) trap, radio-collar and release of feral cats prior to
the baiting program (see Trapping); and 2) detection of site occupancy
using camera trap surveys of feral cat activity. The proportion of
collared feral cats killed, i.e. direct mortality, and the difference in
occupancy before and following baiting (determined by activity at camera
trap monitoring plots; see Camera trapping) will be used as a measure of
baiting efficacy.

2. Intensive monitoring program

An intensive monitoring program will be conducted post-baiting to locate
cats that have survived so that they may be removed. Rapid detection of
cats surviving the initial application of baits is critical to
successfully eradicating cats as soon as possible. Detection of cats
will be based on camera trap surveys which will be conducted for a
two-week period each month when staff are not on the island so human
disturbance will be at a minimum and the population will be closed (i.e.
no removal of animals). Analysis of data from each camera trap survey
will be conducted immediately upon return to the island to provide
information on areas of cat activity and inform required control
measures to be undertaken during that field period. In addition to the
camera trap data, evidence of sign, principally cat track activity will
also be used to locate areas where control effort is required. The
network of survey tracks many of which will consist of a sandy surface
substrate will enable cat footprints to be observed and thus daily cat
activity to be recorded along their length. In addition to the
conducting surveys along the monitoring tracks, sweeps along all beaches
will be routinely undertaken. Track activity is unlikely to be used to
provide indices of cat activity as animals will be removed during these
observational periods however; evidence of sign, location and effort
will be recorded. The detection of any cat sign during the course of the
intensive monitoring programs will instigate an intensive ground baiting
of the general location and the implementation of a trapping program at
the site of the sign and surrounding area. It is anticipated that 12
intensive monitoring programs will be conducted in both the southern and
northern sections of the island.

Cats are social animals and when at low numbers are likely to actively
seek other cats in the area, particularly during the breeding season.
The possibility of using several `sentinel cats' will be investigated;
these animals are brought in from outside the control area and
sterilised (not de-sexed) so they maintain normal hormonal activity.
Fitting radio-collars to these animals will allow their locations to be
plotted and their subsequent removal. Information collected from the
animals will be useful on two fronts: -

\begin{enumerate}
\itemsep1pt\parskip0pt\parsep0pt
\item
  firstly, recording these animals on the monitoring transects will give
  further confidence in cat detectability especially at low population
  densities;
\item
  secondly, once the `sentinel cats' have been removed intensive
  searching of the area for subsequent cat sign will indicate whether
  the `sentinel cats' had located additional animals in the area.
\end{enumerate}

3. Verification monitoring using detector dogs to corroborate absence of
cats

Following the completion of each of the two monitoring programs,
specialist detector dogs and their handlers will be contracted to: 1)
detect the presence and location of cats so that they can be destroyed
(shot) and/or 2) further independently verify the absence of cats and
corroborate that eradication has been successfully achieved. The use of
detector dogs can greatly increase the efficacy of monitoring especially
when cats are wary of other methods and/or are at low densities. Dogs
are able to detect cats from wind-borne or ground scents and track them
to their resting places or dens, enabling their removal. Dogs are
especially useful when cat densities are low because of their keen sense
of smell and ability to follow scents over large distances in a
relatively short amount of time.

The two sections of the island will be divided into blocks, allowing
dogs and their handlers to systematically search the island. Blocks will
be worked in an order that reflects wind direction; blocks yet to be
worked remain upwind of teams who maintain a rolling front going into
the wind. This will provide dogs with the best chance of detecting wind
scents and minimises opportunities for cats to detect and avoid dogs.
Blocks will be delineated by GIS and uploaded into GPS units carried by
dog handlers. Handler's GPS units will be programmed to create track
logs, showing where they have been. Dogs will be fitted with GPS
collars, which along with the handler's units will be downloaded each
evening to the GIS, allowing staff to visually track progress, determine
any areas not sufficiently covered.

`Sentinel cats' may again be used to test the ability of the detector
dogs in a blind trial. Dog handlers will not know whether or not
`sentinel cats' have been deployed, their number or location.

4. Surveillance monitoring

Surveillance monitoring for cat activity will be conducted on a seasonal
basis for a following two years prior to the native species
reintroductions, as an insurance policy that no cats have been
overlooked or reintroduced. Surveillance monitoring will employ both
camera trap monitoring and cat sign searches along the survey tracks. It
is anticipated that surveillance monitoring will be conducted over a
10-day period in each of the southern and northern sections every three
months but will be guided by `stopping rules' used to determine the
probability that eradication has occurred (see later).

Camera trapping

Monitoring the abundance of feral cats, like many mammalian carnivores,
is difficult because they occur at low densities, have large home ranges
and tend to be secretive and cryptic (Saunders et al. 1995; Witmer 2005;
Long et al. 2007; Marks et al. 2009). Capture-recapture studies to
estimate abundance are usually impractical (especially in eradication
program where animals are removed when captured) because the animals are
difficult to trap, leading to low capture rates and recapture
probabilities (Saunders et al. 1995). Consequently most monitoring
schemes rely on indices of abundance derived from data such as den
counts (Coman et al. 1991), catch per unit effort indices (Algar and
Kinnear 1992), spotlight surveys (Edwards et al. 2000; Sharp et al.
2001; Vine et al. 2009), scent station counts, (Phillips 1983; Harrison
et al. 2002; Schauster et al. 2002), and track counts (Engeman et al.
1998; Mahon et al. 1998; Edwards et al. 2000; Algar and Burrows 2004;
Engeman 2005; Algar et al. 2013a). Each of these techniques has
advantages and disadvantages but they are unable to identify individuals
and potentially confound animal activity with abundance (Anderson 2001).

Techniques that identify individual animals provide data that can be
analysed using conventional capture-recapture statistical modelling, to
provide robust estimates of abundance. In situations where individual
animals can be identified from photographs (e.g. through variations in
natural pelage patterns or markings), use of remotely deployed automatic
cameras in camera trap studies have been employed to provide estimates
of abundance. Camera trap studies have provided estimates of abundance
for a number of species, particularly felids such as tigers (Panthera
tigris, Karanth 1995; Karanth and Nichols 1998; Karanth et al. 2006)
snow leopards (Uncia uncia, Jackson et al. 2006), bobcats (Lynx rufus,
Heilbrun et al. 2006), jaguars (Panthera onca, Maffei et al. 2004;
Silver et al. 2004; Soisalo and Cavalcanti 2006), ocelots (Leopardus
pardalis, Trolle and Kery 2003). Camera trap studies are useful in
providing information on feral cat presence/absence but usually
individuals cannot be easily distinguished with any degree of certainty
especially if black cats are present in the population. At a rudimentary
level presence/absence data from camera trap sites can be used to
provide indices of relative activity (e.g. Jenks et al. 2011). Raw
detection rates (i.e. total number of events/number of sites) are naive
estimates of occupancy that do not account for probability of detection
(Long et al. 2010). If detection probabilities are determined, estimates
of occupancy can be derived from presence/absence data (MacKenzie et al.
2006; Long and Zielinski 2008). Occupancy is often used as a metric for
estimating various species occurrence and is a function of abundance as
it concerns the probability of a particular animal being at a given site
(MacKenzie et al. 2006; Long et al. 2010; O'Connell and Bailey 2011), in
this case a camera trap plot. In addition, occupancy surveys require
lower sample sizes than abundance surveys (MacKenzie et al. 2006).

Camera traps will provide an ideal technique for monitoring the impact
of control measures through the progression of the eradication campaign
as they will allow remote monitoring of cats following each control
period/activity when staff are not present on-island. Automated cameras
will be installed at a minimum of 50 (see Fig. 4) and 104 (see Fig. 5)
locations in the southern and northern zones respectively to survey for
the presence of feral cats. Additional cameras may be installed once the
monitoring tracks have been positioned and assessed. The cameras to be
used are Reconyx HC600 (Reconyx, Wisconsin; USA) and will be set
horizontally. Cameras are to set on ``Scrape'' program which records
five pictures per trigger, picture interval is on ``RapidFire'' which is
two frames per second; there is no quiet period. At each plot the camera
will be mounted 30 cm above the ground on a 45 cm heavy duty plastic
tent peg. The camera is to face south, and a 3m x 1m strip of vegetation
will be pruned to ground level between camera and lures to provide an
uninterrupted view between lure and camera and minimise false detections
from taller, moving vegetation. A combination of olfactory and visual
lures will be used to attract cats to the camera traps. Plots that do
not have lures often generate sample sizes that are too low to
adequately monitor population changes (Fleming et al. 2001), they also
provide more precise population estimates by increasing the number of
recaptures (Gerber et al. 2012). Lures for the camera trap surveys will
consist of a spice jar with perforated lid containing an oil-based
scented lure (Catastrophic, Outfoxed Pest Control, Victoria) which is
attached to a wooden stake approximately 30 cm from the ground. A 1.5 m
long bamboo cane is joined to the wooden stake, with white synthetic
turkey feathers connected to the cane approximately 30 cm above the
scented lure. A 30 cm length of tinsel is fixed to the top of the stake
in a position where it is not within the field of view of the camera.
This set up and combination of lures has recently proven successful
elsewhere in attracting cats to the camera traps (Tiller et al. 2013;
Comer et al. in draft; Johnston et al. in press). Lures to be used in
the leg-hold trapping program (see Trapping) will not be used for camera
traps.

Camera trap plots will be established well in advance of their use to
avoid any neophobic reaction by cats and remain in place during the
course of the eradication campaign. Lures, memory cards and batteries
will be removed at the end of each survey period (i.e. commencement of
each field trip) and reinstated/refreshed at the commencement of each
survey period (termination of each field trip). At the time of
installation, all cameras will be test-fired to confirm functionality
and correctness of aim. A series of set-up photos will be taken in which
a white board with the location details and date recorded will be held
in front of the camera.

The software program `Camera Base' (Atrium Biodiversity 2013) will be
used for data storage and management. This software program also allows
exporting data into formats for further analysis. Daily collection of
detection--non-detection data will encompass the period from 18:00h one
day to 18:00h the following day rather than the conventional 24h period
and will be considered a sampling occasion.




\subsection*{Biometrician's Endorsement}

granted



\section*{Data management}


\subsection*{No. specimens}






\subsection*{Herbarium Curator's Endorsement}

not required




\subsection*{Animal Ethics Committee's Endorsement}

not required




\subsection*{Data management}






\section*{Budget}

\section*{Consolidated Funds }



\begin{longtabu} to \linewidth { |  X | X | X | X | }
\hline
\rowcolor{infobg}
Source & Year 1 & Year 2 & Year 3\\
\hline
\endhead



FTE Scientist &  &  & \\



FTE Technical &  &  & \\



Equipment &  &  & \\



Vehicle &  &  & \\



Travel &  &  & \\



Other &  &  & \\



Total &  &  & \\


\hline
\end{longtabu}



\section*{External Funds }



\begin{longtabu} to \linewidth { |  X | X | X | X | }
\hline
\rowcolor{infobg}
Source & Year 1 & Year 2 & Year 3\\
\hline
\endhead



Salaries, Wages, OVertime &  &  & \\



Overheads &  &  & \\



Equipment &  &  & \\



Vehicle &  &  & \\



Travel &  &  & \\



Other &  &  & \\



Total &  &  & \\


\hline
\end{longtabu}





%-----------------------------------------------------------------------------%
% Back matter
%\backmatter
\end{document}
%-----------------------------------------------------------------------------%


\documentclass[version=last,
    paper=a4,                               % paper size
    10pt,                                   % default font size
    dvipsnames,
    % twoside,                                % PRINT Binding Correction
    oneside,                              % ONLINE
    headings=openany,                       % open chapters on odd and even pages
    open=any,
    BCOR=7mm,                               % PRINT Binding Correction
    %DIV=13,    % typearea 161.54mm x 228.46mm, top 22.85mm, inner 16.15mm
    %DIV=14,    % 165.00 233.36 21.21 15.00
    DIV=15,     % 168.00 237.60 19.80 14.00
    % toc=chapterentrywithdots              % Table of Contents style
]{scrbook}
\usepackage{typearea}


%------------------------------------------------------------------------------%
% Headers and footers
%------------------------------------------------------------------------------%
\usepackage[automark,headsepline,footsepline,plainfootsepline]{scrlayer-scrpage}
\automark*[section]{}
\addtokomafont{pageheadfoot}{\normalfont\footnotesize\sffamily} % Don't italicise
\renewcommand{\chaptermark}[1]{\markleft{#1}{}}     % Chapter: suppress numbering
\renewcommand{\sectionmark}[1]{\markright{#1}{}}    % Section: suppress numbering

% Header (inner, center, outer)
% \ihead{\href{http://sdis.dbca.wa.gov.au}{\textbf{Concept Plan SP 2014-021}}}
%\chead{\href{http://sdis.dbca.wa.gov.au}{Science Directorate Information System}}
% \ohead{\href{https://www.dbca.wa.gov.au/science/10-biodiversity-and-conservation-science}{
% \includegraphics[height=8mm, keepaspectratio]{/usr/src/app/staticfiles/img/logo-dbca-bcs.jpg}}}

% Footer (inner, center, outer)
% \ifoot{\RaggedRight\leftmark}                       % Chapter
% \cfoot{\RaggedLeft\rightmark}                       % Section
% \ofoot[\bfseries\thepage]{\bfseries\thepage}        % Page number (also [plain])


%------------------------------------------------------------------------------%
% Fonts, encoding
%------------------------------------------------------------------------------%
%\usepackage{avant}             % Use the Avantgarde font for headings
\usepackage{txfonts}
\usepackage{mathptmx}
\usepackage{gensymb}            % provides \textdegree
\renewcommand{\familydefault}{\sfdefault} % Default to Sans Serif font
\usepackage{microtype}          % Slightly tweak font spacing for aesthetics
\usepackage[english]{babel}
\usepackage[utf8]{inputenc}  % Allow letters with accents
\usepackage[utf8]{luainputenc}  % Allow letters with accents
\usepackage[T1]{fontenc}        % Use 8-bit encoding that has 256 glyphs
\usepackage{textcomp}
\usepackage[explicit]{titlesec}           % Customise of titles
%\DeclareUnicodeCharacter{0080}{\textregistered}
\DeclareUnicodeCharacter{00B0}{\textdegree}

%------------------------------------------------------------------------------%
% Tables, columns, layout
%------------------------------------------------------------------------------%
\usepackage{etoolbox}
\AtBeginEnvironment{longtabu}{\footnotesize}{}{}  % Table font size
\usepackage{booktabs}           % Required for nicer horizontal rules in tables
\usepackage{multicol}           % 2 col publications
\usepackage{pdflscape}          % Landscape pages
\usepackage{pdfpages}           % Include PDFs
\usepackage{hanging}            % hanging paragraphs for publications
%\usepackage{titletoc}          % Manipulate the table of contents
\setcounter{tocdepth}{2}        % TOC list down to section
\usepackage{enumerate}          % Enumerations
\usepackage{enumitem}           % Enumerations
\usepackage{longtable}          % Multipage table
\usepackage{tabu}               %
\setlength{\tabulinesep}{1.5mm} % Consistent vertical spacing in tabu
\newcommand{\HRule}{\vspace{8mm}\noindent\rule{\linewidth}{0.1pt}}
\usepackage[export]{adjustbox}  % minipage, image frame


%------------------------------------------------------------------------------%
% Graphics, images, colours
%------------------------------------------------------------------------------%
\usepackage{graphicx} % embedded images
\usepackage{wrapfig}  % wrap figures in text
\usepackage{caption}  % allow unnumbered captions
\usepackage{eso-pic} % Required for specifying an image background in the title page
\usepackage{colortbl} % define custom named colours
\usepackage{xstring} % Conditionals
\usepackage{transparent} % Allow transparent images

\definecolor{RedFire}{RGB}{146,25,28}
% Following PICA branding guidelines
% https://dpaw.sharepoint.com/Divisions/pica/Documents/Branding%20guidelines.pdf
\definecolor{dpawblue}{RGB}{35,97,146}          % Pantone 647
\definecolor{dpaworange}{RGB}{237,139,0}        % Pantone 144
\definecolor{dpawgreen}{RGB}{116,170,80}        % Pantone 7489
\definecolor{dpawred}{RGB}{124,46,44}           % Paul's suggestion

% bootstrap colours
\definecolor{successbg}{RGB}{223,240,216}
\definecolor{errorbg}{RGB}{242,222,222}
\definecolor{warningbg}{RGB}{252,248,227}
\definecolor{infobg}{RGB}{217,237,247}
\definecolor{muted}{RGB}{153,153,153}
\definecolor{success}{RGB}{70,136,71}
\definecolor{error}{RGB}{185,74,72}
\definecolor{warning}{RGB}{192,152,83}
\definecolor{info}{RGB}{58,135,173}

% SDIS approval colours
\definecolor{required}{RGB}{192,152,83}
\definecolor{requiredbg}{RGB}{252,248,227}
\definecolor{denied}{RGB}{185,74,72}
\definecolor{deniedbg}{RGB}{242,222,222}
\definecolor{granted}{RGB}{70,136,71}
\definecolor{grantedbg}{RGB}{223,240,216}
\definecolor{notrequired}{RGB}{153,153,153}
\definecolor{notrequiredbg}{RGB}{255,255,255}

\usepackage{tikz} % Drawing
\usetikzlibrary{arrows,shapes,positioning,shadows,trees}


%------------------------------------------------------------------------------%
% Hyperlinks
%------------------------------------------------------------------------------%
\usepackage[open=true]{bookmark}
\usepackage{nameref}
\usepackage{ifxetex,ifluatex}
\ifxetex
  \usepackage[
    setpagesize=false,        % page size defined by xetex
    unicode=false,            % unicode breaks when used with xetex
    xetex]{hyperref}
\else
  \usepackage[unicode=true]{hyperref}
\fi

\hypersetup{
  backref=true,
  pagebackref=true,
  hyperindex=true,
  breaklinks=true,
  urlcolor=dpawblue,
  bookmarks=true,
  bookmarksopen=false,
  pdfauthor={Biodiversity and Conservation Science, Department of Biodiversity, Conservation and Attractions, WA},
  pdftitle=Concept Plan SP 2014-021
,
  colorlinks=true,
  linkcolor=dpawblue,
  pdfborder={0 0 0}}

\urlstyle{same}                         % don't use monospace font for urlstyle


%------------------------------------------------------------------------------%
% Black magic to linebreak URLs
%------------------------------------------------------------------------------%
\usepackage{url}
\makeatletter\g@addto@macro{\UrlBreaks}{\UrlOrds}\makeatother
\Urlmuskip=0mu plus 1mu


%------------------------------------------------------------------------------%
% Fix latex errors
%------------------------------------------------------------------------------%
\providecommand{\tightlist}{\setlength{\itemsep}{0pt}\setlength{\parskip}{0pt}}

% copy-pasted HTML <span> in SDIS fields becomes \text{} in tex source
\providecommand{\text}{}


%------------------------------------------------------------------------------%
% Custom Tikz node for SDS diagram
%------------------------------------------------------------------------------%
\newcommand\mynode[6][]{
  \node[#1] (#2){
    \parbox{#3\relax}{
      \begin{center}
      \textbf{#4}\\
      #5\\
      \footnotesize{#6}
      \end{center}
    }};}


%------------------------------------------------------------------------------%
% Custom no-pagebreaks-environment
%------------------------------------------------------------------------------%
\newenvironment{absolutelynopagebreak}
  {\par\nobreak\vfil\penalty0\vfilneg\vtop\bgroup}
  {\par\xdef\tpd{\the\prevdepth}\egroup\prevdepth=\tpd}


%------------------------------------------------------------------------------%
% Remove the header from odd empty pages at the end of chapters
%------------------------------------------------------------------------------%
\makeatletter
\renewcommand{\cleardoublepage}{
\clearpage\ifodd\c@page\else
\hbox{}
\vspace*{\fill}
\thispagestyle{empty}
\newpage
\fi}


%----------------------------------------------------------------------------------------
%  Page flow control
%----------------------------------------------------------------------------------------
%\widowpenalty=10000
%\clubpenalty=10000
%\vbadness=1200
%\hbadness=11000


%----------------------------------------------------------------------------------------
%   CHAPTER HEADINGS
%----------------------------------------------------------------------------------------
\newcommand{\thechapterimage}{}
\newcommand{\chapterimage}[1]{\renewcommand{\thechapterimage}{#1}}

% Numbered chapters with mini tableofcontents
\def\thechapter{\arabic{chapter}}
\def\@makechapterhead#1{
%\thispagestyle{plain}
{\centering \normalfont\sffamily
\ifnum \c@secnumdepth >\m@ne
\if@mainmatter
\startcontents
\begin{tikzpicture}[remember picture,overlay]
\node at (current page.north west)
{\begin{tikzpicture}[remember picture,overlay]
\node[anchor=north west,inner sep=0pt] at (0,0) {
\includegraphics[width=\paperwidth,height=0.5\paperwidth]{\thechapterimage}};
%------------------------------------------------------------------------------%
% Small contents box in the chapter heading
% Mini TOC background box
%\fill[color=dpawblue!10!white,opacity=.2] (1cm,0) rectangle (
%  3.5cm, % Mini TOC box width
%  -3.5cm % Mini TOC box height
%);
% Mini TOC text content
%\node[anchor=north west] at (1.1cm,.35cm) {
%  \parbox[t][8cm][t]{6.5cm}{
%    \huge\bfseries\flushleft
%    \printcontents{l}{1}{
%    \setcounter{tocdepth}{1}                   % Mini TOC level depth
%    }
% }
%};
%------------------------------------------------------------------------------%
% Chapter heading
\draw[anchor=west] (5cm,-9cm) node [
rounded corners=20pt,
fill=dpawblue!10!white,
text opacity=1,
draw=dpawblue,
draw opacity=1,
line width=1.5pt,
fill opacity=.2,
inner sep=12pt]{
    \huge\sffamily\bfseries\textcolor{black}{
      \thechapter. #1\strut\makebox[22cm]{}
    }
};
\end{tikzpicture}};
\end{tikzpicture}}
\par\vspace*{240\p@}                            % Push text below chapter image
\fi
\fi}

%------------------------------------------------------------------------------%
% Unnumbered chapters without mini tableofcontents
%------------------------------------------------------------------------------%
\def\@makeschapterhead#1{
%\thispagestyle{plain}
{\centering \normalfont\sffamily
\ifnum \c@secnumdepth >\m@ne
\if@mainmatter
\begin{tikzpicture}[remember picture,overlay]
\node at (current page.north west)
{\begin{tikzpicture}[remember picture,overlay]
\node[anchor=north west,inner sep=0pt] at (0,0) {
  \includegraphics[width=\paperwidth,height=0.5\paperwidth]{\thechapterimage}};
% Mini TOC background box
%\fill[color=dpawblue!10!white,opacity=.2] (1cm,0) rectangle (
%  3.5cm,                                       % Mini TOC box width
%  -3.5cm                                       % Mini TOC box height
%);
% Mini TOC text content
%\node[anchor=north west] at (1.1cm,.35cm) {
%  \parbox[t][8cm][t]{6.5cm}{
%    \huge\bfseries\flushleft
%    \printcontents{l}{1}{
%    \setcounter{tocdepth}{1} % Mini TOC level depth
%    }
%}
%};
\draw[anchor=west] (5cm,-9cm) node [rounded corners=20pt,
  fill=dpawblue!10!white,fill opacity=.6,inner sep=12pt,text opacity=1,
  draw=dpawblue,draw opacity=1,line width=1.5pt]{
  \huge\sffamily\bfseries\textcolor{black}{#1\strut\makebox[22cm]{}}};
\end{tikzpicture}};
\end{tikzpicture}}
\par\vspace*{240\p@}
\fi
\fi
}
\makeatother



\usepackage[automark,headsepline,footsepline,plainfootsepline]{scrlayer-scrpage}
\automark*[section]{}
\addtokomafont{pageheadfoot}{\normalfont\footnotesize\sffamily} % Don't italicise
\renewcommand{\chaptermark}[1]{\markleft{#1}{}}     % Chapter: suppress numbering
\renewcommand{\sectionmark}[1]{\markright{#1}{}}    % Section: suppress numbering

% Header (inner, center, outer)
\ihead{\href{http://sdis.dbca.wa.gov.au/documents/conceptplan/1304/}{Concept Plan SP 2014-021}}
%\chead{\href{http://sdis.dbca.wa.gov.au}{Science Directorate Information System}}
\ohead{\href{https://www.dbca.wa.gov.au/science/10-biodiversity-and-conservation-science}{
\includegraphics[height=6mm, keepaspectratio]{/usr/src/app/staticfiles/img/logo-dbca-bcs.jpg}}}
% Footer (inner, center, outer)
\ifoot{\textbf{Printed}~Fri, 6 Dec 2019 14:15:27 +0800} % inner/left footer
\cfoot{}
\ofoot[\bfseries\thepage]{\bfseries\thepage}        % Page number (also [plain])


\pagestyle{scrheadings}
\setkomafont{pageheadfoot}{\normalfont}

%-----------------------------------------------------------------------------%
\begin{document}
\raggedbottom

%-----------------------------------------------------------------------------%
% Title page
\subject{Concept Plan SP 2014-021
}
\title{Habitat use, distribution and abundance of coastal dolphin species in
the Pilbara
}
\subtitle{Marine Science
}
\author{}
\publishers{\small
    \subsection*{Project Core Team}
\begin{tabu} {X X}
\textbf{Supervising Scientist} & Holly Raudino
\\
\textbf{Data Custodian} & Holly Raudino
\\
\textbf{Site Custodian} & Holly Raudino
\\
\end{tabu}


    \subsection*{Project status as of Dec. 6, 2019, 2:15 p.m.}
\begin{tabu} {X X}
& Approved and active
\\
\end{tabu}

    
\subsection*{Document endorsements and approvals as of Dec. 6, 2019, 2:15 p.m.}
\begin{tabu} {X X}

%\rowcolor{grantedbg}
    \textbf{Project Team} & 
    \textcolor{granted}{ granted}\\

%\rowcolor{grantedbg}
    \textbf{Program Leader} & 
    \textcolor{granted}{ granted}\\

%\rowcolor{grantedbg}
    \textbf{Directorate} & 
    \textcolor{granted}{ granted}\\

\end{tabu}



}
\uppertitleback{}
\lowertitleback{}
\date{}

%-----------------------------------------------------------------------------%
% Front matter
\frontmatter
\maketitle
%-----------------------------------------------------------------------------%
% Main matter
\mainmatter


\section*{Habitat use, distribution and abundance of coastal dolphin species in
the Pilbara
}



\subsection*{Biodiversity and Conservation Science Program}

Marine Science




\subsection*{Departmental Service}

Service 6: Conserving Habitats, Species and Communities




\subsection*{Aims}

Although little is known about population size, distribution and
residency patterns, it is well accepted that Australian snubfin
(\emph{Orcaella} \emph{heinsohni}) and Australian humpback dolphin
(\emph{Sousa sahulensis}) inhabit Australia's tropical north-western
coastal waters (Allen\_, et al. \emph{2012). Indo-Pacific humpback
dolphins occur across Australia's entire northwest coast including
resident populations at Ningaloo Marine Park and most likely the Dampier
Archipelago as well as further north into the Kimberley. The snubfin
dolphin is endemic to northern Australia with identified resident
populations in the Kimberley, Northern Territory and Queensland (Brown},
et al. \emph{2014; Brown}, et al. \emph{2014). While this species has
been sighted occasionally in the Pilbara, their presence and use of this
area is yet to be determined, however the Pilbara is likely to represent
the southern extreme of their range (Allen}, et al.\_ 2012).

Limited surveys have been conducted targeting coastal dolphins in the
Pilbara; exceptions include a dedicated study of humpback dolphins in
Ningaloo Marine Park and Exmouth Gulf (Brown\_, et al. \emph{2012) and
opportunistic surveys and anecdotal sightings throughout the region
(Allen}, et al.\_ 2012). Aerial surveys that were targeting humpback
whales sighted dolphins but were unable to differentiate between species
due to the high altitude flown (1000 ft) (Jenner \& Jenner 2004; Jenner
\& Jenner 2010). Although the presence of several coastal dolphin
species is expected in nearshore waters (humpback, snubfin and
bottlenose dolphins) (Hanf 2014) the residency, degree of use and
habitat characteristics of these species are unknown in the Pilbara.

Human pressures and impacts on these species are increasing, in
particular in the Pilbara through activities associated with the rapid
expansion of resources sector, including oil and gas exploration and
production, coastal infrastructure development and shipping. This is
often a key factor that proponents are required to address to secure
environmental approvals at the State and Commonwealth levels. However,
as noted above, the knowledge base on these species across their range
is very poor. In addition, there are no agreed best practice protocols
or standards for survey design and data collection on these species that
allow for comparison to be made between studies and study sites. A
better understanding of these species and their use of tPilbara coastal
waters is needed to provide good temporal and regional context for
assessing and managing impacts and to reduce uncertainty in the
approvals process. As such, the draft\_ Strategic Research Priorities
for Marine Mammal Conservation and Management in Western Australia
2014\_ recognised both snubfin and humpback dolphins as high priority
species for fundamental research




\subsection*{Expected outcome}

This research will enable a better understanding of coastal dolphin
species at a regional and national scale including distribution,
abundance, habitat use, movement and connectivity. The main outcomes and
benefits will be:

\begin{itemize}
\tightlist
\item
  Distribution and abundance including high density areas and spatial
  and temporal patterns of coastal dolphins will be identified and
  mapped across the Pilbara to allow managers to assess conflicts with
  potential pressures;
\item
  Key habitat will be identified which can be used to assess potential
  overlap with pressures such as habitat loss from coastal development
  and displacement from industrial development leading to better
  informed decision making during Environmental Impact Assessment
  processes;
\item
  Populations will be defined for coastal dolphin species (humpback,
  bottlenose and snubfin, where applicable) which will allow managers to
  assess the relative conservation significance of different populations
  or species in relation to pressures or factors like restricted
  distributions;
\item
  Baseline data will inform ongoing regional monitoring and management
  and for comparison with other regions;
\item
  A state-wide database will be implemented modelled on the Northern
  Territory database 'DolFIN' to archive and manage survey and
  photo-Identification data which will improve information management,
  compatibility and information sharing between jurisdictions
\item
  Data on population abundance and distribution of humpback and snubfin
  dolphins in the Pilbara will allow a more comprehensive assessment of
  their conservation status at a State and National level.
\end{itemize}




\subsection*{Strategic context}

This project will meet the following objectives of the Corporate and
Science Division Strategic Plans

(Corporate Plan) - To conserve, protect and manage the state's native
fauna and flora based on best practice science

(Strategic Science Plan) -

\begin{itemize}
\tightlist
\item
  G2/2.1 Undertake the research needed to resolve the conservation
  status of threatened and priority species; 2.27 Conduct priority
  research on threatened species as a basis for understanding and
  managing threatening processes
\item
  G3/3.1 Provide the scientific basis for, and assist with, the
  development of cost-effective protocols for monitoring resource
  condition at various scales (landscape, ecosystem, protected area and
  species).
\item
  S3/3.4 Review and strengthen current partnerships.
\end{itemize}




\subsection*{Expected collaborations}

This project is funded by Wheatstone Offset funding and is in line with
the Science Plan prepared to meet this offset. Both the Science Plan and
this specific project have been discussed with and circulated to
regional Parks and Wildlife staff responsible for offset programs and
marine conservation and management (including Pilbara Regional Manager
(Alisdair MacDonald), Exmouth District Manager (Arvid Hogstrom), Pilbara
Marine Program Coordinator (Rachael Marshall), Ningaloo Marine Park
Coordinator (Peter Barnes) and Reserves Officer (Carolyn Williams).
These discussions have included opportunities to collaborate with the
Region and other offset programs in shared field work and in field
logistics. Further discussions will be undertaken with these regional
staff in Exmouth and Karratha in both the planning phase and when data
is available. In particular their advice and participation in liaising
with Indigenous rangers (where appropriate in the Pilbara), gathering
local knowledge and planning logistics will be valuable. Local staff
will participate in surveys where possible.

GIS support for the project has been discussed with Kathy Murray in the
GIS section.

Stuart Field (Offset coordinator) has liaised with Glen Young of Chevron
on our behalf about the engagement of the new sea ranger group that is
being led by Chevron in the Onslow area and active participation will be
sought, once the group is established. For the Dampier surveys, the
Murujuga rangers and traditional land owners will be consulted and
included as participants in the project. engaged.

In addition to the internal collaboration, further collaboration will be
sought with research institutions with some capacity and expertise in
marine mammal research. Murdoch University Cetacean Research Unit has
expertise in coastal dolphin survey methodology and in genetic biopsy
sampling of small cetaceans. We are liaising with them over their
involvement in survey work and in tissue sample collection for genetic
analysis as well as the potential for analsying aerial survey data for
dolphin species. We have also initiated discussions with UWA regarding
the analsysis of tissue samples for stable isotopes to inform trophic
level and diet studies of the dolphin species in the Pilbara.


\subsection*{Proposed period of the project}
July 1, 2014 -- June 30, 2018



\subsection*{Staff time allocation }



\begin{longtabu} to \linewidth { |  X | X | X | X | }
\hline
\rowcolor{infobg}
Role & Year 1 & Year 2 & Year 3\\
\hline
\endhead



Scientist & SCL2 1.0 FTE; SCL4 0.1 FTE & SCL2 1.0 FTE; SCL4 0.1 FTE & SCL2 1.0 FTE; SCL4 0.1 FTE\\



Technical & L3 0.5 FTE; L3 0.25 FTE & L3 0.5 FTE; L3 0.25 FTE & L3 0.5 FTE; L3 0.25 FTE\\



Volunteer &  &  & \\



Collaborator &  &  & \\


\hline
\end{longtabu}



\subsection*{Indicative operating budget }



\begin{longtabu} to \linewidth { |  X | X | X | X | }
\hline
\rowcolor{infobg}
Source & Year 1 & Year 2 & Year 3\\
\hline
\endhead



Consolidated Funds (DPaW) &  &  & \\



External Funding & $253,778 & $242,731 & $242,731\\


\hline
\end{longtabu}






%-----------------------------------------------------------------------------%
% Back matter
%\backmatter
\end{document}
%-----------------------------------------------------------------------------%

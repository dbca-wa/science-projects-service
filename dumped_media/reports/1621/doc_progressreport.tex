
\documentclass[version=last,
    paper=a4, % paper size
    10pt, % default font size
    usenames,
    dvipsnames,
    oneside, % ONLINE
    headings=openany, % open chapters on odd and even pages
    %toc=chapterentrywithdots, % Table of Contents style
    %BCOR=7mm, % PRINT Binding Correction
    %DIV=13, % typearea 161.54 mm x 228.46 mm, top margin 22.85 mm, inner margin 16.15 mm
    %DIV=14, % 165.00 233.36 21.21 15.00
    DIV=15 % 168.00 237.60 19.80 14.00
]{scrbook}
\usepackage{typearea}
\usepackage[automark,headsepline,footsepline]{scrlayer-scrpage} % Headers and footers

%%
%% Fonts, encoding, spacing, indentation
%%
\usepackage{txfonts}
\renewcommand{\familydefault}{\sfdefault} % Default to Sans Serif font
\usepackage[english]{babel}
\usepackage[T1]{fontenc}
\usepackage[utf8]{inputenc}

% Paragraph spacing
%\usepackage{parskip}    % Paragraph spacing
%\setlength{\parindent}{0em} % Don't indent paragraphs - ONLINE
%\setlength{\parskip}{1.3 ex plus 0.5ex minus 0.3ex} % 1-1.8 ex vertical space between paragraphs - ONLINE

% Spacing of headings
%\RedeclareSectionCommand[afterskip=3pt]{section} % less space after section
%\RedeclareSectionCommand[beforeskip=0cm]{subsection} % less space between HRule and project name
%\RedeclareSectionCommand[afterskip=0.1\baselineskip]{subsubsection} % less space after progressreport subheadings

% Table font size
\usepackage{etoolbox}
\AtBeginEnvironment{longtabu}{\footnotesize}{}{}

%%
%% Tables, columns, layout
%%
\usepackage{multicol}   % 2 col publications
\usepackage{pdflscape}  % Landscape pages
\usepackage{pdfpages}   % Include PDFs
\usepackage{hanging}    % hanging paragraphs for publications
%\usepackage{titletoc}   % Required for manipulating the table of contents
\setcounter{tocdepth}{2} % TOC list down to section
\usepackage{enumerate}  % Enumerations
\usepackage{enumitem}   % Enumerations
\usepackage{longtable}  % Multipage table
\usepackage{tabu}       %
\setlength{\tabulinesep}{1.5mm} % Consistent vertical spacing in tabu

%%
%% Graphics, images, colours
%%
\usepackage{graphicx} % embedded images
\usepackage{eso-pic} %
\usepackage{colortbl} % define custom named colours
\definecolor{RedFire}{RGB}{146,25,28}
\definecolor{ParksWildlife}{RGB}{0,85,144}
\definecolor{successbg}{RGB}{223,240,216}
\definecolor{errorbg}{RGB}{242,222,222}
\definecolor{warningbg}{RGB}{252,248,227}
\definecolor{infobg}{RGB}{217,237,247}
\definecolor{muted}{RGB}{153,153,153}
\definecolor{success}{RGB}{70,136,71}
\definecolor{error}{RGB}{185,74,72}
\definecolor{warning}{RGB}{192,152,83}
\definecolor{info}{RGB}{58,135,173}

\definecolor{required}{RGB}{192,152,83}
\definecolor{requiredbg}{RGB}{252,248,227}
\definecolor{denied}{RGB}{185,74,72}
\definecolor{deniedbg}{RGB}{242,222,222}
\definecolor{granted}{RGB}{70,136,71}
\definecolor{grantedbg}{RGB}{223,240,216}
\definecolor{not reqiured}{RGB}{153,153,153}
\definecolor{not requiredbg}{RGB}{255,255,255}

\usepackage{tikz} % Drawing
\usetikzlibrary{arrows,shapes,positioning,shadows,trees}

%%
%% Links, URLs
%%
\usepackage[
    linktoc=all,
    %colorlinks=false,  %PRINT
    colorlinks=true, % ONLINE
    linkcolor=RedFire, % ONLINE
    urlcolor=ParksWildlife, % ONLINE
    pdftitle=Progress Report SP 2012-025 (FY 2015-2016)
]{hyperref}

% Black magic to linebreak URLs
\usepackage{url}
\makeatletter
\g@addto@macro{\UrlBreaks}{\UrlOrds}
\makeatother

%%
%% Custom macros
%%
% Thick Horizontal rule
\newcommand{\HRule}{\vspace{8mm}\\\noindent\rule{\linewidth}{0.1pt}}

% Custom Tikz node for SDS diagram
\newcommand\mynode[6][]{
    \node[#1] (#2){
        \parbox{#3\relax}{
            \begin{center}
            \textbf{#4}\\
            #5\\
            \footnotesize{#6}
            \end{center}}};}



%-----------------------------------------------------------------------------%
% Headers and Footers
\automark{section}
\ohead{\href{http://sdis.dpaw.wa.gov.au/documents/progressreport/1621/}{Progress Report SP 2012-025
}}
\chead{\href{http://sdis.dpaw.wa.gov.au}{SDIS}} % center header ONLINE
\ihead{\href{http://sdis.dpaw.wa.gov.au}{
        \includegraphics[scale=0.4]{/mnt/projects/sdis/staticfiles/img/logo-dpaw.png}}}
\ifoot{\textbf{Printed}~Tue, 5 Jul 2016 11:49:02 +0800} % inner/left footer
\cfoot{} % center footer
\ofoot{\pagemark} % outer/right footer
\pagestyle{scrheadings}
\setkomafont{pageheadfoot}{\normalfont}

%-----------------------------------------------------------------------------%
\begin{document}
\raggedbottom

%-----------------------------------------------------------------------------%
% Title page
\subject{Progress Report SP 2012-025
}
\title{Barrow Island Threatened and Priority fauna species translocation
program
}
\subtitle{Animal Science
}
\author{}
\publishers{\small
    \subsection*{Project Core Team}
\begin{tabu} {X X}
\textbf{Supervising Scientist} & Keith Morris
\\
\textbf{Data Custodian} & Keith Morris
\\
\textbf{Site Custodian} & Keith Morris
\\
\end{tabu}


    \subsection*{Project status as of July 5, 2016, 11:49 a.m.}
\begin{tabu} {X X}
& Approved and active
\\
\end{tabu}

    
\subsection*{Document endorsements and approvals as of July 5, 2016, 11:49 a.m.}
\begin{tabu} {X X}

%\rowcolor{grantedbg}
    \textbf{Project Team} & 
    \textcolor{granted}{ granted}\\

%\rowcolor{grantedbg}
    \textbf{Program Leader} & 
    \textcolor{granted}{ granted}\\

%\rowcolor{grantedbg}
    \textbf{Directorate} & 
    \textcolor{granted}{ granted}\\

\end{tabu}



}
\uppertitleback{}
\lowertitleback{}
\date{}

%-----------------------------------------------------------------------------%
% Front matter
\frontmatter
\maketitle
%-----------------------------------------------------------------------------%
% Main matter
\mainmatter

\section*{Barrow Island Threatened and Priority fauna species translocation
program
}

K Morris, N Thomas, AH Burbidge, J Angus, S Garretson


\section*{Context}
Barrow Island nature reserve is one of Australia's most important
conservation reserves, particularly for mammal and marine turtle
conservation. It has also been the site of a producing oil field since
1964. In 2003 the WA Government approved the development of the Gorgon
gas field off the north west of Barrow Island, and associated LNG plant
on Barrow Island, subject to several environmental offset conditions.
One of these was the threatened and priority fauna translocation program
that provides for the translocation of selected Barrow Island fauna
species to other secure island and mainland sites. This will assist in
improving the conservation status of these species, and also allow the
reconstruction of the fauna in some areas. It was also an opportunity to
examine the factors affecting translocation success, and improve these
where necessary. Targetted species are the golden bandicoot, brushtail
possum, spectacled hare-wallaby, boodie, water rat, black and white
fairy-wren, and spinifexbird.



\section*{Aims}
\begin{itemize}
\itemsep1pt\parskip0pt\parsep0pt
\item
  Successfully translocate selected mammal and bird species from Barrow
  Island to other secure island and mainland sites.
\item
  Reconstruct the fauna in areas where these species have become locally
  extinct.
\item
  Ensure ongoing appropriate management at the translocation sites,
  particularly introduced predators.
\item
  Develop and refine protocols for fauna translocation and monitoring.
\end{itemize}



\section*{Progress}
\begin{itemize}
\itemsep1pt\parskip0pt\parsep0pt
\item
  Golden bandicoots from the captive population within the enclosure at
  Matuwa (Lorna Glen) were translocated outside the enclosure in
  September 2015 and appear to have established following an intensive
  feral cat and dog control program.
\item
  Golden bandicoots and boodies translocated from Barrow Island to the
  fenced enclosure at Matuwa in 2010 were monitored and continue to
  maintaining good body/reproductive condition.
\item
  Golden bandicoots introduced to Doole Island from Barrow Island in
  2011 were monitored and found to now occupy most of the island.
\item
  A camera trap array was used to estimate the occupancy of spectacled
  hare-wallabies on Hermite island and showed that the hare-wallabies
  have continued to expand and now occupy most of the island.
\item
  Black-and-white fairy-wrens and spinifexbirds translocated from Barrow
  Island to Hermite Island in 2010/11 were monitored. Both species have
  shown strong increases in population size and spinifexbirds have
  self-dispersed to six additional islands.
\item
  The Barrow Island fauna translocation strategy was finalised.
\item
  An annual report on progress was provided to Chevron.
\item
  A draft Montebello Islands Conservation Park Management Guideline was
  prepared by the Pilbara Region.
\end{itemize}



\section*{Management implications}
Arid zone rangelands fauna reconstruction and conservation techniques
developed by this project will have broad state and national
application. The outcomes of the project will contribute to the
management of Parks and Wildlife rangeland properties and provide
guidance for future fauna reconstruction, e.g. Dirk Hartog Island. It
will also contribute to an improvement in the conservation status of
several threatened fauna taxa. The draft management guidelines for the
Montebello Islands Conservation Park identify that the Pilbara Region
will assume responsibility for monitoring fauna on the Montebello
Islands after 2023, when current Gorgon funding ceases.



\section*{Future directions}
\begin{itemize}
\itemsep1pt\parskip0pt\parsep0pt
\item
  Continue to implement the Barrow Island fauna translocation strategy.
\item
  Develop and implement the fauna translocation plan based on outcomes
  from March 2015 workshop.
\item
  Research into the effectiveness of feral cat baiting at Cape Range to
  be undertaken in 2016/17 so that an integrated fox/cat baiting regime
  can be developed to support the locality becoming a possible
  reintroduction site.
\item
  Ongoing monitoring of the translocated mammals and birds at all the
  release sites.~
\item
  Continue to plan for an expanded fenced enclosure at Matuwa in the
  context of a joint~management agreement for the~Indigenous Protected
  Area with Martu~traditional owners.
\end{itemize}



%-----------------------------------------------------------------------------%
% Back matter
%\backmatter
\end{document}
%-----------------------------------------------------------------------------%


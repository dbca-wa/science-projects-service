
\documentclass[version=last,
    paper=a4, % paper size
    10pt, % default font size
    usenames,
    dvipsnames,
    oneside, % ONLINE
    headings=openany, % open chapters on odd and even pages
    %toc=chapterentrywithdots, % Table of Contents style
    %BCOR=7mm, % PRINT Binding Correction
    %DIV=13, % typearea 161.54 mm x 228.46 mm, top margin 22.85 mm, inner margin 16.15 mm
    %DIV=14, % 165.00 233.36 21.21 15.00
    DIV=15 % 168.00 237.60 19.80 14.00
]{scrbook}
\usepackage{typearea}
\usepackage[automark,headsepline,footsepline]{scrlayer-scrpage} % Headers and footers

%%
%% Fonts, encoding, spacing, indentation
%%
\usepackage{txfonts}
\renewcommand{\familydefault}{\sfdefault} % Default to Sans Serif font
\usepackage[english]{babel}
\usepackage[T1]{fontenc}
\usepackage[utf8]{inputenc}

% Paragraph spacing
%\usepackage{parskip}    % Paragraph spacing
%\setlength{\parindent}{0em} % Don't indent paragraphs - ONLINE
%\setlength{\parskip}{1.3 ex plus 0.5ex minus 0.3ex} % 1-1.8 ex vertical space between paragraphs - ONLINE

% Spacing of headings
%\RedeclareSectionCommand[afterskip=3pt]{section} % less space after section
%\RedeclareSectionCommand[beforeskip=0cm]{subsection} % less space between HRule and project name
%\RedeclareSectionCommand[afterskip=0.1\baselineskip]{subsubsection} % less space after progressreport subheadings

% Table font size
\usepackage{etoolbox}
\AtBeginEnvironment{longtabu}{\footnotesize}{}{}

%%
%% Tables, columns, layout
%%
\usepackage{multicol}   % 2 col publications
\usepackage{pdflscape}  % Landscape pages
\usepackage{pdfpages}   % Include PDFs
\usepackage{hanging}    % hanging paragraphs for publications
%\usepackage{titletoc}   % Required for manipulating the table of contents
\setcounter{tocdepth}{2} % TOC list down to section
\usepackage{enumerate}  % Enumerations
\usepackage{enumitem}   % Enumerations
\usepackage{longtable}  % Multipage table
\usepackage{tabu}       %
\setlength{\tabulinesep}{1.5mm} % Consistent vertical spacing in tabu

%%
%% Graphics, images, colours
%%
\usepackage{graphicx} % embedded images
\usepackage{eso-pic} %
\usepackage{colortbl} % define custom named colours
\definecolor{RedFire}{RGB}{146,25,28}
\definecolor{ParksWildlife}{RGB}{0,85,144}
\definecolor{successbg}{RGB}{223,240,216}
\definecolor{errorbg}{RGB}{242,222,222}
\definecolor{warningbg}{RGB}{252,248,227}
\definecolor{infobg}{RGB}{217,237,247}
\definecolor{muted}{RGB}{153,153,153}
\definecolor{success}{RGB}{70,136,71}
\definecolor{error}{RGB}{185,74,72}
\definecolor{warning}{RGB}{192,152,83}
\definecolor{info}{RGB}{58,135,173}

\definecolor{required}{RGB}{192,152,83}
\definecolor{requiredbg}{RGB}{252,248,227}
\definecolor{denied}{RGB}{185,74,72}
\definecolor{deniedbg}{RGB}{242,222,222}
\definecolor{granted}{RGB}{70,136,71}
\definecolor{grantedbg}{RGB}{223,240,216}
\definecolor{not reqiured}{RGB}{153,153,153}
\definecolor{not requiredbg}{RGB}{255,255,255}

\usepackage{tikz} % Drawing
\usetikzlibrary{arrows,shapes,positioning,shadows,trees}

%%
%% Links, URLs
%%
\usepackage[
    linktoc=all,
    %colorlinks=false,  %PRINT
    colorlinks=true, % ONLINE
    linkcolor=RedFire, % ONLINE
    urlcolor=ParksWildlife, % ONLINE
    pdftitle=Progress Report SP 2006-003 (FY 2015-2016)
]{hyperref}

% Black magic to linebreak URLs
\usepackage{url}
\makeatletter
\g@addto@macro{\UrlBreaks}{\UrlOrds}
\makeatother

%%
%% Custom macros
%%
% Thick Horizontal rule
\newcommand{\HRule}{\vspace{8mm}\\\noindent\rule{\linewidth}{0.1pt}}

% Custom Tikz node for SDS diagram
\newcommand\mynode[6][]{
    \node[#1] (#2){
        \parbox{#3\relax}{
            \begin{center}
            \textbf{#4}\\
            #5\\
            \footnotesize{#6}
            \end{center}}};}



%-----------------------------------------------------------------------------%
% Headers and Footers
\automark{section}
\ohead{\href{http://sdis.dpaw.wa.gov.au/documents/progressreport/1661/}{Progress Report SP 2006-003
}}
\chead{\href{http://sdis.dpaw.wa.gov.au}{SDIS}} % center header ONLINE
\ihead{\href{http://sdis.dpaw.wa.gov.au}{
        \includegraphics[scale=0.4]{/mnt/projects/sdis/staticfiles/img/logo-dpaw.png}}}
\ifoot{\textbf{Printed}~Mon, 11 Jul 2016 10:50:50 +0800} % inner/left footer
\cfoot{} % center footer
\ofoot{\pagemark} % outer/right footer
\pagestyle{scrheadings}
\setkomafont{pageheadfoot}{\normalfont}

%-----------------------------------------------------------------------------%
\begin{document}
\raggedbottom

%-----------------------------------------------------------------------------%
% Title page
\subject{Progress Report SP 2006-003
}
\title{FORESTCHECK: Integrated site-based monitoring of the effects of timber
harvesting and silviculture in the jarrah forest
}
\subtitle{Ecosystem Science
}
\author{}
\publishers{\small
    \subsection*{Project Core Team}
\begin{tabu} {X X}
\textbf{Supervising Scientist} & Richard Robinson
\\
\textbf{Data Custodian} & 
\\
\textbf{Site Custodian} & 
\\
\end{tabu}


    \subsection*{Project status as of July 11, 2016, 10:50 a.m.}
\begin{tabu} {X X}
& Approved and active
\\
\end{tabu}

    
\subsection*{Document endorsements and approvals as of July 11, 2016, 10:50 a.m.}
\begin{tabu} {X X}

%\rowcolor{grantedbg}
    \textbf{Project Team} & 
    \textcolor{granted}{ granted}\\

%\rowcolor{grantedbg}
    \textbf{Program Leader} & 
    \textcolor{granted}{ granted}\\

%\rowcolor{grantedbg}
    \textbf{Directorate} & 
    \textcolor{granted}{ granted}\\

\end{tabu}



}
\uppertitleback{}
\lowertitleback{}
\date{}

%-----------------------------------------------------------------------------%
% Front matter
\frontmatter
\maketitle
%-----------------------------------------------------------------------------%
% Main matter
\mainmatter

\section*{FORESTCHECK: Integrated site-based monitoring of the effects of timber
harvesting and silviculture in the jarrah forest
}

L Mccaw, J Farr, G Liddelow, V Tunsell, B Ward, A Wills


\section*{Context}
Forestcheck is a long-term monitoring program and results will be used
by forest managers to report against Montreal Process criteria and
indicators for ecologically sustainable forest management. Initiated as
a Ministerial Condition on the \emph{Forest Management Plan 1994-2003},
Forestcheck has continued to be incorporated in the \emph{Forest
Management Plan 2014}-\emph{2023} as a strategy for increasing knowledge
on the maintenance of biodiversity and management effectiveness in
Western Australian forests.



\section*{Aims}
Quantify the effects of current timber harvesting and silvicultural
practices in the jarrah forest (gap creation, shelterwood, post-harvest
burning) on forest structural attributes, soil and foliar nutrients,
soil compaction and the composition of the major biodiversity groups
including: macrofungi, cryptogams, vascular plants, invertebrates,
terrestrial vertebrates and birds.



\section*{Progress}
\begin{itemize}
\itemsep1pt\parskip0pt\parsep0pt
\item
  A progress report on monitoring undertaken in Blackwood and Perth
  Hills Districts during 2014 was finalised, following the completion of
  specimen identification and curation.
\item
  Work continued on the preparation of scientific papers reporting the
  second round of monitoring completed between 2007 and 2012. These
  papers will examine changes in species assemblages between the first
  and second rounds of monitoring in relation to climatic factors,
  changes in forest structure and improved sampling methods.
\item
  Work continued to review the capacity and process for delivering
  integrated forest monitoring into the future.
\item
  Seven monitoring grids in Wellington District burnt by the large Lower
  Hotham bushfire in February 2015 were re-sampled. Invertebrate pitfall
  and light trap sampling was undertaken in spring 2015 and autumn 2016
  on burnt sites. Monthly inspections to identify vascular plant species
  in flower have revealed a number of species not recorded previously at
  these grids, including a significant range extension and a potential
  new species. Crown recovery of overstorey trees on burnt grids was
  assessed in November 2015.
\item
  An analysis of factors affecting the consumption of coarse woody
  debris was undertaken using data gathered from 20 monitoring grids
  burnt by prescribed fire and bushfire.
\item
  Stand structure and fuel load were re-sampled in the Nalyerin block
  external reference grid which is in a fire exclusion reference area
  last burnt in 1987. Fuel load has been sampled previously at this grid
  in 2005 and 2011.
\end{itemize}



\section*{Management implications}
Forestcheck provides a systematic framework for evaluating the effects
of current silvicultural practices across a range of forest types and
provides a sound basis for adaptive management. Sixty five monitoring
grids have now been established, with 50 of these sampled at least
twice.

Findings from the project continue to inform a variety of forest
management policies and practices and have been incorporated in periodic
revision of silvicultural guidance documents. Monitoring data have been
used to verify predictive models for forest growth and species
occurrence.

The network of Forestcheck grids also provides a framework for
monitoring responses to random disturbance events such as bushfires and
extreme droughts, and for examining the impacts of a changing climate
over the longer term.



\section*{Future directions}
\begin{itemize}
\itemsep1pt\parskip0pt\parsep0pt
\item
  Finalise analysis of data from the 10-year monitoring period
  (2002-2012) and publication of 10-year results.
\item
  Review monitoring protocols and incorporate new techniques where these
  will improve efficiency and quality of data collected.
\item
  In consultation with Forest and Ecosystem Management Division and the
  Forest Products Commission determine a future program of monitoring
  for 2016 to 2018.
\item
  Prepare manuscripts reporting on consumption of coarse woody debris
  and the initial response of vascular plants and invertebrates
  following the 2015 Lower Hotham bushfire.
\end{itemize}



%-----------------------------------------------------------------------------%
% Back matter
%\backmatter
\end{document}
%-----------------------------------------------------------------------------%


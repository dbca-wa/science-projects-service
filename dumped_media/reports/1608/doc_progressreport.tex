
\documentclass[version=last,
    paper=a4, % paper size
    10pt, % default font size
    usenames,
    dvipsnames,
    oneside, % ONLINE
    headings=openany, % open chapters on odd and even pages
    %toc=chapterentrywithdots, % Table of Contents style
    %BCOR=7mm, % PRINT Binding Correction
    %DIV=13, % typearea 161.54 mm x 228.46 mm, top margin 22.85 mm, inner margin 16.15 mm
    %DIV=14, % 165.00 233.36 21.21 15.00
    DIV=15 % 168.00 237.60 19.80 14.00
]{scrbook}
\usepackage{typearea}
\usepackage[automark,headsepline,footsepline]{scrlayer-scrpage} % Headers and footers

%%
%% Fonts, encoding, spacing, indentation
%%
\usepackage{txfonts}
\renewcommand{\familydefault}{\sfdefault} % Default to Sans Serif font
\usepackage[english]{babel}
\usepackage[T1]{fontenc}
\usepackage[utf8]{inputenc}

% Paragraph spacing
%\usepackage{parskip}    % Paragraph spacing
%\setlength{\parindent}{0em} % Don't indent paragraphs - ONLINE
%\setlength{\parskip}{1.3 ex plus 0.5ex minus 0.3ex} % 1-1.8 ex vertical space between paragraphs - ONLINE

% Spacing of headings
%\RedeclareSectionCommand[afterskip=3pt]{section} % less space after section
%\RedeclareSectionCommand[beforeskip=0cm]{subsection} % less space between HRule and project name
%\RedeclareSectionCommand[afterskip=0.1\baselineskip]{subsubsection} % less space after progressreport subheadings

% Table font size
\usepackage{etoolbox}
\AtBeginEnvironment{longtabu}{\footnotesize}{}{}

%%
%% Tables, columns, layout
%%
\usepackage{multicol}   % 2 col publications
\usepackage{pdflscape}  % Landscape pages
\usepackage{pdfpages}   % Include PDFs
\usepackage{hanging}    % hanging paragraphs for publications
%\usepackage{titletoc}   % Required for manipulating the table of contents
\setcounter{tocdepth}{2} % TOC list down to section
\usepackage{enumerate}  % Enumerations
\usepackage{enumitem}   % Enumerations
\usepackage{longtable}  % Multipage table
\usepackage{tabu}       %
\setlength{\tabulinesep}{1.5mm} % Consistent vertical spacing in tabu

%%
%% Graphics, images, colours
%%
\usepackage{graphicx} % embedded images
\usepackage{eso-pic} %
\usepackage{colortbl} % define custom named colours
\definecolor{RedFire}{RGB}{146,25,28}
\definecolor{ParksWildlife}{RGB}{0,85,144}
\definecolor{successbg}{RGB}{223,240,216}
\definecolor{errorbg}{RGB}{242,222,222}
\definecolor{warningbg}{RGB}{252,248,227}
\definecolor{infobg}{RGB}{217,237,247}
\definecolor{muted}{RGB}{153,153,153}
\definecolor{success}{RGB}{70,136,71}
\definecolor{error}{RGB}{185,74,72}
\definecolor{warning}{RGB}{192,152,83}
\definecolor{info}{RGB}{58,135,173}

\definecolor{required}{RGB}{192,152,83}
\definecolor{requiredbg}{RGB}{252,248,227}
\definecolor{denied}{RGB}{185,74,72}
\definecolor{deniedbg}{RGB}{242,222,222}
\definecolor{granted}{RGB}{70,136,71}
\definecolor{grantedbg}{RGB}{223,240,216}
\definecolor{not reqiured}{RGB}{153,153,153}
\definecolor{not requiredbg}{RGB}{255,255,255}

\usepackage{tikz} % Drawing
\usetikzlibrary{arrows,shapes,positioning,shadows,trees}

%%
%% Links, URLs
%%
\usepackage[
    linktoc=all,
    %colorlinks=false,  %PRINT
    colorlinks=true, % ONLINE
    linkcolor=RedFire, % ONLINE
    urlcolor=ParksWildlife, % ONLINE
    pdftitle=Progress Report SP 2013-004 (FY 2015-2016)
]{hyperref}

% Black magic to linebreak URLs
\usepackage{url}
\makeatletter
\g@addto@macro{\UrlBreaks}{\UrlOrds}
\makeatother

%%
%% Custom macros
%%
% Thick Horizontal rule
\newcommand{\HRule}{\vspace{8mm}\\\noindent\rule{\linewidth}{0.1pt}}

% Custom Tikz node for SDS diagram
\newcommand\mynode[6][]{
    \node[#1] (#2){
        \parbox{#3\relax}{
            \begin{center}
            \textbf{#4}\\
            #5\\
            \footnotesize{#6}
            \end{center}}};}



%-----------------------------------------------------------------------------%
% Headers and Footers
\automark{section}
\ohead{\href{http://sdis.dpaw.wa.gov.au/documents/progressreport/1608/}{Progress Report SP 2013-004
}}
\chead{\href{http://sdis.dpaw.wa.gov.au}{SDIS}} % center header ONLINE
\ihead{\href{http://sdis.dpaw.wa.gov.au}{
        \includegraphics[scale=0.4]{/mnt/projects/sdis/staticfiles/img/logo-dpaw.png}}}
\ifoot{\textbf{Printed}~Mon, 11 Jul 2016 09:36:50 +0800} % inner/left footer
\cfoot{} % center footer
\ofoot{\pagemark} % outer/right footer
\pagestyle{scrheadings}
\setkomafont{pageheadfoot}{\normalfont}

%-----------------------------------------------------------------------------%
\begin{document}
\raggedbottom

%-----------------------------------------------------------------------------%
% Title page
\subject{Progress Report SP 2013-004
}
\title{Restoring natural riparian vegetation systems along the Warren and
Donnelly Rivers
}
\subtitle{Ecosystem Science
}
\author{}
\publishers{\small
    \subsection*{Project Core Team}
\begin{tabu} {X X}
\textbf{Supervising Scientist} & Margaret Byrne
\\
\textbf{Data Custodian} & 
\\
\textbf{Site Custodian} & 
\\
\end{tabu}


    \subsection*{Project status as of July 11, 2016, 9:36 a.m.}
\begin{tabu} {X X}
& Approved and active
\\
\end{tabu}

    
\subsection*{Document endorsements and approvals as of July 11, 2016, 9:36 a.m.}
\begin{tabu} {X X}

%\rowcolor{grantedbg}
    \textbf{Project Team} & 
    \textcolor{granted}{ granted}\\

%\rowcolor{grantedbg}
    \textbf{Program Leader} & 
    \textcolor{granted}{ granted}\\

%\rowcolor{grantedbg}
    \textbf{Directorate} & 
    \textcolor{granted}{ granted}\\

\end{tabu}



}
\uppertitleback{}
\lowertitleback{}
\date{}

%-----------------------------------------------------------------------------%
% Front matter
\frontmatter
\maketitle
%-----------------------------------------------------------------------------%
% Main matter
\mainmatter

\section*{Restoring natural riparian vegetation systems along the Warren and
Donnelly Rivers
}

M Byrne, T Hopley, T Macfarlane, C Ramalho, C Yates


\section*{Context}
Current practices of seed sourcing for revegetation projects focus on
local seed, based on a premise of maximising adaptation to local
conditions, but this may not be most appropriate under changing climatic
conditions. Identification of patterns of adaptive variation will enable
more informed approaches to species selection and seed sourcing to
maximise establishment and persistence of plants in revegetation
programs.

This project will provide a climate change framework for revegetation of
blackberry-decline sites on the Warren and Donnelly Rivers by
determining the scale of adaptation to climate along the river system
and determining the best seed source strategies to maximise resilience
to future changes in climate in the revegetated populations.



\section*{Aims}
\begin{itemize}
\itemsep1pt\parskip0pt\parsep0pt
\item
  Develop a climate change framework for revegetation of riparian
  vegetation along the Warren and Donnelly Rivers.
\item
  Determine seed sourcing strategies that account for climate adaptation
  to enable resilient restoration of riparian vegetation along the
  Warren River and Donnelly Rivers.
\item
  Test adaptation to climate through experimental plantings under
  operational conditions of establishment.
\end{itemize}



\section*{Progress}
\begin{itemize}
\itemsep1pt\parskip0pt\parsep0pt
\item
  Analysis of genomic data has found the three species (\emph{Astartea
  leptophylla,}~\emph{Callistachys lanceolata,~Taxandria linearifolia})
  to have different patterns of genetic structure across the climate
  gradient. \emph{Astartea leptophylla} is restricted to the main river
  and was found to have the lowest levels of differentiation between
  populations. \emph{Callistachys lanceolata}, a widespread species with
  short distance dispersal had high levels of genetic structure between
  populations. This data will be used to determine genetic adaptation
  between populations and climate zones of the three species.
\item
  Initial outlier analysis has found the \emph{Astartea}~to have low
  numbers of outliers (97) and the majority are under directional
  selection.
\item
  \emph{Callistachys} was found to have a similar number of outliers
  (96) but two-thirds of these are under balancing selection.
\item
  \emph{Taxandria} with moderate levels of genetic diversity between
  populations was found to have high numbers of outliers (264) with
  three-quarters of these under directional selection.
\item
  Experimental plantings were impacted by insect damage and have been
  replanted this year with insect exclusion measures in place.
\item
  A manuscript detailing a spatially explicit approach to support
  decision making for seed provenance selection in ecological
  restoration in a climate change context has been revised and submitted
  for publication.
\end{itemize}



\section*{Management implications}
Changing climates require a re-evaluation of appropriate seed sourcing
strategies for revegetation and restoration of ecological function in
degraded sites. Use of local seed will not provide adequate resilience
to maintain ecological function under changing climates, and
understanding of climate adaptation will provide a scientific basis to
undertake best-practice restoration and facilitate establishment of
biodiverse plantings that maximise ecological function for enhanced
persistence and resilience. Development of a strategic revegetation
program for the riparian areas of the Warren and Donnelly catchments
will provide an integrated approach to habitat restoration that promotes
improved plant community function and improves the knowledge and
capacity of restoration practitioners and land managers.



\section*{Future directions}
\begin{itemize}
\itemsep1pt\parskip0pt\parsep0pt
\item
  Complete analysis of association between allele frequencies of outlier
  loci and ecological variables to identify potential adaptive loci.
\item
  Identify genetic markers subject to selection and associated with
  population-specific climate variables to reveal specific significant
  climatic associations and how these are related to genetic structure
  and gene flow.
\item
  Sample from experimental plantings of \emph{Callistachys} for genomic
  analysis to assess establishment and performance and to determine any
  effects of adaptation to drier environments on current performance of
  germplasm in revegetation projects.
\item
  Identify the scale of climate associations and the implications of
  these for seed sourcing with the aim of maximising resilience in
  restoration projects.
\item
  Prepare manuscripts on genetic structure and outlier identification
  for all three species .
\end{itemize}



%-----------------------------------------------------------------------------%
% Back matter
%\backmatter
\end{document}
%-----------------------------------------------------------------------------%


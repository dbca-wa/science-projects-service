
\documentclass[version=last,
    paper=a4, % paper size
    10pt, % default font size
    usenames,
    dvipsnames,
    oneside, % ONLINE
    headings=openany, % open chapters on odd and even pages
    %toc=chapterentrywithdots, % Table of Contents style
    %BCOR=7mm, % PRINT Binding Correction
    %DIV=13, % typearea 161.54 mm x 228.46 mm, top margin 22.85 mm, inner margin 16.15 mm
    %DIV=14, % 165.00 233.36 21.21 15.00
    DIV=15 % 168.00 237.60 19.80 14.00
]{scrbook}
\usepackage{typearea}
\usepackage[automark,headsepline,footsepline]{scrlayer-scrpage} % Headers and footers

%%
%% Fonts, encoding, spacing, indentation
%%
\usepackage{txfonts}
\renewcommand{\familydefault}{\sfdefault} % Default to Sans Serif font
\usepackage[english]{babel}
\usepackage[T1]{fontenc}
\usepackage[utf8]{inputenc}

% Paragraph spacing
%\usepackage{parskip}    % Paragraph spacing
%\setlength{\parindent}{0em} % Don't indent paragraphs - ONLINE
%\setlength{\parskip}{1.3 ex plus 0.5ex minus 0.3ex} % 1-1.8 ex vertical space between paragraphs - ONLINE

% Spacing of headings
%\RedeclareSectionCommand[afterskip=3pt]{section} % less space after section
%\RedeclareSectionCommand[beforeskip=0cm]{subsection} % less space between HRule and project name
%\RedeclareSectionCommand[afterskip=0.1\baselineskip]{subsubsection} % less space after progressreport subheadings

% Table font size
\usepackage{etoolbox}
\AtBeginEnvironment{longtabu}{\footnotesize}{}{}

%%
%% Tables, columns, layout
%%
\usepackage{multicol}   % 2 col publications
\usepackage{pdflscape}  % Landscape pages
\usepackage{pdfpages}   % Include PDFs
\usepackage{hanging}    % hanging paragraphs for publications
%\usepackage{titletoc}   % Required for manipulating the table of contents
\setcounter{tocdepth}{2} % TOC list down to section
\usepackage{enumerate}  % Enumerations
\usepackage{enumitem}   % Enumerations
\usepackage{longtable}  % Multipage table
\usepackage{tabu}       %
\setlength{\tabulinesep}{1.5mm} % Consistent vertical spacing in tabu

%%
%% Graphics, images, colours
%%
\usepackage{graphicx} % embedded images
\usepackage{eso-pic} %
\usepackage{colortbl} % define custom named colours
\definecolor{RedFire}{RGB}{146,25,28}
\definecolor{ParksWildlife}{RGB}{0,85,144}
\definecolor{successbg}{RGB}{223,240,216}
\definecolor{errorbg}{RGB}{242,222,222}
\definecolor{warningbg}{RGB}{252,248,227}
\definecolor{infobg}{RGB}{217,237,247}
\definecolor{muted}{RGB}{153,153,153}
\definecolor{success}{RGB}{70,136,71}
\definecolor{error}{RGB}{185,74,72}
\definecolor{warning}{RGB}{192,152,83}
\definecolor{info}{RGB}{58,135,173}

\definecolor{required}{RGB}{192,152,83}
\definecolor{requiredbg}{RGB}{252,248,227}
\definecolor{denied}{RGB}{185,74,72}
\definecolor{deniedbg}{RGB}{242,222,222}
\definecolor{granted}{RGB}{70,136,71}
\definecolor{grantedbg}{RGB}{223,240,216}
\definecolor{not reqiured}{RGB}{153,153,153}
\definecolor{not requiredbg}{RGB}{255,255,255}

\usepackage{tikz} % Drawing
\usetikzlibrary{arrows,shapes,positioning,shadows,trees}

%%
%% Links, URLs
%%
\usepackage[
    linktoc=all,
    %colorlinks=false,  %PRINT
    colorlinks=true, % ONLINE
    linkcolor=RedFire, % ONLINE
    urlcolor=ParksWildlife, % ONLINE
    pdftitle=Progress Report SP 2015-002 (FY 2015-2016)
]{hyperref}

% Black magic to linebreak URLs
\usepackage{url}
\makeatletter
\g@addto@macro{\UrlBreaks}{\UrlOrds}
\makeatother

%%
%% Custom macros
%%
% Thick Horizontal rule
\newcommand{\HRule}{\vspace{8mm}\\\noindent\rule{\linewidth}{0.1pt}}

% Custom Tikz node for SDS diagram
\newcommand\mynode[6][]{
    \node[#1] (#2){
        \parbox{#3\relax}{
            \begin{center}
            \textbf{#4}\\
            #5\\
            \footnotesize{#6}
            \end{center}}};}



%-----------------------------------------------------------------------------%
% Headers and Footers
\automark{section}
\ohead{\href{http://sdis.dpaw.wa.gov.au/documents/progressreport/1593/}{Progress Report SP 2015-002
}}
\chead{\href{http://sdis.dpaw.wa.gov.au}{SDIS}} % center header ONLINE
\ihead{\href{http://sdis.dpaw.wa.gov.au}{
        \includegraphics[scale=0.4]{/mnt/projects/sdis/staticfiles/img/logo-dpaw.png}}}
\ifoot{\textbf{Printed}~Tue, 5 Jul 2016 15:11:10 +0800} % inner/left footer
\cfoot{} % center footer
\ofoot{\pagemark} % outer/right footer
\pagestyle{scrheadings}
\setkomafont{pageheadfoot}{\normalfont}

%-----------------------------------------------------------------------------%
\begin{document}
\raggedbottom

%-----------------------------------------------------------------------------%
% Title page
\subject{Progress Report SP 2015-002
}
\title{South West Wetlands Monitoring Program (SWWMP)
}
\subtitle{Wetlands Conservation
}
\author{}
\publishers{\small
    \subsection*{Project Core Team}
\begin{tabu} {X X}
\textbf{Supervising Scientist} & Jim Lane
\\
\textbf{Data Custodian} & Jim Lane
\\
\textbf{Site Custodian} & Jim Lane
\\
\end{tabu}


    \subsection*{Project status as of July 5, 2016, 3:11 p.m.}
\begin{tabu} {X X}
& Approved and active
\\
\end{tabu}

    
\subsection*{Document endorsements and approvals as of July 5, 2016, 3:11 p.m.}
\begin{tabu} {X X}

%\rowcolor{grantedbg}
    \textbf{Project Team} & 
    \textcolor{granted}{ granted}\\

%\rowcolor{grantedbg}
    \textbf{Program Leader} & 
    \textcolor{granted}{ granted}\\

%\rowcolor{grantedbg}
    \textbf{Directorate} & 
    \textcolor{granted}{ granted}\\

\end{tabu}



}
\uppertitleback{}
\lowertitleback{}
\date{}

%-----------------------------------------------------------------------------%
% Front matter
\frontmatter
\maketitle
%-----------------------------------------------------------------------------%
% Main matter
\mainmatter

\section*{South West Wetlands Monitoring Program (SWWMP)
}

J Lane, M Lyons, A Pinder, A Clarke, D Cale, Y Winchcombe


\section*{Context}
Substantial decline in wetland condition has been observed across the
south-west of Western Australia over the past 100 years, particularly in
the Wheatbelt, almost certainly with ongoing loss of biodiversity. The
most pronounced changes to wetlands have been associated with
salinisation and altered hydrology following clearing of native
vegetation in catchments. Broad-scale clearing has largely ceased but
hydrological and fragmentation processes will continue to be expressed
for many decades. Changes in rainfall patterns are also resulting in
significant changes to wetland hydrology, water chemistry and habitats.

While it is known that altered hydrological regimes and salinisation are
major threats to wetland biodiversity, the relationships between
physical expression and loss of biodiversity are poorly documented and
poorly understood. Monitoring of wetland depth and water chemistry in
the south-west began in 1977 to inform duck hunting management. After
continuing at a reduced level following the ban on recreational duck
hunting in 1992, the program was reinvigorated under the State Salinity
Strategy in 1996, supplemented by intensive monitoring of fauna, flora,
water chemistry and shallow groundwater at a subset of 25 wetlands. This
project is delivering vital information on the long-term trends and
variability in key determinants of wetland character and condition and,
to a lesser extent, biological attributes.



\section*{Aims}
\begin{itemize}
\itemsep1pt\parskip0pt\parsep0pt
\item
  To contribute to improved decision making in wetland biodiversity
  conservation by 1) providing analyses of long and short-term changes
  in surface water quantity and quality, shallow groundwater levels and
  biodiversity at representative south-west wetlands in relation to
  threatening processes (particularly dryland salinity and reduced
  rainfall) and 2) assessing the effectiveness of catchment and wetland
  management.
\end{itemize}



\section*{Progress}
\begin{itemize}
\itemsep1pt\parskip0pt\parsep0pt
\item
  Depth and water quality monitoring was undertaken at 105 wetlands,
  with data added to the South West Wetlands Monitoring Program (SWWMP)
  database and supplied to managers and researchers.
\item
  Continuous water level recorders and tipping-bucket rain gauges were
  maintained on nine southern wetlands with high conservation values,
  especially for the Australasian bittern.
\item
  The `Thirty Year (1981-2010) Trends' report (Lane, Winchcombe \&
  Clarke, 2015) was completed, presenting trends in water levels and
  rainfalls of 113 south-western Australian wetlands monitored under
  SWWMP.
\item
  A report on `Water levels and rainfalls of 14 south-western Australian
  wetlands: continuous recordings from 2009-2015' was prepared to final
  draft.
\item
  Lake Jasper was salinity-profiled. While still fresh, Lake Jasper has
  become more saline over the past decade. Long-term monitoring is at a
  single point. Profiling was undertaken to provide a broader basis for
  future comparison and understanding.
\item
  Work continued on a report analysing relationships between wetland
  character and aquatic fauna in 25 representative wetland sites
  monitored between 1996 and 2012.
\item
  Curating and long-term archiving of the 1996 to 2012 aquatic
  invertebrate specimen collection was commenced.
\item
  Data analysis for dominant overstorey trees commenced on Wheatbelt
  wetlands with the production of size class histograms to examine
  population structure and seedling recruitment over time.
\item
  Groundwater data collection was undertaken in autumn 2016.
\end{itemize}



\section*{Management implications}
\begin{itemize}
\itemsep1pt\parskip0pt\parsep0pt
\item
  Rainfalls and water levels in south-western Australia are declining
  and these trends have~adverse consequences and long term implications
  for many species of wetland flora and fauna (such as the threatened
  Australasian bittern) and for the recreational value of wetlands.
  Active management is required to ameliorate impacts and conserve
  threatened species.
\item
  The SWWMP project provides early warnings of changes and helps inform
  where to focus management. Importantly, the long-term nature of this
  project provides a unique context against which to assess the
  significance of contemporary observations during decision-making
  processes and enables prediction of the effects of future change.
\item
  SWWMP data provides vital information for planning and assessing
  management interventions, such as the hydrological interventions to
  reduce water levels in the Warden (Esperance) Ramsar wetlands,
  increase water levels at Jandabup and Thomsons Lakes, manage salinity
  at Lake Toolibin, and manage depths for water skiing at Lake
  Towerinning.
\item
  Analyses of the flora and fauna data from 25 representative wetlands
  will allow managers to predict future impacts of altered hydrology and
  assess management responses in similar wetlands and understand the
  trajectory of Wheatbelt wetland biodiversity more generally.
\end{itemize}



\section*{Future directions}
\begin{itemize}
\itemsep1pt\parskip0pt\parsep0pt
\item
  Complete write-up of the 15 years of fauna and flora monitoring at the
  25 intensively monitored wetlands and archive data.
\item
  Re-design and implement a focused Wheatbelt wetland ecological
  monitoring program to track changes in priority wetlands in relation
  to threats and management.
\item
  Continue to produce annual reports presenting the latest SWWMP data,
  trends and~issues of concern and interest.
\item
  By interpolation and modelling, fill gaps in the SWWMP water level
  time series to substantially increase the number of wetlands that can
  be included in decadal and multi-decadal trend analyses.
\item
  Prepare a 1981-2015 update of the thirty year (1981-2010) trends
  report on water levels and rainfall of the more than 100 south-west
  wetlands of SWWMP.
\item
  Complete the 2009-2015 continuous water level and rainfall recordings
  report.
\item
  Use results of long-term periodic water level, salinity and ph
  monitoring, continuous on-site rainfall and water level monitoring,
  and other datasets, to predict likely futures of wetlands important
  for Australasian bittern and other fauna and flora in different
  climate scenarios.
\item
  Develop~a format to enable upload of vegetation monitoring data,
  including trends, to NatureMap.
\end{itemize}



%-----------------------------------------------------------------------------%
% Back matter
%\backmatter
\end{document}
%-----------------------------------------------------------------------------%



\documentclass[version=last,
    paper=a4, % paper size
    10pt, % default font size
    usenames,
    dvipsnames,
    oneside, % ONLINE
    headings=openany, % open chapters on odd and even pages
    %toc=chapterentrywithdots, % Table of Contents style
    %BCOR=7mm, % PRINT Binding Correction
    %DIV=13, % typearea 161.54 mm x 228.46 mm, top margin 22.85 mm, inner margin 16.15 mm
    %DIV=14, % 165.00 233.36 21.21 15.00
    DIV=15 % 168.00 237.60 19.80 14.00
]{scrbook}
\usepackage{typearea}
\usepackage[automark,headsepline,footsepline]{scrlayer-scrpage} % Headers and footers

%%
%% Fonts, encoding, spacing, indentation
%%
\usepackage{txfonts}
\renewcommand{\familydefault}{\sfdefault} % Default to Sans Serif font
\usepackage[english]{babel}
\usepackage[T1]{fontenc}
\usepackage[utf8]{inputenc}

% Paragraph spacing
%\usepackage{parskip}    % Paragraph spacing
%\setlength{\parindent}{0em} % Don't indent paragraphs - ONLINE
%\setlength{\parskip}{1.3 ex plus 0.5ex minus 0.3ex} % 1-1.8 ex vertical space between paragraphs - ONLINE

% Spacing of headings
%\RedeclareSectionCommand[afterskip=3pt]{section} % less space after section
%\RedeclareSectionCommand[beforeskip=0cm]{subsection} % less space between HRule and project name
%\RedeclareSectionCommand[afterskip=0.1\baselineskip]{subsubsection} % less space after progressreport subheadings

% Table font size
\usepackage{etoolbox}
\AtBeginEnvironment{longtabu}{\footnotesize}{}{}

%%
%% Tables, columns, layout
%%
\usepackage{multicol}   % 2 col publications
\usepackage{pdflscape}  % Landscape pages
\usepackage{pdfpages}   % Include PDFs
\usepackage{hanging}    % hanging paragraphs for publications
%\usepackage{titletoc}   % Required for manipulating the table of contents
\setcounter{tocdepth}{2} % TOC list down to section
\usepackage{enumerate}  % Enumerations
\usepackage{enumitem}   % Enumerations
\usepackage{longtable}  % Multipage table
\usepackage{tabu}       %
\setlength{\tabulinesep}{1.5mm} % Consistent vertical spacing in tabu

%%
%% Graphics, images, colours
%%
\usepackage{graphicx} % embedded images
\usepackage{eso-pic} %
\usepackage{colortbl} % define custom named colours
\definecolor{RedFire}{RGB}{146,25,28}
\definecolor{ParksWildlife}{RGB}{0,85,144}
\definecolor{successbg}{RGB}{223,240,216}
\definecolor{errorbg}{RGB}{242,222,222}
\definecolor{warningbg}{RGB}{252,248,227}
\definecolor{infobg}{RGB}{217,237,247}
\definecolor{muted}{RGB}{153,153,153}
\definecolor{success}{RGB}{70,136,71}
\definecolor{error}{RGB}{185,74,72}
\definecolor{warning}{RGB}{192,152,83}
\definecolor{info}{RGB}{58,135,173}

\definecolor{required}{RGB}{192,152,83}
\definecolor{requiredbg}{RGB}{252,248,227}
\definecolor{denied}{RGB}{185,74,72}
\definecolor{deniedbg}{RGB}{242,222,222}
\definecolor{granted}{RGB}{70,136,71}
\definecolor{grantedbg}{RGB}{223,240,216}
\definecolor{not reqiured}{RGB}{153,153,153}
\definecolor{not requiredbg}{RGB}{255,255,255}

\usepackage{tikz} % Drawing
\usetikzlibrary{arrows,shapes,positioning,shadows,trees}

%%
%% Links, URLs
%%
\usepackage[
    linktoc=all,
    %colorlinks=false,  %PRINT
    colorlinks=true, % ONLINE
    linkcolor=RedFire, % ONLINE
    urlcolor=ParksWildlife, % ONLINE
    pdftitle=Progress Report CF 2011-105 (FY 2014-2015)
]{hyperref}

% Black magic to linebreak URLs
\usepackage{url}
\makeatletter
\g@addto@macro{\UrlBreaks}{\UrlOrds}
\makeatother

%%
%% Custom macros
%%
% Thick Horizontal rule
\newcommand{\HRule}{\vspace{8mm}\\\noindent\rule{\linewidth}{0.1pt}}

% Custom Tikz node for SDS diagram
\newcommand\mynode[6][]{
    \node[#1] (#2){
        \parbox{#3\relax}{
            \begin{center}
            \textbf{#4}\\
            #5\\
            \footnotesize{#6}
            \end{center}}};}



\usepackage[automark,headsepline,footsepline,plainfootsepline]{scrlayer-scrpage}
\automark*[section]{}
\addtokomafont{pageheadfoot}{\normalfont\footnotesize\sffamily} % Don't italicise
\renewcommand{\chaptermark}[1]{\markleft{#1}{}}     % Chapter: suppress numbering
\renewcommand{\sectionmark}[1]{\markright{#1}{}}    % Section: suppress numbering

% Header (inner, center, outer)
\ihead{\href{http://sdis.dpaw.wa.gov.au/documents/progressreport/1426/}{Progress Report CF 2011-105 (FY 2014-2015)}}
%\chead{\href{http://sdis.dpaw.wa.gov.au}{Science Directorate Information System}}
\ohead{\href{https://www.dpaw.wa.gov.au/about-us/science-and-research}{\includegraphics[height=6mm, keepaspectratio]{/mnt/projects/sdis/staticfiles/img/logo-dpaw.png}}}

% Footer (inner, center, outer)
\ifoot{\textbf{Printed}~Tue, 12 Sep 2017 10:15:20 +0800} % inner/left footer
\cfoot{}
\ofoot[\bfseries\thepage]{\bfseries\thepage}        % Page number (also [plain])


\pagestyle{scrheadings}
\setkomafont{pageheadfoot}{\normalfont}

%-----------------------------------------------------------------------------%
\begin{document}
\raggedbottom

%-----------------------------------------------------------------------------%
% Title page
\subject{Progress Report CF 2011-105
}
\title{Herbarium collections management
}
\subtitle{Plant Science and Herbarium
}
\author{}
\publishers{\small
    \subsection*{Project Core Team}
\begin{tabu} {X X}
\textbf{Supervising Scientist} & Karina Knight
\\
\textbf{Data Custodian} & Karina Knight
\\
\textbf{Site Custodian} & Karina Knight
\\
\end{tabu}


    \subsection*{Project status as of Sept. 12, 2017, 10:15 a.m.}
\begin{tabu} {X X}
& Approved and active
\\
\end{tabu}

    
\subsection*{Document endorsements and approvals as of Sept. 12, 2017, 10:15 a.m.}
\begin{tabu} {X X}

%\rowcolor{grantedbg}
    \textbf{Project Team} & 
    \textcolor{granted}{ granted}\\

%\rowcolor{grantedbg}
    \textbf{Program Leader} & 
    \textcolor{granted}{ granted}\\

%\rowcolor{grantedbg}
    \textbf{Directorate} & 
    \textcolor{granted}{ granted}\\

\end{tabu}



}
\uppertitleback{}
\lowertitleback{}
\date{}

%-----------------------------------------------------------------------------%
% Front matter
\frontmatter
\maketitle
%-----------------------------------------------------------------------------%
% Main matter
\mainmatter

\section*{Herbarium collections management
}

K Knight, C Parker, J Huisman, J Percy-Bower, R Rees, S Coffey, M
Falconer, E McGough, E Wood-Ward, M Hislop, R Davis



\section*{Context}

The Western Australian Herbarium's Collection provides the core resource
for knowledge of the state's plants, algae and fungi. The Collection is
growing constantly, through addition of new specimens representing new
taxa and new records of existing taxa. The collection is maintained to a
high standard, and provides Parks and Wildlife and the community with
the fundamental resource providing knowledge of the diversity,
distribution and abundance of plants throughout Western Australia.




\section*{Aims}

\begin{itemize}
\itemsep1pt\parskip0pt\parsep0pt
\item
  Fully document and audit the diversity of Western Australia's plants,
  algae~and fungi.
\item
  Maintain in perpetuity a comprehensive, adequate and representative
  research and archive collection of specimens of all taxa in these
  groups occurring in Western Australia.
\item
  Contribute to, support and service the research, conservation and
  decision-making activities of Parks and Wildlife.
\item
  Contribute to, support and service taxonomic research by the world's
  scientific community.
\end{itemize}




\section*{Progress}

\begin{itemize}
\itemsep1pt\parskip0pt\parsep0pt
\item
  8732~specimens were added to the collection, which now stands at
  759797, a 1.16\% increase in holdings during this period.
\item
  The major plant groups in the collection are summarised in Appendix 1,
  Table 1
\item
  Loans and exchange: loans outward--18 loans consisting of 484
  specimens; loans inward--7 loans consisting of 115~specimens;~loans
  returned to the Herbarium--28 loans consisting of 713 specimens; loans
  returned to their home institution--24 loans consisting of 515
  specimens; exchange inward--1137 specimens; exchange outward--1222
  specimens including 12 requests for~destructive sampling.
\item
  Volunteer participation was significant, totalling~8993 hours which is
  equivalent to 5~full time employees.
\item
  Tasks managed by Collections staff with the assistance of volunteers
  were as follows:~mounting and labelling~8441 specimens; validating the
  name and~occurrence of~2327 incoming specimens for lodgement.
\item
  Recruited 45 volunteers.
\item
  Reference Herbarium: maintained the facility, which has over 14775
  specimens representing~11702 taxa and also added or replaced 180
  specimens.~1108 visitors used this resource to identify plant
  specimens during this period. This year the Swan River Trust Reference
  Collection was merged with the Reference Herbarium.
\item
  Research Collection: accessed by 432 visitors to study taxa or help
  identify specimens.
\item
  Provided~94 high resolution scans of herbarium specimens to Parks and
  Wildlife district staff and industry consultants to aid in the
  identification and location of known and new populations of priority
  and threatened taxa.
\item
  Significant lodgements: industry surveys,~regional DPAW officers,
  DPAW~Kimberley Islands survey, R. Chinnock (cactus), M. \& R. Barrett,
  Desert Discovery, N.Gibson (\emph{Calothamnus} study).
\item
  Reviewed, documented and made available on the Herbarium webpage:
  destructive~sampling policy, exchange policy, loans policy, loans
  conditions and quarantine~guidelines.
\item
  The Herbarium Identification Program provided identifications to a
  range of clients and specialises in taxa and specimens that clients
  find challenging. Our most significant clients included Parks and
  Wildlife, other government agencies, environmental consultancies,
  regional herbaria and the public.
\item
  Educational role: provided tours of the~Herbarium for tertiary
  institutions, Parks and Wildlife staff, environmental consultancies
  and community groups.
\item
  Scanned~532 types for the Global Plants Initiative. The~project was
  completed this year; 7857~WA~Herbarium type specimens~are now
  accessible~to the world's scientific community and other botanical
  resources for study.
\end{itemize}




\section*{Management implications}

\begin{itemize}
\itemsep1pt\parskip0pt\parsep0pt
\item
  Maintenance and curation of the herbarium collections provides an
  authoritative inventory of the plant biodiversity of Western
  Australia.
\item
  The collections are drawn upon constantly by Parks and Wildlife staff,
  consultants and others for validating specimen records from biological
  surveys and for assessing the conservation status of all native taxa.
\item
  Many taxa in Western Australia are yet undiscovered, but many of these
  are already represented by specimens in the Herbarium, awaiting
  recognition by taxonomists.
\end{itemize}




\section*{Future directions}

\begin{itemize}
\itemsep1pt\parskip0pt\parsep0pt
\item
  Continue to maintain the collection to an authoritative standard for
  all users.
\item
  Continue to review and document collections management~policy
  and~procedures to effect efficiencies and reflect modern herbarium
  practices,~and where applicable make available on the Herbarium
  webpage.
\item
  Recruit and retain 50 new volunteers to~assist in key Herbarium
  functions.
\end{itemize}



%-----------------------------------------------------------------------------%
% Back matter
%\backmatter
\end{document}
%-----------------------------------------------------------------------------%

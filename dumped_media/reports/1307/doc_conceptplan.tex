
\documentclass[version=last,
    paper=a4, % paper size
    10pt, % default font size
    usenames,
    dvipsnames,
    oneside, % ONLINE
    headings=openany, % open chapters on odd and even pages
    %toc=chapterentrywithdots, % Table of Contents style
    %BCOR=7mm, % PRINT Binding Correction
    %DIV=13, % typearea 161.54 mm x 228.46 mm, top margin 22.85 mm, inner margin 16.15 mm
    %DIV=14, % 165.00 233.36 21.21 15.00
    DIV=15 % 168.00 237.60 19.80 14.00
]{scrbook}
\usepackage{typearea}
\usepackage[automark,headsepline,footsepline]{scrlayer-scrpage} % Headers and footers

%%
%% Fonts, encoding, spacing, indentation
%%
\usepackage{txfonts}
\renewcommand{\familydefault}{\sfdefault} % Default to Sans Serif font
\usepackage[english]{babel}
\usepackage[T1]{fontenc}
\usepackage[utf8]{inputenc}

% Paragraph spacing
%\usepackage{parskip}    % Paragraph spacing
%\setlength{\parindent}{0em} % Don't indent paragraphs - ONLINE
%\setlength{\parskip}{1.3 ex plus 0.5ex minus 0.3ex} % 1-1.8 ex vertical space between paragraphs - ONLINE

% Spacing of headings
%\RedeclareSectionCommand[afterskip=3pt]{section} % less space after section
%\RedeclareSectionCommand[beforeskip=0cm]{subsection} % less space between HRule and project name
%\RedeclareSectionCommand[afterskip=0.1\baselineskip]{subsubsection} % less space after progressreport subheadings

% Table font size
\usepackage{etoolbox}
\AtBeginEnvironment{longtabu}{\footnotesize}{}{}

%%
%% Tables, columns, layout
%%
\usepackage{multicol}   % 2 col publications
\usepackage{pdflscape}  % Landscape pages
\usepackage{pdfpages}   % Include PDFs
\usepackage{hanging}    % hanging paragraphs for publications
%\usepackage{titletoc}   % Required for manipulating the table of contents
\setcounter{tocdepth}{2} % TOC list down to section
\usepackage{enumerate}  % Enumerations
\usepackage{enumitem}   % Enumerations
\usepackage{longtable}  % Multipage table
\usepackage{tabu}       %
\setlength{\tabulinesep}{1.5mm} % Consistent vertical spacing in tabu

%%
%% Graphics, images, colours
%%
\usepackage{graphicx} % embedded images
\usepackage{eso-pic} %
\usepackage{colortbl} % define custom named colours
\definecolor{RedFire}{RGB}{146,25,28}
\definecolor{ParksWildlife}{RGB}{0,85,144}
\definecolor{successbg}{RGB}{223,240,216}
\definecolor{errorbg}{RGB}{242,222,222}
\definecolor{warningbg}{RGB}{252,248,227}
\definecolor{infobg}{RGB}{217,237,247}
\definecolor{muted}{RGB}{153,153,153}
\definecolor{success}{RGB}{70,136,71}
\definecolor{error}{RGB}{185,74,72}
\definecolor{warning}{RGB}{192,152,83}
\definecolor{info}{RGB}{58,135,173}

\definecolor{required}{RGB}{192,152,83}
\definecolor{requiredbg}{RGB}{252,248,227}
\definecolor{denied}{RGB}{185,74,72}
\definecolor{deniedbg}{RGB}{242,222,222}
\definecolor{granted}{RGB}{70,136,71}
\definecolor{grantedbg}{RGB}{223,240,216}
\definecolor{not reqiured}{RGB}{153,153,153}
\definecolor{not requiredbg}{RGB}{255,255,255}

\usepackage{tikz} % Drawing
\usetikzlibrary{arrows,shapes,positioning,shadows,trees}

%%
%% Links, URLs
%%
\usepackage[
    linktoc=all,
    %colorlinks=false,  %PRINT
    colorlinks=true, % ONLINE
    linkcolor=RedFire, % ONLINE
    urlcolor=ParksWildlife, % ONLINE
    pdftitle=Concept Plan SP 2014-025
]{hyperref}

% Black magic to linebreak URLs
\usepackage{url}
\makeatletter
\g@addto@macro{\UrlBreaks}{\UrlOrds}
\makeatother

%%
%% Custom macros
%%
% Thick Horizontal rule
\newcommand{\HRule}{\vspace{8mm}\\\noindent\rule{\linewidth}{0.1pt}}

% Custom Tikz node for SDS diagram
\newcommand\mynode[6][]{
    \node[#1] (#2){
        \parbox{#3\relax}{
            \begin{center}
            \textbf{#4}\\
            #5\\
            \footnotesize{#6}
            \end{center}}};}



\usepackage[automark,headsepline,footsepline,plainfootsepline]{scrlayer-scrpage}
\automark*[section]{}
\addtokomafont{pageheadfoot}{\normalfont\footnotesize\sffamily} % Don't italicise
\renewcommand{\chaptermark}[1]{\markleft{#1}{}}     % Chapter: suppress numbering
\renewcommand{\sectionmark}[1]{\markright{#1}{}}    % Section: suppress numbering

% Header (inner, center, outer)
\ihead{\href{http://sdis.dpaw.wa.gov.au/documents/conceptplan/1307/}{Concept Plan SP 2014-025}}
%\chead{\href{http://sdis.dpaw.wa.gov.au}{Science Directorate Information System}}
\ohead{\href{https://www.dpaw.wa.gov.au/about-us/science-and-research}{\includegraphics[height=6mm, keepaspectratio]{/mnt/projects/sdis/staticfiles/img/logo-dpaw.png}}}

% Footer (inner, center, outer)
\ifoot{\textbf{Printed}~Sun, 27 Aug 2017 17:51:38 +0800} % inner/left footer
\cfoot{}
\ofoot[\bfseries\thepage]{\bfseries\thepage}        % Page number (also [plain])


\pagestyle{scrheadings}
\setkomafont{pageheadfoot}{\normalfont}

%-----------------------------------------------------------------------------%
\begin{document}
\raggedbottom

%-----------------------------------------------------------------------------%
% Title page
\subject{Concept Plan SP 2014-025
}
\title{Taxonomy, zoogeography and conservation status of aquatic invertebrates
}
\subtitle{Wetlands Conservation
}
\author{}
\publishers{\small
    \subsection*{Project Core Team}
\begin{tabu} {X X}
\textbf{Supervising Scientist} & Adrian Pinder
\\
\textbf{Data Custodian} & Adrian Pinder
\\
\textbf{Site Custodian} & Adrian Pinder
\\
\end{tabu}


    \subsection*{Project status as of Aug. 27, 2017, 5:51 p.m.}
\begin{tabu} {X X}
& Approved and active
\\
\end{tabu}

    
\subsection*{Document endorsements and approvals as of Aug. 27, 2017, 5:51 p.m.}
\begin{tabu} {X X}

%\rowcolor{grantedbg}
    \textbf{Project Team} & 
    \textcolor{granted}{ granted}\\

%\rowcolor{grantedbg}
    \textbf{Program Leader} & 
    \textcolor{granted}{ granted}\\

%\rowcolor{grantedbg}
    \textbf{Directorate} & 
    \textcolor{granted}{ granted}\\

\end{tabu}



}
\uppertitleback{}
\lowertitleback{}
\date{}

%-----------------------------------------------------------------------------%
% Front matter
\frontmatter
\maketitle
%-----------------------------------------------------------------------------%
% Main matter
\mainmatter


\section*{Taxonomy, zoogeography and conservation status of aquatic invertebrates
}



\subsection*{Science and Conservation Division Program}

Wetlands Conservation




\subsection*{Parks and Wildlife Service}

Service 2: Conserving Habitats, Species and Ecological Communities




\subsection*{Aims}

The wetlands conservation program has a strong focus on aquatic
invertebrate biodiversity, including spatial patterning and trends over
time in relation to threats. Over half of the species we deal with are
not formally described, but they are consistently named across projects
through maintenance of a voucher specimen collection.

As opportunities and skills allow program staff undertake systematics
studies (primarily species descriptions), sometimes with specialist
co-authors. This allows formal naming and description of Western
Australian endemics that would not otherwise occur and allows species to
be consistently identified by external research groups. Examples include
descriptions of 'giant ostracods by Halse and McRae (Hydrobiologia 524)
and first records of phallodriline oligochaetes from Australian
non-marine waters by Pinder et al. (Zootaxa 1304). Kirsty Quinlan is
currently describing a species of Boeckella copepod from Lorna Glen.

For many groups of invertebrates there are inadequate tools for their
identification and program staff frequently devise in-house keys to help
with identification consistency. On occasion we have been able to
produce tools of sufficient quality to release to the public. Examples
are keys to aquatic oligochaetes (two publications in Museum of Victoria
Science Reports by Pinder) and keys to non-biting midge larvae of
south-western Australia by Leung et al. Adrian Pinder is also working
with Russell Shiel (University of Adelaide) to revise and expand keys to
Australian rotifers.

Kirsty Quinlan is starting to investigate the use of bar coding for our
work, focusing on being able to identify larvae of species normally only
identified from adults. Her recent success with Berosus (scavenger
beetle) larvae demonstrates we have the capacity to undertake this work.
We are also undertaking a molecular study of Glacidorbis snails from
Drummond Nature Reserve (are they Glacidorbis occidentalis?), with
analyses being performed by the Western Australian Museum with funding
from the Drummond NDRC.




\subsection*{Expected outcome}

Improved ability to consistently identify aquatic invertebrates, both
within internally and externally, allowing 1) greater compatibility
between datasets, 2) reduced sample processing times and 3) more
complete identifications (by enabling identification of previously
unidentifiable life stages or sexes). These improvements mean we can
better describe spatial and temporal trends in biodiversity and better
understand species distributions and conservation status.




\subsection*{Strategic context}

This project mainly addresses corporate goals by adding value to other
projects, by allowing more complete and consistent species lists in
monitoring and survey projects. In this way this project contributes to:

DPaW Strategic Goals:

\begin{itemize}
\itemsep1pt\parskip0pt\parsep0pt
\item
  ``Integrated science and nature conservation: Ensure conservation
  management is based best practice science.''
\item
  ``Protection of threatened animals in partnership with other
  organisations, traditional owners and the community.'' (many
  oligochaetes are rare/restricted or part of TECs)
\end{itemize}

A Strategic Plan for Biodiversity Conservation Research:

\begin{itemize}
\itemsep1pt\parskip0pt\parsep0pt
\item
  Goal 1 ``Understand the composition of, and patterning in, terrestrial
  and marine biodiversity'': Strategy 1.15 ``Continue collections and
  descriptions of fungi and invertebrates in association with research,
  survey and monitoring''.
\end{itemize}




\subsection*{Expected collaborations}

There will be numerous opportunities for collaborations with external
taxonomists, as has happened in the past (e.g. Chris Watts from SA
Museum, Russell Sheil from University of Adelaide and Christer Erseus
from Sweden).


\subsection*{Proposed period of the project}
Dec. 9, 2014 -- None



\subsection*{Staff time allocation }



\begin{longtabu} to \linewidth { |  X | X | X | X | }
\hline
\rowcolor{infobg}
Role & Year 1 & Year 2 & Year 3\\
\hline
\endhead



Scientist (Pinder) & 0.1 & 0.1 & 0.1\\



Technical (Quinlan et al.) & 0.1 & 0.1 & 0.1\\



Technical (Cale) & 0.05 &  & \\



Collaborator &  &  & \\


\hline
\end{longtabu}



\subsection*{Indicative operating budget }



\begin{longtabu} to \linewidth { |  X | X | X | X | }
\hline
\rowcolor{infobg}
Source & Year 1 & Year 2 & Year 3\\
\hline
\endhead



Consolidated Funds (DPaW) &  &  & \\



External Funding & ~3000 & ~2000 & ~2000\\


\hline
\end{longtabu}






%-----------------------------------------------------------------------------%
% Back matter
%\backmatter
\end{document}
%-----------------------------------------------------------------------------%

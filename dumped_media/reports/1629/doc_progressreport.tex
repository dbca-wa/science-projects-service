
\documentclass[version=last,
    paper=a4, % paper size
    10pt, % default font size
    usenames,
    dvipsnames,
    oneside, % ONLINE
    headings=openany, % open chapters on odd and even pages
    %toc=chapterentrywithdots, % Table of Contents style
    %BCOR=7mm, % PRINT Binding Correction
    %DIV=13, % typearea 161.54 mm x 228.46 mm, top margin 22.85 mm, inner margin 16.15 mm
    %DIV=14, % 165.00 233.36 21.21 15.00
    DIV=15 % 168.00 237.60 19.80 14.00
]{scrbook}
\usepackage{typearea}
\usepackage[automark,headsepline,footsepline]{scrlayer-scrpage} % Headers and footers

%%
%% Fonts, encoding, spacing, indentation
%%
\usepackage{txfonts}
\renewcommand{\familydefault}{\sfdefault} % Default to Sans Serif font
\usepackage[english]{babel}
\usepackage[T1]{fontenc}
\usepackage[utf8]{inputenc}

% Paragraph spacing
%\usepackage{parskip}    % Paragraph spacing
%\setlength{\parindent}{0em} % Don't indent paragraphs - ONLINE
%\setlength{\parskip}{1.3 ex plus 0.5ex minus 0.3ex} % 1-1.8 ex vertical space between paragraphs - ONLINE

% Spacing of headings
%\RedeclareSectionCommand[afterskip=3pt]{section} % less space after section
%\RedeclareSectionCommand[beforeskip=0cm]{subsection} % less space between HRule and project name
%\RedeclareSectionCommand[afterskip=0.1\baselineskip]{subsubsection} % less space after progressreport subheadings

% Table font size
\usepackage{etoolbox}
\AtBeginEnvironment{longtabu}{\footnotesize}{}{}

%%
%% Tables, columns, layout
%%
\usepackage{multicol}   % 2 col publications
\usepackage{pdflscape}  % Landscape pages
\usepackage{pdfpages}   % Include PDFs
\usepackage{hanging}    % hanging paragraphs for publications
%\usepackage{titletoc}   % Required for manipulating the table of contents
\setcounter{tocdepth}{2} % TOC list down to section
\usepackage{enumerate}  % Enumerations
\usepackage{enumitem}   % Enumerations
\usepackage{longtable}  % Multipage table
\usepackage{tabu}       %
\setlength{\tabulinesep}{1.5mm} % Consistent vertical spacing in tabu

%%
%% Graphics, images, colours
%%
\usepackage{graphicx} % embedded images
\usepackage{eso-pic} %
\usepackage{colortbl} % define custom named colours
\definecolor{RedFire}{RGB}{146,25,28}
\definecolor{ParksWildlife}{RGB}{0,85,144}
\definecolor{successbg}{RGB}{223,240,216}
\definecolor{errorbg}{RGB}{242,222,222}
\definecolor{warningbg}{RGB}{252,248,227}
\definecolor{infobg}{RGB}{217,237,247}
\definecolor{muted}{RGB}{153,153,153}
\definecolor{success}{RGB}{70,136,71}
\definecolor{error}{RGB}{185,74,72}
\definecolor{warning}{RGB}{192,152,83}
\definecolor{info}{RGB}{58,135,173}

\definecolor{required}{RGB}{192,152,83}
\definecolor{requiredbg}{RGB}{252,248,227}
\definecolor{denied}{RGB}{185,74,72}
\definecolor{deniedbg}{RGB}{242,222,222}
\definecolor{granted}{RGB}{70,136,71}
\definecolor{grantedbg}{RGB}{223,240,216}
\definecolor{not reqiured}{RGB}{153,153,153}
\definecolor{not requiredbg}{RGB}{255,255,255}

\usepackage{tikz} % Drawing
\usetikzlibrary{arrows,shapes,positioning,shadows,trees}

%%
%% Links, URLs
%%
\usepackage[
    linktoc=all,
    %colorlinks=false,  %PRINT
    colorlinks=true, % ONLINE
    linkcolor=RedFire, % ONLINE
    urlcolor=ParksWildlife, % ONLINE
    pdftitle=Progress Report SP 2012-005 (FY 2015-2016)
]{hyperref}

% Black magic to linebreak URLs
\usepackage{url}
\makeatletter
\g@addto@macro{\UrlBreaks}{\UrlOrds}
\makeatother

%%
%% Custom macros
%%
% Thick Horizontal rule
\newcommand{\HRule}{\vspace{8mm}\\\noindent\rule{\linewidth}{0.1pt}}

% Custom Tikz node for SDS diagram
\newcommand\mynode[6][]{
    \node[#1] (#2){
        \parbox{#3\relax}{
            \begin{center}
            \textbf{#4}\\
            #5\\
            \footnotesize{#6}
            \end{center}}};}



%-----------------------------------------------------------------------------%
% Headers and Footers
\automark{section}
\ohead{\href{http://sdis.dpaw.wa.gov.au/documents/progressreport/1629/}{Progress Report SP 2012-005
}}
\chead{\href{http://sdis.dpaw.wa.gov.au}{SDIS}} % center header ONLINE
\ihead{\href{http://sdis.dpaw.wa.gov.au}{
        \includegraphics[scale=0.4]{/mnt/projects/sdis/staticfiles/img/logo-dpaw.png}}}
\ifoot{\textbf{Printed}~Wed, 20 Jul 2016 11:56:43 +0800} % inner/left footer
\cfoot{} % center footer
\ofoot{\pagemark} % outer/right footer
\pagestyle{scrheadings}
\setkomafont{pageheadfoot}{\normalfont}

%-----------------------------------------------------------------------------%
\begin{document}
\raggedbottom

%-----------------------------------------------------------------------------%
% Title page
\subject{Progress Report SP 2012-005
}
\title{Western Australian flora surveys
}
\subtitle{Biogeography
}
\author{}
\publishers{\small
    \subsection*{Project Core Team}
\begin{tabu} {X X}
\textbf{Supervising Scientist} & Neil Gibson
\\
\textbf{Data Custodian} & Neil Gibson
\\
\textbf{Site Custodian} & Neil Gibson
\\
\end{tabu}


    \subsection*{Project status as of July 20, 2016, 11:56 a.m.}
\begin{tabu} {X X}
& Approved and active
\\
\end{tabu}

    
\subsection*{Document endorsements and approvals as of July 20, 2016, 11:56 a.m.}
\begin{tabu} {X X}

%\rowcolor{grantedbg}
    \textbf{Project Team} & 
    \textcolor{granted}{ granted}\\

%\rowcolor{grantedbg}
    \textbf{Program Leader} & 
    \textcolor{granted}{ granted}\\

%\rowcolor{grantedbg}
    \textbf{Directorate} & 
    \textcolor{granted}{ granted}\\

\end{tabu}



}
\uppertitleback{}
\lowertitleback{}
\date{}

%-----------------------------------------------------------------------------%
% Front matter
\frontmatter
\maketitle
%-----------------------------------------------------------------------------%
% Main matter
\mainmatter

\section*{Western Australian flora surveys
}

N Gibson, N Casson, G Keighery, R Meissner, M Langley, M Lyons, S van
Leeuwen, A Markey, B Bayliss, R Coppen


\section*{Context}
Flora surveys of targeted areas provide knowledge of vegetation pattern
and structure for conservation management. These surveys are undertaken
for a variety of purposes and for, or in collaboration with, a number of
partner organisations.

Current projects include:

\begin{itemize}
\itemsep1pt\parskip0pt\parsep0pt
\item
  AusPlots Rangeland survey sites as a baseline for long-term
  surveillance monitoring in collaboration with the Terrestrial
  Ecosystems Research Network (TERN). This AusPlots Wester Australian
  campaign will focused on rangeland bioregions including the~South West
  Australia Transitional~Transect (SWATT) in the Great Western
  Woodlands.
\item
  Flora survey of the Katjarra (Carnarvon Range) Indigenous Protected
  Area (IPA) in collaboration with the Birriliburu Native Title
  Claimants to aid future management. Survey campaign funded by Central
  Desert Native Title Services and undertaken in collaboration with
  Birriliburu Rangers and Bush Heritage Australia.
\item
  Floristic survey and mapping of the halophyte-dominated communities of
  the Fortescue Marsh.
\item
  Capture of vegetation mapping data for the Great Western
  Woodlands,~Indian Ocean Drive and central Pilbara to inform natural
  resource management and land use planning.
\item
  Black spot flora survey of the Peterswald 1:100,000 map sheet, funded
  by Federal the Environment.
\item
  Study of the impacts of weeds and grazing following prescribed fire in
  a Banksia woodland.
\item
  Resurvey of threatened claypan commuities on the Swan Coastal Plain
  documenting change over 20 years.
\item
  Floristic survey of the mound springs and surrounding vegetation
  communities of Mandora Marsh / Waylarta in collaboration with the West
  Kimberley District.
\item
  Floristic survey of the coastal wetlands of the Jurien area.
\end{itemize}



\section*{Aims}
\begin{itemize}
\itemsep1pt\parskip0pt\parsep0pt
\item
  To undertake smaller scale surveys for particular purposes (e.g.
  NatureBank; Bush Blitz surveys ; Black Spot surveys ; AusPlots
  establishment; assistance to Traditional Owners to survey high
  conservation IPAs). These short duration surveys are primarily aimed
  at providing specific management advice, monitor long term change in
  vegetation at specific sites or in specific communities, or to fill
  specific data gaps.
\end{itemize}



\section*{Progress}
\begin{itemize}
\itemsep1pt\parskip0pt\parsep0pt
\item
  Flora survey report of Peterswald has been finalised.
\item
  Flora survey of Colville has been completed, plant identifications are
  continuing.
\item
  Paper on impacts of fire in urban remnant has been published.
\item
  Claypan dataset has been compiled, analysis is continuing.
\item
  Floristic communities have been described for the Fortescue Marsh, and
  mapping of halophyte-dominated communities is currently underway.
\item
  Flora survey of Mandora Marsh / Walyarta has been finalised and a
  draft report prepared for the West Kimberley Region.
\item
  Jurien coastal wetlands survey completed, preparation of scientific
  paper underway.
\item
  Vegetation map reconciliation continued with the digital capture of
  over 360,000 ha of mapping in the central Hamersley Range~ between
  Munjina and Weelumurra.
\item
  A Pilbara AusPlots campaign collaboratively~funded by Pilbara
  Corridors - Rangelands NRM,, AusPlots and~Parks and Wildlife
  established 19 plots in the Fortescue catchment between Millstream
  Chichester and Karijini National Parks.
\end{itemize}



\section*{Management implications}
\begin{itemize}
\itemsep1pt\parskip0pt\parsep0pt
\item
  The survey of the Colville area will increased the known flora known
  this area. Data will be made available via NatureMap and Western
  Australian Herbarium database. This will provide information on the
  flora values of the area and assist in land use planning and impact
  assessments for future resource development.
\item
  Fire study will inform weed management following fire in long unburnt
  urban remnants.
\item
  Analysis of the claypan data will provide information on the temporal
  variation in these nationally threatened communities, and develop
  methods to assess ecosystem health.
\item
  Survey of Mandora Marsh / Walyarta and the Fortescue Marsh will
  provide information for draft management plans for existing and
  proposed reserves in the southern Kimberley and Pilbara.
\item
  The Jurien coastal survey will provide information on wetland flora
  values to assist in land use planning and impact assessments for
  mining and water resource developments.
\item
  The digital capture and reconciliation of vegetation maps will~inform
  environmental impact assessment processes associated with native
  Vegetation Clearing permits and major resource development
  proposals.~~ This information will also enable fire management
  planning, particularly the development of burning prescriptions for
  Karijini National Park.
\end{itemize}



\section*{Future directions}
\begin{itemize}
\itemsep1pt\parskip0pt\parsep0pt
\item
  Further surveys will be undertaken as required and when resources
  become available. The development of collaborative arrangements to
  facilitate future surveys is underway and involves discussions with
  Traditional Owners, natural resource managers, resource developers and
  both government and private sector managers of land and biodiversity
  assets.~ Another Pilbara AusPlots campaign is planned for 2016 and
  resources may also be forthcoming to expand this survey effort into
  the Kimberley and onto the Swan Coastal Plain.~ Additional AusPlots
  will also be established along the SWATT in 2016.
\end{itemize}



%-----------------------------------------------------------------------------%
% Back matter
%\backmatter
\end{document}
%-----------------------------------------------------------------------------%


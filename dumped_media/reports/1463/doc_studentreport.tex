
\documentclass[version=last, paper=a4, DIV=18, usenames, dvipsnames]{scrartcl}
\usepackage{txfonts}
\usepackage{pdflscape}
\usepackage{pdfpages}
\usepackage[english]{babel} % English language/hyphenation
%%% Bootstrap colors
\definecolor{RedFire}{RGB}{146,25,28}
\definecolor{ParksWildlife}{RGB}{0,85,144}
\definecolor{successbg}{RGB}{223,240,216}
\definecolor{errorbg}{RGB}{242,222,222}
\definecolor{warningbg}{RGB}{252,248,227}
\definecolor{infobg}{RGB}{217,237,247}
\definecolor{muted}{RGB}{153,153,153}
\definecolor{success}{RGB}{70,136,71}
\definecolor{error}{RGB}{185,74,72}
\definecolor{warning}{RGB}{192,152,83}
\definecolor{info}{RGB}{58,135,173}

\definecolor{required}{HTML}{D9534F}
\definecolor{denied}{HTML}{D9534F}
\definecolor{granted}{HTML}{47A447}
\definecolor{not required}{RGB}{200, 200, 200}

\usepackage[colorlinks=true,pdftitle=doc\_studentreport.pdf
,linktoc=all,linkcolor=RedFire,urlcolor=ParksWildlife]{hyperref}
\usepackage{colortbl}
\usepackage{longtable}
\usepackage{tabu}
\setlength{\tabulinesep}{1.5mm}
\usepackage{enumerate}
\usepackage{enumitem}
\usepackage{fancyhdr}
\usepackage{lastpage}
\usepackage{graphicx}
\usepackage{eso-pic}
\usepackage{hyphenat}
\renewcommand{\familydefault}{\sfdefault}



\newcommand{\HRule}{\rule{\linewidth}{0.1pt}}

\newcommand{\placetextbox}[3]{% \placetextbox{<horizontal pos>}{<vertical pos>}{<stuff>}
  \setbox0=\hbox{#3}% Put <stuff> in a box
  \AddToShipoutPictureFG*{% Add <stuff> to current page foreground
    \put(\LenToUnit{#1\paperwidth},\LenToUnit{#2\paperheight}){\vtop{{\null}\makebox[0pt][c]{#3}}}%
  }%
}%




%-----------------------------------------------------------------------------%
% Headers and footers
%
\fancypagestyle{plain}{
  \fancyhf{}
  \setlength\headheight{60pt} % push page content below header
  \renewcommand{\headrulewidth}{0.1pt}
  \renewcommand{\footrulewidth}{0.1pt}
  
  
  \fancyhead[L]{ 
    \href{http://sdis.dpaw.wa.gov.au}{
    \includegraphics[scale=0.6]{/mnt/projects/sdis/staticfiles/img/logo-dpaw.png}}
  }
  \fancyhead[R]{ 
      \hfill
      \href{http://sdis.dpaw.wa.gov.au}{Science Directorate Information System} 
      \newline 
      \href{http://sdis.dpaw.wa.gov.au/documents/studentreport/1463/}{Progress Report 2012-225 (FY 2014-2015)} 
  }
  
  
  
  
  \fancyfoot[L]{ \leftmark\newline\textbf{Printed}\textit{ June 24, 2015, 3:46 p.m. }}
  \fancyfoot[R]{  \, \newline Page \thepage\ of \pageref{LastPage} }
  
  
}
\pagestyle{plain}
%
% end Headers
%-----------------------------------------------------------------------------%

\begin{document}

%-----------------------------------------------------------------------------%
% Title page
%

%
% end title page
%-----------------------------------------------------------------------------%




\section*{Progress Report}
This project aims to: i) determine if a reliable estimate of quokka
abundance can be obtained from indicators of activity including scats,
tracks and runnels; ii) identify the preferred habitat of quokka in
southern forests; iii) determine the mobility and activity patterns of
quokka in the southern forests; iv) identify the influence of fire on
distribution and abundance of quokka in the southern forests; and v) in
collaboration with others determine whether the sub-populations
constitute a functional meta-population. Occupancy models were generated
from presence/absence data and have identified the density of the
near-surface fuel layer, vegetation structure and proximity to a
different fuel age as the subset of variables that best predict the
probability of occupancy of habitat by quokka. Associated monitoring by
cage and camera trapping indicates that feral cats were responsible for
almost complete recruitment failure over a four year period due to
predation of young immediately after pouch emergence.

Home range and movement patterns have been investigated using 29
collared quokkas and results indicate a mean home range of 71ha (core
range 18ha) with movements averaging between 0.4 and 2.4km/night.
Largest movements were recorded in summer and autumn and were linked to
requirements to forage further afield for water and food during hot dry
conditions. Collared animals spent 40\% of their time in riparian
habitat within a stable home range and emigrating individuals travelled
distances of up to 14.2km, using riparian vegetation as corridors.
Forest areas with fire treatment and comparable unburnt sites have been
examined for quokka abundance and habitat quality pre- and post-fire to
determine the effect of fire on habitat use and the time taken for
habitat to become re-colonised post-fire. DNA has been provided to staff
at Murdoch University, who will be assisting with DNA processing. A
paper presenting an effective and efficient survey method for quokka has
been published.




\clearpage



\end{document}


\documentclass[version=last,
    paper=a4, % paper size
    10pt, % default font size
    usenames,
    dvipsnames,
    oneside, % ONLINE
    headings=openany, % open chapters on odd and even pages
    %toc=chapterentrywithdots, % Table of Contents style
    %BCOR=7mm, % PRINT Binding Correction
    %DIV=13, % typearea 161.54 mm x 228.46 mm, top margin 22.85 mm, inner margin 16.15 mm
    %DIV=14, % 165.00 233.36 21.21 15.00
    DIV=15 % 168.00 237.60 19.80 14.00
]{scrbook}
\usepackage{typearea}
\usepackage[automark,headsepline,footsepline]{scrlayer-scrpage} % Headers and footers

%%
%% Fonts, encoding, spacing, indentation
%%
\usepackage{txfonts}
\renewcommand{\familydefault}{\sfdefault} % Default to Sans Serif font
\usepackage[english]{babel}
\usepackage[T1]{fontenc}
\usepackage[utf8]{inputenc}

% Paragraph spacing
%\usepackage{parskip}    % Paragraph spacing
%\setlength{\parindent}{0em} % Don't indent paragraphs - ONLINE
%\setlength{\parskip}{1.3 ex plus 0.5ex minus 0.3ex} % 1-1.8 ex vertical space between paragraphs - ONLINE

% Spacing of headings
%\RedeclareSectionCommand[afterskip=3pt]{section} % less space after section
%\RedeclareSectionCommand[beforeskip=0cm]{subsection} % less space between HRule and project name
%\RedeclareSectionCommand[afterskip=0.1\baselineskip]{subsubsection} % less space after progressreport subheadings

% Table font size
\usepackage{etoolbox}
\AtBeginEnvironment{longtabu}{\footnotesize}{}{}

%%
%% Tables, columns, layout
%%
\usepackage{multicol}   % 2 col publications
\usepackage{pdflscape}  % Landscape pages
\usepackage{pdfpages}   % Include PDFs
\usepackage{hanging}    % hanging paragraphs for publications
%\usepackage{titletoc}   % Required for manipulating the table of contents
\setcounter{tocdepth}{2} % TOC list down to section
\usepackage{enumerate}  % Enumerations
\usepackage{enumitem}   % Enumerations
\usepackage{longtable}  % Multipage table
\usepackage{tabu}       %
\setlength{\tabulinesep}{1.5mm} % Consistent vertical spacing in tabu

%%
%% Graphics, images, colours
%%
\usepackage{graphicx} % embedded images
\usepackage{eso-pic} %
\usepackage{colortbl} % define custom named colours
\definecolor{RedFire}{RGB}{146,25,28}
\definecolor{ParksWildlife}{RGB}{0,85,144}
\definecolor{successbg}{RGB}{223,240,216}
\definecolor{errorbg}{RGB}{242,222,222}
\definecolor{warningbg}{RGB}{252,248,227}
\definecolor{infobg}{RGB}{217,237,247}
\definecolor{muted}{RGB}{153,153,153}
\definecolor{success}{RGB}{70,136,71}
\definecolor{error}{RGB}{185,74,72}
\definecolor{warning}{RGB}{192,152,83}
\definecolor{info}{RGB}{58,135,173}

\definecolor{required}{RGB}{192,152,83}
\definecolor{requiredbg}{RGB}{252,248,227}
\definecolor{denied}{RGB}{185,74,72}
\definecolor{deniedbg}{RGB}{242,222,222}
\definecolor{granted}{RGB}{70,136,71}
\definecolor{grantedbg}{RGB}{223,240,216}
\definecolor{not reqiured}{RGB}{153,153,153}
\definecolor{not requiredbg}{RGB}{255,255,255}

\usepackage{tikz} % Drawing
\usetikzlibrary{arrows,shapes,positioning,shadows,trees}

%%
%% Links, URLs
%%
\usepackage[
    linktoc=all,
    %colorlinks=false,  %PRINT
    colorlinks=true, % ONLINE
    linkcolor=RedFire, % ONLINE
    urlcolor=ParksWildlife, % ONLINE
    pdftitle=Concept Plan SP 2015-019
]{hyperref}

% Black magic to linebreak URLs
\usepackage{url}
\makeatletter
\g@addto@macro{\UrlBreaks}{\UrlOrds}
\makeatother

%%
%% Custom macros
%%
% Thick Horizontal rule
\newcommand{\HRule}{\vspace{8mm}\\\noindent\rule{\linewidth}{0.1pt}}

% Custom Tikz node for SDS diagram
\newcommand\mynode[6][]{
    \node[#1] (#2){
        \parbox{#3\relax}{
            \begin{center}
            \textbf{#4}\\
            #5\\
            \footnotesize{#6}
            \end{center}}};}



\usepackage[automark,headsepline,footsepline,plainfootsepline]{scrlayer-scrpage}
\automark*[section]{}
\addtokomafont{pageheadfoot}{\normalfont\footnotesize\sffamily} % Don't italicise
\renewcommand{\chaptermark}[1]{\markleft{#1}{}}     % Chapter: suppress numbering
\renewcommand{\sectionmark}[1]{\markright{#1}{}}    % Section: suppress numbering

% Header (inner, center, outer)
\ihead{\href{http://sdis.dpaw.wa.gov.au/documents/conceptplan/1515/}{Concept Plan SP 2015-019}}
%\chead{\href{http://sdis.dpaw.wa.gov.au}{Science Directorate Information System}}
\ohead{\href{https://www.dpaw.wa.gov.au/about-us/science-and-research}{\includegraphics[height=6mm, keepaspectratio]{/mnt/projects/sdis/staticfiles/img/logo-dpaw.png}}}

% Footer (inner, center, outer)
\ifoot{\textbf{Printed}~Mon, 3 Jul 2017 13:41:50 +0800} % inner/left footer
\cfoot{}
\ofoot[\bfseries\thepage]{\bfseries\thepage}        % Page number (also [plain])


\pagestyle{scrheadings}
\setkomafont{pageheadfoot}{\normalfont}

%-----------------------------------------------------------------------------%
\begin{document}
\raggedbottom

%-----------------------------------------------------------------------------%
% Title page
\subject{Concept Plan SP 2015-019
}
\title{Recovery of the Numbat \emph{Myrmecobius fasciatus}
}
\subtitle{Animal Science
}
\author{}
\publishers{\small
    \subsection*{Project Core Team}
\begin{tabu} {X X}
\textbf{Supervising Scientist} & Tony Friend
\\
\textbf{Data Custodian} & Tony Friend
\\
\textbf{Site Custodian} & 
\\
\end{tabu}


    \subsection*{Project status as of July 3, 2017, 1:41 p.m.}
\begin{tabu} {X X}
& New project, pending concept plan approval
\\
\end{tabu}

    
\subsection*{Document endorsements and approvals as of July 3, 2017, 1:41 p.m.}
\begin{tabu} {X X}

%\rowcolor{requiredbg}
    \textbf{Project Team} & 
    \textcolor{required}{ required}\\

%\rowcolor{requiredbg}
    \textbf{Program Leader} & 
    \textcolor{required}{ required}\\

%\rowcolor{requiredbg}
    \textbf{Directorate} & 
    \textcolor{required}{ required}\\

\end{tabu}



}
\uppertitleback{}
\lowertitleback{}
\date{}

%-----------------------------------------------------------------------------%
% Front matter
\frontmatter
\maketitle
%-----------------------------------------------------------------------------%
% Main matter
\mainmatter


\section*{Recovery of the Numbat \emph{Myrmecobius fasciatus}
}



\subsection*{Science and Conservation Division Program}

Animal Science




\subsection*{Parks and Wildlife Service}

Service 2: Conserving Habitats, Species and Ecological Communities




\subsection*{Aims}

Background

The numbat is the only marsupial termite specialist and is of high
conservation significance due to its phylogenetic distinctness.~ Its
recorded decline since European settlement saw its distribution reduced
from widespread across southern Australia to just two small populations
in south-western WA (Friend 1989).~ In the early 1980s a research
program to determine the causes of decline showed that control of foxes
using 1080 meat baits was followed by a dramatic recovery of the
Dryandra numbat population.~ Based on this finding, a vigorous program
of reintroduction by Parks \& Wildlife (and its predecessors)~with
collaborating organisations commencing in 1985 increased the number of
populations to eight (Friend and Thomas 2004) with several more in
various stages of establishment. ~

Despite this success, one of the two surviving populations, at Dryandra
Woodland, has declined rapidly to a very low level.~ Another SPP,
``Feral cat control and numbat recovery in Dryandra woodland and other
sites'' is concerned with this issue, using the hypothesis that, under
an effective fox control regime, predation by cats has taken over from
predation by foxes as the regulating influence on numbat populations. ~

Under this Science Project, the numbat reintroduction program will
continue, but with a monitoring program enhanced by use of~DNA
technology that will determine with certainty the species and sometimes
the individual identity of the responsible predator.~ This will allow
critical intervention strategies to be developed before the numbat
population becomes extinct.~~

Project Aim

To provide management solutions to effect improvement in the
conservation status of the numbat, currently listed as Endangered by
IUCN and Vulnerable under the EPBC Act.~




\subsection*{Expected outcome}

1) Increases in the number of numbat populations and the total numbers
of numbats, resulting in population data that will support relisting by
IUCN from Endangered to Vulnerable.~

2) Demonstration of management practices that will result in numbat
populations achieving stability at carrying capacity.~

3) Demonstration that Parks \& Wildlife, in collaboration with other
groups, can prevent the extinction of Western Australia's mammal emblem.




\subsection*{Strategic context}

Captive breeding and translocation are key actions in the Recovery Plans
for both the numbat (Friend and Page 2015) and dibbler (Friend 2004).~
Improved conservation security for the numbat is one of the proposed
outcomes of the Department's Strategic Directions for 2014-2017.
Conservation of the numbat is supported by a Government election
commitment aimed at protecting the population at Dryandra, now the
subject of a major Wheatbelt Region project.




\subsection*{Expected collaborations}

Collaborations are currently under way with a number of agencies and
institutions.~

\begin{itemize}
\itemsep1pt\parskip0pt\parsep0pt
\item
  ~ ~ Parks \& Wildlife staff in Wellington, Perth Hills and Great
  Southern Districts report numbat sightings and assist in numbat
  monitoring at times. ~
\item
  ~~~ We have a close relationship with Perth Zoo, where the breeding
  colony is based and in the past we have collaborated in research
  projects. ~
\item
  ~~~ A Ph.D. project at UWA involving behavioural assessment of
  individual numbats (and dibblers) prior to release is nearing
  completion.~
\item
  ~~~ Analysis of genetic variability in numbat populations will be
  carried out as a student project through Murdoch University (Peter
  Spencer's lab).~
\item
  ~~~ The community group Project Numbat is closely involved with the
  numbat recovery program, and has provided funds for radio-collars,
  tracking flights and trail cameras.~ We are currently negotiating a
  regular survey program to monitor all WA numbat sites.~
\item
  ~~~ The numbat recovery project also has close ties with Australian
  Wildlife Conservancy and Arid Recovery where numbat populations exist
  or are in process of establishment and we are collaborating in
  reintroduction programs with both groups.~
\item
  ~~~ Volunteers are often involved in aspects of numbat research,
  either recruited from the community or through Project Numbat and the
  Friends of the Fitzgerald River National Park.~~
\end{itemize}


\subsection*{Proposed period of the project}
Oct. 6, 2015 -- June 30, 2019



\subsection*{Staff time allocation }



\begin{longtabu} to \linewidth { |  X | X | X | X | }
\hline
\rowcolor{infobg}
Role & Year 1 & Year 2 & Year 3\\
\hline
\endhead



Scientist & 0.2 & 0.2 & 0.2\\



Technical &  &  & \\



Volunteer & 0.1 & 0.1 & 0.1\\



Collaborator & 0.1 & 0.1 & 0.1\\


\hline
\end{longtabu}



\subsection*{Indicative operating budget }



\begin{longtabu} to \linewidth { |  X | X | X | X | }
\hline
\rowcolor{infobg}
Source & Year 1 & Year 2 & Year 3\\
\hline
\endhead



Consolidated Funds (DPaW) &  &  & \\



External Funding & 20,000 & 20,000 & 20,000\\


\hline
\end{longtabu}






%-----------------------------------------------------------------------------%
% Back matter
%\backmatter
\end{document}
%-----------------------------------------------------------------------------%


\documentclass[version=last, 
    paper=a4, % paper size
    10pt, % default font size
    usenames,
    dvipsnames, 
    oneside, % ONLINE
    headings=openany, % open chapters on odd and even pages
    %toc=chapterentrywithdots, % Table of Contents style
    %BCOR=7mm, % PRINT Binding Correction
    %DIV=13, % typearea 161.54 mm x 228.46 mm, top margin 22.85 mm, inner margin 16.15 mm
    %DIV=14, % 165.00 233.36 21.21 15.00
    DIV=15 % 168.00 237.60 19.80 14.00
]{scrbook}
\usepackage{typearea}
\usepackage[automark,headsepline,footsepline]{scrlayer-scrpage} % Headers and footers

%%
%% Fonts, encoding, spacing, indentation
%%
\usepackage{txfonts}
\renewcommand{\familydefault}{\sfdefault} % Default to Sans Serif font
\usepackage[english]{babel}
\usepackage[T1]{fontenc}
\usepackage[utf8]{inputenc}

% Paragraph spacing
%\usepackage{parskip}    % Paragraph spacing
%\setlength{\parindent}{0em} % Don't indent paragraphs - ONLINE
%\setlength{\parskip}{1.3 ex plus 0.5ex minus 0.3ex} % 1-1.8 ex vertical space between paragraphs - ONLINE

% Spacing of headings
%\RedeclareSectionCommand[afterskip=3pt]{section} % less space after section
%\RedeclareSectionCommand[beforeskip=0cm]{subsection} % less space between HRule and project name
%\RedeclareSectionCommand[afterskip=0.1\baselineskip]{subsubsection} % less space after progressreport subheadings

% Table font size
\usepackage{etoolbox}
\AtBeginEnvironment{longtabu}{\footnotesize}{}{}

%%
%% Tables, columns, layout
%%
\usepackage{multicol}   % 2 col publications
\usepackage{pdflscape}  % Landscape pages
\usepackage{pdfpages}   % Include PDFs
\usepackage{hanging}    % hanging paragraphs for publications
%\usepackage{titletoc}   % Required for manipulating the table of contents
\setcounter{tocdepth}{2} % TOC list down to section
\usepackage{enumerate}  % Enumerations
\usepackage{enumitem}   % Enumerations
\usepackage{longtable}  % Multipage table
\usepackage{tabu}       % 
\setlength{\tabulinesep}{1.5mm} % Consistent vertical spacing in tabu

%%
%% Graphics, images, colours
%%
\usepackage{graphicx} % embedded images
\usepackage{eso-pic} % 
\usepackage{colortbl} % define custom named colours
\definecolor{RedFire}{RGB}{146,25,28}
\definecolor{ParksWildlife}{RGB}{0,85,144}
\definecolor{successbg}{RGB}{223,240,216}
\definecolor{errorbg}{RGB}{242,222,222}
\definecolor{warningbg}{RGB}{252,248,227}
\definecolor{infobg}{RGB}{217,237,247}
\definecolor{muted}{RGB}{153,153,153}
\definecolor{success}{RGB}{70,136,71}
\definecolor{error}{RGB}{185,74,72}
\definecolor{warning}{RGB}{192,152,83}
\definecolor{info}{RGB}{58,135,173}

\definecolor{required}{RGB}{192,152,83}
\definecolor{requiredbg}{RGB}{252,248,227}
\definecolor{denied}{RGB}{185,74,72}
\definecolor{deniedbg}{RGB}{242,222,222}
\definecolor{granted}{RGB}{70,136,71}
\definecolor{grantedbg}{RGB}{223,240,216}
\definecolor{not reqiured}{RGB}{153,153,153}
\definecolor{not requiredbg}{RGB}{255,255,255}

\usepackage{tikz} % Drawing
\usetikzlibrary{arrows,shapes,positioning,shadows,trees}

%%
%% Links, URLs
%%
\usepackage[
    linktoc=all,
    %colorlinks=false,  %PRINT
    colorlinks=true, % ONLINE
    linkcolor=RedFire, % ONLINE
    urlcolor=ParksWildlife, % ONLINE
    pdftitle=doc\_progressreport.pdf
]{hyperref}

% Black magic to linebreak URLs
\usepackage{url}
\makeatletter
\g@addto@macro{\UrlBreaks}{\UrlOrds}
\makeatother

%%
%% Custom macros
%%
% Thick Horizontal rule
\newcommand{\HRule}{\vspace{8mm}\\\noindent\rule{\linewidth}{0.1pt}}

% Custom Tikz node for SDS diagram
\newcommand\mynode[6][]{\node[#1] (#2){\parbox{#3\relax}{\begin{center}\textbf{#4}\\#5\\\footnotesize{#6}\end{center}}};}




%-----------------------------------------------------------------------------%
% Headers and Footers
\automark{section}
\ohead{\href{http://sdis.dpaw.wa.gov.au/documents/progressreport/1664/}{Progress Report SP 2004-003
}}
\chead{\href{http://sdis.dpaw.wa.gov.au}{SDIS}} % center header ONLINE
\ihead{\href{http://sdis.dpaw.wa.gov.au}{
        \includegraphics[scale=0.4]{/mnt/projects/sdis/staticfiles/img/logo-dpaw.png}}}
\ifoot{\textbf{Printed}~Thu, 2 Jun 2016 15:47:15 +0800} % inner/left footer
\cfoot{} % center footer
\ofoot{\pagemark} % outer/right footer
\pagestyle{scrheadings}
\setkomafont{pageheadfoot}{\normalfont}

%-----------------------------------------------------------------------------%
\begin{document}
\raggedbottom

%-----------------------------------------------------------------------------%
% Title page
\subject{Progress Report SP 2004-003
}
\title{Management of environmental risk in perennial land use systems
}
\subtitle{Ecosystem Science
}
\author{}
\publishers{\small
    \subsection*{Project Core Team}
\begin{tabu} {X X}
\textbf{Supervising Scientist} & Margaret Byrne
\\
\textbf{Data Custodian} & 
\\
\textbf{Site Custodian} & 
\\
\end{tabu}


    \subsection*{Project status as of June 2, 2016, 3:47 p.m.}
\begin{tabu} {X X}
& Update requested
\\
\end{tabu}

    
\subsection*{Document endorsements and approvals as of June 2, 2016, 3:47 p.m.}
\begin{tabu} {X X}

%\rowcolor{requiredbg}
    \textbf{Project Team} & 
    \textcolor{required}{ required}\\

%\rowcolor{requiredbg}
    \textbf{Program Leader} & 
    \textcolor{required}{ required}\\

%\rowcolor{requiredbg}
    \textbf{Directorate} & 
    \textcolor{required}{ required}\\

\end{tabu}



}
\uppertitleback{}
\lowertitleback{}
\date{}

%-----------------------------------------------------------------------------%
% Front matter
\frontmatter
\maketitle
%-----------------------------------------------------------------------------%
% Main matter
\mainmatter

\section*{Management of environmental risk in perennial land use systems
}

M Byrne, C Munday, K Bettink, J Sampson, M Millar


\section*{Context}
The development of perennial-based land use systems for management of
dryland salinity and to increase the productivity of agricultural
systems promises significant environmental and economic benefits, but
there are also risks to existing natural biodiversity. These risks
include the establishment of plant species in new locations where they
may become environmental weeds and the possible gene flow from
cultivated populations into natural populations with the potential for
hybridisation with native species. Both of these may result in a loss of
biodiversity from natural environments. Risk assessment systems can be
used to inform selection and management of agriculturally useful species
to minimise the risk to natural environments.



\section*{Aims}
\begin{itemize}
\itemsep1pt\parskip0pt\parsep0pt
\item
  Develop and implement procedures for the management of environmental
  risk in the form of assessment and management protocols to be applied
  to all germplasm under research and development within the Future Farm
  Industries Cooperative Research Centre (FFI CRC).
\item
  Disseminate information about these processes to a wide audience of
  researchers, land managers and the community via FFI CRC publications,
  national weed risk forums and conferences.
\item
  Publish weed and genetic risk assessment protocols and provide advice
  to encourage adoption of risk assessment procedures within and outside
  the FFI CRC.
\end{itemize}



\section*{Progress}
\begin{itemize}
\itemsep1pt\parskip0pt\parsep0pt
\item
  An information sheet on weed risk and the FFI CRC assessment protocol
  has been completed and published on the FFI CRC website.
\item
  New weed risk assessments have been completed for native forage
  species that may be used outside their natural range and these have
  been published online on the FFI CRC website.
\item
  A paper describing the FFI CRC environmental risk strategy for
  minimising the risk to the environment from agriculturally useful
  species was presented at the 5th Victorian Weeds Conference in
  Geelong, Victoria. The audience represented a wide range of
  organisations engaged in the control of environmental weeds from
  policy development to identification, monitoring and on-ground
  control.
\item
  The concepts of weed and genetic risk have continued to be promoted in
  FFI CRC publications and raised in forums with stakeholders in Western
  Australia and nationally.
\item
  An environmental risk strategy and framework was developed for the FFI
  CRC. All the components have been published, promoted and implemented.
  Assessments and other material prepared within this project are
  published on the Department of Parks and Wildlife website.
\item
  The weed risk assessment protocol, genetic risk assessment protocols,
  species management guides and field trial guideline have been prepared
  and published for some species promoted by the FFI CRC to inform
  management to minimise the risk to natural environments.
\item
  The environmental risk strategy, framework and its components have
  been promoted widely and a weed risk note provided for a publication
  on tropical grasses published by the FFI CRC.
\end{itemize}



\section*{Management implications}
\begin{itemize}
\itemsep1pt\parskip0pt\parsep0pt
\item
  Promotion of the concepts of weed and genetic risk management both
  within and outside the FFI CRC and the development and use of
  appropriate assessment techniques will reduce the risk of large-scale
  plantings of new perennial species introduced from outside Australia,
  native species used outside their natural range or newly developed
  cultivars becoming environmental weeds.
\item
  Adoption of the genetic risk assessment process will enable the risk
  of genetic contamination and hybridisation to be assessed on a
  site-specific basis. This will help in the development and
  implementation of processes to manage these risks. The information may
  also indicate where further research is needed to understand gene flow
  in the environment. Guidelines and risk assessment will inform species
  selection and trial and planting plans to minimise the risk of
  agriculturally useful species to native environments during research,
  breeding and production system development.
\end{itemize}



\section*{Future directions}
\begin{itemize}
\itemsep1pt\parskip0pt\parsep0pt
\item
  The project has been completed with the end of the FFI CRC.
\end{itemize}



%-----------------------------------------------------------------------------%
% Back matter
%\backmatter
\end{document}
%-----------------------------------------------------------------------------%


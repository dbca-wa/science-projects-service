
\documentclass[version=last,
    paper=a4, % paper size
    10pt, % default font size
    usenames,
    dvipsnames,
    oneside, % ONLINE
    headings=openany, % open chapters on odd and even pages
    %toc=chapterentrywithdots, % Table of Contents style
    %BCOR=7mm, % PRINT Binding Correction
    %DIV=13, % typearea 161.54 mm x 228.46 mm, top margin 22.85 mm, inner margin 16.15 mm
    %DIV=14, % 165.00 233.36 21.21 15.00
    DIV=15 % 168.00 237.60 19.80 14.00
]{scrbook}
\usepackage{typearea}
\usepackage[automark,headsepline,footsepline]{scrlayer-scrpage} % Headers and footers

%%
%% Fonts, encoding, spacing, indentation
%%
\usepackage{txfonts}
\renewcommand{\familydefault}{\sfdefault} % Default to Sans Serif font
\usepackage[english]{babel}
\usepackage[T1]{fontenc}
\usepackage[utf8]{inputenc}

% Paragraph spacing
%\usepackage{parskip}    % Paragraph spacing
%\setlength{\parindent}{0em} % Don't indent paragraphs - ONLINE
%\setlength{\parskip}{1.3 ex plus 0.5ex minus 0.3ex} % 1-1.8 ex vertical space between paragraphs - ONLINE

% Spacing of headings
%\RedeclareSectionCommand[afterskip=3pt]{section} % less space after section
%\RedeclareSectionCommand[beforeskip=0cm]{subsection} % less space between HRule and project name
%\RedeclareSectionCommand[afterskip=0.1\baselineskip]{subsubsection} % less space after progressreport subheadings

% Table font size
\usepackage{etoolbox}
\AtBeginEnvironment{longtabu}{\footnotesize}{}{}

%%
%% Tables, columns, layout
%%
\usepackage{multicol}   % 2 col publications
\usepackage{pdflscape}  % Landscape pages
\usepackage{pdfpages}   % Include PDFs
\usepackage{hanging}    % hanging paragraphs for publications
%\usepackage{titletoc}   % Required for manipulating the table of contents
\setcounter{tocdepth}{2} % TOC list down to section
\usepackage{enumerate}  % Enumerations
\usepackage{enumitem}   % Enumerations
\usepackage{longtable}  % Multipage table
\usepackage{tabu}       %
\setlength{\tabulinesep}{1.5mm} % Consistent vertical spacing in tabu

%%
%% Graphics, images, colours
%%
\usepackage{graphicx} % embedded images
\usepackage{eso-pic} %
\usepackage{colortbl} % define custom named colours
\definecolor{RedFire}{RGB}{146,25,28}
\definecolor{ParksWildlife}{RGB}{0,85,144}
\definecolor{successbg}{RGB}{223,240,216}
\definecolor{errorbg}{RGB}{242,222,222}
\definecolor{warningbg}{RGB}{252,248,227}
\definecolor{infobg}{RGB}{217,237,247}
\definecolor{muted}{RGB}{153,153,153}
\definecolor{success}{RGB}{70,136,71}
\definecolor{error}{RGB}{185,74,72}
\definecolor{warning}{RGB}{192,152,83}
\definecolor{info}{RGB}{58,135,173}

\definecolor{required}{RGB}{192,152,83}
\definecolor{requiredbg}{RGB}{252,248,227}
\definecolor{denied}{RGB}{185,74,72}
\definecolor{deniedbg}{RGB}{242,222,222}
\definecolor{granted}{RGB}{70,136,71}
\definecolor{grantedbg}{RGB}{223,240,216}
\definecolor{not reqiured}{RGB}{153,153,153}
\definecolor{not requiredbg}{RGB}{255,255,255}

\usepackage{tikz} % Drawing
\usetikzlibrary{arrows,shapes,positioning,shadows,trees}

%%
%% Links, URLs
%%
\usepackage[
    linktoc=all,
    %colorlinks=false,  %PRINT
    colorlinks=true, % ONLINE
    linkcolor=RedFire, % ONLINE
    urlcolor=ParksWildlife, % ONLINE
    pdftitle=Progress Report SP 2014-004 (FY 2015-2016)
]{hyperref}

% Black magic to linebreak URLs
\usepackage{url}
\makeatletter
\g@addto@macro{\UrlBreaks}{\UrlOrds}
\makeatother

%%
%% Custom macros
%%
% Thick Horizontal rule
\newcommand{\HRule}{\vspace{8mm}\\\noindent\rule{\linewidth}{0.1pt}}

% Custom Tikz node for SDS diagram
\newcommand\mynode[6][]{
    \node[#1] (#2){
        \parbox{#3\relax}{
            \begin{center}
            \textbf{#4}\\
            #5\\
            \footnotesize{#6}
            \end{center}}};}



%-----------------------------------------------------------------------------%
% Headers and Footers
\automark{section}
\ohead{\href{http://sdis.dpaw.wa.gov.au/documents/progressreport/1601/}{Progress Report SP 2014-004
}}
\chead{\href{http://sdis.dpaw.wa.gov.au}{SDIS}} % center header ONLINE
\ihead{\href{http://sdis.dpaw.wa.gov.au}{
        \includegraphics[scale=0.4]{/mnt/projects/sdis/staticfiles/img/logo-dpaw.png}}}
\ifoot{\textbf{Printed}~Mon, 11 Jul 2016 12:59:47 +0800} % inner/left footer
\cfoot{} % center footer
\ofoot{\pagemark} % outer/right footer
\pagestyle{scrheadings}
\setkomafont{pageheadfoot}{\normalfont}

%-----------------------------------------------------------------------------%
\begin{document}
\raggedbottom

%-----------------------------------------------------------------------------%
% Title page
\subject{Progress Report SP 2014-004
}
\title{Improving the understanding of West Pilbara marine habitats and
associated taxa: their connectivity and recovery potential following
natural and human induced disturbance
}
\subtitle{Marine Science
}
\author{}
\publishers{\small
    \subsection*{Project Core Team}
\begin{tabu} {X X}
\textbf{Supervising Scientist} & Richard D Evans
\\
\textbf{Data Custodian} & Richard D Evans
\\
\textbf{Site Custodian} & Richard D Evans
\\
\end{tabu}


    \subsection*{Project status as of July 11, 2016, 12:59 p.m.}
\begin{tabu} {X X}
& Approved and active
\\
\end{tabu}

    
\subsection*{Document endorsements and approvals as of July 11, 2016, 12:59 p.m.}
\begin{tabu} {X X}

%\rowcolor{grantedbg}
    \textbf{Project Team} & 
    \textcolor{granted}{ granted}\\

%\rowcolor{grantedbg}
    \textbf{Program Leader} & 
    \textcolor{granted}{ granted}\\

%\rowcolor{grantedbg}
    \textbf{Directorate} & 
    \textcolor{granted}{ granted}\\

\end{tabu}



}
\uppertitleback{}
\lowertitleback{}
\date{}

%-----------------------------------------------------------------------------%
% Front matter
\frontmatter
\maketitle
%-----------------------------------------------------------------------------%
% Main matter
\mainmatter

\section*{Improving the understanding of West Pilbara marine habitats and
associated taxa: their connectivity and recovery potential following
natural and human induced disturbance
}

RD Evans, S Wilson, M Byrne, R Douglas, R Binks, B Macdonald


\section*{Context}
The focus of work for Wheatstone Development Offset Project B will be to
add to the understanding of west Pilbara marine habitats (including
coral and seagrass communities) and associated taxa, including their
level of connectivity and their recovery potential should they be
impacted by natural and human induced disturbance. This research aims to
build on existing knowledge and integrate with current and proposed
connectivity projects on habitat-forming taxa and associated taxa in the
tropical north-west of Australia. Broad-scale connectivity studies of
flora and fauna within and between the offshore islands of the
north-west continental shelf have shown varying levels of connectivity.
Previous studies have also shown limited connectivity between inshore
and offshore marine communities but there have been no studies looking
at connectivity and recovery potential between locations within the
Pilbara region, and their connections with the broader inshore locations
of Ningaloo to the south-west, and the Kimberley to the north-east.



\section*{Aims}
\begin{itemize}
\itemsep1pt\parskip0pt\parsep0pt
\item
  Determine levels of population connectivity and assess the extent and
  spatial scales of local adaptation.
\item
  Correlate genetic parameters with modeling of environmental variables
  to determine factors that have a significant influence on
  connectivity.
\item
  Investigate coral demographics and recruitment to understand how the
  environment influences the corals in the Pilbara.
\end{itemize}



\section*{Progress}
\begin{itemize}
\itemsep1pt\parskip0pt\parsep0pt
\item
  Coral recruitment settlement tiles were deployed and collected in the
  Onslow region for the third year of the temporal study of recruitment
  processes.
\item
  A third in-situ assessment of recruit corals on Onslow region reefs
  was undertaken using quadrats with underwater visual census and
  digital photographs.
\item
  Pre-dredging benthic images provided by Chevron were analysed to
  understand the size-class frequency distribution of corals in the
  Onslow region.
\item
  DNA extractions for mangroves, seagrass and fish were completed.
\item
  Sequencing for fish, mangroves, and seagrass were completed and data
  analyses have commenced.
\item
  Coral extractions are still in progress with some delays caused by
  taxonomic and technical issues.
\item
  One paper on techniques to identify cryptic species of coral is in
  press with \emph{Coral Reefs.}
\item
  A paper on coral population genomics and connectivity is in review
  with \emph{Global Change Biology} and a review of dredging impacts on
  fishes has been submitted to \emph{Fish and Fisheries Review}.
\item
  A \emph{Landscope} article titled \emph{Understanding Marine
  Connectivity} is in press.
\end{itemize}



\section*{Management implications}
\begin{itemize}
\itemsep1pt\parskip0pt\parsep0pt
\item
  The project will improve our understanding of how well populations of
  marine species are linked, providing an indication on how fast they
  are likely to recover following natural and anthropogenic
  disturbances, with a focus on key habitat forming species that support
  important ecological processes.
\item
  Understanding the extent of connectivity for different taxa will
  inform marine planners about how reserves and management zones can be
  configured to best facilitate propagule transfer among meta
  populations, therefore improving recovery potential after disturbance.
\item
  Improved temporal understanding of the impact of natural and human
  disturbance in the Pilbara, as well as the demography and recovery
  potential of coral communities, will allow resource managers and
  industry to understand the resilience of the system, and allow for
  better spatial and temporal planning of developments and general use
  management zoning.
\end{itemize}



\section*{Future directions}
\begin{itemize}
\itemsep1pt\parskip0pt\parsep0pt
\item
  Laboratory work will continue for the coral connectivity study,
  including DNA extracting and sequencing, data analyses and manuscript
  preparation.
\item
  Analyses of genetic data for fish, mangroves and seagrass species will
  continue.
\item
  A manuscript for all genetic results will be prepared.
\item
  The third year of data on coral settlement will be analysed.
\item
  Analysis of benthic images from `post dredge operations' period for
  coral demographics assessment and reporting.
\item
  Analysis and prepare manuscript on~coral recovery potential in the
  Onslow region.
\item
  Coral settlement tiles will be re-deployed in February and May 2017 to
  determine settlement differentials across the period of spawning.
\end{itemize}



%-----------------------------------------------------------------------------%
% Back matter
%\backmatter
\end{document}
%-----------------------------------------------------------------------------%


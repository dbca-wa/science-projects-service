\documentclass[version=last, paper=a4, DIV=18, usenames, dvipsnames]{scrartcl}
\usepackage{txfonts}
\usepackage{pdflscape}
\usepackage{pdfpages}
\usepackage[english]{babel} % English language/hyphenation
%%% Bootstrap colors
\definecolor{RedFire}{RGB}{146,25,28}
\definecolor{ParksWildlife}{RGB}{0,85,144}
\definecolor{successbg}{RGB}{223,240,216}
\definecolor{errorbg}{RGB}{242,222,222}
\definecolor{warningbg}{RGB}{252,248,227}
\definecolor{infobg}{RGB}{217,237,247}
\definecolor{muted}{RGB}{153,153,153}
\definecolor{success}{RGB}{70,136,71}
\definecolor{error}{RGB}{185,74,72}
\definecolor{warning}{RGB}{192,152,83}
\definecolor{info}{RGB}{58,135,173}
\usepackage[colorlinks=true,pdftitle=doc\_progressreport.pdf,linktoc=all,linkcolor=RedFire,urlcolor=ParksWildlife]{hyperref}
\usepackage{colortbl}
\usepackage{longtable}
\usepackage{tabu}
\setlength{\tabulinesep}{1.5mm}
\usepackage{enumerate}
\usepackage{enumitem}
\usepackage{fancyhdr}
\usepackage{lastpage}
\usepackage{graphicx}
\usepackage{eso-pic}
\usepackage{hyphenat}



%%% Custom headers/footers (fancyhdr package)
\fancypagestyle{plain}{
\fancyhf{}
\setlength\headheight{40pt}
\renewcommand{\headrulewidth}{0.1pt}
\renewcommand{\footrulewidth}{0.1pt}



    \fancyhead[L]{ \href{http://sdis.dpaw.wa.gov.au/documents/progressreport/1164/download/}{} \newline }
\fancyhead[R]{ \hfill\href{http://www.dpaw.wa.gov.au}{Department of Parks and Wildlife}\newline\href{http://sdis.dpaw.wa.gov.au}{Pythia}}




\fancyfoot[L]{ \leftmark\newline\textbf{Last Modified}\textit{ }\quad\textbf{Printed}\textit{ June 26, 2014, 9:26 a.m. } }
\fancyfoot[R]{  \, \newline Page \thepage\ of \pageref{LastPage} } % Pagenumbering


}
\pagestyle{plain}


\newcommand{\HRule}{\rule{\linewidth}{0.1pt}}

\newcommand{\placetextbox}[3]{% \placetextbox{<horizontal pos>}{<vertical pos>}{<stuff>}
  \setbox0=\hbox{#3}% Put <stuff> in a box
  \AddToShipoutPictureFG*{% Add <stuff> to current page foreground
    \put(\LenToUnit{#1\paperwidth},\LenToUnit{#2\paperheight}){\vtop{{\null}\makebox[0pt][c]{#3}}}%
  }%
}%

\begin{document}

\setcounter{secnumdepth}{-1}


\begin{titlepage}
\begin{center}
% Upper part of the page
\begin{minipage}[t]{0.28\textwidth}
\begin{flushleft}
\href{http://www.dpaw.wa.gov.au}{\includegraphics[scale=0.6]{/var/www/sdis_8271/staticfiles/img/logo-dpaw.png}}
\end{flushleft}
\end{minipage}
\begin{minipage}[b]{0.7\textwidth}
\begin{flushright}
    \href{http://sdis.dpaw.wa.gov.au/documents/progressreport/1164/download/}{}) \\
\end{flushright}
\end{minipage}
\HRule \\[0.4cm]
\vfill
\textsc{\Huge Science project 2011-20 Long-term stand dynamics of regrowth karri forest in relation to site productivity and climate \newline }
\vfill
\textsc{\Huge Progress Report}

\vfill\vfill\vfill\vfill
title and summary

\vfill\vfill\vfill\vfill\vfill\vfill\vfill\vfill

\textbf{Version created on} June 26, 2014, 9:26 a.m.
\vfill
\textbf{Last Modified on}  by 
\vfill\vfill
\textbf{Report Status}\\\,
\begin{tabu} to \linewidth { | X[l] | X | }
\hline
\rowcolor{infobg}
Status & Last Updated \\
\hline
\textbf{Planning - } \\
\hline
\end{tabu}
\vfill
\textbf{Science Project Overview}\\\,
\begin{tabu} to \linewidth { | X[l] | X | }
\hline
\rowcolor{infobg}
Part & Checklist Last Updated \\
\hline
\textbf{Part A - Summary \& Approval} & bla \\
\hline
\end{tabu}

\end{center}
\end{titlepage}

\setcounter{tocdepth}{2}
\tableofcontents
\clearpage






\section{Context Summary}



This project provides information to underpin the management of karri in the immature stage of stand development (25-120 years old). Regenerated karri stands have important values for future timber production, biodiversity conservation and as a store of terrestrial carbon. Immature stands regenerated following timber harvesting and bushfire comprise more than 50000 ha and represent around one third of the area of karri forest managed by the department. There are a number of well-designed experiments that investigate the dynamics of naturally regenerated and planted stands managed at a range of stand densities. These experiments span a range of site productivity and climatic gradients in the karri forest, and have been measured repeatedly over a period of several decades, providing important information to support and improve management practices. This project addresses emerging issues likely to be of growing importance in the next decade, including climate change and declining groundwater levels, interactions with pests and pathogens, and increased recognition of the role of forests in maintaining global carbon cycles.






\section{Aims Summary}



To quantify the response of immature karri stands to management practices that manipulate stand density at establishment or through intervention by thinning. Responses will be measured by tree and stand growth, tree health and other indicators as appropriate (e.g. leaf water potential, leaf area index).






\section{Progress}



\begin{itemize}

  \item Findings from the Warren block thinning experiment were reported in a book on long term ecological ecological monitoring studies in Australia (Biodiversity and environmental change: CSIRO Publishing, February 2014).

\end{itemize}






\section{Management implications}



\begin{itemize}

  \item Thinning concentrates the growth potential of a site onto selected trees and provides forest managers with options to manage stands for particular structural characteristics.

  \item Tree mortality associated with \emph{Armillaria} root disease appears to reduce in older stands, and small gaps created by dead trees become less obvious as stands mature. Localised tree mortality can be regarded as a natural process and is likely to contribute to patchiness in the mature forest. However, the extent of tree mortality in silviculturally managed stands should be monitored to ensure that stand productivity and other forest values remain within acceptable ranges.

\end{itemize}






\section{Future directions}



\begin{itemize}

  \item Analyse and report on trends in tree and stand growth, with a focus on possible links between climate and growth.

  \item Analyse trends in the incidence and severity of \emph{Armillaria} root disease at Warren block since 2000.

  \item Develop a plan for a second thinning at Warren block.

\end{itemize}






\clearpage



\end{document}

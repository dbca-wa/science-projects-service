\documentclass[version=last, paper=a4, DIV=18, usenames, dvipsnames]{scrartcl}
\usepackage{txfonts}
\usepackage{pdflscape}
\usepackage{pdfpages}
\usepackage[english]{babel} % English language/hyphenation
%%% Bootstrap colors
\definecolor{RedFire}{RGB}{146,25,28}
\definecolor{ParksWildlife}{RGB}{0,85,144}
\definecolor{successbg}{RGB}{223,240,216}
\definecolor{errorbg}{RGB}{242,222,222}
\definecolor{warningbg}{RGB}{252,248,227}
\definecolor{infobg}{RGB}{217,237,247}
\definecolor{muted}{RGB}{153,153,153}
\definecolor{success}{RGB}{70,136,71}
\definecolor{error}{RGB}{185,74,72}
\definecolor{warning}{RGB}{192,152,83}
\definecolor{info}{RGB}{58,135,173}
\usepackage[colorlinks=true,pdftitle=doc\_progressreport.pdf,linktoc=all,linkcolor=RedFire,urlcolor=ParksWildlife]{hyperref}
\usepackage{colortbl}
\usepackage{longtable}
\usepackage{tabu}
\setlength{\tabulinesep}{1.5mm}
\usepackage{enumerate}
\usepackage{enumitem}
\usepackage{fancyhdr}
\usepackage{lastpage}
\usepackage{graphicx}
\usepackage{eso-pic}
\usepackage{hyphenat}



%%% Custom headers/footers (fancyhdr package)
\fancypagestyle{plain}{
\fancyhf{}
\setlength\headheight{40pt}
\renewcommand{\headrulewidth}{0.1pt}
\renewcommand{\footrulewidth}{0.1pt}



    \fancyhead[L]{ \href{http://sdis.dpaw.wa.gov.au/documents/progressreport/1142/download/}{} \newline }
\fancyhead[R]{ \hfill\href{http://www.dpaw.wa.gov.au}{Department of Parks and Wildlife}\newline\href{http://sdis.dpaw.wa.gov.au}{Pythia}}




\fancyfoot[L]{ \leftmark\newline\textbf{Last Modified}\textit{ }\quad\textbf{Printed}\textit{ June 17, 2014, 11:28 a.m. } }
\fancyfoot[R]{  \, \newline Page \thepage\ of \pageref{LastPage} } % Pagenumbering


}
\pagestyle{plain}


\newcommand{\HRule}{\rule{\linewidth}{0.1pt}}

\newcommand{\placetextbox}[3]{% \placetextbox{<horizontal pos>}{<vertical pos>}{<stuff>}
  \setbox0=\hbox{#3}% Put <stuff> in a box
  \AddToShipoutPictureFG*{% Add <stuff> to current page foreground
    \put(\LenToUnit{#1\paperwidth},\LenToUnit{#2\paperheight}){\vtop{{\null}\makebox[0pt][c]{#3}}}%
  }%
}%

\begin{document}

\setcounter{secnumdepth}{-1}


\begin{titlepage}
\begin{center}
% Upper part of the page
\begin{minipage}[t]{0.28\textwidth}
\begin{flushleft}
\href{http://www.dpaw.wa.gov.au}{\includegraphics[scale=0.6]{/var/www/sdis_8271/staticfiles/img/logo-dpaw.png}}
\end{flushleft}
\end{minipage}
\begin{minipage}[b]{0.7\textwidth}
\begin{flushright}
    \href{http://sdis.dpaw.wa.gov.au/documents/progressreport/1142/download/}{}) \\
\end{flushright}
\end{minipage}
\HRule \\[0.4cm]
\vfill
\textsc{\Huge Science project 2013-5 Improving the use of remote cameras as a survey and monitoring tool \newline }
\vfill
\textsc{\Huge Progress Report}

\vfill\vfill\vfill\vfill
title and summary

\vfill\vfill\vfill\vfill\vfill\vfill\vfill\vfill

\textbf{Version created on} June 17, 2014, 11:28 a.m.
\vfill
\textbf{Last Modified on}  by 
\vfill\vfill
\textbf{Report Status}\\\,
\begin{tabu} to \linewidth { | X[l] | X | }
\hline
\rowcolor{infobg}
Status & Last Updated \\
\hline
\textbf{Planning - } \\
\hline
\end{tabu}
\vfill
\textbf{Science Project Overview}\\\,
\begin{tabu} to \linewidth { | X[l] | X | }
\hline
\rowcolor{infobg}
Part & Checklist Last Updated \\
\hline
\textbf{Part A - Summary \& Approval} & bla \\
\hline
\end{tabu}

\end{center}
\end{titlepage}

\setcounter{tocdepth}{2}
\tableofcontents
\clearpage






\section{Context Summary}



The use of remote cameras is often regarded as an effective tool for fauna survey and monitoring with the assumption that they provide high quality, cost effective data.  However, our understanding of appropriate methods for general survey and species detection, particularly in the small to medium sized range of mammals, remains poorly understood.  Within DEC use of remote cameras to date has usually been restricted to simple species inventories or behavioural studies and beyond this there has been little assessment of deployment methods or appropriate analytical techniques. This has sometimes led to erroneous conclusions being derived from captured images. Camera traps have the potential to offer a comparatively reliable and relatively unbiased method for monitoring medium to large native and introduced mammal species throughout the state, including a number of significant cryptic species that are currently not incorporated under the Western Shield fauna monitoring program. However, research is required to validate and test different survey designs (temporal and spatial components) and methods of deploying camera traps, and to interpret the results in a more meaningful way. In particular, work is needed to determine how best to use remote cameras to provide rigorous data on species detectability, and species richness and density.






\section{Aims Summary}



\begin{itemize}

  \item Establish suitable methods for estimating the presence and relative abundances of native and introduced mammals species in the south-west of Western Australia.

  \item Determine the amount of deployment time required to accurately determine mammal species richness within Dryandra Woodland.

  \item Improve and standardise use of remote cameras within DEC.

  \item Investigate the effectiveness of baited (active) and unbaited (passive) cameras sets to  inventory targeted species.

  \item Compare the detection rates of different makes and models of camera traps.

  \item Investigate and assess the most appropriate methods of image analysis and data storage.

  \item Establish the minimum spatial distance required between camera traps to ensure independence of detections.

\end{itemize}






\section{Progress}



\begin{itemize}

  \item An initial trial of baited (active) verses unbaited (passive) camera traps was completed.  This indicated that baited camera traps are effective in attracting certain species but these species subsequently dominate the cameras and actively exclude other species, resulting in biased data.

  \item A preliminary comparison of a select number of makes and model of camera traps showed that Reconyx cameras were the most suitable for research and operational purposes.

  \item Camera traps at Dryandra Woodland detected a number of threatened species that were either not monitored at all, or were unreliably detected, through conventional Western Shield monitoring programs.

  \item Several different data capture methods were assessed, including commercial image processing software, manual data entry, using an open source database, and an application designed specifically for camera-trap data capture. The most appropriate system for DEC use has been determined.

\end{itemize}






\section{Management implications}



Camera traps appear to be an effective tool in detecting a suite of species currently not adequately monitored by the Western Shield monitoring program. Their use should be considered in the Western Shield monitoring program, either to complement the trapping program, or as an separate fauna monitoring tool.






\section{Future directions}



\begin{itemize}

  \item Validate camera traps against other traditional methods of fauna monitoring, such as cage trapping or sand plots.

  \item Investigate methods to use camera traps to qualitatively and quantitatively monitor invasive species.

  \item Investigate how sensitive camera-trap data are to detecting changes in relative abundance and occupancy of targeted species over time.

\end{itemize}






\clearpage



\end{document}

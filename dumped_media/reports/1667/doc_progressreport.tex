
\documentclass[version=last,
    paper=a4, % paper size
    10pt, % default font size
    usenames,
    dvipsnames,
    oneside, % ONLINE
    headings=openany, % open chapters on odd and even pages
    %toc=chapterentrywithdots, % Table of Contents style
    %BCOR=7mm, % PRINT Binding Correction
    %DIV=13, % typearea 161.54 mm x 228.46 mm, top margin 22.85 mm, inner margin 16.15 mm
    %DIV=14, % 165.00 233.36 21.21 15.00
    DIV=15 % 168.00 237.60 19.80 14.00
]{scrbook}
\usepackage{typearea}
\usepackage[automark,headsepline,footsepline]{scrlayer-scrpage} % Headers and footers

%%
%% Fonts, encoding, spacing, indentation
%%
\usepackage{txfonts}
\renewcommand{\familydefault}{\sfdefault} % Default to Sans Serif font
\usepackage[english]{babel}
\usepackage[T1]{fontenc}
\usepackage[utf8]{inputenc}

% Paragraph spacing
%\usepackage{parskip}    % Paragraph spacing
%\setlength{\parindent}{0em} % Don't indent paragraphs - ONLINE
%\setlength{\parskip}{1.3 ex plus 0.5ex minus 0.3ex} % 1-1.8 ex vertical space between paragraphs - ONLINE

% Spacing of headings
%\RedeclareSectionCommand[afterskip=3pt]{section} % less space after section
%\RedeclareSectionCommand[beforeskip=0cm]{subsection} % less space between HRule and project name
%\RedeclareSectionCommand[afterskip=0.1\baselineskip]{subsubsection} % less space after progressreport subheadings

% Table font size
\usepackage{etoolbox}
\AtBeginEnvironment{longtabu}{\footnotesize}{}{}

%%
%% Tables, columns, layout
%%
\usepackage{multicol}   % 2 col publications
\usepackage{pdflscape}  % Landscape pages
\usepackage{pdfpages}   % Include PDFs
\usepackage{hanging}    % hanging paragraphs for publications
%\usepackage{titletoc}   % Required for manipulating the table of contents
\setcounter{tocdepth}{2} % TOC list down to section
\usepackage{enumerate}  % Enumerations
\usepackage{enumitem}   % Enumerations
\usepackage{longtable}  % Multipage table
\usepackage{tabu}       %
\setlength{\tabulinesep}{1.5mm} % Consistent vertical spacing in tabu

%%
%% Graphics, images, colours
%%
\usepackage{graphicx} % embedded images
\usepackage{eso-pic} %
\usepackage{colortbl} % define custom named colours
\definecolor{RedFire}{RGB}{146,25,28}
\definecolor{ParksWildlife}{RGB}{0,85,144}
\definecolor{successbg}{RGB}{223,240,216}
\definecolor{errorbg}{RGB}{242,222,222}
\definecolor{warningbg}{RGB}{252,248,227}
\definecolor{infobg}{RGB}{217,237,247}
\definecolor{muted}{RGB}{153,153,153}
\definecolor{success}{RGB}{70,136,71}
\definecolor{error}{RGB}{185,74,72}
\definecolor{warning}{RGB}{192,152,83}
\definecolor{info}{RGB}{58,135,173}

\definecolor{required}{RGB}{192,152,83}
\definecolor{requiredbg}{RGB}{252,248,227}
\definecolor{denied}{RGB}{185,74,72}
\definecolor{deniedbg}{RGB}{242,222,222}
\definecolor{granted}{RGB}{70,136,71}
\definecolor{grantedbg}{RGB}{223,240,216}
\definecolor{not reqiured}{RGB}{153,153,153}
\definecolor{not requiredbg}{RGB}{255,255,255}

\usepackage{tikz} % Drawing
\usetikzlibrary{arrows,shapes,positioning,shadows,trees}

%%
%% Links, URLs
%%
\usepackage[
    linktoc=all,
    %colorlinks=false,  %PRINT
    colorlinks=true, % ONLINE
    linkcolor=RedFire, % ONLINE
    urlcolor=ParksWildlife, % ONLINE
    pdftitle=Progress Report SP 2003-005 (FY 2015-2016)
]{hyperref}

% Black magic to linebreak URLs
\usepackage{url}
\makeatletter
\g@addto@macro{\UrlBreaks}{\UrlOrds}
\makeatother

%%
%% Custom macros
%%
% Thick Horizontal rule
\newcommand{\HRule}{\vspace{8mm}\\\noindent\rule{\linewidth}{0.1pt}}

% Custom Tikz node for SDS diagram
\newcommand\mynode[6][]{
    \node[#1] (#2){
        \parbox{#3\relax}{
            \begin{center}
            \textbf{#4}\\
            #5\\
            \footnotesize{#6}
            \end{center}}};}



%-----------------------------------------------------------------------------%
% Headers and Footers
\automark{section}
\ohead{\href{http://sdis.dpaw.wa.gov.au/documents/progressreport/1667/}{Progress Report SP 2003-005
}}
\chead{\href{http://sdis.dpaw.wa.gov.au}{SDIS}} % center header ONLINE
\ihead{\href{http://sdis.dpaw.wa.gov.au}{
        \includegraphics[scale=0.4]{/mnt/projects/sdis/staticfiles/img/logo-dpaw.png}}}
\ifoot{\textbf{Printed}~Tue, 5 Jul 2016 13:56:49 +0800} % inner/left footer
\cfoot{} % center footer
\ofoot{\pagemark} % outer/right footer
\pagestyle{scrheadings}
\setkomafont{pageheadfoot}{\normalfont}

%-----------------------------------------------------------------------------%
\begin{document}
\raggedbottom

%-----------------------------------------------------------------------------%
% Title page
\subject{Progress Report SP 2003-005
}
\title{Development of effective broad-scale aerial baiting strategies for the
control of feral cats
}
\subtitle{Animal Science
}
\author{}
\publishers{\small
    \subsection*{Project Core Team}
\begin{tabu} {X X}
\textbf{Supervising Scientist} & Dave Algar
\\
\textbf{Data Custodian} & 
\\
\textbf{Site Custodian} & 
\\
\end{tabu}


    \subsection*{Project status as of July 5, 2016, 1:56 p.m.}
\begin{tabu} {X X}
& Approved and active
\\
\end{tabu}

    
\subsection*{Document endorsements and approvals as of July 5, 2016, 1:56 p.m.}
\begin{tabu} {X X}

%\rowcolor{grantedbg}
    \textbf{Project Team} & 
    \textcolor{granted}{ granted}\\

%\rowcolor{grantedbg}
    \textbf{Program Leader} & 
    \textcolor{granted}{ granted}\\

%\rowcolor{grantedbg}
    \textbf{Directorate} & 
    \textcolor{granted}{ granted}\\

\end{tabu}



}
\uppertitleback{}
\lowertitleback{}
\date{}

%-----------------------------------------------------------------------------%
% Front matter
\frontmatter
\maketitle
%-----------------------------------------------------------------------------%
% Main matter
\mainmatter

\section*{Development of effective broad-scale aerial baiting strategies for the
control of feral cats
}

D Algar, N Hamilton


\section*{Context}
The effective control of feral cats is one of the most important native
fauna conservation issues in Australia. Development of an effective
broad-scale baiting technique, and the incorporation of a suitable toxin
for feral cats, is cited as a high priority in the National Threat
Abatement Plan for Predation of Feral Cats, as it is most likely to
yield a practical, cost-effective method to control feral cat numbers in
strategic areas and promote the recovery of threatened fauna.



\section*{Aims}
\begin{itemize}
\itemsep1pt\parskip0pt\parsep0pt
\item
  Design and develop a bait medium that is readily consumed by feral
  cats.
\item
  Examine bait uptake in relation to the time of year, to enable baiting
  programs to be conducted when bait uptake is at its peak and therefore
  maximise efficiency.
\item
  Examine baiting intensity in relation to baiting efficiency to
  optimise control.
\item
  Examine baiting frequency required to provide long-term and sustained
  effective control.
\item
  Assess the potential impact of baiting programs on non-target species
  and populations and devise methods to reduce the potential risk where
  possible.
\item
  Provide a technique for the reliable estimation of cat abundance.
\end{itemize}



\section*{Progress}
\begin{itemize}
\itemsep1pt\parskip0pt\parsep0pt
\item
  Research into bait composition continues with the objective of further
  improving bait uptake. Chemical synthesis of several compounds that
  elicit a chewing response by cats has been achieved. One of these
  compounds is currently being assessed in bait uptake trials. Bait
  production is being reviewed with the objective of further improving
  bait palatability~and longevity in the field.~
\item
  Feral cat baiting programs on the Fortescue Marsh (Pilbara) has been
  conducted yearly since 2012. All campaigns have resulted in
  statistically significant declines in cat occupancy rates in the
  baiting area.
\item
  Research into the effectiveness of baiting strategies is continuing to
  be assessed under the temperate climatic conditions of the south-west
  at sites including Cape Arid and Fitzgerald River National Parks. The
  baiting programs fortuitously conducted prior to the Cape Arid
  National Park wildfire in November 2015 contributed to an apparent
  stabilisation in the critically endangered western ground parrot
  population and significant population increases in number of other
  species, including the southern brown bandicoot. Similar results have
  been achieved at Fitzgerald River National Park where anecdotal
  increases in a number of native bird and mammal species have been
  observed.
\item
  Stage 1 of the management plan for the control of cats on Christmas
  Island has been completed with all domestic cats now desexed,
  microchipped and registered. Stage 2 of the plan is continuing and
  involves the removal of all stray/feral cats from residential areas
  and surrounds. Stage 3 of the plan - island-wide eradication of feral
  cats commenced in 2015 and control effort continue for the next two
  years prior to a surveillance period to confirm eradication success.
\item
  Work continues on improving and refining cat lure options.
\end{itemize}



\section*{Management implications}
\begin{itemize}
\itemsep1pt\parskip0pt\parsep0pt
\item
  Development of effective baiting methods across climatic regions will
  ultimately provide efficient feral cat control at strategic locations
  across the state and lead to conservation benefits.
\item
  Successful eradication of feral cats from a number of islands off the
  Western Australian mainland has occurred over the past ten years (i.e.
  Hermite, Faure and Rottnest islands), allowing the persistence of the
  native fauna on these islands and enabling effective reintroductions
  of mammals where appropriate. Eradication of cats on Dirk Hartog
  Island and Christmas Island will significantly add to the conservation
  of biodiversity in Western Australia.
\end{itemize}



\section*{Future directions}
\begin{itemize}
\itemsep1pt\parskip0pt\parsep0pt
\item
  Continue refinement of bait medium to improve bait consumption by
  feral cats.
\item
  Analyse baiting effectiveness at the various research sites and refine
  the method of operation where necessary to optimise baiting efficacy.
\item
  Further investigation of bait consumption by non-target species and
  devise methods to minimise risk (eg. toxin encapsulation).
\item
  Refine and optimise cat lure options.
\end{itemize}



%-----------------------------------------------------------------------------%
% Back matter
%\backmatter
\end{document}
%-----------------------------------------------------------------------------%


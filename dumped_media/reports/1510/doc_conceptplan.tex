
\documentclass[version=last, paper=a4, DIV=18, usenames, dvipsnames]{scrartcl}
\usepackage{txfonts}
\usepackage{pdflscape}
\usepackage{pdfpages}
\usepackage[english]{babel} % English language/hyphenation
%%% Bootstrap colors
\definecolor{RedFire}{RGB}{146,25,28}
\definecolor{ParksWildlife}{RGB}{0,85,144}
\definecolor{successbg}{RGB}{223,240,216}
\definecolor{errorbg}{RGB}{242,222,222}
\definecolor{warningbg}{RGB}{252,248,227}
\definecolor{infobg}{RGB}{217,237,247}
\definecolor{muted}{RGB}{153,153,153}
\definecolor{success}{RGB}{70,136,71}
\definecolor{error}{RGB}{185,74,72}
\definecolor{warning}{RGB}{192,152,83}
\definecolor{info}{RGB}{58,135,173}

\definecolor{required}{HTML}{D9534F}
\definecolor{denied}{HTML}{D9534F}
\definecolor{granted}{HTML}{47A447}
\definecolor{not required}{RGB}{200, 200, 200}

\usepackage[colorlinks=true,pdftitle=doc\_conceptplan.pdf
,linktoc=all,linkcolor=RedFire,urlcolor=ParksWildlife]{hyperref}
\usepackage{colortbl}
\usepackage{longtable}
\usepackage{tabu}
\setlength{\tabulinesep}{1.5mm}
\usepackage{enumerate}
\usepackage{enumitem}
\usepackage{fancyhdr}
\usepackage{lastpage}
\usepackage{graphicx}
\usepackage{eso-pic}
\usepackage{hyphenat}
\renewcommand{\familydefault}{\sfdefault}



\newcommand{\HRule}{\rule{\linewidth}{0.1pt}}

\newcommand{\placetextbox}[3]{% \placetextbox{<horizontal pos>}{<vertical pos>}{<stuff>}
  \setbox0=\hbox{#3}% Put <stuff> in a box
  \AddToShipoutPictureFG*{% Add <stuff> to current page foreground
    \put(\LenToUnit{#1\paperwidth},\LenToUnit{#2\paperheight}){\vtop{{\null}\makebox[0pt][c]{#3}}}%
  }%
}%




%-----------------------------------------------------------------------------%
% Headers and footers
%
\fancypagestyle{plain}{
  \fancyhf{}
  \setlength\headheight{60pt} % push page content below header
  \renewcommand{\headrulewidth}{0.1pt}
  \renewcommand{\footrulewidth}{0.1pt}
  
  
  \fancyhead[L]{ 
    \href{http://sdis.dpaw.wa.gov.au}{
    \includegraphics[scale=0.6]{/mnt/projects/sdis/staticfiles/img/logo-dpaw.png}}
  }
  \fancyhead[R]{ 
      \hfill
      \href{http://sdis.dpaw.wa.gov.au}{Science Directorate Information System} 
      \newline 
      \href{http://sdis.dpaw.wa.gov.au/documents/conceptplan/1510/}{Concept Plan 2015-017} 
  }
  
  
  
  
  \fancyfoot[L]{ \leftmark\newline\textbf{Printed}\textit{ Sept. 21, 2015, 11:16 a.m. }}
  \fancyfoot[R]{  \, \newline Page \thepage\ of \pageref{LastPage} }
  
  
}
\pagestyle{plain}
%
% end Headers
%-----------------------------------------------------------------------------%

\begin{document}

%-----------------------------------------------------------------------------%
% Title page
%

%
% end title page
%-----------------------------------------------------------------------------%



\section*{Concept Plan 2015-017}

\subsection*{Project title}
Responses of aquatic invertebrate communities to changing hydrology and
water quality in streams and significant wetlands of the south-west
forests of Western Australia.



\subsection*{Science and Conservation Division Program}
Wetland Conservation



\subsection*{Parks and Wildlife Service}
Service 4: Forest Management Plan Implementation



\subsection*{Background and Aims}
Aquatic habitats in the south-west of WA are under increasing threat
from changes in hydrology, water quality and fire as a result of the
drying climate and historical and current land use. At present, there is
an inadequate understanding of the responses of aquatic communities to
these threats to inform the management of many aquatic systems in the
Forest Management Plan area, including the Muir-Byenup Ramsar wetlands.

This project has two components:

\emph{Re-surveys of aquatic invertebrates in Muir-Byenup Ramsar wetlands
sampled in 1994 and 2004 and suites of wetlands further south sampled in
1993. This addresses KPI3 of the 2014-23 FMP.}

The FMP area has many high value wetlands, particularly in the Warren
region. Some of these are listed as nationally or internationally
significant and some are priorities in regional nature conservation
plans. These support numerous priority flora species, priority
ecological communities, significant waterbirds, 6 of the 8 species of
south-west endemic fish and a very high diversity and endemicity of
invertebrates. Threats to many of these wetlands have intensified over
the last decade. The available biodiversity data is 10 to 20 year old
and up to date information is required to assess responses to threats
and inform the allocation of resources to management actions.

\emph{Continued monitoring of high condition streams, with a focus on
effects of the drying climate and forest management. This addresses KPI1
of the 2014-23 FMP.}

KPI20 of the previous FMP scored 24 of 51 monitored stream sites as
impaired. This was not clearly related to forestry activities but could,
in part, be related to reduced rainfall. This project would see
continued monitoring at `reference condition' streams and those that
already affected by reduced rainfall. A focus on these streams aligns
with KPI1 of the current FMP which focuses on change in `currently
healthy ecosystems' and will allow us to track condition in relation to
the ongoing decline in rainfall combined with forest management. In a
region with high climatic variability long-term studies are essential to
understand ecosystem responses.

\emph{Aims:}

\begin{itemize}
\item
  To address KPI1 of the 2014-2023 FMP by monitoring the condition of
  currently healthy streams in relation to reduced rainfall and forest
  management practices.
\item
  To address KPI3 of the 2014-2023 FMP by determining responses of
  faunas of high value Warren region wetlands to changes in hydrology,
  water chemistry and fire over the last 10 to 20 years.
\item
  Provide baseline data for some internationally significant wetlands,
  e.g. Lake Muir.
\item
  Use the above information to report on the current conservation
  significance of key DPaW managed wetlands and their response and
  vulnerability to threats.
\end{itemize}



\subsection*{Expected outcome}
\begin{enumerate}
\item
  FMP commitments met with regard to measuring and assessing change in
  condition of 1) currently healthy (reference condition) stream
  ecosystems (KPI1) and 2) Ramsar and nationally listed wetlands (KPI3).
\item
  Warren Region conservation managers will have the information needed
  to address a priority identified in the 2009-14 Warren Region Nature
  Conservation Plan: Target 5, candidate action 1, including the
  milestones:

  ``\emph{Analyse condition trends} {[}of 7 nationally listed
  wetlands{]} \emph{and update adaptive management targets on the basis
  of these trends}'' and

  ``\emph{Establish and consolidate benchmark information for Broke
  Inlet, Doggerup, Marringup, Mt Soho Swamp, and Byenup to determine
  condition, identify threats and to determine interim management
  actions.''}
\item
  DPaW will be able to report on the condition of a significant Ramsar
  site.
\end{enumerate}



\subsection*{Strategic context}
\textbf{Forest Management Plan 2014-23}: Addresses Key Performance
Indicators 1 and 3.

\textbf{DPaW Strategic Directions 2014-17}

\begin{itemize}
\item
  \textbf{Implementation of the Forest Management Plan}: Key FMP
  requirements met.
\item
  \textbf{Integrated forest and ecosystem management}: Focus resources
  on highest priority ecosystem management requirements.
\item
  \textbf{Integrated science and nature conservation}: Focus
  conservation science on management priorities: Ensure conservation
  management is based on best practice science.
\end{itemize}

\textbf{Science and Conservation Division Strategic Plan 2014-17}

\begin{itemize}
\item
  \textbf{Integrated science and conservation}: Ensure that science
  programs address the gaps in knowledge and reflect the applied nature
  of advice required by the Department to deliver effective
  conservation, protection and management of flora, fauna, ecological
  communities and conservation reserves.
\item
  \textbf{Understand management options to enhance biodiversity
  resilience in a changing climate}: Undertake research and identify
  external science that provides a basis for informed management of
  climate change impacts on biodiversity.
\item
  \textbf{Monitoring of forest ecosystems}: Undertake research and
  monitoring to support ecologically sustainable forest management.
\end{itemize}

\textbf{Wildlife Management Service Strategic Priorities 2014-2017}:
WM-4-A (FMP), WM-5-A+B.

\textbf{Warren Region Nature Conservation Plan (2009-2014)}: Targets
1,5,6,8.



\subsection*{Expected collaborations}
Expert taxonomists, such as Russell Shiel (University of Adelaide) and
DPaW Regional and District staff


\subsection*{Proposed period of the project}
July 1, 2014 -- June 30, 2023

\subsection*{Staff time allocation }



\begin{longtabu} to \linewidth { |  X | X | X | X | }
\hline
\rowcolor{infobg}
Role & Year 1 & Year 2 & Year 3\\
\hline
\endhead



Scientist & 0.5 & 0.5 & 0.5\\



Technical & 0.3 & 0.3 & 0.3\\



Volunteer &  &  & \\



Collaborator &  &  & \\


\hline
\end{longtabu}



\subsection*{Indicative operating budget }



\begin{longtabu} to \linewidth { |  X | X | X | X | }
\hline
\rowcolor{infobg}
Source & Year 1 & Year 2 & Year 3\\
\hline
\endhead



Consolidated Funds (DPaW) & 100,000 & 100,000 & 100,000\\



External Funding & 0 & 0 & 0\\


\hline
\end{longtabu}





\subsection*{Endorsements}
Endorsements and approvals as of Sept. 21, 2015, 11:16 a.m.:\\
\begin{tabu} {| X | X |}
\hline

\rowcolor{granted}
Project Team & granted\\
\hline

\rowcolor{granted}
Program Leader & granted\\
\hline

\rowcolor{granted}
Directorate & granted\\
\hline

\end{tabu}





\end{document}


\documentclass[version=last,
    paper=a4, % paper size
    10pt, % default font size
    usenames,
    dvipsnames,
    oneside, % ONLINE
    headings=openany, % open chapters on odd and even pages
    %toc=chapterentrywithdots, % Table of Contents style
    %BCOR=7mm, % PRINT Binding Correction
    %DIV=13, % typearea 161.54 mm x 228.46 mm, top margin 22.85 mm, inner margin 16.15 mm
    %DIV=14, % 165.00 233.36 21.21 15.00
    DIV=15 % 168.00 237.60 19.80 14.00
]{scrbook}
\usepackage{typearea}
\usepackage[automark,headsepline,footsepline]{scrlayer-scrpage} % Headers and footers

%%
%% Fonts, encoding, spacing, indentation
%%
\usepackage{txfonts}
\renewcommand{\familydefault}{\sfdefault} % Default to Sans Serif font
\usepackage[english]{babel}
\usepackage[T1]{fontenc}
\usepackage[utf8]{inputenc}

% Paragraph spacing
%\usepackage{parskip}    % Paragraph spacing
%\setlength{\parindent}{0em} % Don't indent paragraphs - ONLINE
%\setlength{\parskip}{1.3 ex plus 0.5ex minus 0.3ex} % 1-1.8 ex vertical space between paragraphs - ONLINE

% Spacing of headings
%\RedeclareSectionCommand[afterskip=3pt]{section} % less space after section
%\RedeclareSectionCommand[beforeskip=0cm]{subsection} % less space between HRule and project name
%\RedeclareSectionCommand[afterskip=0.1\baselineskip]{subsubsection} % less space after progressreport subheadings

% Table font size
\usepackage{etoolbox}
\AtBeginEnvironment{longtabu}{\footnotesize}{}{}

%%
%% Tables, columns, layout
%%
\usepackage{multicol}   % 2 col publications
\usepackage{pdflscape}  % Landscape pages
\usepackage{pdfpages}   % Include PDFs
\usepackage{hanging}    % hanging paragraphs for publications
%\usepackage{titletoc}   % Required for manipulating the table of contents
\setcounter{tocdepth}{2} % TOC list down to section
\usepackage{enumerate}  % Enumerations
\usepackage{enumitem}   % Enumerations
\usepackage{longtable}  % Multipage table
\usepackage{tabu}       %
\setlength{\tabulinesep}{1.5mm} % Consistent vertical spacing in tabu

%%
%% Graphics, images, colours
%%
\usepackage{graphicx} % embedded images
\usepackage{eso-pic} %
\usepackage{colortbl} % define custom named colours
\definecolor{RedFire}{RGB}{146,25,28}
\definecolor{ParksWildlife}{RGB}{0,85,144}
\definecolor{successbg}{RGB}{223,240,216}
\definecolor{errorbg}{RGB}{242,222,222}
\definecolor{warningbg}{RGB}{252,248,227}
\definecolor{infobg}{RGB}{217,237,247}
\definecolor{muted}{RGB}{153,153,153}
\definecolor{success}{RGB}{70,136,71}
\definecolor{error}{RGB}{185,74,72}
\definecolor{warning}{RGB}{192,152,83}
\definecolor{info}{RGB}{58,135,173}

\definecolor{required}{RGB}{192,152,83}
\definecolor{requiredbg}{RGB}{252,248,227}
\definecolor{denied}{RGB}{185,74,72}
\definecolor{deniedbg}{RGB}{242,222,222}
\definecolor{granted}{RGB}{70,136,71}
\definecolor{grantedbg}{RGB}{223,240,216}
\definecolor{not reqiured}{RGB}{153,153,153}
\definecolor{not requiredbg}{RGB}{255,255,255}

\usepackage{tikz} % Drawing
\usetikzlibrary{arrows,shapes,positioning,shadows,trees}

%%
%% Links, URLs
%%
\usepackage[
    linktoc=all,
    %colorlinks=false,  %PRINT
    colorlinks=true, % ONLINE
    linkcolor=RedFire, % ONLINE
    urlcolor=ParksWildlife, % ONLINE
    pdftitle=Progress Report SP 2014-005 (FY 2015-2016)
]{hyperref}

% Black magic to linebreak URLs
\usepackage{url}
\makeatletter
\g@addto@macro{\UrlBreaks}{\UrlOrds}
\makeatother

%%
%% Custom macros
%%
% Thick Horizontal rule
\newcommand{\HRule}{\vspace{8mm}\\\noindent\rule{\linewidth}{0.1pt}}

% Custom Tikz node for SDS diagram
\newcommand\mynode[6][]{
    \node[#1] (#2){
        \parbox{#3\relax}{
            \begin{center}
            \textbf{#4}\\
            #5\\
            \footnotesize{#6}
            \end{center}}};}



%-----------------------------------------------------------------------------%
% Headers and Footers
\automark{section}
\ohead{\href{http://sdis.dpaw.wa.gov.au/documents/progressreport/1600/}{Progress Report SP 2014-005
}}
\chead{\href{http://sdis.dpaw.wa.gov.au}{SDIS}} % center header ONLINE
\ihead{\href{http://sdis.dpaw.wa.gov.au}{
        \includegraphics[scale=0.4]{/mnt/projects/sdis/staticfiles/img/logo-dpaw.png}}}
\ifoot{\textbf{Printed}~Mon, 11 Jul 2016 12:53:39 +0800} % inner/left footer
\cfoot{} % center footer
\ofoot{\pagemark} % outer/right footer
\pagestyle{scrheadings}
\setkomafont{pageheadfoot}{\normalfont}

%-----------------------------------------------------------------------------%
\begin{document}
\raggedbottom

%-----------------------------------------------------------------------------%
% Title page
\subject{Progress Report SP 2014-005
}
\title{Access and human use at Penguin Island and related implications for
management of Marine Park assets and visitor risk
}
\subtitle{Marine Science
}
\author{}
\publishers{\small
    \subsection*{Project Core Team}
\begin{tabu} {X X}
\textbf{Supervising Scientist} & George Shedrawi
\\
\textbf{Data Custodian} & George Shedrawi
\\
\textbf{Site Custodian} & George Shedrawi
\\
\end{tabu}


    \subsection*{Project status as of July 11, 2016, 12:53 p.m.}
\begin{tabu} {X X}
& Approved and active
\\
\end{tabu}

    
\subsection*{Document endorsements and approvals as of July 11, 2016, 12:53 p.m.}
\begin{tabu} {X X}

%\rowcolor{grantedbg}
    \textbf{Project Team} & 
    \textcolor{granted}{ granted}\\

%\rowcolor{grantedbg}
    \textbf{Program Leader} & 
    \textcolor{granted}{ granted}\\

%\rowcolor{grantedbg}
    \textbf{Directorate} & 
    \textcolor{granted}{ granted}\\

\end{tabu}



}
\uppertitleback{}
\lowertitleback{}
\date{}

%-----------------------------------------------------------------------------%
% Front matter
\frontmatter
\maketitle
%-----------------------------------------------------------------------------%
% Main matter
\mainmatter

\section*{Access and human use at Penguin Island and related implications for
management of Marine Park assets and visitor risk
}

G Shedrawi, A Kendrick, M Rule


\section*{Context}
Penguin Island is part of the Shoalwater Islands Marine Park and is the
most northern significant breeding location for little
penguins,~\emph{Eudyptula minor,} in Wester Australia. The presence of
migrating and resident seabirds and the unspoilt beaches makes Penguin
Island an important seabird breeding colony and an attractive
destination for residents and tourists, who generally access the island
by ferry. A number of visitors choose to wade or swim to the island and
Parks and Wildlife managers have identified this activity as a
significant risk to visitor safety. Historically, such methods of
crossing have resulted in near drownings that required Department staff
to rescue people in the water and more recently, a drowning incident.
Parks and Wildlife are continuing to implement a range of management
strategies to mitigate this risk including recommendations from the
Coroners working group. This project has been established at the request
from Swan Coastal District to develop a better understanding of visitor
crossings to Penguin Island by visitors, thus providing managers with
relevant information for the design of mitigation strategies and
actions.~



\section*{Aims}
\begin{itemize}
\itemsep1pt\parskip0pt\parsep0pt
\item
  Determine social and environmental factors that characterise periods
  of high use of the sand bar crossing.
\item
  Provide information to assist in the development and implementation of
  new and existing mitigation strategies that minimises visitor risk.
\item
  Establish a system for recording beach arrivals of nesting little
  penguins, and high risk crossings by visitors to Penguin Island.
\end{itemize}



\section*{Progress}
\begin{itemize}
\itemsep1pt\parskip0pt\parsep0pt
\item
  Installed cameras and a web-based application that enabled Parks and
  Wildlife staff~to view near real-time video footage of people crossing
  the sandbar spit between Mersey Point and Penguin Island. This system
  includes the automated conversion, upload and corporate storage of
  video footage of both little penguin beach arrivals and visitor
  sandbar crossings.
\item
  Data collected during 2014-15 were analysed to determine the
  proportion of Penguin Island visitors using the sandbar as an access
  route instead of the ferry service. This identified peak periods and
  environmental conditions when visitors used the sandbar as an access
  point to Penguin Island.
\item
  Data and preliminary findings collected during 2014-15 were summarised
  and presented to Swan Coastal District and the Coroner's working
  group.~
\item
  A Department-hosted web application was implemented that enabled
  staff~and community volunteers to analyse infrared recordings~of
  little penguin beach arrivals.
\end{itemize}



\section*{Management implications}
The monitoring tool provides managers with information to assist the
development of visitor risk mitigation strategies aimed at
decreasing~high risk crossings to Penguin Island. The preliminary
information provided to managers indicates that management intervention
strategies such as sandbar closures and signage may lower the
proportion~of visitors using~the sandbar to access Penguin Island and
indicates the relative value of patrols~by Surf Life Saving Association
life guards.~

Marine park managers now have an improved understanding of the numbers
of little penguins landing on two major beaches during the breeding
season which can be used to document long-term trends in abundance. This
information is used to adapt on-ground management and to support
reporting and community education.



\section*{Future directions}
\begin{itemize}
\itemsep1pt\parskip0pt\parsep0pt
\item
  Compile a final report on finding related to visitor use of the
  Penguin Island sandbar.
\item
  Improve the camera~system's utility as a monitoring tool by increasing
  the resolution of~imagery.
\item
  Research~further options to promote community engagement with little
  penguin monitoring.
\item
  Integrate this technology and methods into the Department's long-term
  marine monitoring program for little penguins
\end{itemize}



%-----------------------------------------------------------------------------%
% Back matter
%\backmatter
\end{document}
%-----------------------------------------------------------------------------%


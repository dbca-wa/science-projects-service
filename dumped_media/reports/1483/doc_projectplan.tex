
\documentclass[version=last,
    paper=a4,                               % paper size
    10pt,                                   % default font size
    dvipsnames,
    % twoside,                                % PRINT Binding Correction
    oneside,                              % ONLINE
    headings=openany,                       % open chapters on odd and even pages
    open=any,
    BCOR=7mm,                               % PRINT Binding Correction
    %DIV=13,    % typearea 161.54mm x 228.46mm, top 22.85mm, inner 16.15mm
    %DIV=14,    % 165.00 233.36 21.21 15.00
    DIV=15,     % 168.00 237.60 19.80 14.00
    % toc=chapterentrywithdots              % Table of Contents style
]{scrbook}
\usepackage{typearea}


%------------------------------------------------------------------------------%
% Headers and footers
%------------------------------------------------------------------------------%
\usepackage[automark,headsepline,footsepline,plainfootsepline]{scrlayer-scrpage}
\automark*[section]{}
\addtokomafont{pageheadfoot}{\normalfont\footnotesize\sffamily} % Don't italicise
\renewcommand{\chaptermark}[1]{\markleft{#1}{}}     % Chapter: suppress numbering
\renewcommand{\sectionmark}[1]{\markright{#1}{}}    % Section: suppress numbering

% Header (inner, center, outer)
% \ihead{\href{http://sdis.dpaw.wa.gov.au}{\textbf{Project Plan SP 2014-018}}}
%\chead{\href{http://sdis.dpaw.wa.gov.au}{Science Directorate Information System}}
% \ohead{\href{https://www.dpaw.wa.gov.au/about-us/science-and-research}{\includegraphics[height=8mm, keepaspectratio]{/mnt/projects/sdis/staticfiles/img/logo-dpaw.png}}}

% Footer (inner, center, outer)
% \ifoot{\RaggedRight\leftmark}                       % Chapter
% \cfoot{\RaggedLeft\rightmark}                       % Section
% \ofoot[\bfseries\thepage]{\bfseries\thepage}        % Page number (also [plain])


%------------------------------------------------------------------------------%
% Fonts, encoding
%------------------------------------------------------------------------------%
%\usepackage{avant}             % Use the Avantgarde font for headings
\usepackage{txfonts}
\usepackage{mathptmx}
\usepackage{gensymb}            % provides \textdegree
\renewcommand{\familydefault}{\sfdefault} % Default to Sans Serif font
\usepackage{microtype}          % Slightly tweak font spacing for aesthetics
\usepackage[english]{babel}
\usepackage[utf8]{inputenc}  % Allow letters with accents
\usepackage[utf8]{luainputenc}  % Allow letters with accents
\usepackage[T1]{fontenc}        % Use 8-bit encoding that has 256 glyphs
\usepackage{textcomp}
\usepackage[explicit]{titlesec}           % Customise of titles
%\DeclareUnicodeCharacter{0080}{\textregistered}
\DeclareUnicodeCharacter{00B0}{\textdegree}

%------------------------------------------------------------------------------%
% Tables, columns, layout
%------------------------------------------------------------------------------%
\usepackage{etoolbox}
\AtBeginEnvironment{longtabu}{\footnotesize}{}{}  % Table font size
\usepackage{booktabs}           % Required for nicer horizontal rules in tables
\usepackage{multicol}           % 2 col publications
\usepackage{pdflscape}          % Landscape pages
\usepackage{pdfpages}           % Include PDFs
\usepackage{hanging}            % hanging paragraphs for publications
%\usepackage{titletoc}          % Manipulate the table of contents
\setcounter{tocdepth}{2}        % TOC list down to section
\usepackage{enumerate}          % Enumerations
\usepackage{enumitem}           % Enumerations
\usepackage{longtable}          % Multipage table
\usepackage{tabu}               %
\setlength{\tabulinesep}{1.5mm} % Consistent vertical spacing in tabu
\newcommand{\HRule}{\vspace{8mm}\noindent\rule{\linewidth}{0.1pt}}
\usepackage[export]{adjustbox}  % minipage, image frame


%------------------------------------------------------------------------------%
% Graphics, images, colours
%------------------------------------------------------------------------------%
\usepackage{graphicx} % embedded images
\usepackage{wrapfig}  % wrap figures in text
\usepackage{caption}  % allow unnumbered captions
\usepackage{eso-pic} % Required for specifying an image background in the title page
\usepackage{colortbl} % define custom named colours
\usepackage{xstring} % Conditionals
\usepackage{transparent} % Allow transparent images

\definecolor{RedFire}{RGB}{146,25,28}
% Following PICA branding guidelines
% https://dpaw.sharepoint.com/Divisions/pica/Documents/Branding%20guidelines.pdf
\definecolor{dpawblue}{RGB}{35,97,146}          % Pantone 647
\definecolor{dpaworange}{RGB}{237,139,0}        % Pantone 144
\definecolor{dpawgreen}{RGB}{116,170,80}        % Pantone 7489
\definecolor{dpawred}{RGB}{124,46,44}           % Paul's suggestion

% bootstrap colours
\definecolor{successbg}{RGB}{223,240,216}
\definecolor{errorbg}{RGB}{242,222,222}
\definecolor{warningbg}{RGB}{252,248,227}
\definecolor{infobg}{RGB}{217,237,247}
\definecolor{muted}{RGB}{153,153,153}
\definecolor{success}{RGB}{70,136,71}
\definecolor{error}{RGB}{185,74,72}
\definecolor{warning}{RGB}{192,152,83}
\definecolor{info}{RGB}{58,135,173}

% SDIS approval colours
\definecolor{required}{RGB}{192,152,83}
\definecolor{requiredbg}{RGB}{252,248,227}
\definecolor{denied}{RGB}{185,74,72}
\definecolor{deniedbg}{RGB}{242,222,222}
\definecolor{granted}{RGB}{70,136,71}
\definecolor{grantedbg}{RGB}{223,240,216}
\definecolor{notrequired}{RGB}{153,153,153}
\definecolor{notrequiredbg}{RGB}{255,255,255}

\usepackage{tikz} % Drawing
\usetikzlibrary{arrows,shapes,positioning,shadows,trees}


%------------------------------------------------------------------------------%
% Hyperlinks
%------------------------------------------------------------------------------%
\usepackage[open=true]{bookmark}
\usepackage{nameref}
\usepackage{ifxetex,ifluatex}
\ifxetex
  \usepackage[
    setpagesize=false,        % page size defined by xetex
    unicode=false,            % unicode breaks when used with xetex
    xetex]{hyperref}
\else
  \usepackage[unicode=true]{hyperref}
\fi

\hypersetup{
  backref=true,
  pagebackref=true,
  hyperindex=true,
  breaklinks=true,
  urlcolor=dpawblue,
  bookmarks=true,
  bookmarksopen=false,
  pdfauthor={Biodiversity and Conservation Science, Department of Biodiversity, Conservation and Attractions, WA},
  pdftitle=Project Plan SP 2014-018
,
  colorlinks=true,
  linkcolor=dpawblue,
  pdfborder={0 0 0}}

\urlstyle{same}                         % don't use monospace font for urlstyle


%------------------------------------------------------------------------------%
% Black magic to linebreak URLs
%------------------------------------------------------------------------------%
\usepackage{url}
\makeatletter\g@addto@macro{\UrlBreaks}{\UrlOrds}\makeatother
\Urlmuskip=0mu plus 1mu


%------------------------------------------------------------------------------%
% Fix latex errors
%------------------------------------------------------------------------------%
\providecommand{\tightlist}{\setlength{\itemsep}{0pt}\setlength{\parskip}{0pt}}

% copy-pasted HTML <span> in SDIS fields becomes \text{} in tex source
\providecommand{\text}{}


%------------------------------------------------------------------------------%
% Custom Tikz node for SDS diagram
%------------------------------------------------------------------------------%
\newcommand\mynode[6][]{
  \node[#1] (#2){
    \parbox{#3\relax}{
      \begin{center}
      \textbf{#4}\\
      #5\\
      \footnotesize{#6}
      \end{center}
    }};}


%------------------------------------------------------------------------------%
% Custom no-pagebreaks-environment
%------------------------------------------------------------------------------%
\newenvironment{absolutelynopagebreak}
  {\par\nobreak\vfil\penalty0\vfilneg\vtop\bgroup}
  {\par\xdef\tpd{\the\prevdepth}\egroup\prevdepth=\tpd}


%------------------------------------------------------------------------------%
% Remove the header from odd empty pages at the end of chapters
%------------------------------------------------------------------------------%
\makeatletter
\renewcommand{\cleardoublepage}{
\clearpage\ifodd\c@page\else
\hbox{}
\vspace*{\fill}
\thispagestyle{empty}
\newpage
\fi}


%----------------------------------------------------------------------------------------
%  Page flow control
%----------------------------------------------------------------------------------------
%\widowpenalty=10000
%\clubpenalty=10000
%\vbadness=1200
%\hbadness=11000


%----------------------------------------------------------------------------------------
%   CHAPTER HEADINGS
%----------------------------------------------------------------------------------------
\newcommand{\thechapterimage}{}
\newcommand{\chapterimage}[1]{\renewcommand{\thechapterimage}{#1}}

% Numbered chapters with mini tableofcontents
\def\thechapter{\arabic{chapter}}
\def\@makechapterhead#1{
%\thispagestyle{plain}
{\centering \normalfont\sffamily
\ifnum \c@secnumdepth >\m@ne
\if@mainmatter
\startcontents
\begin{tikzpicture}[remember picture,overlay]
\node at (current page.north west)
{\begin{tikzpicture}[remember picture,overlay]
\node[anchor=north west,inner sep=0pt] at (0,0) {
\includegraphics[width=\paperwidth,height=0.5\paperwidth]{\thechapterimage}};
%------------------------------------------------------------------------------%
% Small contents box in the chapter heading
% Mini TOC background box
%\fill[color=dpawblue!10!white,opacity=.2] (1cm,0) rectangle (
%  3.5cm, % Mini TOC box width
%  -3.5cm % Mini TOC box height
%);
% Mini TOC text content
%\node[anchor=north west] at (1.1cm,.35cm) {
%  \parbox[t][8cm][t]{6.5cm}{
%    \huge\bfseries\flushleft
%    \printcontents{l}{1}{
%    \setcounter{tocdepth}{1}                   % Mini TOC level depth
%    }
% }
%};
%------------------------------------------------------------------------------%
% Chapter heading
\draw[anchor=west] (5cm,-9cm) node [
rounded corners=20pt,
fill=dpawblue!10!white,
text opacity=1,
draw=dpawblue,
draw opacity=1,
line width=1.5pt,
fill opacity=.2,
inner sep=12pt]{
    \huge\sffamily\bfseries\textcolor{black}{
      \thechapter. #1\strut\makebox[22cm]{}
    }
};
\end{tikzpicture}};
\end{tikzpicture}}
\par\vspace*{240\p@}                            % Push text below chapter image
\fi
\fi}

%------------------------------------------------------------------------------%
% Unnumbered chapters without mini tableofcontents
%------------------------------------------------------------------------------%
\def\@makeschapterhead#1{
%\thispagestyle{plain}
{\centering \normalfont\sffamily
\ifnum \c@secnumdepth >\m@ne
\if@mainmatter
\begin{tikzpicture}[remember picture,overlay]
\node at (current page.north west)
{\begin{tikzpicture}[remember picture,overlay]
\node[anchor=north west,inner sep=0pt] at (0,0) {
  \includegraphics[width=\paperwidth,height=0.5\paperwidth]{\thechapterimage}};
% Mini TOC background box
%\fill[color=dpawblue!10!white,opacity=.2] (1cm,0) rectangle (
%  3.5cm,                                       % Mini TOC box width
%  -3.5cm                                       % Mini TOC box height
%);
% Mini TOC text content
%\node[anchor=north west] at (1.1cm,.35cm) {
%  \parbox[t][8cm][t]{6.5cm}{
%    \huge\bfseries\flushleft
%    \printcontents{l}{1}{
%    \setcounter{tocdepth}{1} % Mini TOC level depth
%    }
%}
%};
\draw[anchor=west] (5cm,-9cm) node [rounded corners=20pt,
  fill=dpawblue!10!white,fill opacity=.6,inner sep=12pt,text opacity=1,
  draw=dpawblue,draw opacity=1,line width=1.5pt]{
  \huge\sffamily\bfseries\textcolor{black}{#1\strut\makebox[22cm]{}}};
\end{tikzpicture}};
\end{tikzpicture}}
\par\vspace*{240\p@}
\fi
\fi
}
\makeatother



\usepackage[automark,headsepline,footsepline,plainfootsepline]{scrlayer-scrpage}
\automark*[section]{}
\addtokomafont{pageheadfoot}{\normalfont\footnotesize\sffamily} % Don't italicise
\renewcommand{\chaptermark}[1]{\markleft{#1}{}}     % Chapter: suppress numbering
\renewcommand{\sectionmark}[1]{\markright{#1}{}}    % Section: suppress numbering

% Header (inner, center, outer)
\ihead{\href{http://sdis.dpaw.wa.gov.au/documents/projectplan/1483/}{Project Plan SP 2014-018}}
%\chead{\href{http://sdis.dpaw.wa.gov.au}{Science Directorate Information System}}
\ohead{\href{https://www.dpaw.wa.gov.au/about-us/science-and-research}{\includegraphics[height=6mm, keepaspectratio]{/mnt/projects/sdis/staticfiles/img/logo-dpaw.png}}}

% Footer (inner, center, outer)
\ifoot{\textbf{Printed}~Thu, 5 Jul 2018 13:43:02 +0800} % inner/left footer
\cfoot{}
\ofoot[\bfseries\thepage]{\bfseries\thepage}        % Page number (also [plain])


\pagestyle{scrheadings}
\setkomafont{pageheadfoot}{\normalfont}

%-----------------------------------------------------------------------------%
\begin{document}
\raggedbottom

%-----------------------------------------------------------------------------%
% Title page
\subject{Project Plan SP 2014-018
}
\title{Distribution and abundance estimate of Australian snubfin dolphins
(\emph{Orcaella heinsohni}) at a key site in the Kimberley region,
Western Australia
}
\subtitle{Marine Science
}
\author{}
\publishers{\small
    \subsection*{Project Core Team}
\begin{tabu} {X X}
\textbf{Supervising Scientist} & Kelly Waples
\\
\textbf{Data Custodian} & Holly Raudino
\\
\textbf{Site Custodian} & Holly Raudino
\\
\end{tabu}


    \subsection*{Project status as of July 5, 2018, 1:43 p.m.}
\begin{tabu} {X X}
& Update requested
\\
\end{tabu}

    
\subsection*{Document endorsements and approvals as of July 5, 2018, 1:43 p.m.}
\begin{tabu} {X X}

%\rowcolor{grantedbg}
    \textbf{Project Team} & 
    \textcolor{granted}{ granted}\\

%\rowcolor{grantedbg}
    \textbf{Program Leader} & 
    \textcolor{granted}{ granted}\\

%\rowcolor{grantedbg}
    \textbf{Directorate} & 
    \textcolor{granted}{ granted}\\

%\rowcolor{grantedbg}
    \textbf{Biometrician} & 
    \textcolor{granted}{ granted}\\

%\rowcolor{not requiredbg}
    \textbf{Herbarium Curator} & 
    \textcolor{not required}{ not required}\\

%\rowcolor{not requiredbg}
    \textbf{Animal Ethics Committee} & 
    \textcolor{not required}{ not required}\\

\end{tabu}



}
\uppertitleback{}
\lowertitleback{}
\date{}

%-----------------------------------------------------------------------------%
% Front matter
\frontmatter
\maketitle
%-----------------------------------------------------------------------------%
% Main matter
\mainmatter



\section*{Distribution and abundance estimate of Australian snubfin dolphins
(\emph{Orcaella heinsohni}) at a key site in the Kimberley region,
Western Australia
}



\subsection*{Biodiversity and Conservation Science Program}

Marine Science




\subsection*{Departmental Service}

Service 2: Conserving Habitats, Species and Ecological Communities


\subsection*{Project Staff}
\begin{tabu} {X X X}
%\rowcolor{infobg}
\textbf{Role} & \textbf{Person} & \textbf{Time allocation (FTE)}\\

Supervising Scientist & Kelly Waples & 0.03\\

Research Scientist & Holly Raudino & 0.1\\

\end{tabu}




\subsection*{Related Science Projects}

None


\subsection*{Proposed period of the project}
May 31, 2014 -- June 30, 2018



\section*{Relevance and Outcomes}


\subsection*{Background}

The Australian snubfin dolphin (Orcaella heinsohni) was formally
differentiated from the Irrawaddy dolphin (Orcaella brevirostris) in
2005 (Beasley et al., 2005). Currently, there is limited information
available for the species and local population are likely to be small in
size (Parra \emph{et al}. 2006, Cagnazzi \emph{et al}. 2010, Palmer
\emph{et al.} in press) and geographically or genetically isolated
(Cagnazzi 2010). Fewer than 100 individuals of snubfin wereestimated at
three sites in the tropical and sub-tropical waters of Queensland (Parra
\emph{et al}. 2006, Cagnazzi 2010). ~More recently in the NT (2008 --
2010), photo-identification surveys were undertaken in Port Essington,
Cobourg Peninsula and Darwin Harbour (Palmer et al., 2010, Palmer 2010),
however, there were too few data on snubfin in Darwin Harbour to support
capture-recapture modelling (Palmer \emph{et al.} submitted). In Port
Essington, total abundance estimates for snubfin dolphins were highly
variable across different sampling periods and varied from 136 to 222
(Palmer \emph{et al}. submitted). In 2010, a nomination was made for the
listing of this species as threatened under the EPBC Act, based on its
known biological characteristics and rising pressures resulting from
coastal developments and numerous industrial activities. However, the
conservation status of the species could not be properly assessed due to
a lack of information on the animals' population dynamics and
distribution.~ A coordinated research strategy is being developed by the
Commonwealth to provide some direction and prioritisation for research
to fill this information gap, leading to a better understanding of the
conservation status of this species and the current threats that it
faces.

~

This project will begin to address these information needs by focusing
on the data that already exist within WA. A primary objective of the
project will be to compile and analyse these data so that they are
compatible and comparable with those from other areas within the
species' range and will be available for comparison with ongoing and
future research effort. This will result in a better understanding of
snubfin dolphin distribution and abundance in northern Australia.

~

Specifically, the project aims to assess dolphin distribution via
collation and review of existing sighting data collected across the
Kimberley region between 2004-2012. The first abundance estimates at one
key site in the Kimberley region (Roebuck Bay) will be analysed, extent
of occurrence and area of occupancy, including integrating traditional
knowledge that has been contributed by Indigenous communities through
sea ranger surveys over the greater Kimberley area. To collate the data
for abundance estimation, we propose to migrate survey data into DolFin,
an Oracle database with Access interface that has been developed by the
NT government. It has been agreed that the database technology will be
made available to other states (WA and QLD) to achieve standardisation
of data storage and facilitate comparisons between sites and datasets.
This will enable longer-range movements and distributions to be examined
across the snubfins' range, aligning with a key priority identified by
in the Commonwealth Tropical Inshore Dolphin Draft Research Strategy.




\subsection*{Aims}

The project objectives are therefore to use existing data to:

~

1) Provide a quantitative abundance estimate for Roebuck Bay, WA that
will stand as a baseline for this population and will also enable
comparison with abundance estimates of the species from sites in QLD and
NT e.g. Cleveland Bay and Port Essington

2) Compare methods for abundance estimation (Mark-recapture versus
distance sampling) and the suitability of these methods for abundance
estimation of this species.

3) Map the extent of occurrence and area of occupancy of snubfin
dolphins in the Kimberley, WA by combining traditional knowledge and
dolphin sightings from Indigenous communities by integrating sea ranger
and scientific survey sightings.

4) Integrate data into a standardised database system for the management
of data and the facilitation of data sharing between jurisdictions and
sites~




\subsection*{Expected outcome}

The project will assess dolphin distribution via collation and review of
existing sighting data collected across the Kimberley region between
2004-2012. This will provide an abundance estimate for snubfin dolphins
in Roebuck Bay, a key asset of the proposed marine park, which can be
used for future monitoring programs.~ The project will also bring a
shared database to WA that is currently in use in the Northern
Territory. Sharing a database and compatible data collection and
analysis methods, will enable longer-range movements and distributions
to be examined across the snubfins' range, aligning with a key priority
identified by in the Commonwealth Tropical Inshore Dolphin Draft
Research Strategy. DPAW will have an improved understanding of a key
species in the northwest including baseline data, monitoring protocol
and distribution across the broader Kimberley to focus future research
and monitoring effort.~~




\subsection*{Knowledge transfer}

The primary users of the knowledge gained will be within DPAW, namely
the Marine Science Program and Kimberley Region. The research findings
will be used to establish and inform research and monitoring of inshore
tropical dolphins in the northwest for both of these groups. Traditional
owners will have an interest in the outcomes as they participated in
some field work.~ All data will be readily provided to TOs and to the
Regions and discussions will be held to address future monitoring
protocols and ongoing use of the database.~




\subsection*{Tasks and Milestones}

\begin{longtable}[c]{@{}ll@{}}
\toprule\addlinespace
\begin{minipage}[t]{0.47\columnwidth}\raggedright
\textbf{Milestone}
\end{minipage} & \begin{minipage}[t]{0.47\columnwidth}\raggedright
\textbf{Timing}
\end{minipage}
\\\addlinespace
\begin{minipage}[t]{0.47\columnwidth}\raggedright
Integrating distribution and mapping data ino GIS system and data
validation for DolFIN
\end{minipage} & \begin{minipage}[t]{0.47\columnwidth}\raggedright
~
\end{minipage}
\\\addlinespace
\begin{minipage}[t]{0.47\columnwidth}\raggedright
Installation and training in DolFIN
\end{minipage} & \begin{minipage}[t]{0.47\columnwidth}\raggedright
May 14
\end{minipage}
\\\addlinespace
\begin{minipage}[t]{0.47\columnwidth}\raggedright
Processing photo id data including entry into DolFIN
\end{minipage} & \begin{minipage}[t]{0.47\columnwidth}\raggedright
June 14
\end{minipage}
\\\addlinespace
\begin{minipage}[t]{0.47\columnwidth}\raggedright
Progress report to AMMC
\end{minipage} & \begin{minipage}[t]{0.47\columnwidth}\raggedright
July 14
\end{minipage}
\\\addlinespace
\begin{minipage}[t]{0.47\columnwidth}\raggedright
Workshop with TOs and users
\end{minipage} & \begin{minipage}[t]{0.47\columnwidth}\raggedright
October 14
\end{minipage}
\\\addlinespace
\begin{minipage}[t]{0.47\columnwidth}\raggedright
Abundance estimate mark-recapture
\end{minipage} & \begin{minipage}[t]{0.47\columnwidth}\raggedright
Nov 14
\end{minipage}
\\\addlinespace
\begin{minipage}[t]{0.47\columnwidth}\raggedright
Abundance estimate- distance sampling
\end{minipage} & \begin{minipage}[t]{0.47\columnwidth}\raggedright
Nov 14
\end{minipage}
\\\addlinespace
\begin{minipage}[t]{0.47\columnwidth}\raggedright
Final report
\end{minipage} & \begin{minipage}[t]{0.47\columnwidth}\raggedright
April 15
\end{minipage}
\\\addlinespace
\begin{minipage}[t]{0.47\columnwidth}\raggedright
Publication
\end{minipage} & \begin{minipage}[t]{0.47\columnwidth}\raggedright
December 15
\end{minipage}
\\\addlinespace
\bottomrule
\end{longtable}




\subsection*{References}

Allen, S. J., Cagnazzi, D. D., Hodgson, A. J., Loneragan, N. R. \&
Bejder, L. 2012. Tropical inshore dolphins of north-western Australia:
Unknown populations in a rapidly changing region. Pacific Conservation
Biology, 18, 56-63.

~

Beasley I., K.M. Robertson \& P. Arnold (2005). Description of a new
dolphin: The Australian snubfin dolphin~Orcaella heinsohni~sp.n.
(Cetacea, Delphinidae).~Marine Mammal Science. 21(3):365-400.

~

Bejder, L., Hodgson, A., Loneragan, N. \& Allen, S. 2012. Coastal
dolphins in north-western Australia: The need for re-evaluation of
species listings and short-comings in the Environmental Impact
Assessment process. Pacific Conservation Biology, 18, 22-25.

~

Buckland, S. (2001). Shipboard sighting surveys: Methodological
developments to meet practical needs. \emph{Bulletin of the
International Statistical Institute} Book 1, 315-318.

~

Cagnazzi, D. (2010). Conservation Status of Australian snubfin dolphin,
Orcaella heinsohni, and Indo-Pacific humpback dolphin, Sousa chinensis,
in the Capricorn Coast, Central Queensland, Australia PhD Thesis,
Southern Cross University.

~

Palmer, C., Parra, G. Rogers, T. and Woinarski, J. C. Z. (In Press).
Collation and review of sightings and distribution of three coastal
dolphin species in waters of the Northern Territory, Australia. Pacific
Conservation Biology

~

~

Hines, J., E., et al. ``Tigers on trails: occupancy modeling for cluster
sampling.'' \emph{Ecological Applications} 20.5 (2010): 1456-1466.

~

Palmer, C., Murphy, S. A., Thiele, D., Parra, G. J., Robertson, K. M.,
Beasley, I. \& Austin, C. M. 2011. Analysis of mitochondrial DNA
clarifies the taxonomy and distribution of the Australian snubfin
dolphin (Orcaella heinsohni) in northern Australian waters. Marine and
Freshwater Research, 62, 1303-1307.

~

Palmer, C., 2010. Interim Report: Darwin Harbour Coastal Dolphin
Project. Northern Territory Department of Natural Resources,
Environment, the Arts and Sport. 16 pp.

~

~Palmer, C., Fitzgerald, P., Wood, A. and McKenzie, A., 2010.
Conservation Assessment of Priority Non-Fish Marine Threatened Species
in the NT: Project no. 2007 / 134. Monitoring and assessment of inshore
dolphins in Coburg Marine Park (Garig Gunak Barlu National Park), Final
Report. Call no. 599.33, Palmerston, Northern Territory.

~

Parra, G. J. 2006. Resource partitioning in sympatric delphinids: space
use and habitat preferences of Australian snubfin and Indo-Pacific
humpback dolphins. Journal of Animal Ecology, 75, 862-874.

~

Parra, G. J., Corkeron, P. J. \& Arnold, P. 2011. Grouping and
fission--fusion dynamics in Australian snubfin and Indo-Pacific humpback
dolphins. Animal Behaviour.

~

Parra, G. J., Corkeron, P. J. \& Marsh, H. 2006a. Population sizes, site
fidelity and residence patterns of Australian snubfin and Indo-Pacific
humback dolphins: Implications for conservation. Biological
Conservation, 129, 167-180.

~

Parra, G., Schick, R. \& Corkeron, P. J. 2006b. Spatial distribution and
environmental correlates of Australian snubfin and Indo-Pacific humpback
dolphins. Ecography, 29, 396-406.

~

Smith, B. D. and Reeves, R. R. 2000. Methods for studying Freshwater
Cetaceans. Survey Methods for Population Assessment of Asian River
Dolphins. In Biology and Conservation of Freshwater Cetaceans in Asia.
IUCN Species Survival Commission, Occasional Paper of the IUCN Species
Survival Commission No. 23. Pp. 97 - 115.

~

Thiele, D. 2005. Report of an opportunistic survey for Irrawaddy
dolphins, \emph{Orcaella brevirostris}, off the Kimberley coast,
northwest Australia. Working paper presented at the 57th International
Whaling Commission annual meeting, Small Cetaceans Scientific Committee
(IWC/SC57/SM2), Ulsan, Republic of Korea.

Thiele, D. 2008. Ecology of inshore and riverine dolphin species in
northwestern Australian waters. Kimberley coast Orcaella project.
Summary of current knowledge -- EPBC listed inshore dolphin species on
the Kimberley Coast. Unpublished background report to DSEWPaC.

Thiele, D. 2010a. Collision course: snubfin dolphin injuries in Roebuck
Bay. p. 15 pp.: WWF - Australia.

Thiele, D. 2010b. Iconic marine wildlife surveys. Unpublished final
report to WWF.

Torres, L. G., Reid, A. J., Halpin, P. 2008. Fine scale habitat modeling
of a top marine predator: Do prey data improve predictive capacity?
Ecological Applications 18 (7): 1702 -- 1717.



\section*{Study design}


\subsection*{Methodology}

The project will analyse existing data gathered from both dedicated and
opportunistic vessel-based surveys conducted in the Kimberley between
2004 and 2012.~

~

(i) \textbf{ROEBUCK BAY}-Abundance Estimation

\emph{~}

\textbf{Survey design \& methods:}

Dedicated boat-based surveys have been conducted in Roebuck Bay:

in 2007 (April/May), 2008 (July \& December) using a standard zig zag
line transect design (Map 1); and

In 2009 (May-Sept) with a total of 28 survey days completed using a
standardised grid survey method (Map 2).

~

The 2009 survey design used multiple grids to achieve consistent
sampling effort across the full range of nearshore environments in the
bay and regular parallel sweeps along the inshore 1 metre depth contour
(NB: 1 metre depth contour changes with the tide) to ensure that the
grid surveys were not missing significant number of dolphins between
grids. Once a full set of the grids was completed a parallel sweep would
follow. The order of grids was randomly selected each day but dependent
on sea conditions in the survey area (Roebuck Bay). Abundance estimates
will be calculated for each year (season) but inter-annual comparisons
will not be made due to the difference in sampling effort and survey
design.

~

Map 1- Survey design for Roebuck Bay, Kimberley WA depicting grid of
transect lines used in 2007 and 2008.

~

Map 2- Survey design for Roebuck Bay, Kimberley WA depicting grid of
transect lines used in 2009.

~

Observers searched ahead of the vessel (following line transect methods
(Buckland, 2001) and one observer checked behind the vessel. When a
sighting occurred the vessel was stopped and the distance and angle to
the dolphin group was recorded, along with the GPS position of the
vessel, and measurements of water depth, sea surface temperature were
taken The dolphin group was approached for the purpose of
photo-identification if they were approachable and did not behave
evasively. The aim was to photograph all individuals in a group without
compromising the survey of transect lines and grids and avoiding
changing the distribution of the animals by causing responsive movement
to the vessel. The grid was repeated if photo-identification occurred
over a protracted period.

~

In total, 387 nautical miles of transect lines were covered on effort
resulting in snubfin sightings 64/217 (group/individuals).

Total effort (includes grid transects and transits between transect
lines (off-effort)) 903nautical miles; snubfin sightings 139/480.

\textbf{~}

\textbf{Abundance estimation methods:}

~

\emph{Distance sampling:}

Public domain programme DISTANCE v6.0 will be used to generate absolute
abundance estimates (accounting for detection probability) for snubfin
dolphins within the Roebuck Bay region, based on visual sightings
recorded during the vessel-based surveys described above.

The computer software programmes DISTANCE (PB), R, and ArcGIS (PB\&DT)
will be used for the density and abundance estimate and broad-scale
spatial analysis.

~

\emph{Mark-recapture:}

Photo-ID images for each individual dolphin with the accompanying survey
and spatial information will need to be migrated into the DolFin
database. This database was developed for Department of Land and
Resource Management, NT and is currently used for all dolphin photo id
work.~~ CP and an experienced Research Assistant working with this team
(AP) will be contracted to quality-check the photo-identification images
for and enter them into the database. AP and DT will work
collaboratively to ensure the quality of the data and a successful
migration into the new database.

~

To estimate abundance, the computer software programme MARK (HR\&LB)
will be used with model types such as POPAN or Robust Design. The
sighting history files will need to be constructed from the DolFin
database in a binary (0,1) format that can be read by the mark-recapture
software MARK. These sighting histories will indicate the presence or
absence of an individual and will likely be compiled over several
temporal scales (seasons and years). These binary data will be read into
MARK and the models parameterised based on relevant survey design and
sampling regime assumptions. Multiple models will be fitted and ranked
to achieve the best abundance estimate with corresponding confidence
intervals.

~

The resulting abundance estimates and confidence intervals will be
compared. This comparison will yield insights into the feasibility and
robustness of each method as a candidate technique for the future
monitoring of snubfin populations.

~

(ii) \textbf{KIMBERLEY COAST}- Mapping distribution

\textbf{~}

\textbf{Survey design \& methods}

Two broad-scale surveys were conducted by DT (2004 \& 2006) along the
Kimberley coast:

~

The 2004 survey (Darwin to Broome 14 days) was conducted during
September (4\textsuperscript{th} -- 18th) aboard a yacht, RV Pelican.
The purpose of the survey was to investigate the presence of snubfin
dolphins in riverine and estuarine habitats. Survey effort was conducted
throughout all daylight hours when the vessel was motoring or sailing
unless weather conditions were poor. DT, D. Cook and others on the
vessel maintained a constant watch 180º ahead of the vessel from the
bow, side bridge and upper bridge platforms.

~

All effort, sighting, habitat, position, weather and associated data
(including evidence of human activities e.g. fishing vessels, tourist
vessels, pearl farming etc.) were recorded using the IFAW LOGGER
software program. Group size estimates for cetaceans included best, high
and low estimates. Sightings of all wildlife were recorded, and images
of habitats were regularly taken. Cetaceans, turtles, sea snakes and
dugong were recorded individually, and seabirds were recorded by species
and flock size

~

Photographs of cetacean sightings were collected wherever possible.
Images were also collected for human activities (i.e. fishing vessels,
pearl farms, tourist vessels) and as records of habitat along river and
estuarine transits. When sightings of Irrawaddy dolphins were made, the
vessel would slow or stop or move closer in order to confirm
identification at slow speed. Although the vessel did maintain
position/distance in relation to groups of dolphins until positive
identification was made, animals were not directly followed or pursued
at any time.

~

Groups of snubfin dolphins (7/36) were found in nearshore waters near
Cape Londonderry, Anjo Peninsula, Vansittart Bay, Augustus Island and
Roebuck Bay in Broome. No groups were found in the rivers surveyed (the
Berkeley River, King George River and Sampson Inlet). Survey effort
125h; snubfin sightings 7/36.

~

The 2006 survey (Darwin to Broome 14 days) was conducted during May
(14\textsuperscript{th} -- 28th) aboard a mackerel boat, the RV Rachel.
The main vessel was supported by a small vessel, used to survey in
shallow estuarine areas and for close work (e.g. photo-ID, biopsy). The
purpose of the survey was to investigate the presence of snubfin
dolphins in riverine and estuarine habitats and to obtain photo-ID
records, acoustic recordings and biopsies. Search effort on the main
vessel was conducted from the main deck in front \& sides of the bridge.
Searching was conducted by two to five observers (most often four
observers) using the naked eye and 7x50 Fujinon binoculars. Additional
survey protocols were as noted above (2004 survey).

Survey effort 125h1m 1m; snubfin sightings 15/88

~

A series of surveys with six coastal ranger groups were also conducted
between August 2009 and June 2012. The surveys covered two broad regions
\& the Prince Regent River:

~

(i) north and south of Kalumburu (Napier Broome Bay including the
Drysdale River); and

(ii) the region from Talbot Bay through to King Sound.

Most surveys were supported by a large vessel while utilising small
boats for survey transects during the day.~ Travelling more than 4200km
in boats, a total of 392 hours of survey effort was spent looking for
snubfin and Indo-Pacific Humpback dolphins as the focus species, but
also recording a wide range of other species as they were encountered.
~Total snubfin dolphin sightings:7/15.

~

\textbf{ARC GIS MAPPING METHODS}

The information from these surveys will be used for distribution
analysis but not for abundance estimation. Maps will be produced using
ArcGIS that depict extent of occurrence and area of occupancy using
existing data from scientific surveys and traditional knowledge
contributed by Indigenous communities. GIS mapping the distribution
(extent of occurrence and area of occupancy) using existing sighting
data from surveyed areas of the Kimberley.~




\subsection*{Biometrician's Endorsement}

granted



\section*{Data management}


\subsection*{No. specimens}






\subsection*{Herbarium Curator's Endorsement}

not required




\subsection*{Animal Ethics Committee's Endorsement}

not required




\subsection*{Data management}

Data will be maintained at the Marine Science Program, Science and
Conservation Division, and stored on the DPaW Science and Conservation
Division server (T drive) in DolFIN database, an oracle database
acquired from the Northern Territory Government.

~

~~~~~~~~~ Hard copies of the any reports/publications will be held at
the following locations:

~~~~~~~~~ DPaW Kimberley Region,

~~~~~~~~~ Marine Science Program, Science Division, DPaW, 17 Dick Perry
Avenue, WA, 6151

~~~~~~~~~ Conservation Library, Science Division, DPaW, 17 Dick Perry
Avenue, WA, 6151

~




\section*{Budget}

\section*{Consolidated Funds }



\begin{longtabu} to \linewidth { |  X | X | X | X | }
\hline
\rowcolor{infobg}
Source & Year 1 & Year 2 & Year 3\\
\hline
\endhead



FTE Scientist &  &  & \\



FTE Technical &  &  & \\



Equipment &  &  & \\



Vehicle &  &  & \\



Travel &  &  & \\



Other &  &  & \\



Total &  &  & \\


\hline
\end{longtabu}



\section*{External Funds }



\begin{longtabu} to \linewidth { |  X | X | X | X | }
\hline
\rowcolor{infobg}
Source & Year 1 & Year 2 & Year 3\\
\hline
\endhead



Salaries, Wages, Overtime & 40,204 &  & \\



Overheads &  &  & \\



Equipment &  &  & \\



Vehicle &  &  & \\



Travel & 10,250 &  & \\



Other & 10,900 &  & \\



Total & 61354 &  & \\


\hline
\end{longtabu}





%-----------------------------------------------------------------------------%
% Back matter
%\backmatter
\end{document}
%-----------------------------------------------------------------------------%

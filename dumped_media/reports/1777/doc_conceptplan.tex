
\documentclass[version=last,
    paper=a4, % paper size
    10pt, % default font size
    usenames,
    dvipsnames,
    oneside, % ONLINE
    headings=openany, % open chapters on odd and even pages
    %toc=chapterentrywithdots, % Table of Contents style
    %BCOR=7mm, % PRINT Binding Correction
    %DIV=13, % typearea 161.54 mm x 228.46 mm, top margin 22.85 mm, inner margin 16.15 mm
    %DIV=14, % 165.00 233.36 21.21 15.00
    DIV=15 % 168.00 237.60 19.80 14.00
]{scrbook}
\usepackage{typearea}
\usepackage[automark,headsepline,footsepline]{scrlayer-scrpage} % Headers and footers

%%
%% Fonts, encoding, spacing, indentation
%%
\usepackage{txfonts}
\renewcommand{\familydefault}{\sfdefault} % Default to Sans Serif font
\usepackage[english]{babel}
\usepackage[T1]{fontenc}
\usepackage[utf8]{inputenc}

% Paragraph spacing
%\usepackage{parskip}    % Paragraph spacing
%\setlength{\parindent}{0em} % Don't indent paragraphs - ONLINE
%\setlength{\parskip}{1.3 ex plus 0.5ex minus 0.3ex} % 1-1.8 ex vertical space between paragraphs - ONLINE

% Spacing of headings
%\RedeclareSectionCommand[afterskip=3pt]{section} % less space after section
%\RedeclareSectionCommand[beforeskip=0cm]{subsection} % less space between HRule and project name
%\RedeclareSectionCommand[afterskip=0.1\baselineskip]{subsubsection} % less space after progressreport subheadings

% Table font size
\usepackage{etoolbox}
\AtBeginEnvironment{longtabu}{\footnotesize}{}{}

%%
%% Tables, columns, layout
%%
\usepackage{multicol}   % 2 col publications
\usepackage{pdflscape}  % Landscape pages
\usepackage{pdfpages}   % Include PDFs
\usepackage{hanging}    % hanging paragraphs for publications
%\usepackage{titletoc}   % Required for manipulating the table of contents
\setcounter{tocdepth}{2} % TOC list down to section
\usepackage{enumerate}  % Enumerations
\usepackage{enumitem}   % Enumerations
\usepackage{longtable}  % Multipage table
\usepackage{tabu}       %
\setlength{\tabulinesep}{1.5mm} % Consistent vertical spacing in tabu

%%
%% Graphics, images, colours
%%
\usepackage{graphicx} % embedded images
\usepackage{eso-pic} %
\usepackage{colortbl} % define custom named colours
\definecolor{RedFire}{RGB}{146,25,28}
\definecolor{ParksWildlife}{RGB}{0,85,144}
\definecolor{successbg}{RGB}{223,240,216}
\definecolor{errorbg}{RGB}{242,222,222}
\definecolor{warningbg}{RGB}{252,248,227}
\definecolor{infobg}{RGB}{217,237,247}
\definecolor{muted}{RGB}{153,153,153}
\definecolor{success}{RGB}{70,136,71}
\definecolor{error}{RGB}{185,74,72}
\definecolor{warning}{RGB}{192,152,83}
\definecolor{info}{RGB}{58,135,173}

\definecolor{required}{RGB}{192,152,83}
\definecolor{requiredbg}{RGB}{252,248,227}
\definecolor{denied}{RGB}{185,74,72}
\definecolor{deniedbg}{RGB}{242,222,222}
\definecolor{granted}{RGB}{70,136,71}
\definecolor{grantedbg}{RGB}{223,240,216}
\definecolor{not reqiured}{RGB}{153,153,153}
\definecolor{not requiredbg}{RGB}{255,255,255}

\usepackage{tikz} % Drawing
\usetikzlibrary{arrows,shapes,positioning,shadows,trees}

%%
%% Links, URLs
%%
\usepackage[
    linktoc=all,
    %colorlinks=false,  %PRINT
    colorlinks=true, % ONLINE
    linkcolor=RedFire, % ONLINE
    urlcolor=ParksWildlife, % ONLINE
    pdftitle=Concept Plan SP 2016-067
]{hyperref}

% Black magic to linebreak URLs
\usepackage{url}
\makeatletter
\g@addto@macro{\UrlBreaks}{\UrlOrds}
\makeatother

%%
%% Custom macros
%%
% Thick Horizontal rule
\newcommand{\HRule}{\vspace{8mm}\\\noindent\rule{\linewidth}{0.1pt}}

% Custom Tikz node for SDS diagram
\newcommand\mynode[6][]{
    \node[#1] (#2){
        \parbox{#3\relax}{
            \begin{center}
            \textbf{#4}\\
            #5\\
            \footnotesize{#6}
            \end{center}}};}



\usepackage[automark,headsepline,footsepline,plainfootsepline]{scrlayer-scrpage}
\automark*[section]{}
\addtokomafont{pageheadfoot}{\normalfont\footnotesize\sffamily} % Don't italicise
\renewcommand{\chaptermark}[1]{\markleft{#1}{}}     % Chapter: suppress numbering
\renewcommand{\sectionmark}[1]{\markright{#1}{}}    % Section: suppress numbering

% Header (inner, center, outer)
\ihead{\href{http://sdis.dpaw.wa.gov.au/documents/conceptplan/1777/}{Concept Plan SP 2016-067}}
%\chead{\href{http://sdis.dpaw.wa.gov.au}{Science Directorate Information System}}
\ohead{\href{https://www.dpaw.wa.gov.au/about-us/science-and-research}{\includegraphics[height=6mm, keepaspectratio]{/mnt/projects/sdis/staticfiles/img/logo-dpaw.png}}}

% Footer (inner, center, outer)
\ifoot{\textbf{Printed}~Wed, 9 Aug 2017 13:36:52 +0800} % inner/left footer
\cfoot{}
\ofoot[\bfseries\thepage]{\bfseries\thepage}        % Page number (also [plain])


\pagestyle{scrheadings}
\setkomafont{pageheadfoot}{\normalfont}

%-----------------------------------------------------------------------------%
\begin{document}
\raggedbottom

%-----------------------------------------------------------------------------%
% Title page
\subject{Concept Plan SP 2016-067
}
\title{Improved fauna conservation in the wheatbelt
}
\subtitle{Animal Science
}
\author{}
\publishers{\small
    \subsection*{Project Core Team}
\begin{tabu} {X X}
\textbf{Supervising Scientist} & Keith Morris
\\
\textbf{Data Custodian} & Keith Morris
\\
\textbf{Site Custodian} & 
\\
\end{tabu}


    \subsection*{Project status as of Aug. 9, 2017, 1:36 p.m.}
\begin{tabu} {X X}
& New project, pending concept plan approval
\\
\end{tabu}

    
\subsection*{Document endorsements and approvals as of Aug. 9, 2017, 1:36 p.m.}
\begin{tabu} {X X}

%\rowcolor{requiredbg}
    \textbf{Project Team} & 
    \textcolor{required}{ required}\\

%\rowcolor{requiredbg}
    \textbf{Program Leader} & 
    \textcolor{required}{ required}\\

%\rowcolor{requiredbg}
    \textbf{Directorate} & 
    \textcolor{required}{ required}\\

\end{tabu}



}
\uppertitleback{}
\lowertitleback{}
\date{}

%-----------------------------------------------------------------------------%
% Front matter
\frontmatter
\maketitle
%-----------------------------------------------------------------------------%
% Main matter
\mainmatter


\section*{Improved fauna conservation in the wheatbelt
}



\subsection*{Science and Conservation Division Program}

Animal Science




\subsection*{Parks and Wildlife Service}

Service 2: Conserving Habitats, Species and Ecological Communities




\subsection*{Aims}

Predations by introduced predators, the red fox and feral cat,~are
regarded as the most significant threatening processes for most of
Australia's threatened mammal species. Fox control, through broadscale
toxic baiting under the \emph{Western Shield} program has contributed
significantly to the conservation of many native mammal species in
Western Australia. However it has now been demonstrated that where fox
abundance has been reduced through baiting, activity of feral cats has
increased, and many of the mammal declines observed since 2000 in
Western Australia are thought to be due to increased predation by feral
cats.~In 2014 the \emph{Eradicat\textsuperscript{®}} cat bait was
registered for operational use in the south-west of Western Australia,
and~an operational trial to integrate cat baiting with fox baiting is
currently underway at four sites in the south-west of WA (SW Fauna
Recovery Project at Kalbarri, Dryandra, Upper Warren and South Coast).
Lake Magenta Nature Reserve is one of the largest wheatbelt reserves and
still supports populations of threatened mammals species such as the
chuditch (\emph{Dasyurus geoffroii}), red-tailed phascogale
(\emph{Phascogale calura}) and heath mouse (\emph{Pseudomys
shortridgei}). The mesopredator research program undertaken at Lake
Magenta 2006-2009 demonstrated that fox control was~effective at
reducing fox activity at Lake Magenta, compared to an unbaited control
site (Dunn Rock nature reserve). It also showed~that feral cat activity
was correspondingly higher at Lake Magenta. Recent fauna surveys of Lake
Magenta have detected fewer chuditch and heath mice than previously, and
it is possible that predation by feral cats has replaced fox predation
as a factor in these declines. This project proposes to assess the
impact of introducing cat baiting to Lake Magenta nature reserve,
and~integrating this with the current fox baiting regime being
implemented there.




\subsection*{Expected outcome}

\begin{itemize}
\itemsep1pt\parskip0pt\parsep0pt
\item
  A demonstration of reduced fox and feral cat activity at Lake Magenta
  Nature Reserve (compared to pre-cat baiting baseline) following an
  annual cat baiting program that is integrated into the current fox
  baiting program.
\item
  A reduction in overall predation pressure on native species and an
  increase in abundance of threatened mammal species.
\item
  Contribution to the SW Fauna Recovery Project in determining the
  appropriate fox and feral cat baiting regime(s) in the south-west of
  Western Australia.
\item
  This project builds on the mesopredator research project undertaken at
  Lake Magenta 2006-2009 which had planned to implement feral cat
  baiting, but which didn't eventuate because the
  \emph{Eradicat\textsuperscript{®}} bait did not get registered in
  time.
\end{itemize}




\subsection*{Strategic context}

\begin{itemize}
\itemsep1pt\parskip0pt\parsep0pt
\item
  Department of Parks and Wildlife strategic goal (Wildlife): Conserve,
  protect and manage the State's native plants and animals, and achieve
  habitat, ecosystem and landscape-scale conservation and protection
  based on~best practice science.
\item
  Science and Conservation Strategic Plan 2014-2017: Recovery of key
  animal species, understand and manage threats from pest animals.
\item
  With regards to the Department's ``Framework for Fauna Conservation'',
  this project will \textbf{manage threats} to \textbf{create secure
  populations}, in \textbf{conservation estate}.
\end{itemize}




\subsection*{Expected collaborations}

\begin{itemize}
\itemsep1pt\parskip0pt\parsep0pt
\item
  Wheatbelt Region
\item
  Wheatbelt NRM
\end{itemize}


\subsection*{Proposed period of the project}
Oct. 3, 2016 -- Sept. 30, 2019



\subsection*{Staff time allocation }



\begin{longtabu} to \linewidth { |  X | X | X | X | }
\hline
\rowcolor{infobg}
Role & Year 1 & Year 2 & Year 3\\
\hline
\endhead



Scientist & 0.10 & 0.10 & 0.10\\



Technical & 0.05 & 0.05 & 0.05\\



Volunteer & 0.10 & 0.10 & 0.10\\



Collaborator & 0.10 & 0.10 & 0.10\\


\hline
\end{longtabu}



\subsection*{Indicative operating budget }



\begin{longtabu} to \linewidth { |  X | X | X | X | }
\hline
\rowcolor{infobg}
Source & Year 1 & Year 2 & Year 3\\
\hline
\endhead



Consolidated Funds (DPaW) & 11000 & 11200 & 11500\\



External Funding &  &  & \\


\hline
\end{longtabu}






%-----------------------------------------------------------------------------%
% Back matter
%\backmatter
\end{document}
%-----------------------------------------------------------------------------%

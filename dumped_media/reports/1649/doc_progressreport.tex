
\documentclass[version=last,
    paper=a4, % paper size
    10pt, % default font size
    usenames,
    dvipsnames,
    oneside, % ONLINE
    headings=openany, % open chapters on odd and even pages
    %toc=chapterentrywithdots, % Table of Contents style
    %BCOR=7mm, % PRINT Binding Correction
    %DIV=13, % typearea 161.54 mm x 228.46 mm, top margin 22.85 mm, inner margin 16.15 mm
    %DIV=14, % 165.00 233.36 21.21 15.00
    DIV=15 % 168.00 237.60 19.80 14.00
]{scrbook}
\usepackage{typearea}
\usepackage[automark,headsepline,footsepline]{scrlayer-scrpage} % Headers and footers

%%
%% Fonts, encoding, spacing, indentation
%%
\usepackage{txfonts}
\renewcommand{\familydefault}{\sfdefault} % Default to Sans Serif font
\usepackage[english]{babel}
\usepackage[T1]{fontenc}
\usepackage[utf8]{inputenc}

% Paragraph spacing
%\usepackage{parskip}    % Paragraph spacing
%\setlength{\parindent}{0em} % Don't indent paragraphs - ONLINE
%\setlength{\parskip}{1.3 ex plus 0.5ex minus 0.3ex} % 1-1.8 ex vertical space between paragraphs - ONLINE

% Spacing of headings
%\RedeclareSectionCommand[afterskip=3pt]{section} % less space after section
%\RedeclareSectionCommand[beforeskip=0cm]{subsection} % less space between HRule and project name
%\RedeclareSectionCommand[afterskip=0.1\baselineskip]{subsubsection} % less space after progressreport subheadings

% Table font size
\usepackage{etoolbox}
\AtBeginEnvironment{longtabu}{\footnotesize}{}{}

%%
%% Tables, columns, layout
%%
\usepackage{multicol}   % 2 col publications
\usepackage{pdflscape}  % Landscape pages
\usepackage{pdfpages}   % Include PDFs
\usepackage{hanging}    % hanging paragraphs for publications
%\usepackage{titletoc}   % Required for manipulating the table of contents
\setcounter{tocdepth}{2} % TOC list down to section
\usepackage{enumerate}  % Enumerations
\usepackage{enumitem}   % Enumerations
\usepackage{longtable}  % Multipage table
\usepackage{tabu}       %
\setlength{\tabulinesep}{1.5mm} % Consistent vertical spacing in tabu

%%
%% Graphics, images, colours
%%
\usepackage{graphicx} % embedded images
\usepackage{eso-pic} %
\usepackage{colortbl} % define custom named colours
\definecolor{RedFire}{RGB}{146,25,28}
\definecolor{ParksWildlife}{RGB}{0,85,144}
\definecolor{successbg}{RGB}{223,240,216}
\definecolor{errorbg}{RGB}{242,222,222}
\definecolor{warningbg}{RGB}{252,248,227}
\definecolor{infobg}{RGB}{217,237,247}
\definecolor{muted}{RGB}{153,153,153}
\definecolor{success}{RGB}{70,136,71}
\definecolor{error}{RGB}{185,74,72}
\definecolor{warning}{RGB}{192,152,83}
\definecolor{info}{RGB}{58,135,173}

\definecolor{required}{RGB}{192,152,83}
\definecolor{requiredbg}{RGB}{252,248,227}
\definecolor{denied}{RGB}{185,74,72}
\definecolor{deniedbg}{RGB}{242,222,222}
\definecolor{granted}{RGB}{70,136,71}
\definecolor{grantedbg}{RGB}{223,240,216}
\definecolor{not reqiured}{RGB}{153,153,153}
\definecolor{not requiredbg}{RGB}{255,255,255}

\usepackage{tikz} % Drawing
\usetikzlibrary{arrows,shapes,positioning,shadows,trees}

%%
%% Links, URLs
%%
\usepackage[
    linktoc=all,
    %colorlinks=false,  %PRINT
    colorlinks=true, % ONLINE
    linkcolor=RedFire, % ONLINE
    urlcolor=ParksWildlife, % ONLINE
    pdftitle=Progress Report SP 2009-009 (FY 2015-2016)
]{hyperref}

% Black magic to linebreak URLs
\usepackage{url}
\makeatletter
\g@addto@macro{\UrlBreaks}{\UrlOrds}
\makeatother

%%
%% Custom macros
%%
% Thick Horizontal rule
\newcommand{\HRule}{\vspace{8mm}\\\noindent\rule{\linewidth}{0.1pt}}

% Custom Tikz node for SDS diagram
\newcommand\mynode[6][]{
    \node[#1] (#2){
        \parbox{#3\relax}{
            \begin{center}
            \textbf{#4}\\
            #5\\
            \footnotesize{#6}
            \end{center}}};}



%-----------------------------------------------------------------------------%
% Headers and Footers
\automark{section}
\ohead{\href{http://sdis.dpaw.wa.gov.au/documents/progressreport/1649/}{Progress Report SP 2009-009
}}
\chead{\href{http://sdis.dpaw.wa.gov.au}{SDIS}} % center header ONLINE
\ihead{\href{http://sdis.dpaw.wa.gov.au}{
        \includegraphics[scale=0.4]{/mnt/projects/sdis/staticfiles/img/logo-dpaw.png}}}
\ifoot{\textbf{Printed}~Mon, 4 Jul 2016 16:16:17 +0800} % inner/left footer
\cfoot{} % center footer
\ofoot{\pagemark} % outer/right footer
\pagestyle{scrheadings}
\setkomafont{pageheadfoot}{\normalfont}

%-----------------------------------------------------------------------------%
\begin{document}
\raggedbottom

%-----------------------------------------------------------------------------%
% Title page
\subject{Progress Report SP 2009-009
}
\title{Taxonomic review and floristic studies of the benthic marine algae of
north-western Australian and floristic surveys of Western Australian
marine benthic algae
}
\subtitle{Plant Science and Herbarium
}
\author{}
\publishers{\small
    \subsection*{Project Core Team}
\begin{tabu} {X X}
\textbf{Supervising Scientist} & John Huisman
\\
\textbf{Data Custodian} & John Huisman
\\
\textbf{Site Custodian} & John Huisman
\\
\end{tabu}


    \subsection*{Project status as of July 4, 2016, 4:16 p.m.}
\begin{tabu} {X X}
& Approved and active
\\
\end{tabu}

    
\subsection*{Document endorsements and approvals as of July 4, 2016, 4:16 p.m.}
\begin{tabu} {X X}

%\rowcolor{grantedbg}
    \textbf{Project Team} & 
    \textcolor{granted}{ granted}\\

%\rowcolor{grantedbg}
    \textbf{Program Leader} & 
    \textcolor{granted}{ granted}\\

%\rowcolor{grantedbg}
    \textbf{Directorate} & 
    \textcolor{granted}{ granted}\\

\end{tabu}



}
\uppertitleback{}
\lowertitleback{}
\date{}

%-----------------------------------------------------------------------------%
% Front matter
\frontmatter
\maketitle
%-----------------------------------------------------------------------------%
% Main matter
\mainmatter

\section*{Taxonomic review and floristic studies of the benthic marine algae of
north-western Australian and floristic surveys of Western Australian
marine benthic algae
}

J Huisman, C Parker


\section*{Context}
This project involves systematic research into a poorly known group of
Western Australian plants and is directly relevant to the Department's
nature conservation strategy. It includes floristic studies of the
marine plants of several existing/proposed marine parks and also areas
of commercial interest (Shoalwater, Marmion, Ningaloo, Dampier
Archipelago, Barrow Island, Montebello Islands, Rowley Shoals, Scott
Reef, Maret Islands, etc.) to provide baseline information that will
enable a more comprehensive assessment of the Western Australian marine
biodiversity.



\section*{Aims}
\begin{itemize}
\itemsep1pt\parskip0pt\parsep0pt
\item
  Collect, curate and establish a collection of marine plants
  representative of the Western Australian marine flora, supplementing
  the existing Western Australian Herbarium collection.
\item
  Assess the biodiversity of the marine flora of Western Australia,
  concentrating initially on the poorly-known flora of the tropics.
\item
  Prepare a marine Flora for north-western Australia, documenting this
  biodiversity.
\end{itemize}



\section*{Progress}
\begin{itemize}
\itemsep1pt\parskip0pt\parsep0pt
\item
  Substantial progress has been made towards finalising the second book
  in the series, \emph{Algae of Australia: Marine Benthic Flora of
  North-western Australia, 2. The Red Algae}. The majority of the text
  and illustrations have been prepared and edited/formatted by
  Australian Biological Resources Study in readiness for
  submission/publication in 2016. This book will include descriptions of
  several hundred species, over 70 of which are new to science.
\item
  Participation in a field survey to Coral Bay by John
  Huisman,~resulting in numerous new collections that have~added several
  hundred specimens to the herbarium holdings. These collections include
  new species of the red algae \emph{Aphanta}~(also representing a new
  generic record for Australia) and \emph{Kallymenia}, which~will be
  described in the abovementioned book.
\item
  Several additional major papers have been published concerning aspects
  of the north-western Australian marine flora, including~the
  description of a new genus \emph{Rhytimenia}, based on a species
  collected from Indonesia in 1899 but not recollected until recent
  surveys of Ashmore Reef by John Huisman.
\item
  Publication of~a paper describing the presence of \emph{Codium tenue},
  a rare South African species, in Walpole/Nornalup Inlet.~
\item
  450 new specimens of marine benthic algae have been added to the
  Herbarium collection.
\end{itemize}



\section*{Management implications}
\begin{itemize}
\itemsep1pt\parskip0pt\parsep0pt
\item
  Easier identification of marine plant species leads to a more
  comprehensive understanding of their conservation status, recognition
  of regions with high biodiversity and/or rare species, recognition of
  rare species, recognition of potentially introduced species, and
  discrimination of closely-related native species
\item
  Enhanced knowledge of marine plant species allows a more accurate
  assessment of management needs and potential impacts of environmental
  change, including change conferred by resource~developments.
\end{itemize}



\section*{Future directions}
\begin{itemize}
\itemsep1pt\parskip0pt\parsep0pt
\item
  Further surveys in 2016-17 of the marine algae of Western Australian
  including at Coral Bay, the Capes region in the south-west of Western
  Australia, and sites in the Perth region including Cape Peron.
\item
  Publication of papers describing new and existing genera, species and
  other categories; contributions to FloraBase.~
\item
  Finalise production of \emph{Algae of Australia:} The \emph{Marine
  Benthic Flora of North-western Australia, 2. The Red Algae}, to be
  published by the Australian Biological Resources Study in 2016.
\item
  Prepare a paper describing a new species of the brown alga
  \emph{Rosenvingea}, collected in early 2016 from Cape Peron.~
\item
  Finalise a paper (in collaboration with international colleagues)
  describing several new genera in~the family Kallymeniaceae.~
\end{itemize}



%-----------------------------------------------------------------------------%
% Back matter
%\backmatter
\end{document}
%-----------------------------------------------------------------------------%



\documentclass[version=last,
    paper=a4,                               % paper size
    10pt,                                   % default font size
    dvipsnames,
    % twoside,                                % PRINT Binding Correction
    oneside,                              % ONLINE
    headings=openany,                       % open chapters on odd and even pages
    open=any,
    BCOR=7mm,                               % PRINT Binding Correction
    %DIV=13,    % typearea 161.54mm x 228.46mm, top 22.85mm, inner 16.15mm
    %DIV=14,    % 165.00 233.36 21.21 15.00
    DIV=15,     % 168.00 237.60 19.80 14.00
    % toc=chapterentrywithdots              % Table of Contents style
]{scrbook}
\usepackage{typearea}


%------------------------------------------------------------------------------%
% Headers and footers
%------------------------------------------------------------------------------%
\usepackage[automark,headsepline,footsepline,plainfootsepline]{scrlayer-scrpage}
\automark*[section]{}
\addtokomafont{pageheadfoot}{\normalfont\footnotesize\sffamily} % Don't italicise
\renewcommand{\chaptermark}[1]{\markleft{#1}{}}     % Chapter: suppress numbering
\renewcommand{\sectionmark}[1]{\markright{#1}{}}    % Section: suppress numbering

% Header (inner, center, outer)
% \ihead{\href{http://sdis.dbca.wa.gov.au}{\textbf{Project Plan SP 2017-001}}}
%\chead{\href{http://sdis.dbca.wa.gov.au}{Science Directorate Information System}}
% \ohead{\href{https://www.dbca.wa.gov.au/science/10-biodiversity-and-conservation-science}{
% \includegraphics[height=8mm, keepaspectratio]{/usr/src/app/staticfiles/img/logo-dbca-bcs.jpg}}}

% Footer (inner, center, outer)
% \ifoot{\RaggedRight\leftmark}                       % Chapter
% \cfoot{\RaggedLeft\rightmark}                       % Section
% \ofoot[\bfseries\thepage]{\bfseries\thepage}        % Page number (also [plain])


%------------------------------------------------------------------------------%
% Fonts, encoding
%------------------------------------------------------------------------------%
%\usepackage{avant}             % Use the Avantgarde font for headings
\usepackage{txfonts}
\usepackage{mathptmx}
\usepackage{gensymb}            % provides \textdegree
\renewcommand{\familydefault}{\sfdefault} % Default to Sans Serif font
\usepackage{microtype}          % Slightly tweak font spacing for aesthetics
\usepackage[english]{babel}
\usepackage[utf8]{inputenc}  % Allow letters with accents
\usepackage[utf8]{luainputenc}  % Allow letters with accents
\usepackage[T1]{fontenc}        % Use 8-bit encoding that has 256 glyphs
\usepackage{textcomp}
\usepackage[explicit]{titlesec}           % Customise of titles
%\DeclareUnicodeCharacter{0080}{\textregistered}
\DeclareUnicodeCharacter{00B0}{\textdegree}

%------------------------------------------------------------------------------%
% Tables, columns, layout
%------------------------------------------------------------------------------%
\usepackage{etoolbox}
\AtBeginEnvironment{longtabu}{\footnotesize}{}{}  % Table font size
\usepackage{booktabs}           % Required for nicer horizontal rules in tables
\usepackage{multicol}           % 2 col publications
\usepackage{pdflscape}          % Landscape pages
\usepackage{pdfpages}           % Include PDFs
\usepackage{hanging}            % hanging paragraphs for publications
%\usepackage{titletoc}          % Manipulate the table of contents
\setcounter{tocdepth}{2}        % TOC list down to section
\usepackage{enumerate}          % Enumerations
\usepackage{enumitem}           % Enumerations
\usepackage{longtable}          % Multipage table
\usepackage{tabu}               %
\setlength{\tabulinesep}{1.5mm} % Consistent vertical spacing in tabu
\newcommand{\HRule}{\vspace{8mm}\noindent\rule{\linewidth}{0.1pt}}
\usepackage[export]{adjustbox}  % minipage, image frame


%------------------------------------------------------------------------------%
% Graphics, images, colours
%------------------------------------------------------------------------------%
\usepackage{graphicx} % embedded images
\usepackage{wrapfig}  % wrap figures in text
\usepackage{caption}  % allow unnumbered captions
\usepackage{eso-pic} % Required for specifying an image background in the title page
\usepackage{colortbl} % define custom named colours
\usepackage{xstring} % Conditionals
\usepackage{transparent} % Allow transparent images

\definecolor{RedFire}{RGB}{146,25,28}
% Following PICA branding guidelines
% https://dpaw.sharepoint.com/Divisions/pica/Documents/Branding%20guidelines.pdf
\definecolor{dpawblue}{RGB}{35,97,146}          % Pantone 647
\definecolor{dpaworange}{RGB}{237,139,0}        % Pantone 144
\definecolor{dpawgreen}{RGB}{116,170,80}        % Pantone 7489
\definecolor{dpawred}{RGB}{124,46,44}           % Paul's suggestion

% bootstrap colours
\definecolor{successbg}{RGB}{223,240,216}
\definecolor{errorbg}{RGB}{242,222,222}
\definecolor{warningbg}{RGB}{252,248,227}
\definecolor{infobg}{RGB}{217,237,247}
\definecolor{muted}{RGB}{153,153,153}
\definecolor{success}{RGB}{70,136,71}
\definecolor{error}{RGB}{185,74,72}
\definecolor{warning}{RGB}{192,152,83}
\definecolor{info}{RGB}{58,135,173}

% SDIS approval colours
\definecolor{required}{RGB}{192,152,83}
\definecolor{requiredbg}{RGB}{252,248,227}
\definecolor{denied}{RGB}{185,74,72}
\definecolor{deniedbg}{RGB}{242,222,222}
\definecolor{granted}{RGB}{70,136,71}
\definecolor{grantedbg}{RGB}{223,240,216}
\definecolor{notrequired}{RGB}{153,153,153}
\definecolor{notrequiredbg}{RGB}{255,255,255}

\usepackage{tikz} % Drawing
\usetikzlibrary{arrows,shapes,positioning,shadows,trees}


%------------------------------------------------------------------------------%
% Hyperlinks
%------------------------------------------------------------------------------%
\usepackage[open=true]{bookmark}
\usepackage{nameref}
\usepackage{ifxetex,ifluatex}
\ifxetex
  \usepackage[
    setpagesize=false,        % page size defined by xetex
    unicode=false,            % unicode breaks when used with xetex
    xetex]{hyperref}
\else
  \usepackage[unicode=true]{hyperref}
\fi

\hypersetup{
  backref=true,
  pagebackref=true,
  hyperindex=true,
  breaklinks=true,
  urlcolor=dpawblue,
  bookmarks=true,
  bookmarksopen=false,
  pdfauthor={Biodiversity and Conservation Science, Department of Biodiversity, Conservation and Attractions, WA},
  pdftitle=Project Plan SP 2017-001
,
  colorlinks=true,
  linkcolor=dpawblue,
  pdfborder={0 0 0}}

\urlstyle{same}                         % don't use monospace font for urlstyle


%------------------------------------------------------------------------------%
% Black magic to linebreak URLs
%------------------------------------------------------------------------------%
\usepackage{url}
\makeatletter\g@addto@macro{\UrlBreaks}{\UrlOrds}\makeatother
\Urlmuskip=0mu plus 1mu


%------------------------------------------------------------------------------%
% Fix latex errors
%------------------------------------------------------------------------------%
\providecommand{\tightlist}{\setlength{\itemsep}{0pt}\setlength{\parskip}{0pt}}

% copy-pasted HTML <span> in SDIS fields becomes \text{} in tex source
\providecommand{\text}{}


%------------------------------------------------------------------------------%
% Custom Tikz node for SDS diagram
%------------------------------------------------------------------------------%
\newcommand\mynode[6][]{
  \node[#1] (#2){
    \parbox{#3\relax}{
      \begin{center}
      \textbf{#4}\\
      #5\\
      \footnotesize{#6}
      \end{center}
    }};}


%------------------------------------------------------------------------------%
% Custom no-pagebreaks-environment
%------------------------------------------------------------------------------%
\newenvironment{absolutelynopagebreak}
  {\par\nobreak\vfil\penalty0\vfilneg\vtop\bgroup}
  {\par\xdef\tpd{\the\prevdepth}\egroup\prevdepth=\tpd}


%------------------------------------------------------------------------------%
% Remove the header from odd empty pages at the end of chapters
%------------------------------------------------------------------------------%
\makeatletter
\renewcommand{\cleardoublepage}{
\clearpage\ifodd\c@page\else
\hbox{}
\vspace*{\fill}
\thispagestyle{empty}
\newpage
\fi}


%----------------------------------------------------------------------------------------
%  Page flow control
%----------------------------------------------------------------------------------------
%\widowpenalty=10000
%\clubpenalty=10000
%\vbadness=1200
%\hbadness=11000


%----------------------------------------------------------------------------------------
%   CHAPTER HEADINGS
%----------------------------------------------------------------------------------------
\newcommand{\thechapterimage}{}
\newcommand{\chapterimage}[1]{\renewcommand{\thechapterimage}{#1}}

% Numbered chapters with mini tableofcontents
\def\thechapter{\arabic{chapter}}
\def\@makechapterhead#1{
%\thispagestyle{plain}
{\centering \normalfont\sffamily
\ifnum \c@secnumdepth >\m@ne
\if@mainmatter
\startcontents
\begin{tikzpicture}[remember picture,overlay]
\node at (current page.north west)
{\begin{tikzpicture}[remember picture,overlay]
\node[anchor=north west,inner sep=0pt] at (0,0) {
\includegraphics[width=\paperwidth,height=0.5\paperwidth]{\thechapterimage}};
%------------------------------------------------------------------------------%
% Small contents box in the chapter heading
% Mini TOC background box
%\fill[color=dpawblue!10!white,opacity=.2] (1cm,0) rectangle (
%  3.5cm, % Mini TOC box width
%  -3.5cm % Mini TOC box height
%);
% Mini TOC text content
%\node[anchor=north west] at (1.1cm,.35cm) {
%  \parbox[t][8cm][t]{6.5cm}{
%    \huge\bfseries\flushleft
%    \printcontents{l}{1}{
%    \setcounter{tocdepth}{1}                   % Mini TOC level depth
%    }
% }
%};
%------------------------------------------------------------------------------%
% Chapter heading
\draw[anchor=west] (5cm,-9cm) node [
rounded corners=20pt,
fill=dpawblue!10!white,
text opacity=1,
draw=dpawblue,
draw opacity=1,
line width=1.5pt,
fill opacity=.2,
inner sep=12pt]{
    \huge\sffamily\bfseries\textcolor{black}{
      \thechapter. #1\strut\makebox[22cm]{}
    }
};
\end{tikzpicture}};
\end{tikzpicture}}
\par\vspace*{240\p@}                            % Push text below chapter image
\fi
\fi}

%------------------------------------------------------------------------------%
% Unnumbered chapters without mini tableofcontents
%------------------------------------------------------------------------------%
\def\@makeschapterhead#1{
%\thispagestyle{plain}
{\centering \normalfont\sffamily
\ifnum \c@secnumdepth >\m@ne
\if@mainmatter
\begin{tikzpicture}[remember picture,overlay]
\node at (current page.north west)
{\begin{tikzpicture}[remember picture,overlay]
\node[anchor=north west,inner sep=0pt] at (0,0) {
  \includegraphics[width=\paperwidth,height=0.5\paperwidth]{\thechapterimage}};
% Mini TOC background box
%\fill[color=dpawblue!10!white,opacity=.2] (1cm,0) rectangle (
%  3.5cm,                                       % Mini TOC box width
%  -3.5cm                                       % Mini TOC box height
%);
% Mini TOC text content
%\node[anchor=north west] at (1.1cm,.35cm) {
%  \parbox[t][8cm][t]{6.5cm}{
%    \huge\bfseries\flushleft
%    \printcontents{l}{1}{
%    \setcounter{tocdepth}{1} % Mini TOC level depth
%    }
%}
%};
\draw[anchor=west] (5cm,-9cm) node [rounded corners=20pt,
  fill=dpawblue!10!white,fill opacity=.6,inner sep=12pt,text opacity=1,
  draw=dpawblue,draw opacity=1,line width=1.5pt]{
  \huge\sffamily\bfseries\textcolor{black}{#1\strut\makebox[22cm]{}}};
\end{tikzpicture}};
\end{tikzpicture}}
\par\vspace*{240\p@}
\fi
\fi
}
\makeatother



\usepackage[automark,headsepline,footsepline,plainfootsepline]{scrlayer-scrpage}
\automark*[section]{}
\addtokomafont{pageheadfoot}{\normalfont\footnotesize\sffamily} % Don't italicise
\renewcommand{\chaptermark}[1]{\markleft{#1}{}}     % Chapter: suppress numbering
\renewcommand{\sectionmark}[1]{\markright{#1}{}}    % Section: suppress numbering

% Header (inner, center, outer)
\ihead{\href{http://sdis.dbca.wa.gov.au/documents/projectplan/1796/}{Project Plan SP 2017-001}}
%\chead{\href{http://sdis.dbca.wa.gov.au}{Science Directorate Information System}}
\ohead{\href{https://www.dbca.wa.gov.au/science/10-biodiversity-and-conservation-science}{
\includegraphics[height=6mm, keepaspectratio]{/usr/src/app/staticfiles/img/logo-dbca-bcs.jpg}}}
% Footer (inner, center, outer)
\ifoot{\textbf{Printed}~Tue, 6 Oct 2020 12:37:21 +0800} % inner/left footer
\cfoot{}
\ofoot[\bfseries\thepage]{\bfseries\thepage}        % Page number (also [plain])


\pagestyle{scrheadings}
\setkomafont{pageheadfoot}{\normalfont}

%-----------------------------------------------------------------------------%
\begin{document}
\raggedbottom

%-----------------------------------------------------------------------------%
% Title page
\subject{Project Plan SP 2017-001
}
\title{Understanding and reducing python predation of the endangered Gilbert's
potoroo
}
\subtitle{Animal Science
}
\author{}
\publishers{\small
    \subsection*{Project Core Team}
\begin{tabu} {X X}
\textbf{Supervising Scientist} & David Pearson
\\
\textbf{Data Custodian} & David Pearson
\\
\textbf{Site Custodian} & 
\\
\end{tabu}


    \subsection*{Project status as of Oct. 6, 2020, 12:37 p.m.}
\begin{tabu} {X X}
& Approved and active
\\
\end{tabu}

    
\subsection*{Document endorsements and approvals as of Oct. 6, 2020, 12:37 p.m.}
\begin{tabu} {X X}

%\rowcolor{grantedbg}
    \textbf{Project Team} & 
    \textcolor{granted}{ granted}\\

%\rowcolor{grantedbg}
    \textbf{Program Leader} & 
    \textcolor{granted}{ granted}\\

%\rowcolor{grantedbg}
    \textbf{Directorate} & 
    \textcolor{granted}{ granted}\\

%\rowcolor{grantedbg}
    \textbf{Biometrician} & 
    \textcolor{granted}{ granted}\\

%\rowcolor{not requiredbg}
    \textbf{Herbarium Curator} & 
    \textcolor{not required}{ not required}\\

%\rowcolor{grantedbg}
    \textbf{Animal Ethics Committee} & 
    \textcolor{granted}{ granted}\\

\end{tabu}



}
\uppertitleback{}
\lowertitleback{}
\date{}

%-----------------------------------------------------------------------------%
% Front matter
\frontmatter
\maketitle
%-----------------------------------------------------------------------------%
% Main matter
\mainmatter



\section*{Understanding and reducing python predation of the endangered Gilbert's
potoroo
}



\subsection*{Biodiversity and Conservation Science Program}

Animal Science




\subsection*{Departmental Service}

Service 6: Conserving Habitats, Species and Communities


\subsection*{Project Staff}
\begin{tabu} {X X X}
%\rowcolor{infobg}
\textbf{Role} & \textbf{Person} & \textbf{Time allocation (FTE)}\\

Supervising Scientist & David Pearson & 0.1\\

Technical Officer & Stephanie Hill & 0.1\\

\end{tabu}




\subsection*{Related Science Projects}

Gilbert's Potoroo SP 1996-008


\subsection*{Proposed period of the project}
March 27, 2017 -- June 30, 2019



\section*{Relevance and Outcomes}


\subsection*{Background}

Carpet Pythons are predators of a range of threatened mammal fauna
(Pearson 2002, Pearson et al. 2002), including the endangered Gilbert's
Potoroo (draft Recovery Plan, Department of Parks and Wildlife 2016).
Python predation, especially when populations are small or under
pressure from drought or other predators, can potentially reduce adult
survival as well as curtailing recruitment. This is particularly
problematic when threatened mammal populations are highly confined by
availability of habitat or if housed in enclosures for breeding or
conservation. Current ``predator proof'' fences used to protect WA
threatened mammal fauna, while effective at reducing or eliminating
predation by foxes and feral cats, are likely to have little or no
effect on levels of python predation.

The draft Gilbert's Potoroo Recovery Plan (Department of Parks and
Wildlife 2016) identified python predation as a significant threat to
the growth of the potoroo population in the Waychinicup NP enclosure. A
total of 49 potoroos have been placed in this enclosure since 2010 and
there have been 8 known python predation events of potoroos with tail
radio-transmitters. Since only a proportion of potoroos are fitted with
tail transmitters at any time, the number actually taken by pythons is
likely to be much higher. Tail transmitters typically only stay attached
to the tail for 4-6 weeks.

The estimated size of the potoroo population inside the enclosure in May
2016 was just 12 individuals. In a recent review of options following
the Two Peoples Bay fire, it was stated that without management
intervention, python predation was likely to cause their extinction
within the enclosure (Tony Friend, options paper, December 2015). It
appears that python predation is limiting population growth and hence
the production of progeny for translocation to other sites.

The potoroo population at Two Peoples Bay was believed to be relatively
stable between 2001 and 2012 (31 animals in 2008), but then declined to
around one-third of its previous level (9 in March 2014, Friend 2016).
Low rainfall rather than python predation was probably the main cause,
but nonetheless predation pressure on a low population will stop or slow
recruitment and recovery. Despite consistent production of young
potoroos, the Waychinicup population has not shown strong growth. The
low survival of founders compared with the Bald Island translocation,
together with the generally good condition of individuals, indicates
that a significant level of predation is likely to be a major
contributing factor to the lack of population growth (Department of
Parks and Wildlife 2016)..

Carpet Python Biology

Carpet pythons are known predators of a range of native mammals up to
the size of tammar wallabies (around 3 kg). They hunt primarily by
ambush and since prey are swallowed whole, there is pronounced niche
partitioning between juvenile and adult pythons, and in some areas
between the sexes (Pearson et al. 2002a). Carpet pythons typically hunt
in the warmer months to ensure timely digestion of prey and may cease to
eat and remain inactive for many months in areas with cool winters such
as Dryandra (Pearson, Shine and Williams 2005). Prey are taken by large
pythons on the ground, while juveniles hunt on the ground as well as
from raised positions in shrubs or even in the canopy of trees. However
observations at Waychinicup indicate that Gilbert's Potoroos are taken
by pythons all year round (Tony Friend, pers. comm.), which suggests
that pythons are able to maintain warm enough body temperatures at
Waychinicup to digest these relatively small meals (\textless{} 1kg).\\
Telemetered pythons typically move around 100m a week in the summer
season between ambush sites at Garden Island and Dryandra. Home ranges
averaged around 17 ha; however adult males travel widely in the breeding
season to locate receptive females. Exclusive territories are not
apparent, with adults and juveniles having home ranges that are
overlapping (Pearson et al. 2005). Growth rates are strongly driven by
prey availability and there is pronounced sexual size dimorphism with
males rarely exceeding 1.5 m in length and 1 kg in weight, while adult
females can reach up to 2.5 m and 4 kg (Pearson, Shine and Williams
2002b).

Python predation risk to Potoroos

The relatively small size of Gilbert's potoroos means that they are
vulnerable to predation from a range of python cohorts -- sub-adult and
adult females and adult males. Predation is most likely in summer months
when warmer temperatures allow more frequent feeding and growth. Large
adult female pythons involved in reproduction do not to feed during egg
development and incubation; this is likely to constitute around one
quarter to one third of the adult female population in any given summer.
However, once egg incubation is completed in March, these females have
lost up to 50\% of their post-oviposition weight, are hungry and will
attempt to rapidly regain weight by feeding on whatever prey are
available.

The home range is relatively small and non-exclusive, so densities of
carpet pythons may be high if there is suitable habitat with abundant
prey. Inside the 380 ha Waychinicup enclosure there is likely to be a
substantial number of resident pythons, with some inflow and outflow by
juveniles establishing new home ranges, as well as adults looking for
prey; and in the breeding season, males searching for receptive females.

An options paper for the control and removal of pythons in the
Waychinicup enclosure was prepared by Pearson and Friend (2016). It
listed fencing, the removal of resident pythons by hand, trapping
pythons and using radio-telemetry of adult males in the breeding season
to locate female pythons. Retro-fitting the existing Waychinicup fence
to make it python proof was considered by senior staff to be too
expensive, so this proposal focuses on the location and removal of
pythons within and around the enclosure to reduce the level of python
predation on Gilbert's Potoroo.

In addition, it may be possible to reduce python predation on the
existing Mt Gardner population as it recovers from the November 2015
bushfire. In this area, the fitting of radio-transmitters to male
pythons could assist in the location of adult females which are likely
to be the most significant cohort preying on potoroos.

~




\subsection*{Aims}

To determine the most effective ways to locate, trap and remove carpet
pythons from in and and around Gilbert's Potoroo populations and so
significantly reduce the current level of python predation.

~




\subsection*{Expected outcome}

Reduced python predation of the critically endangered Gilbert's Potoroo,
to prevent its extinction within the Waychinicup enclosure and so that
recruitment is improved and more potoroos are available for
translocations to other sites.\\
Since carpet pythons are important predators of a number of threatened
mammals, the project would have applications beyond potoroo conservation
and could assist with reducing python predation in other enclosures and
even in field situations with wild populations if required.




\subsection*{Knowledge transfer}

Conservation managers and planners responsible for managing small
populations of threatened species that may be threatened by reptile
predation. This would include Parks and Wildlife operational staff and
managers, but also be potentially useful in other jurisdictions.~




\subsection*{Tasks and Milestones}

Application for animal ethics approval January 2017- Completed, animal
ethics approved.

2. Acquistion of python transmitters- underway, transmitters to arrive
by February 2017

3. Capture and implantation of male pythons- commencing February 2017
and ongoing for life of project

4. Briefing and training of South Coast field staff and Science staff-
February 2017

5. Design and production of python trap boxes- completed by April 2017

6. Deployment of trap boxes at Waychinicup and Garden Island and
monitoring effectiveness- April 2017

7. Trap box monitoring- April to June; assessment of effectiveness by
July 2017

8. Radio-telemetry monitoring and capture of pythons; ongoing for
project, but intensified during the breeding season, November to
December

9. Spring trap trials from October to December; assessment of
effectiveness by January 2018

10. Annual report January 2018 summarising the first year of project's
operation; annual animal ethics report

11. Continuing trap tests and modifications if required at Waychinicup
and Garden Island January-March 2018.

12. Continuing occasional radio-telemetry of males- ongoing

13. Intensified radio-tracking of male pythons to locate female pythons-
November-December 2018.

14. Annual report January 2019, summarising second year of project;
annual animal ethics report.

15. Continuation of trapping of pythons if effective trap developed and
continued radio-telemetry of males if successful in locating females.

16. Assessment of the project and final report January 2020.

17. Preparation of a refereed scientific papers on trap design for
pythons, the characteristics of python predation on Gilbert's Potoroos,
natural history of pythons on the south coast WA including breeding
observations and diet; implications for the management of other
threatened species impacted by native predators. Drafts complete by June
2021.




\subsection*{References}

Courtenay, J. and Friend, T (2004). Gilbert's Potoroo (Potorous
gilberti) Recovery Plan. Wildlife Management Program No. 32, Department
of Conservation and Land Management, Perth.

Department of Parks and Wildlife (2016 draft). Gilbert's Potoroo
(Potorous gilberti) Recovery Plan. Wildlife Management Program (number
not yet allocated). Department of Parks and Wildlife, Perth.

Friend, T. (2016). Translocation Proposal- Gilbert's Potoroo (Potorous
gilbertii). Bald Island and Two Peoples Bay to Michaelmas Island
February 2016. Department of Parks and Wildlife, Science Division
Albany.

Friend, T. and Button, T. (2016). Predation of Gilbert's Potoroos by
carpet pythons. Unpublished report, Department of Parks and Wildlife,
Perth.

Pearson, D. (2002). Ecology and Conservation of South-western Carpet
Pythons (Morelia spilota imbricata). PhD thesis, University of Sydney

Pearson, D. and Friend, T. (2016). Reducing predation of Gilbert's
Potoroos by Carpet Pythons inside the Waychinicup enclosure. Unpublished
internal report, Department of Parks and Wildlife.

Pearson, D., Shine, R. and Williams, A. (2002a). Sex-specific niche
partitioning and sexual size dimorphism in Australian pythons (Morelia
spilota imbricata). Biological Journal of the Linnean Society 77:
113-125.

Pearson, D., Shine, R. and Williams, A.(2002b). Geographic variation in
sexual size dimorphism within a single snake species (Morelia spilota,
Pythonidae). Oecologia 131: 418-426.

Pearson, D., Shine, R. and Williams, A. (2005). Spatial ecology of a
threatened python (Morelia spilota imbricata) and the effects of
anthropogenic habitat change. Austral Ecology 30: 261-274.



\section*{Study design}


\subsection*{Methodology}

This project is essentially an adaptive research and management
initiative to reduce python predation of Gilbert's Potoroos as quickly
as possible. Several methods will be investigated and their
effectiveness determined.

Searching and hand capture- to be undertaken by Albany-based Science and
Regional staff during normal operations within the enclosure. Specific
targeted searches will be undertaken during the spring and summer months
by staff and volunteers (predominantly amateur herpetologists).\\
Body parameters will be recorded of captured pythons (SVL, gape size)
and the contents of the lower intestinal tract gently manipulated to
collect faecal samples. These will be analysed using hair, claws and
bone in the pellets and compared with python characteristics. Pythons
will then be released \textgreater{} 3 km from the enclosure.

Trap boxes using a heat pad and animal odour to attact pythons- A small
portable box design will be developed with a solar-powered heat pad. Ten
of these will be deployed in Waychinicup and another ten on Garden
Island and their effectiveness determine by regular checks by staff and
remote cameras. Designs may be modified as required and then tested
against earlier designs in which cause simple visitation metrics will be
used for Chi-squared testing.

Radio-telemetry of male pythons to locate females- Male pythons will be
implanted with Holohil transmitters (life 3 months-2 years depending on
python size) and tracked intensively during the breeding season to
locate adult females. Radio-telemetry data will be stored in Access
tables and imported into Ranges to examine daily movement rates and
mating ranges of adult males. These data will be compared with existing
data for Garden Island and Dryandra via Chi-squared tests.

~




\subsection*{Biometrician's Endorsement}

granted



\section*{Data management}


\subsection*{No. specimens}






\subsection*{Herbarium Curator's Endorsement}

not required




\subsection*{Animal Ethics Committee's Endorsement}

granted




\subsection*{Data management}

Python data and radio-telemetry data will be stored in Access and Excel
and basic statistics and graph drawn in Excel. The program 'Ranges' will
be used to analyse telemetry data and map home ranges and movement
patterns.




\section*{Budget}

\section*{Consolidated Funds }



\begin{longtabu} to \linewidth { |  X | X | X | X | }
\hline
\rowcolor{infobg}
Source & Year 1 & Year 2 & Year 3\\
\hline
\endhead



FTE Scientist & 11000 & 11000 & 11000\\



FTE Technical & 14000 & 14000 & 14000\\



Equipment & 10000 & 4000 & 4000\\



Vehicle & 6000 & 6000 & 6000\\



Travel & 3000 & 3000 & 3000\\



Other (volunteer costs) & 1000 & 2000 & 2000\\



Total & 45000 & 40000 & 40000\\


\hline
\end{longtabu}



\section*{External Funds }



\begin{longtabu} to \linewidth { |  X | X | X | X | }
\hline
\rowcolor{infobg}
Source & Year 1 & Year 2 & Year 3\\
\hline
\endhead



Salaries, Wages, Overtime &  &  & \\



Overheads &  &  & \\



Equipment &  &  & \\



Vehicle &  &  & \\



Travel &  &  & \\



Other &  &  & \\



Total &  &  & \\


\hline
\end{longtabu}





%-----------------------------------------------------------------------------%
% Back matter
%\backmatter
\end{document}
%-----------------------------------------------------------------------------%

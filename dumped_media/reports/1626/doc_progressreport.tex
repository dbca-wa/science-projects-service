
\documentclass[version=last,
    paper=a4, % paper size
    10pt, % default font size
    usenames,
    dvipsnames,
    oneside, % ONLINE
    headings=openany, % open chapters on odd and even pages
    %toc=chapterentrywithdots, % Table of Contents style
    %BCOR=7mm, % PRINT Binding Correction
    %DIV=13, % typearea 161.54 mm x 228.46 mm, top margin 22.85 mm, inner margin 16.15 mm
    %DIV=14, % 165.00 233.36 21.21 15.00
    DIV=15 % 168.00 237.60 19.80 14.00
]{scrbook}
\usepackage{typearea}
\usepackage[automark,headsepline,footsepline]{scrlayer-scrpage} % Headers and footers

%%
%% Fonts, encoding, spacing, indentation
%%
\usepackage{txfonts}
\renewcommand{\familydefault}{\sfdefault} % Default to Sans Serif font
\usepackage[english]{babel}
\usepackage[T1]{fontenc}
\usepackage[utf8]{inputenc}

% Paragraph spacing
%\usepackage{parskip}    % Paragraph spacing
%\setlength{\parindent}{0em} % Don't indent paragraphs - ONLINE
%\setlength{\parskip}{1.3 ex plus 0.5ex minus 0.3ex} % 1-1.8 ex vertical space between paragraphs - ONLINE

% Spacing of headings
%\RedeclareSectionCommand[afterskip=3pt]{section} % less space after section
%\RedeclareSectionCommand[beforeskip=0cm]{subsection} % less space between HRule and project name
%\RedeclareSectionCommand[afterskip=0.1\baselineskip]{subsubsection} % less space after progressreport subheadings

% Table font size
\usepackage{etoolbox}
\AtBeginEnvironment{longtabu}{\footnotesize}{}{}

%%
%% Tables, columns, layout
%%
\usepackage{multicol}   % 2 col publications
\usepackage{pdflscape}  % Landscape pages
\usepackage{pdfpages}   % Include PDFs
\usepackage{hanging}    % hanging paragraphs for publications
%\usepackage{titletoc}   % Required for manipulating the table of contents
\setcounter{tocdepth}{2} % TOC list down to section
\usepackage{enumerate}  % Enumerations
\usepackage{enumitem}   % Enumerations
\usepackage{longtable}  % Multipage table
\usepackage{tabu}       %
\setlength{\tabulinesep}{1.5mm} % Consistent vertical spacing in tabu

%%
%% Graphics, images, colours
%%
\usepackage{graphicx} % embedded images
\usepackage{eso-pic} %
\usepackage{colortbl} % define custom named colours
\definecolor{RedFire}{RGB}{146,25,28}
\definecolor{ParksWildlife}{RGB}{0,85,144}
\definecolor{successbg}{RGB}{223,240,216}
\definecolor{errorbg}{RGB}{242,222,222}
\definecolor{warningbg}{RGB}{252,248,227}
\definecolor{infobg}{RGB}{217,237,247}
\definecolor{muted}{RGB}{153,153,153}
\definecolor{success}{RGB}{70,136,71}
\definecolor{error}{RGB}{185,74,72}
\definecolor{warning}{RGB}{192,152,83}
\definecolor{info}{RGB}{58,135,173}

\definecolor{required}{RGB}{192,152,83}
\definecolor{requiredbg}{RGB}{252,248,227}
\definecolor{denied}{RGB}{185,74,72}
\definecolor{deniedbg}{RGB}{242,222,222}
\definecolor{granted}{RGB}{70,136,71}
\definecolor{grantedbg}{RGB}{223,240,216}
\definecolor{not reqiured}{RGB}{153,153,153}
\definecolor{not requiredbg}{RGB}{255,255,255}

\usepackage{tikz} % Drawing
\usetikzlibrary{arrows,shapes,positioning,shadows,trees}

%%
%% Links, URLs
%%
\usepackage[
    linktoc=all,
    %colorlinks=false,  %PRINT
    colorlinks=true, % ONLINE
    linkcolor=RedFire, % ONLINE
    urlcolor=ParksWildlife, % ONLINE
    pdftitle=Progress Report SP 2012-008 (FY 2015-2016)
]{hyperref}

% Black magic to linebreak URLs
\usepackage{url}
\makeatletter
\g@addto@macro{\UrlBreaks}{\UrlOrds}
\makeatother

%%
%% Custom macros
%%
% Thick Horizontal rule
\newcommand{\HRule}{\vspace{8mm}\\\noindent\rule{\linewidth}{0.1pt}}

% Custom Tikz node for SDS diagram
\newcommand\mynode[6][]{
    \node[#1] (#2){
        \parbox{#3\relax}{
            \begin{center}
            \textbf{#4}\\
            #5\\
            \footnotesize{#6}
            \end{center}}};}



%-----------------------------------------------------------------------------%
% Headers and Footers
\automark{section}
\ohead{\href{http://sdis.dpaw.wa.gov.au/documents/progressreport/1626/}{Progress Report SP 2012-008
}}
\chead{\href{http://sdis.dpaw.wa.gov.au}{SDIS}} % center header ONLINE
\ihead{\href{http://sdis.dpaw.wa.gov.au}{
        \includegraphics[scale=0.4]{/mnt/projects/sdis/staticfiles/img/logo-dpaw.png}}}
\ifoot{\textbf{Printed}~Mon, 11 Jul 2016 13:18:39 +0800} % inner/left footer
\cfoot{} % center footer
\ofoot{\pagemark} % outer/right footer
\pagestyle{scrheadings}
\setkomafont{pageheadfoot}{\normalfont}

%-----------------------------------------------------------------------------%
\begin{document}
\raggedbottom

%-----------------------------------------------------------------------------%
% Title page
\subject{Progress Report SP 2012-008
}
\title{The Western Australian Marine Monitoring Program (WAMMP)
}
\subtitle{Marine Science
}
\author{}
\publishers{\small
    \subsection*{Project Core Team}
\begin{tabu} {X X}
\textbf{Supervising Scientist} & Kim Friedman
\\
\textbf{Data Custodian} & Florian W Mayer
\\
\textbf{Site Custodian} & Florian W Mayer
\\
\end{tabu}


    \subsection*{Project status as of July 11, 2016, 1:18 p.m.}
\begin{tabu} {X X}
& Approved and active
\\
\end{tabu}

    
\subsection*{Document endorsements and approvals as of July 11, 2016, 1:18 p.m.}
\begin{tabu} {X X}

%\rowcolor{grantedbg}
    \textbf{Project Team} & 
    \textcolor{granted}{ granted}\\

%\rowcolor{grantedbg}
    \textbf{Program Leader} & 
    \textcolor{granted}{ granted}\\

%\rowcolor{grantedbg}
    \textbf{Directorate} & 
    \textcolor{granted}{ granted}\\

\end{tabu}



}
\uppertitleback{}
\lowertitleback{}
\date{}

%-----------------------------------------------------------------------------%
% Front matter
\frontmatter
\maketitle
%-----------------------------------------------------------------------------%
% Main matter
\mainmatter

\section*{The Western Australian Marine Monitoring Program (WAMMP)
}

K Friedman, K Bancroft, G Shedrawi, T Holmes, M Rule, AR Halford, A
Kendrick, S Wilson, S Whiting


\section*{Context}
A state-wide system of marine protected areas is being established in
Western Australia as part of Australia's National Representative System
of Marine Protected Areas. Long-term monitoring of the condition of
environmental assets and social values is recognised as an integral
aspect of adaptive management. The Department's marine monitoring
program~is a state-wide, long-term, marine monitoring, evaluation and
reporting program that is being developed and implemented to increase
the efficiency and effectiveness of marine reserve and threatened marine
fauna conservation and management.



\section*{Aims}
\begin{itemize}
\itemsep1pt\parskip0pt\parsep0pt
\item
  Develop and implement a long-term monitoring program for Western
  Auslralia's marine parks and reserves and threatened marine fauna to
  facilitate and promote management effectiveness in the protection and
  conservation of marine biodiversity and related social values.
\item
  Conduct research projects that provide methodological information
  necessary for the implemetation of monitoring programs.
\item
  Communicate (through formal advice, peer reveiewed publications,
  presentations and the popular media) the findings of the monitoring
  programs to stakeholders and the general public where appropriate.
\end{itemize}



\section*{Progress}
\begin{itemize}
\itemsep1pt\parskip0pt\parsep0pt
\item
  A review was initiated into the scope and operations of the marine
  monitoring program to ensure that it is being conducted in an
  efficient manner, and in line with changes to reporting structuress
  and operational capacity.
\item
  Monitoring was undertaken for numerous biophysical assets (e.g.
  finfish, coral, seagrass, macroalgae, mangrove, penguins, turtles,
  little penguin, water quality, human use) across twelve marine
  reserves from Walpole Nornalup Inlets Marine Park in the south to
  Lalang-garram / Camden Sound Marine Park in the north.
\item
  Monitoring reports were provided to Marine Park Coordinators on the
  condition of biodiversity assets and the significance of pressures
  acting on them in 12 marine parks and reserves to inform adaptive
  management strategies and Departmental reporting processes.
\item
  Marine monitoring datasets were integrated into the Divisional CKAN
  data catalogue.
\item
  Training on monitoring protocols for ecological assets such as coral,
  fish, seagrass, mangroves and little penguins was provided to
  Departmental staff, interns and volunteers.
\item
  A paper on the distribution, abundance, diversity and habitat
  associations of fishes across a bioregion experiencing rapid coastal
  development was published in \emph{Estuarine, Coastal and Shelf
  Science}.
\item
  Presentations were made to regional staff, the Cockburn Sound
  Management Council, South West Catchments Council, science/management
  peers and the general public on findings and management significance
  of the marine monitoring program.
\item
  A presentation based on marine monitoring program data was made at the
  2016 \emph{International Coral Reef Symposium}.
\end{itemize}



\section*{Management implications}
\begin{itemize}
\itemsep1pt\parskip0pt\parsep0pt
\item
  The long-term marine monitoring program provides data that informs the
  evidence-based adaptive management of Western Australia's marine parks
  and reserves and threatened and specially protected marine fauna.
\item
  Monitoring data is collected on key ecological assets, the pressures
  acting on those assets and the management response. This performance
  assessment and adaptive management framework allows conservation
  managers to respond appropriately to changes as they become apparent,
  and to refine approaches to managing ecological assets based on
  rigorous scientific evidence.
\end{itemize}



\section*{Future directions}
\begin{itemize}
\itemsep1pt\parskip0pt\parsep0pt
\item
  Finalise and publish supporting documentation that describes the the
  aims and structure of the marine monitoring program, including
  rationale for the selection of monitoring indicators and methods for
  key ecological assets.
\item
  Continue the design and implementation of ecological asset monitoring
  across the marine reserve system, including at new and proposed
  reserves at Ngari Capes Marine Park and in WA's Kimberley region.
\item
  Continue to provide marine park managers with evidence-based knowledge
  of the condition of key ecological assets and the pressures acting on
  them to inform and assist the delivery of adaptive management.
\item
  Continue to provide the scientific knowledge required for the
  Department's marine parks and reserves reporting process.
\end{itemize}



%-----------------------------------------------------------------------------%
% Back matter
%\backmatter
\end{document}
%-----------------------------------------------------------------------------%


\documentclass[version=last, paper=a4, DIV=18, usenames, dvipsnames]{scrartcl}
\usepackage{txfonts}
\usepackage{pdflscape}
\usepackage{pdfpages}
\usepackage[english]{babel} % English language/hyphenation
%%% Bootstrap colors
\definecolor{RedFire}{RGB}{146,25,28}
\definecolor{ParksWildlife}{RGB}{0,85,144}
\definecolor{successbg}{RGB}{223,240,216}
\definecolor{errorbg}{RGB}{242,222,222}
\definecolor{warningbg}{RGB}{252,248,227}
\definecolor{infobg}{RGB}{217,237,247}
\definecolor{muted}{RGB}{153,153,153}
\definecolor{success}{RGB}{70,136,71}
\definecolor{error}{RGB}{185,74,72}
\definecolor{warning}{RGB}{192,152,83}
\definecolor{info}{RGB}{58,135,173}
\usepackage[colorlinks=true,pdftitle=doc\_progressreport.pdf,linktoc=all,linkcolor=RedFire,urlcolor=ParksWildlife]{hyperref}
\usepackage{colortbl}
\usepackage{longtable}
\usepackage{tabu}
\setlength{\tabulinesep}{1.5mm}
\usepackage{enumerate}
\usepackage{enumitem}
\usepackage{fancyhdr}
\usepackage{lastpage}
\usepackage{graphicx}
\usepackage{eso-pic}
\usepackage{hyphenat}



%%% Custom headers/footers (fancyhdr package)
\fancypagestyle{plain}{
\fancyhf{}
\setlength\headheight{40pt}
\renewcommand{\headrulewidth}{0.1pt}
\renewcommand{\footrulewidth}{0.1pt}



    \fancyhead[L]{ \href{http://sdis.dpaw.wa.gov.au/documents/progressreport/1223/download/}{} \newline }
\fancyhead[R]{ \hfill\href{http://www.dpaw.wa.gov.au}{Department of Parks and Wildlife}\newline\href{http://sdis.dpaw.wa.gov.au}{Pythia}}




\fancyfoot[L]{ \leftmark\newline\textbf{Last Modified}\textit{ }\quad\textbf{Printed}\textit{ June 26, 2014, 3:23 p.m. } }
\fancyfoot[R]{  \, \newline Page \thepage\ of \pageref{LastPage} } % Pagenumbering


}
\pagestyle{plain}


\newcommand{\HRule}{\rule{\linewidth}{0.1pt}}

\newcommand{\placetextbox}[3]{% \placetextbox{<horizontal pos>}{<vertical pos>}{<stuff>}
  \setbox0=\hbox{#3}% Put <stuff> in a box
  \AddToShipoutPictureFG*{% Add <stuff> to current page foreground
    \put(\LenToUnit{#1\paperwidth},\LenToUnit{#2\paperheight}){\vtop{{\null}\makebox[0pt][c]{#3}}}%
  }%
}%

\begin{document}

\setcounter{secnumdepth}{-1}


\begin{titlepage}
\begin{center}
% Upper part of the page
\begin{minipage}[t]{0.28\textwidth}
\begin{flushleft}
\href{http://www.dpaw.wa.gov.au}{\includegraphics[scale=0.6]{/var/www/sdis_8271/staticfiles/img/logo-dpaw.png}}
\end{flushleft}
\end{minipage}
\begin{minipage}[b]{0.7\textwidth}
\begin{flushright}
    \href{http://sdis.dpaw.wa.gov.au/documents/progressreport/1223/download/}{}) \\
\end{flushright}
\end{minipage}
\HRule \\[0.4cm]
\vfill
\textsc{\Huge Science project 2000-3 Hydrological response to timber harvesting and associated silviculture in the intermediate rainfall zone of the northern jarrah forest \newline }
\vfill
\textsc{\Huge Progress Report}

\vfill\vfill\vfill\vfill
title and summary

\vfill\vfill\vfill\vfill\vfill\vfill\vfill\vfill

\textbf{Version created on} June 26, 2014, 3:23 p.m.
\vfill
\textbf{Last Modified on}  by 
\vfill\vfill
\textbf{Report Status}\\\,
\begin{tabu} to \linewidth { | X[l] | X | }
\hline
\rowcolor{infobg}
Status & Last Updated \\
\hline
\textbf{Planning - } \\
\hline
\end{tabu}
\vfill
\textbf{Science Project Overview}\\\,
\begin{tabu} to \linewidth { | X[l] | X | }
\hline
\rowcolor{infobg}
Part & Checklist Last Updated \\
\hline
\textbf{Part A - Summary \& Approval} & bla \\
\hline
\end{tabu}

\end{center}
\end{titlepage}

\setcounter{tocdepth}{2}
\tableofcontents
\clearpage






\section{Context Summary}



This is a long-term experiment established in 1999 to address part of Ministerial Condition 12-3 attached to the \emph{Forest Management Plan 1994-2003}. Ministerial Condition 12-3 states that DEC shall monitor and report on the status and effectiveness of silvicultural measures in the intermediate rainfall zone (900-1100 mm/yr) of the jarrah forest to protect water quality.






\section{Aims Summary}



Investigate the hydrologic impacts of timber harvesting and associated silvicultural treatments in the intermediate rainfall zone of the jarrah forest.






\section{Progress}



\begin{itemize}

  \item Monitoring of groundwater levels, streamflow, stream salinity and stream turbidity continued in the two treated catchments (Yarragil 4X and 6C) and in the control catchment (Wuraming).

  \item Monitoring of groundwater levels, streamflow, and stream salinity also continued in Yarragil 4L, which was thinned in the mid 1980s, to examine the effect of thinning on stream water qualty and quantity.

  \item The mild steel V-notch weir plate in Yarragil 4X was corroded and has been replaced by a stainless steel plate.

\end{itemize}






\section{Management implications}



\begin{itemize}

  \item These catchments provide a unique long-term record of the hydrological response of the jarrah forest to climate change and forest management practices.

  \item Findings from these studies informed the preparation of FMP 2014-23 in regard to forest management to protect or enhance stream water quality and quantity.

  \item Monitoring at these catchments contributes to reporting to KPI 10 in FMP2014-23, i.e. Stream condition and groundwater level within fully forested catchments.

\end{itemize}






\section{Future directions}



\begin{itemize}

  \item Continue monitoring of groundwater levels, streamflow, stream salinity and turbidity and rainfall.

  \item Remeasure forest density along fixed transects in Yarragil 4X and 6C to determine the forest regeneration response to the timber harvest and silvicultural treatments.

  \item Remeasure tree growth in Yarragil 4L to determine the long-term hydrological response to thinning.

  \item Write a paper on the long-term hydrological response to thinning in Yarragil 4L.

  \item Yarragil 4L mild steel weir plate is corroding and should be replaced by a stainless steel plate before winter of 2015

\end{itemize}






\clearpage



\end{document}

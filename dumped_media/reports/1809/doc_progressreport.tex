
\documentclass[version=last,
    paper=a4, % paper size
    10pt, % default font size
    usenames,
    dvipsnames,
    oneside, % ONLINE
    headings=openany, % open chapters on odd and even pages
    %toc=chapterentrywithdots, % Table of Contents style
    %BCOR=7mm, % PRINT Binding Correction
    %DIV=13, % typearea 161.54 mm x 228.46 mm, top margin 22.85 mm, inner margin 16.15 mm
    %DIV=14, % 165.00 233.36 21.21 15.00
    DIV=15 % 168.00 237.60 19.80 14.00
]{scrbook}
\usepackage{typearea}
\usepackage[automark,headsepline,footsepline]{scrlayer-scrpage} % Headers and footers

%%
%% Fonts, encoding, spacing, indentation
%%
\usepackage{txfonts}
\renewcommand{\familydefault}{\sfdefault} % Default to Sans Serif font
\usepackage[english]{babel}
\usepackage[T1]{fontenc}
\usepackage[utf8]{inputenc}

% Paragraph spacing
%\usepackage{parskip}    % Paragraph spacing
%\setlength{\parindent}{0em} % Don't indent paragraphs - ONLINE
%\setlength{\parskip}{1.3 ex plus 0.5ex minus 0.3ex} % 1-1.8 ex vertical space between paragraphs - ONLINE

% Spacing of headings
%\RedeclareSectionCommand[afterskip=3pt]{section} % less space after section
%\RedeclareSectionCommand[beforeskip=0cm]{subsection} % less space between HRule and project name
%\RedeclareSectionCommand[afterskip=0.1\baselineskip]{subsubsection} % less space after progressreport subheadings

% Table font size
\usepackage{etoolbox}
\AtBeginEnvironment{longtabu}{\footnotesize}{}{}

%%
%% Tables, columns, layout
%%
\usepackage{multicol}   % 2 col publications
\usepackage{pdflscape}  % Landscape pages
\usepackage{pdfpages}   % Include PDFs
\usepackage{hanging}    % hanging paragraphs for publications
%\usepackage{titletoc}   % Required for manipulating the table of contents
\setcounter{tocdepth}{2} % TOC list down to section
\usepackage{enumerate}  % Enumerations
\usepackage{enumitem}   % Enumerations
\usepackage{longtable}  % Multipage table
\usepackage{tabu}       %
\setlength{\tabulinesep}{1.5mm} % Consistent vertical spacing in tabu

%%
%% Graphics, images, colours
%%
\usepackage{graphicx} % embedded images
\usepackage{eso-pic} %
\usepackage{colortbl} % define custom named colours
\definecolor{RedFire}{RGB}{146,25,28}
\definecolor{ParksWildlife}{RGB}{0,85,144}
\definecolor{successbg}{RGB}{223,240,216}
\definecolor{errorbg}{RGB}{242,222,222}
\definecolor{warningbg}{RGB}{252,248,227}
\definecolor{infobg}{RGB}{217,237,247}
\definecolor{muted}{RGB}{153,153,153}
\definecolor{success}{RGB}{70,136,71}
\definecolor{error}{RGB}{185,74,72}
\definecolor{warning}{RGB}{192,152,83}
\definecolor{info}{RGB}{58,135,173}

\definecolor{required}{RGB}{192,152,83}
\definecolor{requiredbg}{RGB}{252,248,227}
\definecolor{denied}{RGB}{185,74,72}
\definecolor{deniedbg}{RGB}{242,222,222}
\definecolor{granted}{RGB}{70,136,71}
\definecolor{grantedbg}{RGB}{223,240,216}
\definecolor{not reqiured}{RGB}{153,153,153}
\definecolor{not requiredbg}{RGB}{255,255,255}

\usepackage{tikz} % Drawing
\usetikzlibrary{arrows,shapes,positioning,shadows,trees}

%%
%% Links, URLs
%%
\usepackage[
    linktoc=all,
    %colorlinks=false,  %PRINT
    colorlinks=true, % ONLINE
    linkcolor=RedFire, % ONLINE
    urlcolor=ParksWildlife, % ONLINE
    pdftitle=Progress Report SP 2014-018 (FY 2016-2017)
]{hyperref}

% Black magic to linebreak URLs
\usepackage{url}
\makeatletter
\g@addto@macro{\UrlBreaks}{\UrlOrds}
\makeatother

%%
%% Custom macros
%%
% Thick Horizontal rule
\newcommand{\HRule}{\vspace{8mm}\\\noindent\rule{\linewidth}{0.1pt}}

% Custom Tikz node for SDS diagram
\newcommand\mynode[6][]{
    \node[#1] (#2){
        \parbox{#3\relax}{
            \begin{center}
            \textbf{#4}\\
            #5\\
            \footnotesize{#6}
            \end{center}}};}



\usepackage[automark,headsepline,footsepline,plainfootsepline]{scrlayer-scrpage}
\automark*[section]{}
\addtokomafont{pageheadfoot}{\normalfont\footnotesize\sffamily} % Don't italicise
\renewcommand{\chaptermark}[1]{\markleft{#1}{}}     % Chapter: suppress numbering
\renewcommand{\sectionmark}[1]{\markright{#1}{}}    % Section: suppress numbering

% Header (inner, center, outer)
\ihead{\href{http://sdis.dpaw.wa.gov.au/documents/progressreport/1809/}{Progress Report SP 2014-018 (FY 2016-2017)}}
%\chead{\href{http://sdis.dpaw.wa.gov.au}{Science Directorate Information System}}
\ohead{\href{https://www.dpaw.wa.gov.au/about-us/science-and-research}{\includegraphics[height=6mm, keepaspectratio]{/mnt/projects/sdis/staticfiles/img/logo-dpaw.png}}}

% Footer (inner, center, outer)
\ifoot{\textbf{Printed}~Mon, 19 Jun 2017 14:42:29 +0800} % inner/left footer
\cfoot{}
\ofoot[\bfseries\thepage]{\bfseries\thepage}        % Page number (also [plain])


\pagestyle{scrheadings}
\setkomafont{pageheadfoot}{\normalfont}

%-----------------------------------------------------------------------------%
\begin{document}
\raggedbottom

%-----------------------------------------------------------------------------%
% Title page
\subject{Progress Report SP 2014-018
}
\title{Distribution and abundance estimate of Australian snubfin dolphins
(\emph{Orcaella heinsohni}) at a key site in the Kimberley region,
Western Australia
}
\subtitle{Marine Science
}
\author{}
\publishers{\small
    \subsection*{Project Core Team}
\begin{tabu} {X X}
\textbf{Supervising Scientist} & Kelly Waples
\\
\textbf{Data Custodian} & Holly Raudino
\\
\textbf{Site Custodian} & Holly Raudino
\\
\end{tabu}


    \subsection*{Project status as of June 19, 2017, 2:42 p.m.}
\begin{tabu} {X X}
& Update requested
\\
\end{tabu}

    
\subsection*{Document endorsements and approvals as of June 19, 2017, 2:42 p.m.}
\begin{tabu} {X X}

%\rowcolor{grantedbg}
    \textbf{Project Team} & 
    \textcolor{granted}{ granted}\\

%\rowcolor{requiredbg}
    \textbf{Program Leader} & 
    \textcolor{required}{ required}\\

%\rowcolor{requiredbg}
    \textbf{Directorate} & 
    \textcolor{required}{ required}\\

\end{tabu}



}
\uppertitleback{}
\lowertitleback{}
\date{}

%-----------------------------------------------------------------------------%
% Front matter
\frontmatter
\maketitle
%-----------------------------------------------------------------------------%
% Main matter
\mainmatter

\section*{Distribution and abundance estimate of Australian snubfin dolphins
(\emph{Orcaella heinsohni}) at a key site in the Kimberley region,
Western Australia
}

K Waples, H Raudino



\section*{Context}

The current lack of knowledge of the Australian snubfin dolphin
(\emph{Orcaella heinsohni}) meant that its conservation status could not
be properly assessed in 2011 due to insufficent information on
population dynamics and distribution. This species is known from
tropical coastal waters of Australia and New Guinea, but tend to be shy,
evasive and difficult to study. Although they range southwards to the
the Pilbara region of Western Australia, there has been little Western
Australian-based research on this species and much of this remains
unpublished. This project will compile existing data on snubfin dolphins
across the Kimberley to gain a better understanding of their habitat use
and distribution. The collation of data into a single database will also
facilitate the study of population structure and demographics based on
recognised individual animals. This project will assess dolphin
distribution across the Kimberley region between 2004-2012.




\section*{Aims}

\begin{itemize}
\itemsep1pt\parskip0pt\parsep0pt
\item
  Provide a quantitative abundance estimate of snubfin dolphins for
  Roebuck Bay in Western Australia that will be used as a baseline for
  this population and will also enable comparison with abundance
  estimates of the species from sites at Cleveland Bay (Qld) and Port
  Essington (NT).
\item
  Compare methods for abundance estimation (mark-recapture versus
  distance sampling) and the suitability of these methods for abundance
  estimation of this species.
\item
  Map the extent of occurrence and area of occupancy of snubfin dolphins
  in the Kimberley by combining traditional knowledge and dolphin
  sightings from indigenous sea rangers and scientific survey sightings.
\item
  Refine and populate a purpose built and standardised database
  that~will support long term data collection and curation~in WA and
  facilitate data-sharing between jurisdictions.
\end{itemize}




\section*{Progress}

\begin{itemize}
\itemsep1pt\parskip0pt\parsep0pt
\item
  A manuscript on cross-cultural knowledge of the distribution of
  snubfin dolphins in the Kimberley is being revised based on feedback
  from reviewers. This research has been accepted for presentation at
  the 2017 Biennial Conference on the Biology of Marine Mammals.
\item
  A manuscript on snubfin dolphin abundance contrasting distance
  sampling and mark-recapture survey techniques in Roebuck Bay is
  undergoing final statistical analysis and revision prior to
  submission.
\item
  Additional data collected from Roebuck Bay in 2013 and 2014 has been
  added to the database using a grant from the Commonwealth Government
  and administered by James Cook University.
\end{itemize}




\section*{Management implications}

\begin{itemize}
\itemsep1pt\parskip0pt\parsep0pt
\item
  Collation of scientific and traditional knowledge of a poorly
  understood marine mammal species of high conservation value means
  managers now have baseline knowledge of the abundance of snubfin
  dolphins in the proposed Yawuru Nagulagun / Roebuck Bay Marine Park.
\item
  Establishment of a database for all dolphin research and monitoring
  where survey and photo-identification data is collected to ensures
  that standardised data is available for assessing population abundance
  and distribution. It also provides the capacity to develop~sighting
  histories for individual animals, thus providing a better
  understanding of population demographics and life history. This
  database can also be used for information sharing across jurisdictions
  and between research organisations.
\item
  The broad-scale collation of information and modeling has provided
  relevant information on area of occupancy and extent of occurrence
  that can be used to more accurately assess the conservation status of
  snubfin dolphins.
\item
  Partnerships have been established with indigenous sea ranger groups
  to develop survey methodologies, data storage and reporting structures
  that are consistent with healthy country plans and reserve management
  plans.
\end{itemize}




\section*{Future directions}

\begin{itemize}
\itemsep1pt\parskip0pt\parsep0pt
\item
  Submission of a manuscript on a population estimate of the Australian
  snubfin dolphin in Roebuck Bay to a peer-reviewed journal.
\item
  Revision of a manuscript on cross-cultural knowledge of the
  distribution of the Australian snubfin dolphin in the Kimberley to
  include species distribution modelling.
\item
  Utilisation of new (2013 \& 2014) data to investigate long-term site
  fidelity and home range of individual snubfins in Roebuck Bay.
\end{itemize}



%-----------------------------------------------------------------------------%
% Back matter
%\backmatter
\end{document}
%-----------------------------------------------------------------------------%

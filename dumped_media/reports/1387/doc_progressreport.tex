
\documentclass[version=last, paper=a4, DIV=18, usenames, dvipsnames]{scrartcl}
\usepackage{txfonts}
\usepackage{pdflscape}
\usepackage{pdfpages}
\usepackage[english]{babel} % English language/hyphenation
%%% Bootstrap colors
\definecolor{RedFire}{RGB}{146,25,28}
\definecolor{ParksWildlife}{RGB}{0,85,144}
\definecolor{successbg}{RGB}{223,240,216}
\definecolor{errorbg}{RGB}{242,222,222}
\definecolor{warningbg}{RGB}{252,248,227}
\definecolor{infobg}{RGB}{217,237,247}
\definecolor{muted}{RGB}{153,153,153}
\definecolor{success}{RGB}{70,136,71}
\definecolor{error}{RGB}{185,74,72}
\definecolor{warning}{RGB}{192,152,83}
\definecolor{info}{RGB}{58,135,173}

\definecolor{required}{HTML}{D9534F}
\definecolor{denied}{HTML}{D9534F}
\definecolor{granted}{HTML}{47A447}
\definecolor{not required}{RGB}{200, 200, 200}

\usepackage[colorlinks=true,pdftitle=doc\_progressreport.pdf
,linktoc=all,linkcolor=RedFire,urlcolor=ParksWildlife]{hyperref}
\usepackage{colortbl}
\usepackage{longtable}
\usepackage{tabu}
\setlength{\tabulinesep}{1.5mm}
\usepackage{enumerate}
\usepackage{enumitem}
\usepackage{fancyhdr}
\usepackage{lastpage}
\usepackage{graphicx}
\usepackage{eso-pic}
\usepackage{hyphenat}
\renewcommand{\familydefault}{\sfdefault}



\newcommand{\HRule}{\rule{\linewidth}{0.1pt}}

\newcommand{\placetextbox}[3]{% \placetextbox{<horizontal pos>}{<vertical pos>}{<stuff>}
  \setbox0=\hbox{#3}% Put <stuff> in a box
  \AddToShipoutPictureFG*{% Add <stuff> to current page foreground
    \put(\LenToUnit{#1\paperwidth},\LenToUnit{#2\paperheight}){\vtop{{\null}\makebox[0pt][c]{#3}}}%
  }%
}%




%-----------------------------------------------------------------------------%
% Headers and footers
%
\fancypagestyle{plain}{
  \fancyhf{}
  \setlength\headheight{60pt} % push page content below header
  \renewcommand{\headrulewidth}{0.1pt}
  \renewcommand{\footrulewidth}{0.1pt}
  
  
  \fancyhead[L]{ 
    \href{http://sdis.dpaw.wa.gov.au}{
    \includegraphics[scale=0.6]{/mnt/projects/sdis/staticfiles/img/logo-dpaw.png}}
  }
  \fancyhead[R]{ 
      \hfill
      \href{http://sdis.dpaw.wa.gov.au}{Science Directorate Information System} 
      \newline 
      \href{http://sdis.dpaw.wa.gov.au/documents/progressreport/1387/}{Progress Report 2007-2 (FY 2014-2015)} 
  }
  
  
  
  
  \fancyfoot[L]{ \leftmark\newline\textbf{Printed}\textit{ June 24, 2015, 3:28 p.m. }}
  \fancyfoot[R]{  \, \newline Page \thepage\ of \pageref{LastPage} }
  
  
}
\pagestyle{plain}
%
% end Headers
%-----------------------------------------------------------------------------%

\begin{document}

%-----------------------------------------------------------------------------%
% Title page
%

%
% end title page
%-----------------------------------------------------------------------------%




\section*{Context Summary}
The woylie (\emph{Bettongia penicillata}) has declined by about 90\%
since 2001. Population declines have been rapid (\textless{}95\% per
annum), substantial (\textgreater{}90\% lost) and have particularly
impacted the largest and most important populations. Most of the
remaining unaffected populations are small, isolated and inherently
vulnerable. The woylie has been upgraded to Critically Endangered as a
result.



\section*{Aims Summary}
\begin{itemize}
\itemsep1pt\parskip0pt\parsep0pt
\item
  Determine the causal factor(s) responsible for the recent woylie
  declines in the Upper Warren Region of south-western Australia.
\item
  Identify the management required to ameliorate these declines.
\item
  Develop adequate mammal monitoring protocols that will enable future
  changes in population abundances to be quantified and explained.
\end{itemize}



\section*{Progress}
\begin{itemize}
\itemsep1pt\parskip0pt\parsep0pt
\item
  A WWF funded project in collaboration with James Cook University has
  been conducting an outbreak investigation for the woylie. The project
  began in 2013 and is currently preparing a scientific paper for
  submission in 2015. A PhD project is being secured to continue the
  work beyond this.
\item
  An ARC linkage project `The Ecology of Parasite Transmission in Fauna
  Translocations' commenced in 2013 and Parks and Wildlife is an
  industry partner in this, contributing to the project design and
  providing assistance in the field. The translocation of 182 woylies
  from Perup Sanctuary to two sites in Greater Kingston was completed in
  June 2013. Pre and post translocation monitoring is providing evidence
  of the effects of these conservation actions on the populations of
  woylies and sympatric mammals at the source and destination sites. A
  similar process is now also underway for a third translocation of 69
  woylies from sites across the Upper Warren to Dryandra (the other
  remaining natural woylie population) conducted in June 2014.
\item
  Other monitoring of woylie populations and introduced predators within
  the Upper Warren region as part of this project have ceased. Some
  monitoring continues as part of Western Shield and district programs,
  to which this project has continued to provide practical support.
\item
  The evidence remains consistent in indicating that the woylie declines
  have been mortality-driven, principally due to the predation
  (particularly by cats) of individuals that may have become
  increasingly vulnerable due to disease.
\item
  Collaborative disease investigations continue, particularly into the
  key associations with the declines.
\item
  Seven native species have now successively declined since 1994 in the
  Upper Warren region (dunnart, wambenger, bush rat, quenda, ngwayir,
  woylie and western brush wallaby), to similar extents
  (\textgreater{}80\%), at similar rates and with no signs of
  significant or sustained recovery. The chuditch koomal and Tammar
  wallaby have more recently increased in the region.
\item
  Several papers have been recently published in scientific journals and
  others are in preparation.
\end{itemize}



\section*{Management implications}
\begin{itemize}
\itemsep1pt\parskip0pt\parsep0pt
\item
  Insurance populations to conserve the remaining genetic diversity of
  the woylie remains a priority. Continued loss of genetic diversity due
  to important woylie populations remaining small or becoming extinct
  will compromise the recovery prospects and conservation of the
  species.
\item
  More effective control of feral cats and foxes is critical for
  sustaining and facilitating the recovery of important woylie
  populations. Improved control and monitoring of introduced predators
  is therefore very important.
\item
  Wildlife disease may contribute to woylie declines by making animals
  more vulnerable to predation. Resolution of the role of disease in the
  declines will directly inform woylie recovery strategies and
  management.
\item
  The serial decline of multiple mammal species in the Upper Warren
  region is of serious concern requiring action, especially given the
  high conservation value of the area and of the populations it
  supports.
\end{itemize}



\section*{Future directions}
\begin{itemize}
\itemsep1pt\parskip0pt\parsep0pt
\item
  Continued in kind support to the ARC linkage and WWF funded projects,
  Western Shield and district monitoring activities and the students
  assocaited with this project
\item
  Analysis and publication in scientific journals of the research
  conducted to date
\end{itemize}




\clearpage



\end{document}

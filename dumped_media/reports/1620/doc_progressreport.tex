
\documentclass[version=last,
    paper=a4, % paper size
    10pt, % default font size
    usenames,
    dvipsnames,
    oneside, % ONLINE
    headings=openany, % open chapters on odd and even pages
    %toc=chapterentrywithdots, % Table of Contents style
    %BCOR=7mm, % PRINT Binding Correction
    %DIV=13, % typearea 161.54 mm x 228.46 mm, top margin 22.85 mm, inner margin 16.15 mm
    %DIV=14, % 165.00 233.36 21.21 15.00
    DIV=15 % 168.00 237.60 19.80 14.00
]{scrbook}
\usepackage{typearea}
\usepackage[automark,headsepline,footsepline]{scrlayer-scrpage} % Headers and footers

%%
%% Fonts, encoding, spacing, indentation
%%
\usepackage{txfonts}
\renewcommand{\familydefault}{\sfdefault} % Default to Sans Serif font
\usepackage[english]{babel}
\usepackage[T1]{fontenc}
\usepackage[utf8]{inputenc}

% Paragraph spacing
%\usepackage{parskip}    % Paragraph spacing
%\setlength{\parindent}{0em} % Don't indent paragraphs - ONLINE
%\setlength{\parskip}{1.3 ex plus 0.5ex minus 0.3ex} % 1-1.8 ex vertical space between paragraphs - ONLINE

% Spacing of headings
%\RedeclareSectionCommand[afterskip=3pt]{section} % less space after section
%\RedeclareSectionCommand[beforeskip=0cm]{subsection} % less space between HRule and project name
%\RedeclareSectionCommand[afterskip=0.1\baselineskip]{subsubsection} % less space after progressreport subheadings

% Table font size
\usepackage{etoolbox}
\AtBeginEnvironment{longtabu}{\footnotesize}{}{}

%%
%% Tables, columns, layout
%%
\usepackage{multicol}   % 2 col publications
\usepackage{pdflscape}  % Landscape pages
\usepackage{pdfpages}   % Include PDFs
\usepackage{hanging}    % hanging paragraphs for publications
%\usepackage{titletoc}   % Required for manipulating the table of contents
\setcounter{tocdepth}{2} % TOC list down to section
\usepackage{enumerate}  % Enumerations
\usepackage{enumitem}   % Enumerations
\usepackage{longtable}  % Multipage table
\usepackage{tabu}       %
\setlength{\tabulinesep}{1.5mm} % Consistent vertical spacing in tabu

%%
%% Graphics, images, colours
%%
\usepackage{graphicx} % embedded images
\usepackage{eso-pic} %
\usepackage{colortbl} % define custom named colours
\definecolor{RedFire}{RGB}{146,25,28}
\definecolor{ParksWildlife}{RGB}{0,85,144}
\definecolor{successbg}{RGB}{223,240,216}
\definecolor{errorbg}{RGB}{242,222,222}
\definecolor{warningbg}{RGB}{252,248,227}
\definecolor{infobg}{RGB}{217,237,247}
\definecolor{muted}{RGB}{153,153,153}
\definecolor{success}{RGB}{70,136,71}
\definecolor{error}{RGB}{185,74,72}
\definecolor{warning}{RGB}{192,152,83}
\definecolor{info}{RGB}{58,135,173}

\definecolor{required}{RGB}{192,152,83}
\definecolor{requiredbg}{RGB}{252,248,227}
\definecolor{denied}{RGB}{185,74,72}
\definecolor{deniedbg}{RGB}{242,222,222}
\definecolor{granted}{RGB}{70,136,71}
\definecolor{grantedbg}{RGB}{223,240,216}
\definecolor{not reqiured}{RGB}{153,153,153}
\definecolor{not requiredbg}{RGB}{255,255,255}

\usepackage{tikz} % Drawing
\usetikzlibrary{arrows,shapes,positioning,shadows,trees}

%%
%% Links, URLs
%%
\usepackage[
    linktoc=all,
    %colorlinks=false,  %PRINT
    colorlinks=true, % ONLINE
    linkcolor=RedFire, % ONLINE
    urlcolor=ParksWildlife, % ONLINE
    pdftitle=Progress Report SP 2012-027 (FY 2015-2016)
]{hyperref}

% Black magic to linebreak URLs
\usepackage{url}
\makeatletter
\g@addto@macro{\UrlBreaks}{\UrlOrds}
\makeatother

%%
%% Custom macros
%%
% Thick Horizontal rule
\newcommand{\HRule}{\vspace{8mm}\\\noindent\rule{\linewidth}{0.1pt}}

% Custom Tikz node for SDS diagram
\newcommand\mynode[6][]{
    \node[#1] (#2){
        \parbox{#3\relax}{
            \begin{center}
            \textbf{#4}\\
            #5\\
            \footnotesize{#6}
            \end{center}}};}



%-----------------------------------------------------------------------------%
% Headers and Footers
\automark{section}
\ohead{\href{http://sdis.dpaw.wa.gov.au/documents/progressreport/1620/}{Progress Report SP 2012-027
}}
\chead{\href{http://sdis.dpaw.wa.gov.au}{SDIS}} % center header ONLINE
\ihead{\href{http://sdis.dpaw.wa.gov.au}{
        \includegraphics[scale=0.4]{/mnt/projects/sdis/staticfiles/img/logo-dpaw.png}}}
\ifoot{\textbf{Printed}~Mon, 11 Jul 2016 10:07:39 +0800} % inner/left footer
\cfoot{} % center footer
\ofoot{\pagemark} % outer/right footer
\pagestyle{scrheadings}
\setkomafont{pageheadfoot}{\normalfont}

%-----------------------------------------------------------------------------%
\begin{document}
\raggedbottom

%-----------------------------------------------------------------------------%
% Title page
\subject{Progress Report SP 2012-027
}
\title{North Kimberley Landscape Conservation Initiative: monitoring and
evaluation
}
\subtitle{Ecosystem Science
}
\author{}
\publishers{\small
    \subsection*{Project Core Team}
\begin{tabu} {X X}
\textbf{Supervising Scientist} & Ian Radford
\\
\textbf{Data Custodian} & 
\\
\textbf{Site Custodian} & 
\\
\end{tabu}


    \subsection*{Project status as of July 11, 2016, 10:07 a.m.}
\begin{tabu} {X X}
& Approved and active
\\
\end{tabu}

    
\subsection*{Document endorsements and approvals as of July 11, 2016, 10:07 a.m.}
\begin{tabu} {X X}

%\rowcolor{grantedbg}
    \textbf{Project Team} & 
    \textcolor{granted}{ granted}\\

%\rowcolor{grantedbg}
    \textbf{Program Leader} & 
    \textcolor{granted}{ granted}\\

%\rowcolor{grantedbg}
    \textbf{Directorate} & 
    \textcolor{granted}{ granted}\\

\end{tabu}



}
\uppertitleback{}
\lowertitleback{}
\date{}

%-----------------------------------------------------------------------------%
% Front matter
\frontmatter
\maketitle
%-----------------------------------------------------------------------------%
% Main matter
\mainmatter

\section*{North Kimberley Landscape Conservation Initiative: monitoring and
evaluation
}

I Radford, R Fairman


\section*{Context}
This project is a biodiversity monitoring and evaluation program to
inform adaptive management of fire and cattle in the north Kimberley.
The adaptive management program that forms the Landscape Conservation
Initiative (LCI) of the Kimberley Science and Conservation Strategy
commenced in 2011 in response to perceived threats by cattle and fire to
biodiversity conservation in the north Kimberley. This initiative is
based on the hypothesis that large numbers of introduced herbivores and
the impacts of current fire regimes are associated with declines of
critical-weight-range mammals, contraction and degradation of rainforest
patches, and degradation of vegetation structure and habitat condition
in savannas. This monitoring and evaluation program will provide a
report card on performance of landscape management initiatives in the
north Kimberley, particularly prescribed burning and cattle culling, in
maintaining and improving biodiversity status.



\section*{Aims}
\begin{itemize}
\itemsep1pt\parskip0pt\parsep0pt
\item
  Inform management of biodiversity status in representative areas after
  prescribed burning and cattle control programs have been applied.
\item
  Provide warning when landscape ecological thresholds have been
  reached, for example decline of mammals to below 2\% capture rate, or
  decline of mean shrub projected ground cover to \textless{}2\%.
\item
  Compare biodiversity outcomes in intensively managed and unmanaged
  areas so that the effectiveness of management interventions in
  maintaining and improving conservation values can be evaluated.
\end{itemize}



\section*{Progress}
\begin{itemize}
\itemsep1pt\parskip0pt\parsep0pt
\item
  This project is now in its sixth year. A total of 92 sites have been
  surveyed for mammals and vegetation at least once and 112 sites for
  vegetation only. Ten rainforest sites have been surveyed at least
  once. Sites at Mirima National Park, ~Mitchell River National
  Park,~King Leopold Range National Park, Prince Regent National
  Park,~Drysdale River National Park, Bachsten Creek~and Mount
  Elizabethhad all been sampled at least twice.
\item
  Data from monitoring sites, combined with spatial data, confirm that
  mammal distribution patterns are strongly influenced by vegetation
  cover, cattle impacts and fire regime particularly the frequency of
  late dry season fires. An inverse relationship between the amount of
  surrounding country burnt, ground layer vegetation cover and mammal
  abundance confirms the importance of prescribed burning to conserve
  vegetation cover needed by mammals.~
\item
  At the regional scale, monitoring shows that most Kimberley mammal
  species recorded historically are still present and that abundance and
  richness values are well above threshold values seen in the Northern
  Territory where mammal populations have collapsed. King Edward River,
  Drysdale River and Mount Elizabeth have the lowest mammal
  abundance/richness and are localities of greatest conservation
  concern.
\item
  Mammal abundance and richness has increased at the Mitchell Plateau
  compared with earlier surveys. Mammal species have recolonised
  habitats and become more abundant with implementation of LCI
  initiatives since 2008, including greater use of planned burning early
  in the dry season and introduction of a cattle culling program.
\item
  Surveys at Mirima have revealed a~declining mammal abundance trend
  since 2012 when much of the Park was burnt by a large wildfire. Mammal
  abundance remains low despite sufficient time for re-establishment of
  mature vegetation structure.
\end{itemize}



\section*{Management implications}
\begin{itemize}
\itemsep1pt\parskip0pt\parsep0pt
\item
  There is strong evidence that cattle have negative influences on
  critical weight range mammals, including threatened species such as
  \emph{Conilurus penicillatus}. Culling programs should be maintained
  and expanded in important conservation reserves.
\item
  There is statistical evidence that the LCI has shifted the fire regime
  in the North Kimberley from dominance by late dry season bushfires to
  a situation where equal proportions of the country are burnt during
  the early and later periods of the dry season. Monitoring and
  evaluation data suggest that this has benefited threatened mammal
  assemblages, or at least is not detrimental to them, and provides
  evidence that current fire management practices in the North Kimberley
  should be maintained to enhance conservation values.
\item
  Lower mammal abundance and diversity at inland sites in conjunction
  with higher cattle and fire frequency indicates that prescribed
  burning and cattle culling initiatives should be expanded into these
  areas as a matter of priority.
\end{itemize}



\section*{Future directions}
\begin{itemize}
\itemsep1pt\parskip0pt\parsep0pt
\item
  Monitoring and evaluation will continue so that the effectiveness of
  management interventions can be evaluated in the longer term.
\item
  With some sites now having five or more years of repeated survey, the
  project is in a position to use repeated measures analysis to examine
  factors influencing mammal abundance and richness between years. This
  will be undertaken using a statistical modelling approach following
  the 2016 monitoring season which finishes at the end of September.
\item
  Collaborative monitoring programs will be expanded to incorporate
  adjoining areas on pastoral lease and indigenous-owned land to provide
  comparative data on mammal populations and vegetation condition where
  cattle populations remain high and fire regimes are not managed.
\item
  A collaborative project with Charles Darwin University studying the
  abundance and habitat requirements of arboreal mammals in the North
  Kimberley will commence in~2016/2017. This project will investigate
  why some arboreal mammals are more restricted than many terrestrial
  mammals in the region, and the role that fire regimes have~in
  determining the density of tree cavities for nesting habitat as a
  mechanism influencing the abundance of arboreal mammals
\end{itemize}



%-----------------------------------------------------------------------------%
% Back matter
%\backmatter
\end{document}
%-----------------------------------------------------------------------------%


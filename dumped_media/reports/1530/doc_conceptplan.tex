
\documentclass[version=last, 
    paper=a4, % paper size
    10pt, % default font size
    usenames,
    dvipsnames, 
    oneside, % ONLINE
    headings=openany, % open chapters on odd and even pages
    %toc=chapterentrywithdots, % Table of Contents style
    %BCOR=7mm, % PRINT Binding Correction
    %DIV=13, % typearea 161.54 mm x 228.46 mm, top margin 22.85 mm, inner margin 16.15 mm
    %DIV=14, % 165.00 233.36 21.21 15.00
    DIV=15 % 168.00 237.60 19.80 14.00
]{scrbook}
\usepackage{typearea}
\usepackage[automark,headsepline,footsepline]{scrlayer-scrpage} % Headers and footers

%%
%% Fonts, encoding, spacing, indentation
%%
\usepackage{txfonts}
\renewcommand{\familydefault}{\sfdefault} % Default to Sans Serif font
\usepackage[english]{babel}
\usepackage[T1]{fontenc}
\usepackage[utf8]{inputenc}

% Paragraph spacing
%\usepackage{parskip}    % Paragraph spacing
%\setlength{\parindent}{0em} % Don't indent paragraphs - ONLINE
%\setlength{\parskip}{1.3 ex plus 0.5ex minus 0.3ex} % 1-1.8 ex vertical space between paragraphs - ONLINE

% Spacing of headings
%\RedeclareSectionCommand[afterskip=3pt]{section} % less space after section
%\RedeclareSectionCommand[beforeskip=0cm]{subsection} % less space between HRule and project name
%\RedeclareSectionCommand[afterskip=0.1\baselineskip]{subsubsection} % less space after progressreport subheadings

% Table font size
\usepackage{etoolbox}
\AtBeginEnvironment{longtabu}{\footnotesize}{}{}

%%
%% Tables, columns, layout
%%
\usepackage{multicol}   % 2 col publications
\usepackage{pdflscape}  % Landscape pages
\usepackage{pdfpages}   % Include PDFs
\usepackage{hanging}    % hanging paragraphs for publications
%\usepackage{titletoc}   % Required for manipulating the table of contents
\setcounter{tocdepth}{2} % TOC list down to section
\usepackage{enumerate}  % Enumerations
\usepackage{enumitem}   % Enumerations
\usepackage{longtable}  % Multipage table
\usepackage{tabu}       % 
\setlength{\tabulinesep}{1.5mm} % Consistent vertical spacing in tabu

%%
%% Graphics, images, colours
%%
\usepackage{graphicx} % embedded images
\usepackage{eso-pic} % 
\usepackage{colortbl} % define custom named colours
\definecolor{RedFire}{RGB}{146,25,28}
\definecolor{ParksWildlife}{RGB}{0,85,144}
\definecolor{successbg}{RGB}{223,240,216}
\definecolor{errorbg}{RGB}{242,222,222}
\definecolor{warningbg}{RGB}{252,248,227}
\definecolor{infobg}{RGB}{217,237,247}
\definecolor{muted}{RGB}{153,153,153}
\definecolor{success}{RGB}{70,136,71}
\definecolor{error}{RGB}{185,74,72}
\definecolor{warning}{RGB}{192,152,83}
\definecolor{info}{RGB}{58,135,173}

\definecolor{required}{RGB}{192,152,83}
\definecolor{requiredbg}{RGB}{252,248,227}
\definecolor{denied}{RGB}{185,74,72}
\definecolor{deniedbg}{RGB}{242,222,222}
\definecolor{granted}{RGB}{70,136,71}
\definecolor{grantedbg}{RGB}{223,240,216}
\definecolor{not reqiured}{RGB}{153,153,153}
\definecolor{not requiredbg}{RGB}{255,255,255}

\usepackage{tikz} % Drawing
\usetikzlibrary{arrows,shapes,positioning,shadows,trees}

%%
%% Links, URLs
%%
\usepackage[
    linktoc=all,
    %colorlinks=false,  %PRINT
    colorlinks=true, % ONLINE
    linkcolor=RedFire, % ONLINE
    urlcolor=ParksWildlife, % ONLINE
    pdftitle=Concept Plan SP 2016-005
]{hyperref}

% Black magic to linebreak URLs
\usepackage{url}
\makeatletter
\g@addto@macro{\UrlBreaks}{\UrlOrds}
\makeatother

%%
%% Custom macros
%%
% Thick Horizontal rule
\newcommand{\HRule}{\vspace{8mm}\\\noindent\rule{\linewidth}{0.1pt}}

% Custom Tikz node for SDS diagram
\newcommand\mynode[6][]{\node[#1] (#2){\parbox{#3\relax}{\begin{center}\textbf{#4}\\#5\\\footnotesize{#6}\end{center}}};}




%-----------------------------------------------------------------------------%
% Headers and Footers
\automark{section}
\ohead{\href{http://sdis.dpaw.wa.gov.au/documents/conceptplan/1530/}{Concept Plan SP 2016-005
}}
\chead{\href{http://sdis.dpaw.wa.gov.au}{SDIS}} % center header ONLINE
\ihead{\href{http://sdis.dpaw.wa.gov.au}{
        \includegraphics[scale=0.4]{/mnt/projects/sdis/staticfiles/img/logo-dpaw.png}}}
\ifoot{\textbf{Printed}~Fri, 24 Jun 2016 11:55:00 +0800} % inner/left footer
\cfoot{} % center footer
\ofoot{\pagemark} % outer/right footer
\pagestyle{scrheadings}
\setkomafont{pageheadfoot}{\normalfont}

%-----------------------------------------------------------------------------%
\begin{document}
\raggedbottom

%-----------------------------------------------------------------------------%
% Title page
\subject{Concept Plan SP 2016-005
}
\title{Hydrological Function of Critical Ecosystems
}
\subtitle{Wetlands Conservation
}
\author{}
\publishers{\small
    \subsection*{Project Core Team}
\begin{tabu} {X X}
\textbf{Supervising Scientist} & Jasmine Rutherford
\\
\textbf{Data Custodian} & Jasmine Rutherford
\\
\textbf{Site Custodian} & 
\\
\end{tabu}


    \subsection*{Project status as of June 24, 2016, 11:55 a.m.}
\begin{tabu} {X X}
& Pending project plan approval
\\
\end{tabu}

    
\subsection*{Document endorsements and approvals as of June 24, 2016, 11:55 a.m.}
\begin{tabu} {X X}

%\rowcolor{grantedbg}
    \textbf{Project Team} & 
    \textcolor{granted}{ granted}\\

%\rowcolor{grantedbg}
    \textbf{Program Leader} & 
    \textcolor{granted}{ granted}\\

%\rowcolor{grantedbg}
    \textbf{Directorate} & 
    \textcolor{granted}{ granted}\\

\end{tabu}



}
\uppertitleback{}
\lowertitleback{}
\date{}

%-----------------------------------------------------------------------------%
% Front matter
\frontmatter
\maketitle
%-----------------------------------------------------------------------------%
% Main matter
\mainmatter


\section*{Hydrological Function of Critical Ecosystems
}


\subsection*{Science and Conservation Division Program}
Wetlands Conservation



\subsection*{Parks and Wildlife Service}
Service 2: Conserving Habitats, Species and Ecological Communities



\subsection*{Background and Aims}
Hydrology is a primary driver of wetland and riverine ecosystems and
aquatic species distributions, acting directly on spatial and temporal
variation in water volumes or through influences on water chemistry or
habitat (e.g. acidification, salinisation, fire regimes, aquatic plants
for fauna, temperature or woody debris from riparian zones). Most
wetland conservation issues have a hydrological basis, requiring
technical expertise to investigate the problem and advise on management.
This project has been developed to bring short-term hydrological
assessment studies under one umbrella project. Examples of current
hydrological assessment projects include:

\begin{itemize}
\itemsep1pt\parskip0pt\parsep0pt
\item
  Investigations identifying the major aquifers discharging groundwater
  and sustaining the Walyarta (Mandora Marsh) springs. This work will
  provide a basis on which to interpret the potential impacts of water
  resource development in the associated aquifers. (2015-2017). Funded
  by BHP via the Kimberley Region and undertaken in collaboration with
  researchers at The University of Western Australia.
\end{itemize}

\begin{itemize}
\itemsep1pt\parskip0pt\parsep0pt
\item
  Assessment of key hydrological management actions within the Brixton
  Street wetland complex. Funded by Perth Region NRM via~a grant
  application prepared by the Swan Region.
\end{itemize}

Proposals are also in development with~the Rivers and Estuaries Division
to map and understand nutrient fluxes into the Swan Estuary Marine Park
and the Kimberley Region to identify and characterise different
hydrological settings of the monsoon vine thicket TEC and develop
effective monitoring programs.

Project objectives will vary but will generally involve identifying the
critical hydrological parameter to manage that will improve conservation
outcomes. To accomplish this a multi-disciplinary approach is employed
to understand how these ecosystems vary in space and time, frequently
involving collaborations with other researchers.~ Project outputs and
outcomes are designed to provide more informed decision making with
respect to prioritising conservation actions and assessing environmental
impacts of land and water use proposals.

Combined. these projects fill an important role in reducing hydrological
knowledge gaps, building an improved statewide understanding of the
ecohydrology of threatened ecosystems and their critical hydrological
parameters.



\subsection*{Expected outcome}
Improved ecosystem health and resilience through ability of conservation
managers to understand local hydrological function and the critical
parameters requiring management. This information will be used to 1)
assess environmental impacts, 2) prioritise conservation actions, 3) set
limits of acceptable change (LoAC) (e.g. in Ramsar wetlands) and 4)
establish or re-evaluate baseline monitoring programs.

~



\subsection*{Strategic context}
The overarching aims of this project align with Parks and Wildlife's
Strategic Directions 2014-2017, focusing on management priorities,
conducting best practice science to protect threatened and priority
plant and animal species and delivering on-ground conservation outcomes.
Projects will be designed to assist in the development of management
plans and where plans exist will focus on addressing priorities and
targets relevant to the hydrology.~ For Ramsar sites hydrology project
planning will be undertaken to ensure project outcomes reduce existing
knowledge gaps outlined in Ecological Character Descriptions (ECD) and
deliver improved limits of acceptable change (LoAC) to guide management
actions.



\subsection*{Expected collaborations}
Research will be designed and conducted in collaboration with Parks and
Wildlife regional staff and scientists, in particular~ecologists from
the Species and Communities Branch. External links with~research
scientists~from other state and commonwealth agencies, universities and
CSIRO will be maintained and developed further. For example, the Mandora
Marsh work is being undertaken in collaboration with Greg Skyzpek and
Pauline Grierson at The University of Western Australia.


\subsection*{Proposed period of the project}
May 23, 2016 -- May 23, 2025



\subsection*{Staff time allocation }



\begin{longtabu} to \linewidth { |  X | X | X | X | }
\hline
\rowcolor{infobg}
Role & Year 1 (2015-16 & Year 2 (2016-17 & Year 3 (2017-18)\\
\hline
\endhead



Scientist & 0.2 & 0.6 to 1.0 & 0.6 to 1.0\\



Technical & 0.2 & 0.6 to 1.0 & 0.6 to 1.0\\



Volunteer &  &  & \\



Collaborator &  &  & \\


\hline
\end{longtabu}



\subsection*{Indicative operating budget }



\begin{longtabu} to \linewidth { |  X | X | X | X | }
\hline
\rowcolor{infobg}
Source & Year 1 & Year 2 & Year 3\\
\hline
\endhead



Consolidated Funds (DPaW) &  &  & \\



External Funding & 60000 & 100000 & \\


\hline
\end{longtabu}






%-----------------------------------------------------------------------------%
% Back matter
%\backmatter
\end{document}
%-----------------------------------------------------------------------------%



\documentclass[version=last,
    paper=a4, % paper size
    10pt, % default font size
    usenames,
    dvipsnames,
    oneside, % ONLINE
    headings=openany, % open chapters on odd and even pages
    %toc=chapterentrywithdots, % Table of Contents style
    %BCOR=7mm, % PRINT Binding Correction
    %DIV=13, % typearea 161.54 mm x 228.46 mm, top margin 22.85 mm, inner margin 16.15 mm
    %DIV=14, % 165.00 233.36 21.21 15.00
    DIV=15 % 168.00 237.60 19.80 14.00
]{scrbook}
\usepackage{typearea}
\usepackage[automark,headsepline,footsepline]{scrlayer-scrpage} % Headers and footers

%%
%% Fonts, encoding, spacing, indentation
%%
\usepackage{txfonts}
\renewcommand{\familydefault}{\sfdefault} % Default to Sans Serif font
\usepackage[english]{babel}
\usepackage[T1]{fontenc}
\usepackage[utf8]{inputenc}

% Paragraph spacing
%\usepackage{parskip}    % Paragraph spacing
%\setlength{\parindent}{0em} % Don't indent paragraphs - ONLINE
%\setlength{\parskip}{1.3 ex plus 0.5ex minus 0.3ex} % 1-1.8 ex vertical space between paragraphs - ONLINE

% Spacing of headings
%\RedeclareSectionCommand[afterskip=3pt]{section} % less space after section
%\RedeclareSectionCommand[beforeskip=0cm]{subsection} % less space between HRule and project name
%\RedeclareSectionCommand[afterskip=0.1\baselineskip]{subsubsection} % less space after progressreport subheadings

% Table font size
\usepackage{etoolbox}
\AtBeginEnvironment{longtabu}{\footnotesize}{}{}

%%
%% Tables, columns, layout
%%
\usepackage{multicol}   % 2 col publications
\usepackage{pdflscape}  % Landscape pages
\usepackage{pdfpages}   % Include PDFs
\usepackage{hanging}    % hanging paragraphs for publications
%\usepackage{titletoc}   % Required for manipulating the table of contents
\setcounter{tocdepth}{2} % TOC list down to section
\usepackage{enumerate}  % Enumerations
\usepackage{enumitem}   % Enumerations
\usepackage{longtable}  % Multipage table
\usepackage{tabu}       %
\setlength{\tabulinesep}{1.5mm} % Consistent vertical spacing in tabu

%%
%% Graphics, images, colours
%%
\usepackage{graphicx} % embedded images
\usepackage{eso-pic} %
\usepackage{colortbl} % define custom named colours
\definecolor{RedFire}{RGB}{146,25,28}
\definecolor{ParksWildlife}{RGB}{0,85,144}
\definecolor{successbg}{RGB}{223,240,216}
\definecolor{errorbg}{RGB}{242,222,222}
\definecolor{warningbg}{RGB}{252,248,227}
\definecolor{infobg}{RGB}{217,237,247}
\definecolor{muted}{RGB}{153,153,153}
\definecolor{success}{RGB}{70,136,71}
\definecolor{error}{RGB}{185,74,72}
\definecolor{warning}{RGB}{192,152,83}
\definecolor{info}{RGB}{58,135,173}

\definecolor{required}{RGB}{192,152,83}
\definecolor{requiredbg}{RGB}{252,248,227}
\definecolor{denied}{RGB}{185,74,72}
\definecolor{deniedbg}{RGB}{242,222,222}
\definecolor{granted}{RGB}{70,136,71}
\definecolor{grantedbg}{RGB}{223,240,216}
\definecolor{not reqiured}{RGB}{153,153,153}
\definecolor{not requiredbg}{RGB}{255,255,255}

\usepackage{tikz} % Drawing
\usetikzlibrary{arrows,shapes,positioning,shadows,trees}

%%
%% Links, URLs
%%
\usepackage[
    linktoc=all,
    %colorlinks=false,  %PRINT
    colorlinks=true, % ONLINE
    linkcolor=RedFire, % ONLINE
    urlcolor=ParksWildlife, % ONLINE
    pdftitle=Progress Report SP 2015-017 (FY 2015-2016)
]{hyperref}

% Black magic to linebreak URLs
\usepackage{url}
\makeatletter
\g@addto@macro{\UrlBreaks}{\UrlOrds}
\makeatother

%%
%% Custom macros
%%
% Thick Horizontal rule
\newcommand{\HRule}{\vspace{8mm}\\\noindent\rule{\linewidth}{0.1pt}}

% Custom Tikz node for SDS diagram
\newcommand\mynode[6][]{
    \node[#1] (#2){
        \parbox{#3\relax}{
            \begin{center}
            \textbf{#4}\\
            #5\\
            \footnotesize{#6}
            \end{center}}};}



\usepackage[automark,headsepline,footsepline,plainfootsepline]{scrlayer-scrpage}
\automark*[section]{}
\addtokomafont{pageheadfoot}{\normalfont\footnotesize\sffamily} % Don't italicise
\renewcommand{\chaptermark}[1]{\markleft{#1}{}}     % Chapter: suppress numbering
\renewcommand{\sectionmark}[1]{\markright{#1}{}}    % Section: suppress numbering

% Header (inner, center, outer)
\ihead{\href{http://sdis.dpaw.wa.gov.au/documents/progressreport/1591/}{Progress Report SP 2015-017 (FY 2015-2016)}}
%\chead{\href{http://sdis.dpaw.wa.gov.au}{Science Directorate Information System}}
\ohead{\href{https://www.dpaw.wa.gov.au/about-us/science-and-research}{\includegraphics[height=6mm, keepaspectratio]{/mnt/projects/sdis/staticfiles/img/logo-dpaw.png}}}

% Footer (inner, center, outer)
\ifoot{\textbf{Printed}~Mon, 15 May 2017 16:05:13 +0800} % inner/left footer
\cfoot{}
\ofoot[\bfseries\thepage]{\bfseries\thepage}        % Page number (also [plain])


\pagestyle{scrheadings}
\setkomafont{pageheadfoot}{\normalfont}

%-----------------------------------------------------------------------------%
\begin{document}
\raggedbottom

%-----------------------------------------------------------------------------%
% Title page
\subject{Progress Report SP 2015-017
}
\title{Responses of aquatic invertebrate communities to changing hydrology and
water quality in streams and significant wetlands of the south-west
forests of Western Australia
}
\subtitle{Wetlands Conservation
}
\author{}
\publishers{\small
    \subsection*{Project Core Team}
\begin{tabu} {X X}
\textbf{Supervising Scientist} & Melita Pennifold
\\
\textbf{Data Custodian} & Melita Pennifold
\\
\textbf{Site Custodian} & 
\\
\end{tabu}


    \subsection*{Project status as of May 15, 2017, 4:05 p.m.}
\begin{tabu} {X X}
& Approved and active
\\
\end{tabu}

    
\subsection*{Document endorsements and approvals as of May 15, 2017, 4:05 p.m.}
\begin{tabu} {X X}

%\rowcolor{grantedbg}
    \textbf{Project Team} & 
    \textcolor{granted}{ granted}\\

%\rowcolor{grantedbg}
    \textbf{Program Leader} & 
    \textcolor{granted}{ granted}\\

%\rowcolor{grantedbg}
    \textbf{Directorate} & 
    \textcolor{granted}{ granted}\\

\end{tabu}



}
\uppertitleback{}
\lowertitleback{}
\date{}

%-----------------------------------------------------------------------------%
% Front matter
\frontmatter
\maketitle
%-----------------------------------------------------------------------------%
% Main matter
\mainmatter

\section*{Responses of aquatic invertebrate communities to changing hydrology and
water quality in streams and significant wetlands of the south-west
forests of Western Australia
}

M Pennifold, A Pinder



\section*{Context}

Aquatic habitats in the south-west of Western Australia are under
increasing threat from changes in hydrology, water quality and fire as a
result of the drying climate, historical and current land use and water
resource development. The south west of Western Australia has had a
significant reduction in rainfall since the 1970s and~it is predicted
that by 2050 there will be little stream inflow into water supply dams.
At present, there is an inadequate understanding of the responses of
aquatic communities to these threats to inform the management of many
aquatic systems in the Forest Management Plan (FMP) area, including the
Muir-Byenup Ramsar wetlands.~

This project has two components: 1) Re-surveys of aquatic invertebrates
in Muir-Byenup Ramsar wetlands sampled in 1994 and 2004 and suites of
wetlands further south sampled in 1993, addressing KPI3 of the 2014-23
Forest management Plan (FMP) and, 2) Periodic monitoring of high
condition streams, with a focus on effects of the drying climate and
forest management, addressing KPI1 of the 2014-23 FMP.




\section*{Aims}

\begin{itemize}
\itemsep1pt\parskip0pt\parsep0pt
\item
  To address KPI1 of the 2014-2023 FMP by monitoring the condition of
  currently healthy streams in relation to reduced rainfall and forest
  management practices.
\item
  To address KPI3 of the 2014-2023 FMP by determining responses of
  faunas of high value Warren Region wetlands to changes in hydrology,
  water chemistry and fire over the last 10 to 20 years.
\item
  Provide baseline data for some internationally significant wetlands,
  e.g. Lake Muir.
\item
  Use the above information to report on the current conservation
  significance of key Parks and Wildlife managed wetlands and their
  response and vulnerability to threats.
\end{itemize}




\section*{Progress}

\begin{itemize}
\itemsep1pt\parskip0pt\parsep0pt
\item
  A journal article is being prepared on how landscape modelling of
  compositional turnover in aquatic invertebrates informs conservation,
\item
  Conducted~summer 2015 sampling of aquatic invertebrates in Muir-Byenup
  Ramsar wetlands.
\item
  Processing of Muir-Byenup invertebrate samples largely completed.
\item
  Identification and vouchering~of Muir-Byenup invertebrates in
  progress.
\item
  Collaborating with taxonomists to reconcile taxonomic issues between
  1996/97, 2003/4 and 2014/15 data for the Muir-Byenup wetlands.
\end{itemize}




\section*{Management implications}

\begin{itemize}
\itemsep1pt\parskip0pt\parsep0pt
\item
  Re-surveying the Muir-Byenup Ramsar~and other high value wetlands will
  provide the region with knowledge of how these wetlands and their
  fauna have responded to threats over the last 20 years. This, in
  conjunction with results from the peat wetlands project (SPP2014-24),
  will help the Warren Region to make decisions about protecting
  remaining high conservation value wetlands versus taking remedial
  action at those where condition is declining.
\item
  Forest Management Plan commitments will be met with regard to
  measuring and assessing change in condition of 1) currently healthy
  (reference condition) stream ecosystems (KPI1) and 2) Ramsar and
  nationally listed wetlands (KPI3). Results addressing these
  commitments will inform future forest management practices.
\end{itemize}




\section*{Future directions}

\begin{itemize}
\itemsep1pt\parskip0pt\parsep0pt
\item
  Identify and voucher~Muir-Byenup invertebrates collected~in 2014/2015.
\item
  Consolidate Muir-Byenup invertebrate data from 1996/97, 2003/04 and
  2014/15 and produce a report.
\item
  Publish report with summaries of 10 year trends (2005 to 2015) for all
  stream monitoring sites.
\item
  Re-sample selected streams in 2016, with a focus on those considered
  to be in reference condition or in minimally disturbed catchments,
  plus those subject to wildfires, to provide long-term data on the
  response of aquatic invertebrate communities to declining rainfall and
  forest management.
\item
  Publish further papers examining impacts of declining rainfall and
  forest management practices on macroinvertebrate diversity in forest
  streams.
\item
  Re-survey nationally important Warren Region wetlands previously
  sampled by Horwtiz in 1997 (e.g. Owingup, Lake Jasper, Doggerup,
  Marringup, Mt.Soho Swamp) and prioritise~these wetlands for the Warren
  Region Nature Conservation Plan.
\end{itemize}



%-----------------------------------------------------------------------------%
% Back matter
%\backmatter
\end{document}
%-----------------------------------------------------------------------------%


\documentclass[version=last,
    paper=a4,                               % paper size
    10pt,                                   % default font size
    dvipsnames,
    % twoside,                                % PRINT Binding Correction
    oneside,                              % ONLINE
    headings=openany,                       % open chapters on odd and even pages
    open=any,
    BCOR=7mm,                               % PRINT Binding Correction
    %DIV=13,    % typearea 161.54mm x 228.46mm, top 22.85mm, inner 16.15mm
    %DIV=14,    % 165.00 233.36 21.21 15.00
    DIV=15,     % 168.00 237.60 19.80 14.00
    % toc=chapterentrywithdots              % Table of Contents style
]{scrbook}
\usepackage{typearea}


%------------------------------------------------------------------------------%
% Headers and footers
%------------------------------------------------------------------------------%
\usepackage[automark,headsepline,footsepline,plainfootsepline]{scrlayer-scrpage}
\automark*[section]{}
\addtokomafont{pageheadfoot}{\normalfont\footnotesize\sffamily} % Don't italicise
\renewcommand{\chaptermark}[1]{\markleft{#1}{}}     % Chapter: suppress numbering
\renewcommand{\sectionmark}[1]{\markright{#1}{}}    % Section: suppress numbering

% Header (inner, center, outer)
% \ihead{\href{http://sdis.dpaw.wa.gov.au}{\textbf{Project Plan SP 2016-068}}}
%\chead{\href{http://sdis.dpaw.wa.gov.au}{Science Directorate Information System}}
% \ohead{\href{https://www.dbca.wa.gov.au/science/10-biodiversity-and-conservation-science}{
% \includegraphics[height=8mm, keepaspectratio]{/mnt/projects/sdis/staticfiles/img/logo-dbca-bcs.jpg}}}

% Footer (inner, center, outer)
% \ifoot{\RaggedRight\leftmark}                       % Chapter
% \cfoot{\RaggedLeft\rightmark}                       % Section
% \ofoot[\bfseries\thepage]{\bfseries\thepage}        % Page number (also [plain])


%------------------------------------------------------------------------------%
% Fonts, encoding
%------------------------------------------------------------------------------%
%\usepackage{avant}             % Use the Avantgarde font for headings
\usepackage{txfonts}
\usepackage{mathptmx}
\usepackage{gensymb}            % provides \textdegree
\renewcommand{\familydefault}{\sfdefault} % Default to Sans Serif font
\usepackage{microtype}          % Slightly tweak font spacing for aesthetics
\usepackage[english]{babel}
\usepackage[utf8]{inputenc}  % Allow letters with accents
\usepackage[utf8]{luainputenc}  % Allow letters with accents
\usepackage[T1]{fontenc}        % Use 8-bit encoding that has 256 glyphs
\usepackage{textcomp}
\usepackage[explicit]{titlesec}           % Customise of titles
%\DeclareUnicodeCharacter{0080}{\textregistered}
\DeclareUnicodeCharacter{00B0}{\textdegree}

%------------------------------------------------------------------------------%
% Tables, columns, layout
%------------------------------------------------------------------------------%
\usepackage{etoolbox}
\AtBeginEnvironment{longtabu}{\footnotesize}{}{}  % Table font size
\usepackage{booktabs}           % Required for nicer horizontal rules in tables
\usepackage{multicol}           % 2 col publications
\usepackage{pdflscape}          % Landscape pages
\usepackage{pdfpages}           % Include PDFs
\usepackage{hanging}            % hanging paragraphs for publications
%\usepackage{titletoc}          % Manipulate the table of contents
\setcounter{tocdepth}{2}        % TOC list down to section
\usepackage{enumerate}          % Enumerations
\usepackage{enumitem}           % Enumerations
\usepackage{longtable}          % Multipage table
\usepackage{tabu}               %
\setlength{\tabulinesep}{1.5mm} % Consistent vertical spacing in tabu
\newcommand{\HRule}{\vspace{8mm}\noindent\rule{\linewidth}{0.1pt}}
\usepackage[export]{adjustbox}  % minipage, image frame


%------------------------------------------------------------------------------%
% Graphics, images, colours
%------------------------------------------------------------------------------%
\usepackage{graphicx} % embedded images
\usepackage{wrapfig}  % wrap figures in text
\usepackage{caption}  % allow unnumbered captions
\usepackage{eso-pic} % Required for specifying an image background in the title page
\usepackage{colortbl} % define custom named colours
\usepackage{xstring} % Conditionals
\usepackage{transparent} % Allow transparent images

\definecolor{RedFire}{RGB}{146,25,28}
% Following PICA branding guidelines
% https://dpaw.sharepoint.com/Divisions/pica/Documents/Branding%20guidelines.pdf
\definecolor{dpawblue}{RGB}{35,97,146}          % Pantone 647
\definecolor{dpaworange}{RGB}{237,139,0}        % Pantone 144
\definecolor{dpawgreen}{RGB}{116,170,80}        % Pantone 7489
\definecolor{dpawred}{RGB}{124,46,44}           % Paul's suggestion

% bootstrap colours
\definecolor{successbg}{RGB}{223,240,216}
\definecolor{errorbg}{RGB}{242,222,222}
\definecolor{warningbg}{RGB}{252,248,227}
\definecolor{infobg}{RGB}{217,237,247}
\definecolor{muted}{RGB}{153,153,153}
\definecolor{success}{RGB}{70,136,71}
\definecolor{error}{RGB}{185,74,72}
\definecolor{warning}{RGB}{192,152,83}
\definecolor{info}{RGB}{58,135,173}

% SDIS approval colours
\definecolor{required}{RGB}{192,152,83}
\definecolor{requiredbg}{RGB}{252,248,227}
\definecolor{denied}{RGB}{185,74,72}
\definecolor{deniedbg}{RGB}{242,222,222}
\definecolor{granted}{RGB}{70,136,71}
\definecolor{grantedbg}{RGB}{223,240,216}
\definecolor{notrequired}{RGB}{153,153,153}
\definecolor{notrequiredbg}{RGB}{255,255,255}

\usepackage{tikz} % Drawing
\usetikzlibrary{arrows,shapes,positioning,shadows,trees}


%------------------------------------------------------------------------------%
% Hyperlinks
%------------------------------------------------------------------------------%
\usepackage[open=true]{bookmark}
\usepackage{nameref}
\usepackage{ifxetex,ifluatex}
\ifxetex
  \usepackage[
    setpagesize=false,        % page size defined by xetex
    unicode=false,            % unicode breaks when used with xetex
    xetex]{hyperref}
\else
  \usepackage[unicode=true]{hyperref}
\fi

\hypersetup{
  backref=true,
  pagebackref=true,
  hyperindex=true,
  breaklinks=true,
  urlcolor=dpawblue,
  bookmarks=true,
  bookmarksopen=false,
  pdfauthor={Biodiversity and Conservation Science, Department of Biodiversity, Conservation and Attractions, WA},
  pdftitle=Project Plan SP 2016-068
,
  colorlinks=true,
  linkcolor=dpawblue,
  pdfborder={0 0 0}}

\urlstyle{same}                         % don't use monospace font for urlstyle


%------------------------------------------------------------------------------%
% Black magic to linebreak URLs
%------------------------------------------------------------------------------%
\usepackage{url}
\makeatletter\g@addto@macro{\UrlBreaks}{\UrlOrds}\makeatother
\Urlmuskip=0mu plus 1mu


%------------------------------------------------------------------------------%
% Fix latex errors
%------------------------------------------------------------------------------%
\providecommand{\tightlist}{\setlength{\itemsep}{0pt}\setlength{\parskip}{0pt}}

% copy-pasted HTML <span> in SDIS fields becomes \text{} in tex source
\providecommand{\text}{}


%------------------------------------------------------------------------------%
% Custom Tikz node for SDS diagram
%------------------------------------------------------------------------------%
\newcommand\mynode[6][]{
  \node[#1] (#2){
    \parbox{#3\relax}{
      \begin{center}
      \textbf{#4}\\
      #5\\
      \footnotesize{#6}
      \end{center}
    }};}


%------------------------------------------------------------------------------%
% Custom no-pagebreaks-environment
%------------------------------------------------------------------------------%
\newenvironment{absolutelynopagebreak}
  {\par\nobreak\vfil\penalty0\vfilneg\vtop\bgroup}
  {\par\xdef\tpd{\the\prevdepth}\egroup\prevdepth=\tpd}


%------------------------------------------------------------------------------%
% Remove the header from odd empty pages at the end of chapters
%------------------------------------------------------------------------------%
\makeatletter
\renewcommand{\cleardoublepage}{
\clearpage\ifodd\c@page\else
\hbox{}
\vspace*{\fill}
\thispagestyle{empty}
\newpage
\fi}


%----------------------------------------------------------------------------------------
%  Page flow control
%----------------------------------------------------------------------------------------
%\widowpenalty=10000
%\clubpenalty=10000
%\vbadness=1200
%\hbadness=11000


%----------------------------------------------------------------------------------------
%   CHAPTER HEADINGS
%----------------------------------------------------------------------------------------
\newcommand{\thechapterimage}{}
\newcommand{\chapterimage}[1]{\renewcommand{\thechapterimage}{#1}}

% Numbered chapters with mini tableofcontents
\def\thechapter{\arabic{chapter}}
\def\@makechapterhead#1{
%\thispagestyle{plain}
{\centering \normalfont\sffamily
\ifnum \c@secnumdepth >\m@ne
\if@mainmatter
\startcontents
\begin{tikzpicture}[remember picture,overlay]
\node at (current page.north west)
{\begin{tikzpicture}[remember picture,overlay]
\node[anchor=north west,inner sep=0pt] at (0,0) {
\includegraphics[width=\paperwidth,height=0.5\paperwidth]{\thechapterimage}};
%------------------------------------------------------------------------------%
% Small contents box in the chapter heading
% Mini TOC background box
%\fill[color=dpawblue!10!white,opacity=.2] (1cm,0) rectangle (
%  3.5cm, % Mini TOC box width
%  -3.5cm % Mini TOC box height
%);
% Mini TOC text content
%\node[anchor=north west] at (1.1cm,.35cm) {
%  \parbox[t][8cm][t]{6.5cm}{
%    \huge\bfseries\flushleft
%    \printcontents{l}{1}{
%    \setcounter{tocdepth}{1}                   % Mini TOC level depth
%    }
% }
%};
%------------------------------------------------------------------------------%
% Chapter heading
\draw[anchor=west] (5cm,-9cm) node [
rounded corners=20pt,
fill=dpawblue!10!white,
text opacity=1,
draw=dpawblue,
draw opacity=1,
line width=1.5pt,
fill opacity=.2,
inner sep=12pt]{
    \huge\sffamily\bfseries\textcolor{black}{
      \thechapter. #1\strut\makebox[22cm]{}
    }
};
\end{tikzpicture}};
\end{tikzpicture}}
\par\vspace*{240\p@}                            % Push text below chapter image
\fi
\fi}

%------------------------------------------------------------------------------%
% Unnumbered chapters without mini tableofcontents
%------------------------------------------------------------------------------%
\def\@makeschapterhead#1{
%\thispagestyle{plain}
{\centering \normalfont\sffamily
\ifnum \c@secnumdepth >\m@ne
\if@mainmatter
\begin{tikzpicture}[remember picture,overlay]
\node at (current page.north west)
{\begin{tikzpicture}[remember picture,overlay]
\node[anchor=north west,inner sep=0pt] at (0,0) {
  \includegraphics[width=\paperwidth,height=0.5\paperwidth]{\thechapterimage}};
% Mini TOC background box
%\fill[color=dpawblue!10!white,opacity=.2] (1cm,0) rectangle (
%  3.5cm,                                       % Mini TOC box width
%  -3.5cm                                       % Mini TOC box height
%);
% Mini TOC text content
%\node[anchor=north west] at (1.1cm,.35cm) {
%  \parbox[t][8cm][t]{6.5cm}{
%    \huge\bfseries\flushleft
%    \printcontents{l}{1}{
%    \setcounter{tocdepth}{1} % Mini TOC level depth
%    }
%}
%};
\draw[anchor=west] (5cm,-9cm) node [rounded corners=20pt,
  fill=dpawblue!10!white,fill opacity=.6,inner sep=12pt,text opacity=1,
  draw=dpawblue,draw opacity=1,line width=1.5pt]{
  \huge\sffamily\bfseries\textcolor{black}{#1\strut\makebox[22cm]{}}};
\end{tikzpicture}};
\end{tikzpicture}}
\par\vspace*{240\p@}
\fi
\fi
}
\makeatother



\usepackage[automark,headsepline,footsepline,plainfootsepline]{scrlayer-scrpage}
\automark*[section]{}
\addtokomafont{pageheadfoot}{\normalfont\footnotesize\sffamily} % Don't italicise
\renewcommand{\chaptermark}[1]{\markleft{#1}{}}     % Chapter: suppress numbering
\renewcommand{\sectionmark}[1]{\markright{#1}{}}    % Section: suppress numbering

% Header (inner, center, outer)
\ihead{\href{http://sdis.dpaw.wa.gov.au/documents/projectplan/1789/}{Project Plan SP 2016-068}}
%\chead{\href{http://sdis.dpaw.wa.gov.au}{Science Directorate Information System}}
\ohead{\href{https://www.dbca.wa.gov.au/science/10-biodiversity-and-conservation-science}{
\includegraphics[height=6mm, keepaspectratio]{/mnt/projects/sdis/staticfiles/img/logo-dbca-bcs.jpg}}}
% Footer (inner, center, outer)
\ifoot{\textbf{Printed}~Tue, 29 Jan 2019 08:11:04 +0800} % inner/left footer
\cfoot{}
\ofoot[\bfseries\thepage]{\bfseries\thepage}        % Page number (also [plain])


\pagestyle{scrheadings}
\setkomafont{pageheadfoot}{\normalfont}

%-----------------------------------------------------------------------------%
\begin{document}
\raggedbottom

%-----------------------------------------------------------------------------%
% Title page
\subject{Project Plan SP 2016-068
}
\title{South West Threatened Fauna Recovery Project: Southern Jarrah Forest
}
\subtitle{Animal Science
}
\author{}
\publishers{\small
    \subsection*{Project Core Team}
\begin{tabu} {X X}
\textbf{Supervising Scientist} & Adrian Wayne
\\
\textbf{Data Custodian} & Adrian Wayne
\\
\textbf{Site Custodian} & 
\\
\end{tabu}


    \subsection*{Project status as of Jan. 29, 2019, 8:11 a.m.}
\begin{tabu} {X X}
& Approved and active
\\
\end{tabu}

    
\subsection*{Document endorsements and approvals as of Jan. 29, 2019, 8:11 a.m.}
\begin{tabu} {X X}

%\rowcolor{grantedbg}
    \textbf{Project Team} & 
    \textcolor{granted}{ granted}\\

%\rowcolor{grantedbg}
    \textbf{Program Leader} & 
    \textcolor{granted}{ granted}\\

%\rowcolor{grantedbg}
    \textbf{Directorate} & 
    \textcolor{granted}{ granted}\\

%\rowcolor{grantedbg}
    \textbf{Biometrician} & 
    \textcolor{granted}{ granted}\\

%\rowcolor{not requiredbg}
    \textbf{Herbarium Curator} & 
    \textcolor{not required}{ not required}\\

%\rowcolor{grantedbg}
    \textbf{Animal Ethics Committee} & 
    \textcolor{granted}{ granted}\\

\end{tabu}



}
\uppertitleback{}
\lowertitleback{}
\date{}

%-----------------------------------------------------------------------------%
% Front matter
\frontmatter
\maketitle
%-----------------------------------------------------------------------------%
% Main matter
\mainmatter



\section*{South West Threatened Fauna Recovery Project: Southern Jarrah Forest
}



\subsection*{Biodiversity and Conservation Science Program}

Animal Science




\subsection*{Departmental Service}

Service 6: Conserving Habitats, Species and Communities


\subsection*{Project Staff}
\begin{tabu} {X X X}
%\rowcolor{infobg}
\textbf{Role} & \textbf{Person} & \textbf{Time allocation (FTE)}\\

Supervising Scientist & Adrian Wayne & 0.5\\

Technical Officer & Marika Maxwell & 0.9\\

Technical Officer & Colin Ward & 0.9\\

\end{tabu}




\subsection*{Related Science Projects}

\emph{other SWTFRP projects}


\subsection*{Proposed period of the project}
Feb. 1, 2016 -- Dec. 31, 2018



\section*{Relevance and Outcomes}


\subsection*{Background}

Over the last 200 years, 50 per cent of the world's mammal extinctions
have occurred in Australia (Short and Smith 1994). In Western Australia
alone, 12 mammals and two birds have become extinct and at 30 June 2015,
39 terrestrial mammals and 23 terrestrial birds native to the State were
classified as being threatened with extinction. Of these, three mammals
and two birds are ranked as critically endangered. In mammals, most
extinctions and declines have occurred in medium size species, in the so
called `critical weight range' of 35g to 5500g (Burbidge and McKenzie
1989).

Predation by feral cats and foxes is a key threatening process in the
decline of many, if not all, these species. Foxes have been successfully
controlled for many years in a range of locations across Western
Australia using dried meat sausage baits containing 1080 poison. Over a
decade of research by Parks and Wildlife scientists has led to the
development of the \emph{Eradicat®} feral cat bait, a moister meat bait,
also containing 1080, that is more palatable to feral cats.
\emph{Eradicat®} was registered for operational use in Western Australia
in December 2014, providing the opportunity to integrate broadscale
feral cat control with existing fox baiting programs and other actions,
such as translocation, that are implemented to improve the recovery of
threatened native animals.~

On 16 July 2015, the Minister for Environment, Hon Albert Jacob MLA, and
the Commonwealth Minister for the Environment, Hon Greg Hunt MP,
announced \$1.7 million in funding for the Department of Parks and
Wildlife to assist threatened animal recovery. The funding will be used
to integrate the new \emph{Eradicat®} feral cat bait with current fox
baiting in four different Western Australian environments. Sites have
been specifically selected to direct the funding to improving
conservation of species identified in the Commonwealth's Threatened
Species Strategy, and to align with fauna recovery programs already
underway or planned by Parks and Wildlife in the south-west of WA.

This project fits within a broader suite of actions delivered by Parks
and Wildlife to reduce the impacts of feral cats and foxes on threatened
species in Western Australia. This includes \emph{Western Shield}, the
Integrated Fauna Recovery Project on the south coast, Rangelands
Restoration at Matuwa and many other smaller scale projects.

The primary goal of the project is to contribute to the recovery of key
threatened mammal and bird species found at each of these sites, through
integrating feral cat baiting with existing fox baiting to reduce the
impact of introduced predators, and undertaking translocations to
establish new, secure populations, where necessary.

~While the \emph{Eradicat\textsuperscript{®}} feral cat bait is a key
tool in achieving effective feral cat control, other methods, such as
shooting and/or trapping, may also be employed. For example, some
individual cats appear to be reluctant to take baits or local climatic
conditions may reduce the effectiveness of baiting alone. Some
flexibility is built into the project plan to allow this suite of
control options to be employed, as necessary.

The project is funded for three years over the 2015-16, 2016-17 and
2017-18 financial years. A project agreement has been signed with the
Commonwealth Government outlining funding to be provided in each
financial year and key deliverables.




\subsection*{Aims}

To sustainably recover the wild populations of woylies and numbats in
the Upper Warren region, through:

(a) developing protocols to allow the effective integration of feral cat
control with existing fox control in the southern jarrah forest; and

(b) effective neighbour engagement.




\subsection*{Expected outcome}

Integration of feral cat control with the current fox control program,
including developing standard protocols for delivery of
\emph{Eradicat\textsuperscript{®}} to maximise its effectiveness and the
use of other introduced predator control techniques designed to minimise
the impact of introduced predators on native fauna;

\begin{itemize}
\itemsep1pt\parskip0pt\parsep0pt
\item
  Identification of the efficacy of Eradicat\emph{\textsuperscript{®}}
  baiting according to current operational delivery methods aerial and
  ground) and time of year.
\item
  Quantification of the risk to potentially vulnerable non-target native
  mammals in the southern jarrah forest to operational use of
  Eradicat\emph{\textsuperscript{®}}
\item
  Understand the effect of integrated fox and cat control on i)
  introduced predators, ii) non-target native mammals potentially at
  risk from management actions (i.e. chuditch, brush-tailed phascogale,
  quenda), and iii) numbats, woylies and other native species vulnerable
  to introduced predators by means of monitoring.
\item
  Community awareness and engagement to ensure understanding of and
  support for management actions.
\end{itemize}




\subsection*{Knowledge transfer}

Western Shield

Managers of introduced predators

Managers of threatened fauna species threatened by introduced predators
(e.g. critical weight range species)




\subsection*{Tasks and Milestones}

Year 1- 2015/16

\begin{longtable}[c]{@{}lllll@{}}
\toprule\addlinespace
\begin{minipage}[t]{0.17\columnwidth}\raggedright
Task
\end{minipage} & \begin{minipage}[t]{0.17\columnwidth}\raggedright
Milestones
\end{minipage} & \begin{minipage}[t]{0.17\columnwidth}\raggedright
Outputs
\end{minipage} & \begin{minipage}[t]{0.17\columnwidth}\raggedright
Timeframe
\end{minipage} & \begin{minipage}[t]{0.17\columnwidth}\raggedright
Responsibility
\end{minipage}
\\\addlinespace
\begin{minipage}[t]{0.17\columnwidth}\raggedright
Undertake research to inform the development of baiting prescriptions
\end{minipage} & \begin{minipage}[t]{0.17\columnwidth}\raggedright
Pilot trials to develop bait uptake trial methods

~
\end{minipage} & \begin{minipage}[t]{0.17\columnwidth}\raggedright
Protocols for use of remote sensor cameras

~
\end{minipage} & \begin{minipage}[t]{0.17\columnwidth}\raggedright
March -- June 2016

~
\end{minipage} & \begin{minipage}[t]{0.17\columnwidth}\raggedright
Science -- A. Wayne
\end{minipage}
\\\addlinespace
\begin{minipage}[t]{0.17\columnwidth}\raggedright
Undertake Woylie monitoring
\end{minipage} & \begin{minipage}[t]{0.17\columnwidth}\raggedright
Trapping -- Moopinup, Warrup, Boyicup
\end{minipage} & \begin{minipage}[t]{0.17\columnwidth}\raggedright
Abundance estimates
\end{minipage} & \begin{minipage}[t]{0.17\columnwidth}\raggedright
Feb-May 2016
\end{minipage} & \begin{minipage}[t]{0.17\columnwidth}\raggedright
RFMS
\end{minipage}
\\\addlinespace
\begin{minipage}[t]{0.17\columnwidth}\raggedright
Undertake Numbat monitoring
\end{minipage} & \begin{minipage}[t]{0.17\columnwidth}\raggedright
~
\end{minipage} & \begin{minipage}[t]{0.17\columnwidth}\raggedright
~
\end{minipage} & \begin{minipage}[t]{0.17\columnwidth}\raggedright
~
\end{minipage} & \begin{minipage}[t]{0.17\columnwidth}\raggedright
RFMS
\end{minipage}
\\\addlinespace
\begin{minipage}[t]{0.17\columnwidth}\raggedright
Undertake monitoring of Dasyurids
\end{minipage} & \begin{minipage}[t]{0.17\columnwidth}\raggedright
i) Trapping (chuditch) \& Nestbox transect (phascogales) -- Moopinup,
Warrup, Boyicup

ii) Nestbox grid (K4, K5, K6, Stretch)
\end{minipage} & \begin{minipage}[t]{0.17\columnwidth}\raggedright
i) Abundance estimates

ii) abundance estimates and infrastructure condition assessment
\end{minipage} & \begin{minipage}[t]{0.17\columnwidth}\raggedright
Feb-May 2016
\end{minipage} & \begin{minipage}[t]{0.17\columnwidth}\raggedright
RFMS
\end{minipage}
\\\addlinespace
\bottomrule
\end{longtable}

~

Year 2- 2016/17

\begin{longtable}[c]{@{}lllll@{}}
\toprule\addlinespace
\begin{minipage}[t]{0.17\columnwidth}\raggedright
Task
\end{minipage} & \begin{minipage}[t]{0.17\columnwidth}\raggedright
Milestones
\end{minipage} & \begin{minipage}[t]{0.17\columnwidth}\raggedright
Outputs
\end{minipage} & \begin{minipage}[t]{0.17\columnwidth}\raggedright
Timeframe
\end{minipage} & \begin{minipage}[t]{0.17\columnwidth}\raggedright
Responsibility
\end{minipage}
\\\addlinespace
\begin{minipage}[t]{0.17\columnwidth}\raggedright
Optimising Eradicat\textsuperscript{®} baiting protocols
\end{minipage} & \begin{minipage}[t]{0.17\columnwidth}\raggedright
Eradicat\textsuperscript{®} bait uptake trials
\end{minipage} & \begin{minipage}[t]{0.17\columnwidth}\raggedright
i) trial site selection and set up

ii) trials conducted to identify the most effective time and delivery
method for Eradicat\textsuperscript{®} bait deployment
\end{minipage} & \begin{minipage}[t]{0.17\columnwidth}\raggedright
i) Aug - Sept 2016

ii) Sept 2016 -- Oct 2017
\end{minipage} & \begin{minipage}[t]{0.17\columnwidth}\raggedright
i \& ii) Science -- A. Wayne
\end{minipage}
\\\addlinespace
\begin{minipage}[t]{0.17\columnwidth}\raggedright
Undertake cat/fox monitoring
\end{minipage} & \begin{minipage}[t]{0.17\columnwidth}\raggedright
GPS collars on cats to understand spatial ecology~

~
\end{minipage} & \begin{minipage}[t]{0.17\columnwidth}\raggedright
Identify how cats use the landscape (i.e. where best to target baiting)
\end{minipage} & \begin{minipage}[t]{0.17\columnwidth}\raggedright
Mar- Nov 2017
\end{minipage} & \begin{minipage}[t]{0.17\columnwidth}\raggedright
Science -- A. Wayne
\end{minipage}
\\\addlinespace
\begin{minipage}[t]{0.17\columnwidth}\raggedright
Undertake Woylie monitoring
\end{minipage} & \begin{minipage}[t]{0.17\columnwidth}\raggedright
Woylie monitored by trapping
\end{minipage} & \begin{minipage}[t]{0.17\columnwidth}\raggedright
Woylie abundance estimate
\end{minipage} & \begin{minipage}[t]{0.17\columnwidth}\raggedright
Oct-Nov 2016, Mar-Apr 2017

~
\end{minipage} & \begin{minipage}[t]{0.17\columnwidth}\raggedright
RFMS
\end{minipage}
\\\addlinespace
\begin{minipage}[t]{0.17\columnwidth}\raggedright
Undertake Numbat monitoring
\end{minipage} & \begin{minipage}[t]{0.17\columnwidth}\raggedright
Numbat monitoring by RS cameras
\end{minipage} & \begin{minipage}[t]{0.17\columnwidth}\raggedright
Numbat index of abundance
\end{minipage} & \begin{minipage}[t]{0.17\columnwidth}\raggedright
Sept 2016 - April 2017
\end{minipage} & \begin{minipage}[t]{0.17\columnwidth}\raggedright
RSFM \& Science
\end{minipage}
\\\addlinespace
\begin{minipage}[t]{0.17\columnwidth}\raggedright
Undertake monitoring of Dasyurids
\end{minipage} & \begin{minipage}[t]{0.17\columnwidth}\raggedright
i) Phascogale nest boxes monitored

ii) Chuditch monitored by trapping
\end{minipage} & \begin{minipage}[t]{0.17\columnwidth}\raggedright
i) Phascogale index of abundance

ii) chuditch abundance estimate
\end{minipage} & \begin{minipage}[t]{0.17\columnwidth}\raggedright
i) Feb \& June 2017

ii) Oct-Nov 2016, Mar-Apr 2017

~
\end{minipage} & \begin{minipage}[t]{0.17\columnwidth}\raggedright
RFMS
\end{minipage}
\\\addlinespace
\begin{minipage}[t]{0.17\columnwidth}\raggedright
Review and adapt monitoring
\end{minipage} & \begin{minipage}[t]{0.17\columnwidth}\raggedright
Report and review monitoring program
\end{minipage} & \begin{minipage}[t]{0.17\columnwidth}\raggedright
Monitoring plan beyond June 2017 for Stage 2
\end{minipage} & \begin{minipage}[t]{0.17\columnwidth}\raggedright
June 2017
\end{minipage} & \begin{minipage}[t]{0.17\columnwidth}\raggedright
RFMS
\end{minipage}
\\\addlinespace
\begin{minipage}[t]{0.17\columnwidth}\raggedright
Undertake neighbour engagement
\end{minipage} & \begin{minipage}[t]{0.17\columnwidth}\raggedright
Public information on introduced predator control
\end{minipage} & \begin{minipage}[t]{0.17\columnwidth}\raggedright
Increased awareness
\end{minipage} & \begin{minipage}[t]{0.17\columnwidth}\raggedright
June 2017
\end{minipage} & \begin{minipage}[t]{0.17\columnwidth}\raggedright
RFMS
\end{minipage}
\\\addlinespace
\begin{minipage}[t]{0.17\columnwidth}\raggedright
Undertake media /awareness
\end{minipage} & \begin{minipage}[t]{0.17\columnwidth}\raggedright
~
\end{minipage} & \begin{minipage}[t]{0.17\columnwidth}\raggedright
~
\end{minipage} & \begin{minipage}[t]{0.17\columnwidth}\raggedright
~
\end{minipage} & \begin{minipage}[t]{0.17\columnwidth}\raggedright
PICA~
\end{minipage}
\\\addlinespace
\begin{minipage}[t]{0.17\columnwidth}\raggedright
Reporting
\end{minipage} & \begin{minipage}[t]{0.17\columnwidth}\raggedright
Progress report
\end{minipage} & \begin{minipage}[t]{0.17\columnwidth}\raggedright
Progress report
\end{minipage} & \begin{minipage}[t]{0.17\columnwidth}\raggedright
June 2017
\end{minipage} & \begin{minipage}[t]{0.17\columnwidth}\raggedright
RFMS~~ \& Science
\end{minipage}
\\\addlinespace
\bottomrule
\end{longtable}

~

~

Year 3 -- 2017/18

\begin{longtable}[c]{@{}lllll@{}}
\toprule\addlinespace
\begin{minipage}[t]{0.17\columnwidth}\raggedright
Task
\end{minipage} & \begin{minipage}[t]{0.17\columnwidth}\raggedright
Milestones
\end{minipage} & \begin{minipage}[t]{0.17\columnwidth}\raggedright
Outputs
\end{minipage} & \begin{minipage}[t]{0.17\columnwidth}\raggedright
Timeframe
\end{minipage} & \begin{minipage}[t]{0.17\columnwidth}\raggedright
Responsibility
\end{minipage}
\\\addlinespace
\begin{minipage}[t]{0.17\columnwidth}\raggedright
Undertake operational feral cat baiting
\end{minipage} & \begin{minipage}[t]{0.17\columnwidth}\raggedright
Operational scale deployment of Eradicat\textsuperscript{®} baits
\end{minipage} & \begin{minipage}[t]{0.17\columnwidth}\raggedright
Cat baiting in the optimum conditions to maximise efficacy with low risk
to non-targets
\end{minipage} & \begin{minipage}[t]{0.17\columnwidth}\raggedright
TBC based on earlier work that identifies best timing (nominally Feb
2018)
\end{minipage} & \begin{minipage}[t]{0.17\columnwidth}\raggedright
RFMS
\end{minipage}
\\\addlinespace
\begin{minipage}[t]{0.17\columnwidth}\raggedright
Undertake cat/fox monitoring
\end{minipage} & \begin{minipage}[t]{0.17\columnwidth}\raggedright
Camera monitoring

Radio-collars on cats
\end{minipage} & \begin{minipage}[t]{0.17\columnwidth}\raggedright
Effect of Eradicat\textsuperscript{®} baiting event on introduced
predator abundance and survivorship (i.e. efficacy quantified)
\end{minipage} & \begin{minipage}[t]{0.17\columnwidth}\raggedright
TBC dependent on the optimal timing of baiting, which will be identified
in stage 1 of this project
\end{minipage} & \begin{minipage}[t]{0.17\columnwidth}\raggedright
Science -- A. Wayne
\end{minipage}
\\\addlinespace
\begin{minipage}[t]{0.17\columnwidth}\raggedright
Undertake Woylie monitoring
\end{minipage} & \begin{minipage}[t]{0.17\columnwidth}\raggedright
Woylie monitored by trapping
\end{minipage} & \begin{minipage}[t]{0.17\columnwidth}\raggedright
Effect of Eradicat\textsuperscript{®} baiting event on Woylie abundance
\end{minipage} & \begin{minipage}[t]{0.17\columnwidth}\raggedright
As above
\end{minipage} & \begin{minipage}[t]{0.17\columnwidth}\raggedright
RFMS
\end{minipage}
\\\addlinespace
\begin{minipage}[t]{0.17\columnwidth}\raggedright
Undertake Numbat monitoring
\end{minipage} & \begin{minipage}[t]{0.17\columnwidth}\raggedright
Numbat monitoring by RS cameras
\end{minipage} & \begin{minipage}[t]{0.17\columnwidth}\raggedright
Effect of Eradicat\textsuperscript{®} baiting event on Numbat abundance
\end{minipage} & \begin{minipage}[t]{0.17\columnwidth}\raggedright
As above
\end{minipage} & \begin{minipage}[t]{0.17\columnwidth}\raggedright
RSFM \& Science
\end{minipage}
\\\addlinespace
\begin{minipage}[t]{0.17\columnwidth}\raggedright
Undertake monitoring of Dasyurids
\end{minipage} & \begin{minipage}[t]{0.17\columnwidth}\raggedright
i) Phascogale nest boxes monitored

ii) Chuditch monitored by trapping

iii) Chuditch survivorship monitoring by radio-telemetry (funding
dependent)
\end{minipage} & \begin{minipage}[t]{0.17\columnwidth}\raggedright
Effect of Eradicat\textsuperscript{®} baiting event on Dasyurid
abundance and survivorship (i.e. risk to potentially vulnerable
non-targets quantified)
\end{minipage} & \begin{minipage}[t]{0.17\columnwidth}\raggedright
As above
\end{minipage} & \begin{minipage}[t]{0.17\columnwidth}\raggedright
RFMS
\end{minipage}
\\\addlinespace
\begin{minipage}[t]{0.17\columnwidth}\raggedright
Review and adapt monitoring
\end{minipage} & \begin{minipage}[t]{0.17\columnwidth}\raggedright
Report and review monitoring program
\end{minipage} & \begin{minipage}[t]{0.17\columnwidth}\raggedright
Monitoring plan beyond Stage 2
\end{minipage} & \begin{minipage}[t]{0.17\columnwidth}\raggedright
June 2018
\end{minipage} & \begin{minipage}[t]{0.17\columnwidth}\raggedright
RFMS
\end{minipage}
\\\addlinespace
\begin{minipage}[t]{0.17\columnwidth}\raggedright
Undertake neighbour engagement
\end{minipage} & \begin{minipage}[t]{0.17\columnwidth}\raggedright
Public information on introduced predator control
\end{minipage} & \begin{minipage}[t]{0.17\columnwidth}\raggedright
Increased awareness
\end{minipage} & \begin{minipage}[t]{0.17\columnwidth}\raggedright
June 2017
\end{minipage} & \begin{minipage}[t]{0.17\columnwidth}\raggedright
RFMS
\end{minipage}
\\\addlinespace
\begin{minipage}[t]{0.17\columnwidth}\raggedright
Undertake media /awareness
\end{minipage} & \begin{minipage}[t]{0.17\columnwidth}\raggedright
~
\end{minipage} & \begin{minipage}[t]{0.17\columnwidth}\raggedright
~
\end{minipage} & \begin{minipage}[t]{0.17\columnwidth}\raggedright
~
\end{minipage} & \begin{minipage}[t]{0.17\columnwidth}\raggedright
PICA~
\end{minipage}
\\\addlinespace
\begin{minipage}[t]{0.17\columnwidth}\raggedright
Reporting
\end{minipage} & \begin{minipage}[t]{0.17\columnwidth}\raggedright
Final project report
\end{minipage} & \begin{minipage}[t]{0.17\columnwidth}\raggedright
Final project report
\end{minipage} & \begin{minipage}[t]{0.17\columnwidth}\raggedright
Dec 2018
\end{minipage} & \begin{minipage}[t]{0.17\columnwidth}\raggedright
RFMS \& Science
\end{minipage}
\\\addlinespace
\bottomrule
\end{longtable}

~~

~




\subsection*{References}

Burbidge, A.A., McKenzie, N.L., 1989. Patterns in the modern decline of
Western Australia's vertebrate fauna: causes and conservation
implications. Biological Conservation 50, 143-198.

Forest and Ecosystem Management Division, 2015. \emph{Feral Cat Baiting
Prescription}. Department of Parks and Wildlife

Martin, G.R., Twigg, L.E., Marlow, N.J., Kirkpatrick, W.E., King, D.R.,
Gaikhorst, G., 2002. The acceptability of three types of predator baits
to captive non-target animals. Wildlife Research 29, 489-502.

Parks and Wildlife, 2012. Training Manual: Safe and effective use of
1080 for vertebrate pest control. Training version 5 (1/5/2012)

Short, J., Smith, A., 1994. Mammal decline and recovery in Australia.
Journal of Mammalogy 75(2), 288-297.

Thomas, M.L., Parry, L.J., Allan, R.A., Elgar, M.A., 1999. Geographic
affinity, cuticular hydrocarbons and colony recognition in the
Australian meat ant \emph{Iridomyrmex purpureus}. Naturwissenschaften
86, 87-92.

Van Dyck, S., Strahan, R., 2008. The mammals of Australia. Thrid
edition. Reed New Holland, Sydney.

Wayne, A.F., Maxwell, M.A., Ward, C.G., Vellios, C.V., Wilson, I.J.,
Dawson, K.E., 2013. Woylie Conservation and Research Project: Progress
Report 2010-2013. Department of Parks and Wildlife, Perth, Western
Australia.

Wayne, A.F., Maxwell, M.A., Ward, C.G., Wayne, J.C., Vellios, C.V.,
Wilson, I., in review. Recoveries and cascading declines of native
mammals associated with control of an introduced predator. Journal of
Mammalogy.

~



\section*{Study design}


\subsection*{Methodology}

\section{Project summary}

GOAL: To sustainably recover wild populations of woylies and numbats in
the Upper Warren region through effective integration of feral cat
control with existing fox control.

\textbf{~}

OBJECTIVE: Develop an effective cat and fox control program using
Eradicat\emph{\textsuperscript{®}} and Probait baits that reduces the
impact of introduced predators sufficiently to support sustainable
populations of vulnerable threatened native fauna in the southern jarrah
forests.

~

\textbf{Stage 1 -- Optimising
Eradicat}\emph{\textsuperscript{®}}\textbf{baiting protocols}

Specific objective: Determine when, how and where to deploy
Eradicat\emph{\textsuperscript{®}} baits using the current operational
protocols for aerial and ground deployment (Feral Cat Baiting
Prescription; Forest and Ecosystem Management Division, 2015) that
maximises the number/proportion of cats killed by baits and minimises
the potential risks to non-target native species within the southern
jarrah forest bioregion.

Strategies: i) use bait uptake trials (using remote sensor cameras) and
ii) understand the spatial ecology (using GPS collars and
radio-telemetry) of a sample of feral cats to identify when, how and
where Eradicat\textsuperscript{®} baits may maximise their impact on
cats and minimise their possible risk to non-target native species.

~

\textbf{Stage 2 -- Operational deployment of
Eradicat\textsuperscript{®}}

Specific objectives: Quantify the effects of an optimised operational
use of Eradicat\textsuperscript{®} ~baits (informed by stage 1) at the
individual and population levels of i) feral cats and foxes, ii)
potentially vulnerable non-target native species, and iii) priority
native species threatened by introduced predators (Numbats, Woylies,
etc) within the Upper Warren region.

Strategy: Conduct an Eradicat\textsuperscript{®} baiting program within
a limited area of the Upper Warren (e.g. 225-400 km\textsuperscript{2}),
within a comparative / experimental framework, to assess the
effectiveness on target species (cats and foxes) and effects on
non-target species (with a focus on potentially vulnerable and priority
beneficiaries).

~

\textbf{Subsequent stages beyond the scope of the current project}

Stage 3 Objective: Operationally integrate Eradicat\textsuperscript{®}
and Probait into the introduced predator control and monitoring program
across the Upper Warren region (Western Shield's Manjimup baiting cell).

Stage 4 Objective: Consider the merits of rolling out the improved
introduced predator control and monitoring program across other forested
landscapes in south-western Australia.

~

\section{Project area}

Stage 1 (optimising baiting protocols) will be conducted in the southern
jarrah forest ecosystems managed by Parks and Wildlife (predominantly in
the Southern Jarrah Forest (JAF02) IBRA subregion), where standard fox
baiting is currently applied and where feral cat control is a priority.

Stages 2 \& 3 (larger scale operational baiting trial) will be conducted
in the Upper Warren region given that it is a priority area for
threatened mammal conservation.

~

\section{Stage 1 - Optimising Eradicat\textsuperscript{®} baiting
protocols}

Proposed program overview:

\begin{enumerate}
\itemsep1pt\parskip0pt\parsep0pt
\item
  a) Pilot trial and method development
\item
  b) Eradicat\textsuperscript{®} bait uptake trials to determine when,
  how and where the best time to bait is
\item
  c) Bait condition trials to determine how the application of Coopex
  and environmental conditions affect Eradicat\textsuperscript{®} bait
  characteristics {[}dependent on securing additional funds{]}
\item
  d) Spatial ecology of cats to determine where in the landscape to
  target baiting to maximise the chance of baits being encountered by
  cats {[}dependent on securing additional funds{]}
\end{enumerate}

\section{~}

\section{1a) Pilot trial and method development}

Summary: Technical and methodological trials will be conducted to
determine the most effective and efficient protocols for the use of
remote sensor cameras (RSC) to detect the consumers of
Eradicat\textsuperscript{®} baits deployed in the southern jarrah
forest.

Objectives:

\begin{enumerate}
\itemsep1pt\parskip0pt\parsep0pt
\item
  Determine the longevity of Eradicat\textsuperscript{®} baits in the
  southern jarrah forest (i.e. the duration until the bait is
  consumed/removed to ascertain how long RSCs need to be deployed to
  capture the consumption of baits during the subsequent bait uptake
  trials).
\item
  Identify the optimal set up of the RSCs to maximize the detection of
  fauna interacting with the Eradicat\textsuperscript{®} bait (i.e. by
  means of a comparison of camera orientations and distances from the
  bait).
\end{enumerate}

~

\section{1b) Bait uptake trials}

Summary: Bait uptake trials will be conducted over at least 12-14 months
using covert remote sensor cameras to observe species/individuals
consuming Eradicat\textsuperscript{®} baits. This is the preferred
method based on affordability, efficiency and return for effort, for
directly informing when, how and where it is best to bait for cats and
assessing the risk to potential vulnerable non-target species.

~

Key principles:

\begin{enumerate}
\itemsep1pt\parskip0pt\parsep0pt
\item
  The Eradicat\textsuperscript{®} bait trials should resemble
  operational conditions as closely as possible. E.g. - mimic the
  spatial pattern of bait deployment by aircraft and ground deployment
  (Note bait location will not be biased to open areas where cameras can
  detect animals and observe their interactions with the bait) - bait
  treatment and characteristics (e.g. sweating, coopex, age, toxicity,
  etc)
\item
  Observer effects should be minimised as much as practically possible
  E.g. - remote sensor cameras need to be as covert as practically
  possible - minimise disturbance to sites as much as practically
  possible (including anthropogenic cues such as smell, etc) - allow
  enough time for animals to become accustomed to cameras and other
  infrastructure - minimise the cues for bait deployment (e.g. dummy
  cameras to be in place for an extended period beforehand so animals do
  not learn that cameras mean food)
\end{enumerate}

Other considerations:

\begin{itemize}
\itemsep1pt\parskip0pt\parsep0pt
\item
  Spatial independence between sites is required to maximise the
  encounters of different cats and minimise the influence of earlier
  trials nearby (e.g. removal of cats by toxic baits, opportunities for
  learning by individuals) that in turn affect inferences, etc (i.e.
  trials need to be conducted across all or much of the southern jarrah
  forest).
\item
  A balance is needed between having a design that is robust and that
  will provide adequate and appropriate data that can be used to
  confidently identify the most effective cat baiting regime in the
  southern jarrah forests, and having a design that is practical and
  affordable with the time and resources available.
\item
  Trials conducted across larger spatial and temporal scales will
  provide better results - More replicate sites will provide more data
  and results that can better account for the variability between sites
  - Trials over multiple years will more confidently confirm when and
  under what circumstances baiting is likely to be most effective (but
  this is not currently planned and may not be practical).
\item
  Repeat trials on the same sites should be avoided or minimised.
  Ideally sites should not be subject to repeat trials more than once
  every 12 months; to reduce issues associated with the effects of
  earlier trials influencing the results of subsequent trials (e.g.
  learning by animals, removal of individuals by toxic baits).
\item
  Additional benefits from this project include information on the
  distribution and activity/occupancy/abundance of mammal fauna (e.g.
  numbats, quokka, woylies, ngwayir, wambenger, tammar, quenda,
  chuditch, koomal, western brush wallaby, introduced mammals, etc) in
  the southern jarrah forest.
\end{itemize}

Aerial and ground baiting trials:

\begin{itemize}
\itemsep1pt\parskip0pt\parsep0pt
\item
  Both aerial and ground bait delivery protocols should be tested given
  differences in cat detection on and off track (e.g. Wayne\emph{et al.}
  2013); Aerial deployment = 50 baits km\textsuperscript{-2} scattered
  as 50 baits within a 200m x 40m area every 1 km along transect lines
  space 1 km apart Ground deployment = 100 m intervals along road/track
  access allowing for replacement during the ``baiting window'' but not
  exceeding 50 baits/linear km/year
\item
  Ground baiting is likely to be more successful at baiting cats (i.e.
  adjacent to tracks) but operationally more limiting (i.e. higher
  operational costs over equivalent areas; patchy delivery given
  limitations to tracks that provide adequate forest access; and,
  potentially access limitations at some times of the year due to
  dieback hygiene constraints). During the bait uptake trials, camera
  security is likely to be substantially more problematic on- versus
  off-track.
\end{itemize}

~

\subsubsection{Best pragmatic design for Eradicat\textsuperscript{®}
bait uptake trials}

2 x deployment methods (aerial and ground)

2 x replicates

10 x trials

= 40 sites. (Table 1).

\textbf{Table 1. The experimental design for the
Eradicat}\textsuperscript{®}\textbf{bait uptake trials, showing in which
site (as indicated by its number) each trial will be conducted.}

In each of 10 sessions four randomly assigned sites will have
Eradicat\textsuperscript{®} baits deployed in a similar spatial pattern
to operational protocols for aerial or ground baiting, with two
replicates for each deployment method. Sites (numbered 1-40 in the table
matrix) are spatially independent (\textgreater{}10 km) and randomly
selected grid cells (5 x 5 km) in southern jarrah forest ecosystem type.

\begin{longtable}[c]{@{}lllll@{}}
\toprule\addlinespace
\begin{minipage}[t]{0.17\columnwidth}\raggedright
Deployment protocol
\end{minipage} & \begin{minipage}[t]{0.17\columnwidth}\raggedright
Aerial
\end{minipage} & \begin{minipage}[t]{0.17\columnwidth}\raggedright
Aerial
\end{minipage} & \begin{minipage}[t]{0.17\columnwidth}\raggedright
Ground
\end{minipage} & \begin{minipage}[t]{0.17\columnwidth}\raggedright
Ground
\end{minipage}
\\\addlinespace
\begin{minipage}[t]{0.17\columnwidth}\raggedright
Replicate
\end{minipage} & \begin{minipage}[t]{0.17\columnwidth}\raggedright
1
\end{minipage} & \begin{minipage}[t]{0.17\columnwidth}\raggedright
2
\end{minipage} & \begin{minipage}[t]{0.17\columnwidth}\raggedright
1
\end{minipage} & \begin{minipage}[t]{0.17\columnwidth}\raggedright
2
\end{minipage}
\\\addlinespace
\begin{minipage}[t]{0.17\columnwidth}\raggedright
Session
\end{minipage} & \begin{minipage}[t]{0.17\columnwidth}\raggedright
~
\end{minipage} & \begin{minipage}[t]{0.17\columnwidth}\raggedright
~
\end{minipage} & \begin{minipage}[t]{0.17\columnwidth}\raggedright
~
\end{minipage} & \begin{minipage}[t]{0.17\columnwidth}\raggedright
~
\end{minipage}
\\\addlinespace
\begin{minipage}[t]{0.17\columnwidth}\raggedright
1
\end{minipage} & \begin{minipage}[t]{0.17\columnwidth}\raggedright
39
\end{minipage} & \begin{minipage}[t]{0.17\columnwidth}\raggedright
23
\end{minipage} & \begin{minipage}[t]{0.17\columnwidth}\raggedright
8
\end{minipage} & \begin{minipage}[t]{0.17\columnwidth}\raggedright
6
\end{minipage}
\\\addlinespace
\begin{minipage}[t]{0.17\columnwidth}\raggedright
2
\end{minipage} & \begin{minipage}[t]{0.17\columnwidth}\raggedright
16
\end{minipage} & \begin{minipage}[t]{0.17\columnwidth}\raggedright
11
\end{minipage} & \begin{minipage}[t]{0.17\columnwidth}\raggedright
20
\end{minipage} & \begin{minipage}[t]{0.17\columnwidth}\raggedright
9
\end{minipage}
\\\addlinespace
\begin{minipage}[t]{0.17\columnwidth}\raggedright
3
\end{minipage} & \begin{minipage}[t]{0.17\columnwidth}\raggedright
29
\end{minipage} & \begin{minipage}[t]{0.17\columnwidth}\raggedright
4
\end{minipage} & \begin{minipage}[t]{0.17\columnwidth}\raggedright
34
\end{minipage} & \begin{minipage}[t]{0.17\columnwidth}\raggedright
17
\end{minipage}
\\\addlinespace
\begin{minipage}[t]{0.17\columnwidth}\raggedright
4
\end{minipage} & \begin{minipage}[t]{0.17\columnwidth}\raggedright
38
\end{minipage} & \begin{minipage}[t]{0.17\columnwidth}\raggedright
10
\end{minipage} & \begin{minipage}[t]{0.17\columnwidth}\raggedright
31
\end{minipage} & \begin{minipage}[t]{0.17\columnwidth}\raggedright
40
\end{minipage}
\\\addlinespace
\begin{minipage}[t]{0.17\columnwidth}\raggedright
5
\end{minipage} & \begin{minipage}[t]{0.17\columnwidth}\raggedright
26
\end{minipage} & \begin{minipage}[t]{0.17\columnwidth}\raggedright
2
\end{minipage} & \begin{minipage}[t]{0.17\columnwidth}\raggedright
21
\end{minipage} & \begin{minipage}[t]{0.17\columnwidth}\raggedright
5
\end{minipage}
\\\addlinespace
\begin{minipage}[t]{0.17\columnwidth}\raggedright
6
\end{minipage} & \begin{minipage}[t]{0.17\columnwidth}\raggedright
7
\end{minipage} & \begin{minipage}[t]{0.17\columnwidth}\raggedright
33
\end{minipage} & \begin{minipage}[t]{0.17\columnwidth}\raggedright
14
\end{minipage} & \begin{minipage}[t]{0.17\columnwidth}\raggedright
12
\end{minipage}
\\\addlinespace
\begin{minipage}[t]{0.17\columnwidth}\raggedright
7
\end{minipage} & \begin{minipage}[t]{0.17\columnwidth}\raggedright
37
\end{minipage} & \begin{minipage}[t]{0.17\columnwidth}\raggedright
32
\end{minipage} & \begin{minipage}[t]{0.17\columnwidth}\raggedright
25
\end{minipage} & \begin{minipage}[t]{0.17\columnwidth}\raggedright
18
\end{minipage}
\\\addlinespace
\begin{minipage}[t]{0.17\columnwidth}\raggedright
8
\end{minipage} & \begin{minipage}[t]{0.17\columnwidth}\raggedright
3
\end{minipage} & \begin{minipage}[t]{0.17\columnwidth}\raggedright
28
\end{minipage} & \begin{minipage}[t]{0.17\columnwidth}\raggedright
22
\end{minipage} & \begin{minipage}[t]{0.17\columnwidth}\raggedright
24
\end{minipage}
\\\addlinespace
\begin{minipage}[t]{0.17\columnwidth}\raggedright
9
\end{minipage} & \begin{minipage}[t]{0.17\columnwidth}\raggedright
15
\end{minipage} & \begin{minipage}[t]{0.17\columnwidth}\raggedright
35
\end{minipage} & \begin{minipage}[t]{0.17\columnwidth}\raggedright
36
\end{minipage} & \begin{minipage}[t]{0.17\columnwidth}\raggedright
27
\end{minipage}
\\\addlinespace
\begin{minipage}[t]{0.17\columnwidth}\raggedright
10
\end{minipage} & \begin{minipage}[t]{0.17\columnwidth}\raggedright
30
\end{minipage} & \begin{minipage}[t]{0.17\columnwidth}\raggedright
1
\end{minipage} & \begin{minipage}[t]{0.17\columnwidth}\raggedright
13
\end{minipage} & \begin{minipage}[t]{0.17\columnwidth}\raggedright
19
\end{minipage}
\\\addlinespace
\bottomrule
\end{longtable}

~

\begin{itemize}
\itemsep1pt\parskip0pt\parsep0pt
\item
  Missing data will make analysis much more difficult. The orthogonal
  and balanced design is very important to maintain throughout the
  study.
\end{itemize}

\begin{itemize}
\itemsep1pt\parskip0pt\parsep0pt
\item
  Having 40 sites provides some flexibility in the timing of trials but
  an incomplete picture throughout the calendar year (i.e. aerial and
  ground bait trials will only be active during 33\% and 67\% of the
  year, respectively). The times of year that are omitted can be regular
  (e.g. 2 weeks between each trial) or strategic (e.g. avoid periods
  when baiting effectiveness may otherwise be expected to be poor, such
  as the wettest times of the year when bait condition may be
  compromised).
\end{itemize}

\subsubsection{Site selection for Eradicat\textsuperscript{®} bait
uptake trials:}

The southern jarrah forest ecosystem will be subdivided into 5 x 5 km
cells (i.e. 2,500 ha each; Figure 1). Criteria for candidate cells
include;

\begin{itemize}
\itemsep1pt\parskip0pt\parsep0pt
\item
  Study area: southern jarrah forest (SJF) ecosystem and other minor
  jarrah forest ecosystem types immediately adjacent to SJF (Jarrah -
  Unicup, woodland, and north east ecosystem types) within the IBRA
  subregion JAF02 and Warren IBRA bioregion (WAR)), i.e.
  \textasciitilde{}197 cells between Nannup in the northwest, Tonebride
  in the north east, Rocky Gully and Nornalup in the south east and
  Peerabeelup in the southwest.
\item
  \textgreater{}75\% area having Conservation Commission land tenure
  and/or land managed by the Department of Parks and Wildlife.
\item
  \textgreater{}75\% currently subject to Western Shield fox control
\item
  Practical access available and relatively close to Manjimup (i.e. to
  reduce costs)
\item
  Candidate areas planned to be burned or timber-harvested during
  2016/17 will need to be considered carefully to co-ordinate the trials
  where possible when/where they won't be adversely impacted
\end{itemize}

~

Cell selection from the pool of candidates will be stratified-random --
i.e. the initial cell will be randomly selected and then subsequent
cells will be selected based on having a spatial separation of at least
10 km from other selected cells. Each selected cell will be randomly
assigned to the sequential order of when and how (aerial or a ground)
the baits are deployed.

~

~

\textbf{Figure 1.} a) 5 x 5 km grid cells with predominantly southern
jarrah forest ecosystems on lands managed by the Dept. of Parks and
Wildlife. b) Indicative location of the 40 Stage 1b sites to determine
how, when, where best to bait using Eradicat\textsuperscript{®}.

\subsubsection{Site layout for Eradicat\textsuperscript{®} bait uptake
trials}

Aerial bait treatment sites will be;

\begin{itemize}
\itemsep1pt\parskip0pt\parsep0pt
\item
  located as close to the centre of the cell as possible (to maximise
  spatial separation from other selected treatment sites)
\item
  \textgreater{} 100 m from trafficable tracks and \textgreater{}200 m
  from main public roads (for site security)
\item
  Each site will have 50 bait stations roughly evenly spaced in an
  approximate grid layout (4 x 13 rows at 13 m x 16 m intervals) within
  an area of about 200 m x 40 m -- to resemble operational protocols for
  aerial cat baiting as closely as possible (Algar et al. 2013; ``Feral
  Cat Baiting Prescription'', Forest and Ecosystem Management Division
  2015).
\item
  Note: the spatial separations between aerial bait sites will be much
  greater than the operational deployment of 50 baits
  km\textsuperscript{-1} -- this will affect how the results can be
  interpreted (e.g. the likelihood of detecting a bait consumption by an
  introduced predator during the trials may be higher than if the baits
  were deployed at operational densities (conversely the proportion of
  cats removed will be lower); similarly, the lower than operational
  density of baits at the landscape level during the trials will the
  lower the risk to chuditch but not quantify the risk to individuals
  that may encounter multiple bait clusters deployed at operational
  densities).
\end{itemize}

~

Ground bait deployment treatments sites will be;

\begin{itemize}
\itemsep1pt\parskip0pt\parsep0pt
\item
  5 km transects positioned as centrally as possible within the cell and
  maximising the distance from other selected cells nearby
\item
  Forest tracks frequently used by vehicles will be avoided to minimise
  the disruption cause by temporary road closures.
\item
  The roads will be closed to unauthorised access along the baiting
  transect during the baiting trial on that site (i.e. 4-8 weeks) by
  means of signs and flagging tape. This is to reduce the baiting risk
  to humans and their animal companions and to reduce the risk of
  interference or theft of cameras.
\item
  Baits will be at 100 m intervals along the 5 km road/track transect.
  The bait stations will be 5-10 m off the track with the cameras
  positioned to conceal them from the track as much as practically
  possible.
\end{itemize}

~

\subsubsection{Bait stations for both aerial and ground treatments:}

Each bait station will have a dummy remote sensor camera deployed for at
least 4 weeks prior to the start of the bait trials. A real remote
sensor camera (a randomly allocated Reconyx HC 600 or PC900 model) with
a toxic Eradicat\textsuperscript{®} bait at the centre of the field of
view will be deployed at the beginning of the bait trials at each site.

The camera will be as concealed as practically possible to reduce
detection (i.e. within or adjacent to existing natural structures such
as vegetation, logs or debris) but adjacent to and focussed on an open
area in which the bait can be placed and monitored by the camera.

Some minimal modification of the vegetation in front of the camera and
around the bait location may be required (e.g. selective light pruning)
to improve camera surveillance of the bait and the animals around the
bait.

The dummy cameras are necessary to reduce the cues and learning by
animals of baits being available and to reduce the aversion of some
animals to new and artificial objects in their environment (i.e.
minimise behavioural responses of animals to the trials -- i.e. observer
effect).

~

Camera set-up protocols for Eradicat\textsuperscript{®} bait uptake
trials:

\begin{itemize}
\itemsep1pt\parskip0pt\parsep0pt
\item
  Cameras to face between SE and SW (ideally S)
\item
  Distance to bait 1.5 m
\item
  Cameras to be at height 20-30 cm above ground
\item
  Camera settings: 10 images per trigger (HC600 models) or more (PC900
  models). No delay between triggers. High sensitivity, motion sensor
  on.
\item
  Security measures: code lock activated, cameras labelled indicating
  security code protected and engraved,
\item
  no flagging tape or any other visible indications of the location of
  the site or camera/bait stations. GPS co-ordinates of each camera to
  be recorded.
\end{itemize}

\subsubsection{Bait preparation for Eradicat\textsuperscript{®} bait
uptake trials}

Bait characteristics will be identical to those used operationally
including, manufacturing process, toxicity, age and preparation.
Eradicat\textsuperscript{®} bait preparation will be done in accordance
with the ``Feral Cat Baiting Prescription'' (Forest and Ecosystem
Management Division, 2015). Prior to deployment;

\begin{itemize}
\itemsep1pt\parskip0pt\parsep0pt
\item
  Sweat on racks under sunny conditions to allow the oils and
  lipid-soluble digest material to exude from the surface of the bait.
\item
  All baits are sprayed on all sides, during the sweating process, with
  an ant deterrent compound (Coopex) at a concentration of 12.5 g
  l\textsuperscript{-1} as per the manufacturer's instructions.
\end{itemize}

\subsubsection{Timing of Eradicat\textsuperscript{®} bait uptake trials}

\begin{itemize}
\itemsep1pt\parskip0pt\parsep0pt
\item
  Four trials (sites) will run simultaneously within a session; 2
  replicates of each of two treatments (aerial and ground baiting)
  (Table 1, Appendix 3).
\item
  The length of each trial will include 2-4 weeks with dummy cameras
  then 2 and 4 weeks for the aerial and ground bait trials,
  respectively, using real cameras and Eradicat\textsuperscript{®}
  baits.
\item
  The bait uptake trials will be conducted predominantly in the
  warmer-drier periods of the year (October -- April) and
  opportunistically during the forecast drier spells (i.e. 6-7 days with
  \textless{}5-10 mm total rainfall forecast) in the cooler-wetter
  months between May and September inclusive (i.e. \textgreater{}50 mm
  average per month). This is in general accordance with the baiting
  prescriptions designed to avoid moist conditions in which the bait
  quality may be reduced (Forest and Ecosystem Management Division
  2015). While specifics are not provided, a general operational
  guideline is to aim for ≤5mm of rainfall on the day of baiting and a
  minimum of 5 days post baiting (Gareth Watkins pers comm.). But we
  consider temperature, humidity and evaporation rates also to be
  important factors that affect the duration and magnitudes of ambient
  moisture levels and there is currently insufficient detail available
  on how environmental conditions affect bait quality or palatability.
  This may be examined as part of this study (see part 1c details
  below). Note that Western Shield protocols for the use of 1080 poison
  grain baits and dried meat baits is for deployment in conditions when
  \textless{} 6 mm and \textless{}40- 50 mm rainfall is forecast ,
  respectively (Parks and Wildlife 2012).
\item
  The timing of trials within the group of four sites within a session
  will be staggered (i.e. start on four successive days, as they are set
  up) so long as rainfall and extreme weather events are avoided during
  the four successive setup days and the 3-4 initial days post bait
  deployment (i.e. similar starting conditions for the four sites). The
  order of staggered entry within a group will be randomised. Some
  flexibility in the timing of the start of the trials will be needed to
  accommodate weather forecasts. If poor weather is forecast, the
  alternative is to have a staggered setup and simultaneous trial start
  (i.e. set up cameras over successive days then deploy the baits
  simultaneously at all four sites when the weather is clear - note this
  requires more resources / visits to the field sites). Note at least 2
  days (3-4 people) are required to set up the 50 camera/bait stations
  at a site and another full day to deploy and activate the cameras.
\item
  Aerial baiting treatments will only have one deployment of 50 baits
  per trial. The ground baiting treatments will have baits replenished
  on multiple occasions (\textless{} 5 baits per station) within the
  four week trial period at each site, in accordance with the protocols
  (``\ldots{}Baits can be replaced during the ``baiting window'' if
  taken or spoiled but should not exceed 50 baits in total/linear
  km/year\ldots{}'', Forest and Ecosystem Management Division, 2015).
  i.e. up to 250 baits per trial
\end{itemize}

\subsubsection{Research questions and analysis approach for
Eradicat\textsuperscript{®} bait uptake trials}

~

\textbf{General theme: Eradicat}\textsuperscript{®}\textbf{bait
consumption differences in relation to when, how and where baits are
deployed and other factors (i.e. bait effectiveness)}

~

Common response variable(s):

\begin{itemize}
\itemsep1pt\parskip0pt\parsep0pt
\item
  Number or proportion of baits consumed by species `\emph{x'} per
  site/trial
\item
  Number of individuals of species `\emph{x'} that consumed 1 or more
  baits per site/trial
\end{itemize}

Note: data may be sparse (lots of zeros) for bait consumption by cats,
which will have implications for analysis

~

~

\textbf{Q1: Does Eradicat}\textsuperscript{®}\textbf{bait consumption by
the target (feral cat) and non-target species vary according to time of
year?}

Plot: Average number and proportion of baits consumed by species
`\emph{x'} per trial within each session (y-axis) by start date of each
session (x-axis)

Analysis: possibly just descriptive graphics given temporal factors will
be examined in more detail in subsequent questions

Results: Identifies generally when bait consumption by cats and other
species is greatest but does not consider differences or interactions
due to other factors such as deployment method, habitat and landscape
factors, etc.

\textbf{~}

\textbf{Q2: Does Eradicat}\textsuperscript{®}\textbf{bait consumption by
the target (feral cat) and non-target species vary according to season,
within season (session), and deployment method?}

~

Analysis options:

\begin{itemize}
\itemsep1pt\parskip0pt\parsep0pt
\item
  Multifactor ANOVA or mixed effects model. The appropriate approach
  will depend on whether all factors are fixed or whether the time
  factors represent random effects (session dates probably do).
\item
  Log rank tests to examine differences between deployment method.
\end{itemize}

Response variable = number of baits consumed by each species / number of
baits deployed or number of trials (i.e. the former takes into account
bait replenishment during ground baiting, the latter compares the
deployment methods).

~

\textbf{Q3: Are there environmental factors (e.g. weather, proximity to
agriculture, landscape position) or biological factors (e.g.
competitors, food availability and breeding seasons) that relate to
changes in Eradicat}\textsuperscript{®}\textbf{bait consumption by i)
target and ii) non-target species?}

Analysis:

\begin{itemize}
\itemsep1pt\parskip0pt\parsep0pt
\item
  Potential covariates include landscape attributes, site attributes,
  animal attributes, detectability of the bait, bait condition, and
  environmental attributes - see Appendix 4 for details
\item
  Multivariate analyses would be needed to relate differences in bait
  consumption or `survival' to other factors such as time of year,
  spatial factors, bait uptake by other species, environmental and
  biological factors, etc.
\item
  With only 40 sites the number of covariates in the analysis will be
  limited (e.g. \textasciitilde{}4-6 covariates). Therefore a process of
  culling covariates will be necessary (e.g. autocorrelation, step-wise
  approaches, etc)
\item
  The use of the information theoretic approach to model selection may
  be appropriate with this data.
\end{itemize}

\textbf{~}

\textbf{General theme: Nature of
Eradicat}\textsuperscript{®}\textbf{bait consumption}

\textbf{~}

\textbf{Q4: How long do Eradicat}\textsuperscript{®}\textbf{baits last
until they are consumed or removed (i.e. bait longevity)?}

Analysis:

\begin{itemize}
\itemsep1pt\parskip0pt\parsep0pt
\item
  Bait longevity can be described as a range (min-max), mean (+SE)
  and/or median number of days until 50\%, 95\% or 100\% of baits were
  removed
\item
  Plot of mean number of baits (+SE) remaining per day
\item
  If there is a high level of variance in the mean number of baits
  remaining per day and/or the time taken for 95-100\% of baits to be
  removed, then further analyses could seek to account for these
  differences -- e.g. landscape position, subregion, season, etc\ldots{}
\item
  Bait survival or Bait consumption by a particular species (i.e. the
  primary response variable) can be measured using a survival analysis
  approach (i.e. consumption by other species and camera failures are
  censored). Pollock's staggered entry model or Kaplan-Meir survival
  analysis approaches may be appropriate.
\end{itemize}

~

\textbf{General theme: Animal behaviour in relation to baits (i.e. to
understand how to improve bait effectiveness)}

\textbf{~}

\textbf{Q5: How do target and non-target species interact with the
baits? Specifically, Do targets and non-targets detect baits? And. once
a bait has been detected, do targets and non-targets consume baits?}

Once a bait has been detected, do targets and non-targets consume baits?

This will focus on information that provides some indications how to
improve target and non-target consumption rates.

General descriptive metrics will include;

\begin{itemize}
\itemsep1pt\parskip0pt\parsep0pt
\item
  Absolute number of baits consumed by each species
\item
  Absolute number of individuals of focal species (e.g. cat, fox,
  chuditch) that consumed at least one bait
\item
  Proportion of individuals of focal species detected that consumed at
  least one bait
\end{itemize}

Detailed metrics for focal species will include;

\begin{itemize}
\itemsep1pt\parskip0pt\parsep0pt
\item
  Detection rate = \# bait detection events / \# events animals are
  within close proximity of the bait (i.e. within field of view of the
  camera)
\item
  Investigation (approach) rate = \# baits investigated/ \# bait
  detection events
\item
  Contact rate = \# baits contacted (i.e. close investigation or
  physical contact)/ \# bait detection events
\item
  Consumption rate = \# consumed/\# bait detection events
\end{itemize}

Other possible metrics include;

\begin{itemize}
\itemsep1pt\parskip0pt\parsep0pt
\item
  Frequency or proportion of partial v whole consumption (i.e. what is
  the risk of sublethal bait consumption by target species?)
\end{itemize}

\begin{itemize}
\itemsep1pt\parskip0pt\parsep0pt
\item
  Frequency or proportion of baits consumed in situ or removed (i.e. are
  removed baits likely to be consumed or cached elsewhere; is there a
  risk that removed baits remain available)
\item
  Were solitary or multiple individuals involved (i.e. potential for
  learning, sub-lethal doses, etc)?
\end{itemize}

~

\textbf{General theme: Potential impact to individuals and populations
of Eradicat}\textsuperscript{®}\textbf{baiting on target and non-target
species}

\textbf{~}

\textbf{Q6: Do individuals consume multiple baits in a day or within a
baiting session?}

i.e. what is the possible risk to vulnerable non-targets?

\begin{itemize}
\itemsep1pt\parskip0pt\parsep0pt
\item
  Absolute number and proportion of detected individuals of species
  \emph{x} that consumed multiple baits.
\end{itemize}

\textbf{~}

\textbf{Q7: What is the potential and relative impact of
Eradicat}\textsuperscript{®}\textbf{baits on individuals and populations
of feral cat, fox and potentially vulnerable non-target species?}

i.e. how many and what proportion of individuals might have been killed
by toxic baits? Does this change over space and time? Are there any
patterns in the types of individuals that consumed the baits (e.g. size,
age, gender)? Is this number or proportion of individuals that consumed
baits likely to have a significant impact at the population level?

\textbf{~}

\textbf{General theme: Recommended
Eradicat}\textsuperscript{®}\textbf{baiting protocol to maximise
efficacy}

\textbf{~}

\textbf{Q8: When, how and where is it best to conduct
Eradicat}\textsuperscript{®}\textbf{baiting to maximise bait consumption
by feral cats and minimise the risk to non-target species?}

I.e. a synthesis of the results from the previous questions and an
understanding of the biology and ecology of the target and non-target
species to make an assessment of when, how and where
Eradicat\textsuperscript{®} bating is likely to be most
efficient/effective.

Also consider what needs to be done to be informed how to make further
improvements on the timing and location of the baits.

~

\subsection{Other considerations}

\textbf{Low bait consumption rates by cats is a major risk for the bait
uptake trials}

\begin{itemize}
\itemsep1pt\parskip0pt\parsep0pt
\item
  Low detection rates of cats and even lower bait uptake rates by cats
  is expected. A major challenge of this study will be to get sufficient
  data on cat interactions with baits to satisfactorily inform when, how
  and where to most effectively use Eradicat\textsuperscript{®} to
  control cats and minimise risks to non-targets. Cat and fox densities
  are estimated to be about 1 each per 500 ha -- 2000 ha (A. Wayne
  unpublished data). Therefore bait uptake is likely to be \textless{}1
  cat and \textless{}1 fox per trial. This means that to get a routine
  and operationally realistic appreciation of when baits might be most
  effective to target cats, it is likely to require a lot of replicate
  trials to get sufficient data.
\item
  It would be possible to estimate how many trials might be required but
  the estimates would be very rough, particularly since bait consumption
  rates are dependent on the probability a cat or fox comes within close
  proximity with a bait and guesswork as to what proportion might then
  actually detect, investigate and consume the bait.
\item
  Conversely, based on the proposed experimental design with 40 trials,
  it might be possible to estimate what consumption rate of baits might
  be required to get sufficient data to identify significant differences
  in bait consumption by cats in relation to when, how and where the
  baits are deployed.
\item
  Having trials of the ground baiting protocols are likely to be very
  important in getting enough data on target species interactions with
  baits because; 1) introduced predators are detected more frequently on
  track than off track (Wayne et al. 2013) and replenishment of the
  ground-deployed baits during the trials increases the opportunities
  for cats to encounter baits
\item
  Data from the uptake trials will be regularly analysed to ensure
  sufficient data is being generated by the trials. If there is
  insufficient data then an increase in the number of concurrent trials
  may be required (e.g. greater than 4 concurrent trials across
  additional sites, requiring additional cameras and/or the period of
  the trials may need to be extended).
\item
  Note that under the operational conditions of an aerial baiting event
  with introduced predator densities at these levels, 1 bait consumed by
  a cat (or fox) for every 250 -- 1000 baits deployed at 50
  baits/km\textsuperscript{2} (i.e. within a 500 ha -- 2000 ha area)
  would result in a 100\% target removal - this is very low (i.e. 99.2\%
  - 99.8\% non-target uptake of baits could indicate a successful
  baiting program for both cats and foxes).
\item
  Need to think what else could be done to improve detection and bait
  consumption rates -- by cats (i.e. more data) e.g. target areas with
  high cat densities.
\end{itemize}

\textbf{Repeat trials and confounding between learning by animals and
time of year}

Repeat trials on the same sites runs the risk of animals learning and
changing their behaviour in relation to the baits (e.g. increased bait
attraction or aversion), which is confounded with time of year. This
could have a major effect on the study being able to identify when the
best time to bait is (e.g. increased non-target uptake by possums or
corvids could substantially reduce bait availability to targets over
time). Therefore, the number of repeat trials should be minimised and
the intervals between repeat trials should be maximised or avoided
altogether if possible.

Note: The extent of learning within and between trials might be possible
to quantify to some extent if individuals can be confidently identified.
But this will be difficult for some problem species such as brush tail
possums and corvids.

~

\textbf{Toxic or non-toxic baits?}

These trials could be done with toxic or non-toxic
Eradicat\textsuperscript{®} baits. It is recommended that toxic baits
are used.

~

Using toxic baits:

Advantages

\begin{itemize}
\itemsep1pt\parskip0pt\parsep0pt
\item
  Trials closer to real operational conditions
\item
  Some cat removal may be a direct outcome of the trial (but it will
  make negligible difference at a population level given the density of
  trial sites relative to cat densities)
\end{itemize}

Disadvantages

\begin{itemize}
\itemsep1pt\parskip0pt\parsep0pt
\item
  Necessitates greater/complete spatial independence between sites (i.e.
  \textless{}10 km separation) the consequences of which include a
  larger study area, higher logistic costs (e.g. vehicle running and
  travel time), greater variance in the results due to greater variation
  between sites over a larger area.
\item
  There is some risk to non-target native individuals (e.g. chuditch,
  phascogale, etc) but given the large spatial separation between sites
  there should be negligible impact at the population level)
\item
  There is some risk to domestic cats and dogs that visit the trial
  sites and nearby if some baits are relocated by wildlife (a risk
  assessment would be required as part of the baiting approval process)
\item
  It will complicate the stage 2 comparative operational trials by
  potentially removing some cats across all treatments and potentially
  having some surviving individuals with a prior experience with
  Eradicat\textsuperscript{®} baits, albeit at very low densities (i.e.
  50 baits per 100 km\textsuperscript{2}). Therefore the first
  application of Eradicat\textsuperscript{®} at an operational scale
  will not be to an entirely naïve population (i.e. the scale of impact
  of the initial operational deployment of Eradicat\textsuperscript{®}
  baits may be very slightly less than what might have been achieved had
  the population been absolutely naïve).
\end{itemize}

Using non-toxic baits:

Advantages

\begin{itemize}
\itemsep1pt\parskip0pt\parsep0pt
\item
  No risk to non-target individuals
\item
  Spatial independence between trial sites is not as critical (because
  cats from earlier trials will not have been removed from the
  opportunity to encounter baits in subsequent trials nearby)
\item
  The bait uptake trials will not influence cat densities for the
  subsequent Stage 2 comparative trials of the effectiveness of an
  operational-scale baiting program
\end{itemize}

Disadvantages

\begin{itemize}
\itemsep1pt\parskip0pt\parsep0pt
\item
  Not identical to operational baits, therefore some additional
  assumptions as to how the results relate to operational conditions
  would need to be made
\item
  No removal of any cats
\item
  Uptake rates of both targets and non-targets (because of multiple
  consumptions by the same individual) may be greater than with toxic
  baits. This might be partially accountable if individuals are
  distinguishable but this may not be possible for some species such as
  koomal.
\end{itemize}

~

\textbf{Camera security}

Camera management and location will be important for security from theft
and damage (e.g. from fire)

\begin{itemize}
\itemsep1pt\parskip0pt\parsep0pt
\item
  Covert as possible for reducing detectability (also beneficial to the
  study if it reduces detectability by fauna -- i.e. reduced observer
  effects) -- based on location (next to/within/under natural vegetation
  and structures), camouflage (shape, shine, silhouette, surface,
  shadow, smell), etc
\item
  Aerial deployment trials should be relatively low risk - located
  \textgreater{}100 m from any trafficable vehicle tracks and
  \textgreater{}200 m from frequently used public roads.
\item
  Vehicle access to tracks near the cameras should be temporarily closed
  during the trials wherever possible. This will also reduce
  human-related risks of using toxic baits. Departmentally-managed roads
  can be closed under existing authority provisions available to the
  department. The roads could be closed with signage, danger tape,
  official 1080 signs and potentially have a covert camera for incursion
  surveillance. Signs can be produced cheaply on corflute boards and
  could be used for other departmental activities such as pig control.
  The text could state something like,
\end{itemize}

~

~

\begin{longtable}[c]{@{}l@{}}
\toprule\addlinespace
\begin{minipage}[t]{0.97\columnwidth}\raggedright
\textbf{DANGER - DO NOT ENTER.}

Road Temporarily Closed. \textbf{No unauthorised access}\textbf{.}

Pest control operations for conservation currently underway.

\textbf{1080 Poison baits \& firearms may be in use}\textbf{.}

For more information contact the local Parks and Wildlife office Ph:
97761207

Planned date of completion: \_\_\_ / \_\_\_ / \_\_\_\_\_\_~~~~~~~~~~~~
\end{minipage}
\\\addlinespace
\bottomrule
\end{longtable}

~

~

~

~

~

~

~

~

~

~

~

~

\begin{itemize}
\itemsep1pt\parskip0pt\parsep0pt
\item
  All cameras to be code locked activated, cameras labelled indicating
  security code protected and engraved
\item
  No flagging tape or any other visible indications of the location of
  the site or camera/bait stations
\end{itemize}

\begin{itemize}
\itemsep1pt\parskip0pt\parsep0pt
\item
  Consider additional dedicated surveillance cameras to record human
  visits to trial sites
\item
  Consider insurance policy in case of theft.
\item
  Avoid areas imminently planned to be burned during the trial period.
  Have all sites registered with local district operations and fire
  management staff so they are aware of their location and can notify
  the project in case of imminent wildfire risks and other disturbance
  activities. Having the forest tracks closed with signage will also
  make workers in the area aware and thereby reduce the risk of damage
  due to planned burns or wildfire.
\end{itemize}

\section{1c) Bait condition trials in relation to Coopex treatment and
environmental conditions}

\textbf{Subject to additional funds being secured}: experimental bait
condition trials may be conducted concurrently to the bait uptake trials
to;

\begin{enumerate}
\itemsep1pt\parskip0pt\parsep0pt
\item
  Determine the effectiveness of different Coopex application methods to
  baits aimed at reducing invertebrate impact on
  Eradicat\textsuperscript{®} bait characteristics
\item
  Determine how environmental conditions influence
  Eradicat\textsuperscript{®} bait characteristics, including
  spatio-temporal factors, the weather and ambient conditions,
  invertebrate interference, etc.
\end{enumerate}

~

Note: This would make an excellent honours or masters student project;
Year 1 -- data and sample collection (with or without the student), Year
2 - sample analysis and write up.

~

\subsubsection{Testing the effectiveness of Coopex applications}

Design:

3 bait treatments -- sprayed, dipped, no Coopex

3 replicates of each treatment per site

5 sites per trial - each of the 4 bait uptake trial sites and a
reference site (i.e. the 4 bait uptake trial sites will vary between
trials (total of 40 sites) and the reference site (in jarrah forest near
Manjimup) will remain the same throughout the successive trials (i.e. 41
sites in total).

10 trials conducted over a year

~

Method:

\begin{itemize}
\itemsep1pt\parskip0pt\parsep0pt
\item
  The bait condition trial sites will be 1-3 km from the bait uptake
  trial sites for independence (i.e. to reduce the influence of one
  trial on the results of the other). Note that independence will not be
  absolute (because some vertebrates may encounter both trials) but
  environmental conditions should be comparable between the paired
  trials sites.
\item
  Experimental unit / Bait station = 1 Eradicat\textsuperscript{®} bait
  placed in contact with the ground within a 5-sided wire cage cube
  securely pegged to the ground (i.e. to allow invertebrate access but
  not vertebrates. Note the size of the cube and the mesh needs to
  overcome interference from possums and canids that might dig).
\item
  Each bait station will be at least 100 m from other bait stations
  (based on estimated Australian meat ant (\emph{Iridomyrmex purureus})
  territory sizes being \textless{}5900 m\textsuperscript{2} (Thomas
  \emph{et al.} 1999; Note that \emph{Iridomyrmex conifer} is likely to
  be the most problematic species in the forest)
\item
  The trials will be conducted concurrently to the bait up-take trials
  and run for two weeks.
\item
  Bait condition (photographed, weighed, visual assessment), soil
  moisture and invertebrate activity will be recorded at the beginning
  or the trials, at least once during the trials (day 4-10; for the
  trials associated with the ground baiting and reference sites only)
  and at the end of the trial (day 14 for aerial deployment trials). At
  the end of the trials the baits will be retained and frozen for later,
  more detailed, assessment if required.
\item
  Environmental data to be collected at each trial site will include
  ambient conditions (temperature, relative humidity, rainfall, and soil
  moisture by means of data loggers and rainfall gauges; evaporation
  rates).
\item
  Invertebrate species will be sampled i) that are found on the baits
  when they are checked during the trials, and ii) by means of a pair of
  delta traps with replaceable sticky bases (one in each pair being
  baited with 4 cm\textsuperscript{3} piece of non-toxic
  Eradicat\textsuperscript{®} bait), located \textasciitilde{}10 m from
  each Eradicat\textsuperscript{®} bait station.
\end{itemize}

~

Note: Being able to explain the factors affecting the rates and
differences in the degradation of Eradicat\textsuperscript{®} baits is
important to being able to understand the scale of the problem and to
identify solutions, and when and where they may be needed.

~

\section{1d) Spatial ecology of cats}

\textbf{Subject to additional funds being secured}. Summary:
Radio-telemetry using GPS collars on up to 10 cats will be used to get
spatio-temporal data within the Upper Warren region to provide a general
understanding of;

\begin{itemize}
\itemsep1pt\parskip0pt\parsep0pt
\item
  home range size, area covered per night and other information on
  movement patterns (e.g. useful to help estimate density, probability
  of encounter of potential bait locations, etc)
\item
  how they use the landscape (e.g. habitat preferences/selection --
  landscape position, proximity to agriculture, use of roads and tracks,
  etc).
\item
  try to identify when/where cats are more likely to eat baits, such as
  feeding areas as distinct from other activities such as shelter,
  territorial maintenance, breeding, social interaction, etc.
\item
  This work is currently unfunded except for the procurement of the GPS
  collars.
\end{itemize}

Cat selection:

Cats for collaring would preferentially be selected in the area(s)
within the Upper Warren region anticipated to be involved in Stage 2 of
the project -- i.e. the treatment and comparative reference areas for
the operational-scale deployment of baits. This is to maximise
efficiency and information return from investment (e.g. the same cats
and collars could be subsequently used to track comparative survivorship
of cats during operational baiting).

\begin{itemize}
\itemsep1pt\parskip0pt\parsep0pt
\item
  Sources: farm shed strays/ferals from properties within or adjacent to
  expansive forest areas may be easier targets, but at least some
  individuals should be sourced from forest areas
\item
  Capture methods will include cage and leg holds and opportunistic
  captures from wildlife monitoring. Catching cats is the greatest
  challenge for this project component. Assistance and support from
  expertise in cat trapping will be critical to its success.
\item
  Landholder engagement and support will be important in finding source
  animals (i.e. an opportunity to integrate this with the community
  elements of this project).
\item
  Reconnaissance using cat sightings, sign and remote sensor cameras
  will be needed to identify areas and individuals to target for capture
  and collaring
\item
  Given the small sample size it may be difficult to derive clear gender
  differences in behaviour and movement. Consider only collaring
  individuals of one gender (Sarah Comer pers. comm.)
\end{itemize}

GPS Collars:

\begin{itemize}
\itemsep1pt\parskip0pt\parsep0pt
\item
  About 20 fixes per day (hourly between 1600 and 0800 hrs and at 1000,
  1200 and 1400 hrs)
\item
  Approximate battery life of 150 days per collar
\item
  Remote download every month
\end{itemize}

\section{~}

\section{Complementary approaches that could help identify when, how and
where to bait for effective cat control for fauna conservation}

\begin{itemize}
\itemsep1pt\parskip0pt\parsep0pt
\item
  Biologically informed -- cat biology/ecology, native prey
  biology/ecology, predator prey interactions, spatial ecology, etc e.g.
  risk to chuditch may be reduced if the timing of baiting avoids the
  period when young emerge from the den and start feeding themselves
\item
  Modelling
\end{itemize}

\section{Stage 2 - Operational deployment of
Eradicat\textsuperscript{®}}

Specific objectives: Quantify the effects of an optimised operational
use of Eradicat\textsuperscript{®} baits (informed by stage 1) at the
individual and population levels of i) feral cats and foxes, ii)
potentially vulnerable non-target native species (e.g. dasyurids), and
iii) priority native species threatened by introduced predators
(Numbats, Woylies, etc) within the Upper Warren region.

Strategy: Conduct an Eradicat\textsuperscript{®} baiting program within
a limited area of the Upper Warren (e.g. 225-400 km\textsuperscript{2}),
within a comparative / experimental framework, to assess the
effectiveness on target species (cats and foxes) and effects on
non-target species (with a focus on potentially vulnerable and priority
beneficiaries).

~

\begin{itemize}
\itemsep1pt\parskip0pt\parsep0pt
\item
  Comparative treatments will include; 1) Eradicat\textsuperscript{®}
  baiting treatment area, 2) Standard quarterly fox baiting area, 3)
  Monthly fox baiting area (Perup core, since 2010), 4) An unbaited
  (Probait and Eradicat\textsuperscript{®}) area
\item
  The effectiveness of the Eradicat\textsuperscript{®} baiting event
  will be examined using a BACI design (before, after, control, impact);
  considered the most powerful means of making this assessment
\item
  Responses of target and priority non-target fauna will be measured at
  the individual and population levels because each level provides
  important insights that the other cannot.
\item
  Monitoring methods before and after the Eradicat\textsuperscript{®}
  baiting will include (Table 2); - Radio-telemetry (target, and
  possibly vulnerable non-target species if additional funds can be
  secured - Remote sensor cameras (all species, especially cats, foxes,
  numbats, quenda) - Trapping (especially woylies also other native
  mammals) - Spotlighting (especially western ringtail possums, also
  phascogales and macropods) - Nestboxes (phascogales)
\item
  The monitoring program will use and build on the existing monitoring
  infrastructure in the Upper Warren to provide best use of historical
  data and minimise costs of monitoring during this project (Figure 2).
  New monitoring infrastructure will include 3 nestbox grids, and a
  trapping and spotlight transect in Talling forest block in the
  Eradicat\textsuperscript{®} baiting area (southern Perup) (Table 3).
\item
  Potentially vulnerable non-target native species in the Upper Warren
  are presented in Table 4. Wambenger (brush-tailed phascogale), bush
  rat, chuditch and quenda are considered potentially the most
  vulnerable mammals based on a theoretical worst-case scenario
  assessment. All four species need to consume less than one or two
  Eradicat\textsuperscript{®} baits to potentially receive a lethal dose
  of 1080. Relative to their body mass all four are capable of consuming
  their respective quantities of bait within 24 hrs (i.e. \textless{}9\%
  of their body mass). Consumption rates of dried meat baits (DMB) of
  \textless{}9\% body mass have been repeatedly observed being consumed
  by dasyurids in laboratory trials (Martin et al. 2002 and references
  within). Because the amount of bait consumed is inversely proportional
  to the hardness of the food, it is reasonable to assume that these
  species can eat more Eradicat\textsuperscript{®} bait (which is
  softer) than DMBs. Most small dasyurids consume about 20-30\% of their
  body mass each day (see Martin et al. 2002).
\item
  Bush rats will not be assessed as part of this study because none have
  been detected in the Upper Warren region since 2005, having been
  abundant in places in the 1970s (Per Christensen pers comm.) but
  having declined since at least 1994 (Wayne et al. in review).
\item
  The varanid lizards (\emph{Varanus gouldii}and \emph{V. rosenbergi})
  and some birds (e.g. Australian raven, Australia magpie, some raptors)
  may also be potentially vulnerable (Table 4), however there is
  currently insufficient information to more accurately quantify the
  risks. These species will not be monitored, given the limited
  resources available. While some traps captures of \emph{Varanus
  rosenbergi}are expected during other planned monitoring activities it
  is not likely that there will be sufficient data to assess population
  changes without a greater, more targeted monitoring program.
\item
  The timing of the operational use of Eradicat\textsuperscript{®} will
  be determined by Stage 1. This in turn will affect the timing of
  monitoring before and after the baiting operation (Table 5).
\end{itemize}

~

\textbf{Design and analysis}

BACI design with ANOVA

Response variables:

Individual level -- survivorship of radio-collared (cats, ideally also
foxes, chuditch and phascogales) or trapped animals (chuditch)

Population level -- detection rate (Remote sensor camera, spotlight),
abundance estimate (trapping) or occupancy rate (nest boxes) of each
species

~~

\textbf{Figure 2.}Map of the Upper Warren region with existing
monitoring infrastructure and indicative treatment areas for the
comparative operational deployment of Eradicat\textsuperscript{®} baits

~

\textbf{Table 2.} Monitoring methods to be used for species of interest
before and after the Stage 2 operational Eradicat\textsuperscript{®}
baiting. 1= most appropriate, 2=reasonable method, 3 = possibly useful.

~

~

\textbf{Individual}

\textbf{Population responses}

\textbf{Category}

\textbf{Species}

\textbf{Radio-telemetry}

\textbf{RS Cameras}

\textbf{Cage Trap}

\textbf{Spot-light}

\textbf{Nest-box}

Target

Cat

1

1

-

3

-

~

Fox

Ideally

1

-

3

-

Vulnerable natives

Wambenger

Ideally

3

-

3

1

~

Bush rat

-

3

1

-

-

~

Chuditch

Ideally

2

1

-

-

~

Quenda

Ideally

1

2

3

-

~

Varanus

Ideally

3

1

-

-

Priority natives

Woylie

-

2

1

3

-

~

Ngwayir

-

3

-

1

-

~

Numbat

-

1

-

-

-

Other native beneficiaries

e.g. Brushtail possum

-

2

1

2-3

-

Non-native beneficiaries

e.g. rabbits

-

1

-

2-3

-

\textbf{~}

\textbf{~}

\textbf{~}

\textbf{Table 3.} Monitoring infrastructure to be used for species of
interest before and after the Stage 2 operational
Eradicat\textsuperscript{®} baiting.

~

\begin{longtable}[c]{@{}lllllll@{}}
\toprule\addlinespace
\begin{minipage}[t]{0.12\columnwidth}\raggedright
\textbf{Treatment area}
\end{minipage} & \begin{minipage}[t]{0.12\columnwidth}\raggedright
\textbf{Forest block}
\end{minipage} & \begin{minipage}[t]{0.12\columnwidth}\raggedright
\textbf{RS Cameras}
\end{minipage} & \begin{minipage}[t]{0.12\columnwidth}\raggedright
\textbf{Cage Trap}
\end{minipage} & \begin{minipage}[t]{0.12\columnwidth}\raggedright
\textbf{Spotlight}
\end{minipage} & \begin{minipage}[t]{0.12\columnwidth}\raggedright
\textbf{Nest-box grid}
\end{minipage} & \begin{minipage}[t]{0.12\columnwidth}\raggedright
\textbf{Nest-box transect}
\end{minipage}
\\\addlinespace
\begin{minipage}[t]{0.12\columnwidth}\raggedright
Perup Core
\end{minipage} & \begin{minipage}[t]{0.12\columnwidth}\raggedright
Moopinup
\end{minipage} & \begin{minipage}[t]{0.12\columnwidth}\raggedright
~
\end{minipage} & \begin{minipage}[t]{0.12\columnwidth}\raggedright
Existing
\end{minipage} & \begin{minipage}[t]{0.12\columnwidth}\raggedright
Existing
\end{minipage} & \begin{minipage}[t]{0.12\columnwidth}\raggedright
Existing - Stretch Rd
\end{minipage} & \begin{minipage}[t]{0.12\columnwidth}\raggedright
Existing
\end{minipage}
\\\addlinespace
\begin{minipage}[t]{0.12\columnwidth}\raggedright
~
\end{minipage} & \begin{minipage}[t]{0.12\columnwidth}\raggedright
Balban
\end{minipage} & \begin{minipage}[t]{0.12\columnwidth}\raggedright
Existing
\end{minipage} & \begin{minipage}[t]{0.12\columnwidth}\raggedright
Existing
\end{minipage} & \begin{minipage}[t]{0.12\columnwidth}\raggedright
Existing
\end{minipage} & \begin{minipage}[t]{0.12\columnwidth}\raggedright
New
\end{minipage} & \begin{minipage}[t]{0.12\columnwidth}\raggedright
~
\end{minipage}
\\\addlinespace
\begin{minipage}[t]{0.12\columnwidth}\raggedright
Std. W/S
\end{minipage} & \begin{minipage}[t]{0.12\columnwidth}\raggedright
Warrup
\end{minipage} & \begin{minipage}[t]{0.12\columnwidth}\raggedright
~
\end{minipage} & \begin{minipage}[t]{0.12\columnwidth}\raggedright
Existing
\end{minipage} & \begin{minipage}[t]{0.12\columnwidth}\raggedright
Existing
\end{minipage} & \begin{minipage}[t]{0.12\columnwidth}\raggedright
~
\end{minipage} & \begin{minipage}[t]{0.12\columnwidth}\raggedright
Existing
\end{minipage}
\\\addlinespace
\begin{minipage}[t]{0.12\columnwidth}\raggedright
~
\end{minipage} & \begin{minipage}[t]{0.12\columnwidth}\raggedright
Winnejup
\end{minipage} & \begin{minipage}[t]{0.12\columnwidth}\raggedright
~
\end{minipage} & \begin{minipage}[t]{0.12\columnwidth}\raggedright
Existing
\end{minipage} & \begin{minipage}[t]{0.12\columnwidth}\raggedright
~
\end{minipage} & \begin{minipage}[t]{0.12\columnwidth}\raggedright
~
\end{minipage} & \begin{minipage}[t]{0.12\columnwidth}\raggedright
~
\end{minipage}
\\\addlinespace
\begin{minipage}[t]{0.12\columnwidth}\raggedright
~
\end{minipage} & \begin{minipage}[t]{0.12\columnwidth}\raggedright
Kingston
\end{minipage} & \begin{minipage}[t]{0.12\columnwidth}\raggedright
~
\end{minipage} & \begin{minipage}[t]{0.12\columnwidth}\raggedright
~
\end{minipage} & \begin{minipage}[t]{0.12\columnwidth}\raggedright
Existing
\end{minipage} & \begin{minipage}[t]{0.12\columnwidth}\raggedright
2 x Existing
\end{minipage} & \begin{minipage}[t]{0.12\columnwidth}\raggedright
~
\end{minipage}
\\\addlinespace
\begin{minipage}[t]{0.12\columnwidth}\raggedright
~
\end{minipage} & \begin{minipage}[t]{0.12\columnwidth}\raggedright
Warrup East
\end{minipage} & \begin{minipage}[t]{0.12\columnwidth}\raggedright
Existing
\end{minipage} & \begin{minipage}[t]{0.12\columnwidth}\raggedright
~
\end{minipage} & \begin{minipage}[t]{0.12\columnwidth}\raggedright
~
\end{minipage} & \begin{minipage}[t]{0.12\columnwidth}\raggedright
~
\end{minipage} & \begin{minipage}[t]{0.12\columnwidth}\raggedright
~
\end{minipage}
\\\addlinespace
\begin{minipage}[t]{0.12\columnwidth}\raggedright
Eradicat\textsuperscript{®}
\end{minipage} & \begin{minipage}[t]{0.12\columnwidth}\raggedright
Boyicup
\end{minipage} & \begin{minipage}[t]{0.12\columnwidth}\raggedright
Existing
\end{minipage} & \begin{minipage}[t]{0.12\columnwidth}\raggedright
Existing
\end{minipage} & \begin{minipage}[t]{0.12\columnwidth}\raggedright
Existing
\end{minipage} & \begin{minipage}[t]{0.12\columnwidth}\raggedright
New
\end{minipage} & \begin{minipage}[t]{0.12\columnwidth}\raggedright
Existing
\end{minipage}
\\\addlinespace
\begin{minipage}[t]{0.12\columnwidth}\raggedright
~
\end{minipage} & \begin{minipage}[t]{0.12\columnwidth}\raggedright
Talling
\end{minipage} & \begin{minipage}[t]{0.12\columnwidth}\raggedright
~
\end{minipage} & \begin{minipage}[t]{0.12\columnwidth}\raggedright
New
\end{minipage} & \begin{minipage}[t]{0.12\columnwidth}\raggedright
New
\end{minipage} & \begin{minipage}[t]{0.12\columnwidth}\raggedright
New
\end{minipage} & \begin{minipage}[t]{0.12\columnwidth}\raggedright
~
\end{minipage}
\\\addlinespace
\bottomrule
\end{longtable}

~

\textbf{Table 4.}A worst case scenario theoretical assessment of the
potentially vulnerable non-target native animals in the Southern Jarrah
Forest based on published estimates of approximate lethal dose (ALD) for
1080 poison and the minimum recorded adult body masses. The degree of
vulnerability is expressed in relation to the minimum number of
Eradicat\textsuperscript{®} and Probait baits required to consume an ALD
(assuming 1080 is evenly distributed throughout the bait) and in
relation to the amount of bait consumed proportionate to body mass. i.e.
species that need to consume less than 1 or 2 baits and/or
\textless{}20\% of their adult body mass to receive an ALD are
considered vulnerable. No information on the Australian raven or
Australian magpie tolerances to 1080 could be found -- the Little crow
is presented as a surrogate.

~

* Source: Martin et al. (2002); \^{} Source: van Dyck and Strahan
(2008); \textsuperscript{\#} ALD unknown, LD\textsubscript{50} = 50, 235
and 12.8 mg 1080 kg\textsuperscript{-1} for \emph{V. gouldii,} \emph{V.
rosenbergi}and Little crow \emph{(Corvus bennetti)} respectively, (Dept.
of Agriculture 2002); adult body mass estimates are also rough (for
Varanus 500g is used as an interim minimum adult body mass, for the
Little crow 380 g is the mean adult body mass)

~

\begin{longtable}[c]{@{}lllllll@{}}
\toprule\addlinespace
\begin{minipage}[t]{0.12\columnwidth}\raggedright
\textbf{Species}
\end{minipage} & \begin{minipage}[t]{0.12\columnwidth}\raggedright
\textbf{*Approx. Lethal Dose (mg 1080 kg\textsuperscript{-1)}}
\end{minipage} & \begin{minipage}[t]{0.12\columnwidth}\raggedright
\textbf{\^{}Adult Body mass (g)}~
\end{minipage} & \begin{minipage}[t]{0.12\columnwidth}\raggedright
\textbf{Approx min. \# of Eradicat}\textsuperscript{®}\textbf{baits (4.5
mg 1080) for ALD}
\end{minipage} & \begin{minipage}[t]{0.12\columnwidth}\raggedright
\textbf{Amount of Eradicat}\textsuperscript{®}\textbf{bait consumed to
reach ALD proportionate to body mass}
\end{minipage} & \begin{minipage}[t]{0.12\columnwidth}\raggedright
\textbf{Approx min. \# Probaits (3 mg 1080) for ALD}
\end{minipage} & \begin{minipage}[t]{0.12\columnwidth}\raggedright
\textbf{Amount of Probait consumed to reach ALD proportionate to body
mass}
\end{minipage}
\\\addlinespace
\begin{minipage}[t]{0.12\columnwidth}\raggedright
Wambenger
\end{minipage} & \begin{minipage}[t]{0.12\columnwidth}\raggedright
6
\end{minipage} & \begin{minipage}[t]{0.12\columnwidth}\raggedright
106-212 F, 175-234 M
\end{minipage} & \begin{minipage}[t]{0.12\columnwidth}\raggedright
0.14
\end{minipage} & \begin{minipage}[t]{0.12\columnwidth}\raggedright
1.9\%
\end{minipage} & \begin{minipage}[t]{0.12\columnwidth}\raggedright
0.21
\end{minipage} & \begin{minipage}[t]{0.12\columnwidth}\raggedright
9.0\%
\end{minipage}
\\\addlinespace
\begin{minipage}[t]{0.12\columnwidth}\raggedright
Bush rat
\end{minipage} & \begin{minipage}[t]{0.12\columnwidth}\raggedright
27.6
\end{minipage} & \begin{minipage}[t]{0.12\columnwidth}\raggedright
40-225
\end{minipage} & \begin{minipage}[t]{0.12\columnwidth}\raggedright
0.25
\end{minipage} & \begin{minipage}[t]{0.12\columnwidth}\raggedright
8.9\%
\end{minipage} & \begin{minipage}[t]{0.12\columnwidth}\raggedright
0.37
\end{minipage} & \begin{minipage}[t]{0.12\columnwidth}\raggedright
41.4\%
\end{minipage}
\\\addlinespace
\begin{minipage}[t]{0.12\columnwidth}\raggedright
Chuditch
\end{minipage} & \begin{minipage}[t]{0.12\columnwidth}\raggedright
7.5
\end{minipage} & \begin{minipage}[t]{0.12\columnwidth}\raggedright
615-1130 F; 710-2185 M
\end{minipage} & \begin{minipage}[t]{0.12\columnwidth}\raggedright
1.03
\end{minipage} & \begin{minipage}[t]{0.12\columnwidth}\raggedright
2.4\%
\end{minipage} & \begin{minipage}[t]{0.12\columnwidth}\raggedright
1.54
\end{minipage} & \begin{minipage}[t]{0.12\columnwidth}\raggedright
11.3\%
\end{minipage}
\\\addlinespace
\begin{minipage}[t]{0.12\columnwidth}\raggedright
Little crow
\end{minipage} & \begin{minipage}[t]{0.12\columnwidth}\raggedright
\textsuperscript{\#}12.8
\end{minipage} & \begin{minipage}[t]{0.12\columnwidth}\raggedright
*380
\end{minipage} & \begin{minipage}[t]{0.12\columnwidth}\raggedright
1.08
\end{minipage} & \begin{minipage}[t]{0.12\columnwidth}\raggedright
4.1\%
\end{minipage} & \begin{minipage}[t]{0.12\columnwidth}\raggedright
1.62
\end{minipage} & \begin{minipage}[t]{0.12\columnwidth}\raggedright
19.2\%
\end{minipage}
\\\addlinespace
\begin{minipage}[t]{0.12\columnwidth}\raggedright
Quenda
\end{minipage} & \begin{minipage}[t]{0.12\columnwidth}\raggedright
14.1
\end{minipage} & \begin{minipage}[t]{0.12\columnwidth}\raggedright
400-1200 F, 500-1850 M
\end{minipage} & \begin{minipage}[t]{0.12\columnwidth}\raggedright
1.25
\end{minipage} & \begin{minipage}[t]{0.12\columnwidth}\raggedright
4.5\%
\end{minipage} & \begin{minipage}[t]{0.12\columnwidth}\raggedright
1.88
\end{minipage} & \begin{minipage}[t]{0.12\columnwidth}\raggedright
21.2\%
\end{minipage}
\\\addlinespace
\begin{minipage}[t]{0.12\columnwidth}\raggedright
Brown falcon
\end{minipage} & \begin{minipage}[t]{0.12\columnwidth}\raggedright
20
\end{minipage} & \begin{minipage}[t]{0.12\columnwidth}\raggedright
*400-480
\end{minipage} & \begin{minipage}[t]{0.12\columnwidth}\raggedright
1.81
\end{minipage} & \begin{minipage}[t]{0.12\columnwidth}\raggedright
6.6\%
\end{minipage} & \begin{minipage}[t]{0.12\columnwidth}\raggedright
2.72
\end{minipage} & \begin{minipage}[t]{0.12\columnwidth}\raggedright
30.6\%
\end{minipage}
\\\addlinespace
\begin{minipage}[t]{0.12\columnwidth}\raggedright
Varanus
\end{minipage} & \begin{minipage}[t]{0.12\columnwidth}\raggedright
\textsuperscript{\#}50
\end{minipage} & \begin{minipage}[t]{0.12\columnwidth}\raggedright
\textless{}1000 F \textless{}1900 M
\end{minipage} & \begin{minipage}[t]{0.12\columnwidth}\raggedright
5.56
\end{minipage} & \begin{minipage}[t]{0.12\columnwidth}\raggedright
16.1\%
\end{minipage} & \begin{minipage}[t]{0.12\columnwidth}\raggedright
8.33
\end{minipage} & \begin{minipage}[t]{0.12\columnwidth}\raggedright
75.0\%
\end{minipage}
\\\addlinespace
\begin{minipage}[t]{0.12\columnwidth}\raggedright
Woylie
\end{minipage} & \begin{minipage}[t]{0.12\columnwidth}\raggedright
105.8
\end{minipage} & \begin{minipage}[t]{0.12\columnwidth}\raggedright
750-1500 F, 980-1850 M
\end{minipage} & \begin{minipage}[t]{0.12\columnwidth}\raggedright
17.63
\end{minipage} & \begin{minipage}[t]{0.12\columnwidth}\raggedright
34.1\%
\end{minipage} & \begin{minipage}[t]{0.12\columnwidth}\raggedright
26.45
\end{minipage} & \begin{minipage}[t]{0.12\columnwidth}\raggedright
158.7\%
\end{minipage}
\\\addlinespace
\begin{minipage}[t]{0.12\columnwidth}\raggedright
Koomal
\end{minipage} & \begin{minipage}[t]{0.12\columnwidth}\raggedright
92
\end{minipage} & \begin{minipage}[t]{0.12\columnwidth}\raggedright
1200-3500 F, 1300-4500 M
\end{minipage} & \begin{minipage}[t]{0.12\columnwidth}\raggedright
24.53
\end{minipage} & \begin{minipage}[t]{0.12\columnwidth}\raggedright
29.6\%
\end{minipage} & \begin{minipage}[t]{0.12\columnwidth}\raggedright
36.80
\end{minipage} & \begin{minipage}[t]{0.12\columnwidth}\raggedright
138.0\%
\end{minipage}
\\\addlinespace
\bottomrule
\end{longtable}

~

~

\textbf{Table 5.} Indicative timeline for the Southern Jarrah Forest
component of the South West Threatened Fauna Recovery Project. Principle
responsibility Green = Science, Blue = RFMS, Orange = Western
Shield/RFMS

~

Jan-16

Feb-16

Mar-16

Apr-16

May-16

Jun-16

Jul-16

Aug-16

Sep-16

Oct-16

Nov-16

Dec-16

Jan-17

Feb-17

Mar-17

Apr-17

May-17

Jun-17

Jul-17

Aug-17

Sep-17

Oct-17

Nov-17

Dec-17

Jan-18

Feb-18

Mar-18

Apr-18

May-18

Jun-18

Jul-18

Aug-18

Sep-18

Oct-18

Nov-18

Dec-18

\textbf{Stage 1: Bait protocol optimisation}

~

~

~

~

~

~

~

~

~

~

~

~

~

~

~

~

~

~

~

~

~

~

~

~

~

~

~

~

~

~

~

~

~

~

~

~

Pilot trial

~

~

~

~

~

~

~

~

~

~

~

~

~

~

~

~

~

~

~

~

~

~

~

~

~

~

~

~

~

~

~

~

~

~

~

~

Site selection and set up

~

~

~

~

~

~

~

~

~

~

~

~

~

~

~

~

~

~

~

~

~

~

~

~

~

~

~

~

~

~

~

~

~

~

~

~

Bait uptake trials

~

~

~

~

~

~

~

~

~

~

~

~

~

~

~

~

~

~

~

~

~

~

~

~

~

~

~

~

~

~

~

~

~

~

~

~

~

Camera data management

~

~

~

~

~

~

~

~

~

~

~

~

~

~

~

~

~

~

~

~

~

~

~

~

~

~

~

~

~

~

~

~

~

~

~

~

~

Analysis

~

~

~

~

~

~

~

~

~

~

~

~

~

~

~

~

~

~

~

~

~

~

~

~

~

~

~

~

~

~

~

~

~

~

~

~

~

Cat spatial ecology

~

~

~

~

~

~

~

~

~

~

~

~

~

~

~

~

~

~

~

~

~

~

~

~

~

~

~

~

~

~

~

~

\textbf{Stage 2: Integrated fox and cat control}

~

~

~

~

~

~

~

~

~

~

~

~

~

~

~

~

~

~

~

~

~

~

~

~

~

~

~

~

~

~

~

~

~

~

~

~

Eradicat bait - 1 or 2 sessions per year, 50baits/km x 300 km2 = 15,000
baits

~

~

~

~

~

~

~

~

~

~

~

~

~

~

~

~

~

~

~

~

~

~

~

~

~

~

~

~

~

~

~

~

~

~

~

~

\textbf{Stage 2: Fauna Monitoring}

~

~

~

~

~

~

~

~

~

~

~

~

~

~

~

~

~

~

~

~

~

~

~

~

~

~

~

~

~

~

~

~

~

~

~

~

Camera monitoring (cats, numbats)

~

~

~

~

~

~

~

~

~

~

~

~

~

~

~

~

~

~

~

~

~

~

~

~

~

~

~

~

~

~

~

~

~

~

~

~

Radio-collaring cats

~

~

~

~

~

~

~

~

~

~

~

~

~

~

~

~

~

~

~

~

~

~

~

~

~

~

~

~

~

~

~

~

~

~

Radio-collaring chuditch

~

~

~

~

~

~

~

~

~

~

~

~

~

~

~

~

~

~

~

~

~

~

~

~

~

~

~

~

~

~

~

~

~

~

Radio-collar monitoring

~

~

~

~

~

~

~

~

~

~

~

~

~

~

~

~

~

~

~

~

~

~

~

~

~

~

~

~

~

~

~

~

~

~

~

~

Cage Trapping (chuditch, woylie, brushtail possum, quenda)

~

~

~

~

~

~

~

~

~

~

~

~

~

~

~

~

~

~

~

~

~

~

~

~

~

~

~

~

~

~

~

~

~

~

~

Spotlighting (western ringtail possums, tammar, western brush wallaby)

~

~

~

~

~

~

~

~

~

~

~

~

~

~

~

~

~

~

~

~

~

~

~

~

~

~

~

~

~

~

~

~

~

~

~

Nest boxes (phascogales)

~

~

~

~

~

~

~

~

~

~

~

~

~

~

~

~

~

~

~

~

~

~

~

~

~

~

~

~

~

~

~

~

~

~

~

~

Numbat monitoring

~

~

~

~

~

~

~

~

~

~

~

~

~

~

~

~

~

~

~

~

~

~

~

~

~

~

~

~

~

~

~

~

~

~

~

~

~

~

~

~

~

~

~

~

~

~

~

~

~

~

~

~

~

~

~

~

~

~

~

~

~

~

~

~

~

~

~

~

~

~

~

~

~

~

~

Note: the timing of the Operation Eradicat\textsuperscript{®} baiting in
Stage 2 is to be determined by the results of Stage 1. This in turn will
affect the timing of the fauna monitoring before, during and after the
baiting event

Figures provided for Varanus and Koomal are for larger adults than
observed in UW, resulting in an overestimation of the min \# baits
required to reach ALD.




\subsection*{Biometrician's Endorsement}

granted



\section*{Data management}


\subsection*{No. specimens}






\subsection*{Herbarium Curator's Endorsement}

not required




\subsection*{Animal Ethics Committee's Endorsement}

granted




\subsection*{Data management}

Hard copies of data will be kept on file in the Manjimup Work Centre. A
digital copy of the data will be maintained in Fauna File. Images will
be stored and managed in CPW Photo Warehouse.




\section*{Budget}

\section*{Consolidated Funds }



\begin{longtabu} to \linewidth { |  X | X | X | X | }
\hline
\rowcolor{infobg}
Source & Year 1 & Year 2 & Year 3\\
\hline
\endhead



FTE Scientist &  &  & \\



FTE Technical &  &  & \\



Equipment &  &  & \\



Vehicle &  &  & \\



Travel &  &  & \\



Other &  &  & \\



Total &  &  & \\


\hline
\end{longtabu}



\section*{External Funds }



\begin{longtabu} to \linewidth { |  X | X | X | X | }
\hline
\rowcolor{infobg}
Source & Year 1 & Year 2 & Year 3\\
\hline
\endhead



Salaries, Wages, Overtime &  &  & \\



Overheads &  &  & \\



Equipment &  &  & \\



Vehicle &  &  & \\



Travel &  &  & \\



Other &  &  & \\



Total &  &  & \\


\hline
\end{longtabu}





%-----------------------------------------------------------------------------%
% Back matter
%\backmatter
\end{document}
%-----------------------------------------------------------------------------%

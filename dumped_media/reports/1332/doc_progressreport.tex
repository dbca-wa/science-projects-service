
\documentclass[version=last, paper=a4, DIV=18, usenames, dvipsnames]{scrartcl}
\usepackage{txfonts}
\usepackage{pdflscape}
\usepackage{pdfpages}
\usepackage[english]{babel} % English language/hyphenation
%%% Bootstrap colors
\definecolor{RedFire}{RGB}{146,25,28}
\definecolor{ParksWildlife}{RGB}{0,85,144}
\definecolor{successbg}{RGB}{223,240,216}
\definecolor{errorbg}{RGB}{242,222,222}
\definecolor{warningbg}{RGB}{252,248,227}
\definecolor{infobg}{RGB}{217,237,247}
\definecolor{muted}{RGB}{153,153,153}
\definecolor{success}{RGB}{70,136,71}
\definecolor{error}{RGB}{185,74,72}
\definecolor{warning}{RGB}{192,152,83}
\definecolor{info}{RGB}{58,135,173}

\definecolor{required}{HTML}{D9534F}
\definecolor{denied}{HTML}{D9534F}
\definecolor{granted}{HTML}{47A447}
\definecolor{not required}{RGB}{200, 200, 200}

\usepackage[colorlinks=true,pdftitle=doc\_progressreport.pdf
,linktoc=all,linkcolor=RedFire,urlcolor=ParksWildlife]{hyperref}
\usepackage{colortbl}
\usepackage{longtable}
\usepackage{tabu}
\setlength{\tabulinesep}{1.5mm}
\usepackage{enumerate}
\usepackage{enumitem}
\usepackage{fancyhdr}
\usepackage{lastpage}
\usepackage{graphicx}
\usepackage{eso-pic}
\usepackage{hyphenat}
\renewcommand{\familydefault}{\sfdefault}



\newcommand{\HRule}{\rule{\linewidth}{0.1pt}}

\newcommand{\placetextbox}[3]{% \placetextbox{<horizontal pos>}{<vertical pos>}{<stuff>}
  \setbox0=\hbox{#3}% Put <stuff> in a box
  \AddToShipoutPictureFG*{% Add <stuff> to current page foreground
    \put(\LenToUnit{#1\paperwidth},\LenToUnit{#2\paperheight}){\vtop{{\null}\makebox[0pt][c]{#3}}}%
  }%
}%




%-----------------------------------------------------------------------------%
% Headers and footers
%
\fancypagestyle{plain}{
  \fancyhf{}
  \setlength\headheight{60pt} % push page content below header
  \renewcommand{\headrulewidth}{0.1pt}
  \renewcommand{\footrulewidth}{0.1pt}
  
  
  \fancyhead[L]{ 
    \href{http://sdis.dpaw.wa.gov.au}{
    \includegraphics[scale=0.6]{/mnt/projects/sdis/staticfiles/img/logo-dpaw.png}}
  }
  \fancyhead[R]{ 
      \hfill
      \href{http://sdis.dpaw.wa.gov.au}{Science Directorate Information System} 
      \newline 
      \href{http://sdis.dpaw.wa.gov.au/documents/progressreport/1332/}{Progress Report 2012-36 (FY 2014-2015)} 
  }
  
  
  
  
  \fancyfoot[L]{ \leftmark\newline\textbf{Printed}\textit{ June 29, 2015, 4:02 p.m. }}
  \fancyfoot[R]{  \, \newline Page \thepage\ of \pageref{LastPage} }
  
  
}
\pagestyle{plain}
%
% end Headers
%-----------------------------------------------------------------------------%

\begin{document}

%-----------------------------------------------------------------------------%
% Title page
%

%
% end title page
%-----------------------------------------------------------------------------%




\section*{Context Summary}
Shrubland ecosystems are widespread in south-western Australia and are
the predominant vegetation type in coastal areas between Geraldton and
Esperance. Coastal shrublands are renowned for their flammability, and
fires can be fast-moving and intense when dead fine fuels are dry and
wind speeds exceed 15 km h\textsuperscript{-1}. Fires may transition
abruptly from the litter layer to the shrub layer in response to minor
changes in wind speed and fuel dryness, making it difficult to use
prescribed fire reliably to meet management objectives. Currently the
Department does not have a fire behaviour prediction guide specific to
coastal shrublands, and this represents a significant gap in
science-based decision making to underpin the use of fire for bushfire
risk management and biodiversity conservation. This issue was
highlighted by the Special Inquiry into the November 2011 Margaret River
bushfire conducted by the Hon. Mick Keelty. This project addresses
Recommendation 4 of the Keelty Special Inquiry that the Department be
supported to conduct further research into the fuel management of
coastal heath in the south-west of Western Australia exploring
alternatives to burning as well as best practice for burning.



\section*{Aims Summary}
\begin{itemize}
\itemsep1pt\parskip0pt\parsep0pt
\item
  Provide a systematic approach for describing fuel characteristics and
  predicting fire behaviour in coastal shrublands in order to more
  effectively manage prescribed burning and bushfires.
\item
  Facilitate evaluation of the effectiveness of prescribed fire and
  other fuel management practices for mitigating the impact of
  bushfires.
\end{itemize}



\section*{Progress}
\begin{itemize}
\itemsep1pt\parskip0pt\parsep0pt
\item
  A pilot study was undertaken at seven sites to test the
  cost-effectiveness and practicality of different fuel sampling
  techniques.
\item
  Four sites suitable for collecting fire behaviour data have been
  established within planned burn areas in Albany, Frankland, Perth
  Hills and Moora districts.
\item
  Data from fires in Western Australian shrublands have been included in
  a fire spread model developed collaboratively by researchers from
  Australia, New Zealand and Mediterranean Europe.
\end{itemize}



\section*{Management implications}
\begin{itemize}
\itemsep1pt\parskip0pt\parsep0pt
\item
  Development of a systematic approach to describing fuels and
  predicting fire behaviour in coastal shrublands will allow the
  Department to better implement its fire management program.
\item
  Improved knowledge of factors determining fire behaviour in shrublands
  will contribute to more effective training programs for fire managers
  and fire-fighters from the Department and other organisations.
\end{itemize}



\section*{Future directions}
\begin{itemize}
\itemsep1pt\parskip0pt\parsep0pt
\item
  Collect fire behaviour from planned burns as these are implemented.
\item
  Plan and conduct further experimental burning to quantify threshold
  conditions for sustained fire spread in shrublands of different
  structure and time since fire.
\item
  Evaluate and verify the performance of the collaboratively-developed
  fire spread model for Western Australian shrublands.
\end{itemize}




\clearpage



\end{document}

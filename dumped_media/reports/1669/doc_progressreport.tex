
\documentclass[version=last,
    paper=a4, % paper size
    10pt, % default font size
    usenames,
    dvipsnames,
    oneside, % ONLINE
    headings=openany, % open chapters on odd and even pages
    %toc=chapterentrywithdots, % Table of Contents style
    %BCOR=7mm, % PRINT Binding Correction
    %DIV=13, % typearea 161.54 mm x 228.46 mm, top margin 22.85 mm, inner margin 16.15 mm
    %DIV=14, % 165.00 233.36 21.21 15.00
    DIV=15 % 168.00 237.60 19.80 14.00
]{scrbook}
\usepackage{typearea}
\usepackage[automark,headsepline,footsepline]{scrlayer-scrpage} % Headers and footers

%%
%% Fonts, encoding, spacing, indentation
%%
\usepackage{txfonts}
\renewcommand{\familydefault}{\sfdefault} % Default to Sans Serif font
\usepackage[english]{babel}
\usepackage[T1]{fontenc}
\usepackage[utf8]{inputenc}

% Paragraph spacing
%\usepackage{parskip}    % Paragraph spacing
%\setlength{\parindent}{0em} % Don't indent paragraphs - ONLINE
%\setlength{\parskip}{1.3 ex plus 0.5ex minus 0.3ex} % 1-1.8 ex vertical space between paragraphs - ONLINE

% Spacing of headings
%\RedeclareSectionCommand[afterskip=3pt]{section} % less space after section
%\RedeclareSectionCommand[beforeskip=0cm]{subsection} % less space between HRule and project name
%\RedeclareSectionCommand[afterskip=0.1\baselineskip]{subsubsection} % less space after progressreport subheadings

% Table font size
\usepackage{etoolbox}
\AtBeginEnvironment{longtabu}{\footnotesize}{}{}

%%
%% Tables, columns, layout
%%
\usepackage{multicol}   % 2 col publications
\usepackage{pdflscape}  % Landscape pages
\usepackage{pdfpages}   % Include PDFs
\usepackage{hanging}    % hanging paragraphs for publications
%\usepackage{titletoc}   % Required for manipulating the table of contents
\setcounter{tocdepth}{2} % TOC list down to section
\usepackage{enumerate}  % Enumerations
\usepackage{enumitem}   % Enumerations
\usepackage{longtable}  % Multipage table
\usepackage{tabu}       %
\setlength{\tabulinesep}{1.5mm} % Consistent vertical spacing in tabu

%%
%% Graphics, images, colours
%%
\usepackage{graphicx} % embedded images
\usepackage{eso-pic} %
\usepackage{colortbl} % define custom named colours
\definecolor{RedFire}{RGB}{146,25,28}
\definecolor{ParksWildlife}{RGB}{0,85,144}
\definecolor{successbg}{RGB}{223,240,216}
\definecolor{errorbg}{RGB}{242,222,222}
\definecolor{warningbg}{RGB}{252,248,227}
\definecolor{infobg}{RGB}{217,237,247}
\definecolor{muted}{RGB}{153,153,153}
\definecolor{success}{RGB}{70,136,71}
\definecolor{error}{RGB}{185,74,72}
\definecolor{warning}{RGB}{192,152,83}
\definecolor{info}{RGB}{58,135,173}

\definecolor{required}{RGB}{192,152,83}
\definecolor{requiredbg}{RGB}{252,248,227}
\definecolor{denied}{RGB}{185,74,72}
\definecolor{deniedbg}{RGB}{242,222,222}
\definecolor{granted}{RGB}{70,136,71}
\definecolor{grantedbg}{RGB}{223,240,216}
\definecolor{not reqiured}{RGB}{153,153,153}
\definecolor{not requiredbg}{RGB}{255,255,255}

\usepackage{tikz} % Drawing
\usetikzlibrary{arrows,shapes,positioning,shadows,trees}

%%
%% Links, URLs
%%
\usepackage[
    linktoc=all,
    %colorlinks=false,  %PRINT
    colorlinks=true, % ONLINE
    linkcolor=RedFire, % ONLINE
    urlcolor=ParksWildlife, % ONLINE
    pdftitle=Progress Report SP 2002-001 (FY 2015-2016)
]{hyperref}

% Black magic to linebreak URLs
\usepackage{url}
\makeatletter
\g@addto@macro{\UrlBreaks}{\UrlOrds}
\makeatother

%%
%% Custom macros
%%
% Thick Horizontal rule
\newcommand{\HRule}{\vspace{8mm}\\\noindent\rule{\linewidth}{0.1pt}}

% Custom Tikz node for SDS diagram
\newcommand\mynode[6][]{
    \node[#1] (#2){
        \parbox{#3\relax}{
            \begin{center}
            \textbf{#4}\\
            #5\\
            \footnotesize{#6}
            \end{center}}};}



%-----------------------------------------------------------------------------%
% Headers and Footers
\automark{section}
\ohead{\href{http://sdis.dpaw.wa.gov.au/documents/progressreport/1669/}{Progress Report SP 2002-001
}}
\chead{\href{http://sdis.dpaw.wa.gov.au}{SDIS}} % center header ONLINE
\ihead{\href{http://sdis.dpaw.wa.gov.au}{
        \includegraphics[scale=0.4]{/mnt/projects/sdis/staticfiles/img/logo-dpaw.png}}}
\ifoot{\textbf{Printed}~Mon, 4 Jul 2016 16:16:25 +0800} % inner/left footer
\cfoot{} % center footer
\ofoot{\pagemark} % outer/right footer
\pagestyle{scrheadings}
\setkomafont{pageheadfoot}{\normalfont}

%-----------------------------------------------------------------------------%
\begin{document}
\raggedbottom

%-----------------------------------------------------------------------------%
% Title page
\subject{Progress Report SP 2002-001
}
\title{Genetic and ecological viability of plant populations in remnant
vegetation
}
\subtitle{Plant Science and Herbarium
}
\author{}
\publishers{\small
    \subsection*{Project Core Team}
\begin{tabu} {X X}
\textbf{Supervising Scientist} & Dave Coates
\\
\textbf{Data Custodian} & Dave Coates
\\
\textbf{Site Custodian} & Dave Coates
\\
\end{tabu}


    \subsection*{Project status as of July 4, 2016, 4:16 p.m.}
\begin{tabu} {X X}
& Approved and active
\\
\end{tabu}

    
\subsection*{Document endorsements and approvals as of July 4, 2016, 4:16 p.m.}
\begin{tabu} {X X}

%\rowcolor{grantedbg}
    \textbf{Project Team} & 
    \textcolor{granted}{ granted}\\

%\rowcolor{grantedbg}
    \textbf{Program Leader} & 
    \textcolor{granted}{ granted}\\

%\rowcolor{grantedbg}
    \textbf{Directorate} & 
    \textcolor{granted}{ granted}\\

\end{tabu}



}
\uppertitleback{}
\lowertitleback{}
\date{}

%-----------------------------------------------------------------------------%
% Front matter
\frontmatter
\maketitle
%-----------------------------------------------------------------------------%
% Main matter
\mainmatter

\section*{Genetic and ecological viability of plant populations in remnant
vegetation
}

D Coates, M Byrne, C Yates, T Llorens, M Millar, S McArthur, N Gibson, J
Sampson


\section*{Context}
A priority for long-term conservation of remnant vegetation is the
maintenance of viable plant populations. However, little is currently
known about what biological factors actually affect population
persistence. This project quantifies genetic and ecological factors that
influence the viability of plant populations in fragmented Western
Australian agricultural landscapes and explores how these are affected
by remnant vegetation characteristics such as size, shape, isolation,
disturbance and landscape position.



\section*{Aims}
\begin{itemize}
\itemsep1pt\parskip0pt\parsep0pt
\item
  Identify and quantify the genetic and demographic factors that affect
  the viability of plant populations in vegetation remnants. The focus
  will be on the effects of genetic erosion, inbreeding and pollinator
  limitation on seed production and seedling fitness.
\item
  Examine and model the relationships between key genetic and
  demographic factors affecting viability and remnant vegetation
  characteristics such as size, disturbance and landscape position.
\item
  Develop specific genetic and demographic guidelines for management of
  remnant populations of the target taxa and general landscape design
  principles for major plant life-history types that will maximise the
  probability of population persistence.
\item
  Develop an understanding of the population biology, mating systems and
  gene flow of flora with distributions centred on the seasonally wet
  Busselton ironstone communities to inform management for long-term
  conservation in relation to population viability (population size and
  degree of connection) and appropriate fire frequency.
\end{itemize}



\section*{Progress}
\begin{itemize}
\itemsep1pt\parskip0pt\parsep0pt
\item
  Analysis of the genetic diversity, mating system and reproductive
  biology of \emph{Hakea oldfieldii} has been completed. One paper has
  been published in \emph{Biological Journal of the Linnean Society} and
  another paper has been submitted. The three population areas in Perth
  Hills, Busselton and south coast showed significant genetic divergence
  as expected but there was also significant population divergence
  within these areas indicating low historical connectivity. Low
  diversity is associated with historical processes rather than recent
  fragmentation. Populations were predominantly outcrossed even when
  severely reduced in size, indicating little effect of inbreeding in
  small populations, but reproductive parameters were higher in small
  populations with intact vegetation compared to disturbed sites,
  highlighting effects of understorey on pollinator abundance and
  behaviour.
\item
  Analysis of reproductive output, mating system variation, progeny
  fitness and genetic diversity in relation to habitat fragmentation has
  been completed for~\emph{Eucalyptus wandoo.}~A paper is in the final
  preparation stage. Higher levels of soil electrical conductivity were
  strongly associated with greatly reduced fruit set, suggesting
  significant sub-lethal effects of secondary soil salinity on
  reproduction. Levels of pollination were surprisingly high in small
  populations but probably involve high levels of self-pollination,
  leading to low seed set in small populations. Seedling survival in a
  shadehouse trial was~higher for~populations that had a higher edge to
  area ratio, which may be attributed to increased nutrient availability
  in the~agricultural matrix.
\item
  Analysis of reproductive output, mating system variation, progeny
  fitness and genetic diversity~in fragmented populations
  of~\emph{Eremaea pauciflora} is close to completion.
\end{itemize}



\section*{Management implications}
\begin{itemize}
\itemsep1pt\parskip0pt\parsep0pt
\item
  The ability to rapidly and accurately assess the conservation value of
  a vegetation remnant is a critical step in landscape management aimed
  at integrating the goals of conservation and agricultural production.
  Currently much of this assessment is based on best guesses using
  anecdotal species-specific evidence, on the general principle that
  bigger is better, and on simple presence and absence data that take
  little account of long-term remnant trajectories. Improved accuracy of
  assessment of long-term persistence of broad classes of plant species
  will facilitate improved prioritisation of remnants for conservation
  and therefore better allocation of limited management resources.
\item
  Establishment of realistic empirically-based goals for remnant size,
  shape and landscape configuration that maximise regional persistence
  of plant species will allow more efficient conservation efforts at the
  landscape level by facilitating cost-benefit analyses for remnant
  management and restoration work.
\item
  The \emph{H. oldfieldii} study showed that conserving populations in
  intact habitat is a high priority to maintain the genetic and
  ecological processes in naturally fragmented and insular species.
  Management interventions, such as enrichment planting to increase the
  diversity of pollen donors, should also include habitat and
  understorey restoration to facilitate effective mating patterns. Seed
  collection should prioritise populations with intact habitat to
  maximise genetic diversity.
\item
  Levels of secondary soil salinity should be assessed and monitored in
  high-value vegetation remnants in agricultural landscapes.
  The~\emph{E. wandoo~}study~revealed that low to moderate levels of
  soil salinity can have highly significant sub-lethal effects on
  reproductive output, which are~likely to translate to reduced
  population viability.
\end{itemize}



\section*{Future directions}
\begin{itemize}
\itemsep1pt\parskip0pt\parsep0pt
\item
  Prepare papers on mating system variation and reproductive output in
  \emph{E. pauciflora}~and genetic diversity in~\emph{Calothamnus
  quadrifidus}.
\item
  Finalise paper on genetic diversity, pollen dispersal and mating
  systems in \emph{Banksia nivea} ssp. \emph{uliginosa}.
\item
  Finalise paper on reproductive biology and demography in \emph{B.
  nivea} ssp. \emph{uliginosa}.
\item
  Finalise paper on reproductive output,~mating system variation,
  progeny fitness and genetic diversity in \emph{E. wandoo.}
\end{itemize}



%-----------------------------------------------------------------------------%
% Back matter
%\backmatter
\end{document}
%-----------------------------------------------------------------------------%


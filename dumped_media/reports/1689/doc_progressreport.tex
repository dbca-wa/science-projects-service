
\documentclass[version=last,
    paper=a4, % paper size
    10pt, % default font size
    usenames,
    dvipsnames,
    oneside, % ONLINE
    headings=openany, % open chapters on odd and even pages
    %toc=chapterentrywithdots, % Table of Contents style
    %BCOR=7mm, % PRINT Binding Correction
    %DIV=13, % typearea 161.54 mm x 228.46 mm, top margin 22.85 mm, inner margin 16.15 mm
    %DIV=14, % 165.00 233.36 21.21 15.00
    DIV=15 % 168.00 237.60 19.80 14.00
]{scrbook}
\usepackage{typearea}
\usepackage[automark,headsepline,footsepline]{scrlayer-scrpage} % Headers and footers

%%
%% Fonts, encoding, spacing, indentation
%%
\usepackage{txfonts}
\renewcommand{\familydefault}{\sfdefault} % Default to Sans Serif font
\usepackage[english]{babel}
\usepackage[T1]{fontenc}
\usepackage[utf8]{inputenc}

% Paragraph spacing
%\usepackage{parskip}    % Paragraph spacing
%\setlength{\parindent}{0em} % Don't indent paragraphs - ONLINE
%\setlength{\parskip}{1.3 ex plus 0.5ex minus 0.3ex} % 1-1.8 ex vertical space between paragraphs - ONLINE

% Spacing of headings
%\RedeclareSectionCommand[afterskip=3pt]{section} % less space after section
%\RedeclareSectionCommand[beforeskip=0cm]{subsection} % less space between HRule and project name
%\RedeclareSectionCommand[afterskip=0.1\baselineskip]{subsubsection} % less space after progressreport subheadings

% Table font size
\usepackage{etoolbox}
\AtBeginEnvironment{longtabu}{\footnotesize}{}{}

%%
%% Tables, columns, layout
%%
\usepackage{multicol}   % 2 col publications
\usepackage{pdflscape}  % Landscape pages
\usepackage{pdfpages}   % Include PDFs
\usepackage{hanging}    % hanging paragraphs for publications
%\usepackage{titletoc}   % Required for manipulating the table of contents
\setcounter{tocdepth}{2} % TOC list down to section
\usepackage{enumerate}  % Enumerations
\usepackage{enumitem}   % Enumerations
\usepackage{longtable}  % Multipage table
\usepackage{tabu}       %
\setlength{\tabulinesep}{1.5mm} % Consistent vertical spacing in tabu

%%
%% Graphics, images, colours
%%
\usepackage{graphicx} % embedded images
\usepackage{eso-pic} %
\usepackage{colortbl} % define custom named colours
\definecolor{RedFire}{RGB}{146,25,28}
\definecolor{ParksWildlife}{RGB}{0,85,144}
\definecolor{successbg}{RGB}{223,240,216}
\definecolor{errorbg}{RGB}{242,222,222}
\definecolor{warningbg}{RGB}{252,248,227}
\definecolor{infobg}{RGB}{217,237,247}
\definecolor{muted}{RGB}{153,153,153}
\definecolor{success}{RGB}{70,136,71}
\definecolor{error}{RGB}{185,74,72}
\definecolor{warning}{RGB}{192,152,83}
\definecolor{info}{RGB}{58,135,173}

\definecolor{required}{RGB}{192,152,83}
\definecolor{requiredbg}{RGB}{252,248,227}
\definecolor{denied}{RGB}{185,74,72}
\definecolor{deniedbg}{RGB}{242,222,222}
\definecolor{granted}{RGB}{70,136,71}
\definecolor{grantedbg}{RGB}{223,240,216}
\definecolor{not reqiured}{RGB}{153,153,153}
\definecolor{not requiredbg}{RGB}{255,255,255}

\usepackage{tikz} % Drawing
\usetikzlibrary{arrows,shapes,positioning,shadows,trees}

%%
%% Links, URLs
%%
\usepackage[
    linktoc=all,
    %colorlinks=false,  %PRINT
    colorlinks=true, % ONLINE
    linkcolor=RedFire, % ONLINE
    urlcolor=ParksWildlife, % ONLINE
    pdftitle=Progress Report CF 2011-116 (FY 2015-2016)
]{hyperref}

% Black magic to linebreak URLs
\usepackage{url}
\makeatletter
\g@addto@macro{\UrlBreaks}{\UrlOrds}
\makeatother

%%
%% Custom macros
%%
% Thick Horizontal rule
\newcommand{\HRule}{\vspace{8mm}\\\noindent\rule{\linewidth}{0.1pt}}

% Custom Tikz node for SDS diagram
\newcommand\mynode[6][]{
    \node[#1] (#2){
        \parbox{#3\relax}{
            \begin{center}
            \textbf{#4}\\
            #5\\
            \footnotesize{#6}
            \end{center}}};}



%-----------------------------------------------------------------------------%
% Headers and Footers
\automark{section}
\ohead{\href{http://sdis.dpaw.wa.gov.au/documents/progressreport/1689/}{Progress Report CF 2011-116
}}
\chead{\href{http://sdis.dpaw.wa.gov.au}{SDIS}} % center header ONLINE
\ihead{\href{http://sdis.dpaw.wa.gov.au}{
        \includegraphics[scale=0.4]{/mnt/projects/sdis/staticfiles/img/logo-dpaw.png}}}
\ifoot{\textbf{Printed}~Mon, 11 Jul 2016 13:47:41 +0800} % inner/left footer
\cfoot{} % center footer
\ofoot{\pagemark} % outer/right footer
\pagestyle{scrheadings}
\setkomafont{pageheadfoot}{\normalfont}

%-----------------------------------------------------------------------------%
\begin{document}
\raggedbottom

%-----------------------------------------------------------------------------%
% Title page
\subject{Progress Report CF 2011-116
}
\title{WAMSI 1 Node 3: science administration, coordination and integration
}
\subtitle{Marine Science
}
\author{}
\publishers{\small
    \subsection*{Project Core Team}
\begin{tabu} {X X}
\textbf{Supervising Scientist} & Kelly Waples
\\
\textbf{Data Custodian} & 
\\
\textbf{Site Custodian} & 
\\
\end{tabu}


    \subsection*{Project status as of July 11, 2016, 1:47 p.m.}
\begin{tabu} {X X}
& Approved and active
\\
\end{tabu}

    
\subsection*{Document endorsements and approvals as of July 11, 2016, 1:47 p.m.}
\begin{tabu} {X X}

%\rowcolor{grantedbg}
    \textbf{Project Team} & 
    \textcolor{granted}{ granted}\\

%\rowcolor{grantedbg}
    \textbf{Program Leader} & 
    \textcolor{granted}{ granted}\\

%\rowcolor{grantedbg}
    \textbf{Directorate} & 
    \textcolor{granted}{ granted}\\

\end{tabu}



}
\uppertitleback{}
\lowertitleback{}
\date{}

%-----------------------------------------------------------------------------%
% Front matter
\frontmatter
\maketitle
%-----------------------------------------------------------------------------%
% Main matter
\mainmatter

\section*{WAMSI 1 Node 3: science administration, coordination and integration
}

K Waples, A Kendrick


\section*{Context}
In 2005, the State Government allocated \$5 million to undertake
research that would underpin~management of Ningaloo Marine Park. A
research plan was developed to address key strategies in the Ningaloo
Marine Park Management Plan. In 2007, a joint research body, the Western
Australian Marine Science Institution (WAMSI) was formed. Parks and
Wildlife~was the leader of Node 3 of WAMSI, which addressed research in
marine biodiversity and conservation. At the same time as the
development of WAMSI, CSIRO Wealth from Oceans National Research
Flagship program established the Ningaloo Collaboration Cluster (the
Cluster) to address the integration of knowledge of reef use,
biodiversity and socio-economics into a Management Strategy Evaluation
(MSE) model for Ningaloo Marine Park and the Gascoyne Region in general.
The research program of the Cluster complemented that undertaken through
WAMSI and collectively these two programs, along with core research
undertaken by the Australian Institute of Marine Science (AIMS) at
Ningaloo, have become known as the Ningaloo Research Program (NRP).

A key focus of the WAMSI component of the program was to ensure the
transfer and uptake of knowledge generated through the research into
Parks~and Wildlife~management policies, practices and actions.~The
research program was completed in December 2011 and the final WAMSI Node
3 report submitted in 2102.

A new component to this project was initiated in 2015 to assess
the~extent to which new and evolving knowledge generated through NRP has
been integrated into decision making processes and identify options to
improve its uptake in the future.~

~



\section*{Aims}
Ensure the coordination and administration of the research program.

Ensure the integration of this research program with other research
within WAMSI and with external programs relevant to the Ningaloo Marine
Park

Ensure the outputs of research undertaken through the NRP reach target
audiences.

Ensure that knowledge transfer and uptake occurs between scientists,
resource managers and decision makers.

Ensure the long-term storage and custodianship of data from the research
undertaken.

Evaluate the effectiveness of the knowledge transfer and uptake process
so that lessons learned can be applied to future WAMSI and Parks and
Wildlife research programs.



\section*{Progress}
\begin{itemize}
\itemsep1pt\parskip0pt\parsep0pt
\item
  A project was initiated with an external collaborator to assess the
  effectiveness of the WAMSI Node 3 knowledge exchange process through
  assessing the extent to which new knowledge generated through the NRP
  is useful to management and has been uptaken into management activity
  and processes.
\item
  Parks and Wildlife operational headquarters, regional~and district
  staff participated in interviews with the project leader.
\item
  The project has been expanded to include evaluating changes to social
  licence based on the investment in research and knowledge exchange
  activities in the Exmouth community.
\item
  Two seminars were held at Kensington to share the process of this
  research and the outcomes.
\end{itemize}



\section*{Management implications}
\begin{itemize}
\itemsep1pt\parskip0pt\parsep0pt
\item
  This project will improve the exchange of knowledge between science
  and managers. ~It will identify the successes of the NRP as well as
  the areas where there could be opportunities for~improvement and make
  recommendations on how to best manage knowledge so that it is relevant
  to managers and reaches target audiences when it is needed. ~
\item
  The recommendations will inform knowledge transfer activities within
  Science and Conservation Division and the broader Department.~
\end{itemize}



\section*{Future directions}
\begin{itemize}
\itemsep1pt\parskip0pt\parsep0pt
\item
  A manuscript is in preparation on the capacities and processes that
  may assist in overcoming knowledge transfer barriers.~
\item
  A project report will be produced.
\end{itemize}



%-----------------------------------------------------------------------------%
% Back matter
%\backmatter
\end{document}
%-----------------------------------------------------------------------------%


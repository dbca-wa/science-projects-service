
\documentclass[version=last,
    paper=a4, % paper size
    10pt, % default font size
    usenames,
    dvipsnames,
    oneside, % ONLINE
    headings=openany, % open chapters on odd and even pages
    %toc=chapterentrywithdots, % Table of Contents style
    %BCOR=7mm, % PRINT Binding Correction
    %DIV=13, % typearea 161.54 mm x 228.46 mm, top margin 22.85 mm, inner margin 16.15 mm
    %DIV=14, % 165.00 233.36 21.21 15.00
    DIV=15 % 168.00 237.60 19.80 14.00
]{scrbook}
\usepackage{typearea}
\usepackage[automark,headsepline,footsepline]{scrlayer-scrpage} % Headers and footers

%%
%% Fonts, encoding, spacing, indentation
%%
\usepackage{txfonts}
\renewcommand{\familydefault}{\sfdefault} % Default to Sans Serif font
\usepackage[english]{babel}
\usepackage[T1]{fontenc}
\usepackage[utf8]{inputenc}

% Paragraph spacing
%\usepackage{parskip}    % Paragraph spacing
%\setlength{\parindent}{0em} % Don't indent paragraphs - ONLINE
%\setlength{\parskip}{1.3 ex plus 0.5ex minus 0.3ex} % 1-1.8 ex vertical space between paragraphs - ONLINE

% Spacing of headings
%\RedeclareSectionCommand[afterskip=3pt]{section} % less space after section
%\RedeclareSectionCommand[beforeskip=0cm]{subsection} % less space between HRule and project name
%\RedeclareSectionCommand[afterskip=0.1\baselineskip]{subsubsection} % less space after progressreport subheadings

% Table font size
\usepackage{etoolbox}
\AtBeginEnvironment{longtabu}{\footnotesize}{}{}

%%
%% Tables, columns, layout
%%
\usepackage{multicol}   % 2 col publications
\usepackage{pdflscape}  % Landscape pages
\usepackage{pdfpages}   % Include PDFs
\usepackage{hanging}    % hanging paragraphs for publications
%\usepackage{titletoc}   % Required for manipulating the table of contents
\setcounter{tocdepth}{2} % TOC list down to section
\usepackage{enumerate}  % Enumerations
\usepackage{enumitem}   % Enumerations
\usepackage{longtable}  % Multipage table
\usepackage{tabu}       %
\setlength{\tabulinesep}{1.5mm} % Consistent vertical spacing in tabu

%%
%% Graphics, images, colours
%%
\usepackage{graphicx} % embedded images
\usepackage{eso-pic} %
\usepackage{colortbl} % define custom named colours
\definecolor{RedFire}{RGB}{146,25,28}
\definecolor{ParksWildlife}{RGB}{0,85,144}
\definecolor{successbg}{RGB}{223,240,216}
\definecolor{errorbg}{RGB}{242,222,222}
\definecolor{warningbg}{RGB}{252,248,227}
\definecolor{infobg}{RGB}{217,237,247}
\definecolor{muted}{RGB}{153,153,153}
\definecolor{success}{RGB}{70,136,71}
\definecolor{error}{RGB}{185,74,72}
\definecolor{warning}{RGB}{192,152,83}
\definecolor{info}{RGB}{58,135,173}

\definecolor{required}{RGB}{192,152,83}
\definecolor{requiredbg}{RGB}{252,248,227}
\definecolor{denied}{RGB}{185,74,72}
\definecolor{deniedbg}{RGB}{242,222,222}
\definecolor{granted}{RGB}{70,136,71}
\definecolor{grantedbg}{RGB}{223,240,216}
\definecolor{not reqiured}{RGB}{153,153,153}
\definecolor{not requiredbg}{RGB}{255,255,255}

\usepackage{tikz} % Drawing
\usetikzlibrary{arrows,shapes,positioning,shadows,trees}

%%
%% Links, URLs
%%
\usepackage[
    linktoc=all,
    %colorlinks=false,  %PRINT
    colorlinks=true, % ONLINE
    linkcolor=RedFire, % ONLINE
    urlcolor=ParksWildlife, % ONLINE
    pdftitle=Progress Report SP 2013-006 (FY 2015-2016)
]{hyperref}

% Black magic to linebreak URLs
\usepackage{url}
\makeatletter
\g@addto@macro{\UrlBreaks}{\UrlOrds}
\makeatother

%%
%% Custom macros
%%
% Thick Horizontal rule
\newcommand{\HRule}{\vspace{8mm}\\\noindent\rule{\linewidth}{0.1pt}}

% Custom Tikz node for SDS diagram
\newcommand\mynode[6][]{
    \node[#1] (#2){
        \parbox{#3\relax}{
            \begin{center}
            \textbf{#4}\\
            #5\\
            \footnotesize{#6}
            \end{center}}};}



%-----------------------------------------------------------------------------%
% Headers and Footers
\automark{section}
\ohead{\href{http://sdis.dpaw.wa.gov.au/documents/progressreport/1606/}{Progress Report SP 2013-006
}}
\chead{\href{http://sdis.dpaw.wa.gov.au}{SDIS}} % center header ONLINE
\ihead{\href{http://sdis.dpaw.wa.gov.au}{
        \includegraphics[scale=0.4]{/mnt/projects/sdis/staticfiles/img/logo-dpaw.png}}}
\ifoot{\textbf{Printed}~Mon, 11 Jul 2016 13:03:52 +0800} % inner/left footer
\cfoot{} % center footer
\ofoot{\pagemark} % outer/right footer
\pagestyle{scrheadings}
\setkomafont{pageheadfoot}{\normalfont}

%-----------------------------------------------------------------------------%
\begin{document}
\raggedbottom

%-----------------------------------------------------------------------------%
% Title page
\subject{Progress Report SP 2013-006
}
\title{The influence of macroalgal fields on coral reef fish
}
\subtitle{Marine Science
}
\author{}
\publishers{\small
    \subsection*{Project Core Team}
\begin{tabu} {X X}
\textbf{Supervising Scientist} & Shaun Wilson
\\
\textbf{Data Custodian} & 
\\
\textbf{Site Custodian} & 
\\
\end{tabu}


    \subsection*{Project status as of July 11, 2016, 1:03 p.m.}
\begin{tabu} {X X}
& Approved and active
\\
\end{tabu}

    
\subsection*{Document endorsements and approvals as of July 11, 2016, 1:03 p.m.}
\begin{tabu} {X X}

%\rowcolor{grantedbg}
    \textbf{Project Team} & 
    \textcolor{granted}{ granted}\\

%\rowcolor{grantedbg}
    \textbf{Program Leader} & 
    \textcolor{granted}{ granted}\\

%\rowcolor{grantedbg}
    \textbf{Directorate} & 
    \textcolor{granted}{ granted}\\

\end{tabu}



}
\uppertitleback{}
\lowertitleback{}
\date{}

%-----------------------------------------------------------------------------%
% Front matter
\frontmatter
\maketitle
%-----------------------------------------------------------------------------%
% Main matter
\mainmatter

\section*{The influence of macroalgal fields on coral reef fish
}

S Wilson, T Holmes


\section*{Context}
Macroalgae are a prominent component of tropical benthic communities
along the north-west coast of Australia. Within the Ningaloo Reef
lagoon, large fields of macroalgae are a distinct feature of the marine
park, covering \textasciitilde{}2000 ha. These macroalgal fields are
important habitat for fish targeted by recreational fishers and are a
focal area for boating activity within the park. Moreover, large
seasonal shifts in algal biomass on these and other tropical reefs
suggest macroalgae play an important role in nutrient fluxes in Ningaloo
and similar systems. Recent work at Ningaloo has quantitatively assessed
seasonal variation in biomass and diversity of macroalgal communities
and assessed methods for estimating coverage of macroalgae using remote
sensing. This project will build on the information gained from these
initial studies to improve understanding of how macroalgae are
distributed across the Ningaloo lagoon and better define the role of
macroalgal fields as habitat for fish recruits and adults.



\section*{Aims}
\begin{itemize}
\itemsep1pt\parskip0pt\parsep0pt
\item
  Quantify spatial variance in macroalgal fields at Ningaloo Marine
  Park, and determine the relative importance of physical and biological
  drivers of algal abundance and diversity.
\item
  Identify attributes of macroalgal fields favoured by juvenile fish and
  examine the relative importance of habitat quality and predation on
  juvenile abundance.
\item
  Assess the influence of juvenile fish on replenishment and future
  adult abundance.
\end{itemize}



\section*{Progress}
\begin{itemize}
\itemsep1pt\parskip0pt\parsep0pt
\item
  A paper documenting the movement of fish to macroalgal patches with
  greater cover and canopy height in response to seasonal shifts in
  macroalgal assemblages has been published in the journal
  \emph{Ecosphere.\\}
\item
  Work on seasonal fluxes in macroalgal biomass and the importance of
  tropical macroalgae~as habitat for fish has been presented as seminars
  at the University of Stockholm, Sweden, The University of Singapore
  and to Parks and Wildlife staff in the Exmouth District. It has also
  been incorporated into two posts on Departmental social media.
\item
  A PhD project (STP 2015-006) examining how within- and between-patch
  habitat structure influences reef fish diversity in macroalgal
  habitats has begun.
\item
  Field~data for four summers and three winters has now been collected.
\item
  Data collected from this project has been incorporated into a wider
  geographic analysis of shifts in macroalgal communities along the
  Western Australian~coast, which has now been accepted for publication
  at the journal \emph{Science}.
\end{itemize}



\section*{Management implications}
\begin{itemize}
\itemsep1pt\parskip0pt\parsep0pt
\item
  Marine conservation managers and planners will have a greater
  understanding of the ecological and social importance of macroalgal
  habitats in Western Auslralia's tropical north. This includes the
  ecological importance of macroalgal fields as habitat for fishes and
  their role in supporting key processes like recruitment. This
  knowledge will also improve the capacity to predict future abundances
  of adult fishes, particularly those threatened by changes to habitat,
  climate and fishing pressure, and will help to maintain important
  social values like recreational fishing.
\item
  An improved understanding of the distribution of macroalgal fields in
  tropical Wester Australia will improve predictions of the ecological
  significance of algal biomass when planning and managing marine
  reserves.
\item
  This study of macroalgal communities will provide baseline data for
  future long-term monitoring of the condition of macroalgal communities
  in marine reserves.
\end{itemize}



\section*{Future directions}
\begin{itemize}
\itemsep1pt\parskip0pt\parsep0pt
\item
  Compile and analyse~data to understand links between juvenile and
  adult fish, assessing the relative importance of juvenile abundance
  and suitable habitat.
\end{itemize}



%-----------------------------------------------------------------------------%
% Back matter
%\backmatter
\end{document}
%-----------------------------------------------------------------------------%



\documentclass[version=last, 
    paper=a4, % paper size
    10pt, % default font size
    usenames,
    dvipsnames, 
    oneside, % ONLINE
    headings=openany, % open chapters on odd and even pages
    %toc=chapterentrywithdots, % Table of Contents style
    %BCOR=7mm, % PRINT Binding Correction
    %DIV=13, % typearea 161.54 mm x 228.46 mm, top margin 22.85 mm, inner margin 16.15 mm
    %DIV=14, % 165.00 233.36 21.21 15.00
    DIV=15 % 168.00 237.60 19.80 14.00
]{scrbook}
\usepackage{typearea}
\usepackage[automark,headsepline,footsepline]{scrlayer-scrpage} % Headers and footers

%%
%% Fonts, encoding, spacing, indentation
%%
\usepackage{txfonts}
\renewcommand{\familydefault}{\sfdefault} % Default to Sans Serif font
\usepackage[english]{babel}
\usepackage[T1]{fontenc}
\usepackage[utf8]{inputenc}

% Paragraph spacing
%\usepackage{parskip}    % Paragraph spacing
%\setlength{\parindent}{0em} % Don't indent paragraphs - ONLINE
%\setlength{\parskip}{1.3 ex plus 0.5ex minus 0.3ex} % 1-1.8 ex vertical space between paragraphs - ONLINE

% Spacing of headings
%\RedeclareSectionCommand[afterskip=3pt]{section} % less space after section
%\RedeclareSectionCommand[beforeskip=0cm]{subsection} % less space between HRule and project name
%\RedeclareSectionCommand[afterskip=0.1\baselineskip]{subsubsection} % less space after progressreport subheadings

% Table font size
\usepackage{etoolbox}
\AtBeginEnvironment{longtabu}{\footnotesize}{}{}

%%
%% Tables, columns, layout
%%
\usepackage{multicol}   % 2 col publications
\usepackage{pdflscape}  % Landscape pages
\usepackage{pdfpages}   % Include PDFs
\usepackage{hanging}    % hanging paragraphs for publications
%\usepackage{titletoc}   % Required for manipulating the table of contents
\setcounter{tocdepth}{2} % TOC list down to section
\usepackage{enumerate}  % Enumerations
\usepackage{enumitem}   % Enumerations
\usepackage{longtable}  % Multipage table
\usepackage{tabu}       % 
\setlength{\tabulinesep}{1.5mm} % Consistent vertical spacing in tabu

%%
%% Graphics, images, colours
%%
\usepackage{graphicx} % embedded images
\usepackage{eso-pic} % 
\usepackage{colortbl} % define custom named colours
\definecolor{RedFire}{RGB}{146,25,28}
\definecolor{ParksWildlife}{RGB}{0,85,144}
\definecolor{successbg}{RGB}{223,240,216}
\definecolor{errorbg}{RGB}{242,222,222}
\definecolor{warningbg}{RGB}{252,248,227}
\definecolor{infobg}{RGB}{217,237,247}
\definecolor{muted}{RGB}{153,153,153}
\definecolor{success}{RGB}{70,136,71}
\definecolor{error}{RGB}{185,74,72}
\definecolor{warning}{RGB}{192,152,83}
\definecolor{info}{RGB}{58,135,173}

\definecolor{required}{RGB}{192,152,83}
\definecolor{requiredbg}{RGB}{252,248,227}
\definecolor{denied}{RGB}{185,74,72}
\definecolor{deniedbg}{RGB}{242,222,222}
\definecolor{granted}{RGB}{70,136,71}
\definecolor{grantedbg}{RGB}{223,240,216}
\definecolor{not reqiured}{RGB}{153,153,153}
\definecolor{not requiredbg}{RGB}{255,255,255}

\usepackage{tikz} % Drawing
\usetikzlibrary{arrows,shapes,positioning,shadows,trees}

%%
%% Links, URLs
%%
\usepackage[
    linktoc=all,
    %colorlinks=false,  %PRINT
    colorlinks=true, % ONLINE
    linkcolor=RedFire, % ONLINE
    urlcolor=ParksWildlife, % ONLINE
    pdftitle=doc\_progressreport.pdf
]{hyperref}

% Black magic to linebreak URLs
\usepackage{url}
\makeatletter
\g@addto@macro{\UrlBreaks}{\UrlOrds}
\makeatother

%%
%% Custom macros
%%
% Thick Horizontal rule
\newcommand{\HRule}{\vspace{8mm}\\\noindent\rule{\linewidth}{0.1pt}}

% Custom Tikz node for SDS diagram
\newcommand\mynode[6][]{\node[#1] (#2){\parbox{#3\relax}{\begin{center}\textbf{#4}\\#5\\\footnotesize{#6}\end{center}}};}




%-----------------------------------------------------------------------------%
% Headers and Footers
\automark{section}
\ohead{\href{http://sdis.dpaw.wa.gov.au/documents/progressreport/1401/}{Progress Report 0-2003
}}
\chead{\href{http://sdis.dpaw.wa.gov.au}{SDIS}} % center header ONLINE
\ihead{\href{http://sdis.dpaw.wa.gov.au}{
        \includegraphics[scale=0.4]{/mnt/projects/sdis/staticfiles/img/logo-dpaw.png}}}
\ifoot{\textbf{Printed}~Mon, 11 Apr 2016 16:17:31 +0800} % inner/left footer
\cfoot{} % center footer
\ofoot{\pagemark} % outer/right footer
\pagestyle{scrheadings}
\setkomafont{pageheadfoot}{\normalfont}

%-----------------------------------------------------------------------------%
\begin{document}
\raggedbottom

%-----------------------------------------------------------------------------%
% Title page
\subject{Progress Report 0-2003
}
\title{Development of effective broad-scale aerial baiting strategies for the
control of feral cats
}
\subtitle{Animal Science
}
\author{}
\publishers{\small
    \subsection*{Project Core Team}
\begin{tabu} {X X}
\textbf{Supervising Scientist} & Dave Algar
\\
\textbf{Data Custodian} & 
\\
\textbf{Site Custodian} & 
\\
\end{tabu}


    \subsection*{Project status as of April 11, 2016, 4:17 p.m.}
\begin{tabu} {X X}
& Approved and active
\\
\end{tabu}

    
\subsection*{Document endorsements and approvals as of April 11, 2016, 4:17 p.m.}
\begin{tabu} {X X}

%\rowcolor{grantedbg}
    \textbf{Project Team} & 
    \textcolor{granted}{ granted}\\

%\rowcolor{grantedbg}
    \textbf{Program Leader} & 
    \textcolor{granted}{ granted}\\

%\rowcolor{grantedbg}
    \textbf{Directorate} & 
    \textcolor{granted}{ granted}\\

\end{tabu}



}
\uppertitleback{}
\lowertitleback{}
\date{}

%-----------------------------------------------------------------------------%
% Front matter
\frontmatter
\maketitle
%-----------------------------------------------------------------------------%
% Main matter
\mainmatter

\section*{Development of effective broad-scale aerial baiting strategies for the
control of feral cats
}

D Algar, N Hamilton


\section*{Context}
The effective control of feral cats is one of the most important native
fauna conservation issues in Australia. Development of an effective
broad-scale baiting technique, and the incorporation of a suitable toxin
for feral cats, is cited as a high priority in the National Threat
Abatement Plan for Predation of Feral Cats, as it is most likely to
yield a practical, cost-effective method to control feral cat numbers in
strategic areas and promote the recovery of threatened fauna.



\section*{Aims}
\begin{itemize}
\itemsep1pt\parskip0pt\parsep0pt
\item
  Design and develop a bait medium that is readily consumed by feral
  cats.
\item
  Examine bait uptake in relation to the time of year, to enable baiting
  programs to be conducted when bait uptake is at its peak and therefore
  maximise efficiency.
\item
  Examine baiting intensity in relation to baiting efficiency to
  optimise control.
\item
  Examine baiting frequency required to provide long-term and sustained
  effective control.
\item
  Assess the potential impact of baiting programs on non-target species
  and populations and devise methods to reduce the potential risk where
  possible.
\item
  Provide a technique for the reliable estimation of cat abundance.
\end{itemize}



\section*{Progress}
\begin{itemize}
\itemsep1pt\parskip0pt\parsep0pt
\item
  Research into bait composition is continuing with the objective of
  further improving bait uptake. Chemical synthesis of several compounds
  that elicit a chewing response by cats has been achieved. One of these
  compounds is being manufactured at a scale that will enable
  incorporation into baits and reliable assessment of any improvement to
  bait uptake. In addition, the surface coating of baits with mould
  inhibitors is continuing.
\item
  Feral cat baiting programs on the Fortescue Marsh (Pilbara) were
  conducted in 2012, 2013 and 2014. All campaigns resulted in
  statistically significant declines in cat occupancy rates in the
  baiting area. A further baiting program is being conducted this
  winter. Research into the effectiveness of baiting strategies is
  continuing to be assessed under the temperate climatic conditions of
  the south-west at sites including Cape Arid and Fitzgerald River
  National Parks. The baiting programs conducted to date at Cape Arid
  National Park have contributed to an apparent stabilisation in the
  critically endangered western ground parrot population and significant
  population increases in number of other species, including the
  southern brown bandicoot. Similar results have been achieved at
  Fitzgerald River National Park where anecdotal increases in a number
  of native bird and mammal species have been observed.
\item
  Stage 1 of the management plan for the control of cats on the tropical
  Christmas Island has been completed with all domestic cats having been
  desexed, microchipped and registered. Stage 2 of the plan is
  continuing and involves the removal of all stray/feral cats from the
  residential area and surrounds. Stage 3 of the plan - island-wide
  eradication of feral cats commenced in 2015 following the funding
  being secured to see the project to its conclusion.
\item
  An assessment of bait consumption by the northern quoll is to be
  undertaken later this year. The bait medium will contain an
  encapsulated 1080 toxin. If the encapsulated toxin is demonstrated to
  be reliably rejected by quolls it will pave the way for feral cat
  campaigns to be conducted in northern Australia.
\item
  Work has been completed on the lure for the active camera traps. A
  combination of olfactory and visual attractants are used and have been
  shown to be successful in attracting cats to the camera traps across
  temperate, semi-arid and tropical environments. Also, a new audio lure
  is currently being tested as a further trap attractant.
\end{itemize}



\section*{Management implications}
\begin{itemize}
\itemsep1pt\parskip0pt\parsep0pt
\item
  Development of effective baiting methods across climatic regions will
  ultimately provide efficient feral cat control at strategic locations
  across the state and lead to conservation benefits.
\item
  Successful eradication of cats from a number of islands off the
  Western Australian mainland has occurred over the past ten years (i.e.
  Hermite, Faure and Rottnest islands), allowing the persistence of the
  native fauna of the islands and enabling effective reintroductions of
  mammals where necessary. Eradication of cats on Dirk Hartog Island and
  Christmas Island will significantly add to conservation of
  biodiversity.
\end{itemize}



\section*{Future directions}
\begin{itemize}
\itemsep1pt\parskip0pt\parsep0pt
\item
  Continue refinement of bait medium to improve bait consumption by
  feral cats.
\item
  Analyse baiting effectiveness at the various research sites and refine
  the method of operation where necessary to optimise baiting efficacy.
\item
  Further investigation of bait consumption by non-target species and
  devise methods to minimise risk (eg. toxin encapsulation).
\item
  Provide a standard operating procedure for camera trap lures for feral
  cats.
\end{itemize}



%-----------------------------------------------------------------------------%
% Back matter
%\backmatter
\end{document}
%-----------------------------------------------------------------------------%


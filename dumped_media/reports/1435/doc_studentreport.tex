
\documentclass[version=last, paper=a4, DIV=18, usenames, dvipsnames]{scrartcl}
\usepackage{txfonts}
\usepackage{pdflscape}
\usepackage{pdfpages}
\usepackage[english]{babel} % English language/hyphenation
%%% Bootstrap colors
\definecolor{RedFire}{RGB}{146,25,28}
\definecolor{ParksWildlife}{RGB}{0,85,144}
\definecolor{successbg}{RGB}{223,240,216}
\definecolor{errorbg}{RGB}{242,222,222}
\definecolor{warningbg}{RGB}{252,248,227}
\definecolor{infobg}{RGB}{217,237,247}
\definecolor{muted}{RGB}{153,153,153}
\definecolor{success}{RGB}{70,136,71}
\definecolor{error}{RGB}{185,74,72}
\definecolor{warning}{RGB}{192,152,83}
\definecolor{info}{RGB}{58,135,173}

\definecolor{required}{HTML}{D9534F}
\definecolor{denied}{HTML}{D9534F}
\definecolor{granted}{HTML}{47A447}
\definecolor{not required}{RGB}{200, 200, 200}

\usepackage[colorlinks=true,pdftitle=doc\_studentreport.pdf
,linktoc=all,linkcolor=RedFire,urlcolor=ParksWildlife]{hyperref}
\usepackage{colortbl}
\usepackage{longtable}
\usepackage{tabu}
\setlength{\tabulinesep}{1.5mm}
\usepackage{enumerate}
\usepackage{enumitem}
\usepackage{fancyhdr}
\usepackage{lastpage}
\usepackage{graphicx}
\usepackage{eso-pic}
\usepackage{hyphenat}
\renewcommand{\familydefault}{\sfdefault}



\newcommand{\HRule}{\rule{\linewidth}{0.1pt}}

\newcommand{\placetextbox}[3]{% \placetextbox{<horizontal pos>}{<vertical pos>}{<stuff>}
  \setbox0=\hbox{#3}% Put <stuff> in a box
  \AddToShipoutPictureFG*{% Add <stuff> to current page foreground
    \put(\LenToUnit{#1\paperwidth},\LenToUnit{#2\paperheight}){\vtop{{\null}\makebox[0pt][c]{#3}}}%
  }%
}%




%-----------------------------------------------------------------------------%
% Headers and footers
%
\fancypagestyle{plain}{
  \fancyhf{}
  \setlength\headheight{60pt} % push page content below header
  \renewcommand{\headrulewidth}{0.1pt}
  \renewcommand{\footrulewidth}{0.1pt}
  
  
  \fancyhead[L]{ 
    \href{http://sdis.dpaw.wa.gov.au}{
    \includegraphics[scale=0.6]{/mnt/projects/sdis/staticfiles/img/logo-dpaw.png}}
  }
  \fancyhead[R]{ 
      \hfill
      \href{http://sdis.dpaw.wa.gov.au}{Science Directorate Information System} 
      \newline 
      \href{http://sdis.dpaw.wa.gov.au/documents/studentreport/1435/}{Progress Report 2014-9 (FY 2014-2015)} 
  }
  
  
  
  
  \fancyfoot[L]{ \leftmark\newline\textbf{Printed}\textit{ June 24, 2015, 3:52 p.m. }}
  \fancyfoot[R]{  \, \newline Page \thepage\ of \pageref{LastPage} }
  
  
}
\pagestyle{plain}
%
% end Headers
%-----------------------------------------------------------------------------%

\begin{document}

%-----------------------------------------------------------------------------%
% Title page
%

%
% end title page
%-----------------------------------------------------------------------------%




\section*{Progress Report}
Infectious pathogens (e.g., \emph{Trypanosoma} spp.) may play a role the
recent \textgreater{}90\% declines of the woylie; thus, characterising
factors influencing pathogen transmission is a priority and the focus of
this project.

Research will occur in two semi-free-ranging populations near Perth
(high density Karakamia Sanctuary and Whiteman Park, which will shift
from high-to-low density due to enclosure expansion) and in free-ranging
woylie populations in the Upper Warren region. At the \emph{community}
level, validated methods will be applied to identify and evaluate the
prevalence of gastrointestinal and hemopathogens in woylies and
sympatric marsupials. Multi-host (vector-borne, where applicable)
transmission models will be used to evaluate the contribution of various
co-host species to the basic reproductive number (R\textsubscript{0}) of
targeted pathogens. At the \emph{population} level, woylies at Whiteman
Park and Karakamia Sanctuary will be fitted with GPS collars to monitor
movements; then social network analysis will be used to map pathogen
transmission pathways and their relationship with density. These data
will be used to develop networks that reflect potential transmission
pathways for contagious, refuge-based, or environmental pathogens. In
addition, the effects of translocation will be assessed by looking at
members of the donor and recipient woylie populations before, during,
and after translocation. Finally, at the \emph{individual} level,
screening for pathogens while assessing health, reproduction, and
behavioural attributes will allow assessment of risk factors and
potential fitness effects of pathogens in isolation or combination.
Furthermore, network transmission models can facilitate the
identification of behavioural traits (e.g., connectedness) or
demographic factors (e.g., age, sex) key to pathogen propagation.

In this first year of the project, collection of field data has begun,
with successful sampling at all sites and collar deployment at the two
northern sites. Data collection will continue through 2015, as well as
initial laboratory and data analysis.




\clearpage



\end{document}


\documentclass[version=last,
    paper=a4, % paper size
    10pt, % default font size
    usenames,
    dvipsnames,
    oneside, % ONLINE
    headings=openany, % open chapters on odd and even pages
    %toc=chapterentrywithdots, % Table of Contents style
    %BCOR=7mm, % PRINT Binding Correction
    %DIV=13, % typearea 161.54 mm x 228.46 mm, top margin 22.85 mm, inner margin 16.15 mm
    %DIV=14, % 165.00 233.36 21.21 15.00
    DIV=15 % 168.00 237.60 19.80 14.00
]{scrbook}
\usepackage{typearea}
\usepackage[automark,headsepline,footsepline]{scrlayer-scrpage} % Headers and footers

%%
%% Fonts, encoding, spacing, indentation
%%
\usepackage{txfonts}
\renewcommand{\familydefault}{\sfdefault} % Default to Sans Serif font
\usepackage[english]{babel}
\usepackage[T1]{fontenc}
\usepackage[utf8]{inputenc}

% Paragraph spacing
%\usepackage{parskip}    % Paragraph spacing
%\setlength{\parindent}{0em} % Don't indent paragraphs - ONLINE
%\setlength{\parskip}{1.3 ex plus 0.5ex minus 0.3ex} % 1-1.8 ex vertical space between paragraphs - ONLINE

% Spacing of headings
%\RedeclareSectionCommand[afterskip=3pt]{section} % less space after section
%\RedeclareSectionCommand[beforeskip=0cm]{subsection} % less space between HRule and project name
%\RedeclareSectionCommand[afterskip=0.1\baselineskip]{subsubsection} % less space after progressreport subheadings

% Table font size
\usepackage{etoolbox}
\AtBeginEnvironment{longtabu}{\footnotesize}{}{}

%%
%% Tables, columns, layout
%%
\usepackage{multicol}   % 2 col publications
\usepackage{pdflscape}  % Landscape pages
\usepackage{pdfpages}   % Include PDFs
\usepackage{hanging}    % hanging paragraphs for publications
%\usepackage{titletoc}   % Required for manipulating the table of contents
\setcounter{tocdepth}{2} % TOC list down to section
\usepackage{enumerate}  % Enumerations
\usepackage{enumitem}   % Enumerations
\usepackage{longtable}  % Multipage table
\usepackage{tabu}       %
\setlength{\tabulinesep}{1.5mm} % Consistent vertical spacing in tabu

%%
%% Graphics, images, colours
%%
\usepackage{graphicx} % embedded images
\usepackage{eso-pic} %
\usepackage{colortbl} % define custom named colours
\definecolor{RedFire}{RGB}{146,25,28}
\definecolor{ParksWildlife}{RGB}{0,85,144}
\definecolor{successbg}{RGB}{223,240,216}
\definecolor{errorbg}{RGB}{242,222,222}
\definecolor{warningbg}{RGB}{252,248,227}
\definecolor{infobg}{RGB}{217,237,247}
\definecolor{muted}{RGB}{153,153,153}
\definecolor{success}{RGB}{70,136,71}
\definecolor{error}{RGB}{185,74,72}
\definecolor{warning}{RGB}{192,152,83}
\definecolor{info}{RGB}{58,135,173}

\definecolor{required}{RGB}{192,152,83}
\definecolor{requiredbg}{RGB}{252,248,227}
\definecolor{denied}{RGB}{185,74,72}
\definecolor{deniedbg}{RGB}{242,222,222}
\definecolor{granted}{RGB}{70,136,71}
\definecolor{grantedbg}{RGB}{223,240,216}
\definecolor{not reqiured}{RGB}{153,153,153}
\definecolor{not requiredbg}{RGB}{255,255,255}

\usepackage{tikz} % Drawing
\usetikzlibrary{arrows,shapes,positioning,shadows,trees}

%%
%% Links, URLs
%%
\usepackage[
    linktoc=all,
    %colorlinks=false,  %PRINT
    colorlinks=true, % ONLINE
    linkcolor=RedFire, % ONLINE
    urlcolor=ParksWildlife, % ONLINE
    pdftitle=Progress Report SP 2006-002 (FY 2014-2015)
]{hyperref}

% Black magic to linebreak URLs
\usepackage{url}
\makeatletter
\g@addto@macro{\UrlBreaks}{\UrlOrds}
\makeatother

%%
%% Custom macros
%%
% Thick Horizontal rule
\newcommand{\HRule}{\vspace{8mm}\\\noindent\rule{\linewidth}{0.1pt}}

% Custom Tikz node for SDS diagram
\newcommand\mynode[6][]{
    \node[#1] (#2){
        \parbox{#3\relax}{
            \begin{center}
            \textbf{#4}\\
            #5\\
            \footnotesize{#6}
            \end{center}}};}



\usepackage[automark,headsepline,footsepline,plainfootsepline]{scrlayer-scrpage}
\automark*[section]{}
\addtokomafont{pageheadfoot}{\normalfont\footnotesize\sffamily} % Don't italicise
\renewcommand{\chaptermark}[1]{\markleft{#1}{}}     % Chapter: suppress numbering
\renewcommand{\sectionmark}[1]{\markright{#1}{}}    % Section: suppress numbering

% Header (inner, center, outer)
\ihead{\href{http://sdis.dpaw.wa.gov.au/documents/progressreport/1395/}{Progress Report SP 2006-002 (FY 2014-2015)}}
%\chead{\href{http://sdis.dpaw.wa.gov.au}{Science Directorate Information System}}
\ohead{\href{https://www.dpaw.wa.gov.au/about-us/science-and-research}{\includegraphics[height=6mm, keepaspectratio]{/mnt/projects/sdis/staticfiles/img/logo-dpaw.png}}}

% Footer (inner, center, outer)
\ifoot{\textbf{Printed}~Fri, 2 Jun 2017 10:48:44 +0800} % inner/left footer
\cfoot{}
\ofoot[\bfseries\thepage]{\bfseries\thepage}        % Page number (also [plain])


\pagestyle{scrheadings}
\setkomafont{pageheadfoot}{\normalfont}

%-----------------------------------------------------------------------------%
\begin{document}
\raggedbottom

%-----------------------------------------------------------------------------%
% Title page
\subject{Progress Report SP 2006-002
}
\title{Monitoring stream biodiversity (KPI 20 of the Forest Management Plan)
}
\subtitle{Wetlands Conservation
}
\author{}
\publishers{\small
    \subsection*{Project Core Team}
\begin{tabu} {X X}
\textbf{Supervising Scientist} & Melita Pennifold
\\
\textbf{Data Custodian} & 
\\
\textbf{Site Custodian} & 
\\
\end{tabu}


    \subsection*{Project status as of June 2, 2017, 10:48 a.m.}
\begin{tabu} {X X}
& Final update requested
\\
\end{tabu}

    
\subsection*{Document endorsements and approvals as of June 2, 2017, 10:48 a.m.}
\begin{tabu} {X X}

%\rowcolor{grantedbg}
    \textbf{Project Team} & 
    \textcolor{granted}{ granted}\\

%\rowcolor{grantedbg}
    \textbf{Program Leader} & 
    \textcolor{granted}{ granted}\\

%\rowcolor{grantedbg}
    \textbf{Directorate} & 
    \textcolor{granted}{ granted}\\

\end{tabu}



}
\uppertitleback{}
\lowertitleback{}
\date{}

%-----------------------------------------------------------------------------%
% Front matter
\frontmatter
\maketitle
%-----------------------------------------------------------------------------%
% Main matter
\mainmatter

\section*{Monitoring stream biodiversity (KPI 20 of the Forest Management Plan)
}

A Pinder, M Pennifold



\section*{Context}

Key performance indicator 20 of the Forest Management Plan 2004-2013 was
the percentage of water bodies with significant variance of biodiversity
from the historic range of variability. This was addressed by monitoring
invertebrates in representative stream sites in the south-west forests,
particularly in relation to forest management practices.

With the release of Forest Management Plan 2014-2023 a new project is
being developed to address:

\begin{itemize}
\itemsep1pt\parskip0pt\parsep0pt
\item
  KPI1 - "\emph{Measurement and analysis of changes in spatial extent of
  healthy ecosystems and spatial extent of lower condition ecosystems
  from a current state}.''
\item
  KPI3~- ``\emph{Measurement and analysis of changes in spatial extent,
  vegetation condition, fauna communities and water quality of the
  (Ramsar) wetlands as a function of time and as defined by the relevant
  regional nature conservation plans}''.
\end{itemize}

~This project will address these KPIs by:

\begin{itemize}
\item
  Continuing to monitor a subset of higher condition streams from the
  KPI20 project to determine responses to declining rainfall and forest
  management.
\item
  Re-surveying aquatic invertebrates in suites of important~Warren
  Region wetlands (including the Muir-Byenup Ramsar wetlands) to
  quantify responses to changes in hydrology, water chemistry and other
  threats such as fire, over the last 10 to 20 years.
\end{itemize}




\section*{Aims}

\begin{itemize}
\itemsep1pt\parskip0pt\parsep0pt
\item
  Assess the condition of representative south-west forest streams
  (including in relation to forest management practices) by comparing
  the richness of aquatic invertebrates to that predicted by a
  previously constructed model (AusRivAS) developed using data from
  `minimally disturbed' reference sites.
\item
  Assess changes in invertebrate communities in key Warren Region
  wetlands (starting with the Muir-Byenup Ramsar wetlands) as a result
  of altered hydrology over the last decade and provide advice to
  regions on where to focus management activities.
\end{itemize}




\section*{Progress}

\begin{itemize}
\itemsep1pt\parskip0pt\parsep0pt
\item
  ~A journal article is being prepared in collaboration with scientists
  from CSIRO in Canberra: ``Aquatic bioregionalisation derived from
  generalised dissimilarity models of compositional patterns in aquatic
  invertebrate fauna: an example from southwest Western Australia.''
\item
  Sampled aquatic invertebrates in Muir-Byenup wetlands in spring 2014
  and summer 2015 and commenced processing these.
\end{itemize}




\section*{Management implications}

\begin{itemize}
\itemsep1pt\parskip0pt\parsep0pt
\item
  On the whole, there was no evidence that current forest management
  practices were having a significant effect on stream biodiversity,
  probably due, in part, to the practice of leaving unharvested buffers
  around streams.
\item
  Stream sites with greatest divergence in diversity from reference
  condition were generally in the drier parts of the northern and
  eastern jarrah forest or were naturally saline or acidic. Part of the
  reason for these sites being apparently impaired was that the AusRivAS
  models were produced with few reference sites in such streams, so the
  model is likely to have overestimated richness.
\item
  A few stream sites were not in these categories and require further
  monitoring and investigation to examine the cause of the reduced
  diversity.
\item
  The new work will allow the Warren Region to prioritise conservation
  efforts within the Muir-Byenup Ramsar wetlands by identifying those
  with lesser or greater resilience to change and those in higher or
  lower condition.
\end{itemize}




\section*{Future directions}

\begin{itemize}
\itemsep1pt\parskip0pt\parsep0pt
\item
  Re-sample streams in 2015, with a focus on those considered to be in
  minimally disturbed catchments, to provide long-term data on the
  response of aquatic invertebrate communities to declining rainfall
  (addresses KPI1 of the 2014-2023 FMP)
\item
  Continue to update fire and logging history for catchment areas.
\item
  Publish further papers examining impacts of declining rainfall and
  forest management practices on macroinvertebrate diversity in forest
  streams.
\item
  Publish report with summaries of~10 year trends (2005 to 2015) for all
  stream monitoring sites.
\item
  Develop new SPP to address KPI1 and KPI3 of the 2014-2023 FMP.
\item
  Identify Muir-Byenup invertebrates collected in 2014/2015.
\item
  Re-survey nationally important wetlands from south~west previously
  sampled by Horwtiz in 1997~(e.g.Owingup, Lake Jasper, Doggerup,
  Marringup, Mt Soho Swamp) and identified as priorities in the Warren
  Region Nature Conservation Plan.
\end{itemize}



%-----------------------------------------------------------------------------%
% Back matter
%\backmatter
\end{document}
%-----------------------------------------------------------------------------%

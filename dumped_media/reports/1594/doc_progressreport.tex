
\documentclass[version=last,
    paper=a4, % paper size
    10pt, % default font size
    usenames,
    dvipsnames,
    oneside, % ONLINE
    headings=openany, % open chapters on odd and even pages
    %toc=chapterentrywithdots, % Table of Contents style
    %BCOR=7mm, % PRINT Binding Correction
    %DIV=13, % typearea 161.54 mm x 228.46 mm, top margin 22.85 mm, inner margin 16.15 mm
    %DIV=14, % 165.00 233.36 21.21 15.00
    DIV=15 % 168.00 237.60 19.80 14.00
]{scrbook}
\usepackage{typearea}
\usepackage[automark,headsepline,footsepline]{scrlayer-scrpage} % Headers and footers

%%
%% Fonts, encoding, spacing, indentation
%%
\usepackage{txfonts}
\renewcommand{\familydefault}{\sfdefault} % Default to Sans Serif font
\usepackage[english]{babel}
\usepackage[T1]{fontenc}
\usepackage[utf8]{inputenc}

% Paragraph spacing
%\usepackage{parskip}    % Paragraph spacing
%\setlength{\parindent}{0em} % Don't indent paragraphs - ONLINE
%\setlength{\parskip}{1.3 ex plus 0.5ex minus 0.3ex} % 1-1.8 ex vertical space between paragraphs - ONLINE

% Spacing of headings
%\RedeclareSectionCommand[afterskip=3pt]{section} % less space after section
%\RedeclareSectionCommand[beforeskip=0cm]{subsection} % less space between HRule and project name
%\RedeclareSectionCommand[afterskip=0.1\baselineskip]{subsubsection} % less space after progressreport subheadings

% Table font size
\usepackage{etoolbox}
\AtBeginEnvironment{longtabu}{\footnotesize}{}{}

%%
%% Tables, columns, layout
%%
\usepackage{multicol}   % 2 col publications
\usepackage{pdflscape}  % Landscape pages
\usepackage{pdfpages}   % Include PDFs
\usepackage{hanging}    % hanging paragraphs for publications
%\usepackage{titletoc}   % Required for manipulating the table of contents
\setcounter{tocdepth}{2} % TOC list down to section
\usepackage{enumerate}  % Enumerations
\usepackage{enumitem}   % Enumerations
\usepackage{longtable}  % Multipage table
\usepackage{tabu}       %
\setlength{\tabulinesep}{1.5mm} % Consistent vertical spacing in tabu

%%
%% Graphics, images, colours
%%
\usepackage{graphicx} % embedded images
\usepackage{eso-pic} %
\usepackage{colortbl} % define custom named colours
\definecolor{RedFire}{RGB}{146,25,28}
\definecolor{ParksWildlife}{RGB}{0,85,144}
\definecolor{successbg}{RGB}{223,240,216}
\definecolor{errorbg}{RGB}{242,222,222}
\definecolor{warningbg}{RGB}{252,248,227}
\definecolor{infobg}{RGB}{217,237,247}
\definecolor{muted}{RGB}{153,153,153}
\definecolor{success}{RGB}{70,136,71}
\definecolor{error}{RGB}{185,74,72}
\definecolor{warning}{RGB}{192,152,83}
\definecolor{info}{RGB}{58,135,173}

\definecolor{required}{RGB}{192,152,83}
\definecolor{requiredbg}{RGB}{252,248,227}
\definecolor{denied}{RGB}{185,74,72}
\definecolor{deniedbg}{RGB}{242,222,222}
\definecolor{granted}{RGB}{70,136,71}
\definecolor{grantedbg}{RGB}{223,240,216}
\definecolor{not reqiured}{RGB}{153,153,153}
\definecolor{not requiredbg}{RGB}{255,255,255}

\usepackage{tikz} % Drawing
\usetikzlibrary{arrows,shapes,positioning,shadows,trees}

%%
%% Links, URLs
%%
\usepackage[
    linktoc=all,
    %colorlinks=false,  %PRINT
    colorlinks=true, % ONLINE
    linkcolor=RedFire, % ONLINE
    urlcolor=ParksWildlife, % ONLINE
    pdftitle=Progress Report SP 2015-001 (FY 2015-2016)
]{hyperref}

% Black magic to linebreak URLs
\usepackage{url}
\makeatletter
\g@addto@macro{\UrlBreaks}{\UrlOrds}
\makeatother

%%
%% Custom macros
%%
% Thick Horizontal rule
\newcommand{\HRule}{\vspace{8mm}\\\noindent\rule{\linewidth}{0.1pt}}

% Custom Tikz node for SDS diagram
\newcommand\mynode[6][]{
    \node[#1] (#2){
        \parbox{#3\relax}{
            \begin{center}
            \textbf{#4}\\
            #5\\
            \footnotesize{#6}
            \end{center}}};}



%-----------------------------------------------------------------------------%
% Headers and Footers
\automark{section}
\ohead{\href{http://sdis.dpaw.wa.gov.au/documents/progressreport/1594/}{Progress Report SP 2015-001
}}
\chead{\href{http://sdis.dpaw.wa.gov.au}{SDIS}} % center header ONLINE
\ihead{\href{http://sdis.dpaw.wa.gov.au}{
        \includegraphics[scale=0.4]{/mnt/projects/sdis/staticfiles/img/logo-dpaw.png}}}
\ifoot{\textbf{Printed}~Tue, 5 Jul 2016 15:19:06 +0800} % inner/left footer
\cfoot{} % center footer
\ofoot{\pagemark} % outer/right footer
\pagestyle{scrheadings}
\setkomafont{pageheadfoot}{\normalfont}

%-----------------------------------------------------------------------------%
\begin{document}
\raggedbottom

%-----------------------------------------------------------------------------%
% Title page
\subject{Progress Report SP 2015-001
}
\title{Advancing the hydrological understanding of key Wheatbelt catchments and
wetlands to inform adaptive management
}
\subtitle{Wetlands Conservation
}
\author{}
\publishers{\small
    \subsection*{Project Core Team}
\begin{tabu} {X X}
\textbf{Supervising Scientist} & Jasmine Rutherford
\\
\textbf{Data Custodian} & Jasmine Rutherford
\\
\textbf{Site Custodian} & Jasmine Rutherford
\\
\end{tabu}


    \subsection*{Project status as of July 5, 2016, 3:19 p.m.}
\begin{tabu} {X X}
& Approved and active
\\
\end{tabu}

    
\subsection*{Document endorsements and approvals as of July 5, 2016, 3:19 p.m.}
\begin{tabu} {X X}

%\rowcolor{grantedbg}
    \textbf{Project Team} & 
    \textcolor{granted}{ granted}\\

%\rowcolor{grantedbg}
    \textbf{Program Leader} & 
    \textcolor{granted}{ granted}\\

%\rowcolor{grantedbg}
    \textbf{Directorate} & 
    \textcolor{granted}{ granted}\\

\end{tabu}



}
\uppertitleback{}
\lowertitleback{}
\date{}

%-----------------------------------------------------------------------------%
% Front matter
\frontmatter
\maketitle
%-----------------------------------------------------------------------------%
% Main matter
\mainmatter

\section*{Advancing the hydrological understanding of key Wheatbelt catchments and
wetlands to inform adaptive management
}

J Rutherford, L Bourke


\section*{Context}
Changes in the hydrology of Toolibin Lake and the Lake Bryde catchments,
due to land clearing, has resulted in these previously ephemeral fresh
water wetlands developing a connection with deeper, saline groundwater
and becoming degraded. A decline in average rainfall since the 1970s has
seen a further decrease in wetland health as surface water flows and
wetland hydroperiods decrease in quantity and quality. Robust management
decisions require the main hydrological driver(s) of change to be
identified and spatial and temporal fluxes (water and solutes) to be
characterised. This project will significantly advance hydrological
studies at Toolibin Lake and Lake Bryde by making full use of the data
collection and analyses undertaken to date to produce practical tools
for answering the key hydrological management questions.



\section*{Aims}
\begin{itemize}
\itemsep1pt\parskip0pt\parsep0pt
\item
  To produce quantitative conceptual hydrogeological model(s) for
  Toolibin Lake and Lake Bryde;
\item
  To produce a numerical groundwater model to assess the Toolibin Lake
  water balance and determine the effectiveness of groundwater pumping
  (individual pumps) in returning the lake to a perched status;
\item
  To~ evaluate catchment water and salt hydrodynamics (groundwater and
  surface water contributions/fluxes) tested using numerical modeling
  under different climate regimes (Toolibin Lake);
\item
  To investigate the links between key ecological parameters (eg, tree
  and understory health, bird breeding, richness of aquatic
  invertebrates) and hydrological status (Toolibin Lake); and
\item
  To produce risk assessment framework(s) to prioritise conservation
  actions and assess the transferability of research outcomes.
\end{itemize}



\section*{Progress}
\begin{itemize}
\itemsep1pt\parskip0pt\parsep0pt
\item
  Uploaded over 200 hydrological datasets to the Parks and Wildlife Data
  Catalogue.
\item
  Collected borehole geophysical data at Lake Bryde to assign aquifers
  to key bores and report on the spatial distribution of sediments as
  well as the quality of groundwater and soil water.
\item
  Produced scientific reports detailing the achievements of the last 20+
  years of hydrological investigations and monitoring.
\item
  Developed rationalised, long-term hydrological monitoring programs for
  key wetland assets.
\item
  Facilitated a handover of hydrological data collection and management
  to the Wheatbelt Region.
\item
  At Toolibin Lake interpreted quality assured data to construct a
  quantitative conceptual hydrogeological model that is being tested
  numerically to assess the effectiveness of on-ground management
  actions.
\end{itemize}



\section*{Management implications}
\begin{itemize}
\itemsep1pt\parskip0pt\parsep0pt
\item
  Proposed activities, including numerical modelling, will provide a
  much firmer hydrogeological understanding of the threats to high
  conservation value assets associated within the conservation estate in
  catchments of interest to the Department.
\item
  New hydrological informing tools will be available to~ managers to
  help them~make decisions about how best to manage ephemeral wetlands,
  including maintaining, replacing or redesigning existing hydrology
  engineering infrastructure and species selection for re-vegetation
  programs.
\item
  Archiving of Natural Diversity Recovery Catchment data in the Parks
  and Wildlife Data Catalogue will ensure that maximum value can be made
  of this high value resource into the future.
\end{itemize}



\section*{Future directions}
\begin{itemize}
\itemsep1pt\parskip0pt\parsep0pt
\item
  Develop a spatio-temporal conceptual model of deep-rooted vegetation
  resilience on Toolibin Lake and assess hydrological critical criteria
  that influence the success and decline of different plant species.
\item
  Construct a solute transport numerical model to simulate changes in
  soil and aquifer water quality and examine results against
  quantitative conceptual models of changing soil condition and
  vegetation species resilience (Toolibin Lake).
\item
  Archive Toolibin Lake hydrological modelling database to the Parks and
  Wildlife Data Catalogue.
\item
  Complete remaining five reports, summarised as a Science Information
  Sheets, over period 2015-2018.
\end{itemize}



%-----------------------------------------------------------------------------%
% Back matter
%\backmatter
\end{document}
%-----------------------------------------------------------------------------%


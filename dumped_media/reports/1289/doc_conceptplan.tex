
\documentclass[version=last,
    paper=a4,                               % paper size
    10pt,                                   % default font size
    dvipsnames,
    % twoside,                                % PRINT Binding Correction
    oneside,                              % ONLINE
    headings=openany,                       % open chapters on odd and even pages
    open=any,
    BCOR=7mm,                               % PRINT Binding Correction
    %DIV=13,    % typearea 161.54mm x 228.46mm, top 22.85mm, inner 16.15mm
    %DIV=14,    % 165.00 233.36 21.21 15.00
    DIV=15,     % 168.00 237.60 19.80 14.00
    % toc=chapterentrywithdots              % Table of Contents style
]{scrbook}
\usepackage{typearea}


%------------------------------------------------------------------------------%
% Headers and footers
%------------------------------------------------------------------------------%
\usepackage[automark,headsepline,footsepline,plainfootsepline]{scrlayer-scrpage}
\automark*[section]{}
\addtokomafont{pageheadfoot}{\normalfont\footnotesize\sffamily} % Don't italicise
\renewcommand{\chaptermark}[1]{\markleft{#1}{}}     % Chapter: suppress numbering
\renewcommand{\sectionmark}[1]{\markright{#1}{}}    % Section: suppress numbering

% Header (inner, center, outer)
% \ihead{\href{http://sdis.dpaw.wa.gov.au}{\textbf{Concept Plan SP 2014-003}}}
%\chead{\href{http://sdis.dpaw.wa.gov.au}{Science Directorate Information System}}
% \ohead{\href{https://www.dpaw.wa.gov.au/about-us/science-and-research}{\includegraphics[height=8mm, keepaspectratio]{/mnt/projects/sdis/staticfiles/img/logo-dpaw.png}}}

% Footer (inner, center, outer)
% \ifoot{\RaggedRight\leftmark}                       % Chapter
% \cfoot{\RaggedLeft\rightmark}                       % Section
% \ofoot[\bfseries\thepage]{\bfseries\thepage}        % Page number (also [plain])


%------------------------------------------------------------------------------%
% Fonts, encoding
%------------------------------------------------------------------------------%
%\usepackage{avant}             % Use the Avantgarde font for headings
\usepackage{txfonts}
\usepackage{mathptmx}
\usepackage{gensymb}            % provides \textdegree
\renewcommand{\familydefault}{\sfdefault} % Default to Sans Serif font
\usepackage{microtype}          % Slightly tweak font spacing for aesthetics
\usepackage[english]{babel}
\usepackage[utf8]{inputenc}  % Allow letters with accents
\usepackage[utf8]{luainputenc}  % Allow letters with accents
\usepackage[T1]{fontenc}        % Use 8-bit encoding that has 256 glyphs
\usepackage{textcomp}
\usepackage[explicit]{titlesec}           % Customise of titles
%\DeclareUnicodeCharacter{0080}{\textregistered}
\DeclareUnicodeCharacter{00B0}{\textdegree}

%------------------------------------------------------------------------------%
% Tables, columns, layout
%------------------------------------------------------------------------------%
\usepackage{etoolbox}
\AtBeginEnvironment{longtabu}{\footnotesize}{}{}  % Table font size
\usepackage{booktabs}           % Required for nicer horizontal rules in tables
\usepackage{multicol}           % 2 col publications
\usepackage{pdflscape}          % Landscape pages
\usepackage{pdfpages}           % Include PDFs
\usepackage{hanging}            % hanging paragraphs for publications
%\usepackage{titletoc}          % Manipulate the table of contents
\setcounter{tocdepth}{2}        % TOC list down to section
\usepackage{enumerate}          % Enumerations
\usepackage{enumitem}           % Enumerations
\usepackage{longtable}          % Multipage table
\usepackage{tabu}               %
\setlength{\tabulinesep}{1.5mm} % Consistent vertical spacing in tabu
\newcommand{\HRule}{\vspace{8mm}\noindent\rule{\linewidth}{0.1pt}}
\usepackage[export]{adjustbox}  % minipage, image frame


%------------------------------------------------------------------------------%
% Graphics, images, colours
%------------------------------------------------------------------------------%
\usepackage{graphicx} % embedded images
\usepackage{wrapfig}  % wrap figures in text
\usepackage{caption}  % allow unnumbered captions
\usepackage{eso-pic} % Required for specifying an image background in the title page
\usepackage{colortbl} % define custom named colours
\usepackage{xstring} % Conditionals
\usepackage{transparent} % Allow transparent images

\definecolor{RedFire}{RGB}{146,25,28}
% Following PICA branding guidelines
% https://dpaw.sharepoint.com/Divisions/pica/Documents/Branding%20guidelines.pdf
\definecolor{dpawblue}{RGB}{35,97,146}          % Pantone 647
\definecolor{dpaworange}{RGB}{237,139,0}        % Pantone 144
\definecolor{dpawgreen}{RGB}{116,170,80}        % Pantone 7489
\definecolor{dpawred}{RGB}{124,46,44}           % Paul's suggestion

% bootstrap colours
\definecolor{successbg}{RGB}{223,240,216}
\definecolor{errorbg}{RGB}{242,222,222}
\definecolor{warningbg}{RGB}{252,248,227}
\definecolor{infobg}{RGB}{217,237,247}
\definecolor{muted}{RGB}{153,153,153}
\definecolor{success}{RGB}{70,136,71}
\definecolor{error}{RGB}{185,74,72}
\definecolor{warning}{RGB}{192,152,83}
\definecolor{info}{RGB}{58,135,173}

% SDIS approval colours
\definecolor{required}{RGB}{192,152,83}
\definecolor{requiredbg}{RGB}{252,248,227}
\definecolor{denied}{RGB}{185,74,72}
\definecolor{deniedbg}{RGB}{242,222,222}
\definecolor{granted}{RGB}{70,136,71}
\definecolor{grantedbg}{RGB}{223,240,216}
\definecolor{notrequired}{RGB}{153,153,153}
\definecolor{notrequiredbg}{RGB}{255,255,255}

\usepackage{tikz} % Drawing
\usetikzlibrary{arrows,shapes,positioning,shadows,trees}


%------------------------------------------------------------------------------%
% Hyperlinks
%------------------------------------------------------------------------------%
\usepackage[open=true]{bookmark}
\usepackage{nameref}
\usepackage{ifxetex,ifluatex}
\ifxetex
  \usepackage[
    setpagesize=false,        % page size defined by xetex
    unicode=false,            % unicode breaks when used with xetex
    xetex]{hyperref}
\else
  \usepackage[unicode=true]{hyperref}
\fi

\hypersetup{
  backref=true,
  pagebackref=true,
  hyperindex=true,
  breaklinks=true,
  urlcolor=dpawblue,
  bookmarks=true,
  bookmarksopen=false,
  pdfauthor={Science and Conservation Division, Dept Parks and Wildlife, WA},
  pdftitle=Concept Plan SP 2014-003
,
  colorlinks=true,
  linkcolor=dpawblue,
  pdfborder={0 0 0}}

\urlstyle{same}                         % don't use monospace font for urlstyle


%------------------------------------------------------------------------------%
% Black magic to linebreak URLs
%------------------------------------------------------------------------------%
\usepackage{url}
\makeatletter\g@addto@macro{\UrlBreaks}{\UrlOrds}\makeatother
\Urlmuskip=0mu plus 1mu


%------------------------------------------------------------------------------%
% Fix latex errors
%------------------------------------------------------------------------------%
\providecommand{\tightlist}{\setlength{\itemsep}{0pt}\setlength{\parskip}{0pt}}

% copy-pasted HTML <span> in SDIS fields becomes \text{} in tex source
\providecommand{\text}{}


%------------------------------------------------------------------------------%
% Custom Tikz node for SDS diagram
%------------------------------------------------------------------------------%
\newcommand\mynode[6][]{
  \node[#1] (#2){
    \parbox{#3\relax}{
      \begin{center}
      \textbf{#4}\\
      #5\\
      \footnotesize{#6}
      \end{center}
    }};}


%------------------------------------------------------------------------------%
% Custom no-pagebreaks-environment
%------------------------------------------------------------------------------%
\newenvironment{absolutelynopagebreak}
  {\par\nobreak\vfil\penalty0\vfilneg\vtop\bgroup}
  {\par\xdef\tpd{\the\prevdepth}\egroup\prevdepth=\tpd}


%------------------------------------------------------------------------------%
% Remove the header from odd empty pages at the end of chapters
%------------------------------------------------------------------------------%
\makeatletter
\renewcommand{\cleardoublepage}{
\clearpage\ifodd\c@page\else
\hbox{}
\vspace*{\fill}
\thispagestyle{empty}
\newpage
\fi}


%----------------------------------------------------------------------------------------
%  Page flow control
%----------------------------------------------------------------------------------------
%\widowpenalty=10000
%\clubpenalty=10000
%\vbadness=1200
%\hbadness=11000


%----------------------------------------------------------------------------------------
%   CHAPTER HEADINGS
%----------------------------------------------------------------------------------------
\newcommand{\thechapterimage}{}
\newcommand{\chapterimage}[1]{\renewcommand{\thechapterimage}{#1}}

% Numbered chapters with mini tableofcontents
\def\thechapter{\arabic{chapter}}
\def\@makechapterhead#1{
%\thispagestyle{plain}
{\centering \normalfont\sffamily
\ifnum \c@secnumdepth >\m@ne
\if@mainmatter
\startcontents
\begin{tikzpicture}[remember picture,overlay]
\node at (current page.north west)
{\begin{tikzpicture}[remember picture,overlay]
\node[anchor=north west,inner sep=0pt] at (0,0) {
\includegraphics[width=\paperwidth,height=0.5\paperwidth]{\thechapterimage}};
%------------------------------------------------------------------------------%
% Small contents box in the chapter heading
% Mini TOC background box
%\fill[color=dpawblue!10!white,opacity=.2] (1cm,0) rectangle (
%  3.5cm, % Mini TOC box width
%  -3.5cm % Mini TOC box height
%);
% Mini TOC text content
%\node[anchor=north west] at (1.1cm,.35cm) {
%  \parbox[t][8cm][t]{6.5cm}{
%    \huge\bfseries\flushleft
%    \printcontents{l}{1}{
%    \setcounter{tocdepth}{1}                   % Mini TOC level depth
%    }
% }
%};
%------------------------------------------------------------------------------%
% Chapter heading
\draw[anchor=west] (5cm,-9cm) node [
rounded corners=20pt,
fill=dpawblue!10!white,
text opacity=1,
draw=dpawblue,
draw opacity=1,
line width=1.5pt,
fill opacity=.2,
inner sep=12pt]{
    \huge\sffamily\bfseries\textcolor{black}{
      \thechapter. #1\strut\makebox[22cm]{}
    }
};
\end{tikzpicture}};
\end{tikzpicture}}
\par\vspace*{240\p@}                            % Push text below chapter image
\fi
\fi}

%------------------------------------------------------------------------------%
% Unnumbered chapters without mini tableofcontents
%------------------------------------------------------------------------------%
\def\@makeschapterhead#1{
%\thispagestyle{plain}
{\centering \normalfont\sffamily
\ifnum \c@secnumdepth >\m@ne
\if@mainmatter
\begin{tikzpicture}[remember picture,overlay]
\node at (current page.north west)
{\begin{tikzpicture}[remember picture,overlay]
\node[anchor=north west,inner sep=0pt] at (0,0) {
  \includegraphics[width=\paperwidth,height=0.5\paperwidth]{\thechapterimage}};
% Mini TOC background box
%\fill[color=dpawblue!10!white,opacity=.2] (1cm,0) rectangle (
%  3.5cm,                                       % Mini TOC box width
%  -3.5cm                                       % Mini TOC box height
%);
% Mini TOC text content
%\node[anchor=north west] at (1.1cm,.35cm) {
%  \parbox[t][8cm][t]{6.5cm}{
%    \huge\bfseries\flushleft
%    \printcontents{l}{1}{
%    \setcounter{tocdepth}{1} % Mini TOC level depth
%    }
%}
%};
\draw[anchor=west] (5cm,-9cm) node [rounded corners=20pt,
  fill=dpawblue!10!white,fill opacity=.6,inner sep=12pt,text opacity=1,
  draw=dpawblue,draw opacity=1,line width=1.5pt]{
  \huge\sffamily\bfseries\textcolor{black}{#1\strut\makebox[22cm]{}}};
\end{tikzpicture}};
\end{tikzpicture}}
\par\vspace*{240\p@}
\fi
\fi
}
\makeatother



\usepackage[automark,headsepline,footsepline,plainfootsepline]{scrlayer-scrpage}
\automark*[section]{}
\addtokomafont{pageheadfoot}{\normalfont\footnotesize\sffamily} % Don't italicise
\renewcommand{\chaptermark}[1]{\markleft{#1}{}}     % Chapter: suppress numbering
\renewcommand{\sectionmark}[1]{\markright{#1}{}}    % Section: suppress numbering

% Header (inner, center, outer)
\ihead{\href{http://sdis.dpaw.wa.gov.au/documents/conceptplan/1289/}{Concept Plan SP 2014-003}}
%\chead{\href{http://sdis.dpaw.wa.gov.au}{Science Directorate Information System}}
\ohead{\href{https://www.dpaw.wa.gov.au/about-us/science-and-research}{\includegraphics[height=6mm, keepaspectratio]{/mnt/projects/sdis/staticfiles/img/logo-dpaw.png}}}

% Footer (inner, center, outer)
\ifoot{\textbf{Printed}~Tue, 12 Dec 2017 16:44:17 +0800} % inner/left footer
\cfoot{}
\ofoot[\bfseries\thepage]{\bfseries\thepage}        % Page number (also [plain])


\pagestyle{scrheadings}
\setkomafont{pageheadfoot}{\normalfont}

%-----------------------------------------------------------------------------%
\begin{document}
\raggedbottom

%-----------------------------------------------------------------------------%
% Title page
\subject{Concept Plan SP 2014-003
}
\title{Cat Eradication on Dirk Hartog Island
}
\subtitle{Animal Science
}
\author{}
\publishers{\small
    \subsection*{Project Core Team}
\begin{tabu} {X X}
\textbf{Supervising Scientist} & Dave Algar
\\
\textbf{Data Custodian} & M Johnston
\\
\textbf{Site Custodian} & Dave Algar
\\
\end{tabu}


    \subsection*{Project status as of Dec. 12, 2017, 4:44 p.m.}
\begin{tabu} {X X}
& Approved and active
\\
\end{tabu}

    
\subsection*{Document endorsements and approvals as of Dec. 12, 2017, 4:44 p.m.}
\begin{tabu} {X X}

%\rowcolor{grantedbg}
    \textbf{Project Team} & 
    \textcolor{granted}{ granted}\\

%\rowcolor{grantedbg}
    \textbf{Program Leader} & 
    \textcolor{granted}{ granted}\\

%\rowcolor{grantedbg}
    \textbf{Directorate} & 
    \textcolor{granted}{ granted}\\

\end{tabu}



}
\uppertitleback{}
\lowertitleback{}
\date{}

%-----------------------------------------------------------------------------%
% Front matter
\frontmatter
\maketitle
%-----------------------------------------------------------------------------%
% Main matter
\mainmatter


\section*{Cat Eradication on Dirk Hartog Island
}



\subsection*{Science and Conservation Division Program}

Animal Science




\subsection*{Parks and Wildlife Service}

Service 2: Conserving Habitats, Species and Ecological Communities




\subsection*{Aims}

Control of feral cats is recognised as one of the most important fauna
conservation issues in Australia today and as a result, a national
`Threat Abatement Plan (TAP) for Predation by Feral Cats' has been
developed. The TAP seeks to protect affected native species and
ecological communities, and to prevent further species and ecological
communities from becoming threatened. In particular, the first objective
of the TAP is to:~

\begin{itemize}
\itemsep1pt\parskip0pt\parsep0pt
\item
  Prevent feral cats from occupying new areas in Australia and eradicate
  feral cats from high-conservation-value `islands'
\end{itemize}

The eradication of feral cats proposed for Dirk Hartog Island follows a
prescribed course of action (phases) used elsewhere in successful
eradication campaigns. Their first phase involves a succession of
removal events that reduce the pest population to such low levels that
further efforts often do not find and remove any more animals. Their
second phase attempts to validate or assess whether in fact this lack of
detection means eradication may have been achieved. Assuming no more
pests are found, their third phase is one of surveillance to confirm the
assessment and it may continue until a decision is made to stop and
declare the eradication a success. Detecting survivors and interpreting
the lack of such detections to set stop rules are critical elements of
this strategy. Their process collects spatially explicit data on the
numbers of animals removed and on the effort to do this as it proceeds.

The size of Dirk Hartog Island, in particular its length, pose
logistical constraints on conducting an eradication campaign across the
entire island simultaneously. It is not practical or feasible to monitor
for cat activity over such a large area and as such, the eradication
campaign will be conducted in stages. Each of these stages is outlined
briefly below.

Stage 1 (January-April 2014) will be dedicated to establishment of
infrastructure in the southern section (Herald Bay) including
accommodation and equipment storage, installation of the southern
monitoring track network and construction of the barrier fence.
Infrastructure construction will be transportable to provide flexibility
in its use and options for utilization elsewhere at the completion of
the project. Time restrictions due to delays in delivery of the barge
have meant that only infrastructure south of the barrier fence can be
established within this time period. Infrastructure north of the fence
(Sandy Bay accommodation site) and installation of the northern
monitoring track network will need to be established when time permits
later in 2014.

Stage 2 (May/June 2014-May/June 2015) {[}Phase 1{]} a baiting campaign
will be conducted May/June 2014 south of the cat barrier fence, an area
of approximately 220km\textsuperscript{2}. An intensive monitoring
program will be adopted following the baiting campaign to locate any cat
activity. Where warranted, ground-baiting and trapping will be
implemented to remove any cats that remain. {[}Phase 2{]} At the
completion of the monitoring/trapping program, a team of detector dogs
and their handlers will be contracted to independently verify
eradication.

Stage 3(May/June 2015-May/June 2016) {[}Phase 1{]} a baiting campaign
will be conducted May/June 2015 north of the cat barrier fence, an area
of approximately 420km\textsuperscript{2}. As above, an intensive
monitoring program will be adopted following the baiting campaign to
locate any cat activity. Where warranted, ground-baiting and trapping
will be implemented to remove any cats that remain. {[}Phase 2{]} At the
completion of the monitoring/trapping program, a team of detector dogs
and their handlers will be contracted to independently verify
eradication.

Stage 4 June 2016-June 2018) {[}Phase 3{]} a two year surveillance
monitoring program will be instigated for a further two years prior to
any native species reintroductions.

~

There is extensive evidence that domestic cats (Felis catus) introduced
to offshore and oceanic islands around the world have had deleterious
impacts on endemic land vertebrates and breeding bird populations (eg.
van Aarde 1980; Moors and Atkinson 1984; King 1985; Veitch 1985; Bloomer
and Bester 1992; Bester et al. 2002; Keitt et al. 2002; Pontier et al.
2002; Blackburn et al. 2004; Martinez-Gomez and Jacobsen 2004; Nogales
et al. 2004; Ratcliffe et al. 2009; Bonnaud et al. 2010). Feral cats
have been known to drive numerous extinctions of endemic species on
islands and have contributed to at least 14\% of all 238 vertebrate
extinctions recorded globally by the IUCN (Nogales et al., 2013). In
addition, predation by feral cats currently threatens 8\% of the 464
species listed as critically endangered (Medina et al. 2011; Nogales et
al. 2013). Island faunas that have evolved for long periods in the
absence of predators are particularly susceptible to cat predation
(Dickman, 1992). Dirk Hartog Island--once a high biodiversity island--is
no exception.

On Dirk Hartog Island (620km2), which is the largest island off the
Western Australian coast (Abbott and Burbidge 1995), 10 of the 13
species of native terrestrial mammals once present are now locally
extinct (Baynes 1990; McKenzie et al. 2000) probably due to predation by
cats (Burbidge 2001; Burbidge and Manly 2002). The extirpated species of
mainly medium-sized mammals include: boodie (Bettongia lesueur),woylie
(Bettongia penicillata), western barred bandicoot (Perameles
bougainville), chuditch (Dasyurus geoffroii), mulgara (Dasycercus
cristicauda), dibbler (Parantechinus apicalis), greater stick-nest rat
(Leporillus conditor), desert mouse (Pseudomys desertor), Shark Bay
mouse (Pseudomys fieldi), and heath mouse (Pseudomys shortridgei). Only
smaller species still inhabit the island: ash-grey mouse (Pseudomys
albocinereus), sandy inland mouse (Pseudomys hermannsburgensis), and the
little long-tailed dunnart (Sminthopsis dolichura). It is possible that
the banded hare-wallaby (Lagostrophus fasciatus) and rufous hare-wallaby
(Lagorchestes hirsutus) were also on the island as they are both on
nearby Bernier and Dorre Islands, and were once on the adjacent
mainland. The island also contains threatened bird species including:
Dirk Hartog Island white-winged fairy wren (Malurus leucopterus
leucopterus), Dirk Hartog Island southern emu-wren (Stipiturus
malachurus hartogi), and the Dirk Hartog Island rufous fieldwren
(Calamanthus campestris hartogi). A population of the western
spiny-tailed skink (Egernia stokesii badia) found on the island is also
listed as threatened.

Since the 1860s, Dirk Hartog Island has been managed as a pastoral lease
grazed by sheep (Ovis aries) and goats (Capra hircus). More recently,
tourism has been the main commercial activity on the island. Cats were
probably introduced by early pastoralists and became feral during the
late 19th century (Burbidge 2001). The island was established as a
National Park in November 2009, which now provides the opportunity to
reconstruct the native mammal fauna (Algar et al. 2011). Dirk Hartog
Island could potentially support one of the most diverse mammal
assemblages in Australia and contribute significantly to the long-term
conservation of several threatened species. Successful eradication of
feral cats would be a necessary precursor to any mammal reintroductions.




\subsection*{Expected outcome}

The biodiversity outcome from this project will be a measurable decline
in the cat population, eventually to zero when eradication is confirmed.

There will be global interest in the outcome of this project and the
techniques used. Knowledge and technology transfer to other agencies
contemplating cat eradications on islands will be through publication of
manuscripts in scientific journals and presentations at various
conferences.




\subsection*{Strategic context}

This project aligns with the Corporate Plan and Science Division
Strategic Plan for Biodiversity Conservation Research as outlined below.

\textbf{Corporate Plan}

\textbf{1. Conserving biodiversity}

1.5 Protect diversity from threatening processes, agents and activities
including pest animals

Expand and enhance the Western Shield wildlife recovery program
incorporating introduced predator control

Expand programs for the control of pest animals

Implement integrated management strategies to control pests and diseases

Give special attention to the protection of internationally recognised
natural values of World Heritage sites

\textbf{Science Division Strategic Plan for Biodiversity Conservation
Research}

\textbf{G2 Understand the threats to biodiversity and develop
evidence-based management options to ameliorate threats}

\emph{Threatened species and communities}

2.7 Participate in active adaptive management programs that will lead to
improved conservation status of threatened arid zone medium-sized
mammals (links with 2.2), a group that has declined significantly since
European settlement. Adaptive management plans have been developed for
Dirk Hartog Island.

\emph{Threatening processes}

2.20 Complete research into sustained, effective control of feral cats
across a range of biomes.

2.34 Develop safe and effective control technologies for feral cats,
camels, goats and pigs on DPaW-managed lands.

\textbf{G6 Promote and facilitate the uptake of research findings and
communicate the contribution of science to biodiversity conservation and
natural resource management.}

This process has already commenced with a number of manuscripts
published on preliminary work (see below).

Algar, D., Hilmer, S., Onus, M., Hamilton, N. and Moore, J. (2011). New
national park to be cat-free. \emph{LANDSCOPE} \textbf{26(3)}, 39-45.

Algar, D., Johnston, M. and Hilmer, S.S. (2011). A pilot study for the
proposed eradication of feral cats on Dirk Hartog Island, Western
Australia. In: Island Invasives: Eradication and Management (eds. C.R.
Veitch, M.N. Clout and D.R. Towns) pp 10-16. IUCN, Gland, Switzerland.

Bode, M., Brennan, K.E.C., Helmstedt, K., Desmond, A., Smia, R. and
Algar, D. (2013). Interior fences reduce cost and uncertainty when
eradicating invasive species from islands. \emph{Methods in Ecology and
Evolution}\textbf{4(9),} 819-827.

Deller, M. (2013). The role of marine species in the diet of the feral
cat, \emph{Felis catus}, on Dirk Hartog Island: a dietary analysis.
Bachelor of Science (Conservation Biology and Management) SCIE4501-4
FNAS Research Thesis Faculty of Science. The University of Western
Australia.

Hilmer, S.S., Algar, D. and Johnston, M. (2010). Opportunistic
observation of predation of Loggerhead turtle hatchlings by feral cats
on Dirk Hartog Island, Western Australia. \emph{Journal of the Royal
Society of Western Australia} \textbf{93},141-146.

Johnston, M., Algar, D., O'Donoghue, M. and Morris, J. (2011). Field
efficacy of the Curiosity\textsuperscript{®} feral cat bait on three
Australian islands. In: Island Invasives: Eradication and Management
(eds. C.R. Veitch, M.N. Clout and D.R. Towns) pp 182-187. IUCN, Gland,
Switzerland.

Johnston, M., Algar, D., Onus, M., Hamilton, N. Hilmer, S. Withnell, B.
and Koch, K. (2010). A bait efficacy trial for the management of feral
cats on Dirk Hartog Island. Arthur Rylah Institute for Environmental
Research Technical Report Series No. 205. Department of Sustainability
and Environment, Heidelberg, Victoria. 42 pp.

Koch, K., Algar, D. and Schwenk, K. (accepted). Population structure and
management of invasive cats on an Australian island. \emph{Journal of
Wildlife Management}.

The project also aligns directly with a number of strategic objectives
from the 'Dirk Hartog Island National Park Ecological Restoration
Project Submission, November 2012.

\emph{Ecological objectives for DHI}

\begin{enumerate}
\itemsep1pt\parskip0pt\parsep0pt
\item
  \emph{Eradicate cats from Dirk Hartog Island}.
\item
  \emph{Reintroduce 10 mammal species that are locally extinct on the
  island and introduce two mammal species for conservation purposes.}
\end{enumerate}

Sucessful eradication of feral cats would be a necessary precursor to
any mammal reintroductions.

\textbf{Community objectives}

\begin{enumerate}
\itemsep1pt\parskip0pt\parsep0pt
\item
  Promote scientific research associated with the ecological research
  project, and publish reports on the project as well as scientific
  findings.
\end{enumerate}




\subsection*{Expected collaborations}

This project has been discussed at length with District/Regional staff
and various Dirk Hartog committees since the project was proposed almost
a decade ago. The umbrella program `Dirk Hartog Island National Park
Ecological Restoration Project' is a collaboration with other Science
and Conservation Division Programs. Collaboration with departmental
staff, other government agencies, industry, traditional land owners and
the broader community has commenced and will continue as the project
progresses.

Collaboration with a number of universities is expected. Several
projects have already been undertaken with universities as outlined
below.

Genetic analysis of DHI and adjacent mainland cat populations
(biosecurity) - Johann Wolfgang Goethe University (Frankfurt, Germany)

Koch, K., Algar, D. and Schwenk, K. (accepted). Population structure and
management of invasive cats on an Australian island. \emph{Journal of
Wildlife Management}.

Cat diet on DHI using stable isotope analysis - University of Western
Australia

Deller, M. (2013). The role of marine species in the diet of the feral
cat, \emph{Felis catus}, on Dirk Hartog Island: a dietary analysis.
Bachelor of Science (Conservation Biology and Management) SCIE4501-4
FNAS Research Thesis Faculty of Science. The University of Western
Australia.


\subsection*{Proposed period of the project}
Jan. 2, 2014 -- Dec. 31, 2018



\subsection*{Staff time allocation }



\begin{longtabu} to \linewidth { |  X | X | X | X | }
\hline
\rowcolor{infobg}
Role & Year 1 & Year 2 & Year 3\\
\hline
\endhead



Scientist & 1.5 & 1.5 & 1.5\\



Technical & 4.5 & 4.5 & 4.5\\



Volunteer & unknown & unknown & unknown\\



Collaborator & unknown & unknown & unknown\\


\hline
\end{longtabu}



\subsection*{Indicative operating budget }



\begin{longtabu} to \linewidth { |  X | X | X | X | }
\hline
\rowcolor{infobg}
Source & Year 1 & Year 2 & Year 3\\
\hline
\endhead



Consolidated Funds (DPaW) & 205,000 & 205,000 & 205,000\\



External Funding & 1,600,000 & 750,000 & 700,000\\


\hline
\end{longtabu}






%-----------------------------------------------------------------------------%
% Back matter
%\backmatter
\end{document}
%-----------------------------------------------------------------------------%

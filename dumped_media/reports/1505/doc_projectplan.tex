
\documentclass[version=last,
    paper=a4,                               % paper size
    10pt,                                   % default font size
    dvipsnames,
    % twoside,                                % PRINT Binding Correction
    oneside,                              % ONLINE
    headings=openany,                       % open chapters on odd and even pages
    open=any,
    BCOR=7mm,                               % PRINT Binding Correction
    %DIV=13,    % typearea 161.54mm x 228.46mm, top 22.85mm, inner 16.15mm
    %DIV=14,    % 165.00 233.36 21.21 15.00
    DIV=15,     % 168.00 237.60 19.80 14.00
    % toc=chapterentrywithdots              % Table of Contents style
]{scrbook}
\usepackage{typearea}


%------------------------------------------------------------------------------%
% Headers and footers
%------------------------------------------------------------------------------%
\usepackage[automark,headsepline,footsepline,plainfootsepline]{scrlayer-scrpage}
\automark*[section]{}
\addtokomafont{pageheadfoot}{\normalfont\footnotesize\sffamily} % Don't italicise
\renewcommand{\chaptermark}[1]{\markleft{#1}{}}     % Chapter: suppress numbering
\renewcommand{\sectionmark}[1]{\markright{#1}{}}    % Section: suppress numbering

% Header (inner, center, outer)
% \ihead{\href{http://sdis.dbca.wa.gov.au}{\textbf{Project Plan SP 2015-016}}}
%\chead{\href{http://sdis.dbca.wa.gov.au}{Science Directorate Information System}}
% \ohead{\href{https://www.dbca.wa.gov.au/science/10-biodiversity-and-conservation-science}{
% \includegraphics[height=8mm, keepaspectratio]{/usr/src/app/staticfiles/img/logo-dbca-bcs.jpg}}}

% Footer (inner, center, outer)
% \ifoot{\RaggedRight\leftmark}                       % Chapter
% \cfoot{\RaggedLeft\rightmark}                       % Section
% \ofoot[\bfseries\thepage]{\bfseries\thepage}        % Page number (also [plain])


%------------------------------------------------------------------------------%
% Fonts, encoding
%------------------------------------------------------------------------------%
%\usepackage{avant}             % Use the Avantgarde font for headings
\usepackage{txfonts}
\usepackage{mathptmx}
\usepackage{gensymb}            % provides \textdegree
\renewcommand{\familydefault}{\sfdefault} % Default to Sans Serif font
\usepackage{microtype}          % Slightly tweak font spacing for aesthetics
\usepackage[english]{babel}
\usepackage[utf8]{inputenc}  % Allow letters with accents
\usepackage[utf8]{luainputenc}  % Allow letters with accents
\usepackage[T1]{fontenc}        % Use 8-bit encoding that has 256 glyphs
\usepackage{textcomp}
\usepackage[explicit]{titlesec}           % Customise of titles
%\DeclareUnicodeCharacter{0080}{\textregistered}
\DeclareUnicodeCharacter{00B0}{\textdegree}

%------------------------------------------------------------------------------%
% Tables, columns, layout
%------------------------------------------------------------------------------%
\usepackage{etoolbox}
\AtBeginEnvironment{longtabu}{\footnotesize}{}{}  % Table font size
\usepackage{booktabs}           % Required for nicer horizontal rules in tables
\usepackage{multicol}           % 2 col publications
\usepackage{pdflscape}          % Landscape pages
\usepackage{pdfpages}           % Include PDFs
\usepackage{hanging}            % hanging paragraphs for publications
%\usepackage{titletoc}          % Manipulate the table of contents
\setcounter{tocdepth}{2}        % TOC list down to section
\usepackage{enumerate}          % Enumerations
\usepackage{enumitem}           % Enumerations
\usepackage{longtable}          % Multipage table
\usepackage{tabu}               %
\setlength{\tabulinesep}{1.5mm} % Consistent vertical spacing in tabu
\newcommand{\HRule}{\vspace{8mm}\noindent\rule{\linewidth}{0.1pt}}
\usepackage[export]{adjustbox}  % minipage, image frame


%------------------------------------------------------------------------------%
% Graphics, images, colours
%------------------------------------------------------------------------------%
\usepackage{graphicx} % embedded images
\usepackage{wrapfig}  % wrap figures in text
\usepackage{caption}  % allow unnumbered captions
\usepackage{eso-pic} % Required for specifying an image background in the title page
\usepackage{colortbl} % define custom named colours
\usepackage{xstring} % Conditionals
\usepackage{transparent} % Allow transparent images

\definecolor{RedFire}{RGB}{146,25,28}
% Following PICA branding guidelines
% https://dpaw.sharepoint.com/Divisions/pica/Documents/Branding%20guidelines.pdf
\definecolor{dpawblue}{RGB}{35,97,146}          % Pantone 647
\definecolor{dpaworange}{RGB}{237,139,0}        % Pantone 144
\definecolor{dpawgreen}{RGB}{116,170,80}        % Pantone 7489
\definecolor{dpawred}{RGB}{124,46,44}           % Paul's suggestion

% bootstrap colours
\definecolor{successbg}{RGB}{223,240,216}
\definecolor{errorbg}{RGB}{242,222,222}
\definecolor{warningbg}{RGB}{252,248,227}
\definecolor{infobg}{RGB}{217,237,247}
\definecolor{muted}{RGB}{153,153,153}
\definecolor{success}{RGB}{70,136,71}
\definecolor{error}{RGB}{185,74,72}
\definecolor{warning}{RGB}{192,152,83}
\definecolor{info}{RGB}{58,135,173}

% SDIS approval colours
\definecolor{required}{RGB}{192,152,83}
\definecolor{requiredbg}{RGB}{252,248,227}
\definecolor{denied}{RGB}{185,74,72}
\definecolor{deniedbg}{RGB}{242,222,222}
\definecolor{granted}{RGB}{70,136,71}
\definecolor{grantedbg}{RGB}{223,240,216}
\definecolor{notrequired}{RGB}{153,153,153}
\definecolor{notrequiredbg}{RGB}{255,255,255}

\usepackage{tikz} % Drawing
\usetikzlibrary{arrows,shapes,positioning,shadows,trees}


%------------------------------------------------------------------------------%
% Hyperlinks
%------------------------------------------------------------------------------%
\usepackage[open=true]{bookmark}
\usepackage{nameref}
\usepackage{ifxetex,ifluatex}
\ifxetex
  \usepackage[
    setpagesize=false,        % page size defined by xetex
    unicode=false,            % unicode breaks when used with xetex
    xetex]{hyperref}
\else
  \usepackage[unicode=true]{hyperref}
\fi

\hypersetup{
  backref=true,
  pagebackref=true,
  hyperindex=true,
  breaklinks=true,
  urlcolor=dpawblue,
  bookmarks=true,
  bookmarksopen=false,
  pdfauthor={Biodiversity and Conservation Science, Department of Biodiversity, Conservation and Attractions, WA},
  pdftitle=Project Plan SP 2015-016
,
  colorlinks=true,
  linkcolor=dpawblue,
  pdfborder={0 0 0}}

\urlstyle{same}                         % don't use monospace font for urlstyle


%------------------------------------------------------------------------------%
% Black magic to linebreak URLs
%------------------------------------------------------------------------------%
\usepackage{url}
\makeatletter\g@addto@macro{\UrlBreaks}{\UrlOrds}\makeatother
\Urlmuskip=0mu plus 1mu


%------------------------------------------------------------------------------%
% Fix latex errors
%------------------------------------------------------------------------------%
\providecommand{\tightlist}{\setlength{\itemsep}{0pt}\setlength{\parskip}{0pt}}

% copy-pasted HTML <span> in SDIS fields becomes \text{} in tex source
\providecommand{\text}{}


%------------------------------------------------------------------------------%
% Custom Tikz node for SDS diagram
%------------------------------------------------------------------------------%
\newcommand\mynode[6][]{
  \node[#1] (#2){
    \parbox{#3\relax}{
      \begin{center}
      \textbf{#4}\\
      #5\\
      \footnotesize{#6}
      \end{center}
    }};}


%------------------------------------------------------------------------------%
% Custom no-pagebreaks-environment
%------------------------------------------------------------------------------%
\newenvironment{absolutelynopagebreak}
  {\par\nobreak\vfil\penalty0\vfilneg\vtop\bgroup}
  {\par\xdef\tpd{\the\prevdepth}\egroup\prevdepth=\tpd}


%------------------------------------------------------------------------------%
% Remove the header from odd empty pages at the end of chapters
%------------------------------------------------------------------------------%
\makeatletter
\renewcommand{\cleardoublepage}{
\clearpage\ifodd\c@page\else
\hbox{}
\vspace*{\fill}
\thispagestyle{empty}
\newpage
\fi}


%----------------------------------------------------------------------------------------
%  Page flow control
%----------------------------------------------------------------------------------------
%\widowpenalty=10000
%\clubpenalty=10000
%\vbadness=1200
%\hbadness=11000


%----------------------------------------------------------------------------------------
%   CHAPTER HEADINGS
%----------------------------------------------------------------------------------------
\newcommand{\thechapterimage}{}
\newcommand{\chapterimage}[1]{\renewcommand{\thechapterimage}{#1}}

% Numbered chapters with mini tableofcontents
\def\thechapter{\arabic{chapter}}
\def\@makechapterhead#1{
%\thispagestyle{plain}
{\centering \normalfont\sffamily
\ifnum \c@secnumdepth >\m@ne
\if@mainmatter
\startcontents
\begin{tikzpicture}[remember picture,overlay]
\node at (current page.north west)
{\begin{tikzpicture}[remember picture,overlay]
\node[anchor=north west,inner sep=0pt] at (0,0) {
\includegraphics[width=\paperwidth,height=0.5\paperwidth]{\thechapterimage}};
%------------------------------------------------------------------------------%
% Small contents box in the chapter heading
% Mini TOC background box
%\fill[color=dpawblue!10!white,opacity=.2] (1cm,0) rectangle (
%  3.5cm, % Mini TOC box width
%  -3.5cm % Mini TOC box height
%);
% Mini TOC text content
%\node[anchor=north west] at (1.1cm,.35cm) {
%  \parbox[t][8cm][t]{6.5cm}{
%    \huge\bfseries\flushleft
%    \printcontents{l}{1}{
%    \setcounter{tocdepth}{1}                   % Mini TOC level depth
%    }
% }
%};
%------------------------------------------------------------------------------%
% Chapter heading
\draw[anchor=west] (5cm,-9cm) node [
rounded corners=20pt,
fill=dpawblue!10!white,
text opacity=1,
draw=dpawblue,
draw opacity=1,
line width=1.5pt,
fill opacity=.2,
inner sep=12pt]{
    \huge\sffamily\bfseries\textcolor{black}{
      \thechapter. #1\strut\makebox[22cm]{}
    }
};
\end{tikzpicture}};
\end{tikzpicture}}
\par\vspace*{240\p@}                            % Push text below chapter image
\fi
\fi}

%------------------------------------------------------------------------------%
% Unnumbered chapters without mini tableofcontents
%------------------------------------------------------------------------------%
\def\@makeschapterhead#1{
%\thispagestyle{plain}
{\centering \normalfont\sffamily
\ifnum \c@secnumdepth >\m@ne
\if@mainmatter
\begin{tikzpicture}[remember picture,overlay]
\node at (current page.north west)
{\begin{tikzpicture}[remember picture,overlay]
\node[anchor=north west,inner sep=0pt] at (0,0) {
  \includegraphics[width=\paperwidth,height=0.5\paperwidth]{\thechapterimage}};
% Mini TOC background box
%\fill[color=dpawblue!10!white,opacity=.2] (1cm,0) rectangle (
%  3.5cm,                                       % Mini TOC box width
%  -3.5cm                                       % Mini TOC box height
%);
% Mini TOC text content
%\node[anchor=north west] at (1.1cm,.35cm) {
%  \parbox[t][8cm][t]{6.5cm}{
%    \huge\bfseries\flushleft
%    \printcontents{l}{1}{
%    \setcounter{tocdepth}{1} % Mini TOC level depth
%    }
%}
%};
\draw[anchor=west] (5cm,-9cm) node [rounded corners=20pt,
  fill=dpawblue!10!white,fill opacity=.6,inner sep=12pt,text opacity=1,
  draw=dpawblue,draw opacity=1,line width=1.5pt]{
  \huge\sffamily\bfseries\textcolor{black}{#1\strut\makebox[22cm]{}}};
\end{tikzpicture}};
\end{tikzpicture}}
\par\vspace*{240\p@}
\fi
\fi
}
\makeatother



\usepackage[automark,headsepline,footsepline,plainfootsepline]{scrlayer-scrpage}
\automark*[section]{}
\addtokomafont{pageheadfoot}{\normalfont\footnotesize\sffamily} % Don't italicise
\renewcommand{\chaptermark}[1]{\markleft{#1}{}}     % Chapter: suppress numbering
\renewcommand{\sectionmark}[1]{\markright{#1}{}}    % Section: suppress numbering

% Header (inner, center, outer)
\ihead{\href{http://sdis.dbca.wa.gov.au/documents/projectplan/1505/}{Project Plan SP 2015-016}}
%\chead{\href{http://sdis.dbca.wa.gov.au}{Science Directorate Information System}}
\ohead{\href{https://www.dbca.wa.gov.au/science/10-biodiversity-and-conservation-science}{
\includegraphics[height=6mm, keepaspectratio]{/usr/src/app/staticfiles/img/logo-dbca-bcs.jpg}}}
% Footer (inner, center, outer)
\ifoot{\textbf{Printed}~Tue, 22 Sep 2020 11:32:30 +0800} % inner/left footer
\cfoot{}
\ofoot[\bfseries\thepage]{\bfseries\thepage}        % Page number (also [plain])


\pagestyle{scrheadings}
\setkomafont{pageheadfoot}{\normalfont}

%-----------------------------------------------------------------------------%
\begin{document}
\raggedbottom

%-----------------------------------------------------------------------------%
% Title page
\subject{Project Plan SP 2015-016
}
\title{Improved fauna recovery in the Pilbara -- benefitting the endangered
northern quoll through broad-scale feral cat baiting.
}
\subtitle{Animal Science
}
\author{}
\publishers{\small
    \subsection*{Project Core Team}
\begin{tabu} {X X}
\textbf{Supervising Scientist} & Russell Palmer
\\
\textbf{Data Custodian} & Russell Palmer
\\
\textbf{Site Custodian} & 
\\
\end{tabu}


    \subsection*{Project status as of Sept. 22, 2020, 11:32 a.m.}
\begin{tabu} {X X}
& Approved and active
\\
\end{tabu}

    
\subsection*{Document endorsements and approvals as of Sept. 22, 2020, 11:32 a.m.}
\begin{tabu} {X X}

%\rowcolor{grantedbg}
    \textbf{Project Team} & 
    \textcolor{granted}{ granted}\\

%\rowcolor{grantedbg}
    \textbf{Program Leader} & 
    \textcolor{granted}{ granted}\\

%\rowcolor{grantedbg}
    \textbf{Directorate} & 
    \textcolor{granted}{ granted}\\

%\rowcolor{grantedbg}
    \textbf{Biometrician} & 
    \textcolor{granted}{ granted}\\

%\rowcolor{not requiredbg}
    \textbf{Herbarium Curator} & 
    \textcolor{not required}{ not required}\\

%\rowcolor{grantedbg}
    \textbf{Animal Ethics Committee} & 
    \textcolor{granted}{ granted}\\

\end{tabu}



}
\uppertitleback{}
\lowertitleback{}
\date{}

%-----------------------------------------------------------------------------%
% Front matter
\frontmatter
\maketitle
%-----------------------------------------------------------------------------%
% Main matter
\mainmatter



\section*{Improved fauna recovery in the Pilbara -- benefitting the endangered
northern quoll through broad-scale feral cat baiting.
}



\subsection*{Biodiversity and Conservation Science Program}

Animal Science




\subsection*{Departmental Service}

Service 6: Conserving Habitats, Species and Communities


\subsection*{Project Staff}
\begin{tabu} {X X X}
%\rowcolor{infobg}
\textbf{Role} & \textbf{Person} & \textbf{Time allocation (FTE)}\\

Supervising Scientist & Russell Palmer & 1.0\\

Technical Officer & Hannah Anderson & 1.0\\

Technical Officer & Brooke Richards & 1.0\\

\end{tabu}




\subsection*{Related Science Projects}

None


\subsection*{Proposed period of the project}
Aug. 1, 2014 -- Dec. 31, 2015



\section*{Relevance and Outcomes}


\subsection*{Background}

Since European settlement, 29 (9.2\%) of Australia's terrestrial mammal
species have become extinct, and another 57 species (18.3\%) have
declined significantly and are considered threatened (Woinarski \emph{et
al.} 2014). Predation by introduced predators (particularly the European
red fox and feral cat) has been identified as a significant factor in
mammal declines in Australia. In the 1980-90s, predation by foxes was
shown to be a significant threatening process for native fauna (Kinnear
\emph{et al.} 2002, Morris \emph{et} \emph{al.} 2000). More recently,
feral cat predation has been identified as a major issue (Fisher
\emph{et al.} 2014, Wayne \emph{et al}. 2013) and Woinarski \emph{et
al.} (2014) regard this as the factor now affecting the largest number
of threatened and near threatened mammal taxa.

Twelve species of terrestrial mammal have become extinct on the Pilbara
mainland in the last 200 years, and another seven species have declined
(McKenzie \emph{et al.} 2006). Fortunately, some of the species that
have become extinct on the mainland still persist on offshore islands
(Abbott and Burbidge 1995). ~A review of the conservation values,
threats and management options for biodiversity conservation in the
Pilbara (Carwardine \emph{et al}. 2014) identified that for terrestrial
vertebrates of conservation significance, feral cat control would
provide most benefits but probably had a low chance of success. However,
without cat control it is likely that another five species of
terrestrial vertebrate will become functionally extinct in the Pilbara
in the next 20 years, and another 18 species will continue to decline.

The northern quoll (\emph{Dasyurus hallucatus}) is one of the seven
Pilbara medium-sized mammal species that has persisted in the Pilbara
bioregion (McKenzie \emph{et al.} 2006). All of these species, except
perhaps the echidna (\emph{Tachyglossus aculeatus)}, have declined to
some extent in the Pilbara, and three, including the northern quoll, are
listed as threatened species. ~The northern quoll was once distributed
widely across northern Australia from the Pilbara and Kimberley across
the Top End to southern Queensland, but as now contracted to several
disjunct populations (Braithwaite and Griffiths 1994, Oakwood 2008). An
alarming decrease or complete collapse in once locally abundant
populations of northern quoll has occurred in recent years as a direct
result of the invasion of the cane toad, \emph{Rhinella marina}
(Woinarski \emph{et al}. 2008; Woinarski \emph{et al}. 2010). Three
other factors have also been identified as contributing to the decline
of northern quolls and other medium-sized mammals across northern
Australia: changed habitats through widespread fires, predation by feral
cats, and novel disease (Woinarski \emph{et al.} 2011). ~Due to these
factors, the northern quoll is listed as Endangered under both the
Commonwealth's \emph{Environment Protection and Biodiversity
Conservation Act 1999} (EPBC Act 1999) and the Western Australian
\emph{Wildlife Conservation Act 1950.}

The \emph{Eradicat}® feral cat bait will shortly be registered for
operational use in WA and could potentially be used in the Pilbara to
reduce feral cat (and fox and dog) densities and improve conservation
outcomes for northern quolls and other threatened fauna such as the
bilby (\emph{Macrotis lagotis}) and mulgara (\emph{Dasycercus} sp.).
~Prior to using operationally, potential non-target bait impacts have to
be identified and resolved. As a top order native predator, the northern
quoll is at a potential risk to poisoning after ingestion of the sausage
baits. Based on a 1080 LD \textsubscript{50} of 7.5mg/kg (King \emph{et
al}. 1989), an average size Pilbara northern quoll (380-580g) would only
need to ingest approximately one cat toxic bait to be at risk. Calver
\emph{et al.} (1989) identified that in the laboratory, the northern
quoll was at risk from accidental poisoning from crackle baits
containing 1080 for dingo control. However, King (1989) showed that an
aerial dingo baiting program did not pose a hazard to free ranging
northern quolls. A field trial using toxic cat baits and closely
monitored northern quolls is the only certain way of assessing cat
baiting risk to a free-ranging northern quoll population.~ Assessing the
impact on quolls of baiting to control feral cats in the Pilbara is an
important component in the development of a landscape scale program to
control introduced predators and reduce the extinction risk for northern
quolls and other medium-sized mammals in the Pilbara.




\subsection*{Aims}

a) To assess the field uptake of feral cat baits \emph{Eradicat®}by
northern quolls \emph{Dasyurus} \emph{hallucatus}, and impact on
survivorship.

b) To develop an effective cat control strategy that will benefit the
northern quoll and other threatened fauna in the Pilbara




\subsection*{Expected outcome}

a) An effective feral cat baiting strategy in areas of the Pilbara where
northern quolls occur.

b) Improved conservation outcomes for northern quoll and other
threatened Pilbara fauna.




\subsection*{Knowledge transfer}

Parks and Wildlife regional staff, mining companies, consultants.




\subsection*{Tasks and Milestones}

\begin{longtable}[]{@{}lll@{}}
\toprule
\endhead
\begin{minipage}[t]{0.30\columnwidth}\raggedright
\textbf{Tasks}

\textbf{~}\strut
\end{minipage} & \begin{minipage}[t]{0.30\columnwidth}\raggedright
\textbf{Outputs}\strut
\end{minipage} & \begin{minipage}[t]{0.30\columnwidth}\raggedright
\textbf{Timelines}\strut
\end{minipage}\tabularnewline
\begin{minipage}[t]{0.30\columnwidth}\raggedright
Undertake Yarraloola site reconnaissance.\strut
\end{minipage} & \begin{minipage}[t]{0.30\columnwidth}\raggedright
Knowledge of logistics, access, and potential study sites\strut
\end{minipage} & \begin{minipage}[t]{0.30\columnwidth}\raggedright
August / September 2014\strut
\end{minipage}\tabularnewline
\begin{minipage}[t]{0.30\columnwidth}\raggedright
Organise and hold planning workshop.\strut
\end{minipage} & \begin{minipage}[t]{0.30\columnwidth}\raggedright
Report on reconnaissance, record of meeting, agreement on design of bait
uptake study -- sites, methods.\strut
\end{minipage} & \begin{minipage}[t]{0.30\columnwidth}\raggedright
October 2014\strut
\end{minipage}\tabularnewline
\begin{minipage}[t]{0.30\columnwidth}\raggedright
Identify likely areas for the cat baiting trial -- 20,000ha cat baited
area on Yarraloola, plus Red Hill control site (unbaited).\strut
\end{minipage} & \begin{minipage}[t]{0.30\columnwidth}\raggedright
Maps showing baited and unbaited sites.\strut
\end{minipage} & \begin{minipage}[t]{0.30\columnwidth}\raggedright
October 2014\strut
\end{minipage}\tabularnewline
\begin{minipage}[t]{0.30\columnwidth}\raggedright
Prepare operational introduced predator control program.\strut
\end{minipage} & \begin{minipage}[t]{0.30\columnwidth}\raggedright
Indicative plan prepared, may need to be adjusted following results of
quoll cat bait uptake trial.\strut
\end{minipage} & \begin{minipage}[t]{0.30\columnwidth}\raggedright
October 2014, review and revise if necessary in October 2015.\strut
\end{minipage}\tabularnewline
\begin{minipage}[t]{0.30\columnwidth}\raggedright
Meet with traditional owners regarding 1080 baiting risks.\strut
\end{minipage} & \begin{minipage}[t]{0.30\columnwidth}\raggedright
Agreement from TOs for baiting program to continue.\strut
\end{minipage} & \begin{minipage}[t]{0.30\columnwidth}\raggedright
November 2014\strut
\end{minipage}\tabularnewline
\begin{minipage}[t]{0.30\columnwidth}\raggedright
Pen trial to assess quoll ability to wear GPS radio-collars (includes
sourcing 4 quolls from the wild for trial).\strut
\end{minipage} & \begin{minipage}[t]{0.30\columnwidth}\raggedright
Knowledge on whether GPS collars can be used on quolls.\strut
\end{minipage} & \begin{minipage}[t]{0.30\columnwidth}\raggedright
February 2015\strut
\end{minipage}\tabularnewline
\begin{minipage}[t]{0.30\columnwidth}\raggedright
Undertake 1080 baiting risk assessment, warning signs, notify neighbours
(Pilbara Region). Obtain APVMA approval for cat bait trial (approval for
2014 program obtained, but need to re-apply for 2015).\strut
\end{minipage} & \begin{minipage}[t]{0.30\columnwidth}\raggedright
Approval for cat baiting to proceed.\strut
\end{minipage} & \begin{minipage}[t]{0.30\columnwidth}\raggedright
March 2015\strut
\end{minipage}\tabularnewline
\begin{minipage}[t]{0.30\columnwidth}\raggedright
Trapping of Yarraloola (baited) and Red Hill (unbaited control) sites to
assess quoll numbers, and fit VHF radio-collars. Some GPS collars may
deployed depending on outcomes of trial in Feb and numbers of quolls
trapped.\strut
\end{minipage} & \begin{minipage}[t]{0.30\columnwidth}\raggedright
Sufficient quolls for radio-collaring (n= at least 20 at each site,
approx. equal sex ratio)\strut
\end{minipage} & \begin{minipage}[t]{0.30\columnwidth}\raggedright
April - May 2015\strut
\end{minipage}\tabularnewline
\begin{minipage}[t]{0.30\columnwidth}\raggedright
Select area to be cat baited at Yarraloola to ensure adequate coverage
of sites with radio-collared quolls.\strut
\end{minipage} & \begin{minipage}[t]{0.30\columnwidth}\raggedright
Appropriate cat baiting cell provided to Western Shield.\strut
\end{minipage} & \begin{minipage}[t]{0.30\columnwidth}\raggedright
First week of June 2015\strut
\end{minipage}\tabularnewline
\begin{minipage}[t]{0.30\columnwidth}\raggedright
Monitor \textbf{pre-baiting} survivorship of quolls via aerial and
ground radio-tracking, camera array.~~~~~\strut
\end{minipage} & \begin{minipage}[t]{0.30\columnwidth}\raggedright
Ensure all quolls are alive before baiting.\strut
\end{minipage} & \begin{minipage}[t]{0.30\columnwidth}\raggedright
June/early July 2015\strut
\end{minipage}\tabularnewline
\begin{minipage}[t]{0.30\columnwidth}\raggedright
Aerial baiting undertaken.\strut
\end{minipage} & \begin{minipage}[t]{0.30\columnwidth}\raggedright
Cat baits delivered to impact site 50 baits /
km\textsuperscript{2}\strut
\end{minipage} & \begin{minipage}[t]{0.30\columnwidth}\raggedright
Early July 2015\strut
\end{minipage}\tabularnewline
\begin{minipage}[t]{0.30\columnwidth}\raggedright
Monitor \textbf{post-baiting} survivorship of quolls via aerial and
ground radio-tracking, camera array.\strut
\end{minipage} & \begin{minipage}[t]{0.30\columnwidth}\raggedright
Assess impact of baiting on quolls through survivorship.\strut
\end{minipage} & \begin{minipage}[t]{0.30\columnwidth}\raggedright
Mid -- July to late August\strut
\end{minipage}\tabularnewline
\begin{minipage}[t]{0.30\columnwidth}\raggedright
Monitor breeding success of quoll (sub-lethal 1080 impacts) via
trapping.\strut
\end{minipage} & \begin{minipage}[t]{0.30\columnwidth}\raggedright
Assess impact of baiting on quolls through breeding success (presence
and persistence of pouch young)\strut
\end{minipage} & \begin{minipage}[t]{0.30\columnwidth}\raggedright
September -- October 2015\strut
\end{minipage}\tabularnewline
\begin{minipage}[t]{0.30\columnwidth}\raggedright
Prepare report on cat baiting study for Rio Tinto.\strut
\end{minipage} & \begin{minipage}[t]{0.30\columnwidth}\raggedright
Final report submitted to Rio Tinto\strut
\end{minipage} & \begin{minipage}[t]{0.30\columnwidth}\raggedright
November 2015\strut
\end{minipage}\tabularnewline
\begin{minipage}[t]{0.30\columnwidth}\raggedright
Prepare paper for publication.\strut
\end{minipage} & \begin{minipage}[t]{0.30\columnwidth}\raggedright
Paper submitted to journal\strut
\end{minipage} & \begin{minipage}[t]{0.30\columnwidth}\raggedright
December 2015\strut
\end{minipage}\tabularnewline
\bottomrule
\end{longtable}




\subsection*{References}

Abbott I, Burbidge AA (1995) The occurrence of mammal species on the
islands of Australia: a summary of existing knowledge.
\emph{CALMScience} \textbf{1:} 259-324.

Braithwaite RW, Griffiths AD (1994) Demographic variation and range
contraction in the northern quoll \emph{Dasyurus geoffroii}
(Marsupialia: Dasyuridae). \emph{Wildlife} \emph{Research} \textbf{21}:
203-217.

Calver MC, King DR, Bradley JS, Gardner JL, Martin G (1989) An
assessment of the potential target specificity of 1080 predator baiting
in Western Australia. \emph{Australian Wildlife Research}
\textbf{16}:625-638.

Carwardine J, Nicol S, van Leeuwen S, Walters B, Firn J, Reeson A,
Martin TG, Chades I (2014). Priority threat management for Pilbara
species of conservation significance. CSIRO Ecosystem Sciences,
Brisbane.

Fisher DO, Johnson CN, Lawes MJ, Fitz SA, McCallum H, Blomberg SP,
VanDerWal J, Abbott B, Frank A, Legge S, Letnic M, Thomas CR, Fisher A,
Gordon IJ, Kutt A (2014) The current decline of tropical marsupials in
Australia; is history repeating? \emph{Global Ecology and Biogeography}
\textbf{23}: 181-190.

King DR (1989) An assessment of the hazard posed to northern quolls
(\emph{Dasyurus hallucatus}) by aerial baiting with 1080 to control
dingoes. \emph{Australian Wildlife Research} \textbf{16}: 569-574.

King DR, Twigg LE, Gardner JL (1989a) Tolerance to sodium
monofluoroacetate in dasyurids from Western Australia. \emph{Australian
Wildlife Research} \textbf{6}:131-140.

Kinnear JE, Sumner NR, Onus ML (2002) The red fox in Australia -- an
exotic predator turned biocontrol agent. \emph{Biological Conservation}
\textbf{108}: 335-359.

McKenzie NL, Burbidge AA, Baynes A, Brereton R, Dickman CR, Gibson LA,
Gordon G, Menkhorst RW, Robinson AC, Williams MR, Woinarski JCZ (2006)
Analysis of factors implicated in the recent decline of Australia's
mammalian fauna. \emph{Journal of Biogeography} \textbf{34}:597-611.

Morris K, Johnson B, Orell P, Gaikhorst G, Wayne A, Moro D (2003)
Recovery of the threatened chuditch \emph{Dasyurus geoffroii}: a case
study. Chapter 30 in Predators with Pouches: the biology of carnivorous
marsupials (eds M Jones, C Dickman, M Archer). CSIRO Publishing,
Collingwood, Victoria.

Oakwood M (2008) Northern Quoll \emph{Dasyurus hallucatus} Pp 57-59 in
The Mammals of Australia (3\textsuperscript{rd} Ed). Ed:~ S. van Dyck
and R. Strahan. Reed New Holland Publishers, Australia.

Oakwood M, Foster P (2008). Monitoring extinction of the northern quoll.
\emph{Australian Academy of Science Newsletter} \textbf{71}: 6.

Wayne AF, Maxwell MA, Ward CG, Vellios CV, Wilson I, Wayne JC, Williams
MR (2013) Sudden and rapid decline of the abundant marsupial
\emph{Bettongia penicillata} in Australia. \emph{Oryx}: doi:
10.1017/S0030605313000677.

Woinarski JCZ, Oakwood M, Winter J, Burnett S, Milne D, Foster P, Myles
H, Holmes B (2008) Surviving the toads: patterns of persistence of the
northern quoll \emph{Dasyurus hallucatus} in Queensland. Report
submitted to the Natural Heritage Trust Strategic Reserve Program.
Department of Natural Resources, Environment and The Arts, Darwin.

Woinarski JCZ, Armstrong M, Brennan K, Fisher A, Griffiths AD, Hill B,
Milne DJ, Palmer C, Ward S, Watson M, Winderlich S, Young S (2010)
Monitoring indicates rapid and severe decline of native small mammals in
Kakadu National Park, northern Australia. \emph{Wildlife Research}
\textbf{38}: 307-322.

Woinarski JCZ, Legge S, Fitzsimons JA, Traill BJ, Burbidge AA, Fisher A,
Firth RSC, Gordon IJ, Griffiths AD, Johnson CN, McKenzie NL, Palmer C,
Radford I, Rankmore B, Ritchie EG, Ward S, Ziembicki M (2011) The
disappearing mammal fauna of northern Australia: context, Cause and
response. \emph{Conservation Letters} \textbf{4}: 192-201.

Woinarski JCZ, Burbidge AA, Harrison PL (2014) The Action Plan for
Australian Mammals 2012. CSIRO Publishing, Collingwood, Victoria.



\section*{Study design}


\subsection*{Methodology}

\textbf{a) Study sites:} This trial will be undertaken at two sites in
the western Pilbara region of WA. An approximately 20,000 ha site on
Yarraloola pastoral lease (managed by Rio Tinto) will be baited with~
\emph{Eradicat®} feral cat baits, and another site at Red Hill pastoral
lease, 60 km south of Yarraloola, will be used as an unbaited control
site. The actual sites will be determined in March/April 2015 taking
into account accessibility and presence of adequate numbers of northern
quolls.

Habitat attributes will be recorded at all sites on Yarraloola and Red
Hill where quolls are trapped using the standard habitat data sheets for
the northern quoll Pilbara regional monitoring project (Dunlop \emph{et
al.} 2014). These will be incorporated into the regional predictive
habitat modeling project currently underway with Edith Cowan University.
Fire history and other disturbances such as livestock impacts will also
be assessed.

\textbf{b) Cat baiting:} Approximately 10,000 \emph{Eradicat®} feral cat
baits will be dropped from a twin-engine aircraft at an altitude of 500
`(agl) and travelling at 130kts, at a density of 50 baits /
km\textsuperscript{2} (Algar \emph{et al}. 2013).~ Each bait will
contain 4.5 mg 1080, and the biomarker Rhodamine B. The presence of
Rhodamine is detected in animals as a pink dye around the mouth, in the
gastro-intestinal track, and in scats and provides evidence that a bait
has been consumed. Baiting will be undertaken in early July when diurnal
and nocturnal temperatures are the coolest as this factor enhances
uptake of baits by feral cats (Note: the effectiveness of cat baiting on
feral cats is not being assessed during this trial). The track of the
aircraft and GPS location of each cat bait drop will be recorded, and
overlaid on known quoll locations (from trapping and radio telemetry) to
ensure the quolls are in close proximity to cat baits and have the
opportunity to consume them. Note that an indicative baiting cell will
be provided to Western Shield by end of May 2015, but that this can be
refined until mid-June, to ensure baits are dropped in the areas where
quolls have been trapped, and move to.

\textbf{c) Northern quoll survivorship}: The impact of feral cat baiting
on northern quolls will be assessed by comparing survivorship at the
Yarraloola and Red Hill sites, before and after cat baiting at
Yarraloola.(BACI design). Survivorship will be assessed by two methods
a) survival of radio-collared individuals at each site, and b) detection
of quolls on a camera array.

The analysis of northern quoll survivorship will be undertaken using the
Kaplan-Meier procedure, as used by Pollock \emph{et al}. (1989). The
survivorship rate of quolls (based on radio telemetry study) in the cat
baited area will be included in a Population Viability Analysis (PVA,
VORTEX: Lacy \emph{et al.} 2009) to assess the longer term impact of cat
baiting on the Yarraloola northern quoll population. Prediction of the
risk of population extinction can be made using parameters such as
demographic structure, mortality and survival rates, and reproductive
rates (Brook \emph{et al}. 2000). These can be derived either from this
study (if available), from the Pilbara regional quoll monitoring
program, or from other quoll demographic studies e.g. Oakwood 2000). The
risk to the quoll population will be determined by examining the
population trends provided by the PVA. The risk would be considered
unacceptable if the PVA identifies that the population will decline to
extinction.

\textbf{d) Northern quoll trapping and monitoring:} Quolls will be
trapped using transects of 50 small Sheffield cage traps baited with
peanut butter, oats and sardines. This is similar to the methodology
used by the northern quoll regional monitoring project (Judy Dunlop).
Traps will be covered with a hessian bag for protection of any trapped
animal, and placed in a sheltered, shady location. Traps will be set
along rocky breakaways where quolls are known to be (from preliminary
surveys). Trapped quolls will be weighed, measured and sexed, and a
small tissue sample taken from the ear for DNA analysis. At least 20
adult quolls (380-580 g body weight) will be fitted with VHF neck
mounted radio-transmitters fitted with mortality mode (Sitrack, 18g).
These radio-collars will be tasked with operating only during daylight
hours to prolong battery life for six months. A radio-transmitter
failure rate of 10\% is anticipated and will be taken into account when
trapping quolls. It is intended to fit radio-collars to equal numbers of
males and females, and to have equal numbers of quolls radio-collared at
the Yarraloola baited site and the Red Hill control site. ~Quolls will
be re-trapped in August to remove radio-collars. Trapping will be
repeated in September/October to assess whether breeding is occurring
(presence and persistence of pouch young). A small number of quolls may
be fitted with GPS collars if pen trails with these new radio-collars
are successful. These would provide information on movements and habitat
use of the quolls before and after cat baiting.

Ground and aerial radio-tracking will be undertaken in the 4-6 week
period prior to cat baiting and 8 week period post cat baiting to
determine survivorship of the radio-collared quolls. If dead quolls are
detected through the mortality signal, carcasses will be retrieved where
possible and autopsied to determine cause of death. In particular, the
carcasses will be examined closely for the presence of the Rhodamine B
pink dye.

Two remote cameras (Reconyx Hyperfire HC900) will be set at each of the
trapping transects to monitor presence of quolls at Yarraloola before
and after cat baiting. Where possible individual quolls will be
identified using their unique spot pattern (Hohnen \emph{et al.} 2013).
Remote cameras will also be used to monitor the uptake by quolls of
non-toxic cat baits placed in front of cameras at the Red Hill control
site.

\textbf{References:}

Algar D, Onus M, Hamilton N (2013) Feral cat control as part of
\emph{Rangelands Restoration} at Lorna Glen (Matuwa), Western Australia:
the first seven years. \emph{Conservation Science Western Australia}
\textbf{8}: 367-381.

Brook BW, O''Grady JJ, Chapman AP, Burgman MA, Akcakaya HR, Frankham R
(2000) Predictive accuracy of population viability analysis in
conservation biology. \emph{Nature} \textbf{404}: 385-387.

Dunlop J, Cook A, Morris K (2014) Pilbara northern quoll project --
surveying and monitoring \emph{Dasyurus} \emph{hallucatus} in the
Pilbara, Western Australia. Department of Parks and Wildlife, Perth, WA.

Lacy RC, Borbat M, Pollak JP (2009) VORTEX: a stochastic simulation of
the extinction process, version 9.99. Chicago Zoological Society.
Brookfield, USA.

Oakwood M (2000). Reproduction and demography of the northern quoll
Dasyurus hallucatus in the lowland savanna of northern Australia.
\emph{Australian Journal of Zoology} \textbf{48}: 519-539.

Pollock KH, Winterstein SR, Bunck CM, Curtis PD (1989) Survival analysis
in telemetry studies: the staggered entry design. \emph{Journal of
Wildlife Management} \textbf{53}: 7-15.




\subsection*{Biometrician's Endorsement}

granted



\section*{Data management}


\subsection*{No. specimens}

\textenglish[variant=american]{None anticipated, but any dead quolls
recovered will be autopsied, and provide to the WA Museum if required.~
Ear tissue will be taken from all quolls captured and lodged with the WA
Museum.}




\subsection*{Herbarium Curator's Endorsement}

not required




\subsection*{Animal Ethics Committee's Endorsement}

granted




\subsection*{Data management}

A MS Access data base will be developed and populated with the relevant
spatial and morphometric capture data for the northern quolls.~

Remote camera images will be cataloged and managed with an open source
MS Access data base (Camera Base -
\url{http://www.atrium-biodiversity.org/tools/camerabase} ) which has
been developed specifically for camera trapping data analysis.~

All data will be backed up on the Parks and Wildlife (Woodvale) server.




\section*{Budget}

\section*{Consolidated Funds }



\begin{longtabu} to \linewidth { |  X | X | X | X | }
\hline
\rowcolor{infobg}
Source & Year 1 & Year 2 & Year 3\\
\hline
\endhead



FTE Scientist & 75,000 &  & \\



FTE Technical & 15,000 &  & \\



Equipment & 12,000 &  & \\



Vehicle &  &  & \\



Travel &  &  & \\



Other &  &  & \\



Total & 102,000 &  & \\


\hline
\end{longtabu}



\section*{External Funds }



\begin{longtabu} to \linewidth { |  X | X | X | X | }
\hline
\rowcolor{infobg}
Source & Year 1 & Year 2 & Year 3\\
\hline
\endhead



Salaries, Wages, Overtime & 309,100 &  & \\



Overheads & 18,000 &  & \\



Equipment & 33,200 &  & \\



Vehicle & Covered by Rio Tinto &  & \\



Travel & Covered by Rio Tinto &  & \\



Other:
Fuel
Cat baiting
Radio tracking (DPaW)
Travel (allowance)
Region - liaison with neighbours & -
12,600
11,000
10,200
16,200
20,000 &  & \\



Total & 425,000 &  & \\


\hline
\end{longtabu}





%-----------------------------------------------------------------------------%
% Back matter
%\backmatter
\end{document}
%-----------------------------------------------------------------------------%


\documentclass[version=last, paper=a4, DIV=18, usenames, dvipsnames]{scrartcl}
\usepackage{txfonts}
\usepackage{pdflscape}
\usepackage{pdfpages}
\usepackage[english]{babel} % English language/hyphenation
%%% Bootstrap colors
\definecolor{RedFire}{RGB}{146,25,28}
\definecolor{ParksWildlife}{RGB}{0,85,144}
\definecolor{successbg}{RGB}{223,240,216}
\definecolor{errorbg}{RGB}{242,222,222}
\definecolor{warningbg}{RGB}{252,248,227}
\definecolor{infobg}{RGB}{217,237,247}
\definecolor{muted}{RGB}{153,153,153}
\definecolor{success}{RGB}{70,136,71}
\definecolor{error}{RGB}{185,74,72}
\definecolor{warning}{RGB}{192,152,83}
\definecolor{info}{RGB}{58,135,173}

\definecolor{required}{HTML}{D9534F}
\definecolor{denied}{HTML}{D9534F}
\definecolor{granted}{HTML}{47A447}
\definecolor{not required}{RGB}{200, 200, 200}

\usepackage[colorlinks=true,pdftitle=doc\_studentreport.pdf
,linktoc=all,linkcolor=RedFire,urlcolor=ParksWildlife]{hyperref}
\usepackage{colortbl}
\usepackage{longtable}
\usepackage{tabu}
\setlength{\tabulinesep}{1.5mm}
\usepackage{enumerate}
\usepackage{enumitem}
\usepackage{fancyhdr}
\usepackage{lastpage}
\usepackage{graphicx}
\usepackage{eso-pic}
\usepackage{hyphenat}
\renewcommand{\familydefault}{\sfdefault}



\newcommand{\HRule}{\rule{\linewidth}{0.1pt}}

\newcommand{\placetextbox}[3]{% \placetextbox{<horizontal pos>}{<vertical pos>}{<stuff>}
  \setbox0=\hbox{#3}% Put <stuff> in a box
  \AddToShipoutPictureFG*{% Add <stuff> to current page foreground
    \put(\LenToUnit{#1\paperwidth},\LenToUnit{#2\paperheight}){\vtop{{\null}\makebox[0pt][c]{#3}}}%
  }%
}%




%-----------------------------------------------------------------------------%
% Headers and footers
%
\fancypagestyle{plain}{
  \fancyhf{}
  \setlength\headheight{60pt} % push page content below header
  \renewcommand{\headrulewidth}{0.1pt}
  \renewcommand{\footrulewidth}{0.1pt}
  
  
  \fancyhead[L]{ 
    \href{http://sdis.dpaw.wa.gov.au}{
    \includegraphics[scale=0.6]{/mnt/projects/sdis/staticfiles/img/logo-dpaw.png}}
  }
  \fancyhead[R]{ 
      \hfill
      \href{http://sdis.dpaw.wa.gov.au}{Science Directorate Information System} 
      \newline 
      \href{http://sdis.dpaw.wa.gov.au/documents/studentreport/1464/}{Progress Report 2012-224 (FY 2014-2015)} 
  }
  
  
  
  
  \fancyfoot[L]{ \leftmark\newline\textbf{Printed}\textit{ July 2, 2015, 11:33 a.m. }}
  \fancyfoot[R]{  \, \newline Page \thepage\ of \pageref{LastPage} }
  
  
}
\pagestyle{plain}
%
% end Headers
%-----------------------------------------------------------------------------%

\begin{document}

%-----------------------------------------------------------------------------%
% Title page
%

%
% end title page
%-----------------------------------------------------------------------------%




\section*{Progress Report}
This project aimed to determine if there is regional variation in the
understorey of \emph{Eucalyptus salmonophloia} woodlands across the
Great Western Woodlands (GWW) and, if so, what environmental factors
were influencing it. The project then integrated relevant existing
survey data from across the Wheatbelt to assess the variation across the
two bioregions in which salmon gum woodlands occur, Avon-Wheatbelt and
Coolgardie. This project fills large gaps in the floristic surveys of
the GWW, which have previously focused on the banded ironstone and
greenstone ranges. One hundred sites were sampled in spring 2011 and
2012, in old growth woodlands or woodlands where the timber cutting
and/or grazing history could be estimated. Data was collected on species
composition, cover and height, tree dimensions, site-based variables,
and soil physical and chemical characteristics. Detailed classification
and ordination of the data revealed two main communities; one with an
understorey of mainly chenopod species on soils higher in clay found in
the drier north and east of the GWW, and the other with non-chenopod
species (e.g. \emph{Eremophila} spp., \emph{Acacia} spp., \emph{Scaevola
spinescens} and \emph{Alyxia buxifolia}) found on sandier soils in the
wetter south and west. Precipitation, monthly precipitation variability
and temperature, and to a lesser extent soil phosphorous, pH, silt
content, and cover of organic crust influenced the patterns in floristic
composition and differentiated between the two main communities. When
data from the Wheatbelt was incorporated the two GWW communities
remained prominent and were joined by two Wheatbelt communities and one
community (with \emph{Melaleuca pauperiflora}) that traversed the two
regions. Across this larger area the influence of the annual
precipitation gradient and ratio of summer to winter rainfall (less in
the east) was strong. Generally regional factors (such as climate) were
more influential on the floristic patterns that local (such as soil)
factors. This project has contributed to knowledge about these woodlands
relevant to their conservation status, delineation of subregional
boundaries and land management activities. The salmon gum - chenopod
shrublands burn less frequently as they are less flammable and have a
more sparse cover that the salmon gum - eremophila woodlands which are
experiencing fire more frequently and consequently being reduced in
extent. Vouchers for all species collected will be lodged in the Perth
Herbarium and the field data will be lodged with TERN-ÆKOS.




\clearpage



\end{document}

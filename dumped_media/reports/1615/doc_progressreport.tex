
\documentclass[version=last,
    paper=a4, % paper size
    10pt, % default font size
    usenames,
    dvipsnames,
    oneside, % ONLINE
    headings=openany, % open chapters on odd and even pages
    %toc=chapterentrywithdots, % Table of Contents style
    %BCOR=7mm, % PRINT Binding Correction
    %DIV=13, % typearea 161.54 mm x 228.46 mm, top margin 22.85 mm, inner margin 16.15 mm
    %DIV=14, % 165.00 233.36 21.21 15.00
    DIV=15 % 168.00 237.60 19.80 14.00
]{scrbook}
\usepackage{typearea}
\usepackage[automark,headsepline,footsepline]{scrlayer-scrpage} % Headers and footers

%%
%% Fonts, encoding, spacing, indentation
%%
\usepackage{txfonts}
\renewcommand{\familydefault}{\sfdefault} % Default to Sans Serif font
\usepackage[english]{babel}
\usepackage[T1]{fontenc}
\usepackage[utf8]{inputenc}

% Paragraph spacing
%\usepackage{parskip}    % Paragraph spacing
%\setlength{\parindent}{0em} % Don't indent paragraphs - ONLINE
%\setlength{\parskip}{1.3 ex plus 0.5ex minus 0.3ex} % 1-1.8 ex vertical space between paragraphs - ONLINE

% Spacing of headings
%\RedeclareSectionCommand[afterskip=3pt]{section} % less space after section
%\RedeclareSectionCommand[beforeskip=0cm]{subsection} % less space between HRule and project name
%\RedeclareSectionCommand[afterskip=0.1\baselineskip]{subsubsection} % less space after progressreport subheadings

% Table font size
\usepackage{etoolbox}
\AtBeginEnvironment{longtabu}{\footnotesize}{}{}

%%
%% Tables, columns, layout
%%
\usepackage{multicol}   % 2 col publications
\usepackage{pdflscape}  % Landscape pages
\usepackage{pdfpages}   % Include PDFs
\usepackage{hanging}    % hanging paragraphs for publications
%\usepackage{titletoc}   % Required for manipulating the table of contents
\setcounter{tocdepth}{2} % TOC list down to section
\usepackage{enumerate}  % Enumerations
\usepackage{enumitem}   % Enumerations
\usepackage{longtable}  % Multipage table
\usepackage{tabu}       %
\setlength{\tabulinesep}{1.5mm} % Consistent vertical spacing in tabu

%%
%% Graphics, images, colours
%%
\usepackage{graphicx} % embedded images
\usepackage{eso-pic} %
\usepackage{colortbl} % define custom named colours
\definecolor{RedFire}{RGB}{146,25,28}
\definecolor{ParksWildlife}{RGB}{0,85,144}
\definecolor{successbg}{RGB}{223,240,216}
\definecolor{errorbg}{RGB}{242,222,222}
\definecolor{warningbg}{RGB}{252,248,227}
\definecolor{infobg}{RGB}{217,237,247}
\definecolor{muted}{RGB}{153,153,153}
\definecolor{success}{RGB}{70,136,71}
\definecolor{error}{RGB}{185,74,72}
\definecolor{warning}{RGB}{192,152,83}
\definecolor{info}{RGB}{58,135,173}

\definecolor{required}{RGB}{192,152,83}
\definecolor{requiredbg}{RGB}{252,248,227}
\definecolor{denied}{RGB}{185,74,72}
\definecolor{deniedbg}{RGB}{242,222,222}
\definecolor{granted}{RGB}{70,136,71}
\definecolor{grantedbg}{RGB}{223,240,216}
\definecolor{not reqiured}{RGB}{153,153,153}
\definecolor{not requiredbg}{RGB}{255,255,255}

\usepackage{tikz} % Drawing
\usetikzlibrary{arrows,shapes,positioning,shadows,trees}

%%
%% Links, URLs
%%
\usepackage[
    linktoc=all,
    %colorlinks=false,  %PRINT
    colorlinks=true, % ONLINE
    linkcolor=RedFire, % ONLINE
    urlcolor=ParksWildlife, % ONLINE
    pdftitle=Progress Report SP 2012-034 (FY 2015-2016)
]{hyperref}

% Black magic to linebreak URLs
\usepackage{url}
\makeatletter
\g@addto@macro{\UrlBreaks}{\UrlOrds}
\makeatother

%%
%% Custom macros
%%
% Thick Horizontal rule
\newcommand{\HRule}{\vspace{8mm}\\\noindent\rule{\linewidth}{0.1pt}}

% Custom Tikz node for SDS diagram
\newcommand\mynode[6][]{
    \node[#1] (#2){
        \parbox{#3\relax}{
            \begin{center}
            \textbf{#4}\\
            #5\\
            \footnotesize{#6}
            \end{center}}};}



%-----------------------------------------------------------------------------%
% Headers and Footers
\automark{section}
\ohead{\href{http://sdis.dpaw.wa.gov.au/documents/progressreport/1615/}{Progress Report SP 2012-034
}}
\chead{\href{http://sdis.dpaw.wa.gov.au}{SDIS}} % center header ONLINE
\ihead{\href{http://sdis.dpaw.wa.gov.au}{
        \includegraphics[scale=0.4]{/mnt/projects/sdis/staticfiles/img/logo-dpaw.png}}}
\ifoot{\textbf{Printed}~Tue, 5 Jul 2016 11:32:37 +0800} % inner/left footer
\cfoot{} % center footer
\ofoot{\pagemark} % outer/right footer
\pagestyle{scrheadings}
\setkomafont{pageheadfoot}{\normalfont}

%-----------------------------------------------------------------------------%
\begin{document}
\raggedbottom

%-----------------------------------------------------------------------------%
% Title page
\subject{Progress Report SP 2012-034
}
\title{Genetic assessment for conservation of rare and threatened fauna
}
\subtitle{Animal Science
}
\author{}
\publishers{\small
    \subsection*{Project Core Team}
\begin{tabu} {X X}
\textbf{Supervising Scientist} & Kym Ottewell
\\
\textbf{Data Custodian} & Kym Ottewell
\\
\textbf{Site Custodian} & 
\\
\end{tabu}


    \subsection*{Project status as of July 5, 2016, 11:32 a.m.}
\begin{tabu} {X X}
& Approved and active
\\
\end{tabu}

    
\subsection*{Document endorsements and approvals as of July 5, 2016, 11:32 a.m.}
\begin{tabu} {X X}

%\rowcolor{grantedbg}
    \textbf{Project Team} & 
    \textcolor{granted}{ granted}\\

%\rowcolor{grantedbg}
    \textbf{Program Leader} & 
    \textcolor{granted}{ granted}\\

%\rowcolor{grantedbg}
    \textbf{Directorate} & 
    \textcolor{granted}{ granted}\\

\end{tabu}



}
\uppertitleback{}
\lowertitleback{}
\date{}

%-----------------------------------------------------------------------------%
% Front matter
\frontmatter
\maketitle
%-----------------------------------------------------------------------------%
% Main matter
\mainmatter

\section*{Genetic assessment for conservation of rare and threatened fauna
}

K Ottewell, M Byrne, K Morris, D Coates


\section*{Context}
Genetic analysis of threatened species can provide important information
to support and guide conservation management. In particular, genetic
tools can be used to aid resolution of the taxonomic identity of species
to determine whether they have appropriate conservation listing. At a
population level, analysis of the genetic diversity present in extant
populations provides information on genetic `health' of threatened
species and how this may be maintained or improved through management
actions, leading to long-term positive conservation outcomes.



\section*{Aims}
\begin{itemize}
\itemsep1pt\parskip0pt\parsep0pt
\item
  Resolve taxonomic boundaries of Western Australian bandicoots
  (\emph{Isoodon} sp.), particularly \emph{I. auratus} and \emph{I.
  obesulus} and their subspecies, to determine appropriate conservation
  rankings.
\item
  In collaboration with Brian Chambers (UWA) investigate the role of
  fauna underpasses in providing connectivity between quenda (\emph{I.
  obesulus} ssp. \emph{fusciventer}) populations impacted by main road
  construction.
\item
  In collaboration with Mark Eldridge (Australian Museum), assess the
  genetic diversity and genetic structure of extant populations of
  black-flanked rock wallaby (\emph{Petrogale lateralis} ssp.
  \emph{lateralis}) to inform future conservation management, including
  translocations.
\item
  Use of DNA barcoding to confirm species identifications.
\end{itemize}



\section*{Progress}
\begin{itemize}
\itemsep1pt\parskip0pt\parsep0pt
\item
  DNA sequencing of \emph{I. obesulus}, \emph{I. auratus}, \emph{I.
  macrourus} and their subspecies using mitochondrial and nuclear
  markers for taxonomic analysis~has been undertaken and a preliminary
  report written and submitted. Further analyses using a more powerful
  genomics approach are required to fully resolve species boundaries
  between \emph{I. obesulus} and \emph{I. auratus;}however, a population
  genetic analysis is currently underway to designate `management units'
  in Western Australian bandicoots.
\item
  Population viability analyses of urban quenda populations impacted by
  road construction have been completed exploring the long-term
  trajectory of populations with and without fauna underpasses, and with
  an additional range of threats (fire, urban expansion, inbreeding). A
  manuscript has been submitted~and a further manuscript documenting
  quenda mating patterns in remnant populations is being prepared.
\item
  Preliminary data analysis of genetic diversity and structure of
  wheatbelt and mid-north rock wallaby populations has been completed.
\item
  DNA barcoding was used to identify a stranded dolphin carcass as a
  Australian snubfin dolphin hybrid, most likely with a Indo-Pacific
  humpback dolphin.
\end{itemize}



\section*{Management implications}
\begin{itemize}
\itemsep1pt\parskip0pt\parsep0pt
\item
  An Australia-wide phylogenetic assessment of \emph{I. obesulus} and
  related species/subspecies has enabled a more informed evaluation of~
  taxonomic boundaries, showing that \emph{I. o. obesulus} is restricted
  to eastern and southeastern Australia and identifying a range
  extension of \emph{I. o. fusciventer} (Western Australia) into South
  Australia. The threat status is currently being evaluated at the
  Commonwealth level for the eastern and Tasmanian sub-species of
  \emph{I. obesulus}.
\item
  Genetic and population viability analysis showed that quenda
  populations in small, isolated patches of remnant vegetation in the
  urban matrix are vulnerable to genetic erosion, inbreeding and
  population decline, particularly when connectivity within (fauna
  underpasses) or between (increased urbanisation) habitat patches is
  inhibited. The impact of fauna underpasses on population persistence
  is somewhat context-specific, but extinction risks are predicted to
  increase in populations without fauna underpasses.
\item
  Genetic information on rock wallaby populations will enable evaluation
  of the current status of nearly all extant populations, including the
  assessment of the effectiveness of past management interventions, and
  will contribute to planning of future conservation actions, including
  translocations. ~
\item
  The presence of rare dolphin hybrids in Australian waters contributes
  to our knowledge of hybridization in the marine environment.
\end{itemize}



\section*{Future directions}
\begin{itemize}
\itemsep1pt\parskip0pt\parsep0pt
\item
  Continue phylogenomic analysis of the genus \emph{Isoodon} to formally
  resolve the species boundaries across the group.
\item
  Complete manuscripts of urban quenda populations (population viability
  analysis and mating systems, including parentage assignment of
  individuals using fauna underpasses).
\item
  Comparison of historic and contemporary population genetics of
  wheatbelt rock wallaby populations to monitor genetic change and
  investigate impact of past management actions. Develop population
  viability analyses to predict future trajectory of threatened
  populations.
\item
  Utilize DNA barcoding to assist in fauna identifications when
  required.
\end{itemize}



%-----------------------------------------------------------------------------%
% Back matter
%\backmatter
\end{document}
%-----------------------------------------------------------------------------%



\documentclass[version=last,
    paper=a4, % paper size
    10pt, % default font size
    usenames,
    dvipsnames,
    oneside, % ONLINE
    headings=openany, % open chapters on odd and even pages
    %toc=chapterentrywithdots, % Table of Contents style
    %BCOR=7mm, % PRINT Binding Correction
    %DIV=13, % typearea 161.54 mm x 228.46 mm, top margin 22.85 mm, inner margin 16.15 mm
    %DIV=14, % 165.00 233.36 21.21 15.00
    DIV=15 % 168.00 237.60 19.80 14.00
]{scrbook}
\usepackage{typearea}
\usepackage[automark,headsepline,footsepline]{scrlayer-scrpage} % Headers and footers

%%
%% Fonts, encoding, spacing, indentation
%%
\usepackage{txfonts}
\renewcommand{\familydefault}{\sfdefault} % Default to Sans Serif font
\usepackage[english]{babel}
\usepackage[T1]{fontenc}
\usepackage[utf8]{inputenc}

% Paragraph spacing
%\usepackage{parskip}    % Paragraph spacing
%\setlength{\parindent}{0em} % Don't indent paragraphs - ONLINE
%\setlength{\parskip}{1.3 ex plus 0.5ex minus 0.3ex} % 1-1.8 ex vertical space between paragraphs - ONLINE

% Spacing of headings
%\RedeclareSectionCommand[afterskip=3pt]{section} % less space after section
%\RedeclareSectionCommand[beforeskip=0cm]{subsection} % less space between HRule and project name
%\RedeclareSectionCommand[afterskip=0.1\baselineskip]{subsubsection} % less space after progressreport subheadings

% Table font size
\usepackage{etoolbox}
\AtBeginEnvironment{longtabu}{\footnotesize}{}{}

%%
%% Tables, columns, layout
%%
\usepackage{multicol}   % 2 col publications
\usepackage{pdflscape}  % Landscape pages
\usepackage{pdfpages}   % Include PDFs
\usepackage{hanging}    % hanging paragraphs for publications
%\usepackage{titletoc}   % Required for manipulating the table of contents
\setcounter{tocdepth}{2} % TOC list down to section
\usepackage{enumerate}  % Enumerations
\usepackage{enumitem}   % Enumerations
\usepackage{longtable}  % Multipage table
\usepackage{tabu}       %
\setlength{\tabulinesep}{1.5mm} % Consistent vertical spacing in tabu

%%
%% Graphics, images, colours
%%
\usepackage{graphicx} % embedded images
\usepackage{eso-pic} %
\usepackage{colortbl} % define custom named colours
\definecolor{RedFire}{RGB}{146,25,28}
\definecolor{ParksWildlife}{RGB}{0,85,144}
\definecolor{successbg}{RGB}{223,240,216}
\definecolor{errorbg}{RGB}{242,222,222}
\definecolor{warningbg}{RGB}{252,248,227}
\definecolor{infobg}{RGB}{217,237,247}
\definecolor{muted}{RGB}{153,153,153}
\definecolor{success}{RGB}{70,136,71}
\definecolor{error}{RGB}{185,74,72}
\definecolor{warning}{RGB}{192,152,83}
\definecolor{info}{RGB}{58,135,173}

\definecolor{required}{RGB}{192,152,83}
\definecolor{requiredbg}{RGB}{252,248,227}
\definecolor{denied}{RGB}{185,74,72}
\definecolor{deniedbg}{RGB}{242,222,222}
\definecolor{granted}{RGB}{70,136,71}
\definecolor{grantedbg}{RGB}{223,240,216}
\definecolor{not reqiured}{RGB}{153,153,153}
\definecolor{not requiredbg}{RGB}{255,255,255}

\usepackage{tikz} % Drawing
\usetikzlibrary{arrows,shapes,positioning,shadows,trees}

%%
%% Links, URLs
%%
\usepackage[
    linktoc=all,
    %colorlinks=false,  %PRINT
    colorlinks=true, % ONLINE
    linkcolor=RedFire, % ONLINE
    urlcolor=ParksWildlife, % ONLINE
    pdftitle=Progress Report SP 2014-001 (FY 2015-2016)
]{hyperref}

% Black magic to linebreak URLs
\usepackage{url}
\makeatletter
\g@addto@macro{\UrlBreaks}{\UrlOrds}
\makeatother

%%
%% Custom macros
%%
% Thick Horizontal rule
\newcommand{\HRule}{\vspace{8mm}\\\noindent\rule{\linewidth}{0.1pt}}

% Custom Tikz node for SDS diagram
\newcommand\mynode[6][]{
    \node[#1] (#2){
        \parbox{#3\relax}{
            \begin{center}
            \textbf{#4}\\
            #5\\
            \footnotesize{#6}
            \end{center}}};}



%-----------------------------------------------------------------------------%
% Headers and Footers
\automark{section}
\ohead{\href{http://sdis.dpaw.wa.gov.au/documents/progressreport/1603/}{Progress Report SP 2014-001
}}
\chead{\href{http://sdis.dpaw.wa.gov.au}{SDIS}} % center header ONLINE
\ihead{\href{http://sdis.dpaw.wa.gov.au}{
        \includegraphics[scale=0.4]{/mnt/projects/sdis/staticfiles/img/logo-dpaw.png}}}
\ifoot{\textbf{Printed}~Mon, 11 Jul 2016 09:13:14 +0800} % inner/left footer
\cfoot{} % center footer
\ofoot{\pagemark} % outer/right footer
\pagestyle{scrheadings}
\setkomafont{pageheadfoot}{\normalfont}

%-----------------------------------------------------------------------------%
\begin{document}
\raggedbottom

%-----------------------------------------------------------------------------%
% Title page
\subject{Progress Report SP 2014-001
}
\title{Understanding the changing fire environment of south-west Western
Australia
}
\subtitle{Ecosystem Science
}
\author{}
\publishers{\small
    \subsection*{Project Core Team}
\begin{tabu} {X X}
\textbf{Supervising Scientist} & Lachie Mccaw
\\
\textbf{Data Custodian} & Lachie Mccaw
\\
\textbf{Site Custodian} & Lachie Mccaw
\\
\end{tabu}


    \subsection*{Project status as of July 11, 2016, 9:13 a.m.}
\begin{tabu} {X X}
& Approved and active
\\
\end{tabu}

    
\subsection*{Document endorsements and approvals as of July 11, 2016, 9:13 a.m.}
\begin{tabu} {X X}

%\rowcolor{grantedbg}
    \textbf{Project Team} & 
    \textcolor{granted}{ granted}\\

%\rowcolor{grantedbg}
    \textbf{Program Leader} & 
    \textcolor{granted}{ granted}\\

%\rowcolor{grantedbg}
    \textbf{Directorate} & 
    \textcolor{granted}{ granted}\\

\end{tabu}



}
\uppertitleback{}
\lowertitleback{}
\date{}

%-----------------------------------------------------------------------------%
% Front matter
\frontmatter
\maketitle
%-----------------------------------------------------------------------------%
% Main matter
\mainmatter

\section*{Understanding the changing fire environment of south-west Western
Australia
}

L Mccaw, B Ward


\section*{Context}
Fire environment is the resultant effect of factors that influence the
ignition, behaviour and extent of fires in a landscape. These factors
include climate and weather, topography, vegetation and fuel, and
ignition. The climate of south-west Western Australia is becoming drier
and warmer, and reduced autumn and winter rainfall is causing the
landscape to become drier, thereby extending the duration of the
traditional fire season. A combination of land use, socio-economic and
organisational factors has resulted in more widespread extent of lands
unburnt for two decades or more, increasing the risk of high severity
fires with adverse impacts on the community and the environment. Much of
the science linking interactions between climate, fire weather and fire
behaviour was established in the 1960s and 1970s, and there is a need to
review and update baseline information that underpins bushfire risk
management and the program of planned burning undertaken by the
Department. This project will draw upon data held by the Department and
other organisations with expertise in climate and bushfire science.



\section*{Aims}
\begin{itemize}
\itemsep1pt\parskip0pt\parsep0pt
\item
  Provide an objective basis to review and revise management guidelines
  and practices based on past research and experience during wetter
  climate phases
\item
  Provide contextual information for investigations of the role and
  effects of fire in the south-west Australian environment
\end{itemize}



\section*{Progress}
\begin{itemize}
\itemsep1pt\parskip0pt\parsep0pt
\item
  Preliminary analysis of trends in the Soil Dryness Index from 2000 to
  2014 suggest that the duration of the peak dryness period when the
  Index exceeds 150 mm has increased by up to 30 days in the southern
  forests represented by Bridgetown and Pemberton observation sites.
  Observations from Pearce, Bickley and Rocky Gully show a reduced
  number of days in the peak dryness range. These trends would be
  explained by change in the spatial pattern of summer rainfall.
\item
  The spread and behaviour of the Waroona bushfire in early January 2016
  was reconstructed in order to identify significant fuel and weather
  factors that influenced the behaviour of the fire during different
  phases of its spread. Findings were presented to the Waroona Special
  Bushfire Inquiry in April 2016. Fire behaviour data have been made
  available to the Bushfire and Natural Hazards Cooperative Research
  Centre for a project on coupled fire-atmosphere modelling, and for a
  national project evaluating the performance of bushfire simulators.
\item
  In conjunction with Warren Region and Fire Management Services Branch,
  an adaptive management program has been developed to facilitate and
  evaluate prescribed burning in young regrowth stands of jarrah and
  karri. Achievement of prescribed burning objectives is being monitored
  using ground-based methods and assessment of burn severity derived
  from satellite remote sensing, and where possible linked to existing
  Forestcheck monitoring.
\end{itemize}



\section*{Management implications}
Understanding the factors that influence the location and timing of
bushfire ignitions is important for developing effective management
strategies to minimise the risks posed by unplanned fires, and to guide
the level of resourcing required for bushfire suppression in different
management areas. Lightning is an important cause of bushfire ignition
in south-west Western Australia and the area burnt by lightning-caused
fires has been disproportionately large relative to the number of
ignitions during the past decade. Better understanding of the links
between climatic patterns and lightning ignition could provide advance
warning of above-normal activity and the opportunity for improved
preparation and resource deployment.

The increased occurrence of large and damaging bushfires in the past
five years has led to a re-focus on the importance of managing fuels
with prescribed fire. In order to achieve a safe and effective
prescribed burning program there is a need to understand how weather and
climate influence opportunities for burning, and how these opportunities
may be changing over time.



\section*{Future directions}
\begin{itemize}
\itemsep1pt\parskip0pt\parsep0pt
\item
  Finalise and submit a manuscript analysing temporal and spatial
  patterns of lightning ignition for the Warren Region, and continue to
  investigate climatic factors associated with lightning ignition.
\item
  Further analyse data to investigate trends in fuel moisture content
  and soil dryness during the past 30 years.
\item
  Prepare a manuscript examining weather and fire behaviour during the
  Waroona bushfire in collaboration with co-authors from the Bureau of
  Meteorology.
\item
  Monitor and report on the outcomes of the adaptive management trial of
  prescribed burning in regrowth forest.
\end{itemize}



%-----------------------------------------------------------------------------%
% Back matter
%\backmatter
\end{document}
%-----------------------------------------------------------------------------%


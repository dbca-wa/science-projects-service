
\documentclass[version=last,
    paper=a4, % paper size
    10pt, % default font size
    usenames,
    dvipsnames,
    oneside, % ONLINE
    headings=openany, % open chapters on odd and even pages
    %toc=chapterentrywithdots, % Table of Contents style
    %BCOR=7mm, % PRINT Binding Correction
    %DIV=13, % typearea 161.54 mm x 228.46 mm, top margin 22.85 mm, inner margin 16.15 mm
    %DIV=14, % 165.00 233.36 21.21 15.00
    DIV=15 % 168.00 237.60 19.80 14.00
]{scrbook}
\usepackage{typearea}
\usepackage[automark,headsepline,footsepline]{scrlayer-scrpage} % Headers and footers

%%
%% Fonts, encoding, spacing, indentation
%%
\usepackage{txfonts}
\renewcommand{\familydefault}{\sfdefault} % Default to Sans Serif font
\usepackage[english]{babel}
\usepackage[T1]{fontenc}
\usepackage[utf8]{inputenc}

% Paragraph spacing
%\usepackage{parskip}    % Paragraph spacing
%\setlength{\parindent}{0em} % Don't indent paragraphs - ONLINE
%\setlength{\parskip}{1.3 ex plus 0.5ex minus 0.3ex} % 1-1.8 ex vertical space between paragraphs - ONLINE

% Spacing of headings
%\RedeclareSectionCommand[afterskip=3pt]{section} % less space after section
%\RedeclareSectionCommand[beforeskip=0cm]{subsection} % less space between HRule and project name
%\RedeclareSectionCommand[afterskip=0.1\baselineskip]{subsubsection} % less space after progressreport subheadings

% Table font size
\usepackage{etoolbox}
\AtBeginEnvironment{longtabu}{\footnotesize}{}{}

%%
%% Tables, columns, layout
%%
\usepackage{multicol}   % 2 col publications
\usepackage{pdflscape}  % Landscape pages
\usepackage{pdfpages}   % Include PDFs
\usepackage{hanging}    % hanging paragraphs for publications
%\usepackage{titletoc}   % Required for manipulating the table of contents
\setcounter{tocdepth}{2} % TOC list down to section
\usepackage{enumerate}  % Enumerations
\usepackage{enumitem}   % Enumerations
\usepackage{longtable}  % Multipage table
\usepackage{tabu}       %
\setlength{\tabulinesep}{1.5mm} % Consistent vertical spacing in tabu

%%
%% Graphics, images, colours
%%
\usepackage{graphicx} % embedded images
\usepackage{eso-pic} %
\usepackage{colortbl} % define custom named colours
\definecolor{RedFire}{RGB}{146,25,28}
\definecolor{ParksWildlife}{RGB}{0,85,144}
\definecolor{successbg}{RGB}{223,240,216}
\definecolor{errorbg}{RGB}{242,222,222}
\definecolor{warningbg}{RGB}{252,248,227}
\definecolor{infobg}{RGB}{217,237,247}
\definecolor{muted}{RGB}{153,153,153}
\definecolor{success}{RGB}{70,136,71}
\definecolor{error}{RGB}{185,74,72}
\definecolor{warning}{RGB}{192,152,83}
\definecolor{info}{RGB}{58,135,173}

\definecolor{required}{RGB}{192,152,83}
\definecolor{requiredbg}{RGB}{252,248,227}
\definecolor{denied}{RGB}{185,74,72}
\definecolor{deniedbg}{RGB}{242,222,222}
\definecolor{granted}{RGB}{70,136,71}
\definecolor{grantedbg}{RGB}{223,240,216}
\definecolor{not reqiured}{RGB}{153,153,153}
\definecolor{not requiredbg}{RGB}{255,255,255}

\usepackage{tikz} % Drawing
\usetikzlibrary{arrows,shapes,positioning,shadows,trees}

%%
%% Links, URLs
%%
\usepackage[
    linktoc=all,
    %colorlinks=false,  %PRINT
    colorlinks=true, % ONLINE
    linkcolor=RedFire, % ONLINE
    urlcolor=ParksWildlife, % ONLINE
    pdftitle=Progress Report SP 2011-001 (FY 2015-2016)
]{hyperref}

% Black magic to linebreak URLs
\usepackage{url}
\makeatletter
\g@addto@macro{\UrlBreaks}{\UrlOrds}
\makeatother

%%
%% Custom macros
%%
% Thick Horizontal rule
\newcommand{\HRule}{\vspace{8mm}\\\noindent\rule{\linewidth}{0.1pt}}

% Custom Tikz node for SDS diagram
\newcommand\mynode[6][]{
    \node[#1] (#2){
        \parbox{#3\relax}{
            \begin{center}
            \textbf{#4}\\
            #5\\
            \footnotesize{#6}
            \end{center}}};}



%-----------------------------------------------------------------------------%
% Headers and Footers
\automark{section}
\ohead{\href{http://sdis.dpaw.wa.gov.au/documents/progressreport/1642/}{Progress Report SP 2011-001
}}
\chead{\href{http://sdis.dpaw.wa.gov.au}{SDIS}} % center header ONLINE
\ihead{\href{http://sdis.dpaw.wa.gov.au}{
        \includegraphics[scale=0.4]{/mnt/projects/sdis/staticfiles/img/logo-dpaw.png}}}
\ifoot{\textbf{Printed}~Mon, 4 Jul 2016 16:15:51 +0800} % inner/left footer
\cfoot{} % center footer
\ofoot{\pagemark} % outer/right footer
\pagestyle{scrheadings}
\setkomafont{pageheadfoot}{\normalfont}

%-----------------------------------------------------------------------------%
\begin{document}
\raggedbottom

%-----------------------------------------------------------------------------%
% Title page
\subject{Progress Report SP 2011-001
}
\title{Taxonomy of selected families including legumes, grasses and lilies
}
\subtitle{Plant Science and Herbarium
}
\author{}
\publishers{\small
    \subsection*{Project Core Team}
\begin{tabu} {X X}
\textbf{Supervising Scientist} & Terry Macfarlane
\\
\textbf{Data Custodian} & Terry Macfarlane
\\
\textbf{Site Custodian} & Terry Macfarlane
\\
\end{tabu}


    \subsection*{Project status as of July 4, 2016, 4:15 p.m.}
\begin{tabu} {X X}
& Approved and active
\\
\end{tabu}

    
\subsection*{Document endorsements and approvals as of July 4, 2016, 4:15 p.m.}
\begin{tabu} {X X}

%\rowcolor{grantedbg}
    \textbf{Project Team} & 
    \textcolor{granted}{ granted}\\

%\rowcolor{grantedbg}
    \textbf{Program Leader} & 
    \textcolor{granted}{ granted}\\

%\rowcolor{grantedbg}
    \textbf{Directorate} & 
    \textcolor{granted}{ granted}\\

\end{tabu}



}
\uppertitleback{}
\lowertitleback{}
\date{}

%-----------------------------------------------------------------------------%
% Front matter
\frontmatter
\maketitle
%-----------------------------------------------------------------------------%
% Main matter
\mainmatter

\section*{Taxonomy of selected families including legumes, grasses and lilies
}

T Macfarlane


\section*{Context}
Successful conservation of the flora requires that the conservation
units equate to properly defined, described and named taxa. There are
numerous known and suspected unnamed taxa in the grass, legume and lily
families, as well as numerous cases where keying problems or anomalous
distributions indicate that taxonomic review is required. This is true
of various parts of the families but the main current focus is on
Lepilaena, \emph{Thysanotus}, \emph{Wurmbea},~\emph{Lomandra},
\emph{Neurachne} and \emph{Trithuria}.



\section*{Aims}
\begin{itemize}
\itemsep1pt\parskip0pt\parsep0pt
\item
  Identify plant groups where there are taxonomic issues that need to be
  resolved, including apparently new species to be described and
  unsatisfactory taxonomy that requires clarification.
\item
  Carry out taxonomic revisions using fieldwork, herbarium collections
  and laboratory work, resulting in published journal articles.
\end{itemize}



\section*{Progress}
\begin{itemize}
\itemsep1pt\parskip0pt\parsep0pt
\item
  \emph{Lepilaena}(Potamogetonaceae): a genus of aquatic plants~that
  have been difficult to identify has been studied from all Australian
  herbarium specimen holdings and species boundaries defined.
  Paper~drafted for one new Western Australia species. Other papers in
  preparation on pollen structure and variation, and a revision of the
  genus.
\item
  \emph{Wurmbea} (Colchicaceae): continued field work to assess
  conservation status of poorly known species and obtain photos.
  Progress continued on writing paper~to describe thirty~new species.
\item
  Hydatellaceae: paper published~on phylogeography of \emph{Trithuria
  submersa}; paper in preparation on molecular phylogeny and genetic
  variation in \emph{Trithuria australis}.
\item
  Poaceae: continuing~research collaboration on \emph{Neurachne} and the
  evolution of C4 photosynthesis.
\item
  \emph{Thysanotus}(Asparagaceae): Review of the taxonomy of the twining
  species, the \emph{T. patersonii} group, with extensive filed work,
  preliminary species defining, and preparation of samples for DNA
  analysis.
\item
  \emph{Lomandra\emph{~}}(Asparagaceae): field and herbarium work
  on~\emph{L. suaveolens~}group.~
\item
  Haemodoraceae: paper describing~seven~new Kimberley species published.
\item
  Asparagales: paper on a neglected taxonomically useful flower feature
  in several plant families still~in press.
\item
  Eremosynaceae: book chapter summarising taxonomy and morphology
  published.
\end{itemize}



\section*{Management implications}
\begin{itemize}
\itemsep1pt\parskip0pt\parsep0pt
\item
  Identification of species known or suspected to have a restricted
  distribution will enable re-assessment of the conservation status and
  improve management effectiveness.
\item
  Improved identification tools will enable more effective
  identification of species and the subsequent assessment of their
  conservation status.
\end{itemize}



\section*{Future directions}
\begin{itemize}
\itemsep1pt\parskip0pt\parsep0pt
\item
  Complete and submit papers describing new species of \emph{Wurmbea},
  \emph{Thysanotus, Lepilaena} and \emph{Lomandra}. Conduct appropriate
  field searches for~species or~populations that are insufficiently
  known.
\item
  Continue to revise plant groups and investigate via field and
  herbarium studies various putatively new species in order to improve
  knowledge of the flora, provide stable plant names and provide means
  of identifying species. Current targets are new species of
  \emph{Rytidosperma} (Poaceae) and reviews of
  \emph{Arthropodium}and\emph{Lepilaena} in Western Australia.
\item
  Publish information on selected plant groups for general audiences.
  Articles are in preparation for~\emph{Wurmbea} and \emph{Thysanotus}.
\end{itemize}



%-----------------------------------------------------------------------------%
% Back matter
%\backmatter
\end{document}
%-----------------------------------------------------------------------------%


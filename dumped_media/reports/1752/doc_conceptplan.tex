
\documentclass[version=last,
    paper=a4, % paper size
    10pt, % default font size
    usenames,
    dvipsnames,
    oneside, % ONLINE
    headings=openany, % open chapters on odd and even pages
    %toc=chapterentrywithdots, % Table of Contents style
    %BCOR=7mm, % PRINT Binding Correction
    %DIV=13, % typearea 161.54 mm x 228.46 mm, top margin 22.85 mm, inner margin 16.15 mm
    %DIV=14, % 165.00 233.36 21.21 15.00
    DIV=15 % 168.00 237.60 19.80 14.00
]{scrbook}
\usepackage{typearea}
\usepackage[automark,headsepline,footsepline]{scrlayer-scrpage} % Headers and footers

%%
%% Fonts, encoding, spacing, indentation
%%
\usepackage{txfonts}
\renewcommand{\familydefault}{\sfdefault} % Default to Sans Serif font
\usepackage[english]{babel}
\usepackage[T1]{fontenc}
\usepackage[utf8]{inputenc}

% Paragraph spacing
%\usepackage{parskip}    % Paragraph spacing
%\setlength{\parindent}{0em} % Don't indent paragraphs - ONLINE
%\setlength{\parskip}{1.3 ex plus 0.5ex minus 0.3ex} % 1-1.8 ex vertical space between paragraphs - ONLINE

% Spacing of headings
%\RedeclareSectionCommand[afterskip=3pt]{section} % less space after section
%\RedeclareSectionCommand[beforeskip=0cm]{subsection} % less space between HRule and project name
%\RedeclareSectionCommand[afterskip=0.1\baselineskip]{subsubsection} % less space after progressreport subheadings

% Table font size
\usepackage{etoolbox}
\AtBeginEnvironment{longtabu}{\footnotesize}{}{}

%%
%% Tables, columns, layout
%%
\usepackage{multicol}   % 2 col publications
\usepackage{pdflscape}  % Landscape pages
\usepackage{pdfpages}   % Include PDFs
\usepackage{hanging}    % hanging paragraphs for publications
%\usepackage{titletoc}   % Required for manipulating the table of contents
\setcounter{tocdepth}{2} % TOC list down to section
\usepackage{enumerate}  % Enumerations
\usepackage{enumitem}   % Enumerations
\usepackage{longtable}  % Multipage table
\usepackage{tabu}       %
\setlength{\tabulinesep}{1.5mm} % Consistent vertical spacing in tabu

%%
%% Graphics, images, colours
%%
\usepackage{graphicx} % embedded images
\usepackage{eso-pic} %
\usepackage{colortbl} % define custom named colours
\definecolor{RedFire}{RGB}{146,25,28}
\definecolor{ParksWildlife}{RGB}{0,85,144}
\definecolor{successbg}{RGB}{223,240,216}
\definecolor{errorbg}{RGB}{242,222,222}
\definecolor{warningbg}{RGB}{252,248,227}
\definecolor{infobg}{RGB}{217,237,247}
\definecolor{muted}{RGB}{153,153,153}
\definecolor{success}{RGB}{70,136,71}
\definecolor{error}{RGB}{185,74,72}
\definecolor{warning}{RGB}{192,152,83}
\definecolor{info}{RGB}{58,135,173}

\definecolor{required}{RGB}{192,152,83}
\definecolor{requiredbg}{RGB}{252,248,227}
\definecolor{denied}{RGB}{185,74,72}
\definecolor{deniedbg}{RGB}{242,222,222}
\definecolor{granted}{RGB}{70,136,71}
\definecolor{grantedbg}{RGB}{223,240,216}
\definecolor{not reqiured}{RGB}{153,153,153}
\definecolor{not requiredbg}{RGB}{255,255,255}

\usepackage{tikz} % Drawing
\usetikzlibrary{arrows,shapes,positioning,shadows,trees}

%%
%% Links, URLs
%%
\usepackage[
    linktoc=all,
    %colorlinks=false,  %PRINT
    colorlinks=true, % ONLINE
    linkcolor=RedFire, % ONLINE
    urlcolor=ParksWildlife, % ONLINE
    pdftitle=Concept Plan SP 2016-015
]{hyperref}

% Black magic to linebreak URLs
\usepackage{url}
\makeatletter
\g@addto@macro{\UrlBreaks}{\UrlOrds}
\makeatother

%%
%% Custom macros
%%
% Thick Horizontal rule
\newcommand{\HRule}{\vspace{8mm}\\\noindent\rule{\linewidth}{0.1pt}}

% Custom Tikz node for SDS diagram
\newcommand\mynode[6][]{
    \node[#1] (#2){
        \parbox{#3\relax}{
            \begin{center}
            \textbf{#4}\\
            #5\\
            \footnotesize{#6}
            \end{center}}};}



\usepackage[automark,headsepline,footsepline,plainfootsepline]{scrlayer-scrpage}
\automark*[section]{}
\addtokomafont{pageheadfoot}{\normalfont\footnotesize\sffamily} % Don't italicise
\renewcommand{\chaptermark}[1]{\markleft{#1}{}}     % Chapter: suppress numbering
\renewcommand{\sectionmark}[1]{\markright{#1}{}}    % Section: suppress numbering

% Header (inner, center, outer)
\ihead{\href{http://sdis.dpaw.wa.gov.au/documents/conceptplan/1752/}{Concept Plan SP 2016-015}}
%\chead{\href{http://sdis.dpaw.wa.gov.au}{Science Directorate Information System}}
\ohead{\href{https://www.dpaw.wa.gov.au/about-us/science-and-research}{\includegraphics[height=6mm, keepaspectratio]{/mnt/projects/sdis/staticfiles/img/logo-dpaw.png}}}

% Footer (inner, center, outer)
\ifoot{\textbf{Printed}~Wed, 16 Aug 2017 11:45:04 +0800} % inner/left footer
\cfoot{}
\ofoot[\bfseries\thepage]{\bfseries\thepage}        % Page number (also [plain])


\pagestyle{scrheadings}
\setkomafont{pageheadfoot}{\normalfont}

%-----------------------------------------------------------------------------%
\begin{document}
\raggedbottom

%-----------------------------------------------------------------------------%
% Title page
\subject{Concept Plan SP 2016-015
}
\title{Is restoration working? An ecological assessment
}
\subtitle{Plant Science and Herbarium
}
\author{}
\publishers{\small
    \subsection*{Project Core Team}
\begin{tabu} {X X}
\textbf{Supervising Scientist} & Dave Coates
\\
\textbf{Data Custodian} & Melissa Millar
\\
\textbf{Site Custodian} & 
\\
\end{tabu}


    \subsection*{Project status as of Aug. 16, 2017, 11:45 a.m.}
\begin{tabu} {X X}
& Approved and active
\\
\end{tabu}

    
\subsection*{Document endorsements and approvals as of Aug. 16, 2017, 11:45 a.m.}
\begin{tabu} {X X}

%\rowcolor{grantedbg}
    \textbf{Project Team} & 
    \textcolor{granted}{ granted}\\

%\rowcolor{grantedbg}
    \textbf{Program Leader} & 
    \textcolor{granted}{ granted}\\

%\rowcolor{grantedbg}
    \textbf{Directorate} & 
    \textcolor{granted}{ granted}\\

\end{tabu}



}
\uppertitleback{}
\lowertitleback{}
\date{}

%-----------------------------------------------------------------------------%
% Front matter
\frontmatter
\maketitle
%-----------------------------------------------------------------------------%
% Main matter
\mainmatter


\section*{Is restoration working? An ecological assessment
}



\subsection*{Science and Conservation Division Program}

Plant Science and Herbarium




\subsection*{Parks and Wildlife Service}

Service 3: Conservation Partnerships




\subsection*{Aims}

The recognition of poorly defined success criteria and a lack of long
term monitoring have highlighted the need for the development of post
implementation empirical evaluations of the quality of restoration
activities.~This recognition has led to the hypothesis that the most
ecologically and genetically viable restored populations will be those
where reproductive outputs, plant pollinator interactions, levels of
genetic diversity, mating systems and patterns of pollen dispersal most
closely mimic those found in natural or undisturbed remnant vegetation.
These populations are more likely to persist in the long term and
contribute to effective ecosystem function through integration into the
broader landscape. This project aims to assess the success of
restoration in terms of ecological and genetic viability for plant
species in the Fitzgerald River-Stirling Range region of Western
Australia, where significant investment is being made in restoring
connectivity at a landscape scale. The project intends to compare mating
systems, genetic diversity and pollen dispersal in restored sites with
those of undisturbed natural vegetation. Significantly, the project
moves measures of restoration success beyond that of population
establishment and survival to incorporate the evolutionary processes
that provide long term resilience, persistence and functional
integration of restored populations into broader landscapes.




\subsection*{Expected outcome}

A number of ecological and genetic assessments will be made for each of
six target species (\emph{Banksia media}, \emph{Hakea nitida},
\emph{Hakea laurina}, \emph{Melaleuca acuminata}, \emph{Eucalyptus
occidentalis} and \emph{Acacia cyclops}) at up to three Gondwana Link
restoration sites established with differing seed and seedling
establishment regimes and for differing lengths of time and in nearby
natural remnant populations. Assessments will include,

\begin{itemize}
\itemsep1pt\parskip0pt\parsep0pt
\item
  measures of reproductive output including seed viability and vigour,
\item
  levels of genetic diversity as measured by nuclear microsatellite
  markers,
\item
  mating system parameters and
\item
  for two proteaceous target species (\emph{Banksia media}, and
  \emph{Hakea nitida}), the presence and behaviour of animal pollinators
  and how well pollinator services effect pollen dispersal within and
  among populations.
\end{itemize}

Benchmarking of ecological and genetic processes in restoration
populations against plants in reference populations of remnant natural
vegetation will allow determination of

\begin{itemize}
\itemsep1pt\parskip0pt\parsep0pt
\item
  when population reproductive output is sufficient to maintain
  outcrossing and combat inbreeding,
\item
  if comparable amounts of genetic variation has been captured,
\item
  when augmentation of numbers and genetics is no longer required, and
\item
  when plant-pollinator interactions and pollen dispersal are sufficient
  to maintain genetic health and connectivity within and across
  populations for animal pollinated plants.
\end{itemize}

Monitoring these processes for differing restoration establishment
regimes allows for assessment of the usefulness of specific restoration
activities and the determination of adaptive management and future
restoration actions that would be beneficial and most cost effective.




\subsection*{Strategic context}

This project intends to apply advanced knowledge of how ecological and
genetic viability of restored plant\\populations perform under differing
establishment regimes and to provide improved guidelines to assist in
adaptive management of future restoration activities that are cost
effective, achieve long term resilience and persistence, integrated
functionality of restored populations at regional landscape scales and
greater restoration success.




\subsection*{Expected collaborations}

This project is a collaboration involving The University of Western
Australia, the Botanic Gardens and Parks Authority and Gondwana Link
Limited. The involvement of students from a range of institutions is
envisioned.


\subsection*{Proposed period of the project}
June 1, 2016 -- July 1, 2019



\subsection*{Staff time allocation }



\begin{longtabu} to \linewidth { |  X | X | X | X | }
\hline
\rowcolor{infobg}
Role & Year 1 & Year 2 & Year 3\\
\hline
\endhead



Scientist & 1.1 & 1.1 & 1.1\\



Technical &  &  & \\



Volunteer &  &  & \\



Collaborator &  &  & \\


\hline
\end{longtabu}



\subsection*{Indicative operating budget }



\begin{longtabu} to \linewidth { |  X | X | X | X | }
\hline
\rowcolor{infobg}
Source & Year 1 & Year 2 & Year 3\\
\hline
\endhead



Consolidated Funds (DPaW) &  &  & \\



External Funding & 173,000 & 173,000 & 173,000\\


\hline
\end{longtabu}






%-----------------------------------------------------------------------------%
% Back matter
%\backmatter
\end{document}
%-----------------------------------------------------------------------------%


\documentclass[version=last,
    paper=a4, % paper size
    10pt, % default font size
    usenames,
    dvipsnames,
    oneside, % ONLINE
    headings=openany, % open chapters on odd and even pages
    %toc=chapterentrywithdots, % Table of Contents style
    %BCOR=7mm, % PRINT Binding Correction
    %DIV=13, % typearea 161.54 mm x 228.46 mm, top margin 22.85 mm, inner margin 16.15 mm
    %DIV=14, % 165.00 233.36 21.21 15.00
    DIV=15 % 168.00 237.60 19.80 14.00
]{scrbook}
\usepackage{typearea}
\usepackage[automark,headsepline,footsepline]{scrlayer-scrpage} % Headers and footers

%%
%% Fonts, encoding, spacing, indentation
%%
\usepackage{txfonts}
\renewcommand{\familydefault}{\sfdefault} % Default to Sans Serif font
\usepackage[english]{babel}
\usepackage[T1]{fontenc}
\usepackage[utf8]{inputenc}

% Paragraph spacing
%\usepackage{parskip}    % Paragraph spacing
%\setlength{\parindent}{0em} % Don't indent paragraphs - ONLINE
%\setlength{\parskip}{1.3 ex plus 0.5ex minus 0.3ex} % 1-1.8 ex vertical space between paragraphs - ONLINE

% Spacing of headings
%\RedeclareSectionCommand[afterskip=3pt]{section} % less space after section
%\RedeclareSectionCommand[beforeskip=0cm]{subsection} % less space between HRule and project name
%\RedeclareSectionCommand[afterskip=0.1\baselineskip]{subsubsection} % less space after progressreport subheadings

% Table font size
\usepackage{etoolbox}
\AtBeginEnvironment{longtabu}{\footnotesize}{}{}

%%
%% Tables, columns, layout
%%
\usepackage{multicol}   % 2 col publications
\usepackage{pdflscape}  % Landscape pages
\usepackage{pdfpages}   % Include PDFs
\usepackage{hanging}    % hanging paragraphs for publications
%\usepackage{titletoc}   % Required for manipulating the table of contents
\setcounter{tocdepth}{2} % TOC list down to section
\usepackage{enumerate}  % Enumerations
\usepackage{enumitem}   % Enumerations
\usepackage{longtable}  % Multipage table
\usepackage{tabu}       %
\setlength{\tabulinesep}{1.5mm} % Consistent vertical spacing in tabu

%%
%% Graphics, images, colours
%%
\usepackage{graphicx} % embedded images
\usepackage{eso-pic} %
\usepackage{colortbl} % define custom named colours
\definecolor{RedFire}{RGB}{146,25,28}
\definecolor{ParksWildlife}{RGB}{0,85,144}
\definecolor{successbg}{RGB}{223,240,216}
\definecolor{errorbg}{RGB}{242,222,222}
\definecolor{warningbg}{RGB}{252,248,227}
\definecolor{infobg}{RGB}{217,237,247}
\definecolor{muted}{RGB}{153,153,153}
\definecolor{success}{RGB}{70,136,71}
\definecolor{error}{RGB}{185,74,72}
\definecolor{warning}{RGB}{192,152,83}
\definecolor{info}{RGB}{58,135,173}

\definecolor{required}{RGB}{192,152,83}
\definecolor{requiredbg}{RGB}{252,248,227}
\definecolor{denied}{RGB}{185,74,72}
\definecolor{deniedbg}{RGB}{242,222,222}
\definecolor{granted}{RGB}{70,136,71}
\definecolor{grantedbg}{RGB}{223,240,216}
\definecolor{not reqiured}{RGB}{153,153,153}
\definecolor{not requiredbg}{RGB}{255,255,255}

\usepackage{tikz} % Drawing
\usetikzlibrary{arrows,shapes,positioning,shadows,trees}

%%
%% Links, URLs
%%
\usepackage[
    linktoc=all,
    %colorlinks=false,  %PRINT
    colorlinks=true, % ONLINE
    linkcolor=RedFire, % ONLINE
    urlcolor=ParksWildlife, % ONLINE
    pdftitle=Progress Report SP 2012-023 (FY 2012-2013)
]{hyperref}

% Black magic to linebreak URLs
\usepackage{url}
\makeatletter
\g@addto@macro{\UrlBreaks}{\UrlOrds}
\makeatother

%%
%% Custom macros
%%
% Thick Horizontal rule
\newcommand{\HRule}{\vspace{8mm}\\\noindent\rule{\linewidth}{0.1pt}}

% Custom Tikz node for SDS diagram
\newcommand\mynode[6][]{
    \node[#1] (#2){
        \parbox{#3\relax}{
            \begin{center}
            \textbf{#4}\\
            #5\\
            \footnotesize{#6}
            \end{center}}};}



%-----------------------------------------------------------------------------%
% Headers and Footers
\automark{section}
\ohead{\href{http://sdis.dpaw.wa.gov.au/documents/progressreport/1015/}{Progress Report SP 2012-023
}}
\chead{\href{http://sdis.dpaw.wa.gov.au}{SDIS}} % center header ONLINE
\ihead{\href{http://sdis.dpaw.wa.gov.au}{
        \includegraphics[scale=0.4]{/mnt/projects/sdis/staticfiles/img/logo-dpaw.png}}}
\ifoot{\textbf{Printed}~Tue, 2 Aug 2016 14:41:36 +0800} % inner/left footer
\cfoot{} % center footer
\ofoot{\pagemark} % outer/right footer
\pagestyle{scrheadings}
\setkomafont{pageheadfoot}{\normalfont}

%-----------------------------------------------------------------------------%
\begin{document}
\raggedbottom

%-----------------------------------------------------------------------------%
% Title page
\subject{Progress Report SP 2012-023
}
\title{Feral cat control and numbat recovery in Dryandra woodland and other
sites
}
\subtitle{Animal Science
}
\author{}
\publishers{\small
    \subsection*{Project Core Team}
\begin{tabu} {X X}
\textbf{Supervising Scientist} & Tony Friend
\\
\textbf{Data Custodian} & 
\\
\textbf{Site Custodian} & 
\\
\end{tabu}


    \subsection*{Project status as of Aug. 2, 2016, 2:41 p.m.}
\begin{tabu} {X X}
& Approved and active
\\
\end{tabu}

    
\subsection*{Document endorsements and approvals as of Aug. 2, 2016, 2:41 p.m.}
\begin{tabu} {X X}

%\rowcolor{grantedbg}
    \textbf{Project Team} & 
    \textcolor{granted}{ granted}\\

%\rowcolor{requiredbg}
    \textbf{Program Leader} & 
    \textcolor{required}{ required}\\

%\rowcolor{requiredbg}
    \textbf{Directorate} & 
    \textcolor{required}{ required}\\

\end{tabu}



}
\uppertitleback{}
\lowertitleback{}
\date{}

%-----------------------------------------------------------------------------%
% Front matter
\frontmatter
\maketitle
%-----------------------------------------------------------------------------%
% Main matter
\mainmatter

\section*{Feral cat control and numbat recovery in Dryandra woodland and other
sites
}

A Friend


\section*{Context}
The Dryandra numbat population showed a strong positive response to fox
control during the 1980s and early 1990s, but then declined in 1993 and
has since remained at low levels. Numbers dropped even further between
2006 and 2009 despite continued fox baiting. In response to evidence of
cat predation on radio-collared numbats (2008-09) and Nicky Marlow's
finding that cats are responsible for 60\% of woylie deaths at Dryandra,
an intensive study of predation on numbats and woylies there has been
undertaken under this project.

This project also investigates the effect of cat baiting using
\emph{Eradicat}® baits on GPS-collared cats, rates of mortality amongst
woylies and numbats and their numbers in Dryandra Woodland. Cat-baiting
efficacy is also being investigated at Tutanning NR. Part of this study
is an assessment of the risk posed to other native mammals, particularly
dasyurids, in wheatbelt reserves, by the use of Eradicat® baits.



\section*{Aims}
\begin{itemize}
\itemsep1pt\parskip0pt\parsep0pt
\item
  To define the risks of using Eradicat baits to non-target species in
  Dryandra Woodland, particularly to threatened species like the
  red-tailed phascogale.
\item
  To implement cat control by baiting at Dryandra and measure the effect
  on numbat, woylie, cat and fox population numbers and on causes of
  death in numbats and woylies.
\item
  To determine the effect of ceasing fox control on feral cat movements
  and survival at Dragon Rocks NR.
\end{itemize}



\section*{Progress}
\begin{itemize}
\itemsep1pt\parskip0pt\parsep0pt
\item
  Ten numbats are currently radio-collared in Dryandra Woodland. They
  are checked at least once a fortnight so that predation events can be
  quickly verified.
\item
  Ten cats fitted with GPS collars at Dryandra (five) and Tutanning
  (five) were recovered in June and July 2013 and the data retrieved.
  Five more cats were collared in Tutanning in April 2014, prior to
  baiting in May. Three more cats were collared in Dryandra in May 2014,
  prior to the baiting, carried out in late May.
\item
  Eradicat® baits were distributed in Dryandra and Tutanning in groups
  of 50 baits spaced on a 1 km grid, in line with aerial baiting
  protocols, and also along boundary firebreaks at 200 m intervals.
  Seven of the eight cats were still alive two weeks after the baiting,
  and the other had been killed by a cow.
\item
  Sensor cameras have been deployed on the track system at Dryandra and
  Tutanning, and activated before and after Eradicat® baiting events to
  measure changes in cat and fox activity.
\item
  Four chuditch were collared in Dryandra, and nine red-tailed
  phascogales were collared in Tutanning prior to baiting at each site.
\item
  Twelve cats were fitted with GPS collars in June 2013 at Dragon Rocks
  for six months while fox control was suspended in half of the reserve.
  The data from 10 of the collars has been retrieved. Three cats died
  before the end of the experimental period.
\item
  Sensor cameras were deployed on two 3 x 6 grids with 2km spacing in
  Dragon Rock NR, one in the northern half and one in the south and run
  before and after the suspension of baiting in the north of the
  reserve. An array of cameras at 3 km spacing on tracks was also
  deployed before and after the suspension.
\end{itemize}



\section*{Management implications}
This project has shown that cats are the most important predators of
both woylies and Numbats in Dryandra. Cat control is essential and
should be implemented in such a way as to minimise the impact on
non-target species. Eradicat® can be used safely in the presence of
chuditch and red-tailed phascogales. Most cats in Dryandra, Tutanning
and Dragon Rocks spend significant time in adjacent farmland, reducing
their exposure to baits laid within the reserves.



\section*{Future directions}
\begin{itemize}
\itemsep1pt\parskip0pt\parsep0pt
\item
  Cats will be fitted with GPS collars in Dryandra and Tutanning prior
  to an Eradicat® baiting in March 2015, to determine the efficacy of
  baiting in early autumn.
\item
  Monitoring of radio-collared numbats will continue in order to
  accumulate evidence on causes of death in this species over a longer
  period.
\item
  Further non-target baiting trials with Eradicat® will be implemented,
  aimed at other species, particularly the mardo.
\item
  Trapping on Western Shield traplines will continue to monitor woylie
  numbers.
\item
  Monitoring of known numbat populations will be continued by involving
  district staff and community members and handing over monitoring
  responsibility to districts. Regular monitoring surveys will be
  essential in Dragon Rocks, Dryandra, Boyagin, Tutanning and Cocanarup.
  Other areas will be included if possible
\end{itemize}



%-----------------------------------------------------------------------------%
% Back matter
%\backmatter
\end{document}
%-----------------------------------------------------------------------------%


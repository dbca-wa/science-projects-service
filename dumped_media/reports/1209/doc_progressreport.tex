\documentclass[version=last, paper=a4, DIV=18, usenames, dvipsnames]{scrartcl}
\usepackage{txfonts}
\usepackage{pdflscape}
\usepackage{pdfpages}
\usepackage[english]{babel} % English language/hyphenation
%%% Bootstrap colors
\definecolor{RedFire}{RGB}{146,25,28}
\definecolor{ParksWildlife}{RGB}{0,85,144}
\definecolor{successbg}{RGB}{223,240,216}
\definecolor{errorbg}{RGB}{242,222,222}
\definecolor{warningbg}{RGB}{252,248,227}
\definecolor{infobg}{RGB}{217,237,247}
\definecolor{muted}{RGB}{153,153,153}
\definecolor{success}{RGB}{70,136,71}
\definecolor{error}{RGB}{185,74,72}
\definecolor{warning}{RGB}{192,152,83}
\definecolor{info}{RGB}{58,135,173}
\usepackage[colorlinks=true,pdftitle=doc\_progressreport.pdf,linktoc=all,linkcolor=RedFire,urlcolor=ParksWildlife]{hyperref}
\usepackage{colortbl}
\usepackage{longtable}
\usepackage{tabu}
\setlength{\tabulinesep}{1.5mm}
\usepackage{enumerate}
\usepackage{enumitem}
\usepackage{fancyhdr}
\usepackage{lastpage}
\usepackage{graphicx}
\usepackage{eso-pic}
\usepackage{hyphenat}



%%% Custom headers/footers (fancyhdr package)
\fancypagestyle{plain}{
\fancyhf{}
\setlength\headheight{40pt}
\renewcommand{\headrulewidth}{0.1pt}
\renewcommand{\footrulewidth}{0.1pt}



    \fancyhead[L]{ \href{http://sdis.dpaw.wa.gov.au/documents/progressreport/1209/download/}{} \newline }
\fancyhead[R]{ \hfill\href{http://www.dpaw.wa.gov.au}{Department of Parks and Wildlife}\newline\href{http://sdis.dpaw.wa.gov.au}{Pythia}}




\fancyfoot[L]{ \leftmark\newline\textbf{Last Modified}\textit{ }\quad\textbf{Printed}\textit{ June 23, 2014, 10:24 a.m. } }
\fancyfoot[R]{  \, \newline Page \thepage\ of \pageref{LastPage} } % Pagenumbering


}
\pagestyle{plain}


\newcommand{\HRule}{\rule{\linewidth}{0.1pt}}

\newcommand{\placetextbox}[3]{% \placetextbox{<horizontal pos>}{<vertical pos>}{<stuff>}
  \setbox0=\hbox{#3}% Put <stuff> in a box
  \AddToShipoutPictureFG*{% Add <stuff> to current page foreground
    \put(\LenToUnit{#1\paperwidth},\LenToUnit{#2\paperheight}){\vtop{{\null}\makebox[0pt][c]{#3}}}%
  }%
}%

\begin{document}

\setcounter{secnumdepth}{-1}


\begin{titlepage}
\begin{center}
% Upper part of the page
\begin{minipage}[t]{0.28\textwidth}
\begin{flushleft}
\href{http://www.dpaw.wa.gov.au}{\includegraphics[scale=0.6]{/var/www/sdis_8271/staticfiles/img/logo-dpaw.png}}
\end{flushleft}
\end{minipage}
\begin{minipage}[b]{0.7\textwidth}
\begin{flushright}
    \href{http://sdis.dpaw.wa.gov.au/documents/progressreport/1209/download/}{}) \\
\end{flushright}
\end{minipage}
\HRule \\[0.4cm]
\vfill
\textsc{\Huge Science project 2006-2 Monitoring stream biodiversity (KPI 20 of the Forest Management Plan) \newline }
\vfill
\textsc{\Huge Progress Report}

\vfill\vfill\vfill\vfill
title and summary

\vfill\vfill\vfill\vfill\vfill\vfill\vfill\vfill

\textbf{Version created on} June 23, 2014, 10:24 a.m.
\vfill
\textbf{Last Modified on}  by 
\vfill\vfill
\textbf{Report Status}\\\,
\begin{tabu} to \linewidth { | X[l] | X | }
\hline
\rowcolor{infobg}
Status & Last Updated \\
\hline
\textbf{Planning - } \\
\hline
\end{tabu}
\vfill
\textbf{Science Project Overview}\\\,
\begin{tabu} to \linewidth { | X[l] | X | }
\hline
\rowcolor{infobg}
Part & Checklist Last Updated \\
\hline
\textbf{Part A - Summary \& Approval} & bla \\
\hline
\end{tabu}

\end{center}
\end{titlepage}

\setcounter{tocdepth}{2}
\tableofcontents
\clearpage






\section{Context Summary}



The \emph{Forest Management Plan 2004-2013} includes a range of key performance indicators (KPIs) based on the internationally agreed Montreal Protocols. KPI 20 is the percentage of water bodies (e.g. stream kilometres, lake hectares) with significant variance of biodiversity from the historic range of variability. This project addresses this KPI by monitoring aquatic invertebrates in representative stream sites within the jarrah and karri forests of south-western Australia, particularly in relation to forest management practices.






\section{Aims Summary}



\begin{itemize}

  \item Assess aquatic invertebrate diversity in representative jarrah and karri forest streams against reference condition by comparing the family richness of sampled invertebrates to richness predicted by a previously constructed model (AusRivAS) developed using data from `minimally disturbed' reference sites in the same regions.

  \item Examine relationships between species richness and degree of catchment disturbance (burning and harvesting) for selected invertebrate groups.

\end{itemize}






\section{Progress}



\begin{itemize}

  \item Invertebrates from the 2013 round of sampling were identified.

  \item An article was written for \emph{LANDSCOPE} magazine, ``Saving streams of the south-west forest''.

  \item An analysis of biogeographic patterning of aquatic invertebrates in south-west forest streams was carried out in collaboration with scientists from CSIRO in Canberra. Journal article is in preparation.

\end{itemize}






\section{Management implications}



The \emph{Forest Management Plan 2004-2013} target of no stream sites with fauna significantly different to reference condition was not achieved. However, stream sites with greatest divergence in diversity from reference condition were generally in the drier parts of the northern and eastern jarrah forest or were naturally saline or acidic. Part of the reason for these sites being apparently impaired was that the AusRivAS models were produced with few reference sites in such streams, so the model is likely to have overestimated richness. However, a few stream sites were not in these categories and require further monitoring and investigation to examine the cause of the reduced diversity. There was very little evidence that the proportion of the catchment burned and/or harvested affected any of the stream invertebrate biodiversity measures or overall community composition.






\section{Future directions}



\begin{itemize}

  \item Undertake sampling of streams in 2015, with a focus on those considered to be in minimally disturbed catchments, with a view to providing long-term data on the response of aquatic invertebrate communities to declining rainfall.

  \item Continue to update fire and logging history for catchment areas.

  \item Complete and publish a paper on biogeographic patterning in aquatic invertebrate communities of south-west forests.

  \item Publish further papers examining impacts of declining rainfall and forest management.

  \item Publish report with summaries of 8 year trends (2005 to 2013) for all monitoring sites

\end{itemize}






\clearpage



\end{document}


\documentclass[version=last,
    paper=a4,                               % paper size
    10pt,                                   % default font size
    dvipsnames,
    % twoside,                                % PRINT Binding Correction
    oneside,                              % ONLINE
    headings=openany,                       % open chapters on odd and even pages
    open=any,
    BCOR=7mm,                               % PRINT Binding Correction
    %DIV=13,    % typearea 161.54mm x 228.46mm, top 22.85mm, inner 16.15mm
    %DIV=14,    % 165.00 233.36 21.21 15.00
    DIV=15,     % 168.00 237.60 19.80 14.00
    % toc=chapterentrywithdots              % Table of Contents style
]{scrbook}
\usepackage{typearea}


%------------------------------------------------------------------------------%
% Headers and footers
%------------------------------------------------------------------------------%
\usepackage[automark,headsepline,footsepline,plainfootsepline]{scrlayer-scrpage}
\automark*[section]{}
\addtokomafont{pageheadfoot}{\normalfont\footnotesize\sffamily} % Don't italicise
\renewcommand{\chaptermark}[1]{\markleft{#1}{}}     % Chapter: suppress numbering
\renewcommand{\sectionmark}[1]{\markright{#1}{}}    % Section: suppress numbering

% Header (inner, center, outer)
% \ihead{\href{http://sdis.dbca.wa.gov.au}{\textbf{Project Plan SP 2016-030}}}
%\chead{\href{http://sdis.dbca.wa.gov.au}{Science Directorate Information System}}
% \ohead{\href{https://www.dbca.wa.gov.au/science/10-biodiversity-and-conservation-science}{
% \includegraphics[height=8mm, keepaspectratio]{/usr/src/app/staticfiles/img/logo-dbca-bcs.jpg}}}

% Footer (inner, center, outer)
% \ifoot{\RaggedRight\leftmark}                       % Chapter
% \cfoot{\RaggedLeft\rightmark}                       % Section
% \ofoot[\bfseries\thepage]{\bfseries\thepage}        % Page number (also [plain])


%------------------------------------------------------------------------------%
% Fonts, encoding
%------------------------------------------------------------------------------%
%\usepackage{avant}             % Use the Avantgarde font for headings
\usepackage{txfonts}
\usepackage{mathptmx}
\usepackage{gensymb}            % provides \textdegree
\renewcommand{\familydefault}{\sfdefault} % Default to Sans Serif font
\usepackage{microtype}          % Slightly tweak font spacing for aesthetics
\usepackage[english]{babel}
\usepackage[utf8]{inputenc}  % Allow letters with accents
\usepackage[utf8]{luainputenc}  % Allow letters with accents
\usepackage[T1]{fontenc}        % Use 8-bit encoding that has 256 glyphs
\usepackage{textcomp}
\usepackage[explicit]{titlesec}           % Customise of titles
%\DeclareUnicodeCharacter{0080}{\textregistered}
\DeclareUnicodeCharacter{00B0}{\textdegree}

%------------------------------------------------------------------------------%
% Tables, columns, layout
%------------------------------------------------------------------------------%
\usepackage{etoolbox}
\AtBeginEnvironment{longtabu}{\footnotesize}{}{}  % Table font size
\usepackage{booktabs}           % Required for nicer horizontal rules in tables
\usepackage{multicol}           % 2 col publications
\usepackage{pdflscape}          % Landscape pages
\usepackage{pdfpages}           % Include PDFs
\usepackage{hanging}            % hanging paragraphs for publications
%\usepackage{titletoc}          % Manipulate the table of contents
\setcounter{tocdepth}{2}        % TOC list down to section
\usepackage{enumerate}          % Enumerations
\usepackage{enumitem}           % Enumerations
\usepackage{longtable}          % Multipage table
\usepackage{tabu}               %
\setlength{\tabulinesep}{1.5mm} % Consistent vertical spacing in tabu
\newcommand{\HRule}{\vspace{8mm}\noindent\rule{\linewidth}{0.1pt}}
\usepackage[export]{adjustbox}  % minipage, image frame


%------------------------------------------------------------------------------%
% Graphics, images, colours
%------------------------------------------------------------------------------%
\usepackage{graphicx} % embedded images
\usepackage{wrapfig}  % wrap figures in text
\usepackage{caption}  % allow unnumbered captions
\usepackage{eso-pic} % Required for specifying an image background in the title page
\usepackage{colortbl} % define custom named colours
\usepackage{xstring} % Conditionals
\usepackage{transparent} % Allow transparent images

\definecolor{RedFire}{RGB}{146,25,28}
% Following PICA branding guidelines
% https://dpaw.sharepoint.com/Divisions/pica/Documents/Branding%20guidelines.pdf
\definecolor{dpawblue}{RGB}{35,97,146}          % Pantone 647
\definecolor{dpaworange}{RGB}{237,139,0}        % Pantone 144
\definecolor{dpawgreen}{RGB}{116,170,80}        % Pantone 7489
\definecolor{dpawred}{RGB}{124,46,44}           % Paul's suggestion

% bootstrap colours
\definecolor{successbg}{RGB}{223,240,216}
\definecolor{errorbg}{RGB}{242,222,222}
\definecolor{warningbg}{RGB}{252,248,227}
\definecolor{infobg}{RGB}{217,237,247}
\definecolor{muted}{RGB}{153,153,153}
\definecolor{success}{RGB}{70,136,71}
\definecolor{error}{RGB}{185,74,72}
\definecolor{warning}{RGB}{192,152,83}
\definecolor{info}{RGB}{58,135,173}

% SDIS approval colours
\definecolor{required}{RGB}{192,152,83}
\definecolor{requiredbg}{RGB}{252,248,227}
\definecolor{denied}{RGB}{185,74,72}
\definecolor{deniedbg}{RGB}{242,222,222}
\definecolor{granted}{RGB}{70,136,71}
\definecolor{grantedbg}{RGB}{223,240,216}
\definecolor{notrequired}{RGB}{153,153,153}
\definecolor{notrequiredbg}{RGB}{255,255,255}

\usepackage{tikz} % Drawing
\usetikzlibrary{arrows,shapes,positioning,shadows,trees}


%------------------------------------------------------------------------------%
% Hyperlinks
%------------------------------------------------------------------------------%
\usepackage[open=true]{bookmark}
\usepackage{nameref}
\usepackage{ifxetex,ifluatex}
\ifxetex
  \usepackage[
    setpagesize=false,        % page size defined by xetex
    unicode=false,            % unicode breaks when used with xetex
    xetex]{hyperref}
\else
  \usepackage[unicode=true]{hyperref}
\fi

\hypersetup{
  backref=true,
  pagebackref=true,
  hyperindex=true,
  breaklinks=true,
  urlcolor=dpawblue,
  bookmarks=true,
  bookmarksopen=false,
  pdfauthor={Biodiversity and Conservation Science, Department of Biodiversity, Conservation and Attractions, WA},
  pdftitle=Project Plan SP 2016-030
,
  colorlinks=true,
  linkcolor=dpawblue,
  pdfborder={0 0 0}}

\urlstyle{same}                         % don't use monospace font for urlstyle


%------------------------------------------------------------------------------%
% Black magic to linebreak URLs
%------------------------------------------------------------------------------%
\usepackage{url}
\makeatletter\g@addto@macro{\UrlBreaks}{\UrlOrds}\makeatother
\Urlmuskip=0mu plus 1mu


%------------------------------------------------------------------------------%
% Fix latex errors
%------------------------------------------------------------------------------%
\providecommand{\tightlist}{\setlength{\itemsep}{0pt}\setlength{\parskip}{0pt}}

% copy-pasted HTML <span> in SDIS fields becomes \text{} in tex source
\providecommand{\text}{}


%------------------------------------------------------------------------------%
% Custom Tikz node for SDS diagram
%------------------------------------------------------------------------------%
\newcommand\mynode[6][]{
  \node[#1] (#2){
    \parbox{#3\relax}{
      \begin{center}
      \textbf{#4}\\
      #5\\
      \footnotesize{#6}
      \end{center}
    }};}


%------------------------------------------------------------------------------%
% Custom no-pagebreaks-environment
%------------------------------------------------------------------------------%
\newenvironment{absolutelynopagebreak}
  {\par\nobreak\vfil\penalty0\vfilneg\vtop\bgroup}
  {\par\xdef\tpd{\the\prevdepth}\egroup\prevdepth=\tpd}


%------------------------------------------------------------------------------%
% Remove the header from odd empty pages at the end of chapters
%------------------------------------------------------------------------------%
\makeatletter
\renewcommand{\cleardoublepage}{
\clearpage\ifodd\c@page\else
\hbox{}
\vspace*{\fill}
\thispagestyle{empty}
\newpage
\fi}


%----------------------------------------------------------------------------------------
%  Page flow control
%----------------------------------------------------------------------------------------
%\widowpenalty=10000
%\clubpenalty=10000
%\vbadness=1200
%\hbadness=11000


%----------------------------------------------------------------------------------------
%   CHAPTER HEADINGS
%----------------------------------------------------------------------------------------
\newcommand{\thechapterimage}{}
\newcommand{\chapterimage}[1]{\renewcommand{\thechapterimage}{#1}}

% Numbered chapters with mini tableofcontents
\def\thechapter{\arabic{chapter}}
\def\@makechapterhead#1{
%\thispagestyle{plain}
{\centering \normalfont\sffamily
\ifnum \c@secnumdepth >\m@ne
\if@mainmatter
\startcontents
\begin{tikzpicture}[remember picture,overlay]
\node at (current page.north west)
{\begin{tikzpicture}[remember picture,overlay]
\node[anchor=north west,inner sep=0pt] at (0,0) {
\includegraphics[width=\paperwidth,height=0.5\paperwidth]{\thechapterimage}};
%------------------------------------------------------------------------------%
% Small contents box in the chapter heading
% Mini TOC background box
%\fill[color=dpawblue!10!white,opacity=.2] (1cm,0) rectangle (
%  3.5cm, % Mini TOC box width
%  -3.5cm % Mini TOC box height
%);
% Mini TOC text content
%\node[anchor=north west] at (1.1cm,.35cm) {
%  \parbox[t][8cm][t]{6.5cm}{
%    \huge\bfseries\flushleft
%    \printcontents{l}{1}{
%    \setcounter{tocdepth}{1}                   % Mini TOC level depth
%    }
% }
%};
%------------------------------------------------------------------------------%
% Chapter heading
\draw[anchor=west] (5cm,-9cm) node [
rounded corners=20pt,
fill=dpawblue!10!white,
text opacity=1,
draw=dpawblue,
draw opacity=1,
line width=1.5pt,
fill opacity=.2,
inner sep=12pt]{
    \huge\sffamily\bfseries\textcolor{black}{
      \thechapter. #1\strut\makebox[22cm]{}
    }
};
\end{tikzpicture}};
\end{tikzpicture}}
\par\vspace*{240\p@}                            % Push text below chapter image
\fi
\fi}

%------------------------------------------------------------------------------%
% Unnumbered chapters without mini tableofcontents
%------------------------------------------------------------------------------%
\def\@makeschapterhead#1{
%\thispagestyle{plain}
{\centering \normalfont\sffamily
\ifnum \c@secnumdepth >\m@ne
\if@mainmatter
\begin{tikzpicture}[remember picture,overlay]
\node at (current page.north west)
{\begin{tikzpicture}[remember picture,overlay]
\node[anchor=north west,inner sep=0pt] at (0,0) {
  \includegraphics[width=\paperwidth,height=0.5\paperwidth]{\thechapterimage}};
% Mini TOC background box
%\fill[color=dpawblue!10!white,opacity=.2] (1cm,0) rectangle (
%  3.5cm,                                       % Mini TOC box width
%  -3.5cm                                       % Mini TOC box height
%);
% Mini TOC text content
%\node[anchor=north west] at (1.1cm,.35cm) {
%  \parbox[t][8cm][t]{6.5cm}{
%    \huge\bfseries\flushleft
%    \printcontents{l}{1}{
%    \setcounter{tocdepth}{1} % Mini TOC level depth
%    }
%}
%};
\draw[anchor=west] (5cm,-9cm) node [rounded corners=20pt,
  fill=dpawblue!10!white,fill opacity=.6,inner sep=12pt,text opacity=1,
  draw=dpawblue,draw opacity=1,line width=1.5pt]{
  \huge\sffamily\bfseries\textcolor{black}{#1\strut\makebox[22cm]{}}};
\end{tikzpicture}};
\end{tikzpicture}}
\par\vspace*{240\p@}
\fi
\fi
}
\makeatother



\usepackage[automark,headsepline,footsepline,plainfootsepline]{scrlayer-scrpage}
\automark*[section]{}
\addtokomafont{pageheadfoot}{\normalfont\footnotesize\sffamily} % Don't italicise
\renewcommand{\chaptermark}[1]{\markleft{#1}{}}     % Chapter: suppress numbering
\renewcommand{\sectionmark}[1]{\markright{#1}{}}    % Section: suppress numbering

% Header (inner, center, outer)
\ihead{\href{http://sdis.dbca.wa.gov.au/documents/projectplan/1775/}{Project Plan SP 2016-030}}
%\chead{\href{http://sdis.dbca.wa.gov.au}{Science Directorate Information System}}
\ohead{\href{https://www.dbca.wa.gov.au/science/10-biodiversity-and-conservation-science}{
\includegraphics[height=6mm, keepaspectratio]{/usr/src/app/staticfiles/img/logo-dbca-bcs.jpg}}}
% Footer (inner, center, outer)
\ifoot{\textbf{Printed}~Wed, 25 Nov 2020 14:56:22 +0800} % inner/left footer
\cfoot{}
\ofoot[\bfseries\thepage]{\bfseries\thepage}        % Page number (also [plain])


\pagestyle{scrheadings}
\setkomafont{pageheadfoot}{\normalfont}

%-----------------------------------------------------------------------------%
\begin{document}
\raggedbottom

%-----------------------------------------------------------------------------%
% Title page
\subject{Project Plan SP 2016-030
}
\title{Dirk Hartog Island National Park Ecological Restoration Project -- fauna
reconstruction
}
\subtitle{Animal Science
}
\author{}
\publishers{\small
    \subsection*{Project Core Team}
\begin{tabu} {X X}
\textbf{Supervising Scientist} & Saul Cowen
\\
\textbf{Data Custodian} & Saul Cowen
\\
\textbf{Site Custodian} & 
\\
\end{tabu}


    \subsection*{Project status as of Nov. 25, 2020, 2:56 p.m.}
\begin{tabu} {X X}
& Approved and active
\\
\end{tabu}

    
\subsection*{Document endorsements and approvals as of Nov. 25, 2020, 2:56 p.m.}
\begin{tabu} {X X}

%\rowcolor{grantedbg}
    \textbf{Project Team} & 
    \textcolor{granted}{ granted}\\

%\rowcolor{grantedbg}
    \textbf{Program Leader} & 
    \textcolor{granted}{ granted}\\

%\rowcolor{grantedbg}
    \textbf{Directorate} & 
    \textcolor{granted}{ granted}\\

%\rowcolor{grantedbg}
    \textbf{Biometrician} & 
    \textcolor{granted}{ granted}\\

%\rowcolor{not requiredbg}
    \textbf{Herbarium Curator} & 
    \textcolor{not required}{ not required}\\

%\rowcolor{grantedbg}
    \textbf{Animal Ethics Committee} & 
    \textcolor{granted}{ granted}\\

\end{tabu}



}
\uppertitleback{}
\lowertitleback{}
\date{}

%-----------------------------------------------------------------------------%
% Front matter
\frontmatter
\maketitle
%-----------------------------------------------------------------------------%
% Main matter
\mainmatter



\section*{Dirk Hartog Island National Park Ecological Restoration Project -- fauna
reconstruction
}



\subsection*{Biodiversity and Conservation Science Program}

Animal Science




\subsection*{Departmental Service}

Service 7: Research and Conservation Partnerships


\subsection*{Project Staff}
\begin{tabu} {X X X}
%\rowcolor{infobg}
\textbf{Role} & \textbf{Person} & \textbf{Time allocation (FTE)}\\

Research Scientist & Saul Cowen & 1.0\\

Technical Officer & Colleen Sims & 1.0\\

Supervising Scientist & Lesley Gibson & 0.2\\

Research Scientist & Kym Ottewell & 0.5\\

Technical Officer & Sean Garretson & 1.0\\

Technical Officer & Kelly Rayner & 1.0\\

Technical Officer & John Angus & 0.5\\

Research Scientist & Allan Burbidge & None\\

\end{tabu}




\subsection*{Related Science Projects}

SP 2014-003 Cat eradication on Dirk Hartog Island (D Algar \emph{et al})


\subsection*{Proposed period of the project}
July 1, 2016 -- June 30, 2030



\section*{Relevance and Outcomes}


\subsection*{Background}

The Australian vertebrate fauna, particularly mammals, has undergone a
significant decline since European settlement. Over the last 200 years,
29 mammal species have become extinct, and another 89 taxa are
threatened with extinction (Woinarski \emph{et al}. 2014). Many of these
declines and extinctions have occurred in the semi-arid and arid areas
of Australia (Burbidge and McKenzie 1989, Burbidge \emph{et al}. 2008).
Offshore islands have provided refuge for several mammal species that
otherwise have declined or become extinct on the Australian mainland
(Abbott and Burbidge 1995).

Dirk Hartog Island (DHI) is WA's largest island (58,640 ha) and lies
within the Shark Bay World Heritage Area. In 1616 the Dutch navigator
Dirk Hartog landed at the northen end of DHI and became the first
European to land on the west coast of Australia. From 1860 - 2009 it was
a pastoral lease supporting up to 20,000 sheep (\emph{Ovis aries}) at a
time. Goats (\emph{Capra hircus}) were introduced to the island in the
early 1900s following the construction and operation of a lighthouse at
Cape Inscription, and they became feral soon after. Cats (\emph{Felis
catus}) are believed to have been introduced to DHI several times over
the last 150 years and are now feral (Koch \emph{et al.} 2015). House
mice (\emph{Mus} \emph{domesticus}) have also become feral on the
island, and horses (\emph{Equus caballus}) and camels (\emph{Camelus
dromedarius}) were present on DHI as part of the pastoral operations
(Burbidge and George 1978).

Thirteen species of native terrestrial mammal are known to have occurred
on DHI (Baynes 1990, McKenzie \emph{et al}. 2000), however all but
three, smaller species have become extinct over the last few hundred
years. This project proposes to reconstruct the island's fauna
assemblage by reintroducing 10 species of native mammal and one species
of bird, and introducing two other threatened mammal species for
conservation reasons, over a 12 year period. Eight~of the species that
are proposed for translocation to DHI are listed as threatened under the
WA \emph{Wildlife Conservation Act} \emph{1950} and Commonwealth's
\emph{Environment Protection and Biodiversity Conservation Act}
\emph{1999}, and another two species are Conservation Dependent. Many of
these species are now restricted to a few offshore islands, or fenced
conservation enclosures.

The value of DHI to mammal conservation has been recognised for several
decades. In 1975 the Conservation Through Reserves Committee (CTRC 1975)
recommended that DHI be included in the conservation estate, but this
was not achieved until 2009 when most of DHI became a National Park. In
1995, the pastoral lessee developed an environmental management plan
that proposed the eradication of sheep, feral goats and feral cats, and
the reintroduction of the native mammals that once occurred on DHI
(Saunders 1995). Prior to this, in 1974-78 an attempt was made to
establish a population of banded hare-wallabies (\emph{Lagostrophus
fasciatus}) on DHI (Prince 1979), but this failed due to drought,
overgrazing by sheep and goats and an inability to effectively control
feral cats (Short \emph{et al.} 1992). In 2003, the Environmental
Protection Authority (EPA) supported the use of Net Conservation
Benefits (NCB) funds derived from the Gorgon Gas Development on Barrow
Island to be used for the restoration of DHI and reconstruction of its
mammal fauna. DHI is also within the Shark Bay World Heritage Property.
One of the four criteria for which the Shark Bay area was listed as a
World Heritage site was that the area supported important and
significant natural habitats where threatened species of animals of
outstanding universal value still survive. Reconstructing DHI's mammal
fauna will further enhance the values of the World Heritage Property in
respect of threatened fauna conservation.

NCB funding has allowed realisation of the vision of restoring DHI to a
similar ecological condition to that which existed when Dirk Hartog
landed on the island in October 1616. A proposal for the DHI National
Park Ecological Restoration Project (DHINPERP) was submitted by the then
Department of Environment and Conservation (now Parks and Wildlife) to
the NCB Board in 2011 and proposed a two stage process for achieving
this (DEC 2011). Stage One (2011-2018) has focussed on the eradication
of sheep, feral goats and feral cats, the management of weeds, and
implementing a biosecurity program for the island. This was to be
followed by Stage Two (2018-2030) that would restore the native mammal
species to the island. The DHINPERP was funded by the NCB in 2012 and
work on Stage One commenced immediately. With the successful
implementation of Stage One, planning is now underway for Stage Two to
commence. This project will contribute significantly to the long-term
conservation of several threatened species. In addition, by returning
the mammal species that were once known on DHI, the ecosystem services
provided by their digging, burrowing, grazing and browsing activities
will assist in re-establishing the ecological processes to support plant
communities. It will be the largest ecological restoration project
undertaken in Australia, and possibly the world.

Stage One of the DHINPERP has progressed well and since 2010 over 7,000
sheep and feral goats have been removed from the island. As at June
2017, two radio-collared `Judas' goats remained and these will be
removed by November 2017. The island was split into two management areas
by a cat-proof fence to facilitate effective cat eradication. Following
baiting and trapping programs since 2014, good progress has been made
with removing feral cats and eradication will be confirmed by September
2018. No black rats (\emph{Rattus rattus}) have been detected, either on
DHI, or in the broader Shark Bay area (Palmer 2017). Weed management is
ongoing. Vegetation assessment by Landsat imagery has shown that there
has been a 35\% increase in vegetation cover, predominantly in the south
part of DHI, since sheep and goat removal began (van Dongen and Huntley
2016).

In anticipation of Stage Two of the DHINPERP proceeding, a strategic
framework to guide the fauna reconstruction and conservation program on
DHI 2017 - 2030 was prepared (Morris \emph{et al.} 2017). This Project
Plan provides details about planning and implementing the translocation
program, the research required to support the translocation program, and
the research that could be undertaken once fauna populations have been
established on DHI.

It is proposed to undertake a trial release of 10-12 banded
hare-wallabies and 10-12 rufous hare-wallabies (\emph{Lagorchestes
hirsutus}) in August / September 2017 to trial collection and transport
techniques, and develop adequate monitoring protocols for use on DHI.
Providing this trial is successful, 40-50 of each species will be
translocated in September/October 2018 and 2019. Providing this
translocation is successful, the translocation program for the other 10
mammals species and one bird species, as outlined in the Milestones and
Tasks below, will be implemented over a 12 year period.

~

~




\subsection*{Aims}

The aim of Stage Two of the DHINPERP is to re-establish up to 10
terrestrial native mammal species and one bird species on Dirk Hartog
Island and establish up to two native mammal species that may have
previously occurred there, along with healthy vegetation and ecosystem
processes to sustain the islands biodiversity (Parks and Wildlife 2017).

Specifically this project aims to:

a) Identify the most suitable source populations to act as founders for
new populations on DHI, using the criteria set out in the Strategic
Framework (Morris \emph{et al.} 2017).

b) Establish new populations of 12 mammal species and one birds species
on DHI, using the species selection criteria set out in the Strategic
Framework (Morris \emph{et al.} 2017).

c) Confirm that the translocations are successful and that all new
populations on DHI are healthy and self-sustaining, using criteria set
out in the Strategic Framework (Morris \emph{et al.} 2017) and approved
Translocation Proposals.

d) Promote scientific research associated with the translocations,
monitoring and establishment of fauna, and publish scientific findings.




\subsection*{Expected outcome}

This project will achieve one of the key outcomes identified in the
Wildlife strategic priorities in the Parks and Wildlife Strategic
Directions document 2014-2017 - "With external partnerships, restore the
original suite of native fauna to Dirk Hartog Island National Park".

Specifically, if successful, this project will significantly increase
distributions and populations sizes of eight species of threatened
mammal (rufous hare-wallaby, \emph{Lagorchestes hirsutus;} banded
hare-wallaby, \emph{Lagostrophus fasciatus}; woylie, \emph{Bettongia
penicillata}; dibbler, \emph{Parantechinus apicalis;} chuditch,
\emph{Dasyurus geoffroii}; western barred bandicoot, \emph{Perameles
bougainville}; heath mouse, \emph{Pseudomys shortridgei}; and Shark Bay
mouse, \emph{Pseudomys fieldi}), two Conservation Dependent species
(boodie, \emph{Bettongia lesueur}; and greater stick-nest rat,
\emph{Leporillus conditor}), and a Priority species (mulgara,
\emph{Dasycercus} \emph{blythi}). This will potentially lead to an
improvement in conservation status for all of these species. The only
island population of the desert mouse (\emph{Pseudomys desertor}) will
be established, and the return of the only bird species known to have
become extinct on DHI, the western grasswren (\emph{Amytornis textilis
textilis}) will also be accomplished.

The re-establishment of medium-sized native mammals will also allow many
of the ecological services that these diggers and burrowers provided,
to~be returned to the DHI ecosystem. In addition, opportunities for
research at all stages of the translocation program will be provided
resulting in a better understanding of fauna translocation processes and
the values of fauna on ecosystem management.~




\subsection*{Knowledge transfer}

This will be the largest island fauna restoration project in Australia
and possibly the world, and there will be global interest in the
outcomes and the techniques used. Knowledge and technology transfer to
other organisations contemplating fauna translocations to islands will
be through reports, publication of peer-reviewed manuscripts and
presentations at conferences. Communication of results to the Shark Bay
community will be undertaken via the DHINPERP Community Engagement
Strategy. Media statements will be issued via PICA to highlight
significant events in the projects implementation.




\subsection*{Tasks and Milestones}

\textbf{2017 - Trial translocation of banded hare-wallabies and rufous
hare-wallabies:}

1. February-August : planning for fauna translocations (prepare SPP,
undertake genetic audit, select source populations, prepare
translocation proposals, AEC approvals, staff employment).

2. June-July: undertake site selection for release of hare-wallabies and
dibblers.

3. August: monitor source populations on Bernier and Dorre Islands,

4. October: monitor rufous hare-wallabies on Trimouille Island and Shark
Bay mouse on North West Island (Montebellos), collect further tissue
samples for DNA analysis.

5. September-December: undertake trial translocations of rufous and
banded hare-wallabies to DHI, monitor outcomes.

6. October-November: capture dibblers for captive breeding colony at
Perth Zoo. Undertake monitoring of DHI small vertebrates in conjunction
with Global Gypsies.

\textbf{2018 - Translocations of banded and rufous hare-wallabies, and
dibbler:}

1. January-October: dibbler breeding program at Perth Zoo for release on
DHI in October 2018.

2. February: prepare a publications schedule.

3. February-June: monitor hare-wallabies on DHI, obtain ecological and
biological data.

4. March: report on 2017 monitoring source populations and trends,
confirm sources of hare-wallabies for Sept/Oct translocations, prepare
translocation proposal for dibblers, obtain AEC approvals.

5. June: report on 2017 trial translocations and monitoring of
hare-wallabies and small vertebrates.

6. August-October: monitor source populations on Bernier and Dorre
Islands, Salutation Island, and North West and Trimouille Islands.

7. September-December: undertake translocations of rufous and banded
hare-wallabies (larger numbers), and dibblers (in early Oct using 2018
progeny), monitor outcomes.

8. October-November: capture dibblers for captive breeding colony at
Perth Zoo (for June 2019 restocking), undertake monitoring of DHI small
vertebrates.

\textbf{2019 - Restocking of banded and rufous hare-wallabies and
dibblers, reintroductions of boodies and western barred bandicoots:~}

1. January-March: capture dibbler founders for restocking release on DHI
in April (females with young).

2. February-June: monitor hare-wallabies and dibblers on DHI, obtain
ecological and biological data.

3. March: report on 2018 monitoring source populations and trends,
confirm sources of~boodies and western barred bandicoots for Sept/Oct
translocations, prepare translocation proposals for boodies and western
barred bandicoots, obtain AEC approvals.

4. April: undertake~restocking of hare-wallabies, monitor outcomes.

5. June:~ undertake dibbler restocking with females~and pouch young
(early June), report on 2018 translocations and monitoring of
hare-wallabies and dibblers, and small vertebrates.

6. August-October: monitor source populations on Bernier and Dorre
Islands, Salutation Island, and North West and Trimouille Islands.

7. September-December: undertake translocations of boodies and western
barred bandicoots, monitor outcomes.

8. October-November: undertake monitoring of DHI small vertebrates.

\textbf{2020 - Restocking of boodies and western barred bandicoots,
reintroductions of Shark Bay mice and stick-nest rats:}

1. February-June: monitor hare-wallabies, dibblers, boodies and
western-barred bandicoots on DHI, obtain ecological and biological data.

2. March: report on 2019 monitoring source populations and trends,
confirm sources of Shark Bay mice and stick-nest rats for Sept/Oct
translocations, prepare translocation proposals for Shark Bay mice and
stick-nest rats, obtain AEC approvals.

3. April: undertake restocking of~boodies and western barred bandicoots,
monitor outcomes.

4. June: report on 2019 translocations and monitoring of hare-wallabies,
dibblers, boodies and western barred bandicoots, and small vertebrates.

5. August-October: monitor source populations on Bernier and Dorre
Islands, Salutation Island, and North West and Trimouille Islands, Shark
Bay mainland (grasswren) and south-west locations (woylie).

6. September-December: undertake translocations of Shark Bay mice and
stick-nest rats, monitor outcomes.

7. October-November: undertake monitoring of DHI small vertebrates.

\textbf{2021 - Restocking of Shark Bay mice and stick-nest rats,
reintroduce western grasswren and woylie:}

1. February-June: monitor hare-wallabies, dibblers, boodies,
western-barred bandicoots, Shark Bay mice and stick-nest rats on DHI,
obtain ecological and biological data.

2. March: report on 2020 monitoring source populations and trends,
confirm sources of western grasswren and woylie for Sept/Oct
translocations, prepare translocation proposals for western grasswren
and woylie, obtain AEC approvals.

3. April: undertake restocking of~Shark Bay mice and stick-nest rats,
monitor outcomes.

4. May: survey for heath mice at Lake Magenta, south coast, Victoria;
and Shark Bay area for western grasswren.

5. June: report on 2020 translocations and monitoring of translocated
fauna, and small vertebrates.

6. July: capture heath mice and commence captive breeding program at
Perth Zoo.

7. August-October: monitor source populations on Bernier and Dorre
Islands, Salutation Island, and North West and Trimouille Islands, Shark
Bay mainland (grasswren), south-west locations (woylie), and Matuwa
(desert mouse).

8. September-December: undertake translocations of western grasswren and
woylie, monitor outcomes.

9. October-November: undertake monitoring of DHI small vertebrates.

\textbf{2022 - Restocking of western grasswren and woylie, reintroduce
heath mouse and desert mouse:}

1. February-June: monitor hare-wallabies, dibblers, boodies,
western-barred bandicoots, Shark Bay mice and stick-nest rats, grasswren
and woylie on DHI, obtain ecological and biological data.

2. March: report on 2021 monitoring source populations and trends,
confirm sources of heath mouse and desert mouse for Sept/Oct
translocations, prepare translocation proposals for heath mouse and
desert mouse, obtain AEC approvals.

3. April: undertake restocking of~western grasswren and woylie, monitor
outcomes.

4. May: survey for desert mouse and mulgara at Matuwa.

5. June: report on 2021 translocations and monitoring of translocated
fauna, and small vertebrates.

6. July: assess potential founder stock of heath mice available from
Perth Zoo, provide additional founders if necessary for restocking in
2023.

7. August-October: monitor source populations on Bernier and Dorre
Islands, Salutation Island, and North West and Trimouille Islands, Shark
Bay mainland (grasswren), south-west locations (woylie), and Matuwa
(desert mouse).

8. September-December: undertake translocations of heath mouse (captive
stock) and desert mouse, monitor outcomes.

9. October-November: undertake monitoring of DHI small vertebrates.

\textbf{2023 - Restocking of heath mouse and desert mouse, reintroduce
mulgara:}

1. February-June: monitor hare-wallabies, dibblers, boodies,
western-barred bandicoots, Shark Bay mice and stick-nest rats,
grasswren, woylie, heath mouse, desert mouse on DHI, obtain ecological
and biological data.

2. March: report on 2022 monitoring source populations and trends,
confirm sources of mulgara for Sept/Oct translocations, prepare
translocation proposal for mulgara, obtain AEC approval.

3. April: undertake restocking of~heath mouse (captive stock) and desert
mouse, monitor outcomes.

4. May: survey for mulgara at Matuwa.

5. June: report on 2022 translocations and monitoring of translocated
fauna, and small vertebrates.

6. August-October: monitor source populations on Bernier and Dorre
Islands, Salutation Island, and North West and Trimouille Islands, Shark
Bay mainland (grasswren), south-west locations (woylie), and Matuwa
(desert mouse).

7. September-December: undertake translocations of heath mouse (captive
stock) and desert mouse, monitor outcomes.

8. October-November: undertake monitoring of DHI small vertebrates.

\textbf{2024 -~Restocking of mulgara, reintroduce chuditch:}

1. February-June: monitor hare-wallabies, dibblers, boodies,
western-barred bandicoots, Shark Bay mice and stick-nest rats,
grasswren, woylie, heath mouse, desert mouse and mulgara on DHI, obtain
ecological and biological data.

2. March: report on 2023 monitoring source populations and trends,
confirm source of chuditch for Sept/Oct translocation, prepare
translocation proposal for chuditch, obtain AEC approval.

3. April: undertake restocking of mulgara, monitor outcomes.

4. June: report on 2023 translocations and monitoring of translocated
fauna, and small vertebrates.

5. August-October: monitor source populations on Bernier and Dorre
Islands, Salutation Island, and North West and Trimouille Islands, Shark
Bay mainland (grasswren), south-west locations (woylie, chuditch), and
Matuwa (desert mouse, mulgara).

6. September-December: undertake translocations of chuditch, monitor
outcomes.

7. October-November: undertake monitoring of DHI small vertebrates.

\textbf{2025 - Restocking chuditch, monitoring:}

1. February-June: monitor hare-wallabies, dibblers, boodies,
western-barred bandicoots, Shark Bay mice and stick-nest rats,
grasswren, woylie, heath mouse, desert mouse, mulgara and chuditch on
DHI, obtain ecological and biological data.

2. March: report on 2024 monitoring source populations and trends.

3. April: undertake restocking of chuditch, monitor outcomes.

4. June: report on 2024 translocations and monitoring of translocated
fauna, and small vertebrates.

5. August-October: monitor source populations on Bernier and Dorre
Islands, Salutation Island, and North West and Trimouille Islands, Shark
Bay mainland (grasswren), south-west locations (woylie, chuditch), and
Matuwa (desert mouse, mulgara).

6. September-December: monitor translocated fauna on DHI.

7. October-November: undertake monitoring of DHI small vertebrates.

\textbf{2026 Monitoring, analyse data, publish results:}

1. February-June: monitor hare-wallabies, dibblers, boodies,
western-barred bandicoots, Shark Bay mice and stick-nest rats,
grasswren, woylie, heath mouse, desert mouse, mulgara and chuditch on
DHI, obtain ecological and biological data.

2. March: report on 2025 monitoring source populations and trends.

3. June: report on 2025 translocations and monitoring of translocated
fauna, and small vertebrates.

5. August-October: monitor source populations on Bernier and Dorre
Islands, Salutation Island, and North West and Trimouille Islands, Shark
Bay mainland (grasswren), south-west locations (woylie, chuditch), and
Matuwa (desert mouse, mulgara).

6. September-December: monitor translocated fauna on DHI.

7. October-November: undertake monitoring of DHI small vertebrates.

\textbf{2027 - June 2030 Monitoring, analyse data, publish results:}

1. January-December: analyse data, publish results

2. February-June: monitor translocated fauna on DHI, obtain ecological
and biological data.

3. March: report on monitoring source populations and trends.

4. June: report on monitoring of translocated fauna, and small
vertebrates.

5. August-October: monitor source populations.

6. September-December: monitor translocated fauna on DHI.

7. October-November: undertake monitoring of DHI small vertebrates.




\subsection*{References}

Abbott I, Burbidge AA (1995). The occurrence of mammal species on the
islands of Australia: a summary of existing knowledge.
\emph{CALMScience} \textbf{1,} 259-324.

Asher J, Morris K (2014) Dirk Hartog Island Biosecurity Implementation
Plan: a shared responsibility. Department of Parks and Wildlife, Perth,
WA.

Austin JJ, Joseph L, Pedler LP, Black AB (2013) Uncovering cryptic
evolutionary diversity in extant and extinct populations of the southern
Australian arid zone Western and Thick-billed Grasswrens (Passeriformes:
Maluridae: \emph{Amytornis}). \emph{Conservation Genetics} \textbf{14},
1173-1184.

Baynes A (1990) The mammals of Shark Bay, Western Australia. Pp 313-325
In Research in Shark Bay: Report of the France-Australe Bicentenary
Expedition Committee (ed. PF Berry, SD Bradshaw and BR Wilson), Western
Australian Museum, Perth, WA.

Beard J (1976) The vegetation of the Shark Bay and Edel area, Western
Australia. Map and explanatory memoir, 1:250,000 series. Vegmap
Publications, Perth, WA.

Black A (2011) Western Australia, home of the Grass-wren
(\emph{Amytornis textilis}). \emph{Amytornis} \textbf{3}, 1-12.

Burbidge AA, George AS (1978) The flora and fauna of Dirk Hartog Island,
Western Australia. \emph{Journal of the Royal Society of Western
Australia} \textbf{60,} 71-90.

Burbidge AA, McKenzie NL (1989). Patterns in the modern decline of
Western Australia's vertebrate fauna: causes and conservation
implications. \emph{Biological Conservation} \textbf{50,} 143-198.

Burbidge AA, McKenzie NL, Brennan KEC, Woinarski JCZ, Dickman CR, Baynes
A, Gordon G, Menkhorst PW, Robinson AC (2008). Conservation status and
biogeography of Australia's terrestrial mammals. \emph{Australian
Journal of Zoology} \textbf{56,} 411-422.

Cale B (2003) Thick-billed Grasswren (western subspecies \emph{Amytornis
textilis textilis}) Interim Recovery Plan 2003-2008 No.146, Department
of Conservation and Land Management, Perth, WA.

Chapman TF, Sims C, Thomas ND, Reinhold L (2015) Assessment of mammal
populations on Bernier and Dorre Islands: 2006-2013. Department of Parks
and Wildlife, Perth, WA.

ChevronTexaco Australia (2003) Environmental, Social and Economic Review
of the Gorgon Gas Development on Barrow Island. Gorgon Australian Gas,
Perth, WA.

Christensen J (2008) Shark Bay 1616-1991: The Spread of Science and the
Emergence of Ecology in a World Heritage Area. Doctor of Philosophy
thesis, School of Humanities, The University of Western Australia,
Crawley.

CTRC (1975) Conservation Reserves for Western Australia as recommended
by the Environmental Protection Authority, Perth, WA.

Dampier W (1981) A Voyage to New Holland: The English Voyage of
Discovery to the South Seas in 1699. Alan Sutton, London.

DEC (2008a) Nature Conservation Service Midwest Region Plan 2009-2014.
Department of Environment and Conservation, Perth, WA.

DEC (2008b) Shark Bay World Heritage Property Strategic Plan 2008-2020.
Department of Environment and Conservation (WA), and Department of the
Environment, Water, Heritage and the Arts (Canberra).

DEC (2011) Dirk Hartog Island National Park Ecological Restoration
Project Summary. Department of Environment and Conservation, Perth, WA.

DEC (2012) Shark Bay Terrestrial Reserves and Proposed Reserve
Additions. Management Plan No. 75. Department of Environment and
Conservation, Perth, WA.

DEC (2013) Standard Operating Procedure: Mist net trapping for birds.
SOP No 9.10. Department of Environment and Conservation, Perth.

Dunlop J, Morris K (2012) Chuditch (\emph{Dasyurus geoffroii}) National
Recovery Plan. Western Australian Wildlife Management Program No. 54,
Department of Environment and Conservation, Perth, WA.

Friend JA (2003) Dibbler (\emph{Parantechinus apicalis}) Recovery Plan.
Western Australian Wildlife Management Program No. 38, Department of
Conservation and Land Management, Perth, WA.

Friend JA, Thomas ND (1990) The water-rat \emph{Hydromys chrysogaster}
(Muridae) on Dorre Island, Western Australia. \emph{West Australian
Naturalist} \textbf{18}, 92-93.

IUCN (2013) International Union for the Conservation of Nature -
Guidelines for Reintroductions and Other Conservation Translocations.
Version 1.0. Gland, Switzerland: IUCN Species Survival Commission, viiii
+ 57pp.

Johnstone RE, Burbidge AH, Stone P (2000) Birds of the southern
Carnarvon Basin, Western Australia: distribution, status and historical
changes. \emph{Records of the Western Australian Museum}
\textbf{Supplement No. 61,} 371-448.

King PP (1827) Narrative of a Survey of the Intertropical and Western
Coasts of Australia, performed between the Years 1818 and 1822. John
Murray, London.

Koch K, Algar D, Searle JB, Pfenninger M, Schwenk K (2015) A voyage to
Terra Australis: human mediated dispersal of cats. \emph{BMC
Evolutionary Biology} \textbf{15}, 262-271.

Lagdon R, Moro D (2013) The Gorgon gas development and its environmental
commitments. \emph{Records of the Western Australian Museum}
\textbf{Supplement 83}, 9-11.

McCarthy MA, Armstrong DP, MC Runge (2012) Adaptive management of
reintroduction. In Reintroduction Biology: Integrating Science and
Management. Eds JG Ewen, DP Armstrong, KA Parker, PJ Seddon pp 256-287.
Blackwell Publishing, West Sussex, UK.

McKenzie NL, Hall N, Muir WP (2000) Non-volant mammals of the southern
Carnarvon Basin, Western Australia. \emph{Records of the Western
Australian Museum} \textbf{Supplement No. 61,} 479-510.

Morris K, Speldewinde P, Orell P (2000) Djoongari (Shark Bay Mouse),
\emph{Pseudomys fieldi} Recovery Plan. Western Australian Wildlife
Management Program No.17, Department of Conservation and Land
Management, Perth, WA.

Pacioni C, Wayne AF, Spencer PBS (2010) Effects of habitat fragmentation
on population structure and long-distance gene flow in an endangered
marsupial: the woylie. \emph{Journal of} \emph{Zoology}, doi: 10.1111/j.
1469-7998.2010.00750x.

Palmer R (2017) Dirk Hartog Island invasive rodent survey no. 3; with
notes on the detection of black rats in the Shark Bay World Heritage
Area near Carnarvon. Department of Parks and Wildlife, Perth. March
2017.

Parks and Wildlife (2014) Science and Conservation Division, Strategic
Plan 2014-2017. Department of Parks and Wildlife, Perth, WA.

Parks and Wildlife (2015) Conserving habitats, species and ecological
communities Service Strategic Priorities 2015-2017. Department of Parks
and Wildlife, Perth, WA.

Parks and Wildlife (2016a) Framework for fauna conservation. Department
of Parks and Wildlife, Perth, WA.

Parks and Wildlife (2016b) Conserving Western Australia's Islands --
Visitor Guide. Department of Parks and Wildlife, Perth.

Parks and Wildlife (2017) Project Plan - Dirk Hartog Island National
Park Ecological Restoration Project Stage 2 - Return to 1616.~

Prince RIT (1979) Banded hare-wallaby reintroduction program - Dirk
Hartog Island. \emph{SWANS} \textbf{9}, 28.

Richards JD (2012a).~ Western barred bandicoot, burrowing bettong and
banded hare-wallaby recovery plan. Department of Environment and
Conservation (Perth) and Department of Sustainability, Environment,
Water, Populations and Communities (Canberra).

Richards JD (2012b).~ Rufous hare-wallaby recovery plan. Department of
Environment and Conservation (Perth) and Department of Sustainability,
Environment, Water, Populations and Communities (Canberra).

Saunders K (1995) Dirk Hartog Island Strategic Environmental Management
Plan 1995-2005. Enviroplan, Perth, WA.

Short J, Bradshaw SD, Giles J, Prince RIT, Wilson GR (1992)
Reintroduction of macropods (Marsupialia: Macropodoidea) in Australia -
a review. \emph{Biological Conservation} \textbf{62}, 189-204.

Van Dongen R, Huntley B (2016) Vegetation cover change on Dirk Hartog
Island: 2008-2016. Poster, Department of Parks and Wildlife, Perth, WA.

Van Dyck S, Strahan R (2008) The Mammals of Australia
(3\textsuperscript{rd} Ed). Reed New Holland, Sydney.

Yeatman GJ,~ Groom CJ (2012) National Recovery Plan for the woylie
\emph{Bettongia penicillata}. Wildlife Management Program No.51,
Department of Environment and Conservation, Perth, WA.

Walters CJ, Holling CS (1990) Large-scale management experiments and
learning by doing. \emph{Ecology} \textbf{71}, 2060-2068.

Woinarski JCZ, Burbidge AA, Harrison PL (2014) The action plan for
Australian mammals 2012. CSIRO Publishing, Collingwood, Victoria.



\section*{Study design}


\subsection*{Methodology}

\textbf{1. Site Description}

DHI is WA's largest island with an area of 58,640 ha (Abbott and
Burbidge 1995). It lies at approximately 25°50'S 113°0.5'E (centroid) at
the western edge of the Shark Bay World Heritage Area. The island is
approximately 79km long and has a maximum width of 11km with its long
axis in a south-east to north-west direction.

DHI is classified as a 'Coastal Dune' geomorphic district (Payne
\emph{et al}. 1987) and consists of coastal dunes and undulating plains
of shallow calcareous sand over limestone or calcrete. Five land systems
occur on the island, three of which (Coast, Edel and Inscription
collectively form 99\% of the island; Payne \emph{et al}. 1987) and are
described below:

Coast: occurs along the entire western side of the island and consists
of large long-walled parabolic dunes and narrow swales, unstable
blow-out areas and bare mobile dunes, minor limestone hills and rises
and steep sea cliffs (41.9\%);

Edel: occurs in eastern and south-eastern parts of the island and
consists of undulating sandy plains with minor low dunes, limestone
rises and saline flats (32.5\%);

Inscription: is found in the north-east and central-east of the island.
It consists of gently undulating sandy plains over limestone (24.3\%);

The two remaining land systems are Birrida (0.7\%) and Littoral (0.6\%).

Vegetation on the island is generally sparse, low and open and comprises
spinifex (\emph{Triodia}) hummock grassland with an overstorey of
\emph{Acacia coriacea}, \emph{Pittosporum phylliraeoides} over
\emph{Acacia ligulata}, \emph{Diplolaena dampieri}, \emph{Exocarpus
sparteus} shrubs over \emph{Triodia} sp., \emph{Acanthocarpus preissii}
and \emph{Atriplex bunburyana} hummock grasses, chenopods or shrubs
(Beard 1976). Adjacent to the exposed western coastline is a mixed open
chenopod shrubland of \emph{Atriplex} sp., \emph{Olearia axillaris} and
\emph{Frankenia} sp. and slightly inland in more protected sites,
\emph{Triodia plurinervata}, \emph{Triodia} sp., \emph{Melaleuca
huegelii}, \emph{Thryptomene baeckeacea} and \emph{Atriplex} sp.. There
are patches of bare sand and several birridas (salt pans). On the east
coast there are patches of mixed open heath of \emph{Diplolaena
dampieri}, \emph{Myoporum} sp. and \emph{Conostylis} sp. shrubs (Beard
1976). The overall height of the vegetation reduces towards the north of
the island.

The climate of the region is 'semi-desert Mediterranean' (Beard 1976;
Payne \emph{et al}. 1987). The mean annual rainfall for Denham
(recording station 006044, located 37km to the east of DHI) is 224mm
(Bureau of Meteorology 2013; long-term records 1893-2013). The wettest
month is June with an average of 55mm. February is the hottest month
with a mean daily maximum of 31.8°C while July is the coolest month with
a mean daily maximum of 21.7°C. Prevailing winds are southerly in the
morning swinging to the south-west in the afternoon with the sea breeze
(Bureau of Meteorology 2013).

Between the 1860s and 2009, DHI was managed as a pastoral lease and
grazed by sheep (\emph{Ovis aries}) and goats (\emph{Capra}
\emph{hircus}). More recently, tourism has been the main commercial
activity on the island. Cats were probably introduced by early
pastoralists and became feral during the late 19th century (Burbidge
2001). Most of the island became a National Park in November 2009, and
this provided the opportunity to reconstruct the native mammal fauna as
had been proposed earlier (Saunders 1995). DHI could potentially support
one of the most diverse mammal assemblages in Australia and contribute
significantly to the long-term conservation of several threatened
species.

\textbf{2. Translocation Proposals}

The fauna translocations proposed for DHI will comply with the relevant
Parks and Wildlife Corporate Policy and Guidelines (Parks and Wildlife
2015 a, b), and the \emph{Animal Welfare Act 2002}. ~A Translocation
Proposal (TP) will be prepared for each of the 13 species to be
translocated to DHI. These will contain background biological and
ecological information of the species being translocated, what source
populations will be used and why, and how the founders will be captured
and transported to DHI. Short, medium and long-term criteria for
translocation success or failure will be established, and monitoring
implemented to allow these to be assessed.

Founders will be released at pre-determined sites at dusk, usually
within 24 hours of being captured. Methodologies for monitoring short -
medium term survivorship (fate over time) for most mammal species will
involve the use of either VHF or GPS telemetry attachments (collars or
tail mounted transmitters). Longer term monitoring of survivorship will
use trapping, camera traps and spotlighting techniques. Monitoring of
the the translocated western grasswren will be through the use of
coloured leg bands and field observations. Animal Ethics approvals will
be obtained before translocations commence.

\textbf{3. Genetic audit}

\textbf{a)}~\textbf{Sampling and DNA extraction}

Genetic samples will be taken from animals during each phase of the
fauna reconstruction process, i.e. the initial audit of potential source
populations, founder populations released onto DHI, and new recruits as
part of the regular post-release monitoring.

Ear biopsy samples will be obtained from mammals~using a sterilised
1-2mm ear notching tool following standard operating procedures
(Department of Parks and Wildlife, 2015). Tissue samples will be stored
in 80-100\% analytical grade ethanol under cool conditions before
transport to the lab.~ Genomic DNA will be extracted from tissue samples
using a standard `salting out' protocol (Sunnucks and Hales, 1996) and
stored at -20C until use. Where biopsy samples cannot be obtained, scats
may be used as an alternative DNA source, though these can be less
reliable than tissue. Scat DNA will be extracted using the Qiagen DNA
Stool Mini-Kit following the manufacturer's protocol and stored at -20C
until use.

For birds, blood or feather samples will be obtained and DNA extracted
using the 'salting out' protocol (feathers) or Qiagen DNeasy Blood and
Tissue kit (blood).

\textbf{b) Genotyping}

Different methods will be used to obtain genetic data from sampled
animals dependent on (i) the choice of the most appropriate genetic
marker type to address the research question, and (ii) the availability
and need to integrate with existing data.

\emph{Microsatellites}: For several species, microsatellite marker data
are~available from previously published studies. In cases where it is
most efficient and informative to complement existing data to answer
specific research questions we will utilise these existing markers.
Microsatellite primers will be obtained from published studies and PCR
amplified using the Qiagen Multiplex PCR kit following manufacturer's
protocols. Microsatellite genotype profiles will be obtained through
fragment analysis at the State Agricultural Biotechnology Centre (SABC)
at Murdoch University.

\emph{RADsequencing}: New genomic techniques such as RADseq offer a
means of generating large numbers of informative markers relatively
rapidly and cheaply, increasing the power of genetic analyses. RADseq
will be used to provide genetic assessment of source populations and for
genetic monitoring post-translocation. It is anticipated that RADseq
data will be generated using a commercial service, Diversity Array
Technologies (DART).

\emph{Exon capture}: An alternative genomic approach, exon capture, may
be employed when appropriate to address taxonomic questions (e.g.
distinguishing sub-species of ~mala~\emph{Lagorchestes hirsutus}). Exon
captures and sequencing will be performed at ANU as part of the Oz
Mammal Genomics initiative.

\textbf{c)}~\textbf{Data analysis}

~\emph{Genetic assessment of source populations}

To assess the genetic `health' of source populations we will estimate
genetic diversity using several metrics, (e.g. observed and expected
heterozygosity, allelic richness), the level of observed inbreeding
(F\textsubscript{IS}) and the degree of relatedness (\emph{r}). Genetic
differentiation amongst source populations will be assessed by several
metrics including the use of clustering approaches (e.g. STRUCTURE, PCA)
and estimation of the fixation index, F\textsubscript{ST}.

Where practical we will aim to source animals from populations with high
genetic diversity, low inbreeding and to source animals from multiple
populations to maximise genetic diversity represented in the
translocated population on DHI. Mixing of divergent populations will be
assessed against the framework proposed in Frankham \emph{et al}. (2011)
to minimise risks of outbreeding depression. Population viability
analysis using VORTEX software (Lacy and Pollack, 2014) and
incorporating both genetic and demographic data will be used to inform
translocations by estimating numbers of individuals required to maximise
genetic diversity and genetic representativeness and assess mixing
ratios where multiple source populations are used.

\emph{Genetic monitoring of translocated populations}

We will assess the genetic 'health' of translocated populations on DHI
at regular intervals, with tissue collection occurring during annual
monitoring. It is anticipated that genetic analyses will be conducted at
two year intervals for species with high breeding capacities, and three
year intervals for species with a lower breeding capacity (see Table 1).
At each interval we will estimate the population genetic diversity
(observed and expected heterozygosity, allelic richness) and
inbreeding/relatedness, and assess changes in these metrics over time.
We will aim to sustain genetic diversity levels over the longer time
frame at 95\% that of the starting population; supplementations (genetic
augmentation) may be recommended in order to maintain population genetic
diversity and minimise inbreeding. Population viability models will be
updated with genetic and demographic monitoring data to inform ongoing
management activities, for example, the number and timing of
supplementations (if required).

\textbf{Table 1: Anticipated timetable for genetic analyses to monitor
the genetic health of translocated populations on Dirk Hartog Island. T
= translocation; S = supplementation (re-stocking); GA = genetic
analyses.}

\begin{longtable}[]{@{}llllllllllllllll@{}}
\toprule
\endhead
Species & Breed & 2017 & 2018 & 2019 & 2020 & 2021 & 2022 & 2023 & 2024
& 2025 & 2026 & 2027 & 2028 & 2029 & 2030\tabularnewline
BHW & slow & T & T & S & GA & ~ & ~ & GA & ~ & ~ & GA & ~ & ~ & GA &
~\tabularnewline
RHW & slow & T & T & S & GA & ~ & ~ & GA & ~ & ~ & GA & ~ & ~ & GA &
~\tabularnewline
Dibbler & fast & ~ & T & S & GA & ~ & GA & ~ & GA & ~ & GA & ~ & GA & ~
& GA\tabularnewline
Boodie & slow & ~ & ~ & T & S & GA & ~ & ~ & GA & ~ & ~ & GA & ~ & ~ &
GA\tabularnewline
WBB & fast & ~ & ~ & T & S & GA & ~ & GA & ~ & GA & ~ & GA & ~ & GA &
~\tabularnewline
SBM & fast & ~ & ~ & ~ & T & S & GA & ~ & GA & ~ & GA & ~ & GA & ~ &
GA\tabularnewline
GSNR & fast & ~ & ~ & ~ & T & S & GA & ~ & GA & ~ & GA & ~ & GA & ~ &
GA\tabularnewline
WGWren & slow & ~ & ~ & ~ & ~ & T & S & GA & ~ & ~ & GA & ~ & ~ & GA &
~\tabularnewline
Woylie & fast & ~ & ~ & ~ & ~ & T & S & GA & ~ & GA & ~ & GA & ~ & GA &
~\tabularnewline
HeathM & fast & ~ & ~ & ~ & ~ & ~ & T & S & GA & ~ & GA & ~ & GA & ~ &
GA\tabularnewline
DesertM & fast & ~ & ~ & ~ & ~ & ~ & T & S & GA & ~ & GA & ~ & GA & ~ &
GA\tabularnewline
Mulgara & fast & ~ & ~ & ~ & ~ & ~ & ~ & T & S & GA & ~ & GA & ~ & GA &
~\tabularnewline
Chuditch & slow & ~ & ~ & ~ & ~ & ~ & ~ & ~ & T & S & GA & ~ & ~ & GA &
~\tabularnewline
\bottomrule
\end{longtable}

~

\textbf{4. Monitoring source populations}

~\textbf{a) Spotlight transects for mammals}

\emph{Bernier and Dorre Islands}

Bernier and Dorre Islands were divided into blocks of two or four
transect lines, based on Short \emph{et al.} (1989) and / or placed in
proximity to safe landing points. Five blocks of four transects were
established, 500m apart (total of 20 transects) running in an east-west
direction and were marked for spotlighting on each island from 2006 to
2010.~ Transects were marked with a 1.8 m jarrah stake driven into the
ground and marked at the top with reflective tape, with the start or
finish of the transect set above high water mark or at a cliff-top.
~This was supplemented by reflective tape being placed on high
vegetation such as \emph{Acacia} spp.~ The spacing between the markers
was variable but generally a few hundred metres and where possible
placed on high points, but varied depending on terrain and vegetation.~
From 2011 to 2013 transect locations, number of transects (an additional
seven transects were added) and survey frequencies were modified to meet
Distance analysis requirements.~ The average transect length on Bernier
Island is 4.8 km and on Dorre Island 5.4 km. Their locations have been
mapped, and GPS coordinates recorded.

Between 2006 and 2013, the transects were walked by a team of two people
at no faster that 3km/hr or less.~ One team member would use a 30w
LightForce SL170 Striker spotlight to spot for animals and navigate by
spotting the reflective tape on the markers.~ The spotter needed to
maintain a line as close as possible to the transect line by navigating
between the reflective markers.~ The other team member would walk behind
the spotter and when an animal was sighted would take a GPS waypoint on
the line where the spotter is, then go to where the animal was first
sighted and either take another GPS waypoint or pace the perpendicular
distance back to the transect line.~ The distance from the transect line
would be calculated by either converting the recorders pacing distance
to meters or subtracting the Northing of the animal location from
Northing of the transect line.

From 2016 onwards, each team of two were~equipped with a Trimble Juno
J41 Series 5 hand held computer, TruPulse 360˚B laser range finder
(\href{http://www.technologyonecorp.com/}{www.TechnologyOneCorp.com})
and a Lightforce Striker 170 35w HID spotlight fitted with a Lithium
Ferrous P04 9amp hour battery
(\href{http://www.lightforce.com}{www.lightforce.com}).~ The Juno J41
Series 5 runs Trimble Terrasync
(\href{http://www.Trimble.com}{www.Trimble.com}) mapping and GIS
software and has all the required spotlighting transects for each island
installed as shapefiles, as well as any other required or relevant GIS
mapping information, i.e. landing points, island boundary shape files
etc. The range finder connects wirelessly via Bluetooth to the Juno J41
Series 5 computer.

Teams walk at no more than 3km/hour.~ The team member operating the
spotlight and rangefinder pinpoints the animal's location with the range
finder.~ Navigation and data recording are~undertaken by the other team
member using the Juno computer, and the transect line is maintained by
following the transect shapefile on the hand-held computer.~ When an
animal is sighted the data recorder records the species, number sighted
and location of the team.~ The spotter pinpoints the animal's location
using the rangefinder, the location of the animal is then automatically
transmitted to the hand held computer.

\emph{Trimouille Island}

Mala on Trimouille Island are monitored using Distance sampling as per
the 2016 methodology for hare-wallabies on Bernier and Dorre Island
above.~ Trimouille Island has 10 transects running east west, and are
500m apart, covering the full length of the island. These have been
mapped and their GPS coordinates recorded.

\textbf{b) Trapping for mammals: Cage and Elliott traps}

\emph{Bernier and Dorre Islands}

Additional monitoring on Bernier and Dorre Islands using cage and
Elliott trap grids is undertaken to target western barred bandicoots as
they are likely not adequately monitored and/or potentially
underestimated by Distance analysis alone due to their more cryptic
nature and small size.~ Opportunistically, boodies are capture during
these trapping sessions and are concurrently monitored in the process.~
Shark Bay mice are monitored on Bernier Island using Elliott traps alone
as spotlighting does not adequately census this small species.

At White Beach (Dorre Island) a 7 x 7 trap grid with traps at 40m
spacings has been established to the east of the dunes, covering a mixed
habitat of travertine, \emph{Scaevola} spp. and \emph{Triodia} sp.
vegetation.~ At each trap site, an Elliott and cage trap is set, baited
with a mixture made up of rolled oats and peanut butter.~ Sardines are
not used to reduce the likelihood of the traps being inundated by meat
ants (\emph{Iridomyrmex}~sp.) which can be prolific in some areas of the
islands and pose a risk to entrapped animals.

Two grids have been established on Bernier Island at Red Cliff Bay.~ One
7 x 7 grid (as above) has been established in suitable habitat,
predominately of \emph{Triodia} sp., \emph{Scaevola} sp. and some
fringing \emph{Spinifex} sp. just west of the dunes, approximately 500m
east of Red Cliff.~ A 7 x 3 grid to monitor Shark Bay mice has been
established in suitable habitat (predominately \emph{Spinifex} sp.) in
consolidated dunes approximately 300m west south west of Red Cliff
(Figure 4).~ Each trap site has an Elliott trap baited the standard bait
(as above).

The locations of all the grids and traps on Bernier and Dorre Islands
have been mapped and their coordinates recorded.

Standard morphometric measurements (pes length, head length, and scrotal
width), weight and reproductive status of females recorded, and tissue
samples for DNA analysis taken for each captured animal. Boodies are
western barred bandicoots are permanently marked with Allfex Passive
Integrated Implants
(\href{http://www.allflex.com.au}{www.allflex.com.au}), any captured
rodents are marked using the standard ear clipping numbering system.

\emph{North West Island}

A trapping grid has been established in \emph{Spinifex longifolius}
consolidated coastal habitat on North West Island, Montebello group, to
monitor Shark Bay mice. ~The trapping grid is same layout as on Bernier
Island so that capture rates can be compared. ~The grid location has
been mapped and GPS coordinates recorded. Morphometric measurement,
marking and tissue sampling are as above.

\textbf{c) Hand netting of mammals}

Morphometric and tissue samples are obtained from the two hare-wallaby
species by hand-netting at night.~ A team of six people are required,
made up of four runners/netters, one spotlighter and one equipment
carrier.~ Two runners walk approximately 10m abreast both side of the
spotter with the equipment carrier following up close behind.~ When an
animal is sighted the spotlighter maintains the light on it while the
runners move up slowly encircling the animal, generally the animal
remains still and is easily netted.~ Animals that do move away can be
chased down however, pursuits are restricted to no more than 100m to
reduce stress to the animals and the possibility of females ejecting
pouch young.

\textbf{d) Mist-netting birds and field observations}

It is anticipated that the founder western grasswrens will be sourced
from the nearby Shark Bay mainland, possibly Peron Peninsula. A survey
for potential source sites will be undertaken using experience
ornithologists, and two - three source sites will be selected 12 months
before grasswren translocations are planned. The grasswren populations
at these sites will then be monitored to obtain estimates of abundance.
Mist nets will be used to capture a sample of the population, and
captured birds will be weighed, sexed where possible, measured and
marked with coloured leg bands. ~Swabs and blood samples for health
screening and genetic analysis may be taken.

\textbf{e) Data Analysis}

\emph{Distance (Spotlighting)}

Transect parameters and fauna observations are~modelled using Distance
6.2 software (Thomas \emph{et al.} 2010), to account for diminishing
probability of detection with increasing distance from the transect
line. The design of the analysis was a line transect 80m wide (40m each
side of the transect), single observer and single observations of fauna
at distances perpendicular to the transect. Conventional distance
sampling was used to model the probability of detection as a function of
the distance from the transect line via size bias regression and
bootstrapped variance (Buckland \emph{et al.} 1993; Buckland \emph{et
al.} 2001).

Akaike's Information Criterion was used to select the `best' fitting
models from amongst the following: uniform cosine, uniform simple
polynomial, half-normal cosine, hazard rate cosine and hazard rate
simple polynomial (see Buckland \emph{et al.} 1993; Buckland \emph{et
al.} 2001 for more information). The best fitting models were used to
estimate population sizes for the areas surveyed and these were
extrapolated to estimate total population sizes for each Island. The
data were log transformed to ensure they met the assumptions of the
tests and \emph{t}-tests were used to compare population sizes between
islands.

\emph{Cage and Elliott Trapping}

Spatially explicit capture-recapture (SECR) analysis of the cage and
Elliott capture data is undertaken using Program Density 5.03
(\url{http://www.otago.ac.nz/density}).~ Estimates of ~population
densities of western barred bandicoots, boodies and Shark Bay mice can
also be derived using inverse prediction and/or maximum likelihood
models to fit the data.

\textbf{5. Translocation Protocols}

\emph{Animal capture}

Founder mammals from source populations will be captured using the
trapping and / or hand netting methods described above. Once captured
the mammals will be marked, weighed, measured, and placed in black cloth
bags and tagged with species and capture location. Data will be recorded
either on standardised data sheets or into a hand held computer.

Founder grasswrens will be mist netted at the source sites. The nets
will be erected at dawn, then checked and cleared before the heat of
mid-morning. Nets will be dismantled or furled when not being attended.
Captured birds will be weighed, measured, sexed where possible, fitted
with coloured leg bands, and placed in calico bags in small,
well-ventilated boxes.

\emph{Transport}

Method of transport will be dependent on species being translocated at
the time.~ Larger species will be individually bagged and placed in
airline-approved pet carriers (K9 Pet Carrier 53 x 37 x 37 cm PP20).
~Bags will be secured within the pet carriers to the corners to prevent
animals rolling on each other during transit.~ Due to their relative
sizes, two hare- wallabies will be placed in each pet carrier, boodies
and bandicoots can be transported in groups of up to four.~ Smaller
species (smaller than western barred bandicoots) will be transported
within Elliott traps placed within pet carriers.~ Depending on sea
conditions, animals may need to be taken off the islands in water tight
containers but should only be secured in these for the duration from
transferring them from shore and to the charter vessel. Founder
mammals~will be transported to DHI either onboard a sea-going charter
vessel, or an aircraft (fixed-wing or helicopter), or a combination of
both.

Founder birds will be transported to release site(s) on DHI by either
air-conditioned vehicles, or aircraft, or a combination of both. Birds
will be released in the mid afternoon, within 6-8 hours of capture.

\emph{Stress and sedation: Rufous hare-wallaby / mala}

Rufous hare-wallabies/mala are prone to capture myopathy, causing death,
and there is anecdotal information than the Shark Bay sub-species may be
more prone to capture induced stress than mala found on Trimouille
Island (C. Simms, pers comm.).~

It is recommended to inject rufous-hare-wallabies / mala (particularly
Shark bay RHW) with Vitamin E to reduce the likelihood of capture
myopathy. ~If it is used, Vitamin E injections should be given on
initial capture to prevent possible capture myopathy from subsequent
stress events such as transport. SelVit E (Ilium Selvite E, Troy,
Australia) has been used in the past, administered intramuscular (IM) at
a rate of 0.2ml/kg.~ However, this product is no longer sold in
Australia.~ An alternative injectable prophylactic Vit E will need to be
used.

To reduce the impact of additional stressors such as transport it may be
necessary to sedate the animals. Veterinarian advice recommends the use
of neuroleptic agents to induce a calm and tranquil state, and reduce
stress. The following procedure is recommended to sedate the animals for
transport:

\begin{itemize}
\tightlist
\item
  Initial use (as soon as possible after capture) of Diazepam at a lower
  dose rate (1mg/kg IM). Diazepam (Troy, Australia) will only provide
  sedation for up to a few hours (at most) and suggest combining it with
  a short-acting neuroleptic agent, Azaperone.~
\item
  An hour (approx.) after the Diazepam has been administered --
  administer the neuroleptic Azaperone (Stresnil®, Ausrichter,
  Australia) at a rate of 2 mg/kg IM which has a sedation duration of
  3-8 hours.
\end{itemize}

Neuroleptic agents are less effective if the animal is already stressed
when injected and less effective in creating a tranquil state.
Therefore, it is important to give the Diazepam first (which is not
affected in the same manner by the animal's state).

A decision on whether to use Vitamin e to reduce capture myopathy and /
or neuroleptic agents to reduce stress will be made after further
discussions with veterinarians and others who have translocated rufous
hare-wallabies / male previously.

\textbf{6. Release site selection}

The release sites for the species to be translocated to DHI will be
selected based on knowledge of their habitat and dietary preferences
recorded in the published and unpublished literature. Release sites will
also be selected using physical features (e.g. cat barrier fence, large,
unvegetated sand dunes) to restrict wide-ranging movements in the early
stages of translocations. Recorded interactions between translocated
species (e.g. boodies and woylies) will also be taken into account when
release sites are selected. The details of release sites and the reasons
they were selected will be provided in the Translocation Proposals being
prepared for each species.

The adequate availability and abundance of suitable invertebrate prey
items is important for the success of dibbler translocations. A survey
of prey items on DHI will be undertaken prior to dibbler releases in
2018 and 2019. This will involve the systematic sampling of litter~for
invertebrates \textgreater{} 3mm during periods of moist soil conditions
(so the invertebrates are active). Litter samples will be collected,
labelled~and stored in plastic bags and the invertebrate fauna extracted
using heat lamps. The location and vegetation association of each
collection site will be recorded.~Survey sites will be in the vicinity
of the vehicle~tracks on DHI, which offer transects in~north-south and
east-west directions.

\textbf{7. Determining the success of translocations on DHI}

The success of translocations will be assessed against~short, medium and
long term criteria which will be detailed in the Translocation Proposals
for each species. These criteria will be based on survivorship,
reproduction and recruitment, and occupancy and distribution. Monitoring
using several techniques will be implemented to provide the information
required to determine success / failure.

\textbf{a) Radio Tracking}

Short term monitoring of survivorship and habitat use will be undertaken
using radio-telemetry. Before release on DHI, a proportion of founder
mammals (up to 50\%) will be fitted with either VHF mortality sensing
radio-collars, or GPS radio-collars. As part of the pilot translocation
of banded and rufous hare-wallabies in 2017, all of the founder
hare-wallabies will be fitted with either a VHF Ultimate mortality
sensing radio collars (Sirtrack, New Zealand) to provide information on
survivorship, movements and refuge selection; or a FLR V GPS collar
(Telemetry Solution, USA) to obtain finer scale movement information.~
All collars will have a degradable elastic insert to ensure collars
breakaway over time.

Novel techniques using unmanned aeronautical vehicles (UAVs), or drones,
as aerial platforms for radiotelemetry are currently being developed and
there will be opportunities to use the DHI fauna translocations to
develop these in collaboration with third parties.

Released mammals will be monitored intensively for the first 10-12~weeks
post-release, then less frequently, depending on the outcomes of the
intensive monitoring. It is anticipated that radio-collars will
breakaway, or be removed after trapping the animals by six months
post-release. GPS locations can be down loaded remotely from the FLR V
collars as required.

\textbf{b) Camera Traps}

Once founder populations have established, medium to longer-term
monitoring of relative abundance and habitat use will be achieved using
camera traps, either in linear transects, or in grids.~

\textbf{c) Trapping}

Longer term monitoring of population trends, demograhics and
reproductive biology will be undertaken using trapping grids and
transects.~

\textbf{8. Data analysis}

Morphometric measurements, age, sex and bodyweight data taken for all
animals~captured and translocated will be stored on an MS Access
database. A body condition index will be derived for each mammal
individual using a relationship between body weight and a measure of
size such as pes or head length. Capture data from monitoring sites will
be analysed to detect population trends using Program MARK and used to
assist population viability analyses using VORTEX software. Population
estimates at potential source sites will be undertaken using Program
DISTANCE where spotlighting transects are used and SECR where trapping
is used. Estimates of population size and condition of translocated
populations will be compared annually and reported.

Survivorship of translocated fauna will be determined using the
Kaplan-Meier estimate based on knowledge of numbers of founders released
alive and the number who subsequently die.

~\textbf{9. Other research}

This project offers many research opportunities to improve conservation
management of threatened species, and for assessing the impact of the
removal of pest animal species, and the impact of re-establishing a
suite of native fauna on a large island in a semi-arid environment. A
framework to guide this research has been prepared (Morris 2017) and
this identifies three broad research themes to focus research programs.
These research themes are a) better planning, implementing and
monitoring fauna translocations, b) understanding the role and impacts
of translocated fauna, and c) understanding the ecology and biology of
threatened species.~

Currently two research projects are being planned.~

\textbf{a) Impact of pest species and translocated fauna on DHI
ecosystems}

\emph{Monitoring house mice and small vertebrates}

The house mouse and small vertebrate fauna of DHI were monitored from
2008-2013 (prior to eradication of sheep, goats and cats, and
translocations of native fauna) by Shark Bay District staff. Eight
monitoring sites were established encompassing most of the major
vegetation types on the island. Each site consisted of two paired
parallel lines of pit traps (each line consisted of 3 x 20 L bucket
alternated with 3 x 150mm diam PVC tube, 60 cm deep) approximately 30 m
apart. In addition two parallel lines of six medium Elliott traps were
set approximately 10 m from the pit lines. The beginning and end of each
pit line was marked with a dropper post, and the first pit trap in each
line marked on a GPS. Monitoring was undertaken annually (usually in
October/November) with traps open for 4-5 nights for each trapping
session. Animals captured~were identified, weighed and measured, and
released after being temporarily marked (e.g. with a permanent marking
pen). Species lists and relative abundances have been prepared for each
trapping site, and this equates to baseline data (pre-removal of pest
species and before arrival of translocated species). These pit lines
also offer the opportunity to sample and document the larger
invertebrate fauna of DHI, which has not previously been undertaken.

These monitoring sites will be surveyed again in October / November 2017
and each year for at least the next seven years. Additional monitoring
sites may be added in November 2017 to cover some vegetation types that
are not adequately sampled at present (e.g. \emph{Triodia} dominated
sites, coastal sites). Animals captured will be processed as before and
data recorded on datasheets. Animals that can not be confidently
identified at site will be taken back to camp for identification using
field keys. Some tissue (scale, tail tip etc) may be taken for DNA
analysis. All animals will be released alive near the point of capture
unless a new species for DHI is captured, in which case it will be
either euthanased and preserved, or transported to Perth and provided to
the WA Museum.

\textbf{b) Better planning, implementing and monitoring fauna
translocations}

~\emph{Prey abundance and release site selection for dibblers on Dirk
Hartog Island National Park}

The Endangered dibbler (\emph{Parantechinus} apicalis) is one of the 10
species of mammal proposed for reintroduction to Dirk Hartog Island
(DHI), in the Shark Bay World Heritage Area, as part of the DHI
Ecological Restoration Project. If successful the establishment of a
dibbler population on this 60 000 ha island would substantially improve
its overall abundance and distribution and potentially lead to an
improvement in its conservation status. One of the keys to the
successful re-establishment of dibblers on DHI is their release in
suitable habitat with adequate leaf litter and invertebrate food source.
Assessments of these factors are recommended prior to any dibbler
translocation.

Reintroduction of dibblers to DHI is planned for both October 2018 (sub
adults) and May 2019 (females with pouch young). Prior to this DHI needs
to be surveyed for a) leaf litter and ground debris, and b) medium-large
invertebrates (\textgreater{} 3mm length) that would be suitable as a
dibbler food source. The sampling survey would be based around the
existing north -- south track which runs for the length of the island
(ca 80 km), some east -- west tracks that cross the island (up to 15
km), and some other limited access tracks. Survey methodology would
follow previous dibbler food availability assessments. It would include
mapping areas of litter and rating these as good / abundant litter,
moderate / adequate litter, and poor / little or no litter. Invertebrate
sampling at a selection of good, adequate and poor sites would then be
undertaken. There would be the opportunity to determine any trends in
litter and prey abundance on DHI, and for comparison of results with
other assessments (e.g. Jurien Bay islands and mainland sites). Parks
and Wildlife researchers would use the results of this work to identify
suitable dibbler release sites.

\textbf{References}

Abbott I, Burbidge AA (1995). The occurrence of mammal species on the
islands of Australia: a summary of existing knowledge.
\emph{CALMScience} \textbf{1,} 259-324.

Beard J (1976) The vegetation of the Shark Bay and Edel area, Western
Australia. Map and explanatory memoir, 1:250,000 series. Vegmap
Publications, Perth, WA.

Buckland, S. T., Anderson, D. R., Burnham, K. P. and Laake, J. L. (1993)
Distance Sampling: Estimating Abundance of Biological Populations.
(Chapman and Hall: London.)

Buckland, S. T., Anderson, D. R., Burnham, K. P., Laake, J. L.,
Borchers, D. L. and Thomas, L. (2001) Introduction to Distance Sampling.
(Oxford University Press: London.)

Burbidge A (2001) Our largest island. \emph{Landscope} 17, 16-22.

Department of Parks and Wildlife (2015) Standard Operating Procedure
(SOP) No. 8.4: Tissue sample collection and storage for mammals. Parks
and Wildlife, Perth, Western Australia.

Frankham R, Ballou JD, Eldridge MDB, Lacy RC, Ralls K, Dudash MR,
Fenster CB (2011) Predicting the probability of outbreeding depression.
\emph{Conservation Biology}~\textbf{25}, 465--475.

Lacy RC, Pollack JP (2014) Vortex: A stochastic simulation of the
extinction process. Version 10. Chicago Zoological Society, Brookfield,
Illinois.

Parks and Wildlife (2015a) Conserving threatened species and ecological
communities. Corporate Policy Statement 35, Department of Parks and
Wildlife, Perth.

Parks and Wildlife (2015b) Recovery of threatened species through
translocations and captive breeding or propagation. Corporate Guideline
No. 36. Department of Parks and Wildlife, Perth.

Payne AL, Curry PJ, Spencer GF (1987) An inventory and condition survey
of rangelands in the Carnarvon Basin, Western Australia. Department of
Agriculture Western Australia Technical Bulletin No. 73.

Saunders K (1995) Dirk Hartog Island Strategic Environmental Management
Plan 1995-2005. Enviroplan, Perth, WA.

Short, J., Turner, B. and Majors, C. (1989) The distribution, relative
abundance, and habitat preferences of rare macropods and bandicoots on
Barrow, Boodie, Bernier and Dorre Islands.~ Final report for the 2 year
consultancy program entitled ``Feasibility of Reintroducing the
burrowing Bettong \emph{B. lesueur} to Mainland Western Australia Phase
1". National Kangaroo Monitoring Unit, Australian National Parks and
Wildlife Service and CSIRO, Division of Wildlife and Ecology, Perth.

Sunnucks P, Hales DF (1996) Numerous transposed sequences of
mitochondrial cytochrome oxidase I-II in aphids of the genus
\emph{Sitobion} (Hemiptera: Aphididae). \emph{Molecular Biology and
Evolution}~\textbf{13}, 510--524.

Thomas, L., Buckland, S. T., Rexstad, E. A., Laake, J. L., Strindberg,
S., Hedley, S. L., Bishop, J. R. B., Marques, T. A. and Burnham, K. P.
(2010) Distance software: design and analysis of distance sampling
surveys for estimating population size. \emph{Journal of Applied
Ecology} \textbf{47}: 5-14.
\url{http://dx.doi.org/10.1111/j.1365-2664.2009.01737.x}.

~

~

~

~




\subsection*{Biometrician's Endorsement}

granted



\section*{Data management}


\subsection*{No. specimens}

None of the species targeted for translocation to DHI will be
intentionally taken as specimens, unless they die unexpectedly. This may
occur through the capture and transport process, the fitting of
telemetry devices, or soon after release. Any unexpected deaths will be
accurately recorded and reported to the Animal Ethics Committee.

Tissues taken for genetic analyses will be registered and stored in
ethanol at the WA Conservation Science Centre.

Any individuals~captured during the small vertebrate monitoring program,
and thought to be~new species for DHI will be taken as specimens for the
WA Museum. These may be euthanased on site, or transported to the WA
Museum alive.




\subsection*{Herbarium Curator's Endorsement}

not required




\subsection*{Animal Ethics Committee's Endorsement}

granted




\subsection*{Data management}

Field data will be collected either on a pre-formatted PDA, or
datasheets. It will be uploaded / input into either an MS Excel
spreadsheet, or MS Access database. Data will be backed up on the
Woodvale server. Translocation details will be added to the corporate
translocation database, and trapping data incorporated into FaunaFile.

There will be a coordinated approach to link animal capture / sample
collection data to genetic data and demographic data in one database
location. A unique ID (potentially the PIT number) will be allocated to
each specimen and this will be carried through on all data forms.




\section*{Budget}

\section*{Consolidated Funds }



\begin{longtabu} to \linewidth { |  X | X | X | X | }
\hline
\rowcolor{infobg}
Source & Year 1 & Year 2 & Year 3\\
\hline
\endhead



FTE Scientist & 0.50 & 0.50 & 0.50\\



FTE Technical &  &  & \\



Equipment &  &  & \\



Vehicle &  &  & \\



Travel &  &  & \\



Other &  &  & \\



Total & 100,000 & 100,000 & 100,000\\


\hline
\end{longtabu}



\section*{External Funds }



\begin{longtabu} to \linewidth { |  X | X | X | X | }
\hline
\rowcolor{infobg}
Source & Year 1 & Year 2 & Year 3\\
\hline
\endhead



Salaries, Wages, Overtime & 325,000 & 561,900 & 578,800\\



Overheads & 14,800 & 19,100 & 19,700\\



Equipment & 27,000 & 25,000 & 27,000\\



Vehicle & 30,000 & 33,000 & 35,000\\



Travel & 30,600 & 36,500 & 37,300\\



Other & 319,100 & 400,800 & 306,900\\



Total & 746,500 & 950,200 & 1,004,700\\


\hline
\end{longtabu}





%-----------------------------------------------------------------------------%
% Back matter
%\backmatter
\end{document}
%-----------------------------------------------------------------------------%

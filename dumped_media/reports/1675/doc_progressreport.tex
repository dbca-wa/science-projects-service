
\documentclass[version=last,
    paper=a4, % paper size
    10pt, % default font size
    usenames,
    dvipsnames,
    oneside, % ONLINE
    headings=openany, % open chapters on odd and even pages
    %toc=chapterentrywithdots, % Table of Contents style
    %BCOR=7mm, % PRINT Binding Correction
    %DIV=13, % typearea 161.54 mm x 228.46 mm, top margin 22.85 mm, inner margin 16.15 mm
    %DIV=14, % 165.00 233.36 21.21 15.00
    DIV=15 % 168.00 237.60 19.80 14.00
]{scrbook}
\usepackage{typearea}
\usepackage[automark,headsepline,footsepline]{scrlayer-scrpage} % Headers and footers

%%
%% Fonts, encoding, spacing, indentation
%%
\usepackage{txfonts}
\renewcommand{\familydefault}{\sfdefault} % Default to Sans Serif font
\usepackage[english]{babel}
\usepackage[T1]{fontenc}
\usepackage[utf8]{inputenc}

% Paragraph spacing
%\usepackage{parskip}    % Paragraph spacing
%\setlength{\parindent}{0em} % Don't indent paragraphs - ONLINE
%\setlength{\parskip}{1.3 ex plus 0.5ex minus 0.3ex} % 1-1.8 ex vertical space between paragraphs - ONLINE

% Spacing of headings
%\RedeclareSectionCommand[afterskip=3pt]{section} % less space after section
%\RedeclareSectionCommand[beforeskip=0cm]{subsection} % less space between HRule and project name
%\RedeclareSectionCommand[afterskip=0.1\baselineskip]{subsubsection} % less space after progressreport subheadings

% Table font size
\usepackage{etoolbox}
\AtBeginEnvironment{longtabu}{\footnotesize}{}{}

%%
%% Tables, columns, layout
%%
\usepackage{multicol}   % 2 col publications
\usepackage{pdflscape}  % Landscape pages
\usepackage{pdfpages}   % Include PDFs
\usepackage{hanging}    % hanging paragraphs for publications
%\usepackage{titletoc}   % Required for manipulating the table of contents
\setcounter{tocdepth}{2} % TOC list down to section
\usepackage{enumerate}  % Enumerations
\usepackage{enumitem}   % Enumerations
\usepackage{longtable}  % Multipage table
\usepackage{tabu}       %
\setlength{\tabulinesep}{1.5mm} % Consistent vertical spacing in tabu

%%
%% Graphics, images, colours
%%
\usepackage{graphicx} % embedded images
\usepackage{eso-pic} %
\usepackage{colortbl} % define custom named colours
\definecolor{RedFire}{RGB}{146,25,28}
\definecolor{ParksWildlife}{RGB}{0,85,144}
\definecolor{successbg}{RGB}{223,240,216}
\definecolor{errorbg}{RGB}{242,222,222}
\definecolor{warningbg}{RGB}{252,248,227}
\definecolor{infobg}{RGB}{217,237,247}
\definecolor{muted}{RGB}{153,153,153}
\definecolor{success}{RGB}{70,136,71}
\definecolor{error}{RGB}{185,74,72}
\definecolor{warning}{RGB}{192,152,83}
\definecolor{info}{RGB}{58,135,173}

\definecolor{required}{RGB}{192,152,83}
\definecolor{requiredbg}{RGB}{252,248,227}
\definecolor{denied}{RGB}{185,74,72}
\definecolor{deniedbg}{RGB}{242,222,222}
\definecolor{granted}{RGB}{70,136,71}
\definecolor{grantedbg}{RGB}{223,240,216}
\definecolor{not reqiured}{RGB}{153,153,153}
\definecolor{not requiredbg}{RGB}{255,255,255}

\usepackage{tikz} % Drawing
\usetikzlibrary{arrows,shapes,positioning,shadows,trees}

%%
%% Links, URLs
%%
\usepackage[
    linktoc=all,
    %colorlinks=false,  %PRINT
    colorlinks=true, % ONLINE
    linkcolor=RedFire, % ONLINE
    urlcolor=ParksWildlife, % ONLINE
    pdftitle=Progress Report SP 1999-010 (FY 2015-2016)
]{hyperref}

% Black magic to linebreak URLs
\usepackage{url}
\makeatletter
\g@addto@macro{\UrlBreaks}{\UrlOrds}
\makeatother

%%
%% Custom macros
%%
% Thick Horizontal rule
\newcommand{\HRule}{\vspace{8mm}\\\noindent\rule{\linewidth}{0.1pt}}

% Custom Tikz node for SDS diagram
\newcommand\mynode[6][]{
    \node[#1] (#2){
        \parbox{#3\relax}{
            \begin{center}
            \textbf{#4}\\
            #5\\
            \footnotesize{#6}
            \end{center}}};}



%-----------------------------------------------------------------------------%
% Headers and Footers
\automark{section}
\ohead{\href{http://sdis.dpaw.wa.gov.au/documents/progressreport/1675/}{Progress Report SP 1999-010
}}
\chead{\href{http://sdis.dpaw.wa.gov.au}{SDIS}} % center header ONLINE
\ihead{\href{http://sdis.dpaw.wa.gov.au}{
        \includegraphics[scale=0.4]{/mnt/projects/sdis/staticfiles/img/logo-dpaw.png}}}
\ifoot{\textbf{Printed}~Mon, 4 Jul 2016 16:31:51 +0800} % inner/left footer
\cfoot{} % center footer
\ofoot{\pagemark} % outer/right footer
\pagestyle{scrheadings}
\setkomafont{pageheadfoot}{\normalfont}

%-----------------------------------------------------------------------------%
\begin{document}
\raggedbottom

%-----------------------------------------------------------------------------%
% Title page
\subject{Progress Report SP 1999-010
}
\title{Seed biology, seedbank dynamics and collection and storage of seed of
rare and threatened Western Australian taxa
}
\subtitle{Plant Science and Herbarium
}
\author{}
\publishers{\small
    \subsection*{Project Core Team}
\begin{tabu} {X X}
\textbf{Supervising Scientist} & Anne Cochrane
\\
\textbf{Data Custodian} & Anne Cochrane
\\
\textbf{Site Custodian} & Anne Cochrane
\\
\end{tabu}


    \subsection*{Project status as of July 4, 2016, 4:31 p.m.}
\begin{tabu} {X X}
& Approved and active
\\
\end{tabu}

    
\subsection*{Document endorsements and approvals as of July 4, 2016, 4:31 p.m.}
\begin{tabu} {X X}

%\rowcolor{grantedbg}
    \textbf{Project Team} & 
    \textcolor{granted}{ granted}\\

%\rowcolor{grantedbg}
    \textbf{Program Leader} & 
    \textcolor{granted}{ granted}\\

%\rowcolor{grantedbg}
    \textbf{Directorate} & 
    \textcolor{granted}{ granted}\\

\end{tabu}



}
\uppertitleback{}
\lowertitleback{}
\date{}

%-----------------------------------------------------------------------------%
% Front matter
\frontmatter
\maketitle
%-----------------------------------------------------------------------------%
% Main matter
\mainmatter

\section*{Seed biology, seedbank dynamics and collection and storage of seed of
rare and threatened Western Australian taxa
}

A Cochrane, A Crawford, A Monaghan, D Coates


\section*{Context}
Seed conservation is a specific and targeted action to conserve
biodiversity and entails the banking of genetic material in the form of
seed. Seed banking provides an important opportunity for assessing and
utilising genetic material for \emph{in situ} recovery actions, and for
seed research. Understanding the seed biology and ecology of plant
species is important for the conservation and management of
conservation-significant Western Australian taxa and for developing and
implementing recovery plans for rare and threatened flora.



\section*{Aims}
\begin{itemize}
\itemsep1pt\parskip0pt\parsep0pt
\item
  Provide a cost effective and efficient interim solution to the loss of
  plant genetic diversity by collecting and storing seed of rare and
  threatened Western Australian plant species, and thereby provide a
  focus for flora recovery.
\item
  Increase knowledge of seed biology, ecology and longevity.
\item
  Incorporate all information into a corporate database (WASEED) and
  provide relevant information on seed availability, seed biology,
  storage requirements and viability of seed of rare and threatened taxa
  to assist the development of management prescriptions and preparation
  of interim recovery plans and translocation plans.
\end{itemize}



\section*{Progress}
\begin{itemize}
\itemsep1pt\parskip0pt\parsep0pt
\item
  A total of 202 seed collections representing 107 species were
  accessioned into the Threatened Flora Seed Centre in the past year, of
  which 46 collections (29 species) were Critically Endangered.
\item
  283 accessions representing 216 Declared Rare Flora (DRF) collections
  (80 species), 31 Priority species and 36 common species were banked.
\item
  Duplicates of 40 collections were sent to Millennium Seed Bank Kew as
  a risk management strategy.
\item
  353 accessions from the \emph{Banksia} Woodland Restoration Project in
  the~Swan Region were stored, with 171 of stored collections being
  removed for use in direct seeding.
\item
  66 germination tests conducted.
\item
  Provision of seedlings of six Critically Endangered species for
  translocation.
\item
  The seed bank now contains collections of 330 Threatened flora, 704
  Priority flora and 810 key restoration species.
\end{itemize}



\section*{Management implications}
\begin{itemize}
\itemsep1pt\parskip0pt\parsep0pt
\item
  Seed conservation supports the survival of species in the wild by
  providing the genetic material for reintroduction; seed is provided
  for translocations of rare species and for restoration of
  \emph{Banksia} woodland.
\item
  Provision of seed biology and ecology data increases the success of
  threatened flora recovery actions, particularly through knowledge of
  how pre-treatments may~stimulate seed germination.
\end{itemize}



\section*{Future directions}
\begin{itemize}
\itemsep1pt\parskip0pt\parsep0pt
\item
  Ongoing collection of seed of threatened species for long-term
  conservation and use in re-introductions.
\item
  Ongoing processing and storage of collection backlog.
\item
  Germination testing, storage and monitoring of existing collections.
\item
  Ongoing research into seed biology and seed storage behaviour of
  threatened plant taxa.
\item
  Continued collaboration with other Australian seed banks through the
  Australian Seed Bank Partnership and finalise funding from the
  Australian Seed Bank Partnership.
\item
  Continue collaboration with the Millennium Seed Bank Project, Royal
  Botanic Gardens, UK.
\end{itemize}



%-----------------------------------------------------------------------------%
% Back matter
%\backmatter
\end{document}
%-----------------------------------------------------------------------------%


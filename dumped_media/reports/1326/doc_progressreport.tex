
\documentclass[version=last, paper=a4, DIV=18, usenames, dvipsnames]{scrartcl}
\usepackage{txfonts}
\usepackage{pdflscape}
\usepackage{pdfpages}
\usepackage[english]{babel} % English language/hyphenation
%%% Bootstrap colors
\definecolor{RedFire}{RGB}{146,25,28}
\definecolor{ParksWildlife}{RGB}{0,85,144}
\definecolor{successbg}{RGB}{223,240,216}
\definecolor{errorbg}{RGB}{242,222,222}
\definecolor{warningbg}{RGB}{252,248,227}
\definecolor{infobg}{RGB}{217,237,247}
\definecolor{muted}{RGB}{153,153,153}
\definecolor{success}{RGB}{70,136,71}
\definecolor{error}{RGB}{185,74,72}
\definecolor{warning}{RGB}{192,152,83}
\definecolor{info}{RGB}{58,135,173}

\definecolor{required}{HTML}{D9534F}
\definecolor{denied}{HTML}{D9534F}
\definecolor{granted}{HTML}{47A447}
\definecolor{not required}{RGB}{200, 200, 200}

\usepackage[colorlinks=true,pdftitle=doc\_progressreport.pdf
,linktoc=all,linkcolor=RedFire,urlcolor=ParksWildlife]{hyperref}
\usepackage{colortbl}
\usepackage{longtable}
\usepackage{tabu}
\setlength{\tabulinesep}{1.5mm}
\usepackage{enumerate}
\usepackage{enumitem}
\usepackage{fancyhdr}
\usepackage{lastpage}
\usepackage{graphicx}
\usepackage{eso-pic}
\usepackage{hyphenat}
\renewcommand{\familydefault}{\sfdefault}



\newcommand{\HRule}{\rule{\linewidth}{0.1pt}}

\newcommand{\placetextbox}[3]{% \placetextbox{<horizontal pos>}{<vertical pos>}{<stuff>}
  \setbox0=\hbox{#3}% Put <stuff> in a box
  \AddToShipoutPictureFG*{% Add <stuff> to current page foreground
    \put(\LenToUnit{#1\paperwidth},\LenToUnit{#2\paperheight}){\vtop{{\null}\makebox[0pt][c]{#3}}}%
  }%
}%




%-----------------------------------------------------------------------------%
% Headers and footers
%
\fancypagestyle{plain}{
  \fancyhf{}
  \setlength\headheight{60pt} % push page content below header
  \renewcommand{\headrulewidth}{0.1pt}
  \renewcommand{\footrulewidth}{0.1pt}
  
  
  \fancyhead[L]{ 
    \href{http://sdis.dpaw.wa.gov.au}{
    \includegraphics[scale=0.6]{/mnt/projects/sdis/staticfiles/img/logo-dpaw.png}}
  }
  \fancyhead[R]{ 
      \hfill
      \href{http://sdis.dpaw.wa.gov.au}{Science Directorate Information System} 
      \newline 
      \href{http://sdis.dpaw.wa.gov.au/documents/progressreport/1326/}{Progress Report 2013-5 (FY 2014-2015)} 
  }
  
  
  
  
  \fancyfoot[L]{ \leftmark\newline\textbf{Printed}\textit{ June 29, 2015, 9:38 a.m. }}
  \fancyfoot[R]{  \, \newline Page \thepage\ of \pageref{LastPage} }
  
  
}
\pagestyle{plain}
%
% end Headers
%-----------------------------------------------------------------------------%

\begin{document}

%-----------------------------------------------------------------------------%
% Title page
%

%
% end title page
%-----------------------------------------------------------------------------%




\section*{Context Summary}
The use of camera traps is often regarded as an effective tool for fauna
survey and monitoring with the assumption that they provide high
quality, cost effective data. However, our understanding of appropriate
methods for general survey and species detection, particularly in the
small to medium sized range of mammals, remains poorly understood.
Within Parks and Wildlife use of camera traps to date has usually been
restricted to simple species inventories or behavioural studies and
beyond this there has been little assessment of deployment methods or
appropriate analytical techniques. This has sometimes led to erroneous
conclusions being derived from captured images. Camera traps have the
potential to offer a comparatively reliable and relatively unbiased
method for monitoring medium to large native and introduced mammal
species throughout the state, including a number of significant cryptic
species that are currently not incorporated under the Western Shield
fauna monitoring program. However, research is required to validate and
test different survey designs (temporal and spatial components) and
methods of deploying camera traps, and to interpret the results in a
meaningful way. In particular, work is needed to determine how best to
use remote cameras to provide rigorous data on species detectability,
and species richness and density.



\section*{Aims Summary}
\begin{itemize}
\itemsep1pt\parskip0pt\parsep0pt
\item
  Establish suitable methodology for use of camera traps to estimate the
  presence and relative abundances of native and introduced mammals
  species in the south-west of Western Australia.
\item
  Investigate the effectiveness of baited (active) and un-baited
  (passive) cameras sets to inventory targeted species.
\item
  Investigate and assess the most appropriate methods of image analysis
  and data storage.
\end{itemize}



\section*{Progress}
\begin{itemize}
\itemsep1pt\parskip0pt\parsep0pt
\item
  Completed two camera trap trials in Dryandra Woodland and one in
  Tutanning. Results indicate that camera trapping provides consistent,
  reliable and comparable species accumulation and detection rates.
\item
  Tutanning camera trap trial confirmed that woylies are no longer at
  detectable levels in the reserve.
\item
  Further analysis of bait verses un-baited trial data indicates decline
  in detection rates at baited cameras over time.
\item
  Completed analysis of species relative abundance from cameras deployed
  during a known removal event (translocation of woylies to Perup),
  which showed the method is sensitive enough to detect changes in
  relative abundance of a species.
\item
  Quantified required effort to detect all species (mammals) within
  Dryandra, which can be extrapolated to other reserves.
\item
  Provided advice and methodologies to other sections within Parks and
  Wildlife, and other Tertiary institutions and NGO's.
\end{itemize}



\section*{Management implications}
\begin{itemize}
\itemsep1pt\parskip0pt\parsep0pt
\item
  Camera traps appear to be an effective tool in detecting a suite of
  species currently not adequately monitored by the Western Shield
  monitoring program. Their use should be considered in the Western
  Shield monitoring program, either to complement the trapping program,
  or as a separate fauna monitoring tool.
\item
  Recommend that the Reconyx HC600 series cameras are adopted as the
  minimum standard of camera to be used across the department, but
  preference should be for the PC900 camera due to its reliability and
  greater functionality, including its operation over a wider
  temperature range.
\item
  The open source Access database, Camera Base 1.6
  (\href{http://www.atrium-biodiversity.org/tools/camerabase/}{http://www.atrium-biodiversity.org/tools/camerabase/}),
  be adopted (with some minor modifications) as the standard method of
  capturing and storing camera trap data. Camera Base 1.6 to be used in
  conjunction with an image processing software such as Faststone Image
  Viewer.
\item
  A standardised mounting method that is cohesive and repeatable between
  sites should be adopted for monitoring purposes.
\item
  Camera traps consistently detect species that are not currently
  censused using most other standard detection/monitoring methods, and
  provide an effective mounting method for these species.
\end{itemize}



\section*{Future directions}
\begin{itemize}
\itemsep1pt\parskip0pt\parsep0pt
\item
  Develop a number of standard Access queries to better analyse outputs
  from Camera Base, and provide this to Regional staff using Camera
  Base.
\item
  Validate camera traps against other traditional methods of fauna
  monitoring, such as cage trapping or sand plots.
\item
  Investigate methods to use camera traps to qualitatively and
  quantitatively monitor invasive species.
\item
  Investigate the sensitivity of camera-trap data to detect changes in
  relative abundance and occupancy of targeted species over time and
  season.
\end{itemize}




\clearpage



\end{document}


\documentclass[version=last, 
    paper=a4, % paper size
    10pt, % default font size
    usenames,
    dvipsnames, 
    oneside, % ONLINE
    headings=openany, % open chapters on odd and even pages
    %toc=chapterentrywithdots, % Table of Contents style
    %BCOR=7mm, % PRINT Binding Correction
    %DIV=13, % typearea 161.54 mm x 228.46 mm, top margin 22.85 mm, inner margin 16.15 mm
    %DIV=14, % 165.00 233.36 21.21 15.00
    DIV=15 % 168.00 237.60 19.80 14.00
]{scrbook}
\usepackage{typearea}
\usepackage[automark,headsepline,footsepline]{scrlayer-scrpage} % Headers and footers

%%
%% Fonts, encoding, spacing, indentation
%%
\usepackage{txfonts}
\renewcommand{\familydefault}{\sfdefault} % Default to Sans Serif font
\usepackage[english]{babel}
\usepackage[T1]{fontenc}
\usepackage[utf8]{inputenc}

% Paragraph spacing
%\usepackage{parskip}    % Paragraph spacing
%\setlength{\parindent}{0em} % Don't indent paragraphs - ONLINE
%\setlength{\parskip}{1.3 ex plus 0.5ex minus 0.3ex} % 1-1.8 ex vertical space between paragraphs - ONLINE

% Spacing of headings
%\RedeclareSectionCommand[afterskip=3pt]{section} % less space after section
%\RedeclareSectionCommand[beforeskip=0cm]{subsection} % less space between HRule and project name
%\RedeclareSectionCommand[afterskip=0.1\baselineskip]{subsubsection} % less space after progressreport subheadings

% Table font size
\usepackage{etoolbox}
\AtBeginEnvironment{longtabu}{\footnotesize}{}{}

%%
%% Tables, columns, layout
%%
\usepackage{multicol}   % 2 col publications
\usepackage{pdflscape}  % Landscape pages
\usepackage{pdfpages}   % Include PDFs
\usepackage{hanging}    % hanging paragraphs for publications
%\usepackage{titletoc}   % Required for manipulating the table of contents
\setcounter{tocdepth}{2} % TOC list down to section
\usepackage{enumerate}  % Enumerations
\usepackage{enumitem}   % Enumerations
\usepackage{longtable}  % Multipage table
\usepackage{tabu}       % 
\setlength{\tabulinesep}{1.5mm} % Consistent vertical spacing in tabu

%%
%% Graphics, images, colours
%%
\usepackage{graphicx} % embedded images
\usepackage{eso-pic} % 
\usepackage{colortbl} % define custom named colours
\definecolor{RedFire}{RGB}{146,25,28}
\definecolor{ParksWildlife}{RGB}{0,85,144}
\definecolor{successbg}{RGB}{223,240,216}
\definecolor{errorbg}{RGB}{242,222,222}
\definecolor{warningbg}{RGB}{252,248,227}
\definecolor{infobg}{RGB}{217,237,247}
\definecolor{muted}{RGB}{153,153,153}
\definecolor{success}{RGB}{70,136,71}
\definecolor{error}{RGB}{185,74,72}
\definecolor{warning}{RGB}{192,152,83}
\definecolor{info}{RGB}{58,135,173}

\definecolor{required}{RGB}{192,152,83}
\definecolor{requiredbg}{RGB}{252,248,227}
\definecolor{denied}{RGB}{185,74,72}
\definecolor{deniedbg}{RGB}{242,222,222}
\definecolor{granted}{RGB}{70,136,71}
\definecolor{grantedbg}{RGB}{223,240,216}
\definecolor{not reqiured}{RGB}{153,153,153}
\definecolor{not requiredbg}{RGB}{255,255,255}

\usepackage{tikz} % Drawing
\usetikzlibrary{arrows,shapes,positioning,shadows,trees}

%%
%% Links, URLs
%%
\usepackage[
    linktoc=all,
    %colorlinks=false,  %PRINT
    colorlinks=true, % ONLINE
    linkcolor=RedFire, % ONLINE
    urlcolor=ParksWildlife, % ONLINE
    pdftitle=doc\_studentreport.pdf
]{hyperref}

% Black magic to linebreak URLs
\usepackage{url}
\makeatletter
\g@addto@macro{\UrlBreaks}{\UrlOrds}
\makeatother

%%
%% Custom macros
%%
% Thick Horizontal rule
\newcommand{\HRule}{\vspace{8mm}\\\noindent\rule{\linewidth}{0.1pt}}

% Custom Tikz node for SDS diagram
\newcommand\mynode[6][]{\node[#1] (#2){\parbox{#3\relax}{\begin{center}\textbf{#4}\\#5\\\footnotesize{#6}\end{center}}};}




%-----------------------------------------------------------------------------%
% Headers and Footers
\automark{section}
\ohead{\href{http://sdis.dpaw.wa.gov.au/documents/studentreport/1726/}{Progress Report STP 2013-013 (FY 2015-2016)
}}
\chead{\href{http://sdis.dpaw.wa.gov.au}{SDIS}} % center header ONLINE
\ihead{\href{http://sdis.dpaw.wa.gov.au}{
        \includegraphics[scale=0.4]{/mnt/projects/sdis/staticfiles/img/logo-dpaw.png}}}
\ifoot{\textbf{Printed}~Fri, 17 Jun 2016 15:04:59 +0800} % inner/left footer
\cfoot{} % center footer
\ofoot{\pagemark} % outer/right footer
\pagestyle{scrheadings}
\setkomafont{pageheadfoot}{\normalfont}

%-----------------------------------------------------------------------------%
\begin{document}
\raggedbottom

%-----------------------------------------------------------------------------%
% Title page
\subject{Progress Report STP 2013-013 (FY 2015-2016)
}
\title{Nutrient movement and its impact on aquatic invertebrates as a food
source of waterbirds between different wetland suites within the Lake
Warden Wetland System
}
\subtitle{Wetlands Conservation
}
\author{}
\publishers{\small
    \subsection*{Project Core Team}
\begin{tabu} {X X}
\textbf{Supervising Scientist} & Adrian Pinder
\\
\textbf{Data Custodian} & 
\\
\textbf{Site Custodian} & 
\\
\end{tabu}


    \subsection*{Project status as of June 17, 2016, 3:04 p.m.}
\begin{tabu} {X X}
& Approved and active
\\
\end{tabu}

    
\subsection*{Document endorsements and approvals as of June 17, 2016, 3:04 p.m.}
\begin{tabu} {X X}

%\rowcolor{grantedbg}
    \textbf{Project Team} & 
    \textcolor{granted}{ granted}\\

%\rowcolor{grantedbg}
    \textbf{Program Leader} & 
    \textcolor{granted}{ granted}\\

%\rowcolor{grantedbg}
    \textbf{Directorate} & 
    \textcolor{granted}{ granted}\\

\end{tabu}



}
\uppertitleback{}
\lowertitleback{}
\date{}

%-----------------------------------------------------------------------------%
% Front matter
\frontmatter
\maketitle
%-----------------------------------------------------------------------------%
% Main matter
\mainmatter

\section*{Nutrient movement and its impact on aquatic invertebrates as a food
source of waterbirds between different wetland suites within the Lake
Warden Wetland System
}

J Lizamore, A Pinder, Dr R Vogwill (The University of Western Australia)


\section*{Progress Report}
Canditure for this project has been suspended until January 2017 due to
other commitments. During 2015-16 the following tasks were completed.

\begin{itemize}
\itemsep1pt\parskip0pt\parsep0pt
\item
  Aquatic invertebrates were identified by Parks and Wildlife staff and
  analysed for the pink lakes component of the project.
\item
  Evaporative total dissolved solids analysis were completed.
\item
  A paper on management of the Lake Warden system was presented at the
  South Australia NRM Conference in Adelaide.
\item
  A poster titled `Comparing methods for analysing the microbial
  communities of Lake Hillier --~ an Australian bright-pink hypersaline
  lake'' was prepared.
\end{itemize}

Four paper are planned from this project.

\begin{itemize}
\itemsep1pt\parskip0pt\parsep0pt
\item
  A discussion of actual salt concentrations in relation to electrical
  conductivity of hyper saline water -- case study of natural hyper
  saline lakes on the South-coast of Western Australia
\item
  Managing increased salt loads in Lake Warden and its impacts on
  aquatic invertebrate assemblages as food source for waterbird
  assemblages.~
\item
  A simplified numerical model of water balance and solute load
  differences between Pink Lake and Lake Warden.~
\item
  A study of the feasibility of reinstating the surface water
  hydrological link between Pink Lake and Lake Warden as part of long
  term restoration plans.~
\end{itemize}



%-----------------------------------------------------------------------------%
% Back matter
%\backmatter
\end{document}
%-----------------------------------------------------------------------------%


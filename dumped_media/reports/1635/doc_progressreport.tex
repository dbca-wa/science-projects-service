
\documentclass[version=last,
    paper=a4, % paper size
    10pt, % default font size
    usenames,
    dvipsnames,
    oneside, % ONLINE
    headings=openany, % open chapters on odd and even pages
    %toc=chapterentrywithdots, % Table of Contents style
    %BCOR=7mm, % PRINT Binding Correction
    %DIV=13, % typearea 161.54 mm x 228.46 mm, top margin 22.85 mm, inner margin 16.15 mm
    %DIV=14, % 165.00 233.36 21.21 15.00
    DIV=15 % 168.00 237.60 19.80 14.00
]{scrbook}
\usepackage{typearea}
\usepackage[automark,headsepline,footsepline]{scrlayer-scrpage} % Headers and footers

%%
%% Fonts, encoding, spacing, indentation
%%
\usepackage{txfonts}
\renewcommand{\familydefault}{\sfdefault} % Default to Sans Serif font
\usepackage[english]{babel}
\usepackage[T1]{fontenc}
\usepackage[utf8]{inputenc}

% Paragraph spacing
%\usepackage{parskip}    % Paragraph spacing
%\setlength{\parindent}{0em} % Don't indent paragraphs - ONLINE
%\setlength{\parskip}{1.3 ex plus 0.5ex minus 0.3ex} % 1-1.8 ex vertical space between paragraphs - ONLINE

% Spacing of headings
%\RedeclareSectionCommand[afterskip=3pt]{section} % less space after section
%\RedeclareSectionCommand[beforeskip=0cm]{subsection} % less space between HRule and project name
%\RedeclareSectionCommand[afterskip=0.1\baselineskip]{subsubsection} % less space after progressreport subheadings

% Table font size
\usepackage{etoolbox}
\AtBeginEnvironment{longtabu}{\footnotesize}{}{}

%%
%% Tables, columns, layout
%%
\usepackage{multicol}   % 2 col publications
\usepackage{pdflscape}  % Landscape pages
\usepackage{pdfpages}   % Include PDFs
\usepackage{hanging}    % hanging paragraphs for publications
%\usepackage{titletoc}   % Required for manipulating the table of contents
\setcounter{tocdepth}{2} % TOC list down to section
\usepackage{enumerate}  % Enumerations
\usepackage{enumitem}   % Enumerations
\usepackage{longtable}  % Multipage table
\usepackage{tabu}       %
\setlength{\tabulinesep}{1.5mm} % Consistent vertical spacing in tabu

%%
%% Graphics, images, colours
%%
\usepackage{graphicx} % embedded images
\usepackage{eso-pic} %
\usepackage{colortbl} % define custom named colours
\definecolor{RedFire}{RGB}{146,25,28}
\definecolor{ParksWildlife}{RGB}{0,85,144}
\definecolor{successbg}{RGB}{223,240,216}
\definecolor{errorbg}{RGB}{242,222,222}
\definecolor{warningbg}{RGB}{252,248,227}
\definecolor{infobg}{RGB}{217,237,247}
\definecolor{muted}{RGB}{153,153,153}
\definecolor{success}{RGB}{70,136,71}
\definecolor{error}{RGB}{185,74,72}
\definecolor{warning}{RGB}{192,152,83}
\definecolor{info}{RGB}{58,135,173}

\definecolor{required}{RGB}{192,152,83}
\definecolor{requiredbg}{RGB}{252,248,227}
\definecolor{denied}{RGB}{185,74,72}
\definecolor{deniedbg}{RGB}{242,222,222}
\definecolor{granted}{RGB}{70,136,71}
\definecolor{grantedbg}{RGB}{223,240,216}
\definecolor{not reqiured}{RGB}{153,153,153}
\definecolor{not requiredbg}{RGB}{255,255,255}

\usepackage{tikz} % Drawing
\usetikzlibrary{arrows,shapes,positioning,shadows,trees}

%%
%% Links, URLs
%%
\usepackage[
    linktoc=all,
    %colorlinks=false,  %PRINT
    colorlinks=true, % ONLINE
    linkcolor=RedFire, % ONLINE
    urlcolor=ParksWildlife, % ONLINE
    pdftitle=Progress Report SP 2011-015 (FY 2015-2016)
]{hyperref}

% Black magic to linebreak URLs
\usepackage{url}
\makeatletter
\g@addto@macro{\UrlBreaks}{\UrlOrds}
\makeatother

%%
%% Custom macros
%%
% Thick Horizontal rule
\newcommand{\HRule}{\vspace{8mm}\\\noindent\rule{\linewidth}{0.1pt}}

% Custom Tikz node for SDS diagram
\newcommand\mynode[6][]{
    \node[#1] (#2){
        \parbox{#3\relax}{
            \begin{center}
            \textbf{#4}\\
            #5\\
            \footnotesize{#6}
            \end{center}}};}



%-----------------------------------------------------------------------------%
% Headers and Footers
\automark{section}
\ohead{\href{http://sdis.dpaw.wa.gov.au/documents/progressreport/1635/}{Progress Report SP 2011-015
}}
\chead{\href{http://sdis.dpaw.wa.gov.au}{SDIS}} % center header ONLINE
\ihead{\href{http://sdis.dpaw.wa.gov.au}{
        \includegraphics[scale=0.4]{/mnt/projects/sdis/staticfiles/img/logo-dpaw.png}}}
\ifoot{\textbf{Printed}~Mon, 4 Jul 2016 16:26:41 +0800} % inner/left footer
\cfoot{} % center footer
\ofoot{\pagemark} % outer/right footer
\pagestyle{scrheadings}
\setkomafont{pageheadfoot}{\normalfont}

%-----------------------------------------------------------------------------%
\begin{document}
\raggedbottom

%-----------------------------------------------------------------------------%
% Title page
\subject{Progress Report SP 2011-015
}
\title{Taxonomy of undescribed taxa in the Ericaceae subfamily Styphelioideae,
with an emphasis on those of conservation concern
}
\subtitle{Plant Science and Herbarium
}
\author{}
\publishers{\small
    \subsection*{Project Core Team}
\begin{tabu} {X X}
\textbf{Supervising Scientist} & Michael Hislop
\\
\textbf{Data Custodian} & Michael Hislop
\\
\textbf{Site Custodian} & Michael Hislop
\\
\end{tabu}


    \subsection*{Project status as of July 4, 2016, 4:26 p.m.}
\begin{tabu} {X X}
& Approved and active
\\
\end{tabu}

    
\subsection*{Document endorsements and approvals as of July 4, 2016, 4:26 p.m.}
\begin{tabu} {X X}

%\rowcolor{grantedbg}
    \textbf{Project Team} & 
    \textcolor{granted}{ granted}\\

%\rowcolor{grantedbg}
    \textbf{Program Leader} & 
    \textcolor{granted}{ granted}\\

%\rowcolor{grantedbg}
    \textbf{Directorate} & 
    \textcolor{granted}{ granted}\\

\end{tabu}



}
\uppertitleback{}
\lowertitleback{}
\date{}

%-----------------------------------------------------------------------------%
% Front matter
\frontmatter
\maketitle
%-----------------------------------------------------------------------------%
% Main matter
\mainmatter

\section*{Taxonomy of undescribed taxa in the Ericaceae subfamily Styphelioideae,
with an emphasis on those of conservation concern
}

M Hislop


\section*{Context}
Epacrid classification is undergoing fundamental reassessment at the
generic level as new information on relationships is revealed.
\emph{Leucopogon}, in particular, is species-rich in Western Australia
but is relatively poorly understood and includes many undescribed taxa,
including ones of conservation significance. It is also clear that the
genus cannot be maintained in its current circumscription, although
generic boundaries are still uncertain. This project will continue to
describe new taxa in \emph{Leucopogon, Melichrus} and other genera in
the subfamily Styphelioideae and, in collaboration with partners in
eastern Australia, work towards a generic reclassification of the
subfamily.



\section*{Aims}
\begin{itemize}
\itemsep1pt\parskip0pt\parsep0pt
\item
  Publish new taxa from the tribes \emph{Styphelieae} and
  \emph{Oligarrheneae}, prioritising those of high conservation
  significance.
\item
  Revise generic concepts in line with recent systematic studies.
\item
  Continue a taxonomic assessment of species boundaries across the tribe
  \emph{Styphelieae} (mainly in \emph{Leucopogon}) with a view to
  identifying previously unrecognised taxa, especially those which may
  be geographically restricted.
\end{itemize}



\section*{Progress}
~

\begin{itemize}
\itemsep1pt\parskip0pt\parsep0pt
\item
  A collaborative paper with researchers in eastern Australia and the
  USA is now published in~\emph{Australian Systematic Botany}. The paper
  lays the foundation for the recognition of a greatly
  expanded~\emph{Styphelia~}which will encompass all elements of
  \emph{Leucopogon} not belonging to~\emph{Leucopogon s. str.~}as well
  as~\emph{Astroloma, Croninia~}and~\emph{Coleanthera.~}~~
\item
  A paper describing a new and probably rare species of~\emph{Leucopogon
  s. str.}from the south-west corner of Western Australia published in
  \emph{Nuytsia}.
\item
  A paper describing~six new species of \emph{Leucopogon s. str.~}from
  the Geraldton Sandplains has been submitted to \emph{Nuytsia}.
\item
  A draft paper with descriptions of five new species of
  \emph{Styphelia~}(currently in~\emph{Leucopogon s. lat.}) from the
  Geraldton Sandplains is near completion.~
\item
  A new paper updating the taxonomy of the genus~\emph{Brachyloma~}is
  under way.
\end{itemize}



\section*{Management implications}
The epacrids, of which \emph{Leucopogon} constitutes by far the largest
genus, have a major centre of diversity in south-west Western Australia.
An authoritative source of current information is fundamental to
correctly managing the conservation taxa and the lands on which they
occur for this taxonomically difficult group that is also very
susceptible to a number of major threatening processes, including
salinity and \emph{Phytophthora}.



\section*{Future directions}
\begin{itemize}
\itemsep1pt\parskip0pt\parsep0pt
\item
  Preparation of further papers describing new taxa in \emph{Leucopogon,
  Brachyloma} and \emph{Styphelia}.
\item
  Further field studies to assist in the taxonomic resolution of
  potentially new taxa in~\emph{Leucopogon~}and~\emph{Styphelia s. lat.}
\end{itemize}



%-----------------------------------------------------------------------------%
% Back matter
%\backmatter
\end{document}
%-----------------------------------------------------------------------------%


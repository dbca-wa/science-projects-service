
\documentclass[version=last, paper=a4, DIV=18, usenames, dvipsnames]{scrartcl}
\usepackage{txfonts}
\usepackage{pdflscape}
\usepackage{pdfpages}
\usepackage[english]{babel} % English language/hyphenation
%%% Bootstrap colors
\definecolor{RedFire}{RGB}{146,25,28}
\definecolor{ParksWildlife}{RGB}{0,85,144}
\definecolor{successbg}{RGB}{223,240,216}
\definecolor{errorbg}{RGB}{242,222,222}
\definecolor{warningbg}{RGB}{252,248,227}
\definecolor{infobg}{RGB}{217,237,247}
\definecolor{muted}{RGB}{153,153,153}
\definecolor{success}{RGB}{70,136,71}
\definecolor{error}{RGB}{185,74,72}
\definecolor{warning}{RGB}{192,152,83}
\definecolor{info}{RGB}{58,135,173}

\definecolor{required}{HTML}{D9534F}
\definecolor{denied}{HTML}{D9534F}
\definecolor{granted}{HTML}{47A447}
\definecolor{not required}{RGB}{200, 200, 200}

\usepackage[colorlinks=true,pdftitle=doc\_studentreport.pdf
,linktoc=all,linkcolor=RedFire,urlcolor=ParksWildlife]{hyperref}
\usepackage{colortbl}
\usepackage{longtable}
\usepackage{tabu}
\setlength{\tabulinesep}{1.5mm}
\usepackage{enumerate}
\usepackage{enumitem}
\usepackage{fancyhdr}
\usepackage{lastpage}
\usepackage{graphicx}
\usepackage{eso-pic}
\usepackage{hyphenat}
\renewcommand{\familydefault}{\sfdefault}



\newcommand{\HRule}{\rule{\linewidth}{0.1pt}}

\newcommand{\placetextbox}[3]{% \placetextbox{<horizontal pos>}{<vertical pos>}{<stuff>}
  \setbox0=\hbox{#3}% Put <stuff> in a box
  \AddToShipoutPictureFG*{% Add <stuff> to current page foreground
    \put(\LenToUnit{#1\paperwidth},\LenToUnit{#2\paperheight}){\vtop{{\null}\makebox[0pt][c]{#3}}}%
  }%
}%




%-----------------------------------------------------------------------------%
% Headers and footers
%
\fancypagestyle{plain}{
  \fancyhf{}
  \setlength\headheight{60pt} % push page content below header
  \renewcommand{\headrulewidth}{0.1pt}
  \renewcommand{\footrulewidth}{0.1pt}
  
  
  \fancyhead[L]{ 
    \href{http://sdis.dpaw.wa.gov.au}{
    \includegraphics[scale=0.6]{/mnt/projects/sdis/staticfiles/img/logo-dpaw.png}}
  }
  \fancyhead[R]{ 
      \hfill
      \href{http://sdis.dpaw.wa.gov.au}{Science Directorate Information System} 
      \newline 
      \href{http://sdis.dpaw.wa.gov.au/documents/studentreport/1446/}{Progress Report 2013-19 (FY 2014-2015)} 
  }
  
  
  
  
  \fancyfoot[L]{ \leftmark\newline\textbf{Printed}\textit{ June 24, 2015, 3:32 p.m. }}
  \fancyfoot[R]{  \, \newline Page \thepage\ of \pageref{LastPage} }
  
  
}
\pagestyle{plain}
%
% end Headers
%-----------------------------------------------------------------------------%

\begin{document}

%-----------------------------------------------------------------------------%
% Title page
%

%
% end title page
%-----------------------------------------------------------------------------%




\section*{Progress Report}
While much is known of the impact of trypanosomes on human and livestock
health, trypanosomes in wildlife, although ubiquitous, have largely been
considered to be non-pathogenic. This project aimed to investigate the
genetic diversity and potential pathogenicity of trypanosomes naturally
infecting Western Australian marsupials with particular emphasis on
those parasites associated with the endangered woylie (\emph{Bettongia
penicillata}). 554 blood samples and 250 tissue samples collected from
50 carcasses of sick-euthanised and road-killed animals, belonging to 10
species of marsupials, were screened for the presence of trypanosomes
using a PCR of the 18S rDNA gene. PCR results revealed a rate of
infection of 67\% in blood and 60\% in tissues. Inferred phylogenetic
trees using 18S rDNA and glycosomal glyceraldehyde phosphate
dehydrogenase (gGAPDH) sequences showed the presence of three different
species of \emph{Trypanosoma}: \emph{Trypanosoma copemani},
\emph{Trypanosoma vegrandis}, and \emph{Trypanosoma} sp. H25.
\emph{Trypanosoma} infections were compared in a declining and in a
stable population of the woylie. High rates of infection with
\emph{Trypanosoma copemani} (96\%) were found in the declining
population, whereas in the stable population, \emph{Trypanosoma
vegrandis} was predominant (89\%). Mixed infections were common in
woylies from the declining but not from the stable population.
Histopathological findings associated with either mixed or single
infections involving \emph{Trypanosoma copemani} showed pathological
changes similar to those seen in \emph{Didelphis marsupialis} infected
with the pathogenic \emph{Trypanosoma cruzi} in South America:
myocarditis and tongue degeneration. \emph{T. copemani} was successfully
grown in culture and for the first time it was demonstrated that this
species has the capacity to not only colonise different tissues in the
host but also to invade cells in vitro. This study also showed that
commercial drugs and new compounds developed against the pathogenic
\emph{T. cruzi} are active in vitro against \emph{T. copemani}. These
results provide evidence for the potential role of trypanosomes in the
decline of the woylie and contribute valuable information towards
directing management decisions for endangered species where these
parasites are known to be present at high prevalence levels.

This research has been presented at international and national
conferences. One paper was published in the \emph{International Journal
for Parasitology: Parasites and Wildlife}, and three more papers are in
process of submission.




\clearpage



\end{document}

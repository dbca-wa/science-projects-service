
\documentclass[version=last, 
    paper=a4, % paper size
    10pt, % default font size
    usenames,
    dvipsnames, 
    oneside, % ONLINE
    headings=openany, % open chapters on odd and even pages
    %toc=chapterentrywithdots, % Table of Contents style
    %BCOR=7mm, % PRINT Binding Correction
    %DIV=13, % typearea 161.54 mm x 228.46 mm, top margin 22.85 mm, inner margin 16.15 mm
    %DIV=14, % 165.00 233.36 21.21 15.00
    DIV=15 % 168.00 237.60 19.80 14.00
]{scrbook}
\usepackage{typearea}
\usepackage[automark,headsepline,footsepline]{scrlayer-scrpage} % Headers and footers

%%
%% Fonts, encoding, spacing, indentation
%%
\usepackage{txfonts}
\renewcommand{\familydefault}{\sfdefault} % Default to Sans Serif font
\usepackage[english]{babel}
\usepackage[T1]{fontenc}
\usepackage[utf8]{inputenc}

% Paragraph spacing
%\usepackage{parskip}    % Paragraph spacing
%\setlength{\parindent}{0em} % Don't indent paragraphs - ONLINE
%\setlength{\parskip}{1.3 ex plus 0.5ex minus 0.3ex} % 1-1.8 ex vertical space between paragraphs - ONLINE

% Spacing of headings
%\RedeclareSectionCommand[afterskip=3pt]{section} % less space after section
%\RedeclareSectionCommand[beforeskip=0cm]{subsection} % less space between HRule and project name
%\RedeclareSectionCommand[afterskip=0.1\baselineskip]{subsubsection} % less space after progressreport subheadings

% Table font size
\usepackage{etoolbox}
\AtBeginEnvironment{longtabu}{\footnotesize}{}{}

%%
%% Tables, columns, layout
%%
\usepackage{multicol}   % 2 col publications
\usepackage{pdflscape}  % Landscape pages
\usepackage{pdfpages}   % Include PDFs
\usepackage{hanging}    % hanging paragraphs for publications
%\usepackage{titletoc}   % Required for manipulating the table of contents
\setcounter{tocdepth}{2} % TOC list down to section
\usepackage{enumerate}  % Enumerations
\usepackage{enumitem}   % Enumerations
\usepackage{longtable}  % Multipage table
\usepackage{tabu}       % 
\setlength{\tabulinesep}{1.5mm} % Consistent vertical spacing in tabu

%%
%% Graphics, images, colours
%%
\usepackage{graphicx} % embedded images
\usepackage{eso-pic} % 
\usepackage{colortbl} % define custom named colours
\definecolor{RedFire}{RGB}{146,25,28}
\definecolor{ParksWildlife}{RGB}{0,85,144}
\definecolor{successbg}{RGB}{223,240,216}
\definecolor{errorbg}{RGB}{242,222,222}
\definecolor{warningbg}{RGB}{252,248,227}
\definecolor{infobg}{RGB}{217,237,247}
\definecolor{muted}{RGB}{153,153,153}
\definecolor{success}{RGB}{70,136,71}
\definecolor{error}{RGB}{185,74,72}
\definecolor{warning}{RGB}{192,152,83}
\definecolor{info}{RGB}{58,135,173}

\definecolor{required}{RGB}{192,152,83}
\definecolor{requiredbg}{RGB}{252,248,227}
\definecolor{denied}{RGB}{185,74,72}
\definecolor{deniedbg}{RGB}{242,222,222}
\definecolor{granted}{RGB}{70,136,71}
\definecolor{grantedbg}{RGB}{223,240,216}
\definecolor{not reqiured}{RGB}{153,153,153}
\definecolor{not requiredbg}{RGB}{255,255,255}

\usepackage{tikz} % Drawing
\usetikzlibrary{arrows,shapes,positioning,shadows,trees}

%%
%% Links, URLs
%%
\usepackage[
    linktoc=all,
    %colorlinks=false,  %PRINT
    colorlinks=true, % ONLINE
    linkcolor=RedFire, % ONLINE
    urlcolor=ParksWildlife, % ONLINE
    pdftitle=doc\_studentreport.pdf
]{hyperref}

% Black magic to linebreak URLs
\usepackage{url}
\makeatletter
\g@addto@macro{\UrlBreaks}{\UrlOrds}
\makeatother

%%
%% Custom macros
%%
% Thick Horizontal rule
\newcommand{\HRule}{\vspace{8mm}\\\noindent\rule{\linewidth}{0.1pt}}

% Custom Tikz node for SDS diagram
\newcommand\mynode[6][]{\node[#1] (#2){\parbox{#3\relax}{\begin{center}\textbf{#4}\\#5\\\footnotesize{#6}\end{center}}};}




%-----------------------------------------------------------------------------%
% Headers and Footers
\automark{section}
\ohead{\href{http://sdis.dpaw.wa.gov.au/documents/studentreport/1429/}{Progress Report STP 2014-016 (FY 2014-2015)
}}
\chead{\href{http://sdis.dpaw.wa.gov.au}{SDIS}} % center header ONLINE
\ihead{\href{http://sdis.dpaw.wa.gov.au}{
        \includegraphics[scale=0.4]{/mnt/projects/sdis/staticfiles/img/logo-dpaw.png}}}
\ifoot{\textbf{Printed}~Mon, 16 May 2016 17:37:52 +0800} % inner/left footer
\cfoot{} % center footer
\ofoot{\pagemark} % outer/right footer
\pagestyle{scrheadings}
\setkomafont{pageheadfoot}{\normalfont}

%-----------------------------------------------------------------------------%
\begin{document}
\raggedbottom

%-----------------------------------------------------------------------------%
% Title page
\subject{Progress Report STP 2014-016 (FY 2014-2015)
}
\title{The ecology and interactions of dingoes and feral cats in the arid
Rangelands of Western Australia
}
\subtitle{Animal Science
}
\author{}
\publishers{\small
    \subsection*{Project Core Team}
\begin{tabu} {X X}
\textbf{Supervising Scientist} & Neil Burrows
\\
\textbf{Data Custodian} & Neil Burrows
\\
\textbf{Site Custodian} & Neil Burrows
\\
\end{tabu}


    \subsection*{Project status as of May 16, 2016, 5:37 p.m.}
\begin{tabu} {X X}
& Approved and active
\\
\end{tabu}

    
\subsection*{Document endorsements and approvals as of May 16, 2016, 5:37 p.m.}
\begin{tabu} {X X}

%\rowcolor{grantedbg}
    \textbf{Project Team} & 
    \textcolor{granted}{ granted}\\

%\rowcolor{grantedbg}
    \textbf{Program Leader} & 
    \textcolor{granted}{ granted}\\

%\rowcolor{grantedbg}
    \textbf{Directorate} & 
    \textcolor{granted}{ granted}\\

\end{tabu}



}
\uppertitleback{}
\lowertitleback{}
\date{}

%-----------------------------------------------------------------------------%
% Front matter
\frontmatter
\maketitle
%-----------------------------------------------------------------------------%
% Main matter
\mainmatter

\section*{The ecology and interactions of dingoes and feral cats in the arid
Rangelands of Western Australia
}

N Burrows, K Morris, M Wysong, Prof R Hobbs (University of Western
Australia), Dr E Ritchie (Melbourne University)


\section*{Progress Report}
Research investigating the interactions between feral cats and dingoes
at Lorna Glen began in the winter field season of 2013. During this time
we initiated a pilot camera trap study to trial different camera trap
techniques and investigate changes in predator activity following annual
Eradicat baiting. Eighty cameras were placed either alongside roads or
100m off roads and were either left unbaited or else baited using an
audio call lure. The study showed that the best method for detecting
both feral cats and dingoes was to deploy either baited or unbaited
cameras along roadsides. Cameras alongside roads that were baited showed
a slightly higher detection rate although this difference was not
significant while cameras off road showed virtually no detections
whether baited or unbaited.

Using the on-road camera data from this study we also examined activity
levels of feral cats and dingoes before, during, and after Eradicat
baiting. The results of this study showed that activity of both
predators (measured by the number of photo captures per trap night)
decreased immediately following the baiting. However, by 30 days
post-baiting, dingo activity had decreased to about 23\% of pre-bait
levels whereas cat activity increased to near pre-bait levels. By 60
days, dingo activity rebounded to about 53\% and cat activity fell to
45\% of pre-bait levels. This suggests that high levels of dingo
activity may have some role in supressing cat activity.

The current phase of research for the project seeks to investigate
fine-scale habitat use and diets of these two species to better
understand their extent of spatial and dietary overlap. At present, we
have deployed 136 camera traps across three major habitat types. This
study will run for 21 days prior to the annual Eradicat baiting and
again for 21 days starting two weeks after the baiting. An occupancy
modelling approach will be used to analyse this data and will help us
understand habitat use of dingoes and feral cats and how this habitat
use is impacted by baiting.

To complement the current camera trap study, we have also fitted 16
dingoes and 21 feral cats with high precision GPS collars. These collars
take a location fix every two or four hours and will give us detailed
information on the movements of these predators through space and time
and also help us understand how the baiting impacts their movement
ecology. Finally, by analysing scat contents of these species we can get
a good understanding of both their level of dietary competition and the
impacts that these predators have on their prey species. To date, we
have over 100 scats collected of each species and collections will
continue until the end of 2014.

Preliminary analysis of movement data indicates that dingoes and feral
cats maintain some spatial separation from each other and use different
habitats. This PhD thesis is now being written up and will be submitted
for examination in February 2016.



%-----------------------------------------------------------------------------%
% Back matter
%\backmatter
\end{document}
%-----------------------------------------------------------------------------%


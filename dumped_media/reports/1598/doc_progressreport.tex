
\documentclass[version=last,
    paper=a4, % paper size
    10pt, % default font size
    usenames,
    dvipsnames,
    oneside, % ONLINE
    headings=openany, % open chapters on odd and even pages
    %toc=chapterentrywithdots, % Table of Contents style
    %BCOR=7mm, % PRINT Binding Correction
    %DIV=13, % typearea 161.54 mm x 228.46 mm, top margin 22.85 mm, inner margin 16.15 mm
    %DIV=14, % 165.00 233.36 21.21 15.00
    DIV=15 % 168.00 237.60 19.80 14.00
]{scrbook}
\usepackage{typearea}
\usepackage[automark,headsepline,footsepline]{scrlayer-scrpage} % Headers and footers

%%
%% Fonts, encoding, spacing, indentation
%%
\usepackage{txfonts}
\renewcommand{\familydefault}{\sfdefault} % Default to Sans Serif font
\usepackage[english]{babel}
\usepackage[T1]{fontenc}
\usepackage[utf8]{inputenc}

% Paragraph spacing
%\usepackage{parskip}    % Paragraph spacing
%\setlength{\parindent}{0em} % Don't indent paragraphs - ONLINE
%\setlength{\parskip}{1.3 ex plus 0.5ex minus 0.3ex} % 1-1.8 ex vertical space between paragraphs - ONLINE

% Spacing of headings
%\RedeclareSectionCommand[afterskip=3pt]{section} % less space after section
%\RedeclareSectionCommand[beforeskip=0cm]{subsection} % less space between HRule and project name
%\RedeclareSectionCommand[afterskip=0.1\baselineskip]{subsubsection} % less space after progressreport subheadings

% Table font size
\usepackage{etoolbox}
\AtBeginEnvironment{longtabu}{\footnotesize}{}{}

%%
%% Tables, columns, layout
%%
\usepackage{multicol}   % 2 col publications
\usepackage{pdflscape}  % Landscape pages
\usepackage{pdfpages}   % Include PDFs
\usepackage{hanging}    % hanging paragraphs for publications
%\usepackage{titletoc}   % Required for manipulating the table of contents
\setcounter{tocdepth}{2} % TOC list down to section
\usepackage{enumerate}  % Enumerations
\usepackage{enumitem}   % Enumerations
\usepackage{longtable}  % Multipage table
\usepackage{tabu}       %
\setlength{\tabulinesep}{1.5mm} % Consistent vertical spacing in tabu

%%
%% Graphics, images, colours
%%
\usepackage{graphicx} % embedded images
\usepackage{eso-pic} %
\usepackage{colortbl} % define custom named colours
\definecolor{RedFire}{RGB}{146,25,28}
\definecolor{ParksWildlife}{RGB}{0,85,144}
\definecolor{successbg}{RGB}{223,240,216}
\definecolor{errorbg}{RGB}{242,222,222}
\definecolor{warningbg}{RGB}{252,248,227}
\definecolor{infobg}{RGB}{217,237,247}
\definecolor{muted}{RGB}{153,153,153}
\definecolor{success}{RGB}{70,136,71}
\definecolor{error}{RGB}{185,74,72}
\definecolor{warning}{RGB}{192,152,83}
\definecolor{info}{RGB}{58,135,173}

\definecolor{required}{RGB}{192,152,83}
\definecolor{requiredbg}{RGB}{252,248,227}
\definecolor{denied}{RGB}{185,74,72}
\definecolor{deniedbg}{RGB}{242,222,222}
\definecolor{granted}{RGB}{70,136,71}
\definecolor{grantedbg}{RGB}{223,240,216}
\definecolor{not reqiured}{RGB}{153,153,153}
\definecolor{not requiredbg}{RGB}{255,255,255}

\usepackage{tikz} % Drawing
\usetikzlibrary{arrows,shapes,positioning,shadows,trees}

%%
%% Links, URLs
%%
\usepackage[
    linktoc=all,
    %colorlinks=false,  %PRINT
    colorlinks=true, % ONLINE
    linkcolor=RedFire, % ONLINE
    urlcolor=ParksWildlife, % ONLINE
    pdftitle=Progress Report SP 2014-021 (FY 2015-2016)
]{hyperref}

% Black magic to linebreak URLs
\usepackage{url}
\makeatletter
\g@addto@macro{\UrlBreaks}{\UrlOrds}
\makeatother

%%
%% Custom macros
%%
% Thick Horizontal rule
\newcommand{\HRule}{\vspace{8mm}\\\noindent\rule{\linewidth}{0.1pt}}

% Custom Tikz node for SDS diagram
\newcommand\mynode[6][]{
    \node[#1] (#2){
        \parbox{#3\relax}{
            \begin{center}
            \textbf{#4}\\
            #5\\
            \footnotesize{#6}
            \end{center}}};}



%-----------------------------------------------------------------------------%
% Headers and Footers
\automark{section}
\ohead{\href{http://sdis.dpaw.wa.gov.au/documents/progressreport/1598/}{Progress Report SP 2014-021
}}
\chead{\href{http://sdis.dpaw.wa.gov.au}{SDIS}} % center header ONLINE
\ihead{\href{http://sdis.dpaw.wa.gov.au}{
        \includegraphics[scale=0.4]{/mnt/projects/sdis/staticfiles/img/logo-dpaw.png}}}
\ifoot{\textbf{Printed}~Mon, 11 Jul 2016 12:36:26 +0800} % inner/left footer
\cfoot{} % center footer
\ofoot{\pagemark} % outer/right footer
\pagestyle{scrheadings}
\setkomafont{pageheadfoot}{\normalfont}

%-----------------------------------------------------------------------------%
\begin{document}
\raggedbottom

%-----------------------------------------------------------------------------%
% Title page
\subject{Progress Report SP 2014-021
}
\title{Habitat use, distribution and abundance of coastal dolphin species in
the Pilbara
}
\subtitle{Marine Science
}
\author{}
\publishers{\small
    \subsection*{Project Core Team}
\begin{tabu} {X X}
\textbf{Supervising Scientist} & Holly Raudino
\\
\textbf{Data Custodian} & Holly Raudino
\\
\textbf{Site Custodian} & Holly Raudino
\\
\end{tabu}


    \subsection*{Project status as of July 11, 2016, 12:36 p.m.}
\begin{tabu} {X X}
& Update requested
\\
\end{tabu}

    
\subsection*{Document endorsements and approvals as of July 11, 2016, 12:36 p.m.}
\begin{tabu} {X X}

%\rowcolor{grantedbg}
    \textbf{Project Team} & 
    \textcolor{granted}{ granted}\\

%\rowcolor{grantedbg}
    \textbf{Program Leader} & 
    \textcolor{granted}{ granted}\\

%\rowcolor{requiredbg}
    \textbf{Directorate} & 
    \textcolor{required}{ required}\\

\end{tabu}



}
\uppertitleback{}
\lowertitleback{}
\date{}

%-----------------------------------------------------------------------------%
% Front matter
\frontmatter
\maketitle
%-----------------------------------------------------------------------------%
% Main matter
\mainmatter

\section*{Habitat use, distribution and abundance of coastal dolphin species in
the Pilbara
}

K Waples, H Raudino, R Douglas, C Severin


\section*{Context}
Australian snubfin (\emph{Orcaella} \emph{heinsohni}) and Australian
humpback (\emph{Sousa sahulensis}) dolphins inhabit Australia's
north-western coastal waters, but little is known about the population
sizes, distribution and residency patterns of these species. Current
knowledge of these dolphin species in the Pilbara is currently poor and
is limited to a dedicated study of humpback dolphins in Ningaloo Marine
Park and Exmouth Gulf (Brown\emph{, et al.} 2012) and opportunistic
surveys and anecdotal sightings throughout the region (Allen\emph{, et
al.} 2012). ~Although the presence of several coastal dolphin species is
expected in nearshore Pilbara waters (humpback, snubfin and bottlenose
dolphins), very little is currently known of their residency, degree of
use and habitat characteristics.

Human pressures on these species are increasing in the Pilbara through
activities associated with expansion of the resources sector, including
oil and gas exploration and production, coastal infrastructure
development and shipping. While this is a key factor that proponents are
required to address to secure State and Commonwealth environmental
approvals, impact assessments for these species are complicated by the
lack of best practice protocols and standards for survey design and data
collection, which limits the comparison of different studies and study
sites.~ This project will provide a better understanding of these
species and their spatial and temporal use of Pilbara coastal waters and
lead to greater certainty in assessing and managing impacts that relate
to industrial developments.~This project was designed to meet this
priority need under the Wheatstone Offset C program.~



\section*{Aims}
This research is being conducted to develop a baseline understanding of
key aspects of dolphin ecology in coastal Pilbara waters. The specific
aims are to:

\begin{itemize}
\itemsep1pt\parskip0pt\parsep0pt
\item
  ~~~~~~~~ Determine habitat use, distribution, abundance, residency,
  and movement patterns of dolphins in coastal Pilbara waters; and
\item
  ~~~~~~~~ Identify the characteristics of habitats used by coastal
  dolphins, such as water depth, benthic substrate, timing and seasonal
  variation.~
\end{itemize}



\section*{Progress}
A three year~research program has been initiated which will include
both~annual boat and aerial surveys.~~During this reporting period we
have:

\begin{itemize}
\itemsep1pt\parskip0pt\parsep0pt
\item
  Developed a survey design and protocol for vessel based surveys in the
  impact area (Onslow) as well as other suspected high use areas for
  dolphins.
\item
  Conducted two boat-based dolphin~surveys (autumn and winter) around
  Onslow.~ These data will be used to estimate the density and~abundance
  of coastal dolphins in the area~if the~encounter and re-sighting rate
  is sufficient.
\item
  Designed and conducted an aerial survey of coastal waters from Exmouth
  Gulf to Port Hedland and extending offshore to the 20 metre depth
  contour,including the Montebello Islands. Data collected from this
  survey will be used to produce an estimate of dolphin density~for
  the~surveyed area.
\item
  Prepared the annual report for Chevron on Wheatstone Offset C.
\item
  Developed a relationship with the Murujuga traditional owners to share
  information on dolphin sightings and important areas~in the Dampier
  Archipelago and develop standardised survey protocols and data
  storage.
\end{itemize}



\section*{Management implications}
\begin{itemize}
\itemsep1pt\parskip0pt\parsep0pt
\item
  This research will~provide a baseline understanding of dolphin habitat
  usage of the Pilbara region.
\item
  This knowledge will inform the assessment of environmental impacts
  relating to future coastal developments and will assist to determine
  the conservation status of coastal dolphin species in Pilbara waters.
\item
  The research will establish baseline data and long-term monitoring
  protocols for coastal dolphin species in State waters.~
\end{itemize}



\section*{Future directions}
\begin{itemize}
\itemsep1pt\parskip0pt\parsep0pt
\item
  The established vessel survey methodology and design will continue for
  a third year and additional survey sites may be added in the Dampier
  Archipelago area.
\item
  The aerial survey design will be reviewed based on recent results and
  adjusted as needed for the third aerial survey in 16/17.
\item
  Existing data will be analysed to produce abundance estimates for at
  least two dolphin species (bottlenose and humpback dolphins)~across
  the study area.
\item
  Survey data will be collated with other datasets to produce spatial
  habitat models of dolphin presence and relationships with key
  environmental factors across the Pilbara region.~
\end{itemize}



%-----------------------------------------------------------------------------%
% Back matter
%\backmatter
\end{document}
%-----------------------------------------------------------------------------%



\documentclass[version=last,
    paper=a4, % paper size
    10pt, % default font size
    usenames,
    dvipsnames,
    oneside, % ONLINE
    headings=openany, % open chapters on odd and even pages
    %toc=chapterentrywithdots, % Table of Contents style
    %BCOR=7mm, % PRINT Binding Correction
    %DIV=13, % typearea 161.54 mm x 228.46 mm, top margin 22.85 mm, inner margin 16.15 mm
    %DIV=14, % 165.00 233.36 21.21 15.00
    DIV=15 % 168.00 237.60 19.80 14.00
]{scrbook}
\usepackage{typearea}
\usepackage[automark,headsepline,footsepline]{scrlayer-scrpage} % Headers and footers

%%
%% Fonts, encoding, spacing, indentation
%%
\usepackage{txfonts}
\renewcommand{\familydefault}{\sfdefault} % Default to Sans Serif font
\usepackage[english]{babel}
\usepackage[T1]{fontenc}
\usepackage[utf8]{inputenc}

% Paragraph spacing
%\usepackage{parskip}    % Paragraph spacing
%\setlength{\parindent}{0em} % Don't indent paragraphs - ONLINE
%\setlength{\parskip}{1.3 ex plus 0.5ex minus 0.3ex} % 1-1.8 ex vertical space between paragraphs - ONLINE

% Spacing of headings
%\RedeclareSectionCommand[afterskip=3pt]{section} % less space after section
%\RedeclareSectionCommand[beforeskip=0cm]{subsection} % less space between HRule and project name
%\RedeclareSectionCommand[afterskip=0.1\baselineskip]{subsubsection} % less space after progressreport subheadings

% Table font size
\usepackage{etoolbox}
\AtBeginEnvironment{longtabu}{\footnotesize}{}{}

%%
%% Tables, columns, layout
%%
\usepackage{multicol}   % 2 col publications
\usepackage{pdflscape}  % Landscape pages
\usepackage{pdfpages}   % Include PDFs
\usepackage{hanging}    % hanging paragraphs for publications
%\usepackage{titletoc}   % Required for manipulating the table of contents
\setcounter{tocdepth}{2} % TOC list down to section
\usepackage{enumerate}  % Enumerations
\usepackage{enumitem}   % Enumerations
\usepackage{longtable}  % Multipage table
\usepackage{tabu}       %
\setlength{\tabulinesep}{1.5mm} % Consistent vertical spacing in tabu

%%
%% Graphics, images, colours
%%
\usepackage{graphicx} % embedded images
\usepackage{eso-pic} %
\usepackage{colortbl} % define custom named colours
\definecolor{RedFire}{RGB}{146,25,28}
\definecolor{ParksWildlife}{RGB}{0,85,144}
\definecolor{successbg}{RGB}{223,240,216}
\definecolor{errorbg}{RGB}{242,222,222}
\definecolor{warningbg}{RGB}{252,248,227}
\definecolor{infobg}{RGB}{217,237,247}
\definecolor{muted}{RGB}{153,153,153}
\definecolor{success}{RGB}{70,136,71}
\definecolor{error}{RGB}{185,74,72}
\definecolor{warning}{RGB}{192,152,83}
\definecolor{info}{RGB}{58,135,173}

\definecolor{required}{RGB}{192,152,83}
\definecolor{requiredbg}{RGB}{252,248,227}
\definecolor{denied}{RGB}{185,74,72}
\definecolor{deniedbg}{RGB}{242,222,222}
\definecolor{granted}{RGB}{70,136,71}
\definecolor{grantedbg}{RGB}{223,240,216}
\definecolor{not reqiured}{RGB}{153,153,153}
\definecolor{not requiredbg}{RGB}{255,255,255}

\usepackage{tikz} % Drawing
\usetikzlibrary{arrows,shapes,positioning,shadows,trees}

%%
%% Links, URLs
%%
\usepackage[
    linktoc=all,
    %colorlinks=false,  %PRINT
    colorlinks=true, % ONLINE
    linkcolor=RedFire, % ONLINE
    urlcolor=ParksWildlife, % ONLINE
    pdftitle=Progress Report SP 1998-007 (FY 2015-2016)
]{hyperref}

% Black magic to linebreak URLs
\usepackage{url}
\makeatletter
\g@addto@macro{\UrlBreaks}{\UrlOrds}
\makeatother

%%
%% Custom macros
%%
% Thick Horizontal rule
\newcommand{\HRule}{\vspace{8mm}\\\noindent\rule{\linewidth}{0.1pt}}

% Custom Tikz node for SDS diagram
\newcommand\mynode[6][]{
    \node[#1] (#2){
        \parbox{#3\relax}{
            \begin{center}
            \textbf{#4}\\
            #5\\
            \footnotesize{#6}
            \end{center}}};}



%-----------------------------------------------------------------------------%
% Headers and Footers
\automark{section}
\ohead{\href{http://sdis.dpaw.wa.gov.au/documents/progressreport/1676/}{Progress Report SP 1998-007
}}
\chead{\href{http://sdis.dpaw.wa.gov.au}{SDIS}} % center header ONLINE
\ihead{\href{http://sdis.dpaw.wa.gov.au}{
        \includegraphics[scale=0.4]{/mnt/projects/sdis/staticfiles/img/logo-dpaw.png}}}
\ifoot{\textbf{Printed}~Mon, 11 Jul 2016 12:03:30 +0800} % inner/left footer
\cfoot{} % center footer
\ofoot{\pagemark} % outer/right footer
\pagestyle{scrheadings}
\setkomafont{pageheadfoot}{\normalfont}

%-----------------------------------------------------------------------------%
\begin{document}
\raggedbottom

%-----------------------------------------------------------------------------%
% Title page
\subject{Progress Report SP 1998-007
}
\title{Genetic analysis for the development of vegetation services and
sustainable environmental management
}
\subtitle{Ecosystem Science
}
\author{}
\publishers{\small
    \subsection*{Project Core Team}
\begin{tabu} {X X}
\textbf{Supervising Scientist} & Margaret Byrne
\\
\textbf{Data Custodian} & 
\\
\textbf{Site Custodian} & 
\\
\end{tabu}


    \subsection*{Project status as of July 11, 2016, 12:03 p.m.}
\begin{tabu} {X X}
& Approved and active
\\
\end{tabu}

    
\subsection*{Document endorsements and approvals as of July 11, 2016, 12:03 p.m.}
\begin{tabu} {X X}

%\rowcolor{grantedbg}
    \textbf{Project Team} & 
    \textcolor{granted}{ granted}\\

%\rowcolor{grantedbg}
    \textbf{Program Leader} & 
    \textcolor{granted}{ granted}\\

%\rowcolor{grantedbg}
    \textbf{Directorate} & 
    \textcolor{granted}{ granted}\\

\end{tabu}



}
\uppertitleback{}
\lowertitleback{}
\date{}

%-----------------------------------------------------------------------------%
% Front matter
\frontmatter
\maketitle
%-----------------------------------------------------------------------------%
% Main matter
\mainmatter

\section*{Genetic analysis for the development of vegetation services and
sustainable environmental management
}

M Byrne, D Coates, S van Leeuwen, S McArthur, R Binks, E Levy, B
Macdonald, M Millar


\section*{Context}
Understanding the genetic structure and function of plants is important
for their effective utilisation for revegetation, mine-site
rehabilitation and provision of ecosystem services, such as hydrological
balance, pollination and habitat connectivity.



\section*{Aims}
Provide genetic information for the conservation and utilisation of
plant species for revegetation and rehabilitation. Current work aims to
identify seed collection zones for species used in rehabilitation of
minesites in the Pilbara and the Midwest.



\section*{Progress}
\begin{itemize}
\itemsep1pt\parskip0pt\parsep0pt
\item
  A paper on genetic patterns in \emph{Acacia ancistrocarpa} and
  \emph{A. atkinsiana} has been submitted for publication. \emph{Acacia
  ancistrocarpa} has moderate haplotype diversity with most populations
  showing specific haplotypes. Nuclear diversity was moderate with
  little genetic structure across the Pilbara populations of this
  widespread species. In contrast, the Pilbara endemic, \emph{A.
  atkinsiana} had low haplotype diversity with little geographic
  structure. Nuclear diversity was low and genetic differentiation among
  populations was moderate.
\item
  A paper on genetics patterns in \emph{Eucalyptus leucophloia} has been
  submitted for publication. This species shows genetic signatures of
  the Hamersley Range being a historical refugium, supporting a previous
  hypothesis of inland ranges being refugia in the Australian arid zone.
\item
  A paper on the genetic diversity and differentiation in the rare
  \emph{Aluta quadrata} has been submitted for publication. The species
  shows high nuclear genetic differentiation but no chloroplast
  haplotype variability indicating that the geographic separation of the
  three species locations is leading to contemporary genetic
  differentiation but is not a result of historical isolation.
\item
  Studies have commenced on another eight species in the Pilbara for the
  identification of seed collection zones. Collections and
  microsatellite genotyping have been completed for all~eight~species,
  \emph{Petalostylis labicheoides}, \emph{Indigofera monophylla, Senna
  glutinosa},~\emph{Corymbia hamersleyana},~\emph{A. pruinocarpa, A.
  hilliana, A. spondylophylla} and \emph{Mirbelia} \emph{viminalis}.
  Reports have been written to summarise the results and provide
  provenancing recommendations for all species. Chloroplast sequencing
  is currently underway.~
\item
  A paper on phylog\emph{eo}graphic patterns and genetic diversity in
  \emph{Grevillea paradoxa} and \emph{Melaleuca nematophylla} is ready
  for peer review. In \emph{G. paradoxa}haplotype diversity within
  populations was low, diversity was moderate overall and there was a
  phylogeographic signal in chloroplast DNA. Nuclear diversity was low
  and genetic differentiation among populations was moderate to high
  with no signal of isolation by distance. In\emph{M.
  nematophylla}haplotype diversity within populations was low, diversity
  was moderate overall and there was a phylogeographic signal in
  chloroplast DNA. Nuclear diversity was low and genetic differentiation
  among populations was moderate with a signal of isolation by distance.
\item
  A paper on phylogeographic pattern and genetic diversity in
  \emph{Mirbelia} sp. bursarioides and \emph{G. globosa} has been
  published in \emph{The Botanical Journal of the Linnean Society}. In
  \emph{Mirbelia} \emph{.}sp. bursarioides haplotype diversity within
  populations was low, diversity was moderate overall and there was no
  phylogeographic signal in chloroplast DNA. Nuclear diversity was
  moderate and genetic differentiation among populations low to moderate
  with a signal of isolation by distance. In \emph{G. globosa}haplotype
  diversity within populations was low and diversity was low overall.
  There was no phylogeographic signal in chloroplast DNA. Nuclear
  diversity was moderate and genetic differentiation among populations
  was low with a signal of isolation by distance.
\item
  Comprehensive seed collection and restoration establishment guidelines
  for the four species from the Midwest region have been provided to
  Karara Mining Limited.
\end{itemize}



\section*{Management implications}
\begin{itemize}
\itemsep1pt\parskip0pt\parsep0pt
\item
  Pilbara seed collection zones--The high levels of genetic diversity
  and low levels of differentiation within \emph{E. leucophloia} and
  \emph{A. ancistrocarpa} imply that, for these species, seed resources
  for land rehabilitation and mine-site revegetation programs can be
  selected from a wide distributional range within the Pilbara. However,
  phylogeographic analysis of \emph{E. leucophloia} has identified the
  Hamersley Range as a historical refugia, so seed collections for
  rehabilitation of mine sites using this species should be targeted
  within the Hamersley Range to maximise the diversity of these sites.
  In contrast, the low diversity and high population differentiation in
  \emph{A. atkinsiana} indicates that more restricted seed collection
  zones should be observed.
\item
  \emph{Aluta quadrata}--The significant genetic structure in \emph{A.
  quadrata} indicates three conservation or management units: Western
  Range, Pirraburdoo and Howie's Hole. Given the genetic differences,
  restricted distribution and size of the populations, a precautionary
  approach should be taken to seed collections. Establishment of
  restoration populations within gene flow distance of existing
  populations should be done with seed from the location of that
  population. However, mixing seed collections from the three locations
  for establishment of restoration sites located distant to~existing
  populations would be a means of maximising genetic diversity for
  future conservation.
\item
  \emph{Grevillea paradoxa}--Moderate haplotype diversity and low levels
  of divergence among haplotypes of \emph{G. paradoxa} imply that there
  are no evolutionarily divergent lineages within this species. Genetic
  structuring and divergence in the nuclear genome does imply some
  limitation to pollen dispersal, likely due to territoriality in bird
  pollinators and an ability to self-pollinate. Three regional seed
  collection zones for land rehabilitation and mine-site revegetation
  programs may be appropriate for this species.
\item
  \emph{Melaleuca nematophylla}--Levels of divergence among haplotypes
  suggest the population of \emph{M. nematophylla} within the Murchison
  River gorge be treated as a divergent lineage and not incorporated
  into seed collection for rehabilitation and revegetation programs
  outside of this area. Low levels of divergence among populations in
  the nuclear genome implies that seed collections can otherwise be made
  across wide distributional areas.
\item
  \emph{Mirbelia bursarioides}--Low divergence among haplotypes implies
  a lack of divergent lineages for \emph{M. bursarioides}. A limited
  degree of genetic divergence among populations in the nuclear genome
  suggests that seed collections for rehabilitation and revegetation
  that encompass the distribution may be appropriate for this species.
\item
  \emph{Grevillea globosa}--Limited haplotype diversity and divergence
  and limited genetic structure in the nuclear genome imply that seed
  collections for rehabilitation and revegetation may be made across
  this species entire distribution.
\end{itemize}



\section*{Future directions}
\begin{itemize}
\itemsep1pt\parskip0pt\parsep0pt
\item
  Manuscripts currently in review will be finalised.
\item
  Analysis of the current eight species will be completed and
  recommendations on seed collection zones will be made.
\item
  Genetic diversity and phylogeographic patterns will be investigated in
  two more Pilbara species.
\end{itemize}



%-----------------------------------------------------------------------------%
% Back matter
%\backmatter
\end{document}
%-----------------------------------------------------------------------------%


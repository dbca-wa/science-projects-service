
\documentclass[version=last,
    paper=a4,                               % paper size
    10pt,                                   % default font size
    dvipsnames,
    % twoside,                                % PRINT Binding Correction
    oneside,                              % ONLINE
    headings=openany,                       % open chapters on odd and even pages
    open=any,
    BCOR=7mm,                               % PRINT Binding Correction
    %DIV=13,    % typearea 161.54mm x 228.46mm, top 22.85mm, inner 16.15mm
    %DIV=14,    % 165.00 233.36 21.21 15.00
    DIV=15,     % 168.00 237.60 19.80 14.00
    % toc=chapterentrywithdots              % Table of Contents style
]{scrbook}
\usepackage{typearea}


%------------------------------------------------------------------------------%
% Headers and footers
%------------------------------------------------------------------------------%
\usepackage[automark,headsepline,footsepline,plainfootsepline]{scrlayer-scrpage}
\automark*[section]{}
\addtokomafont{pageheadfoot}{\normalfont\footnotesize\sffamily} % Don't italicise
\renewcommand{\chaptermark}[1]{\markleft{#1}{}}     % Chapter: suppress numbering
\renewcommand{\sectionmark}[1]{\markright{#1}{}}    % Section: suppress numbering

% Header (inner, center, outer)
% \ihead{\href{http://sdis.dbca.wa.gov.au}{\textbf{Project Plan SP 2015-015}}}
%\chead{\href{http://sdis.dbca.wa.gov.au}{Science Directorate Information System}}
% \ohead{\href{https://www.dbca.wa.gov.au/science/10-biodiversity-and-conservation-science}{
% \includegraphics[height=8mm, keepaspectratio]{/usr/src/app/staticfiles/img/logo-dbca-bcs.jpg}}}

% Footer (inner, center, outer)
% \ifoot{\RaggedRight\leftmark}                       % Chapter
% \cfoot{\RaggedLeft\rightmark}                       % Section
% \ofoot[\bfseries\thepage]{\bfseries\thepage}        % Page number (also [plain])


%------------------------------------------------------------------------------%
% Fonts, encoding
%------------------------------------------------------------------------------%
%\usepackage{avant}             % Use the Avantgarde font for headings
\usepackage{txfonts}
\usepackage{mathptmx}
\usepackage{gensymb}            % provides \textdegree
\renewcommand{\familydefault}{\sfdefault} % Default to Sans Serif font
\usepackage{microtype}          % Slightly tweak font spacing for aesthetics
\usepackage[english]{babel}
\usepackage[utf8]{inputenc}  % Allow letters with accents
\usepackage[utf8]{luainputenc}  % Allow letters with accents
\usepackage[T1]{fontenc}        % Use 8-bit encoding that has 256 glyphs
\usepackage{textcomp}
\usepackage[explicit]{titlesec}           % Customise of titles
%\DeclareUnicodeCharacter{0080}{\textregistered}
\DeclareUnicodeCharacter{00B0}{\textdegree}

%------------------------------------------------------------------------------%
% Tables, columns, layout
%------------------------------------------------------------------------------%
\usepackage{etoolbox}
\AtBeginEnvironment{longtabu}{\footnotesize}{}{}  % Table font size
\usepackage{booktabs}           % Required for nicer horizontal rules in tables
\usepackage{multicol}           % 2 col publications
\usepackage{pdflscape}          % Landscape pages
\usepackage{pdfpages}           % Include PDFs
\usepackage{hanging}            % hanging paragraphs for publications
%\usepackage{titletoc}          % Manipulate the table of contents
\setcounter{tocdepth}{2}        % TOC list down to section
\usepackage{enumerate}          % Enumerations
\usepackage{enumitem}           % Enumerations
\usepackage{longtable}          % Multipage table
\usepackage{tabu}               %
\setlength{\tabulinesep}{1.5mm} % Consistent vertical spacing in tabu
\newcommand{\HRule}{\vspace{8mm}\noindent\rule{\linewidth}{0.1pt}}
\usepackage[export]{adjustbox}  % minipage, image frame


%------------------------------------------------------------------------------%
% Graphics, images, colours
%------------------------------------------------------------------------------%
\usepackage{graphicx} % embedded images
\usepackage{wrapfig}  % wrap figures in text
\usepackage{caption}  % allow unnumbered captions
\usepackage{eso-pic} % Required for specifying an image background in the title page
\usepackage{colortbl} % define custom named colours
\usepackage{xstring} % Conditionals
\usepackage{transparent} % Allow transparent images

\definecolor{RedFire}{RGB}{146,25,28}
% Following PICA branding guidelines
% https://dpaw.sharepoint.com/Divisions/pica/Documents/Branding%20guidelines.pdf
\definecolor{dpawblue}{RGB}{35,97,146}          % Pantone 647
\definecolor{dpaworange}{RGB}{237,139,0}        % Pantone 144
\definecolor{dpawgreen}{RGB}{116,170,80}        % Pantone 7489
\definecolor{dpawred}{RGB}{124,46,44}           % Paul's suggestion

% bootstrap colours
\definecolor{successbg}{RGB}{223,240,216}
\definecolor{errorbg}{RGB}{242,222,222}
\definecolor{warningbg}{RGB}{252,248,227}
\definecolor{infobg}{RGB}{217,237,247}
\definecolor{muted}{RGB}{153,153,153}
\definecolor{success}{RGB}{70,136,71}
\definecolor{error}{RGB}{185,74,72}
\definecolor{warning}{RGB}{192,152,83}
\definecolor{info}{RGB}{58,135,173}

% SDIS approval colours
\definecolor{required}{RGB}{192,152,83}
\definecolor{requiredbg}{RGB}{252,248,227}
\definecolor{denied}{RGB}{185,74,72}
\definecolor{deniedbg}{RGB}{242,222,222}
\definecolor{granted}{RGB}{70,136,71}
\definecolor{grantedbg}{RGB}{223,240,216}
\definecolor{notrequired}{RGB}{153,153,153}
\definecolor{notrequiredbg}{RGB}{255,255,255}

\usepackage{tikz} % Drawing
\usetikzlibrary{arrows,shapes,positioning,shadows,trees}


%------------------------------------------------------------------------------%
% Hyperlinks
%------------------------------------------------------------------------------%
\usepackage[open=true]{bookmark}
\usepackage{nameref}
\usepackage{ifxetex,ifluatex}
\ifxetex
  \usepackage[
    setpagesize=false,        % page size defined by xetex
    unicode=false,            % unicode breaks when used with xetex
    xetex]{hyperref}
\else
  \usepackage[unicode=true]{hyperref}
\fi

\hypersetup{
  backref=true,
  pagebackref=true,
  hyperindex=true,
  breaklinks=true,
  urlcolor=dpawblue,
  bookmarks=true,
  bookmarksopen=false,
  pdfauthor={Biodiversity and Conservation Science, Department of Biodiversity, Conservation and Attractions, WA},
  pdftitle=Project Plan SP 2015-015
,
  colorlinks=true,
  linkcolor=dpawblue,
  pdfborder={0 0 0}}

\urlstyle{same}                         % don't use monospace font for urlstyle


%------------------------------------------------------------------------------%
% Black magic to linebreak URLs
%------------------------------------------------------------------------------%
\usepackage{url}
\makeatletter\g@addto@macro{\UrlBreaks}{\UrlOrds}\makeatother
\Urlmuskip=0mu plus 1mu


%------------------------------------------------------------------------------%
% Fix latex errors
%------------------------------------------------------------------------------%
\providecommand{\tightlist}{\setlength{\itemsep}{0pt}\setlength{\parskip}{0pt}}

% copy-pasted HTML <span> in SDIS fields becomes \text{} in tex source
\providecommand{\text}{}


%------------------------------------------------------------------------------%
% Custom Tikz node for SDS diagram
%------------------------------------------------------------------------------%
\newcommand\mynode[6][]{
  \node[#1] (#2){
    \parbox{#3\relax}{
      \begin{center}
      \textbf{#4}\\
      #5\\
      \footnotesize{#6}
      \end{center}
    }};}


%------------------------------------------------------------------------------%
% Custom no-pagebreaks-environment
%------------------------------------------------------------------------------%
\newenvironment{absolutelynopagebreak}
  {\par\nobreak\vfil\penalty0\vfilneg\vtop\bgroup}
  {\par\xdef\tpd{\the\prevdepth}\egroup\prevdepth=\tpd}


%------------------------------------------------------------------------------%
% Remove the header from odd empty pages at the end of chapters
%------------------------------------------------------------------------------%
\makeatletter
\renewcommand{\cleardoublepage}{
\clearpage\ifodd\c@page\else
\hbox{}
\vspace*{\fill}
\thispagestyle{empty}
\newpage
\fi}


%----------------------------------------------------------------------------------------
%  Page flow control
%----------------------------------------------------------------------------------------
%\widowpenalty=10000
%\clubpenalty=10000
%\vbadness=1200
%\hbadness=11000


%----------------------------------------------------------------------------------------
%   CHAPTER HEADINGS
%----------------------------------------------------------------------------------------
\newcommand{\thechapterimage}{}
\newcommand{\chapterimage}[1]{\renewcommand{\thechapterimage}{#1}}

% Numbered chapters with mini tableofcontents
\def\thechapter{\arabic{chapter}}
\def\@makechapterhead#1{
%\thispagestyle{plain}
{\centering \normalfont\sffamily
\ifnum \c@secnumdepth >\m@ne
\if@mainmatter
\startcontents
\begin{tikzpicture}[remember picture,overlay]
\node at (current page.north west)
{\begin{tikzpicture}[remember picture,overlay]
\node[anchor=north west,inner sep=0pt] at (0,0) {
\includegraphics[width=\paperwidth,height=0.5\paperwidth]{\thechapterimage}};
%------------------------------------------------------------------------------%
% Small contents box in the chapter heading
% Mini TOC background box
%\fill[color=dpawblue!10!white,opacity=.2] (1cm,0) rectangle (
%  3.5cm, % Mini TOC box width
%  -3.5cm % Mini TOC box height
%);
% Mini TOC text content
%\node[anchor=north west] at (1.1cm,.35cm) {
%  \parbox[t][8cm][t]{6.5cm}{
%    \huge\bfseries\flushleft
%    \printcontents{l}{1}{
%    \setcounter{tocdepth}{1}                   % Mini TOC level depth
%    }
% }
%};
%------------------------------------------------------------------------------%
% Chapter heading
\draw[anchor=west] (5cm,-9cm) node [
rounded corners=20pt,
fill=dpawblue!10!white,
text opacity=1,
draw=dpawblue,
draw opacity=1,
line width=1.5pt,
fill opacity=.2,
inner sep=12pt]{
    \huge\sffamily\bfseries\textcolor{black}{
      \thechapter. #1\strut\makebox[22cm]{}
    }
};
\end{tikzpicture}};
\end{tikzpicture}}
\par\vspace*{240\p@}                            % Push text below chapter image
\fi
\fi}

%------------------------------------------------------------------------------%
% Unnumbered chapters without mini tableofcontents
%------------------------------------------------------------------------------%
\def\@makeschapterhead#1{
%\thispagestyle{plain}
{\centering \normalfont\sffamily
\ifnum \c@secnumdepth >\m@ne
\if@mainmatter
\begin{tikzpicture}[remember picture,overlay]
\node at (current page.north west)
{\begin{tikzpicture}[remember picture,overlay]
\node[anchor=north west,inner sep=0pt] at (0,0) {
  \includegraphics[width=\paperwidth,height=0.5\paperwidth]{\thechapterimage}};
% Mini TOC background box
%\fill[color=dpawblue!10!white,opacity=.2] (1cm,0) rectangle (
%  3.5cm,                                       % Mini TOC box width
%  -3.5cm                                       % Mini TOC box height
%);
% Mini TOC text content
%\node[anchor=north west] at (1.1cm,.35cm) {
%  \parbox[t][8cm][t]{6.5cm}{
%    \huge\bfseries\flushleft
%    \printcontents{l}{1}{
%    \setcounter{tocdepth}{1} % Mini TOC level depth
%    }
%}
%};
\draw[anchor=west] (5cm,-9cm) node [rounded corners=20pt,
  fill=dpawblue!10!white,fill opacity=.6,inner sep=12pt,text opacity=1,
  draw=dpawblue,draw opacity=1,line width=1.5pt]{
  \huge\sffamily\bfseries\textcolor{black}{#1\strut\makebox[22cm]{}}};
\end{tikzpicture}};
\end{tikzpicture}}
\par\vspace*{240\p@}
\fi
\fi
}
\makeatother



\usepackage[automark,headsepline,footsepline,plainfootsepline]{scrlayer-scrpage}
\automark*[section]{}
\addtokomafont{pageheadfoot}{\normalfont\footnotesize\sffamily} % Don't italicise
\renewcommand{\chaptermark}[1]{\markleft{#1}{}}     % Chapter: suppress numbering
\renewcommand{\sectionmark}[1]{\markright{#1}{}}    % Section: suppress numbering

% Header (inner, center, outer)
\ihead{\href{http://sdis.dbca.wa.gov.au/documents/projectplan/1512/}{Project Plan SP 2015-015}}
%\chead{\href{http://sdis.dbca.wa.gov.au}{Science Directorate Information System}}
\ohead{\href{https://www.dbca.wa.gov.au/science/10-biodiversity-and-conservation-science}{
\includegraphics[height=6mm, keepaspectratio]{/usr/src/app/staticfiles/img/logo-dbca-bcs.jpg}}}
% Footer (inner, center, outer)
\ifoot{\textbf{Printed}~Wed, 11 Dec 2019 13:45:27 +0800} % inner/left footer
\cfoot{}
\ofoot[\bfseries\thepage]{\bfseries\thepage}        % Page number (also [plain])


\pagestyle{scrheadings}
\setkomafont{pageheadfoot}{\normalfont}

%-----------------------------------------------------------------------------%
\begin{document}
\raggedbottom

%-----------------------------------------------------------------------------%
% Title page
\subject{Project Plan SP 2015-015
}
\title{Long-term monitoring in the area of the proposed Dampier Archipelago
marine reserves
}
\subtitle{Marine Science
}
\author{}
\publishers{\small
    \subsection*{Project Core Team}
\begin{tabu} {X X}
\textbf{Supervising Scientist} & Molly Moustaka
\\
\textbf{Data Custodian} & Molly Moustaka
\\
\textbf{Site Custodian} & 
\\
\end{tabu}


    \subsection*{Project status as of Dec. 11, 2019, 1:45 p.m.}
\begin{tabu} {X X}
& Approved and active
\\
\end{tabu}

    
\subsection*{Document endorsements and approvals as of Dec. 11, 2019, 1:45 p.m.}
\begin{tabu} {X X}

%\rowcolor{grantedbg}
    \textbf{Project Team} & 
    \textcolor{granted}{ granted}\\

%\rowcolor{grantedbg}
    \textbf{Program Leader} & 
    \textcolor{granted}{ granted}\\

%\rowcolor{grantedbg}
    \textbf{Directorate} & 
    \textcolor{granted}{ granted}\\

%\rowcolor{grantedbg}
    \textbf{Biometrician} & 
    \textcolor{granted}{ granted}\\

%\rowcolor{not requiredbg}
    \textbf{Herbarium Curator} & 
    \textcolor{not required}{ not required}\\

%\rowcolor{not requiredbg}
    \textbf{Animal Ethics Committee} & 
    \textcolor{not required}{ not required}\\

\end{tabu}



}
\uppertitleback{}
\lowertitleback{}
\date{}

%-----------------------------------------------------------------------------%
% Front matter
\frontmatter
\maketitle
%-----------------------------------------------------------------------------%
% Main matter
\mainmatter



\section*{Long-term monitoring in the area of the proposed Dampier Archipelago
marine reserves
}



\subsection*{Biodiversity and Conservation Science Program}

Marine Science




\subsection*{Departmental Service}

Service 6: Conserving Habitats, Species and Communities


\subsection*{Project Staff}
\begin{tabu} {X X X}
%\rowcolor{infobg}
\textbf{Role} & \textbf{Person} & \textbf{Time allocation (FTE)}\\

Supervising Scientist & Molly Moustaka & 1.0\\

Research Scientist & Thomas Holmes & 0.05\\

Research Scientist & Shaun Wilson & 0.01\\

Research Scientist & Alan Kendrick & 0.01\\

Research Scientist & Kathy Murray & 0.06\\

\end{tabu}




\subsection*{Related Science Projects}

None


\subsection*{Proposed period of the project}
July 15, 2014 -- June 30, 2019



\section*{Relevance and Outcomes}


\subsection*{Background}

The Dampier Archipelago is situated adjacent to Dampier (pop. 1,341) and
Karratha (pop. 25,907), 1,650 km north of Perth in the Pilbara Nearshore
Marine Bioregion (Figure 1). The marine and coastal environment of the
Dampier Archipelago/Cape Preston Region has a unique combination of 42
offshore islands, intertidal and subtidal reefs, mangroves, macroalgal
communities and coral reefs. The area is the second most diverse region
for hard corals in Western Australia, with coral reef habitats
comprising 8\% (approx. 18,300 ha) of the major marine habitats,
consisting of 229 scleractinian coral species from 57 genera (Griffith
1998). In surveys carried out in 1998-2002 the Western Australian Museum
recorded 736 fish species (Hutchins 2001), over 1,227 species of
mollusc, including five regional endemics (Slack-Smith and Bryce 2004),
438 species of crustacean (Hewitt 2004), 275 sponge species (Fromont
1998), and 286 echinoderm species (Marsh and Morrison 2004).

~

Not only is there high species diversity in the area but many of the
islands and beaches provide habitats and feeding areas for threatened
marine fauna. A number of beaches in the area are important nesting
sites for five species of marine turtle. There are at least 16 species
of sea birds known to form large nesting colonies in the region. Over
50\% of the mainland shore is lined with complex mangrove habitats,
comprising six species, and are considered to be of international
significance (Semeniuk and Wurm 1987). The area also contains important
feeding and travelling grounds for marine mammals. The annual humpback
whale migration passes through the Dampier Archipelago, and eight
toothed whale species and four baleen whales have been documented
utilising the area. Dugongs are also regularly sighted using the
extensive seagrass beds within the area.

~

The Dampier Archipelago is rich in cultural and historical value, with
the indigenous history of the Yaburara people extending back 20,000
years. The area is referred to as Murujuga land and contains some of
Australia's earliest art and cultural sites (CALM 2005).~ Other
culturally significant areas include; quarries, middens, fish traps,
rock shelters, ceremonial places, artefact scatters, grinding patches
and stone arrangements.

~

Within the Dampier Archipelago/Cape Preston area, species and habitat
depletion have been recorded (CALM 2005), with the area subject to
increasing human pressures. The nearby town of Karratha has the highest
per capita boat ownership in Australia (CALM 2005)and recreational
fishing pressure in the archipelago waters is likely to be high. There
are also commercial prawn, demersal finfish, and interim mackerel
fisheries, with the area also important to Western Australia's aquarium
fishery. There are also multiple pearl aquaculture sites, a productive
saltworks (opened in 1972) and Dampier Port development, pipeline and
processing facilities and shipping associated with industrial
development.

~

\href{http://internal-data.dpaw.wa.gov.au/dataset/4afb0762-81ee-4d29-aa12-2728f725d85d/resource/d9b76f02-f7ed-4bcc-b653-c39533d24c8f/download/map-2.jpg}{Proposed
MP Zoning}

Figure 1 Map of the Dampier Archipelago showing locality and the three
sections of the proposed marine protected area.~ Zoning in protected
area; Green -- General Use, General Use and Recreation zones. Yellow --
Conservation Zones, Red -- Sanctuary Zones.

~

The Dampier Archipelago/Cape Preston area was identified for
consideration as a marine conservation reserve by the \emph{Marine Parks
and Reserves Selection Working Group} (MPRSWG 1994), which identified
representative and unique areas of Western Australia's marine waters for
consideration as part of a statewide system of marine parks. Waters
surrounding the Dampier Archipelago were reported as having significant
conservation values, and the report highlighted the overlap of
ecological values and high levels of human activity in the area. An
Indicative Management Plan (IMP; CALM 2005) has been written to provide
a framework for how the marine ecological and social values of the area
can be protected. The IMP proposes creation of three separate marine
protected areas; Dampier Marine Park East, Dampier Marine Park West and
the Regnard Marine Management Area (Figure 1). In total the proposed
protected areas cover 214,920 ha of State waters.

~

Woodside's Pluto liquefied natural gas (LNG) project is an approved
operation utilising the Dampier area for processes around the extraction
of gas, transport via a gas pipeline to an onshore processing facility,
storage and then export via a shipping port, loading jetty and channel.
Alongside the LNG processing and export facility, a large scale dredging
program, removing 14 million cubic meters of sediment, has been
occurring since 2003. The EPA approved the development on the condition
that Woodside implement a substantive offset package. The Pluto LNG
Offset `D' Program was initiated as a result of Ministerial Approval
Statement 757 and included a funding agreement between Woodside Burrup
Pty Ltd in its capacity as operator of the Pluto LNG development and the
state of Western Australia. The agreement defines the focus of the Pluto
LNG Offset `D' as `\emph{Research and monitoring consistent with the
Indicative Management Plan for the Dampier Archipelago Marine Park'}.




\subsection*{Aims}

There are four main projects within the Pluto LNG Offset `D' Program.
They are to:

\begin{enumerate}
\def\labelenumi{\roman{enumi}.}
\tightlist
\item
  \emph{Review, assess and summarise historical data relevant to the
  management of the proposed Dampier Archipelago Marine Park and Regnard
  Marine Management Area;}
\item
  \emph{Determine distribution, patterns and key processes of major
  marine communities and large marine fauna of the proposed Dampier
  Archipelago Marine Park and Regnard Marine Management Area;}
\item
  \emph{Describe patterns and trends in human use in the proposed
  Dampier Archipelago Marine Park and Regnard Marine Management Area;}
  and
\item
  \emph{Establish long-term monitoring reference sites in the proposed
  Dampier Archipelago Marine Park and Regnard Marine Management Area.}
\end{enumerate}

The focus of this SPP is on the project `iv' which required the
implementation of monitoring within the Proposed Dampier Archipelago
Marine Park and Regnard Marine Management Area. Monitoring will be
consistent with the State Government's Indicative Management Plan for
the area. The aim of project `iv' is to:

~

\emph{Develop and implement a monitoring, evaluation and reporting (MER)
program of key biodiversity asset condition and their major pressures at
key sites in the proposed Dampier Archipelago Marine Park and Regnard
Marine Management Area}.

~

The project will implement an integrated, effective and efficient
monitoring, evaluation and reporting process that allows measurement and
evaluation of condition and pressure changes in a way that is robust but
practical and cost effective. The objectives of this project are:

\begin{itemize}
\tightlist
\item
  Complete a prioritisation of ecological assets listed in the
  indicative management plan to identify high priority assets for
  monitoring;
\item
  Establish reference sites for monitoring high priority assets in the
  Proposed Dampier Archipelago Marine Park and Regnard Marine Management
  Area, taking into account, spatial and temporal distribution of key
  pressures acting on assets, previous data availability, the standards
  of the Western Australian Marine Monitoring Program and stakeholder
  recommendations;
\item
  Provide quantitative evidence on the status and trends in selected
  indicators of asset condition and the pressure/s on these assets, to
  assist managers to make better-informed decisions within an active
  adaptive management framework;
\item
  Contribute to the development of best practice monitoring techniques
  for assets and pressures where knowledge is lacking;
\item
  Contribute data, as appropriate, to assist development of Parks and
  Wildlife's strategic marine planning and conservation initiatives;
\item
  Assist other government departments and industry environmental
  management programs; and
\item
  To establish links with indigenous owners and encourage their
  involvement in monitoring and marine conservation.
\end{itemize}




\subsection*{Expected outcome}

\begin{itemize}
\tightlist
\item
  \emph{A better understanding of the value, spatial extent and status
  of the shallow tropical communities of the Dampier archipelago.}
\end{itemize}

An established monitoring program will facilitate evidence-based
long-term adaptive management for the proposed reserves, and promote
environmental understanding for a range of stakeholders and the
community.

\begin{itemize}
\tightlist
\item
  ~\emph{Data on ecological assets will inform a baseline of natural and
  human induced change when the marine reserves are gazetted.}
\end{itemize}

This project will produce baseline data on various key ecological assets
and the pressures acting on them. This will inform Parks and Wildlife
managers, the vesting authority, stakeholders and the broader community
of the condition of the marine reserves over time.

\begin{itemize}
\tightlist
\item
  ~~\emph{Parks and Wildlife and the scientific community will benefit
  from having newly developed, tested and documented indicators and
  methods for monitoring Pilbara marine environments.}
\end{itemize}

This project will develop new marine monitoring indicators and methods
for key ecological assets and habitats that are largely novel to
northern WA. Development of these monitoring indicators and methods may
also assist marine monitoring implementation and reserve management in
the Kimberley region.

\begin{itemize}
\tightlist
\item
  \emph{Assistance for other government departments and Industry through
  better understanding of the importance of, and change in, the natural
  assets of the Dampier Region.}
\end{itemize}

The data collected by this project will be accessible in the event of an
environmental incident, so that industry has baseline evidence of the
condition of marine communities. Providing information to major
industrial users of the area will create awareness of the condition and
pressure on assets and values in the proposed marine reserve and it will
also highlight the role of Parks and Wildlife in conservation
management.~

\begin{itemize}
\tightlist
\item
  ~\emph{Foster operational collaborations with traditional owners.}
\end{itemize}

Parks and Wildlife recognizes joint management with Aboriginal people as
a primary objective of conservation management. This project will
nurture collaboration and information sharing through consultation with
the Murujuga Aboriginal Corporation, whose ranger group is to be
involved in field activities.




\subsection*{Knowledge transfer}

The most significant use of this information is for the management of
the marine environment, both through the Parks and Wildlife planning
department, specialist branches (e.g. science and conservation division)
and regional operational staff. The information gained from this project
will be used to inform planning for the proposed marine reserves (by
Parks and Wildlife and the vesting authority), and assist ~~Parks and
Wildlife regional staff to gain an understanding of change in ecological
assets in relation to marine park management.

~

Information collected in this project will also be provided to other
government management agencies. This work may advise the Office of the
Environmental Protect Authority during the Environmental Impact
Assessment and support research and monitoring carried out by Department
of Fisheries and CSIRO. When the park is established, historical data,
as well as data collected through this program, will be ready to feed
directly into the MPRA (Marine Parks and Reserves Authority) reporting
process carried out by Marine Science Program.

~

A copy of all major reports will be forwarded to Woodside and
baseline/monitoring data will be made available to the oil spill
response team in the event of an incident.

~

One of the aims of this work is to involve the indigenous community and
keep them updated on the condition of the system. Researchers on this
project will also seek advice from traditional owners throughout the
program.

~

As part of this work we will communicate information on the current
status of WA marine assets to the community. Communication outputs will
include more `general reading' publications, e.g. Parks and Wildlife
Landscope.




\subsection*{Tasks and Milestones}

The specific outputs of this project are:

Project Administration:

\begin{itemize}
\tightlist
\item
  Annual progress reports to Woodside completed by 30 September each
  year;
\item
  Pluto Offset `D' `iv' final summary report completed and distributed
  by 30 June 2019;
\end{itemize}

Project Science-Management Outputs

\begin{itemize}
\tightlist
\item
  Science outcomes and new knowledge will be published in peer-reviewed
  journals;
\item
  Monitoring related updates sent to Pilbara District and Regional
  managers, and Murujuga Aboriginal Corporation;
\end{itemize}

Project General Communication:

\begin{itemize}
\tightlist
\item
  Literature regarding the Dampier marine environment, methods for
  monitoring assets and indicative long-term trends will be presented in
  Parks and Wildlife publications (e.g. Landscope, brochures and social
  media.
\end{itemize}

\textenglish[variant=american]{The major tasks and milestones for this
project
are~\href{http://internal-data.dpaw.wa.gov.au/dataset/4afb0762-81ee-4d29-aa12-2728f725d85d/resource/61b24cf4-4800-44a8-aa21-5f92762f51d0/download/milestones-pdf.pdf}{Milestone
and Major Activities}}




\subsection*{References}

Babcock RC, Baird AH, Piromvaragorn S, \emph{et al} (2003)
Identification of scleractinian coral recruits from Indo-Pacific reefs.
Zool Stud 42:211--226.

Biota Environmental Sciences (2009) Turtle monitoring at Bells Beach and
selected rookeries of the Dampier~ Archipelago: 2008/09 Season. Rio
Tinto

CALM (2005) Indicative Management Plan for the Proposed Dampier
Archipelago Marine Park and Cape Preston Marine Management Area.
Department of Conservation and Land Management, Perth, W.A.

Fromont J (1998) Porifera (sponges) of the Dampier Archipelago, Western
Australia: habitats and distribution. Aquatic Fauna of the Waters of the
Dampier Archipelago, Western Australia Report on the results of the
Western Australia Museum/Woodside Energy Ltd partnership to explore the
marine biodiversity of the Dampier Archipelago 2001:69--100.

Griffith JK (1998) Scleractinian corals collected during 1998 from the
Dampier Archipelago, Western Australia. Report on the results of the
Western Australia Museum/Woodside Energy Ltd partnership to explore the
marine biodiversity of the Dampier Archipelago 2002:101--120.

~

Harvey E, Fletcher D, Shortis M (2002) Estimation of reef fish length by
divers and by stereo-video: A first comparison of the accuracy and
precision in the field on living fish under operational conditions.
Fisheries Research 57:255--265.

~

Hewitt MA (2004) Crustacea (excluding Cirripedia) of the Dampier
Archipelago, Western Australia. Records of the Western Australian Museum
Supplement 66:169--219.

~

Hutchins BJ (2001) Biodiversity of shallow reef fish assemblages in
Western Australia using a rapid censusing technique. Records of the
Western Australian Museum 20:247--270.

~

Jones DS (2004) Report on the results of the Western Australian
Museum/Woodside Energy Ltd. partnership to explore the marine
biodiversity of the Dampier Archipelago Western Australia 1998-2002.
Western Australian Museum, Perth, Western Australian

~

MPRSWG (1994) A representative marine reserve system for Western
Australia. Department of Conservation and Land Management, Perth, W.A.

~

Marsh LM, Morrison SM (2004) Echinoderms of the Dampier Archipelago,
Western Australia. Records of the Western Australian Museum 66:293--342.

~

Semeniuk V, Wurm P (1987) Mangroves of the Dampier Archipelago, Western
Australia. Journal of The Royal Society of Western Australia 69:29 --
87.

~

Simpson C, Beger M, Colman J, \emph{et al} (2015) Prioritisation of
conservation research and monitoring for Western Australian protected
areas and threatened species. Conservation Science Western Australia
9:227 -- 237.

~

Slack-Smith SM, Bryce CW (2004) A survey of the benthic molluscs of the
Dampier Archipelago, Western Australia. Records of the Western
Australian Museum Supplement 66:221--245.

~

Spiers M, Vitenbergs A, Prince R (2013) Rosemary Island hawksbill turtle
tagging program. In: In Proceedings of the First Western Australian
Marine Turtle Symposium, 28-29th August 2012. (eds B Prince, S Whiting,
H Raudino \emph{et al.}) Department of Parks and Wildlife, Kensington,
W.A.

~

Turner JA, Polunin VC, Field SN, Wilson SK (2015) Measuring coral
size-frequency distribution using stereo video technology, a comparison
with in situ measurements. Environmental Monitoring and Assessment
187:234.



\section*{Study design}


\subsection*{Methodology}

\textbf{Context}

The Indicative Management Plan describes 14 ecological assets detailed
in the Indicative Management Plan, which are: geomorphology, sediment
and water quality, coral reef communities, mangrove communities,
macroalgae and seagrass communities, subtidal soft-bottom communities,
intertidal sand and mudflat communities, rocky shore communities,
turtles, marine mammals, seabirds, finfishes, and invertebrates.

~

\textbf{Prioritisation}

All assets need to be considered for monitoring, however, managers must
prioritise assets to ensure the effective use of limited resources
(Simpson \emph{et al.} 2015). Following the framework outlined in
Simpson \emph{et al} (2015) a preliminary monitoring prioritisation was
completed prior to planning this project and proceeding with field
surveys. This involved requesting the assistance of a panel of experts,
with experience in the Dampier region, to score the value of each asset
and the pressure they perceive this asset to be under. Assets that were
considered of high value and under high pressure will be the initial
focus of monitoring. Twelve experts from government, industry and
academia, with good knowledge of the marine environment in the Dampier
Archipelago, were surveyed and the resultant order of asset priority is
presented in Table 1.

~

Table 1 Results of monitoring prioritisation carried out by 12 experts
with knowledge of the Dampier region. The mean values are presented with
both upper and lower Confidence Intervals. The asset with the highest
mean rank, and all other assets with 95\% confidence intervals that
overlap with highest asset, were highlighted in bold as high priority
for monitoring.

~

\begin{longtable}[]{@{}lllll@{}}
\toprule
\endhead
\textbf{Asset} & \textbf{~ KPI? ~} & \textbf{~ Mean Score ~} & \textbf{~
Lower 95\% CI ~} & \textbf{~ Upper 95\% CI ~}\tabularnewline
Coral communities & Y & 181.28 & 151.42 & 211.13\tabularnewline
Finfish & Y & 169.43 & 141.60 & 197.25\tabularnewline
Turtles & Y & 124.66 & 95.51 & 153.80\tabularnewline
Mangrove communities & Y & 123.26 & 91.88 & 154.64\tabularnewline
Macroalgae and seagrass communities & N & 115.77 & 91.10 &
140.53\tabularnewline
Invertebrates & Y & 103.22 & 78.32 & 128.12\tabularnewline
Marine mammals & N & 100.21 & 59.88 & 140.54\tabularnewline
Water quality & Y & 90.54 & 60.20 & 120.88\tabularnewline
Geomorphology & N & 84.75 & 44.34 & 125.15\tabularnewline
Seabirds & N & 71.56 & 52.01 & 91.11\tabularnewline
Subtidal soft-bottom communities & Y & 66.26 & 40.81 &
91.71\tabularnewline
Intertidal sand and mudflat communities & Y & 64.58 & 42.60 &
86.56\tabularnewline
Sediment quality & N & 63.56 & 40.92 & 86.20\tabularnewline
Rocky shore communities & N & 54.75 & 41.69 & 67.81\tabularnewline
\bottomrule
\end{longtable}

~

Based on the priortisation of assets, coral communities were identified
as having the highest mean rank, though fish communities, mangroves and
turtles were also considered high priority and will be the focus of
monitoring efforts. Macroalgae and seagrass, as well as invertebrates
also ranked highly, but as a lower priority for monitoring. Monitoring
of other assets may be undertaken opportunistically or by other
researchers and we will endeavor to support their programs where
appropriate.

~

\textbf{Site selection process}

Sites are to be selected based on a combination of factors; the
availability of historical data, site descriptions, and vicinity to
anthropogenic pressures. Sites will be selected to maximize potential of
recording change as a result of anthropogenic pressures, while
minimizing the potential range of confounding factors (including natural
pressures). Detailed site descriptions by the Western Australian Museum
(Jones 2004) will assist with site selection. Also, pressure variables
found within datasets, such as water quality and industrial development
projects will be used to advise site selection.

~

A scoping fieldtrip is to be carried out to visit an extended number of
potential sites for monitoring high priority assets. These will be
identified using the above criteria and assessed for suitability.
Wherever possible, sites will be selected where multiple assets are well
represented. For example, at reef sites attempts will be made to monitor
coral, finfish and invertebrate communities concurrently. By monitoring
multiple assets at the same site, results can advise on interactions
between communities and highlight secondary pressures.

~

\textbf{Indicators}

WAMMP `Asset Knowledge Reviews' highlight important asset condition,
pressure and management response (CPR) indicators and recommend methods
for their measurement. This project will measure the condition of
selected assets in the context of possibly interacting natural,
anthropogenic and climate change pressures. Monitoring will use standard
WAMMP methods, except where new indicators and methods need to be
identified.

~

Table 2 Summary table of assets and condition and pressure metrics.
Condition and pressure metrics for tropical macroalgae and seagrass
communities have, as of yet, not been well defined

\begin{longtable}[]{@{}lll@{}}
\toprule
\endhead
\begin{minipage}[t]{0.30\columnwidth}\raggedright
\textbf{~ Assets}\strut
\end{minipage} & \begin{minipage}[t]{0.30\columnwidth}\raggedright
\textbf{~ Asset Condition Metrics}\strut
\end{minipage} & \begin{minipage}[t]{0.30\columnwidth}\raggedright
\textbf{~ Pressure Metrics}\strut
\end{minipage}\tabularnewline
\begin{minipage}[t]{0.30\columnwidth}\raggedright
Coral Communities ~\strut
\end{minipage} & \begin{minipage}[t]{0.30\columnwidth}\raggedright
\% live coral cover

Community composition

Coral recruitment\strut
\end{minipage} & \begin{minipage}[t]{0.30\columnwidth}\raggedright
Water temperature

Predator abundance i.e. Crown of thorns starfish and corallivorous
gastropods

Physical disturbance i.e. storms and cyclones

Sedimentation/Turbidity\strut
\end{minipage}\tabularnewline
\begin{minipage}[t]{0.30\columnwidth}\raggedright
Finfish\strut
\end{minipage} & \begin{minipage}[t]{0.30\columnwidth}\raggedright
Abundance of fishery target spp

Abundance of functionally important spp

Community composition

Community size structure\strut
\end{minipage} & \begin{minipage}[t]{0.30\columnwidth}\raggedright
Habitat change

Fishing pressure\strut
\end{minipage}\tabularnewline
\begin{minipage}[t]{0.30\columnwidth}\raggedright
Turtles\strut
\end{minipage} & \begin{minipage}[t]{0.30\columnwidth}\raggedright
Breeding metrics (assist current programs)

Strandings/incidents\strut
\end{minipage} & \begin{minipage}[t]{0.30\columnwidth}\raggedright
Island visitors

Boat use/ownership\strut
\end{minipage}\tabularnewline
\begin{minipage}[t]{0.30\columnwidth}\raggedright
Mangrove Communities\strut
\end{minipage} & \begin{minipage}[t]{0.30\columnwidth}\raggedright
Spatial extent

Canopy cover

Tree density

Foliage condition (leaf colour)

Mortality\strut
\end{minipage} & \begin{minipage}[t]{0.30\columnwidth}\raggedright
Salinity

Air Temperature

Clearing\strut
\end{minipage}\tabularnewline
\begin{minipage}[t]{0.30\columnwidth}\raggedright
Macroalgae \& Seagrass\strut
\end{minipage} & \begin{minipage}[t]{0.30\columnwidth}\raggedright
~\% cover/density

Community composition\strut
\end{minipage} & \begin{minipage}[t]{0.30\columnwidth}\raggedright
~Water Temperature

Physical disturbance i.e. storms and cyclones

Sedimentation/Turbidity\strut
\end{minipage}\tabularnewline
\begin{minipage}[t]{0.30\columnwidth}\raggedright
Invertebrates\strut
\end{minipage} & \begin{minipage}[t]{0.30\columnwidth}\raggedright
Abundance of fishery target spp

Abundance of functionally important spp\strut
\end{minipage} & \begin{minipage}[t]{0.30\columnwidth}\raggedright
Habitat change

Fishing pressure\strut
\end{minipage}\tabularnewline
\bottomrule
\end{longtable}

~

\textbf{High Priority Assets -- Sites and Methods}

Assets identified as high priority in the prioritisation process will be
monitored initially, with main effort focusing on establishing permanent
sites, collecting baseline datasets and where appropriate publication of
results.

~

\emph{Coral communities}

Digital still imagery along fixed transects will be used to provide data
on the live cover, composition, and diversity of coral communities. At
each site, three replicate 50 m transects will be permanently marked
with star pickets positioned at each end of each transect and marked for
relocation using a GPS at the surface. To help ensure the accurate
resampling along the transect steel marker rods will be fixed to the
substrate every 10m. Benthic community along these transects will be
monitored by taking 50 digital images per transect at a height of 1 m
from the substrate. This will enable analysis of the structural benthic
community metrics in the laboratory using a point count method, while
providing a permanent record for potential investigation.

~

Coral settlement is also highlighted as an important functional metric
for monitoring coral reef communities (coupled with the examination of
shifts in coral size frequency). The use of coral settlement tiles can
estimate larval supply to an area and provide a relatively simple
estimate of recruitment potential (Babcock \emph{et al.} 2003). Crude
assessments of juvenile coral abundance may also be gathered from
digital images (Turner \emph{et al.} 2015).

~

Coral communities will be monitored every two years at a range of sites
over the proposed marine reserves. Fifteen preliminary sites (Figure 2)
have been selected for monitoring and attempts will be made to monitor
other assets (fish and invertebrates) concurrently at these sites.

~

\href{http://internal-data.dpaw.wa.gov.au/dataset/4afb0762-81ee-4d29-aa12-2728f725d85d/resource/0e66905a-8fe6-4099-b2f8-ddb2058d03e8/download/figure-2.pdf}{Coral
and Fish Monitoring Sites}

Figure 2~\textenglish[variant=american]{Preliminary locations (red
points) of fish, coral and invertebrate monitoring sites in the proposed
Dampier Archipelago Marine Park and Regnard Marine Management Area. The
location of temperature loggers are marked with blue crosses. Also
included is a table listing site names.}

~

\emph{Fish}

Fish communities will be monitored for changes in community abundance,
diversity and population structure (major indicators, Table 2) using
stereo video methods, which enables the identification of species and
measurements of fish size (Harvey \emph{et al.} 2002).

~

Six independent 50 meter transects will be completed by employing Diver
Operated Video (DOV). This provides a permanent record of the fish
assemblage and allows analysis of the videos in the laboratory by
experienced staff. At each site, video transects will be swum by a two
diver team. The first person in the dive team swims the DOV unit, with
the second person measuring the transect. The divers leave a 10 meter
gap between each transect to maintain independence of replicates.
Transects are swum at a rate of three-four minutes per 50m in a
relatively straight line (following habitat and depth gradient), in a
direction away from the sun.

~

The stereo-video equipment comprises two forward facing video units
secured to an aluminium frame. A light emitting diode is positioned on a
rod in front of the cameras so the footage can be syncronised when it is
viewed on a computer. The video units are configured to record the fish
from different angles. Analysis of high definition video footage is
completed in the laboratory following the fieldtrip, allowing accurate
identification of individuals and length measurements. Digital imagery
is analysed using the seaGIS software `Event Measure', which allows for
the calculation the accurate size of the fish by mathematical
triangulation.

~

Fish communities will be monitored at the same 15 sites assessed for the
benthic community surveys (Figure 2). This allows fish community data to
be associated with benthic structure and composition information. Fish
communities will be monitored every two years.

~

\emph{Turtles}

Turtles have dedicated research and monitoring programs through the
Marine Science Program. At present there is a large-scale study of the
turtle populations in the Pilbara, being undertaken by Parks and
Wildlife, Marine Science Program (SPP 002-2013), and this will include
an assessment of populations in the Dampier Archipelago. There are
presently multiple programs monitoring turtle nesting running throughout
the Dampier Archipelago. Parks and Wildlife are working with Rio Tinto
to monitor Hawksbill, Green and Loggerhead turtles at Rosemary Island
(Spiers \emph{et al.} 2013). In addition Rio Tinto is also funding a
long term turtle monitoring program at Legendre and Delambre Island
(Biota Environmental Sciences 2009). Since this asset is well covered in
the Dampier region, it would not be cost effective to establish another
program, when time and funding could be spent focusing on other less
researched assets. Therefore, this project will attempt to assist the
current turtle monitoring work where possible through field support and
the collection and contribution of pressure related data.

~

\emph{Mangroves}

The extent, condition and cover of mangrove communities are the main
indicators (Table 2) of health and these will be assessed using remote
sensing techniques. It is a requirement that these images be
ground-truthed, and this will allow for assessment of on ground health
indicators including tree density and foliage condition.

~

Rectified and calibrated satellite imagery allows for fast, cost
effective monitoring of mangroves over large spatial scales. This will
be achieved using ALOS or SPOT 6 imagery, which have a pixel size of ten
and five meters, respectively. A time series of three historical records
will be analysed initially to develop an appropriate baseline time
series index of vegetation trends. Images will then be purchased every
two years of the monitoring program, to reduce cost outlay, and also
align with the bi-annual schedule used for monitoring other assets.
Imagery will be collected in November, when they become available. The
major aim of this monitoring is to identify communities that are stable,
in decline or increasing over the time period of years captured by
quantifying change in mangrove extent and foliage cover (as a proxy for
condition).

~

Ground truthing will be undertaken at predetermined points among the
survey sites. The purpose of ground truthing is to verify data
calculated from remote techniques, it is therefore not necessary to have
ongoing \emph{in situ} work for the monitoring program. However, there
are cases where re-survey may be necessary, such as after a major event
or where difficulty exists interpreting the results of the remote study.

~

Ground truthing sites will be chosen according to satellite image
characteristics. Two types of `set point photograph' will be taken at
each point; the first will be a wide angle overview of the community as
a whole; the second will be a series of close up photographs taken along
the fringes of the communities, or in areas of disturbance (as these are
the areas that are the most accessible and where change is most likely
to occur). In the second series of photos individual trees will be
identified and photographed from a distance of 8 meters. A surveyor's
staff will be included in all photos taken to determine tree height and
foliage width. Four upward facing photos will be taken, capturing the
canopy cover and 4-6 photos of the substrate under/around the tree will
be collected.Field notes will be collected along with the photographs
and these will include basic measurements, including; tree height and
trunk width, tree mortality and damage, and pneumatophore density. A
photo of the tripod position relative to the tree will also be taken for
future reference. Additional GPS marks will be collected for the tree
trunk and tripod position.

~

A total of 10 areas have been selected for mangrove monitoring (Figure
3) based on the availability of GIS imagery encompassing existing
mangrove forests. The sites are distributed inside and outside of the
proposed marine reserves and at varying proximity to development.

~

\href{http://internal-data.dpaw.wa.gov.au/dataset/4afb0762-81ee-4d29-aa12-2728f725d85d/resource/ea96c0ea-1409-4c6d-884a-6fc7e9b084c1/download/figure-3.pdf}{Mangrove
Ground Truth Sites}

Figure 3 Locations (blue boxes) of GIS imagery for long term monitoring
of mangrove communities. Points for ground truthing will be selected
within these boundaries

~

\textbf{Secondary Priority Assets -- Methodological Development}

In order to increase the capacity of the Marine Science Program,
attempts will be made to develop some asset monitoring programs that are
not currently established. This work will be second priority to the
assets identified above, and these projects will only be attempted if
time and capacity allow.

~

\emph{Macroalgae and Seagrass}

Indicators and methods for monitoring tropical seagrass and macroalgae
in northern WA are yet to be identified. With the assistance of the
WAMMP Asset Leader for seagrass and algae, this project will collaborate
with internal and external groups to identify appropriate monitoring
techniques. In order to develop techniques, there will be an initial
collection of research work, drafting a literature review and trial of
appropriate methodologies in the field. A cost-benefit analysis will
then be undertaken to determine a method which is achievable and returns
appropriate data. Once a Standard Operating Procedure is in place,
appropriate sites will be selected using remote imagery, ground truthed,
and monitoring will commence in the Dampier region.

~

Primary productivity, and natural variability, of benthic primary
producers has been identified as potential area of focus for research in
the Dampier Archipelago, and has been highlighted as a research priority
in Pluto Offset `D' project `ii'. The work undertaken in this project,
will compliment and extend the findings of that research.

~

\emph{Invertebrates}

The main indicator for change in invertebrate communities is abundance
(Table 2), particularly those targeted through human activities (e.g.
collection of crustaceans, large clams, topshells, urchins and sea
cucumbers), but also species which may pose a threat to other
communities, e.g. pressure on the condition of corals from crown-of
thorns starfish (\emph{Acanthaster planci}) and the corallivorous
gastropod \emph{Drupella}.

~

Methods have been established to monitor invertebrates on coral reefs;
however, if time allows, part of this project will include research into
the development of monitoring techniques for an array of target
invertebrates harvested from other habitats, such as mangroves and
seagrass. As part of this work, research will be undertaken to identify
priority species targeted by commercial and recreational fisheries and
aquarium collectors in the region.

~

Invertebrate monitoring will be undertaken at multiple sites throughout
the Dampier, and attempts will be made to monitor within a subset of
different benthic communities including coral reefs, mangroves, seagrass
and macroalgae. In some cases, and where appropriate, invertebrate
monitoring may be carried out at complementary benthic monitoring sites.
This co-location of assessment sites will allow for comparison between
benthic structure and composition, to assess how invertebrate abundance
can influence other communities, and how changes in habitat structure
can influence invertebrate communities.

~

\textbf{Pressure Metrics}

The Department's marine monitoring program operates within a condition,
pressure and management response (CPR) framework to enable and
facilitate the assessment of management effectiveness. Monitoring
enables a better understanding of the effects of anthropogenic pressures
on marine assets, and through subsequent targeted but adaptive
management allows for mitigation of their impacts. Understanding
pressures is vital to being able to respond to changes in the condition
of assets and determining management and level of protection required.
As part of this project, it is necessary to understand and monitor the
major pressures on the assets that are sought to be conserved and
managed in the proposed reserve.

~

Pressures to be monitored are listed in Table 2. Some of these are
already addressed through monitoring of other or assets e.g. habitat
loss which is a pressure on fish and invertebrate communities is
measured through monitoring changes in cover of benthic communities. For
other pressure metrics, data will be collected independently.

~

Variation in the physical condition of coastal waters can drive change
in marine communities. Temperature is one of the most significant
pressure parameters for ecological assets. At five coral monitoring
sites, two temperature loggers will be deployed for the extent of the
project, and replaced at alternating six month periods (one swapped in
May, the other in Nov). Each logger will collect an hourly reading every
day for 12 months. Results from these will be assessed to determine how
many are replaced after the first year, and these in-situ recordings
will be compared to remotely collected sea surface temperature data
collected and modeled by NOAA.

~

\textbf{Data Analyses}

Broad spatial and temporal patterns in asset condition will be
investigated using univariate and multivariate analyses.~ Relationships
between the structure of the biological communities and physical
variables and human use will be assessed where possible. The
relationship between condition and pressure can be analysed using a
suite of metrics including regression and correlation techniques. The
proposed marine reserve is extensive and encompasses diverse habitats, a
range of pressures and varied conditions. Spatial variability must be
considered when analysing data from this project. Given the short time
frame of this project, it is unlikely that meaningful long-term
comparisons will be possible. In many cases sites have been selected
where historic sites existed, and attempts will be made to compare the
findings from this program to historic data. However, in many cases the
metrics measured or the techniques used may not be comparable, and other
parameters may limit interpretation of relationships with historic data.
~The data will be regularly tested for power and considered for trend
analysis when the strength of the data allows for confidence in the
results.~




\subsection*{Biometrician's Endorsement}

granted



\section*{Data management}


\subsection*{No. specimens}






\subsection*{Herbarium Curator's Endorsement}

not required




\subsection*{Animal Ethics Committee's Endorsement}

not required




\subsection*{Data management}

All raw data will be presented in a data report and archived on DPaW's
Marine Science Program server and on an archived hard drive.

~

\emph{Hard copies of any reports resulting from the project will be held
at the following locations:}

\begin{enumerate}
\tightlist
\item
  Woodside -- Pluto Program.
\item
  Marine Science Program, Science \& Conservation Division, Department
  of Parks and Wildlife, 17 Dick Perry Avenue, Western Australia, 6152.
  Ph: (08) 9334~0333.
\item
  Woodvale Library, Science \& Conservation Division, Department of
  Parks and Wildlife, Ocean Reef Road, Woodvale, Western Australia,
  6026. Ph: (08) 9405~5100 Fax: (08) 9306~1641.~
\item
  Archives, Woodvale Library, Science \& Conservation Division,
  Department of Parks and Wildlife, Ocean Reef Road, Woodvale, Western
  Australia, 6026. Ph: (08) 9405~5100 Fax: (08) 9306~1641 (CD also
  attached).
\item
  Department of Parks and Wildlife: Pilbara Region, Lot 3 Mardie Rd,
  Karratha, WA, 6714. Ph: (08) 9182 2000 Fax: (08) 9144 1118.
\end{enumerate}

\emph{Digital copies of any reports and data will be held at the
following:}

\begin{enumerate}
\tightlist
\item
  ~~~~ The Science Division Server:
\end{enumerate}

T:\textbackslash529-CALMscience\textbackslash Shared
Data\textbackslash Marine Science Program\textbackslash{}

\begin{enumerate}
\setcounter{enumi}{1}
\tightlist
\item
  ~~~~ Archived hard drive
\item
  ~~~~ Uploaded to the Marine Science Program data catalogue (CKAN)
\end{enumerate}




\section*{Budget}

\section*{Consolidated Funds }



\begin{longtabu} to \linewidth { |  X | X | X | X | }
\hline
\rowcolor{infobg}
Source & Year 1 & Year 2 & Year 3\\
\hline
\endhead



FTE Scientist &  &  & \\



FTE Technical &  &  & \\



Equipment &  &  & \\



Vehicle &  &  & \\



Travel &  &  & \\



Other &  &  & \\



Total &  &  & \\


\hline
\end{longtabu}



\section*{External Funds }



\begin{longtabu} to \linewidth { |  X | X | X | X | }
\hline
\rowcolor{infobg}
Source & Year 1 & Year 2 & Year 3\\
\hline
\endhead



Salaries, Wages, Overtime & 91,014 & 99,716 & 105,381\\



Overheads & 42,642 & 35,000 & 37,000\\



Equipment & 9,491 & 2,500 & 2,500\\



Vehicle & 17,003 & 25,100 & 15,600\\



Travel & 6,092 & 8,100 & 3,300\\



Other & 10,826 & 19,331 & 14,220\\



Total & 173,368 & 189,747 & 178,001\\


\hline
\end{longtabu}





%-----------------------------------------------------------------------------%
% Back matter
%\backmatter
\end{document}
%-----------------------------------------------------------------------------%

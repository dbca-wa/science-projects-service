
\documentclass[version=last,
    paper=a4, % paper size
    10pt, % default font size
    usenames,
    dvipsnames,
    oneside, % ONLINE
    headings=openany, % open chapters on odd and even pages
    %toc=chapterentrywithdots, % Table of Contents style
    %BCOR=7mm, % PRINT Binding Correction
    %DIV=13, % typearea 161.54 mm x 228.46 mm, top margin 22.85 mm, inner margin 16.15 mm
    %DIV=14, % 165.00 233.36 21.21 15.00
    DIV=15 % 168.00 237.60 19.80 14.00
]{scrbook}
\usepackage{typearea}
\usepackage[automark,headsepline,footsepline]{scrlayer-scrpage} % Headers and footers

%%
%% Fonts, encoding, spacing, indentation
%%
\usepackage{txfonts}
\renewcommand{\familydefault}{\sfdefault} % Default to Sans Serif font
\usepackage[english]{babel}
\usepackage[T1]{fontenc}
\usepackage[utf8]{inputenc}

% Paragraph spacing
%\usepackage{parskip}    % Paragraph spacing
%\setlength{\parindent}{0em} % Don't indent paragraphs - ONLINE
%\setlength{\parskip}{1.3 ex plus 0.5ex minus 0.3ex} % 1-1.8 ex vertical space between paragraphs - ONLINE

% Spacing of headings
%\RedeclareSectionCommand[afterskip=3pt]{section} % less space after section
%\RedeclareSectionCommand[beforeskip=0cm]{subsection} % less space between HRule and project name
%\RedeclareSectionCommand[afterskip=0.1\baselineskip]{subsubsection} % less space after progressreport subheadings

% Table font size
\usepackage{etoolbox}
\AtBeginEnvironment{longtabu}{\footnotesize}{}{}

%%
%% Tables, columns, layout
%%
\usepackage{multicol}   % 2 col publications
\usepackage{pdflscape}  % Landscape pages
\usepackage{pdfpages}   % Include PDFs
\usepackage{hanging}    % hanging paragraphs for publications
%\usepackage{titletoc}   % Required for manipulating the table of contents
\setcounter{tocdepth}{2} % TOC list down to section
\usepackage{enumerate}  % Enumerations
\usepackage{enumitem}   % Enumerations
\usepackage{longtable}  % Multipage table
\usepackage{tabu}       %
\setlength{\tabulinesep}{1.5mm} % Consistent vertical spacing in tabu

%%
%% Graphics, images, colours
%%
\usepackage{graphicx} % embedded images
\usepackage{eso-pic} %
\usepackage{colortbl} % define custom named colours
\definecolor{RedFire}{RGB}{146,25,28}
\definecolor{ParksWildlife}{RGB}{0,85,144}
\definecolor{successbg}{RGB}{223,240,216}
\definecolor{errorbg}{RGB}{242,222,222}
\definecolor{warningbg}{RGB}{252,248,227}
\definecolor{infobg}{RGB}{217,237,247}
\definecolor{muted}{RGB}{153,153,153}
\definecolor{success}{RGB}{70,136,71}
\definecolor{error}{RGB}{185,74,72}
\definecolor{warning}{RGB}{192,152,83}
\definecolor{info}{RGB}{58,135,173}

\definecolor{required}{RGB}{192,152,83}
\definecolor{requiredbg}{RGB}{252,248,227}
\definecolor{denied}{RGB}{185,74,72}
\definecolor{deniedbg}{RGB}{242,222,222}
\definecolor{granted}{RGB}{70,136,71}
\definecolor{grantedbg}{RGB}{223,240,216}
\definecolor{not reqiured}{RGB}{153,153,153}
\definecolor{not requiredbg}{RGB}{255,255,255}

\usepackage{tikz} % Drawing
\usetikzlibrary{arrows,shapes,positioning,shadows,trees}

%%
%% Links, URLs
%%
\usepackage[
    linktoc=all,
    %colorlinks=false,  %PRINT
    colorlinks=true, % ONLINE
    linkcolor=RedFire, % ONLINE
    urlcolor=ParksWildlife, % ONLINE
    pdftitle=Progress Report SP 2011-019 (FY 2015-2016)
]{hyperref}

% Black magic to linebreak URLs
\usepackage{url}
\makeatletter
\g@addto@macro{\UrlBreaks}{\UrlOrds}
\makeatother

%%
%% Custom macros
%%
% Thick Horizontal rule
\newcommand{\HRule}{\vspace{8mm}\\\noindent\rule{\linewidth}{0.1pt}}

% Custom Tikz node for SDS diagram
\newcommand\mynode[6][]{
    \node[#1] (#2){
        \parbox{#3\relax}{
            \begin{center}
            \textbf{#4}\\
            #5\\
            \footnotesize{#6}
            \end{center}}};}



%-----------------------------------------------------------------------------%
% Headers and Footers
\automark{section}
\ohead{\href{http://sdis.dpaw.wa.gov.au/documents/progressreport/1633/}{Progress Report SP 2011-019
}}
\chead{\href{http://sdis.dpaw.wa.gov.au}{SDIS}} % center header ONLINE
\ihead{\href{http://sdis.dpaw.wa.gov.au}{
        \includegraphics[scale=0.4]{/mnt/projects/sdis/staticfiles/img/logo-dpaw.png}}}
\ifoot{\textbf{Printed}~Mon, 11 Jul 2016 10:28:38 +0800} % inner/left footer
\cfoot{} % center footer
\ofoot{\pagemark} % outer/right footer
\pagestyle{scrheadings}
\setkomafont{pageheadfoot}{\normalfont}

%-----------------------------------------------------------------------------%
\begin{document}
\raggedbottom

%-----------------------------------------------------------------------------%
% Title page
\subject{Progress Report SP 2011-019
}
\title{Management of invertebrate pests in forests of south-west Western
Australia
}
\subtitle{Ecosystem Science
}
\author{}
\publishers{\small
    \subsection*{Project Core Team}
\begin{tabu} {X X}
\textbf{Supervising Scientist} & Janet Farr
\\
\textbf{Data Custodian} & 
\\
\textbf{Site Custodian} & 
\\
\end{tabu}


    \subsection*{Project status as of July 11, 2016, 10:28 a.m.}
\begin{tabu} {X X}
& Approved and active
\\
\end{tabu}

    
\subsection*{Document endorsements and approvals as of July 11, 2016, 10:28 a.m.}
\begin{tabu} {X X}

%\rowcolor{grantedbg}
    \textbf{Project Team} & 
    \textcolor{granted}{ granted}\\

%\rowcolor{grantedbg}
    \textbf{Program Leader} & 
    \textcolor{granted}{ granted}\\

%\rowcolor{grantedbg}
    \textbf{Directorate} & 
    \textcolor{granted}{ granted}\\

\end{tabu}



}
\uppertitleback{}
\lowertitleback{}
\date{}

%-----------------------------------------------------------------------------%
% Front matter
\frontmatter
\maketitle
%-----------------------------------------------------------------------------%
% Main matter
\mainmatter

\section*{Management of invertebrate pests in forests of south-west Western
Australia
}

J Farr, A Wills


\section*{Context}
Within the history of forest and natural landscape management in Western
Australia, many invertebrates are known to utilise forest biomass for
their survival and in doing so impart some form of damage to leaves,
shoot, roots, stems or branches. There are 10 recognised invertebrate
species with demonstrated significant impact on tree health, vitality
and timber quality within our natural environment. Currently the most
prevalent insect pests of concern in native forests are \emph{Perthida
glyphopa} (jarrah leafminer, JLM), \emph{Phoracantha acanthocera}
(bullseye borer, formerly known as \emph{Tryphocaria acanthocera} BEB)
and \emph{Uraba lugens} (gumleaf skeletoniser, GLS). Both JLM and GLS
have documented population outbreak periods, and BEB incidence appears
to be responsive to drought stress and is likely to increase. However,
Western Australian forests and woodlands also have a history of
developing unexpected insect outbreaks with dramatic consequences for
the ecosystem health and vitality. The decline in mean annual rainfall
in south-west Western Australia since the 1970s and global climate model
predictions of a warmer and drier environment mean conditions for
invertebrate pests will alter significantly in the next decade as our
environment shifts toward a new climatic regime. This project addresses
both recognised and emerging/potential invertebrate forest pests, and is
designed to augment forest health surveillance and management
requirements by providing knowledge on the biological aspects of forest
health threats from invertebrates in the south-west of Western
Australia.



\section*{Aims}
\begin{itemize}
\itemsep1pt\parskip0pt\parsep0pt
\item
  Investigate aspects of pest organism biology, host requirements,
  pathology and environmental conditions (including climatic conditions)
  that influence populations.
\item
  Determine distribution of the invertebrate pests, including outbreak
  boundaries and advancing outbreak fronts, using aerial mapping, remote
  sensing and road surveys.
\item
  Measure relative abundance of invertebrate pests, including
  quantitative population surveys and host/environmental impact studies
  where appropriate and/or possible.
\item
  Utilise appropriate monitoring technologies including GIS and remote
  sensing.
\item
  Liaise with land managers and the community regarding responses to
  pest insect outbreaks.
\end{itemize}



\section*{Progress}
\begin{itemize}
\itemsep1pt\parskip0pt\parsep0pt
\item
  Pheromone trapping of GLS was used to quantify the 2015/16 GLS
  population level. Populations capable of moderate to severe
  defoliation were present in areas around Donnelly Mill north west of
  Manjimup.
\item
  A paper investigating climate effects on GLS outbreaks~ was prepared
  and submitted to \emph{Austral Entomology} and is currently being
  revised after referee comments.
\item
  A preliminary study was initiated investigating effects of understorey
  removal by fire on pheromone trap catch. Catch data have been
  summarized and analyzed. Dense understorey interferes with pheromone
  trap effectiveness. Fire enhances trap catch in forest with a dense
  understorey by removing the shrub layer. Testing for a vegetation
  effect using a dryness index as a proxy for vegetation density showed
  negligible impact on the historical data from long term GLS population
  monitoring sites.
\end{itemize}



\section*{Management implications}
\begin{itemize}
\itemsep1pt\parskip0pt\parsep0pt
\item
  Integration of GLS population and impact data from two major outbreak
  events indicates a strong relationship between GLS outbreak and
  periods of below-normal rainfall at seasonal or longer timescales.
  Further outbreaks are likely given present declining trends in
  rainfall.
\item
  Pheromone trapping is effective in monitoring GLS populations and
  could be used routinely to identify the likelihood of GLS outbreak.
  Moth populations are a good predictor of subsequent larval populations
  and may provide early warning of outbreak events in the context of
  climate data.
\item
  Interference with pheromone based mate finding by vegetation has
  implications for moth species richness at landscape scales because
  pheromone based mate finding is common in moths (including other
  eucalypt defoliators).~ Understanding drivers of moth richness may
  allow spatial predictions of moth species richness and enhance
  understanding of effects of global change on moth biodiversity.
\item
  Vegetation density and structure as well as species composition and
  fire regimes in jarrah forest has implications for rate of spread and
  control of potentially invasive and exotic moth species which use
  pheromone based mate finding.
\end{itemize}



\section*{Future directions}
\begin{itemize}
\itemsep1pt\parskip0pt\parsep0pt
\item
  Continue pheromone trapping at long-term monitoring sites.
\item
  Prepare a manuscript describing fire and vegetation density effects on
  the efficacy of GLS pheromone trap catches.
\item
  Investigate vegetation species composition and fire effects at
  different trap heights using the GLS pheromone system.
\item
  Finalize edits and publish paper describing relationship between long
  term GLS population and climate cycles.
\end{itemize}



%-----------------------------------------------------------------------------%
% Back matter
%\backmatter
\end{document}
%-----------------------------------------------------------------------------%



\documentclass[version=last, paper=a4, DIV=18, usenames, dvipsnames]{scrartcl}
\usepackage{txfonts}
\usepackage{pdflscape}
\usepackage{pdfpages}
\usepackage[english]{babel} % English language/hyphenation
%%% Bootstrap colors
\definecolor{RedFire}{RGB}{146,25,28}
\definecolor{ParksWildlife}{RGB}{0,85,144}
\definecolor{successbg}{RGB}{223,240,216}
\definecolor{errorbg}{RGB}{242,222,222}
\definecolor{warningbg}{RGB}{252,248,227}
\definecolor{infobg}{RGB}{217,237,247}
\definecolor{muted}{RGB}{153,153,153}
\definecolor{success}{RGB}{70,136,71}
\definecolor{error}{RGB}{185,74,72}
\definecolor{warning}{RGB}{192,152,83}
\definecolor{info}{RGB}{58,135,173}

\definecolor{required}{HTML}{D9534F}
\definecolor{denied}{HTML}{D9534F}
\definecolor{granted}{HTML}{47A447}
\definecolor{not required}{RGB}{200, 200, 200}

\usepackage[colorlinks=true,pdftitle=doc\_progressreport.pdf
,linktoc=all,linkcolor=RedFire,urlcolor=ParksWildlife]{hyperref}
\usepackage{colortbl}
\usepackage{longtable}
\usepackage{tabu}
\setlength{\tabulinesep}{1.5mm}
\usepackage{enumerate}
\usepackage{enumitem}
\usepackage{fancyhdr}
\usepackage{lastpage}
\usepackage{graphicx}
\usepackage{eso-pic}
\usepackage{hyphenat}
\renewcommand{\familydefault}{\sfdefault}



\newcommand{\HRule}{\rule{\linewidth}{0.1pt}}

\newcommand{\placetextbox}[3]{% \placetextbox{<horizontal pos>}{<vertical pos>}{<stuff>}
  \setbox0=\hbox{#3}% Put <stuff> in a box
  \AddToShipoutPictureFG*{% Add <stuff> to current page foreground
    \put(\LenToUnit{#1\paperwidth},\LenToUnit{#2\paperheight}){\vtop{{\null}\makebox[0pt][c]{#3}}}%
  }%
}%




%-----------------------------------------------------------------------------%
% Headers and footers
%
\fancypagestyle{plain}{
  \fancyhf{}
  \setlength\headheight{60pt} % push page content below header
  \renewcommand{\headrulewidth}{0.1pt}
  \renewcommand{\footrulewidth}{0.1pt}
  
  
  \fancyhead[L]{ 
    \href{http://sdis.dpaw.wa.gov.au}{
    \includegraphics[scale=0.6]{/mnt/projects/sdis/staticfiles/img/logo-dpaw.png}}
  }
  \fancyhead[R]{ 
      \hfill
      \href{http://sdis.dpaw.wa.gov.au}{Science Directorate Information System} 
      \newline 
      \href{http://sdis.dpaw.wa.gov.au/documents/progressreport/1334/}{Progress Report 2012-34 (FY 2014-2015)} 
  }
  
  
  
  
  \fancyfoot[L]{ \leftmark\newline\textbf{Printed}\textit{ July 28, 2015, 4:36 p.m. }}
  \fancyfoot[R]{  \, \newline Page \thepage\ of \pageref{LastPage} }
  
  
}
\pagestyle{plain}
%
% end Headers
%-----------------------------------------------------------------------------%

\begin{document}

%-----------------------------------------------------------------------------%
% Title page
%

%
% end title page
%-----------------------------------------------------------------------------%




\section*{Context Summary}
Genetic analysis of threatened species can provide important information
to support and guide conservation management. In particular, genetic
tools can be used to aid resolution of the taxonomic identity of species
to determine whether they have appropriate conservation listing. At a
population level, analysis of the genetic diversity present in extant
populations provides information on genetic `health' of threatened
species and how this may be maintained or improved through management
actions, leading to long-term positive conservation outcomes.



\section*{Aims Summary}
\begin{itemize}
\itemsep1pt\parskip0pt\parsep0pt
\item
  Resolve taxonomic boundaries of Western Australian bandicoots
  (\emph{Isoodon} sp.), particularly \emph{I. auratus} and \emph{I.
  obesulus} and their subspecies, to determine appropriate conservation
  rankings.
\item
  In collaboration with Brian Chambers (UWA) investigate the role of
  fauna underpasses in providing connectivity between quenda (\emph{I.
  obesulus} ssp. \emph{fusciventer}) populations impacted by main road
  construction.
\item
  In collaboration with Mark Eldridge (Australian Museum), assess the
  genetic diversity and genetic structure of extant populations of
  black-flanked rock wallaby (\emph{Petrogale lateralis} ssp.
  \emph{lateralis}) to inform future conservation management, including
  translocations.
\item
  Use of DNA barcoding to confirm species identifications.
\end{itemize}



\section*{Progress}
\begin{itemize}
\itemsep1pt\parskip0pt\parsep0pt
\item
  DNA sequencing of \emph{I. obesulus}, \emph{I. auratus}, \emph{I.
  macrourus} and their subspecies using mitochondrial and nuclear
  markers for taxonomic analysis~has been undertaken. Preliminary
  analyses suggest further nuclear sequencing markers are needed to
  fully resolve species boundaries but that~revision of \emph{I.
  obesulus} sub-species designations is likely to be required.
\item
  DNA analysis of quenda populations has been completed and population
  genetic analyses are currently being undertaken. Population viability
  analyses have been completed exploring the long-term trajectory of
  urban populations with and without fauna underpasses, and with an
  additional range of threats (fire, urban expansion).
\item
  DNA analysis of rock wallaby populations has been completed and
  preliminary data analysis has investigated genetic diversity and
  structure of wheatbelt and northern WA populations.
\item
  DNA barcoding was used to identify a stranded whale carcass as Omura's
  whale, constituting a new record for the species in WA~
\end{itemize}



\section*{Management implications}
\begin{itemize}
\itemsep1pt\parskip0pt\parsep0pt
\item
  Resolution of taxonomic boundaries between \emph{I. obesulus} and
  \emph{I. auratus} and their broader relationships with eastern states
  bandicoots should enable revision of current threatened species status
  at state and commonwealth levels.
\item
  Genetic and population viability analysis showed that quenda
  populations in small, isolated patches of remnant vegetation in the
  urban matrix are vulnerable to genetic erosion, inbreeding and
  population decline, particularly when connectivity within (fauna
  underpasses) or between (increased urbanisation) habitat patches
  becomes inhibited. The impact of fauna underpasses on population
  persistence is somewhat context-specific, but extinction risks are
  predicted to increase in the study populations without fauna
  underpasses.
\item
  Genetic information on rock wallaby populations will enable a
  stocktake of the current status of nearly all extant populations,
  including the assessment of the effectiveness of past management
  interventions, and will contribute to planning of future conservation
  actions, including translocations. ~
\item
  The presence of the rare Omura's whale in Australian waters adds to
  our knowledge of the distribution of this species and has been updated
  on the state's fauna list.
\end{itemize}



\section*{Future directions}
\begin{itemize}
\itemsep1pt\parskip0pt\parsep0pt
\item
  Investigate potential of other nuclear markers for taxonomic analysis
  of bandicoots to further resolve species classifications.~Investigate
  use of coalescent models to infer the evolutionary history of the
  genus.
\item
  Complete analysis of population genetics and gene flow in urban quenda
  populations, including parentage assignment of individuals using fauna
  underpasses.
\item
  Comparison of historic and contemporary population genetics of rock
  wallaby wheatbelt populations to monitor genetic change and
  investigate impact of past management actions. Develop population
  viability analyses to predict future trajectory of threatened
  populations.
\end{itemize}




\clearpage



\end{document}

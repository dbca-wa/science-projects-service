
\documentclass[version=last,
    paper=a4, % paper size
    10pt, % default font size
    usenames,
    dvipsnames,
    oneside, % ONLINE
    headings=openany, % open chapters on odd and even pages
    %toc=chapterentrywithdots, % Table of Contents style
    %BCOR=7mm, % PRINT Binding Correction
    %DIV=13, % typearea 161.54 mm x 228.46 mm, top margin 22.85 mm, inner margin 16.15 mm
    %DIV=14, % 165.00 233.36 21.21 15.00
    DIV=15 % 168.00 237.60 19.80 14.00
]{scrbook}
\usepackage{typearea}
\usepackage[automark,headsepline,footsepline]{scrlayer-scrpage} % Headers and footers

%%
%% Fonts, encoding, spacing, indentation
%%
\usepackage{txfonts}
\renewcommand{\familydefault}{\sfdefault} % Default to Sans Serif font
\usepackage[english]{babel}
\usepackage[T1]{fontenc}
\usepackage[utf8]{inputenc}

% Paragraph spacing
%\usepackage{parskip}    % Paragraph spacing
%\setlength{\parindent}{0em} % Don't indent paragraphs - ONLINE
%\setlength{\parskip}{1.3 ex plus 0.5ex minus 0.3ex} % 1-1.8 ex vertical space between paragraphs - ONLINE

% Spacing of headings
%\RedeclareSectionCommand[afterskip=3pt]{section} % less space after section
%\RedeclareSectionCommand[beforeskip=0cm]{subsection} % less space between HRule and project name
%\RedeclareSectionCommand[afterskip=0.1\baselineskip]{subsubsection} % less space after progressreport subheadings

% Table font size
\usepackage{etoolbox}
\AtBeginEnvironment{longtabu}{\footnotesize}{}{}

%%
%% Tables, columns, layout
%%
\usepackage{multicol}   % 2 col publications
\usepackage{pdflscape}  % Landscape pages
\usepackage{pdfpages}   % Include PDFs
\usepackage{hanging}    % hanging paragraphs for publications
%\usepackage{titletoc}   % Required for manipulating the table of contents
\setcounter{tocdepth}{2} % TOC list down to section
\usepackage{enumerate}  % Enumerations
\usepackage{enumitem}   % Enumerations
\usepackage{longtable}  % Multipage table
\usepackage{tabu}       %
\setlength{\tabulinesep}{1.5mm} % Consistent vertical spacing in tabu

%%
%% Graphics, images, colours
%%
\usepackage{graphicx} % embedded images
\usepackage{eso-pic} %
\usepackage{colortbl} % define custom named colours
\definecolor{RedFire}{RGB}{146,25,28}
\definecolor{ParksWildlife}{RGB}{0,85,144}
\definecolor{successbg}{RGB}{223,240,216}
\definecolor{errorbg}{RGB}{242,222,222}
\definecolor{warningbg}{RGB}{252,248,227}
\definecolor{infobg}{RGB}{217,237,247}
\definecolor{muted}{RGB}{153,153,153}
\definecolor{success}{RGB}{70,136,71}
\definecolor{error}{RGB}{185,74,72}
\definecolor{warning}{RGB}{192,152,83}
\definecolor{info}{RGB}{58,135,173}

\definecolor{required}{RGB}{192,152,83}
\definecolor{requiredbg}{RGB}{252,248,227}
\definecolor{denied}{RGB}{185,74,72}
\definecolor{deniedbg}{RGB}{242,222,222}
\definecolor{granted}{RGB}{70,136,71}
\definecolor{grantedbg}{RGB}{223,240,216}
\definecolor{not reqiured}{RGB}{153,153,153}
\definecolor{not requiredbg}{RGB}{255,255,255}

\usepackage{tikz} % Drawing
\usetikzlibrary{arrows,shapes,positioning,shadows,trees}

%%
%% Links, URLs
%%
\usepackage[
    linktoc=all,
    %colorlinks=false,  %PRINT
    colorlinks=true, % ONLINE
    linkcolor=RedFire, % ONLINE
    urlcolor=ParksWildlife, % ONLINE
    pdftitle=Progress Report SP 2010-005 (FY 2015-2016)
]{hyperref}

% Black magic to linebreak URLs
\usepackage{url}
\makeatletter
\g@addto@macro{\UrlBreaks}{\UrlOrds}
\makeatother

%%
%% Custom macros
%%
% Thick Horizontal rule
\newcommand{\HRule}{\vspace{8mm}\\\noindent\rule{\linewidth}{0.1pt}}

% Custom Tikz node for SDS diagram
\newcommand\mynode[6][]{
    \node[#1] (#2){
        \parbox{#3\relax}{
            \begin{center}
            \textbf{#4}\\
            #5\\
            \footnotesize{#6}
            \end{center}}};}



%-----------------------------------------------------------------------------%
% Headers and Footers
\automark{section}
\ohead{\href{http://sdis.dpaw.wa.gov.au/documents/progressreport/1646/}{Progress Report SP 2010-005
}}
\chead{\href{http://sdis.dpaw.wa.gov.au}{SDIS}} % center header ONLINE
\ihead{\href{http://sdis.dpaw.wa.gov.au}{
        \includegraphics[scale=0.4]{/mnt/projects/sdis/staticfiles/img/logo-dpaw.png}}}
\ifoot{\textbf{Printed}~Mon, 4 Jul 2016 15:58:07 +0800} % inner/left footer
\cfoot{} % center footer
\ofoot{\pagemark} % outer/right footer
\pagestyle{scrheadings}
\setkomafont{pageheadfoot}{\normalfont}

%-----------------------------------------------------------------------------%
\begin{document}
\raggedbottom

%-----------------------------------------------------------------------------%
% Title page
\subject{Progress Report SP 2010-005
}
\title{Development of ethically acceptable techniques for invertebrate wet-pit
trapping
}
\subtitle{Biogeography
}
\author{}
\publishers{\small
    \subsection*{Project Core Team}
\begin{tabu} {X X}
\textbf{Supervising Scientist} & Mark Cowan
\\
\textbf{Data Custodian} & Mark Cowan
\\
\textbf{Site Custodian} & Mark Cowan
\\
\end{tabu}


    \subsection*{Project status as of July 4, 2016, 3:58 p.m.}
\begin{tabu} {X X}
& Approved and active
\\
\end{tabu}

    
\subsection*{Document endorsements and approvals as of July 4, 2016, 3:58 p.m.}
\begin{tabu} {X X}

%\rowcolor{grantedbg}
    \textbf{Project Team} & 
    \textcolor{granted}{ granted}\\

%\rowcolor{grantedbg}
    \textbf{Program Leader} & 
    \textcolor{granted}{ granted}\\

%\rowcolor{grantedbg}
    \textbf{Directorate} & 
    \textcolor{granted}{ granted}\\

\end{tabu}



}
\uppertitleback{}
\lowertitleback{}
\date{}

%-----------------------------------------------------------------------------%
% Front matter
\frontmatter
\maketitle
%-----------------------------------------------------------------------------%
% Main matter
\mainmatter

\section*{Development of ethically acceptable techniques for invertebrate wet-pit
trapping
}

M Cowan, N Guthrie


\section*{Context}
Over the past 15 years the technique of invertebrate wet-pit trapping
has become a standard practice in biological survey, biogeographic
research and condition monitoring programs. Relatively small aperture
pits with a preserving fluid are buried flush with the ground and left
\emph{in situ} for extended periods (several months) to sample
terrestrial invertebrates. This has enabled an unprecedented insight
into the temporal and spatial structuring of invertebrate communities -
a highly significant but comparatively poorly understood component of
the Western Australian biota.

However, a consequence of this sampling technique is the inadvertent
capture of vertebrates, which creates an ethical issue. The combination
of glycol and formalin used in these pits is likely to result in a
distressing demise for vertebrates as they are able to swim and stay
afloat in the solution for some time, and the chemical solution is
likely to act as an irritant. Also, the quality of the subsequently
preserved material is of limited use beyond initial species
identifications.



\section*{Aims}
\begin{itemize}
\itemsep1pt\parskip0pt\parsep0pt
\item
  Establish wet-pit trapping chemistry that ensures rapid death to both
  target and non-target fauna with the least distress possible.
\item
  Achieve a level of preservation in captured organisms suitable not
  only for species identification, but also for morphological and
  molecular taxonomic research.
\item
  Solution requires minimal personal protective equipment~for safe use
  and poses no environmental risk or hazard.
\item
  Solution needs to be stable for several weeks or more under variable
  climatic conditions.
\end{itemize}



\section*{Progress}
~

\begin{itemize}
\itemsep1pt\parskip0pt\parsep0pt
\item
  DNA extraction and PCR amplification~trials have been undertaken from
  a number of preserved specimens collected in the field with positive
  results.
\item
  Reports have been provided to the Parks and Wildlife Animal Ethics
  Committee.
\item
  Advice has been requested and provided to the Department's Animal
  Ethics Committee in relation to applications to use invertebrate wet
  pit traps.
\end{itemize}



\section*{Management implications}
\begin{itemize}
\item
  An alternative to the ethylene glycol and formalin solutions used in
  invertebrate wet pitfall trapping that reduce or address existing
  ethical issues around~vertebrate by-catch may allow for the
  continuation of this type of invertebrate sampling method. This is not
  only important for the Department of Parks and Wildlife's existing
  survey~and monitoring programs but for other government
  agencies,~tertiary institutions and environmental consultants that
  utilise this methodology for sampling ground invertebrates. Of
  particular relevance will be the increased value of preserved material
  for morphological and molecular examination, an area that is currently
  severely compromised by material collected using the conventional
  ethylene glycol and formalin~solutions.~~~~
\end{itemize}



\section*{Future directions}
\begin{itemize}
\itemsep1pt\parskip0pt\parsep0pt
\item
  Receive feedback and results from operational work using new chemistry
  in different parts of the state and under different climatic regimes.
\item
  Present findings to the~Department's Animal Ethics Committee.
\item
  Where the opportunity is available assess quality of molecular
  fixation for different target and non-target taxa from operational use
  of traps. ~
\item
  Continue provision of advice to the Department's Animal Ethics
  Committee on the use of invertebrate wet pits.
\item
  Produce a Science Information Sheet.
\item
  Completion of final integrated report.
\end{itemize}



%-----------------------------------------------------------------------------%
% Back matter
%\backmatter
\end{document}
%-----------------------------------------------------------------------------%


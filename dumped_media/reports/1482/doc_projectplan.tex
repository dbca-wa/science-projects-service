
\documentclass[version=last,
    paper=a4,                               % paper size
    10pt,                                   % default font size
    dvipsnames,
    % twoside,                                % PRINT Binding Correction
    oneside,                              % ONLINE
    headings=openany,                       % open chapters on odd and even pages
    open=any,
    BCOR=7mm,                               % PRINT Binding Correction
    %DIV=13,    % typearea 161.54mm x 228.46mm, top 22.85mm, inner 16.15mm
    %DIV=14,    % 165.00 233.36 21.21 15.00
    DIV=15,     % 168.00 237.60 19.80 14.00
    % toc=chapterentrywithdots              % Table of Contents style
]{scrbook}
\usepackage{typearea}


%------------------------------------------------------------------------------%
% Headers and footers
%------------------------------------------------------------------------------%
\usepackage[automark,headsepline,footsepline,plainfootsepline]{scrlayer-scrpage}
\automark*[section]{}
\addtokomafont{pageheadfoot}{\normalfont\footnotesize\sffamily} % Don't italicise
\renewcommand{\chaptermark}[1]{\markleft{#1}{}}     % Chapter: suppress numbering
\renewcommand{\sectionmark}[1]{\markright{#1}{}}    % Section: suppress numbering

% Header (inner, center, outer)
% \ihead{\href{http://sdis.dbca.wa.gov.au}{\textbf{Project Plan SP 2014-021}}}
%\chead{\href{http://sdis.dbca.wa.gov.au}{Science Directorate Information System}}
% \ohead{\href{https://www.dbca.wa.gov.au/science/10-biodiversity-and-conservation-science}{
% \includegraphics[height=8mm, keepaspectratio]{/usr/src/app/staticfiles/img/logo-dbca-bcs.jpg}}}

% Footer (inner, center, outer)
% \ifoot{\RaggedRight\leftmark}                       % Chapter
% \cfoot{\RaggedLeft\rightmark}                       % Section
% \ofoot[\bfseries\thepage]{\bfseries\thepage}        % Page number (also [plain])


%------------------------------------------------------------------------------%
% Fonts, encoding
%------------------------------------------------------------------------------%
%\usepackage{avant}             % Use the Avantgarde font for headings
\usepackage{txfonts}
\usepackage{mathptmx}
\usepackage{gensymb}            % provides \textdegree
\renewcommand{\familydefault}{\sfdefault} % Default to Sans Serif font
\usepackage{microtype}          % Slightly tweak font spacing for aesthetics
\usepackage[english]{babel}
\usepackage[utf8]{inputenc}  % Allow letters with accents
\usepackage[utf8]{luainputenc}  % Allow letters with accents
\usepackage[T1]{fontenc}        % Use 8-bit encoding that has 256 glyphs
\usepackage{textcomp}
\usepackage[explicit]{titlesec}           % Customise of titles
%\DeclareUnicodeCharacter{0080}{\textregistered}
\DeclareUnicodeCharacter{00B0}{\textdegree}

%------------------------------------------------------------------------------%
% Tables, columns, layout
%------------------------------------------------------------------------------%
\usepackage{etoolbox}
\AtBeginEnvironment{longtabu}{\footnotesize}{}{}  % Table font size
\usepackage{booktabs}           % Required for nicer horizontal rules in tables
\usepackage{multicol}           % 2 col publications
\usepackage{pdflscape}          % Landscape pages
\usepackage{pdfpages}           % Include PDFs
\usepackage{hanging}            % hanging paragraphs for publications
%\usepackage{titletoc}          % Manipulate the table of contents
\setcounter{tocdepth}{2}        % TOC list down to section
\usepackage{enumerate}          % Enumerations
\usepackage{enumitem}           % Enumerations
\usepackage{longtable}          % Multipage table
\usepackage{tabu}               %
\setlength{\tabulinesep}{1.5mm} % Consistent vertical spacing in tabu
\newcommand{\HRule}{\vspace{8mm}\noindent\rule{\linewidth}{0.1pt}}
\usepackage[export]{adjustbox}  % minipage, image frame


%------------------------------------------------------------------------------%
% Graphics, images, colours
%------------------------------------------------------------------------------%
\usepackage{graphicx} % embedded images
\usepackage{wrapfig}  % wrap figures in text
\usepackage{caption}  % allow unnumbered captions
\usepackage{eso-pic} % Required for specifying an image background in the title page
\usepackage{colortbl} % define custom named colours
\usepackage{xstring} % Conditionals
\usepackage{transparent} % Allow transparent images

\definecolor{RedFire}{RGB}{146,25,28}
% Following PICA branding guidelines
% https://dpaw.sharepoint.com/Divisions/pica/Documents/Branding%20guidelines.pdf
\definecolor{dpawblue}{RGB}{35,97,146}          % Pantone 647
\definecolor{dpaworange}{RGB}{237,139,0}        % Pantone 144
\definecolor{dpawgreen}{RGB}{116,170,80}        % Pantone 7489
\definecolor{dpawred}{RGB}{124,46,44}           % Paul's suggestion

% bootstrap colours
\definecolor{successbg}{RGB}{223,240,216}
\definecolor{errorbg}{RGB}{242,222,222}
\definecolor{warningbg}{RGB}{252,248,227}
\definecolor{infobg}{RGB}{217,237,247}
\definecolor{muted}{RGB}{153,153,153}
\definecolor{success}{RGB}{70,136,71}
\definecolor{error}{RGB}{185,74,72}
\definecolor{warning}{RGB}{192,152,83}
\definecolor{info}{RGB}{58,135,173}

% SDIS approval colours
\definecolor{required}{RGB}{192,152,83}
\definecolor{requiredbg}{RGB}{252,248,227}
\definecolor{denied}{RGB}{185,74,72}
\definecolor{deniedbg}{RGB}{242,222,222}
\definecolor{granted}{RGB}{70,136,71}
\definecolor{grantedbg}{RGB}{223,240,216}
\definecolor{notrequired}{RGB}{153,153,153}
\definecolor{notrequiredbg}{RGB}{255,255,255}

\usepackage{tikz} % Drawing
\usetikzlibrary{arrows,shapes,positioning,shadows,trees}


%------------------------------------------------------------------------------%
% Hyperlinks
%------------------------------------------------------------------------------%
\usepackage[open=true]{bookmark}
\usepackage{nameref}
\usepackage{ifxetex,ifluatex}
\ifxetex
  \usepackage[
    setpagesize=false,        % page size defined by xetex
    unicode=false,            % unicode breaks when used with xetex
    xetex]{hyperref}
\else
  \usepackage[unicode=true]{hyperref}
\fi

\hypersetup{
  backref=true,
  pagebackref=true,
  hyperindex=true,
  breaklinks=true,
  urlcolor=dpawblue,
  bookmarks=true,
  bookmarksopen=false,
  pdfauthor={Biodiversity and Conservation Science, Department of Biodiversity, Conservation and Attractions, WA},
  pdftitle=Project Plan SP 2014-021
,
  colorlinks=true,
  linkcolor=dpawblue,
  pdfborder={0 0 0}}

\urlstyle{same}                         % don't use monospace font for urlstyle


%------------------------------------------------------------------------------%
% Black magic to linebreak URLs
%------------------------------------------------------------------------------%
\usepackage{url}
\makeatletter\g@addto@macro{\UrlBreaks}{\UrlOrds}\makeatother
\Urlmuskip=0mu plus 1mu


%------------------------------------------------------------------------------%
% Fix latex errors
%------------------------------------------------------------------------------%
\providecommand{\tightlist}{\setlength{\itemsep}{0pt}\setlength{\parskip}{0pt}}

% copy-pasted HTML <span> in SDIS fields becomes \text{} in tex source
\providecommand{\text}{}


%------------------------------------------------------------------------------%
% Custom Tikz node for SDS diagram
%------------------------------------------------------------------------------%
\newcommand\mynode[6][]{
  \node[#1] (#2){
    \parbox{#3\relax}{
      \begin{center}
      \textbf{#4}\\
      #5\\
      \footnotesize{#6}
      \end{center}
    }};}


%------------------------------------------------------------------------------%
% Custom no-pagebreaks-environment
%------------------------------------------------------------------------------%
\newenvironment{absolutelynopagebreak}
  {\par\nobreak\vfil\penalty0\vfilneg\vtop\bgroup}
  {\par\xdef\tpd{\the\prevdepth}\egroup\prevdepth=\tpd}


%------------------------------------------------------------------------------%
% Remove the header from odd empty pages at the end of chapters
%------------------------------------------------------------------------------%
\makeatletter
\renewcommand{\cleardoublepage}{
\clearpage\ifodd\c@page\else
\hbox{}
\vspace*{\fill}
\thispagestyle{empty}
\newpage
\fi}


%----------------------------------------------------------------------------------------
%  Page flow control
%----------------------------------------------------------------------------------------
%\widowpenalty=10000
%\clubpenalty=10000
%\vbadness=1200
%\hbadness=11000


%----------------------------------------------------------------------------------------
%   CHAPTER HEADINGS
%----------------------------------------------------------------------------------------
\newcommand{\thechapterimage}{}
\newcommand{\chapterimage}[1]{\renewcommand{\thechapterimage}{#1}}

% Numbered chapters with mini tableofcontents
\def\thechapter{\arabic{chapter}}
\def\@makechapterhead#1{
%\thispagestyle{plain}
{\centering \normalfont\sffamily
\ifnum \c@secnumdepth >\m@ne
\if@mainmatter
\startcontents
\begin{tikzpicture}[remember picture,overlay]
\node at (current page.north west)
{\begin{tikzpicture}[remember picture,overlay]
\node[anchor=north west,inner sep=0pt] at (0,0) {
\includegraphics[width=\paperwidth,height=0.5\paperwidth]{\thechapterimage}};
%------------------------------------------------------------------------------%
% Small contents box in the chapter heading
% Mini TOC background box
%\fill[color=dpawblue!10!white,opacity=.2] (1cm,0) rectangle (
%  3.5cm, % Mini TOC box width
%  -3.5cm % Mini TOC box height
%);
% Mini TOC text content
%\node[anchor=north west] at (1.1cm,.35cm) {
%  \parbox[t][8cm][t]{6.5cm}{
%    \huge\bfseries\flushleft
%    \printcontents{l}{1}{
%    \setcounter{tocdepth}{1}                   % Mini TOC level depth
%    }
% }
%};
%------------------------------------------------------------------------------%
% Chapter heading
\draw[anchor=west] (5cm,-9cm) node [
rounded corners=20pt,
fill=dpawblue!10!white,
text opacity=1,
draw=dpawblue,
draw opacity=1,
line width=1.5pt,
fill opacity=.2,
inner sep=12pt]{
    \huge\sffamily\bfseries\textcolor{black}{
      \thechapter. #1\strut\makebox[22cm]{}
    }
};
\end{tikzpicture}};
\end{tikzpicture}}
\par\vspace*{240\p@}                            % Push text below chapter image
\fi
\fi}

%------------------------------------------------------------------------------%
% Unnumbered chapters without mini tableofcontents
%------------------------------------------------------------------------------%
\def\@makeschapterhead#1{
%\thispagestyle{plain}
{\centering \normalfont\sffamily
\ifnum \c@secnumdepth >\m@ne
\if@mainmatter
\begin{tikzpicture}[remember picture,overlay]
\node at (current page.north west)
{\begin{tikzpicture}[remember picture,overlay]
\node[anchor=north west,inner sep=0pt] at (0,0) {
  \includegraphics[width=\paperwidth,height=0.5\paperwidth]{\thechapterimage}};
% Mini TOC background box
%\fill[color=dpawblue!10!white,opacity=.2] (1cm,0) rectangle (
%  3.5cm,                                       % Mini TOC box width
%  -3.5cm                                       % Mini TOC box height
%);
% Mini TOC text content
%\node[anchor=north west] at (1.1cm,.35cm) {
%  \parbox[t][8cm][t]{6.5cm}{
%    \huge\bfseries\flushleft
%    \printcontents{l}{1}{
%    \setcounter{tocdepth}{1} % Mini TOC level depth
%    }
%}
%};
\draw[anchor=west] (5cm,-9cm) node [rounded corners=20pt,
  fill=dpawblue!10!white,fill opacity=.6,inner sep=12pt,text opacity=1,
  draw=dpawblue,draw opacity=1,line width=1.5pt]{
  \huge\sffamily\bfseries\textcolor{black}{#1\strut\makebox[22cm]{}}};
\end{tikzpicture}};
\end{tikzpicture}}
\par\vspace*{240\p@}
\fi
\fi
}
\makeatother



\usepackage[automark,headsepline,footsepline,plainfootsepline]{scrlayer-scrpage}
\automark*[section]{}
\addtokomafont{pageheadfoot}{\normalfont\footnotesize\sffamily} % Don't italicise
\renewcommand{\chaptermark}[1]{\markleft{#1}{}}     % Chapter: suppress numbering
\renewcommand{\sectionmark}[1]{\markright{#1}{}}    % Section: suppress numbering

% Header (inner, center, outer)
\ihead{\href{http://sdis.dbca.wa.gov.au/documents/projectplan/1482/}{Project Plan SP 2014-021}}
%\chead{\href{http://sdis.dbca.wa.gov.au}{Science Directorate Information System}}
\ohead{\href{https://www.dbca.wa.gov.au/science/10-biodiversity-and-conservation-science}{
\includegraphics[height=6mm, keepaspectratio]{/usr/src/app/staticfiles/img/logo-dbca-bcs.jpg}}}
% Footer (inner, center, outer)
\ifoot{\textbf{Printed}~Fri, 6 Dec 2019 14:17:32 +0800} % inner/left footer
\cfoot{}
\ofoot[\bfseries\thepage]{\bfseries\thepage}        % Page number (also [plain])


\pagestyle{scrheadings}
\setkomafont{pageheadfoot}{\normalfont}

%-----------------------------------------------------------------------------%
\begin{document}
\raggedbottom

%-----------------------------------------------------------------------------%
% Title page
\subject{Project Plan SP 2014-021
}
\title{Habitat use, distribution and abundance of coastal dolphin species in
the Pilbara
}
\subtitle{Marine Science
}
\author{}
\publishers{\small
    \subsection*{Project Core Team}
\begin{tabu} {X X}
\textbf{Supervising Scientist} & Holly Raudino
\\
\textbf{Data Custodian} & Holly Raudino
\\
\textbf{Site Custodian} & Holly Raudino
\\
\end{tabu}


    \subsection*{Project status as of Dec. 6, 2019, 2:17 p.m.}
\begin{tabu} {X X}
& Approved and active
\\
\end{tabu}

    
\subsection*{Document endorsements and approvals as of Dec. 6, 2019, 2:17 p.m.}
\begin{tabu} {X X}

%\rowcolor{grantedbg}
    \textbf{Project Team} & 
    \textcolor{granted}{ granted}\\

%\rowcolor{grantedbg}
    \textbf{Program Leader} & 
    \textcolor{granted}{ granted}\\

%\rowcolor{grantedbg}
    \textbf{Directorate} & 
    \textcolor{granted}{ granted}\\

%\rowcolor{grantedbg}
    \textbf{Biometrician} & 
    \textcolor{granted}{ granted}\\

%\rowcolor{not requiredbg}
    \textbf{Herbarium Curator} & 
    \textcolor{not required}{ not required}\\

%\rowcolor{not requiredbg}
    \textbf{Animal Ethics Committee} & 
    \textcolor{not required}{ not required}\\

\end{tabu}



}
\uppertitleback{}
\lowertitleback{}
\date{}

%-----------------------------------------------------------------------------%
% Front matter
\frontmatter
\maketitle
%-----------------------------------------------------------------------------%
% Main matter
\mainmatter



\section*{Habitat use, distribution and abundance of coastal dolphin species in
the Pilbara
}



\subsection*{Biodiversity and Conservation Science Program}

Marine Science




\subsection*{Departmental Service}

Service 6: Conserving Habitats, Species and Communities


\subsection*{Project Staff}
\begin{tabu} {X X X}
%\rowcolor{infobg}
\textbf{Role} & \textbf{Person} & \textbf{Time allocation (FTE)}\\

Supervising Scientist & Kelly Waples & 0.25\\

Research Scientist & Holly Raudino & 1.0\\

Technical Officer & Corrine Douglas & 0.25\\

Technical Officer & Ryan Douglas & 0.25\\

\end{tabu}




\subsection*{Related Science Projects}

None


\subsection*{Proposed period of the project}
July 1, 2014 -- June 30, 2018



\section*{Relevance and Outcomes}


\subsection*{Background}

Although little is known about population size, distribution and
residency patterns, it is well accepted that Australian snubfin
(\emph{Orcaella} \emph{heinsohni}) and Australian humpback dolphin
(\emph{Sousa sahulensis}) inhabit Australia's tropical north-western
coastal waters (Allen\emph{, et al.} 2012). Indo-Pacific humpback
dolphins occur across Australia's entire northwest coast including
resident populations at Ningaloo Marine Park and most likely the Dampier
Archipelago as well as further north into the Kimberley.~ The snubfin
dolphin is endemic to northern Australia with identified resident
populations in the Kimberley, Northern Territory and Queensland
(Brown\emph{, et al.} 2014; Brown\emph{, et al.} 2014). While this
species has been sighted occasionally in the Pilbara, their presence and
use of this area is yet to be determined, however the Pilbara is likely
to represent the southern extreme of their range (Allen\emph{, et al.}
2012).

~

Limited surveys have been conducted targeting coastal dolphins in the
Pilbara; exceptions include a dedicated study of humpback dolphins in
Ningaloo Marine Park and Exmouth Gulf (Brown\emph{, et al.} 2012) and
opportunistic surveys and anecdotal sightings throughout the region
(Allen\emph{, et al.} 2012). Aerial surveys that were targeting humpback
whales sighted dolphins but were unable to differentiate between species
due to the high altitude flown (1000 ft) (Jenner \& Jenner 2004; Jenner
\& Jenner 2010). Although the presence of several coastal dolphin
species is expected in nearshore waters (humpback, snubfin and
bottlenose dolphins) (Hanf 2014) the residency, degree of use and
habitat characteristics of these species are unknown in the Pilbara.

~

Human pressures and impacts on these species are increasing, in
particular in the Pilbara through activities associated with the rapid
expansion of resources sector, including oil and gas exploration and
production, coastal infrastructure development and shipping. This is
often a key factor that proponents are required to address to secure
environmental approvals at the State and Commonwealth levels. However,
as noted above, the knowledge base on these species across their range
is very poor. In addition, there are no agreed best practice protocols
or standards for survey design and data collection on these species that
allow for comparison to be made between studies and study sites. ~A
better understanding of these species and their use of tPilbara coastal
waters is needed to provide good temporal and regional context for
assessing and managing impacts and to reduce uncertainty in the
approvals process. As such, the draft \emph{Strategic Research
Priorities for Marine Mammal Conservation and Management in Western
Australia 2014} recognised both snubfin and humpback dolphins as high
priority species for fundamental research.




\subsection*{Aims}

This research is being conducted to develop a baseline understanding of
key aspects of the ecology of dolphins in coastal Pilbara waters. The
specific aims are to:

\begin{itemize}
\tightlist
\item
  ~~~~~~~~ Determine habitat use, distribution, abundance, residency,
  and movement patterns of dolphins in coastal Pilbara waters;
\item
  ~~~~~~~~ Identify the characteristics of habitats used by coastal
  dolphins, such as water depth, benthic substrate, timing and seasonal
  variation; and
\item
  ~~~~~~~~ Determine the trophic niche of dolphins in the Pilbara by
  identifying ~the prey species they consume.
\end{itemize}




\subsection*{Expected outcome}

This research will enable a better understanding of coastal dolphin
species at a regional and national scale including distribution,
abundance, habitat use, movement and connectivity. The main outcomes and
benefits will be:

\begin{itemize}
\tightlist
\item
  ~~~~~~~~ Distribution and abundance including high density areas and
  spatial and temporal patterns of coastal dolphins will be identified
  and mapped across the Pilbara to allow managers to assess conflicts
  with potential pressures;
\item
  ~~~~~~~~ Key habitat will be identified which can be used to assess
  potential overlap with pressures such as habitat loss from coastal
  development and displacement from industrial development leading to
  better informed decision making during Environmental Impact Assessment
  processes;
\item
  ~~~~~~~~ Populations will be defined for coastal dolphin species
  (humpback, bottlenose and snubfin, where applicable) which will allow
  managers to assess the relative conservation significance of different
  populations or species in relation to pressures or factors like
  restricted distributions;
\item
  ~~~~~~~~ Baseline data will inform ongoing regional monitoring and
  management and for comparison with other regions;
\item
  ~~~~~~~~ A state-wide database will be implemented modelled on the
  Northern Territory database `DolFIN' to archive and manage survey and
  photo-Identification data which will improve information management,
  compatibility and information sharing between jurisdictions
\item
  ~~~~~~~~ Data on population abundance and distribution of humpback and
  snubfin dolphins in the Pilbara will allow a more comprehensive
  assessment of their conservation status at a State and National level.
\end{itemize}




\subsection*{Knowledge transfer}

At a state level the Department of Parks and Wildlife and the Office of
Environmental Protection Authority (OEPA) will be the main user of this
knowledge. ~The Environmental Management Branch within Parks and
Wildlife and the OEPA currently have limited information on which to
assess Environmental Impact Assessments (EIA) on these dolphin species
in the Pilbara. This information will be used in EIA processes to
evaluate and mitigate potential impacts on coastal dolphins by informing
the appropriateness of proposed developments in relation to critical
habitat identified through this project. The presence of coastal dolphin
species and the vulnerability of these populations to coastal
development will be more apparent from this study.~ Industry (Port
Authorities, Oil and Gas companies) will be interested in methodology,
e.g. protocols and standard operating procedures for dolphin surveys and
the resulting baseline data to inform future proposals and EIA related
documents.

~

In addition, the findings and information ~will be used by Parks and
Wildlife Marine Science Program to manage marine mammals in the Pilbara
and across the state using this baseline information and protocols to
establish long term monitoring. Data will be relevant to marine
conservation planning through marine protected area planning and through
conservation status assessment. ~

~

When announcing the 2011 Finalised Priority Assessment List of
Threatened Species under the EPBC Act, the Federal Minister for
Sustainability, Environment, Water, Population and Communities, Mr Burke
said he had received advice that there was currently not enough
information to properly assess the snubfin dolphin however he was
``hopeful that we will be able to collect enough information to do a
proper assessment before long''. This project will gather information on
the population status of the snubfin dolphin in the Pilbara region, WA
and this information can be used towards the overall assessment of the
conservation status of this species.




\subsection*{Tasks and Milestones}

~

\begin{longtable}[]{@{}ll@{}}
\toprule
\endhead
\begin{minipage}[t]{0.47\columnwidth}\raggedright
\textbf{Milestone}\strut
\end{minipage} & \begin{minipage}[t]{0.47\columnwidth}\raggedright
\textbf{Completion Time}\strut
\end{minipage}\tabularnewline
\begin{minipage}[t]{0.47\columnwidth}\raggedright
\emph{Project Planning}

~~~ Submission of Science Concept Plan

~~~ Submission of Science Project Plan\strut
\end{minipage} & \begin{minipage}[t]{0.47\columnwidth}\raggedright
~

Jul 2014

Aug 2014\strut
\end{minipage}\tabularnewline
\begin{minipage}[t]{0.47\columnwidth}\raggedright
\emph{Field Work}

Pilot field trip to test survey and sampling design (Southern Pilbara)

Boat surveys in southern Pilbara sampling dolphins

~

Pilot field trip to test survey and sampling design (Northern Pilbara)

Boat surveys in northern Pilbara sampling dolphins

~

Aerial surveys in southern and northern Pilbara

~\strut
\end{minipage} & \begin{minipage}[t]{0.47\columnwidth}\raggedright
~

Oct 2014

Mar, Jun/July, Oct 2015

Mar, Jun/Jul, Oct 2016

~

Apr and Nov 2015

Apr and Nov 2016

~\strut
\end{minipage}\tabularnewline
\begin{minipage}[t]{0.47\columnwidth}\raggedright
\emph{Data processing and Statistical Analysis}

Pilot run of habitat model

Photo-identification and distribution analysis

Occupancy and habitat modeling analysis\strut
\end{minipage} & \begin{minipage}[t]{0.47\columnwidth}\raggedright
~

Nov 2014

Dec of each year 2014-1016\strut
\end{minipage}\tabularnewline
\begin{minipage}[t]{0.47\columnwidth}\raggedright
\emph{Reports (outputs)}

Metadata report for each field trip (summary statistics of sample sizes)

~~~~~~~~~~~~~~~~~~~~~~~~~~

~ Progress reports

~ Final reports\strut
\end{minipage} & \begin{minipage}[t]{0.47\columnwidth}\raggedright
~

Within 3 months of trip completion

Annually Sep.

Jun 2017\strut
\end{minipage}\tabularnewline
\bottomrule
\end{longtable}

~




\subsection*{References}

\textbf{Allen, S. J., Cagnazzi, D. D., Hodgson, A. J., Loneragan, N. R.
\& Bejder, L.} (2012) Tropical inshore dolphins of north-western
Australia: Unknown populations in a rapidly changing region.
\emph{Pacific Conservation Biology} 18: 56-63

\textbf{Bacher, K., Allen, S., Lindholm, A. K., Bejder, L. \& Krutzen,
M.} (2010) Genes or Culture: Are Mitochondrial Genes Associated with
Tool Use in Bottlenose Dolphins (\emph{Tursiops} sp.)? \emph{Behavioural
genetics} 40: 706-714

\textbf{Bilgmann, K., Griffiths, O. J., Allen, S. J. \& Moller, L. M.}
(2006) A biopsy pole system for bow-riding dolphins: sampling success,
behavioral responses, and test for sampling bias. \emph{Marine Mammal
Science} 23: 218--225

\textbf{Brown, A., Bejder, L., Cagnazzi, D., Parra, G. \& Allen, S.}
(2012) The North West Cape, Western Australia: A Potential Hotspot for
Indo-Pacific Humpback Dolphins \emph{Sousa chinensis}? \emph{Pacific
Conservation Biology} 18: 240-246

\textbf{Brown, A. M., Bejder, L., Pollock, K. H. \& Allen, S. J.} (2014)
Abundance of coastal dolphins in Roebuck Bay, Western Australia
\emph{Report to WWF-Australia,} 25

\textbf{Brown, A. M., Kopps, A. M., Allen, S. J., Bejder, L.,
Littleford-Colquhoun, B., Parra, G. J., Cagnazzi, D., Thiele, D.,
Palmer, C. \& Frère, C. H.} (2014) Population Differentiation and
Hybridisation of Australian Snubfin (\emph{Orcaella heinsohni}) and
Indo-Pacific Humpback (\emph{Sousa chinensis}) Dolphins in North-Western
Australia. \emph{PLoS ONE} 9: e101427

\textbf{Fernandez, R., Garcıa-Tiscar, S., Santos, M. B., Lopez, A.,
Martınez-Cedeira, J. A., Newton, J. \& Pierce, G. J.} (2011) Stable
isotope analysis in two sympatric populations of bottlenose dolphins
Tursiops truncatus: evidence of resource partitioning? \emph{Marine
Biology} 158: 1043--1055

\textbf{Frère, C. H., Krzyszczyk, E., Patterson, E. M., Hunter, S. \&
Ginsburg, A.} (2010) Thar she blows! A novel method for DNA collection
from cetacean blow. \emph{Plosone} 5: e12299

\textbf{Gibbs, S. E., Harcourt, R. G. \& Kemper, C. M.} (2011) Niche
differentiation of bottlenose dolphin species in South Australia
revealed by stable isotopes and stomach contents. \emph{Wildlife
Research} 34: 261-270

\textbf{Groom, C. J. \& Coughran, D. K.} (2012) Three decades of
cetacean strandings in Western Australia: 1981 to 2010. \emph{Journal of
the Royal Society of Western Australia} 95: 63-76

\textbf{Jenner, K. C. \& Jenner, M. N.} (2010) A Description of
Megafauna Distribution and Abundance in the SW Pilbara Using Aerial and
Acoustic Surveys -- Final Report \emph{Centre for Whale Research} 54

\textbf{Krützen, M., Barré, L. M., Möller, L. M., Heithaus, M. R.,
Simms, C. \& Sherwin, W. B.} (2002) A biopsy system for small cetaceans:
darting success and wound healing in \emph{Tursiops} spp. \emph{Marine
Mammal Science} 18: 863-878

\textbf{Nicholson, K., Bejder, L., Allen, S. J., Kruetzen, M. \&
Pollock, K. H.} (2012) Abundance, survival and temporary emigration of
bottlenose dolphins (\emph{Tursiops} sp.) off Useless Loop in the
western gulf of Shark Bay, Western Australia. \emph{Marine and
Freshwater Research} 63: 1059-1068

\textbf{Owen, K., Charlton-Robb, K. \& Thompson, R.} (2011) Resolving
the Trophic Relations of Cryptic Species: An Example Using Stable
Isotope Analysis of Dolphin Teeth. \emph{PLoS ONE} 6: 1-10

\textbf{Smith, H. C., Pollock, K., Waples, K., Bradley, S. \& Bejder,
L.} (2013) Use of the Robust Design to Estimate Seasonal Abundance and
Demographic Parameters of a Coastal Bottlenose Dolphin (\emph{Tursiops
aduncus}) Population. \emph{PLoS ONE} 8: e76574

\textbf{Tyne, J. A., Pollock, K. H., Johnston, D. W. \& Bejder, L.}
(2014) Abundance and Survival Rates of the Hawai'i Island Associated
Spinner Dolphin (Stenella longirostris) Stock. \emph{PLoS ONE} 9: e86132

~



\section*{Study design}


\subsection*{Methodology}

\textbf{Field sampling}

A combination of distance sampling techniques (through aerial and boat
surveys) and mark-recapture (through boat surveys) will be used to
investigate dolphin density, distribution and abundance.

~

\emph{Boat surveys}

Sampling will take place three times a year for a two week period,
subject to weather (March, June/July, October in 2014-2016). Surveys
will be conducted along pre-determined line transects from a small 5m
centre console RIB (or equivalent) research vessel driven at a speed of
8 to 12 kn with two observers and one boat driver present during each
survey. Line transects will be designed to run perpendicular to the
mainland coastline where possible to account for the water depth
gradient and distance from coast.

~

\emph{~}

\emph{Transect design}

The study area will be divided into multiple zones for the purpose of
the boat based surveys. The initial four zones have been selected based
on known densities of dolphins and/or suspected presence based on
habitat. Other potential zones include Exmouth Gulf and Cape Preston.

Zone 1- transects will extend offshore a maximum of 3nm from the
mainland coastline and extend alongshore 20 kilometres north and south
of Onslow township.

Zone 2- transects will be centred around Thevenard Island and radiate
from the island to 2 nautical miles and between Thevenard Island and the
mainland (\textasciitilde12km).

Zone 3- transects will centre around Dampier Port and the Dampier
archipelago within state waters (3nm)

Zone 4- transects will centre around Balla Balla and extend alongshore
extending offshore to state waters (3nm)

A sampling design of transects will be prepared for each zone that will
allow for that zone to be surveyed by boat within a single day (weather
permitting) to avoid re-sighting dolphins as they move through the study
area.The overall survey route will be divided into smaller transects of
at least 10 kilometres in length. The 10 kilometre transects will be
completed in a random order and repeated at least 6 times over two
years.

\emph{~}

\emph{~}

\emph{~}

\emph{Sighting data}

When a dolphin is sighted the following data will be recorded; species,
distance and angle between the vessel and the dolphin, and dolphin group
size. The position of the vessel on the transect will also be recorded
before the vessel departs the transect line. Once this initial
information is recorded, the vessel will approach the dolphin group to
within 50 m to capture additional data such as photo-identification of
individual dolphins, age and group composition and predominant behaviour
(feeding, foraging, travelling, resting or socialising). ~When foraging
or feeding is recorded as the predominant behaviour, the species of prey
wil be recorded whenever possible.~

~

The duration of sightings of dolphin groups will be short where possible
(5 minutes minimum to determine predominant behaviour and photograph all
individuals) to maximise the search effort along the transect line and
ensure completion of each transect during suitable weather (Beaufort
scale \textless3). At the completion of each sighting the transect line
will be re-joined where it was departed (marked by the GPS waypoint).

~

\emph{Aerial surveys}

The study area will be divided into two zones for the purpose of aerial
surveys

Zone 1 -- will be the area surveyed and demarked in the dugong research
plan extending south of Onslow and offshore to Barrow Island.

Zone 2- will be the area north of the area already surveyed north of
Onslow, Barrow Island and encompassing the Dampier Archipelago.

~

Aerial surveys will be conducted by a fixed high-wing aircraft flying a
pre-determined survey route of line transects. The aircraft will follow
transect lines at a low altitude (500 ft) and a speed of 80km/hr. There
will be two dedicated observers; one on each side of the aircraft and a
survey leader collating the data. The two observers and the survey
leader will communicate via aviation headsets and the survey leader will
record the data being called by the observers. For each dolphin group
sighted the species, number of individuals visible, position in the
transect and number of calves will be recorded. For the purpose of
species identification and confirming group size, the group will be
circled, counted and then the transect line resumed.

\emph{Tissue Sample Collection -- connectivity and trophic niche
assessment}

Tissue samples will be collected using remote biopsy techniques; either
a modified air rifle with darts (Krützen\emph{, et al.} 2002) or a pole
puncture system for bow-riding individuals (Bilgmann\emph{, et al.}
2006). These samples will be collected during vessel based surveys,
though these surveys will be conducted separate to the line transect
surveys described above. Approximately 20 samples per sample site across
the Pilbara region will be the target sample size (maximum total samples
approximately 100). Approximately 20 samples already exist from the
North West Cape and Dampier Archipelago for humpback dolphins Tissue
samples will be sub-sampled for the purpose of stable isotope and
genetic analysis and stored appropriately following Western Australian
Museum protocols. External researchers will be engaged to collect
samples and the laboratory analysis will be outsourced for this aspect
of the project.

~

~

\textbf{Analysis}

\emph{~}

\emph{Photo identification - processing and archiving}

Photo identification images will be collected in the field. A protocol
will be followed of collecting at least one image for both sides of the
dorsal fin for each individual dolphin regardless of degree of marking
or familiarity. The images will then be graded using standard protocols.
The images will be graded in relation to: 1) quality (angle, focus and
the ratio of fin to frame) and 2) distinctiveness (how marked the
individual is from not marked to severely marked). Survey data will be
entered and managed in a relational database (DolFIN) with survey images
matched to the photo-identification catalogue and linked to the relevant
survey information. This database will be used to generate the capture
histories and distribution data that will then be analysed statistically
using software such as MARK, DISTANCE and Arc GIS.

\emph{~}

\emph{Distribution and abundance estimation}

Mark-recapture methods will be applied to the photo-identification data
that will be stored and processed in the DolFIN database. Capture
histories will be generated for individual dolphins and abundance will
be modeled using either POPAN or the Robust Design in program MARK
23/07/2014(Nicholson\emph{, et al.} 2012; Smith\emph{, et al.} 2013;
Tyne\emph{, et al.} 2014). Parameters such as apparent survival and
temporary emigration will be estimated. Abundance and distribution may
also be calculated using the program DISTANCE for comparative purposes
if time permits.

~

\emph{Habitat modelling}

Dolphin sightings from two platforms (aerial and boat) will be analysed
in ArcGIS and incorporated into a broadscale habitat model per species
with variables such as distance from coast, benthic habitat, SST, water
depth, slope and behaviour to investigate which environmental variables
best explain dolphin distribution across the Pilbara study area. This
analysis will build on preliminary analysis by Hanf 2014 (Murdoch
University masters project) that modelled dolphin sighting data
collected during 2012 dugong aerial surveys for Chevron using MAXENT
modelling (presence-only) software. A pilot project of incorporating
Department of Parks and Wildlife benthic habitat data will be
incorporated into this model in October 2014 to see if benthic habitat
type is an important predictor in dolphin distribution. Boat sighting
data will be incorporated into the habitat model to improve the species
representation (this may be biased due to dugongs being the focal
species of the aerial surveys) and to validate the habitat model.
Potentially other habitat models may be used including Generalised
Linear Models if the presence-only models in MAXENT are deemed to be of
limited value in identifying habitat characteristics important to these
dolphin species in this area and if the model has limited overall
predictive ability of dolphin distribution across the region.

~

\emph{Connectivity}

Genetic analyses will follow methodology by Brown \emph{et al.} 2014 and
will likely include mitochondrial DNA (mtDNA) and microsatellite loci
with PCR conditions as described in Bacher \emph{et al.} 2010 and Frère
\emph{et al.} 2010. Several measures of population differentiation will
likely be calculated from several sample sites across the Pilbara
region. Statistical software such as STRUCTURE, FLOCK and BOTTLENECK may
be used for to examine the differentiation patterns between populations
and to test whether there has been a recent bottleneck (Brown\emph{, et
al.} 2014). This component of the project will likely be contracted to
Murdoch University Cetacean Research Unit.

~

\emph{Trophic Niche/diet}

If carcasses are recovered during stranding events, stomach contents
will be analysed for hard parts (fish otoliths and cephalopod beaks) and
teeth for stable isotope signatures but it is expected that strandings
and recovery of carcasses in the Pilbara region will be uncommon (Groom
\& Coughran 2012).~ Analysis of the carbon and nitrogen signatures of
the tissue samples or teeth compared will indicate whether dolphins are
feeding in brackish water, coastally or further offshore
(Fernandez\emph{, et al.} 2011; Gibbs\emph{, et al.} 2011; Owen\emph{,
et al.} 2011).




\subsection*{Biometrician's Endorsement}

granted



\section*{Data management}


\subsection*{No. specimens}






\subsection*{Herbarium Curator's Endorsement}

not required




\subsection*{Animal Ethics Committee's Endorsement}

required




\subsection*{Data management}

The survey data, including photo identification images will be stored in
DolFIN database, a purpose built database maintained on a secure server.

~

Tissue samples will be lodged with the Western Australian Museum




\section*{Budget}

\section*{Consolidated Funds }



\begin{longtabu} to \linewidth { |  X | X | X | X | }
\hline
\rowcolor{infobg}
Source & Year 1 & Year 2 & Year 3\\
\hline
\endhead



FTE Scientist &  &  & \\



FTE Technical &  &  & \\



Equipment &  &  & \\



Vehicle &  &  & \\



Travel &  &  & \\



Other &  &  & \\



Total &  &  & \\


\hline
\end{longtabu}



\section*{External Funds }



\begin{longtabu} to \linewidth { |  X | X | X | X | }
\hline
\rowcolor{infobg}
Source & Year 1 & Year 2 & Year 3\\
\hline
\endhead



Salaries, Wages, Overtime & $87,940 & $89,535 & $89,535\\



Overheads & $26,868 & $27,356 & $27,356\\



Equipment & $13,130 &  & \\



Vehicle & $15,000 & $15,000 & $15,000\\



Travel & $5,500 & $5,500 & $5,500\\



Other & $105,340 & $105,340 & $105,340\\



Total & $253,778 & $242,731 & $242,731\\


\hline
\end{longtabu}





%-----------------------------------------------------------------------------%
% Back matter
%\backmatter
\end{document}
%-----------------------------------------------------------------------------%

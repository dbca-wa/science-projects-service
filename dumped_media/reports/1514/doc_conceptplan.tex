
\documentclass[version=last,
    paper=a4,                               % paper size
    10pt,                                   % default font size
    dvipsnames,
    % twoside,                                % PRINT Binding Correction
    oneside,                              % ONLINE
    headings=openany,                       % open chapters on odd and even pages
    open=any,
    BCOR=7mm,                               % PRINT Binding Correction
    %DIV=13,    % typearea 161.54mm x 228.46mm, top 22.85mm, inner 16.15mm
    %DIV=14,    % 165.00 233.36 21.21 15.00
    DIV=15,     % 168.00 237.60 19.80 14.00
    % toc=chapterentrywithdots              % Table of Contents style
]{scrbook}
\usepackage{typearea}


%------------------------------------------------------------------------------%
% Headers and footers
%------------------------------------------------------------------------------%
\usepackage[automark,headsepline,footsepline,plainfootsepline]{scrlayer-scrpage}
\automark*[section]{}
\addtokomafont{pageheadfoot}{\normalfont\footnotesize\sffamily} % Don't italicise
\renewcommand{\chaptermark}[1]{\markleft{#1}{}}     % Chapter: suppress numbering
\renewcommand{\sectionmark}[1]{\markright{#1}{}}    % Section: suppress numbering

% Header (inner, center, outer)
% \ihead{\href{http://sdis.dbca.wa.gov.au}{\textbf{Concept Plan SP 2015-018}}}
%\chead{\href{http://sdis.dbca.wa.gov.au}{Science Directorate Information System}}
% \ohead{\href{https://www.dbca.wa.gov.au/science/10-biodiversity-and-conservation-science}{
% \includegraphics[height=8mm, keepaspectratio]{/usr/src/app/staticfiles/img/logo-dbca-bcs.jpg}}}

% Footer (inner, center, outer)
% \ifoot{\RaggedRight\leftmark}                       % Chapter
% \cfoot{\RaggedLeft\rightmark}                       % Section
% \ofoot[\bfseries\thepage]{\bfseries\thepage}        % Page number (also [plain])


%------------------------------------------------------------------------------%
% Fonts, encoding
%------------------------------------------------------------------------------%
%\usepackage{avant}             % Use the Avantgarde font for headings
\usepackage{txfonts}
\usepackage{mathptmx}
\usepackage{gensymb}            % provides \textdegree
\renewcommand{\familydefault}{\sfdefault} % Default to Sans Serif font
\usepackage{microtype}          % Slightly tweak font spacing for aesthetics
\usepackage[english]{babel}
\usepackage[utf8]{inputenc}  % Allow letters with accents
\usepackage[utf8]{luainputenc}  % Allow letters with accents
\usepackage[T1]{fontenc}        % Use 8-bit encoding that has 256 glyphs
\usepackage{textcomp}
\usepackage[explicit]{titlesec}           % Customise of titles
%\DeclareUnicodeCharacter{0080}{\textregistered}
\DeclareUnicodeCharacter{00B0}{\textdegree}

%------------------------------------------------------------------------------%
% Tables, columns, layout
%------------------------------------------------------------------------------%
\usepackage{etoolbox}
\AtBeginEnvironment{longtabu}{\footnotesize}{}{}  % Table font size
\usepackage{booktabs}           % Required for nicer horizontal rules in tables
\usepackage{multicol}           % 2 col publications
\usepackage{pdflscape}          % Landscape pages
\usepackage{pdfpages}           % Include PDFs
\usepackage{hanging}            % hanging paragraphs for publications
%\usepackage{titletoc}          % Manipulate the table of contents
\setcounter{tocdepth}{2}        % TOC list down to section
\usepackage{enumerate}          % Enumerations
\usepackage{enumitem}           % Enumerations
\usepackage{longtable}          % Multipage table
\usepackage{tabu}               %
\setlength{\tabulinesep}{1.5mm} % Consistent vertical spacing in tabu
\newcommand{\HRule}{\vspace{8mm}\noindent\rule{\linewidth}{0.1pt}}
\usepackage[export]{adjustbox}  % minipage, image frame


%------------------------------------------------------------------------------%
% Graphics, images, colours
%------------------------------------------------------------------------------%
\usepackage{graphicx} % embedded images
\usepackage{wrapfig}  % wrap figures in text
\usepackage{caption}  % allow unnumbered captions
\usepackage{eso-pic} % Required for specifying an image background in the title page
\usepackage{colortbl} % define custom named colours
\usepackage{xstring} % Conditionals
\usepackage{transparent} % Allow transparent images

\definecolor{RedFire}{RGB}{146,25,28}
% Following PICA branding guidelines
% https://dpaw.sharepoint.com/Divisions/pica/Documents/Branding%20guidelines.pdf
\definecolor{dpawblue}{RGB}{35,97,146}          % Pantone 647
\definecolor{dpaworange}{RGB}{237,139,0}        % Pantone 144
\definecolor{dpawgreen}{RGB}{116,170,80}        % Pantone 7489
\definecolor{dpawred}{RGB}{124,46,44}           % Paul's suggestion

% bootstrap colours
\definecolor{successbg}{RGB}{223,240,216}
\definecolor{errorbg}{RGB}{242,222,222}
\definecolor{warningbg}{RGB}{252,248,227}
\definecolor{infobg}{RGB}{217,237,247}
\definecolor{muted}{RGB}{153,153,153}
\definecolor{success}{RGB}{70,136,71}
\definecolor{error}{RGB}{185,74,72}
\definecolor{warning}{RGB}{192,152,83}
\definecolor{info}{RGB}{58,135,173}

% SDIS approval colours
\definecolor{required}{RGB}{192,152,83}
\definecolor{requiredbg}{RGB}{252,248,227}
\definecolor{denied}{RGB}{185,74,72}
\definecolor{deniedbg}{RGB}{242,222,222}
\definecolor{granted}{RGB}{70,136,71}
\definecolor{grantedbg}{RGB}{223,240,216}
\definecolor{notrequired}{RGB}{153,153,153}
\definecolor{notrequiredbg}{RGB}{255,255,255}

\usepackage{tikz} % Drawing
\usetikzlibrary{arrows,shapes,positioning,shadows,trees}


%------------------------------------------------------------------------------%
% Hyperlinks
%------------------------------------------------------------------------------%
\usepackage[open=true]{bookmark}
\usepackage{nameref}
\usepackage{ifxetex,ifluatex}
\ifxetex
  \usepackage[
    setpagesize=false,        % page size defined by xetex
    unicode=false,            % unicode breaks when used with xetex
    xetex]{hyperref}
\else
  \usepackage[unicode=true]{hyperref}
\fi

\hypersetup{
  backref=true,
  pagebackref=true,
  hyperindex=true,
  breaklinks=true,
  urlcolor=dpawblue,
  bookmarks=true,
  bookmarksopen=false,
  pdfauthor={Biodiversity and Conservation Science, Department of Biodiversity, Conservation and Attractions, WA},
  pdftitle=Concept Plan SP 2015-018
,
  colorlinks=true,
  linkcolor=dpawblue,
  pdfborder={0 0 0}}

\urlstyle{same}                         % don't use monospace font for urlstyle


%------------------------------------------------------------------------------%
% Black magic to linebreak URLs
%------------------------------------------------------------------------------%
\usepackage{url}
\makeatletter\g@addto@macro{\UrlBreaks}{\UrlOrds}\makeatother
\Urlmuskip=0mu plus 1mu


%------------------------------------------------------------------------------%
% Fix latex errors
%------------------------------------------------------------------------------%
\providecommand{\tightlist}{\setlength{\itemsep}{0pt}\setlength{\parskip}{0pt}}

% copy-pasted HTML <span> in SDIS fields becomes \text{} in tex source
\providecommand{\text}{}


%------------------------------------------------------------------------------%
% Custom Tikz node for SDS diagram
%------------------------------------------------------------------------------%
\newcommand\mynode[6][]{
  \node[#1] (#2){
    \parbox{#3\relax}{
      \begin{center}
      \textbf{#4}\\
      #5\\
      \footnotesize{#6}
      \end{center}
    }};}


%------------------------------------------------------------------------------%
% Custom no-pagebreaks-environment
%------------------------------------------------------------------------------%
\newenvironment{absolutelynopagebreak}
  {\par\nobreak\vfil\penalty0\vfilneg\vtop\bgroup}
  {\par\xdef\tpd{\the\prevdepth}\egroup\prevdepth=\tpd}


%------------------------------------------------------------------------------%
% Remove the header from odd empty pages at the end of chapters
%------------------------------------------------------------------------------%
\makeatletter
\renewcommand{\cleardoublepage}{
\clearpage\ifodd\c@page\else
\hbox{}
\vspace*{\fill}
\thispagestyle{empty}
\newpage
\fi}


%----------------------------------------------------------------------------------------
%  Page flow control
%----------------------------------------------------------------------------------------
%\widowpenalty=10000
%\clubpenalty=10000
%\vbadness=1200
%\hbadness=11000


%----------------------------------------------------------------------------------------
%   CHAPTER HEADINGS
%----------------------------------------------------------------------------------------
\newcommand{\thechapterimage}{}
\newcommand{\chapterimage}[1]{\renewcommand{\thechapterimage}{#1}}

% Numbered chapters with mini tableofcontents
\def\thechapter{\arabic{chapter}}
\def\@makechapterhead#1{
%\thispagestyle{plain}
{\centering \normalfont\sffamily
\ifnum \c@secnumdepth >\m@ne
\if@mainmatter
\startcontents
\begin{tikzpicture}[remember picture,overlay]
\node at (current page.north west)
{\begin{tikzpicture}[remember picture,overlay]
\node[anchor=north west,inner sep=0pt] at (0,0) {
\includegraphics[width=\paperwidth,height=0.5\paperwidth]{\thechapterimage}};
%------------------------------------------------------------------------------%
% Small contents box in the chapter heading
% Mini TOC background box
%\fill[color=dpawblue!10!white,opacity=.2] (1cm,0) rectangle (
%  3.5cm, % Mini TOC box width
%  -3.5cm % Mini TOC box height
%);
% Mini TOC text content
%\node[anchor=north west] at (1.1cm,.35cm) {
%  \parbox[t][8cm][t]{6.5cm}{
%    \huge\bfseries\flushleft
%    \printcontents{l}{1}{
%    \setcounter{tocdepth}{1}                   % Mini TOC level depth
%    }
% }
%};
%------------------------------------------------------------------------------%
% Chapter heading
\draw[anchor=west] (5cm,-9cm) node [
rounded corners=20pt,
fill=dpawblue!10!white,
text opacity=1,
draw=dpawblue,
draw opacity=1,
line width=1.5pt,
fill opacity=.2,
inner sep=12pt]{
    \huge\sffamily\bfseries\textcolor{black}{
      \thechapter. #1\strut\makebox[22cm]{}
    }
};
\end{tikzpicture}};
\end{tikzpicture}}
\par\vspace*{240\p@}                            % Push text below chapter image
\fi
\fi}

%------------------------------------------------------------------------------%
% Unnumbered chapters without mini tableofcontents
%------------------------------------------------------------------------------%
\def\@makeschapterhead#1{
%\thispagestyle{plain}
{\centering \normalfont\sffamily
\ifnum \c@secnumdepth >\m@ne
\if@mainmatter
\begin{tikzpicture}[remember picture,overlay]
\node at (current page.north west)
{\begin{tikzpicture}[remember picture,overlay]
\node[anchor=north west,inner sep=0pt] at (0,0) {
  \includegraphics[width=\paperwidth,height=0.5\paperwidth]{\thechapterimage}};
% Mini TOC background box
%\fill[color=dpawblue!10!white,opacity=.2] (1cm,0) rectangle (
%  3.5cm,                                       % Mini TOC box width
%  -3.5cm                                       % Mini TOC box height
%);
% Mini TOC text content
%\node[anchor=north west] at (1.1cm,.35cm) {
%  \parbox[t][8cm][t]{6.5cm}{
%    \huge\bfseries\flushleft
%    \printcontents{l}{1}{
%    \setcounter{tocdepth}{1} % Mini TOC level depth
%    }
%}
%};
\draw[anchor=west] (5cm,-9cm) node [rounded corners=20pt,
  fill=dpawblue!10!white,fill opacity=.6,inner sep=12pt,text opacity=1,
  draw=dpawblue,draw opacity=1,line width=1.5pt]{
  \huge\sffamily\bfseries\textcolor{black}{#1\strut\makebox[22cm]{}}};
\end{tikzpicture}};
\end{tikzpicture}}
\par\vspace*{240\p@}
\fi
\fi
}
\makeatother



\usepackage[automark,headsepline,footsepline,plainfootsepline]{scrlayer-scrpage}
\automark*[section]{}
\addtokomafont{pageheadfoot}{\normalfont\footnotesize\sffamily} % Don't italicise
\renewcommand{\chaptermark}[1]{\markleft{#1}{}}     % Chapter: suppress numbering
\renewcommand{\sectionmark}[1]{\markright{#1}{}}    % Section: suppress numbering

% Header (inner, center, outer)
\ihead{\href{http://sdis.dbca.wa.gov.au/documents/conceptplan/1514/}{Concept Plan SP 2015-018}}
%\chead{\href{http://sdis.dbca.wa.gov.au}{Science Directorate Information System}}
\ohead{\href{https://www.dbca.wa.gov.au/science/10-biodiversity-and-conservation-science}{
\includegraphics[height=6mm, keepaspectratio]{/usr/src/app/staticfiles/img/logo-dbca-bcs.jpg}}}
% Footer (inner, center, outer)
\ifoot{\textbf{Printed}~Fri, 7 Feb 2020 14:21:33 +0800} % inner/left footer
\cfoot{}
\ofoot[\bfseries\thepage]{\bfseries\thepage}        % Page number (also [plain])


\pagestyle{scrheadings}
\setkomafont{pageheadfoot}{\normalfont}

%-----------------------------------------------------------------------------%
\begin{document}
\raggedbottom

%-----------------------------------------------------------------------------%
% Title page
\subject{Concept Plan SP 2015-018
}
\title{Identification of threats and critical aspects of the ecology of the
threatened Pilbara Olive Python (\emph{Liasis olivaceus barroni}) to aid
management.
}
\subtitle{Animal Science
}
\author{}
\publishers{\small
    \subsection*{Project Core Team}
\begin{tabu} {X X}
\textbf{Supervising Scientist} & David Pearson
\\
\textbf{Data Custodian} & David Pearson
\\
\textbf{Site Custodian} & 
\\
\end{tabu}


    \subsection*{Project status as of Feb. 7, 2020, 2:21 p.m.}
\begin{tabu} {X X}
& Pending project plan approval
\\
\end{tabu}

    
\subsection*{Document endorsements and approvals as of Feb. 7, 2020, 2:21 p.m.}
\begin{tabu} {X X}

%\rowcolor{grantedbg}
    \textbf{Project Team} & 
    \textcolor{granted}{ granted}\\

%\rowcolor{grantedbg}
    \textbf{Program Leader} & 
    \textcolor{granted}{ granted}\\

%\rowcolor{grantedbg}
    \textbf{Directorate} & 
    \textcolor{granted}{ granted}\\

\end{tabu}



}
\uppertitleback{}
\lowertitleback{}
\date{}

%-----------------------------------------------------------------------------%
% Front matter
\frontmatter
\maketitle
%-----------------------------------------------------------------------------%
% Main matter
\mainmatter


\section*{Identification of threats and critical aspects of the ecology of the
threatened Pilbara Olive Python (\emph{Liasis olivaceus barroni}) to aid
management.
}



\subsection*{Biodiversity and Conservation Science Program}

Animal Science




\subsection*{Departmental Service}

Service 6: Conserving Habitats, Species and Communities





\subsection*{Background}

None




\subsection*{Aims}

\textbf{BACKGROUND}

~

The Pilbara Olive Python (\emph{Liasis olivacea barroni}) is a large
species (up to 5 m in length) restricted to the Pilbara and northern
Ashburton regions of Western Australia. It is listed as ``Vulnerable''
under the EPBC Act 1999 and "Threatened" under Schedule 1 of the
Wildlife Conservation Notice 2014 (WA). Limited information is available
on its distribution, ecology, population structure and trends, and
conservation threats.~

Its massive size at maturity, restricted habitat preferences, probable
low densities and a diet containing large vertebrates (including a
number of threatened species) makes the Pilbara Olive Python (POP)
potentially vulnerable to a range of changes to its habitat. No decline
in the overall population size or the distribution has been detected,
but there is insufficient historical and recent data to establish any
trends. A number of potential threats could result in local or wider
extinctions including: habitat destruction or alteratlion by
infrastructure or mining projects; habitat degradation around water
bodies due to cattle and frequent fires; predation of young by foxes and
feral cats; and the loss of important food species (e.g. fruit bats,
quolls, rock-wallabies) due to such factors as feral animals or
inappropriate fire regimes. Information is available on the distribution
of POPs (Smith 1981, Pearson 1993) and some ecological work has been
undertaken by the Department in association with community groups
(Pearson 2003, 2007, Tutt \emph{et al}. 2002, 2004).

In the last decade, the rapid expansion of resources projects and their
associated infrastructure in the Pilbara has resulted in numerous
referrals under the EPBC Act concerning the presence or probable
presence of POPs on mining leases. Its cryptic nature has frustrated
attempts to survey and monitor the species and made assessment of what
constitutes a ``significant'' impact from development problematic.~

This project seeks to examine critical aspects of the ecology and
population dynamics likely to be affected by feral animals and changes
caused by fire and cattle. For instance, radio-telemetry (Pearson in
prep.) suggests that juvenile pythons rely on thick vegetation around
water bodies to hide and to capture prey, while the nesting sites of
adult females~may be~remote from usual home ranges. Females are "capital
breeders". They breed infrequently, relying on depositing substantial
fat reserves to enable the production and incubation of eggs. Loss of
preferred prey items can potentially delay or halt reproduction. The
project will focus on life history parameters such as ~diet, habitat
preferences, reproductive patterns and recruitment of POPs, as well as
determining appropriate techniques for survey and monitoring of the
species.

~

~\textbf{AIMS}

~1. Collate existing information about the Pilbara Olive Python and
publish research data on diet and foraging activity.

~2. Clarify the distribution, population structure and conservation
status of POPs.

~3. Establish the reproductive requirements of female POPs in relation
to body condition and preferred nest sites.

~4. Develop techniques to monitor the recruitment of juveniles, the
population cohort likely to be most affected by feral animals.~

~5. Trial, investigate and improve survey and monitoring techniques to
enable better assessments of potential and future impacts of resource
projects and management activities on POPs.~




\subsection*{Expected outcome}

The project will dramatically improve our knowledge of the ecology of
the Pilbara Olive Python, techniques to survey and monitor populations
and what management actions may be required to maintain its populations.
It will allow the Department and other agencies to better assess the
likely impact of resource developments and management activities on POPs
and techniques to mitigate those impacts (if necessary) and monitor
their effectiveness. It will provide direction to consultants on the
best ways to locate POPs and establish monitoring programs.

To communicate this information, the following outputs are envisaged:

Peer-reviewed papers (with tentative titles) could include;

1. Re-description of the POP based on genetics, meristics and
morphometrics.

2. Diet and foraging behaviour of the POP.

3. Distribution, population structuring and conservation status of the
POP.

4. Habitat preferences and activity patterns of the POP.

5. Reproductive behaviour in POPs: a large capital breeder in an arid
land.

6. Survey and monitoring protocols for a large cryptic predator- the
POP.

7. Radio-telemetry and other monitoring techniques for large cryptic
predatory snakes.~

8. What role does fire management and feral cat control have in the
recruitment and survival of the threatened POP?

~

Other written outputs would include:

1. Report on protocols to survey and monitor POPs for mining companies
and consultants.

2. Annual reports to funding bodies.

3. Articles in Landscope and other popular magazines.

4. Media releases.

~

Other communications:

1. TV/documentaries- large pythons are popular with the public- I have
been involved in 2 documentaries in relation to POPs in the past (German
VoxTierzit and Wildlife Rescue on Australian TV). I would anticipate
future approaches from documentary production companies and television
networks. ~




\subsection*{Strategic context}

The Pilbara Olive Python is an iconic threatened species in the Pilbara
and its possible presence on mining leases has resulted in numerous EPBC
referrals in the last decade. Consultants working on behalf of resource
companies have struggled to find POPs and have not developed effective
ways to monitor the impacts of mining activity on the species. POPs
occur in several national parks and conservation reserves managed by the
Department. It is WA's largest snake and a species popular with the
wider public and Pilbara residents in particular.

It is a listed "Vulnerable" species under the EPBC Act and "Threatened"
under WA legislation. The aims and outcomes of this proposed project are
consistent with the research priorities identified by the Commonwealth
Conservation Advice on \emph{Lialis olivaceus barroni} (Threatened
Species Committee 2008) as outlined below:

"Research priorities that would inform future regional and local
priority actions include:

- Design and implement a monitoring program.

- More precisely assess population size, distribution, ecological
requirements and the relative impacts of threatening processes.

- Undertake survey work in suitable habitat and potential habitat to
locate any populations/occurrences."

It also~aligns with~the seven projects identified by Pearson and Morris
(2011) in "Project Plan- The ecology and conservation of the Pilbara
Olive Python 2011-2016" (Department of Parks and Wildlife):

1. Review of published and unpublished literature.

2. Development of survey and monitoring techniques.

3. Pilbara Olive Python genetics and population structure.

4. Detailed field ecology of Pilbara Olive Pythons.

5. Developing strategies to minimise Pilbara Olive Python mortality.

6. Reducing the impact of mining and infrastructure on Pilbara Olive
Pythons.

7. Monitoring Pilbara Olive Python populations.

A POP workshop was held at the Department of Parks and Wildlife offices
at Kensington on December 10, 2013 and identified the following research
requirements:

"1. Undertake a literature review.

2a. Develop survey techniques.

2b. Develop monitoring techniques.

2c. Better understand habitat requirements.

2d. Better understand breeding biology.

3a. Better understand prey relationships.

3b. Better understand predator relationships."

~

The proposed project examines the most critical of the research
priorities identifed by these three documents, focussing on reviewing
available literature, developing and testing survey/monitoring
techniques, resolving population relationships and structure via genetic
techniques, identifying conservation threats and undertaking detailed
ecological work to understand habitat requirements, diet and
reproduction. The project will primarily use radio-telemetry to follow
the fate of a number of olive pythons in response to fire management and
feral animal (cat) control.

Note that the project has a five year life on account of the apparent
low reproductive frequency of female POPs to enable determination of
factors associated with this. Funding required for Years 4 and 5 would
each be similar to Year 3.




\subsection*{Expected collaborations}

A range of collaborations are anticipated.

The project will work synergisitically where possible with existing
Northern Quoll work at Yarraloola and Red Hill funded by Rio Tinto (and
managed by Keith Morris). It will involve Pilbara regional staff
wherever possible in site selection, capture of study animals,
radio-telemetry and other aspects.

Outside the Department, the project will seek the involvement of
indigenous ranger groups and the general public to report sightings,
collect scales and roadkills, and potentially radio-track pythons.
Environmental consultants will be asked to contribute scale samples for
genetic analysis and will be consulted in relation to survey and
monitoring techniques they are currently using or have attempted to
apply.

A co-operative study of the genetic distinctiveness of POPs in relation
to other populations of olive pythons is already underway with Dr Peter
Spencer of Murdoch University. A preliminary report was produced, but a
better geographic spread of samples collected during this project would
allow a more thorough assessment of population structuring. This would
involve the WA Museum in its role as custodian for samples and
specimens. There may be opportunities to involve students in certain
aspects of the project and this will be explored depending on funding
available.

Funding for the project will most likely to come from threatened species
offsets and so it will be necessary to collaborate with mining companies
and their environmental sections to undertake fieldwork and to carry out
any habitat manipulations such as grazing or fire management.~


\subsection*{Proposed period of the project}
Jan. 1, 2016 -- Jan. 1, 2021



\subsection*{Staff time allocation }



\begin{longtabu} to \linewidth { |  X | X | X | X | }
\hline
\rowcolor{infobg}
Role & Year 1 & Year 2 & Year 3\\
\hline
\endhead



Scientist & 0.2 & 0.2 & 0.2\\



Technical & 1.0 & 1.0 & 0.2\\



Volunteer & 0.2 & 0.2 & 0.2\\



Collaborator & 0.1 & 0.2 & 0.1\\


\hline
\end{longtabu}



\subsection*{Indicative operating budget }



\begin{longtabu} to \linewidth { |  X | X | X | X | }
\hline
\rowcolor{infobg}
Source & Year 1 & Year 2 & Year 3\\
\hline
\endhead



Consolidated Funds (DPaW) &  &  & \\



External Funding & $180,000 over three years &  & \\


\hline
\end{longtabu}






%-----------------------------------------------------------------------------%
% Back matter
%\backmatter
\end{document}
%-----------------------------------------------------------------------------%

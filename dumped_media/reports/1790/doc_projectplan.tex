
\documentclass[version=last,
    paper=a4,                               % paper size
    10pt,                                   % default font size
    dvipsnames,
    % twoside,                                % PRINT Binding Correction
    oneside,                              % ONLINE
    headings=openany,                       % open chapters on odd and even pages
    open=any,
    BCOR=7mm,                               % PRINT Binding Correction
    %DIV=13,    % typearea 161.54mm x 228.46mm, top 22.85mm, inner 16.15mm
    %DIV=14,    % 165.00 233.36 21.21 15.00
    DIV=15,     % 168.00 237.60 19.80 14.00
    % toc=chapterentrywithdots              % Table of Contents style
]{scrbook}
\usepackage{typearea}


%------------------------------------------------------------------------------%
% Headers and footers
%------------------------------------------------------------------------------%
\usepackage[automark,headsepline,footsepline,plainfootsepline]{scrlayer-scrpage}
\automark*[section]{}
\addtokomafont{pageheadfoot}{\normalfont\footnotesize\sffamily} % Don't italicise
\renewcommand{\chaptermark}[1]{\markleft{#1}{}}     % Chapter: suppress numbering
\renewcommand{\sectionmark}[1]{\markright{#1}{}}    % Section: suppress numbering

% Header (inner, center, outer)
% \ihead{\href{http://sdis.dpaw.wa.gov.au}{\textbf{Project Plan CF 2015-024}}}
%\chead{\href{http://sdis.dpaw.wa.gov.au}{Science Directorate Information System}}
% \ohead{\href{https://www.dpaw.wa.gov.au/about-us/science-and-research}{\includegraphics[height=8mm, keepaspectratio]{/mnt/projects/sdis/staticfiles/img/logo-dpaw.png}}}

% Footer (inner, center, outer)
% \ifoot{\RaggedRight\leftmark}                       % Chapter
% \cfoot{\RaggedLeft\rightmark}                       % Section
% \ofoot[\bfseries\thepage]{\bfseries\thepage}        % Page number (also [plain])


%------------------------------------------------------------------------------%
% Fonts, encoding
%------------------------------------------------------------------------------%
%\usepackage{avant}             % Use the Avantgarde font for headings
\usepackage{txfonts}
\usepackage{mathptmx}
\usepackage{gensymb}            % provides \textdegree
\renewcommand{\familydefault}{\sfdefault} % Default to Sans Serif font
\usepackage{microtype}          % Slightly tweak font spacing for aesthetics
\usepackage[english]{babel}
\usepackage[utf8]{inputenc}  % Allow letters with accents
\usepackage[utf8]{luainputenc}  % Allow letters with accents
\usepackage[T1]{fontenc}        % Use 8-bit encoding that has 256 glyphs
\usepackage{textcomp}
\usepackage[explicit]{titlesec}           % Customise of titles
%\DeclareUnicodeCharacter{0080}{\textregistered}
\DeclareUnicodeCharacter{00B0}{\textdegree}

%------------------------------------------------------------------------------%
% Tables, columns, layout
%------------------------------------------------------------------------------%
\usepackage{etoolbox}
\AtBeginEnvironment{longtabu}{\footnotesize}{}{}  % Table font size
\usepackage{booktabs}           % Required for nicer horizontal rules in tables
\usepackage{multicol}           % 2 col publications
\usepackage{pdflscape}          % Landscape pages
\usepackage{pdfpages}           % Include PDFs
\usepackage{hanging}            % hanging paragraphs for publications
%\usepackage{titletoc}          % Manipulate the table of contents
\setcounter{tocdepth}{2}        % TOC list down to section
\usepackage{enumerate}          % Enumerations
\usepackage{enumitem}           % Enumerations
\usepackage{longtable}          % Multipage table
\usepackage{tabu}               %
\setlength{\tabulinesep}{1.5mm} % Consistent vertical spacing in tabu
\newcommand{\HRule}{\vspace{8mm}\noindent\rule{\linewidth}{0.1pt}}
\usepackage[export]{adjustbox}  % minipage, image frame


%------------------------------------------------------------------------------%
% Graphics, images, colours
%------------------------------------------------------------------------------%
\usepackage{graphicx} % embedded images
\usepackage{wrapfig}  % wrap figures in text
\usepackage{caption}  % allow unnumbered captions
\usepackage{eso-pic} % Required for specifying an image background in the title page
\usepackage{colortbl} % define custom named colours
\usepackage{xstring} % Conditionals
\usepackage{transparent} % Allow transparent images

\definecolor{RedFire}{RGB}{146,25,28}
% Following PICA branding guidelines
% https://dpaw.sharepoint.com/Divisions/pica/Documents/Branding%20guidelines.pdf
\definecolor{dpawblue}{RGB}{35,97,146}          % Pantone 647
\definecolor{dpaworange}{RGB}{237,139,0}        % Pantone 144
\definecolor{dpawgreen}{RGB}{116,170,80}        % Pantone 7489
\definecolor{dpawred}{RGB}{124,46,44}           % Paul's suggestion

% bootstrap colours
\definecolor{successbg}{RGB}{223,240,216}
\definecolor{errorbg}{RGB}{242,222,222}
\definecolor{warningbg}{RGB}{252,248,227}
\definecolor{infobg}{RGB}{217,237,247}
\definecolor{muted}{RGB}{153,153,153}
\definecolor{success}{RGB}{70,136,71}
\definecolor{error}{RGB}{185,74,72}
\definecolor{warning}{RGB}{192,152,83}
\definecolor{info}{RGB}{58,135,173}

% SDIS approval colours
\definecolor{required}{RGB}{192,152,83}
\definecolor{requiredbg}{RGB}{252,248,227}
\definecolor{denied}{RGB}{185,74,72}
\definecolor{deniedbg}{RGB}{242,222,222}
\definecolor{granted}{RGB}{70,136,71}
\definecolor{grantedbg}{RGB}{223,240,216}
\definecolor{notrequired}{RGB}{153,153,153}
\definecolor{notrequiredbg}{RGB}{255,255,255}

\usepackage{tikz} % Drawing
\usetikzlibrary{arrows,shapes,positioning,shadows,trees}


%------------------------------------------------------------------------------%
% Hyperlinks
%------------------------------------------------------------------------------%
\usepackage[open=true]{bookmark}
\usepackage{nameref}
\usepackage{ifxetex,ifluatex}
\ifxetex
  \usepackage[
    setpagesize=false,        % page size defined by xetex
    unicode=false,            % unicode breaks when used with xetex
    xetex]{hyperref}
\else
  \usepackage[unicode=true]{hyperref}
\fi

\hypersetup{
  backref=true,
  pagebackref=true,
  hyperindex=true,
  breaklinks=true,
  urlcolor=dpawblue,
  bookmarks=true,
  bookmarksopen=false,
  pdfauthor={Biodiversity and Conservation Science, Department of Biodiversity, Conservation and Attractions, WA},
  pdftitle=Project Plan CF 2015-024
,
  colorlinks=true,
  linkcolor=dpawblue,
  pdfborder={0 0 0}}

\urlstyle{same}                         % don't use monospace font for urlstyle


%------------------------------------------------------------------------------%
% Black magic to linebreak URLs
%------------------------------------------------------------------------------%
\usepackage{url}
\makeatletter\g@addto@macro{\UrlBreaks}{\UrlOrds}\makeatother
\Urlmuskip=0mu plus 1mu


%------------------------------------------------------------------------------%
% Fix latex errors
%------------------------------------------------------------------------------%
\providecommand{\tightlist}{\setlength{\itemsep}{0pt}\setlength{\parskip}{0pt}}

% copy-pasted HTML <span> in SDIS fields becomes \text{} in tex source
\providecommand{\text}{}


%------------------------------------------------------------------------------%
% Custom Tikz node for SDS diagram
%------------------------------------------------------------------------------%
\newcommand\mynode[6][]{
  \node[#1] (#2){
    \parbox{#3\relax}{
      \begin{center}
      \textbf{#4}\\
      #5\\
      \footnotesize{#6}
      \end{center}
    }};}


%------------------------------------------------------------------------------%
% Custom no-pagebreaks-environment
%------------------------------------------------------------------------------%
\newenvironment{absolutelynopagebreak}
  {\par\nobreak\vfil\penalty0\vfilneg\vtop\bgroup}
  {\par\xdef\tpd{\the\prevdepth}\egroup\prevdepth=\tpd}


%------------------------------------------------------------------------------%
% Remove the header from odd empty pages at the end of chapters
%------------------------------------------------------------------------------%
\makeatletter
\renewcommand{\cleardoublepage}{
\clearpage\ifodd\c@page\else
\hbox{}
\vspace*{\fill}
\thispagestyle{empty}
\newpage
\fi}


%----------------------------------------------------------------------------------------
%  Page flow control
%----------------------------------------------------------------------------------------
%\widowpenalty=10000
%\clubpenalty=10000
%\vbadness=1200
%\hbadness=11000


%----------------------------------------------------------------------------------------
%   CHAPTER HEADINGS
%----------------------------------------------------------------------------------------
\newcommand{\thechapterimage}{}
\newcommand{\chapterimage}[1]{\renewcommand{\thechapterimage}{#1}}

% Numbered chapters with mini tableofcontents
\def\thechapter{\arabic{chapter}}
\def\@makechapterhead#1{
%\thispagestyle{plain}
{\centering \normalfont\sffamily
\ifnum \c@secnumdepth >\m@ne
\if@mainmatter
\startcontents
\begin{tikzpicture}[remember picture,overlay]
\node at (current page.north west)
{\begin{tikzpicture}[remember picture,overlay]
\node[anchor=north west,inner sep=0pt] at (0,0) {
\includegraphics[width=\paperwidth,height=0.5\paperwidth]{\thechapterimage}};
%------------------------------------------------------------------------------%
% Small contents box in the chapter heading
% Mini TOC background box
%\fill[color=dpawblue!10!white,opacity=.2] (1cm,0) rectangle (
%  3.5cm, % Mini TOC box width
%  -3.5cm % Mini TOC box height
%);
% Mini TOC text content
%\node[anchor=north west] at (1.1cm,.35cm) {
%  \parbox[t][8cm][t]{6.5cm}{
%    \huge\bfseries\flushleft
%    \printcontents{l}{1}{
%    \setcounter{tocdepth}{1}                   % Mini TOC level depth
%    }
% }
%};
%------------------------------------------------------------------------------%
% Chapter heading
\draw[anchor=west] (5cm,-9cm) node [
rounded corners=20pt,
fill=dpawblue!10!white,
text opacity=1,
draw=dpawblue,
draw opacity=1,
line width=1.5pt,
fill opacity=.2,
inner sep=12pt]{
    \huge\sffamily\bfseries\textcolor{black}{
      \thechapter. #1\strut\makebox[22cm]{}
    }
};
\end{tikzpicture}};
\end{tikzpicture}}
\par\vspace*{240\p@}                            % Push text below chapter image
\fi
\fi}

%------------------------------------------------------------------------------%
% Unnumbered chapters without mini tableofcontents
%------------------------------------------------------------------------------%
\def\@makeschapterhead#1{
%\thispagestyle{plain}
{\centering \normalfont\sffamily
\ifnum \c@secnumdepth >\m@ne
\if@mainmatter
\begin{tikzpicture}[remember picture,overlay]
\node at (current page.north west)
{\begin{tikzpicture}[remember picture,overlay]
\node[anchor=north west,inner sep=0pt] at (0,0) {
  \includegraphics[width=\paperwidth,height=0.5\paperwidth]{\thechapterimage}};
% Mini TOC background box
%\fill[color=dpawblue!10!white,opacity=.2] (1cm,0) rectangle (
%  3.5cm,                                       % Mini TOC box width
%  -3.5cm                                       % Mini TOC box height
%);
% Mini TOC text content
%\node[anchor=north west] at (1.1cm,.35cm) {
%  \parbox[t][8cm][t]{6.5cm}{
%    \huge\bfseries\flushleft
%    \printcontents{l}{1}{
%    \setcounter{tocdepth}{1} % Mini TOC level depth
%    }
%}
%};
\draw[anchor=west] (5cm,-9cm) node [rounded corners=20pt,
  fill=dpawblue!10!white,fill opacity=.6,inner sep=12pt,text opacity=1,
  draw=dpawblue,draw opacity=1,line width=1.5pt]{
  \huge\sffamily\bfseries\textcolor{black}{#1\strut\makebox[22cm]{}}};
\end{tikzpicture}};
\end{tikzpicture}}
\par\vspace*{240\p@}
\fi
\fi
}
\makeatother



\usepackage[automark,headsepline,footsepline,plainfootsepline]{scrlayer-scrpage}
\automark*[section]{}
\addtokomafont{pageheadfoot}{\normalfont\footnotesize\sffamily} % Don't italicise
\renewcommand{\chaptermark}[1]{\markleft{#1}{}}     % Chapter: suppress numbering
\renewcommand{\sectionmark}[1]{\markright{#1}{}}    % Section: suppress numbering

% Header (inner, center, outer)
\ihead{\href{http://sdis.dpaw.wa.gov.au/documents/projectplan/1790/}{Project Plan CF 2015-024}}
%\chead{\href{http://sdis.dpaw.wa.gov.au}{Science Directorate Information System}}
\ohead{\href{https://www.dpaw.wa.gov.au/about-us/science-and-research}{\includegraphics[height=6mm, keepaspectratio]{/mnt/projects/sdis/staticfiles/img/logo-dpaw.png}}}

% Footer (inner, center, outer)
\ifoot{\textbf{Printed}~Fri, 13 Jul 2018 10:58:52 +0800} % inner/left footer
\cfoot{}
\ofoot[\bfseries\thepage]{\bfseries\thepage}        % Page number (also [plain])


\pagestyle{scrheadings}
\setkomafont{pageheadfoot}{\normalfont}

%-----------------------------------------------------------------------------%
\begin{document}
\raggedbottom

%-----------------------------------------------------------------------------%
% Title page
\subject{Project Plan CF 2015-024
}
\title{BioSys -- the Western Australian Biological Survey Database
}
\subtitle{Ecoinformatics
}
\author{}
\publishers{\small
    \subsection*{Project Core Team}
\begin{tabu} {X X}
\textbf{Supervising Scientist} & Paul Gioia
\\
\textbf{Data Custodian} & Paul Gioia
\\
\textbf{Site Custodian} & 
\\
\end{tabu}


    \subsection*{Project status as of July 13, 2018, 10:58 a.m.}
\begin{tabu} {X X}
& Update requested
\\
\end{tabu}

    
\subsection*{Document endorsements and approvals as of July 13, 2018, 10:58 a.m.}
\begin{tabu} {X X}

%\rowcolor{grantedbg}
    \textbf{Project Team} & 
    \textcolor{granted}{ granted}\\

%\rowcolor{grantedbg}
    \textbf{Program Leader} & 
    \textcolor{granted}{ granted}\\

%\rowcolor{grantedbg}
    \textbf{Directorate} & 
    \textcolor{granted}{ granted}\\

%\rowcolor{grantedbg}
    \textbf{Biometrician} & 
    \textcolor{granted}{ granted}\\

%\rowcolor{not requiredbg}
    \textbf{Herbarium Curator} & 
    \textcolor{not required}{ not required}\\

%\rowcolor{not requiredbg}
    \textbf{Animal Ethics Committee} & 
    \textcolor{not required}{ not required}\\

\end{tabu}



}
\uppertitleback{}
\lowertitleback{}
\date{}

%-----------------------------------------------------------------------------%
% Front matter
\frontmatter
\maketitle
%-----------------------------------------------------------------------------%
% Main matter
\mainmatter



\section*{BioSys -- the Western Australian Biological Survey Database
}



\subsection*{Biodiversity and Conservation Science Program}

Ecoinformatics




\subsection*{Departmental Service}

Service 5: Conserving Habitats, Species and Ecological Communities


\subsection*{Project Staff}
\begin{tabu} {X X X}
%\rowcolor{infobg}
\textbf{Role} & \textbf{Person} & \textbf{Time allocation (FTE)}\\

Supervising Scientist & Paul Gioia & 0.5\\

Research Scientist & Florian Mayer & 0.2\\

\end{tabu}




\subsection*{Related Science Projects}

CF 2011-106~Online GIS biodiversity mapping (NatureMap)


\subsection*{Proposed period of the project}
June 30, 2015 -- None



\section*{Relevance and Outcomes}


\subsection*{Background}

One of the department's corporate goals is to conserve biodiversity. A
key strategy for achieving this is acquiring scientific knowledge to
underpin decision-making. The department therefore invests heavily in
ecological survey (from small area- inventory through to major regional
surveys) and on-going monitoring projects from which data are~collected
through field observation and analysed to produce new information.
Because of the size of Western Australia, and the inaccessibility of
many sites, data collection is typically the most expensive component of
producing new information and understanding, and the datasets are often
irreplaceable.

However, the majority of datasets collected through survey are not
lodged in a corporately accessible and managed environment. While the
department maintains some~corporate scientific and management databases,
these capture only a small proportion of data generated through survey,
probably less than 25\%. The remainder of datasets are managed on a
per-project basis, satisfying individual project requirements, but
invariably without due attention to on-going data maintenance and
availability. To compound the issue, an increasing number of scientists
and project officers responsible for historical datasets are close to
retirement age. The department therefore faces an increasing and
significant risk of data loss.

Compounding the situation,~data are stored in various ways ranging from
spreadsheets to Microsoft Access, on desktops, laptops, or local file
servers. As such, security and data protection~are dependent on the data
custodian, resulting in inconsistent practices and increased likelihood
of data loss or corruption. In addition, data access is typically
limited to local users, or restricted by vendor-specific applications.
Departmental requirements for linking to related systems~are~generally
not considered. And, finally, while survey methodologies are well
understood, inadequate attention is paid~to data collection and
management standards (particularly ecological data, which still has
developing standards), and therefore inconsistency and lack of
documentation in how data attributes are defined.




\subsection*{Aims}

\begin{itemize}
\itemsep1pt\parskip0pt\parsep0pt
\item
  Build a central repository for storing, curating and
  distributing~ecological data
\item
  Minimise the risk of data loss to the department
\item
  Increase access to ecological data
\item
  Create a single point of truth for ecological data within the
  department
\item
  Develop best-practice techniques for managing ecological data within
  a~corporate context
\item
  Facilitate the development of ecological data standards and survey
  protocols
\end{itemize}




\subsection*{Expected outcome}

\begin{itemize}
\itemsep1pt\parskip0pt\parsep0pt
\item
  Increased consistency in, and availability of, ecological data
\item
  Data protected in perpetuity against staff turnover and change in
  storage technologies
\item
  Improved security and backup for legacy and operational databases
\item
  Improved integration with other departmental systems
\item
  Improved capacity for reporting, research and analysis of
  observational data.
\item
  Improved compliance with government~requirements for knowledge
  management
\item
  Improved credibility of the department in protecting and making data
  available
\item
  Contribution to the public good by making all intellectual property
  publicly available, and encouraging co-development from sister
  organisations
\end{itemize}

In the longer term BioSys is expected to take over all or~part of the
role of NatureMap, built on modern, supported architecture.




\subsection*{Knowledge transfer}

The novel approaches to ecological data management used within BioSys
will have relevance to any biological survey database. BioSys is being
developed as open source software, and intellectual property and source
code will be made available to the general scientific and informatics
community.

Interest has already been expressed by the NSW Office of Environment and
Heritage in supporting and co-developing the application.




\subsection*{Tasks and Milestones}

\begin{longtable}[c]{@{}ll@{}}
\toprule\addlinespace
Develop minimum viable product (MVP) & 30/06/17
\\\addlinespace
Test database design against Pilbara and other key datasets & 31/12/17
\\\addlinespace
Implement interface to NatureMap for automated data consumption &
30/06/18
\\\addlinespace
Implement standardised ecological data model & 31/12/18
\\\addlinespace
Expose source code as open source and invite collaboration~ & 01/07/19
\\\addlinespace
Assess feasibility of offline data capture tool & 31/12/19
\\\addlinespace
\bottomrule
\end{longtable}




\subsection*{References}

Australian Venture Consultants (2012) Pathway to an Enhanced Western
Australian Terrestrial Biodiversity Knowledge System, August, 2012,
Perth, WA.

Salt, C., Burrows, N., Coates, D. \& van~Leeuwen, S. (2008) Vegetation
information management system: the need for a new vegetation map of WA:
Vegetation Mapping Workshop. Department of Environment and Conservation,
23-24 July 2008, Woodvale, WA.

Science~Division (2008) A Strategic Plan for BIodiversity Conservation
Research 2008-2017. Department of Environment and Conservation, Perth,
WA.



\section*{Study design}


\subsection*{Methodology}

Design principles and approach for BioSys include~two key attributes:
\emph{open source development}, and \emph{scalability}. All intellectual
property created during BioSys development will be made available
through the open source paradigm. While early stages of development will
be kept in-house, once BioSys has reached a basic level of acceptance
and maturity, the source code will be housed within a publicly
accessible repository. Other developers and research institutions will
be able to contribute to the code base so that the software will benefit
from potentially many sources.

Key reasons for making the source publicly available include~a)
providing a public good benefit given that BioSys has been developed
using public funds, and b) benefitting from the keen interest that other
parties have already expressed, and the additional expertise they could
bring to the project.

There has been a high level of difficulty in the conceptual design of
the~database thus far. Ecological data is, by its nature,
complex.~Additionally, survey protocols change over time as~knowledge
and techniques improve, resulting in different attributes being
collected. And, finally, the lack of ecological data standards, together
with~a culture of independence amongst some researchers,~contributes to
even greater variation in survey protocols and~data content and
organisation.

This leads to a~second key design attribute: scalability. It is a
challenge to design a database that will scale to
increasing~data~complexity and variability so that all ecological data
can be captured and curated, while on the other hand providing reports
and outputs in consistent formats for scientific research, conservation
planning or environmental impact assessment, or~consumption by other
information systems.~Traditional methods (e.g. relational databases) do
not have the flexibility to accommodate complex ecological data.~BioSys
is therefore being designed using the most recent techniques for storing
and updating semi-structured data that provide the required scalability
and flexibility.

An inherent tradeoff with the above approach is that variability in data
structure will be replicated within the database. Inconsistencies in how
researchers manage their data will potentially be perpetuated. It is not
possible to guarantee storage~and curation of any dataset, while
also~guaranteeing those datasets can be consolidated into a single,
monolithic, relational data model (e.g. for reporting or data exchange
purposes). That is a much harder~proposition.

The BioSys project is therefore a long-term project. In the first
instance, risk of data loss must be mitigated. It is therefore more
important that data is captured and protected within a corporate system
``as is'', than attempting the harder task of data standardisation.
BioSys will also allow data to be curated, the system potentially
becoming the~point of truth, rather than the desktop. Development will
be incremental and adaptive~- system capability and design will be
enhanced as new datasets and survey protocols extend conceptual
boundaries. In subsequent years, other capabilities will be added
(subject to available resourcing), such as offline data capture tools
that feed into BioSys.

Architecturally, BioSys is being developed as a web-based application
using OIM-preferred and supported technologies.

\textbf{BioSys}~is being implemented in a phased approach:

\begin{itemize}
\itemsep1pt\parskip0pt\parsep0pt
\item
  Build database to support Kimberley Land Conservation Initiative (LCI)
  monitoring data and~Kimberley Island survey data
\item
  Release minimum viable product for acceptance testing
\item
  Test and adapt database design by accommodating Pilbara and other key
  survey data sets
\item
  Implement a standardised ecological data model based on outcomes from
  initiatives such as Essential Measures vegetation working group
\item
  Evaluate feasibility of an offline data entry tool
\item
  Invite co-development from interested parties
\end{itemize}

The phased approach described above is dependent on ongoing funding.
Estimates below are based on current and historical funds provided from
the Kimberley Land Conservation Initiative, at Director's discretion,
but this is not an indication such funds will continue to be provided in
the future. There is the possibility projects dependent on BioSys
architecture might subsidise aspects of BioSys functionality.




\subsection*{Biometrician's Endorsement}

granted



\section*{Data management}


\subsection*{No. specimens}

N/A




\subsection*{Herbarium Curator's Endorsement}

not required




\subsection*{Animal Ethics Committee's Endorsement}

not required




\subsection*{Data management}

\begin{itemize}
\itemsep1pt\parskip0pt\parsep0pt
\item
  Data will be managed using novel techniques to handle complex and
  highly variable data.
\item
  All data will be stored within corporate infrastructure and benefit
  from standard disaster recovery processes.
\item
  Data will be archived periodically and stored off-site.
\item
  Data will be audited for changes at user-level
\end{itemize}




\section*{Budget}

\section*{Consolidated Funds }



\begin{longtabu} to \linewidth { |  X | X | X | X | }
\hline
\rowcolor{infobg}
Source & Year 1 & Year 2 & Year 3\\
\hline
\endhead



FTE Scientist & 0.5 FTE & 0.5 FTE & 0.5 FTE\\



FTE Technical & 0.1 FTE & 0.1 FTE & 0.1 FTE\\



Equipment &  &  & \\



Vehicle &  &  & \\



Travel &  &  & \\



Other & $70K & $70K & $70K\\



Total &  &  & \\


\hline
\end{longtabu}



\section*{External Funds }



\begin{longtabu} to \linewidth { |  X | X | X | X | }
\hline
\rowcolor{infobg}
Source & Year 1 & Year 2 & Year 3\\
\hline
\endhead



Salaries, Wages, Overtime &  &  & \\



Overheads &  &  & \\



Equipment &  &  & \\



Vehicle &  &  & \\



Travel &  &  & \\



Other &  &  & \\



Total &  &  & \\


\hline
\end{longtabu}





%-----------------------------------------------------------------------------%
% Back matter
%\backmatter
\end{document}
%-----------------------------------------------------------------------------%

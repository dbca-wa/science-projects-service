
\documentclass[version=last,
    paper=a4, % paper size
    10pt, % default font size
    usenames,
    dvipsnames,
    oneside, % ONLINE
    headings=openany, % open chapters on odd and even pages
    %toc=chapterentrywithdots, % Table of Contents style
    %BCOR=7mm, % PRINT Binding Correction
    %DIV=13, % typearea 161.54 mm x 228.46 mm, top margin 22.85 mm, inner margin 16.15 mm
    %DIV=14, % 165.00 233.36 21.21 15.00
    DIV=15 % 168.00 237.60 19.80 14.00
]{scrbook}
\usepackage{typearea}
\usepackage[automark,headsepline,footsepline]{scrlayer-scrpage} % Headers and footers

%%
%% Fonts, encoding, spacing, indentation
%%
\usepackage{txfonts}
\renewcommand{\familydefault}{\sfdefault} % Default to Sans Serif font
\usepackage[english]{babel}
\usepackage[T1]{fontenc}
\usepackage[utf8]{inputenc}

% Paragraph spacing
%\usepackage{parskip}    % Paragraph spacing
%\setlength{\parindent}{0em} % Don't indent paragraphs - ONLINE
%\setlength{\parskip}{1.3 ex plus 0.5ex minus 0.3ex} % 1-1.8 ex vertical space between paragraphs - ONLINE

% Spacing of headings
%\RedeclareSectionCommand[afterskip=3pt]{section} % less space after section
%\RedeclareSectionCommand[beforeskip=0cm]{subsection} % less space between HRule and project name
%\RedeclareSectionCommand[afterskip=0.1\baselineskip]{subsubsection} % less space after progressreport subheadings

% Table font size
\usepackage{etoolbox}
\AtBeginEnvironment{longtabu}{\footnotesize}{}{}

%%
%% Tables, columns, layout
%%
\usepackage{multicol}   % 2 col publications
\usepackage{pdflscape}  % Landscape pages
\usepackage{pdfpages}   % Include PDFs
\usepackage{hanging}    % hanging paragraphs for publications
%\usepackage{titletoc}   % Required for manipulating the table of contents
\setcounter{tocdepth}{2} % TOC list down to section
\usepackage{enumerate}  % Enumerations
\usepackage{enumitem}   % Enumerations
\usepackage{longtable}  % Multipage table
\usepackage{tabu}       %
\setlength{\tabulinesep}{1.5mm} % Consistent vertical spacing in tabu

%%
%% Graphics, images, colours
%%
\usepackage{graphicx} % embedded images
\usepackage{eso-pic} %
\usepackage{colortbl} % define custom named colours
\definecolor{RedFire}{RGB}{146,25,28}
\definecolor{ParksWildlife}{RGB}{0,85,144}
\definecolor{successbg}{RGB}{223,240,216}
\definecolor{errorbg}{RGB}{242,222,222}
\definecolor{warningbg}{RGB}{252,248,227}
\definecolor{infobg}{RGB}{217,237,247}
\definecolor{muted}{RGB}{153,153,153}
\definecolor{success}{RGB}{70,136,71}
\definecolor{error}{RGB}{185,74,72}
\definecolor{warning}{RGB}{192,152,83}
\definecolor{info}{RGB}{58,135,173}

\definecolor{required}{RGB}{192,152,83}
\definecolor{requiredbg}{RGB}{252,248,227}
\definecolor{denied}{RGB}{185,74,72}
\definecolor{deniedbg}{RGB}{242,222,222}
\definecolor{granted}{RGB}{70,136,71}
\definecolor{grantedbg}{RGB}{223,240,216}
\definecolor{not reqiured}{RGB}{153,153,153}
\definecolor{not requiredbg}{RGB}{255,255,255}

\usepackage{tikz} % Drawing
\usetikzlibrary{arrows,shapes,positioning,shadows,trees}

%%
%% Links, URLs
%%
\usepackage[
    linktoc=all,
    %colorlinks=false,  %PRINT
    colorlinks=true, % ONLINE
    linkcolor=RedFire, % ONLINE
    urlcolor=ParksWildlife, % ONLINE
    pdftitle=Progress Report SP 2012-023 (FY 2015-2016)
]{hyperref}

% Black magic to linebreak URLs
\usepackage{url}
\makeatletter
\g@addto@macro{\UrlBreaks}{\UrlOrds}
\makeatother

%%
%% Custom macros
%%
% Thick Horizontal rule
\newcommand{\HRule}{\vspace{8mm}\\\noindent\rule{\linewidth}{0.1pt}}

% Custom Tikz node for SDS diagram
\newcommand\mynode[6][]{
    \node[#1] (#2){
        \parbox{#3\relax}{
            \begin{center}
            \textbf{#4}\\
            #5\\
            \footnotesize{#6}
            \end{center}}};}



\usepackage[automark,headsepline,footsepline,plainfootsepline]{scrlayer-scrpage}
\automark*[section]{}
\addtokomafont{pageheadfoot}{\normalfont\footnotesize\sffamily} % Don't italicise
\renewcommand{\chaptermark}[1]{\markleft{#1}{}}     % Chapter: suppress numbering
\renewcommand{\sectionmark}[1]{\markright{#1}{}}    % Section: suppress numbering

% Header (inner, center, outer)
\ihead{\href{http://sdis.dpaw.wa.gov.au/documents/progressreport/1623/}{Progress Report SP 2012-023 (FY 2015-2016)}}
%\chead{\href{http://sdis.dpaw.wa.gov.au}{Science Directorate Information System}}
\ohead{\href{https://www.dpaw.wa.gov.au/about-us/science-and-research}{\includegraphics[height=6mm, keepaspectratio]{/mnt/projects/sdis/staticfiles/img/logo-dpaw.png}}}

% Footer (inner, center, outer)
\ifoot{\textbf{Printed}~Tue, 4 Jul 2017 18:10:35 +0800} % inner/left footer
\cfoot{}
\ofoot[\bfseries\thepage]{\bfseries\thepage}        % Page number (also [plain])


\pagestyle{scrheadings}
\setkomafont{pageheadfoot}{\normalfont}

%-----------------------------------------------------------------------------%
\begin{document}
\raggedbottom

%-----------------------------------------------------------------------------%
% Title page
\subject{Progress Report SP 2012-023
}
\title{Feral cat control and numbat recovery in Dryandra woodland and other
sites
}
\subtitle{Animal Science
}
\author{}
\publishers{\small
    \subsection*{Project Core Team}
\begin{tabu} {X X}
\textbf{Supervising Scientist} & Tony Friend
\\
\textbf{Data Custodian} & 
\\
\textbf{Site Custodian} & 
\\
\end{tabu}


    \subsection*{Project status as of July 4, 2017, 6:10 p.m.}
\begin{tabu} {X X}
& Update requested
\\
\end{tabu}

    
\subsection*{Document endorsements and approvals as of July 4, 2017, 6:10 p.m.}
\begin{tabu} {X X}

%\rowcolor{grantedbg}
    \textbf{Project Team} & 
    \textcolor{granted}{ granted}\\

%\rowcolor{grantedbg}
    \textbf{Program Leader} & 
    \textcolor{granted}{ granted}\\

%\rowcolor{grantedbg}
    \textbf{Directorate} & 
    \textcolor{granted}{ granted}\\

\end{tabu}



}
\uppertitleback{}
\lowertitleback{}
\date{}

%-----------------------------------------------------------------------------%
% Front matter
\frontmatter
\maketitle
%-----------------------------------------------------------------------------%
% Main matter
\mainmatter

\section*{Feral cat control and numbat recovery in Dryandra woodland and other
sites
}

A Friend



\section*{Context}

Dryandra Woodland supports important populations of several threatened
mammals, including the numbat, woylie and red-tailed phascogale, as well
as significant populations of a number of threatened birds. Recent
research has shown that feral cats are responsible for the majority of
numbat and woylie deaths. This project investigates the feasibility and
efficacy of using the \emph{Eradicat\textsuperscript{®}} feral cat bait
to reduce numbat and woylie mortality and promote their recovery at
Dryandra.




\section*{Aims}

\begin{itemize}
\itemsep1pt\parskip0pt\parsep0pt
\item
  To determine the uptake of rhodamine-labelled non-toxic
  \emph{Eradicat}\textsuperscript{®} baits by chuditch, red-tailed
  phascogales and mardos in Dryandra.
\item
  To determine the survival or mortality of groups of radio-collared
  chuditch, red-tailed phascogales and mardos during a baiting campaign
  using toxic \emph{Eradicat\textsuperscript{®}}baits in Dryandra.
\item
  To determine the survival or mortality of feral cats through a baiting
  campaign using toxic \emph{Eradicat\textsuperscript{®}} baits in
  Dryandra.
\end{itemize}




\section*{Progress}

\begin{itemize}
\itemsep1pt\parskip0pt\parsep0pt
\item
  Four of the eight chuditch radio-collared at Dryandra in 2015 were
  recaptured and their collars removed. The other four were not found or
  trapped before their collars failed.
\item
  Camera traps were set at Tutanning Nature Reserve in March 2016 to
  monitor cats before and after an \emph{Eradicat\textsuperscript{®}}
  baiting event. Of 37 cameras deployed around tracks in the reserve, 16
  were stolen. As a result the camera monitoring was suspended.
  \emph{Eradicat\textsuperscript{®}} baiting was carried out as
  programmed in April 2016.
\end{itemize}




\section*{Management implications}

\begin{itemize}
\itemsep1pt\parskip0pt\parsep0pt
\item
  \emph{Eradicat\textsuperscript{®}} can be used with minimal non-target
  impacts in the Dryandra Woodland. This is significant for the
  implementation of integrated fox and feral cat control programs at
  Dryandra and elsewhere in the south-west of Western Australia.
\end{itemize}




\section*{Future directions}

\begin{itemize}
\itemsep1pt\parskip0pt\parsep0pt
\item
  Repeat an \emph{Eradicat\textsuperscript{®}} trial with GPS collared
  cats in Dryandra Woodland in 2016-17 to determine the effectiveness of
  this control method.
\item
  Assess the effectiveness of setting leg-hold traps in raised sites
  (buckets of sand) for cats to reduce non-target capture.
\item
  Monitor cat density before and after baiting using camera traps.
\item
  Carry out an additional trial to determine the risk to red-tailed
  phascogale from \emph{Eradicat\textsuperscript{®}}~baiting by
  monitoring 8-10 radio-collared phascogales through a baiting event at
  Dryandra and Tutanning.
\end{itemize}



%-----------------------------------------------------------------------------%
% Back matter
%\backmatter
\end{document}
%-----------------------------------------------------------------------------%


\documentclass[version=last, paper=a4, DIV=18, usenames, dvipsnames]{scrartcl}
\usepackage{txfonts}
\usepackage{pdflscape}
\usepackage{pdfpages}
\usepackage[english]{babel} % English language/hyphenation
%%% Bootstrap colors
\definecolor{RedFire}{RGB}{146,25,28}
\definecolor{ParksWildlife}{RGB}{0,85,144}
\definecolor{successbg}{RGB}{223,240,216}
\definecolor{errorbg}{RGB}{242,222,222}
\definecolor{warningbg}{RGB}{252,248,227}
\definecolor{infobg}{RGB}{217,237,247}
\definecolor{muted}{RGB}{153,153,153}
\definecolor{success}{RGB}{70,136,71}
\definecolor{error}{RGB}{185,74,72}
\definecolor{warning}{RGB}{192,152,83}
\definecolor{info}{RGB}{58,135,173}

\definecolor{required}{HTML}{D9534F}
\definecolor{denied}{HTML}{D9534F}
\definecolor{granted}{HTML}{47A447}
\definecolor{not required}{RGB}{200, 200, 200}

\usepackage[colorlinks=true,pdftitle=doc\_studentreport.pdf
,linktoc=all,linkcolor=RedFire,urlcolor=ParksWildlife]{hyperref}
\usepackage{colortbl}
\usepackage{longtable}
\usepackage{tabu}
\setlength{\tabulinesep}{1.5mm}
\usepackage{enumerate}
\usepackage{enumitem}
\usepackage{fancyhdr}
\usepackage{lastpage}
\usepackage{graphicx}
\usepackage{eso-pic}
\usepackage{hyphenat}
\renewcommand{\familydefault}{\sfdefault}



\newcommand{\HRule}{\rule{\linewidth}{0.1pt}}

\newcommand{\placetextbox}[3]{% \placetextbox{<horizontal pos>}{<vertical pos>}{<stuff>}
  \setbox0=\hbox{#3}% Put <stuff> in a box
  \AddToShipoutPictureFG*{% Add <stuff> to current page foreground
    \put(\LenToUnit{#1\paperwidth},\LenToUnit{#2\paperheight}){\vtop{{\null}\makebox[0pt][c]{#3}}}%
  }%
}%




%-----------------------------------------------------------------------------%
% Headers and footers
%
\fancypagestyle{plain}{
  \fancyhf{}
  \setlength\headheight{60pt} % push page content below header
  \renewcommand{\headrulewidth}{0.1pt}
  \renewcommand{\footrulewidth}{0.1pt}
  
  
  \fancyhead[L]{ 
    \href{http://sdis.dpaw.wa.gov.au}{
    \includegraphics[scale=0.6]{/mnt/projects/sdis/staticfiles/img/logo-dpaw.png}}
  }
  \fancyhead[R]{ 
      \hfill
      \href{http://sdis.dpaw.wa.gov.au}{Science Directorate Information System} 
      \newline 
      \href{http://sdis.dpaw.wa.gov.au/documents/studentreport/1436/}{Progress Report 2014-8 (FY 2014-2015)} 
  }
  
  
  
  
  \fancyfoot[L]{ \leftmark\newline\textbf{Printed}\textit{ Sept. 10, 2015, 12:45 p.m. }}
  \fancyfoot[R]{  \, \newline Page \thepage\ of \pageref{LastPage} }
  
  
}
\pagestyle{plain}
%
% end Headers
%-----------------------------------------------------------------------------%

\begin{document}

%-----------------------------------------------------------------------------%
% Title page
%

%
% end title page
%-----------------------------------------------------------------------------%




\section*{Progress Report}
This project is currently evaluating how fauna translocations impact the
transmission of parasites in woylies (\emph{Bettongia penicillata}), and
what consequences this has for translocated hosts and other cohabiting
species (Brushtail possum - \emph{Trichosurus vulpecula}; Chuditch -
\emph{Dasyurus geoffroii}). We are testing the hypothesis that fauna
translocations lead to a higher diversity of parasites within the
resultant host-parasite community, and thus a higher incidence of
polyparasitism; which in conjunction with the disruption of established
host-parasite associations, may exacerbate the negative impacts of
parasites on their hosts to the detriment of translocation success.
Secondly, as the effects of anti-parasite treatment in translocated
hosts are relatively unknown; we are also assessing the effect of
parasite removal in translocated hosts. We are testing the hypothesis
that anti-parasite treatment reduces the incidence of polyparasitism,
thereby improving host fitness and survivability. In June 2014, 182
woylies were translocated from Perup Sanctuary to two unfenced sites
within Western Australia. In June 2015, an additional 69 woylies were
translocated into Dryandra Woodland; a second spatially independent
study site. Pre- and post-translocation, woylies from both the source
and destination sites were measured and weighed, and pouch activity was
recorded for females. Blood, ectoparasite and faecal samples were also
collected for parasitological examination. In each destination site,
cohabiting species were sampled to quantify parasite transmission
between species post-translocation. To evaluate the effect of
anti-parasite treatment, we treated half the woylies with Ivermectin
prior to translocation. We have observed changes to the predominant
species of \emph{Trypanosoma} in woylies pre- and post-translocation,
and that anti-parasite treatment has had an effect on both target and
non-target parasites of the translocated hosts.




\clearpage



\end{document}

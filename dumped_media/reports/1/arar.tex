\documentclass[version=last, paper=a4, DIV=18, usenames, dvipsnames]{scrartcl}
\usepackage{txfonts}
\usepackage{pdflscape}
\usepackage{pdfpages}
\usepackage[english]{babel} % English language/hyphenation
%%% Bootstrap colors
\definecolor{RedFire}{RGB}{146,25,28}
\definecolor{ParksWildlife}{RGB}{0,85,144}
\definecolor{successbg}{RGB}{223,240,216}
\definecolor{errorbg}{RGB}{242,222,222}
\definecolor{warningbg}{RGB}{252,248,227}
\definecolor{infobg}{RGB}{217,237,247}
\definecolor{muted}{RGB}{153,153,153}
\definecolor{success}{RGB}{70,136,71}
\definecolor{error}{RGB}{185,74,72}
\definecolor{warning}{RGB}{192,152,83}
\definecolor{info}{RGB}{58,135,173}
\usepackage[colorlinks=true,pdftitle=arar.pdf,linktoc=all,linkcolor=RedFire,urlcolor=ParksWildlife]{hyperref}
\usepackage{colortbl}
\usepackage{longtable}
\usepackage{tabu}
\setlength{\tabulinesep}{1.5mm}
\usepackage{enumerate}
\usepackage{enumitem}
\usepackage{fancyhdr}
\usepackage{lastpage}
\usepackage{graphicx}
\usepackage{eso-pic}
\usepackage{hyphenat}



%%% Custom headers/footers (fancyhdr package)
\fancypagestyle{plain}{
\fancyhf{}
\setlength\headheight{40pt}
\renewcommand{\headrulewidth}{0.1pt}
\renewcommand{\footrulewidth}{0.1pt}



    \fancyhead[L]{ \href{http://sdis.dpaw.wa.gov.au/pythia/ararreport/1/download/tex/}{} \newline }
\fancyhead[R]{ \hfill\href{http://www.dpaw.wa.gov.au}{Department of Parks and Wildlife}\newline\href{http://sdis.dpaw.wa.gov.au}{Pythia}}




\fancyfoot[L]{ \leftmark\newline\textbf{Last Modified}\textit{ }\quad\textbf{Printed}\textit{ Aug. 19, 2014, 1:32 p.m. } }
\fancyfoot[R]{  \, \newline Page \thepage\ of \pageref{LastPage} } % Pagenumbering


}
\pagestyle{plain}


\newcommand{\HRule}{\rule{\linewidth}{0.1pt}}

\newcommand{\placetextbox}[3]{% \placetextbox{<horizontal pos>}{<vertical pos>}{<stuff>}
  \setbox0=\hbox{#3}% Put <stuff> in a box
  \AddToShipoutPictureFG*{% Add <stuff> to current page foreground
    \put(\LenToUnit{#1\paperwidth},\LenToUnit{#2\paperheight}){\vtop{{\null}\makebox[0pt][c]{#3}}}%
  }%
}%

\begin{document}

\setcounter{secnumdepth}{-1}


\begin{titlepage}
\begin{center}
% Upper part of the page
\begin{minipage}[t]{0.28\textwidth}
\begin{flushleft}
\href{http://www.dpaw.wa.gov.au}{\includegraphics[scale=0.6]{/var/www/sdis_8271/staticfiles/img/logo-dpaw.png}}
\end{flushleft}
\end{minipage}
\begin{minipage}[b]{0.7\textwidth}
\begin{flushright}
    \href{http://sdis.dpaw.wa.gov.au/pythia/ararreport/1/download/tex/}{}) \\
\end{flushright}
\end{minipage}
\HRule \\[0.4cm]
\vfill
\textsc{\Huge ARAR Report}
\vfill
\textsc{\Huge Subtitle}

\vfill\vfill\vfill\vfill
title and summary

\vfill\vfill\vfill\vfill\vfill\vfill\vfill\vfill

\textbf{Version created on} Aug. 19, 2014, 1:32 p.m.
\vfill
\textbf{Last Modified on}  by 
\vfill\vfill
\textbf{Report Status}\\\,
\begin{tabu} to \linewidth { | X[l] | X | }
\hline
\rowcolor{infobg}
Status & Last Updated \\
\hline
\textbf{Planning - } \\
\hline
\end{tabu}
\vfill
\textbf{Science Project Overview}\\\,
\begin{tabu} to \linewidth { | X[l] | X | }
\hline
\rowcolor{infobg}
Part & Checklist Last Updated \\
\hline
\textbf{Part A - Summary \& Approval} & bla \\
\hline
\end{tabu}

\end{center}
\end{titlepage}

\setcounter{tocdepth}{2}
\tableofcontents
\clearpage



    
    

\section{Director's Message}



The Department has primary responsibility for protecting and conserving the state's natural assets on behalf of the people of Western Australia. Western Australia is a large state with unique biological diversity including internationally recognised terrestrial and marine biodiversity hotspots, 12 internationally significant wetlands and eight of the 15 national biodiversity hotspots. We have a large number of national parks and nature reserves that protect biodiversity and provide superb natural environments for community recreation. Integrated scientific and management expertise is essential in order to provide the depth of knowledge required to effectively manage these complex and diverse ecological systems. While knowledge in other sectors can be obtained from national and international spheres, the explicit knowledge required to manage Western Australia's unique environments and natural assets requires local expertise.


Investment in strategic research develops the understanding, capacity and innovation necessary to identify and respond to emerging issues before they develop into major environmental pressures that impinge on conservation and sustainable development, and have costly long-term impacts on the environment, the community and the economy. Environmental research is a critical activity that supports the Western Australian economy through the mining, agricultural, oil and gas, urban development and tourism industries. Scientific information is required to value the natural assets of the state and to effectively and efficiently manage those assets for the benefit of all Western Australians. Scientific knowledge creates tourism opportunities and enables community enjoyment of our unique natural environments while supporting management of these areas for future generations.


A coordinated science and research capacity integrated with policy and management functions is a strength of the department, and reflects the vital role of science and information in conserving the state's biodiversity and managing its natural assets. An expert science capability embedded in the
agency has significant benefits including:


\begin{itemize}

  \item end-user, risk-based priority setting and funding allocation processes that ensure investment in research is relevant to Western Australia, and that it is feasible, beneficial, cost effective and directly linked to government needs;

  \item rapid incorporation of up-to-date knowledge to support the department's immediate policy and operational requirements through effective and efficient technology transfer and research uptake;

  \item the ability to undertake vital strategic long term science essential for government decision making;

  \item availability of science based knowledge and information to provide accurate and timely scientific, technical and policy advice to government.

\end{itemize}


The department's internal research capacity is extended and leveraged through attracting external investment to address state priorities, and through strategic collaborations and partnerships with external research agencies and end-users where this provides access to relevant expertise.


There is a clear need for substantial levels of research for effective management of our natural assets, but the resources to carry out such research are limited. This reality necessitates rigorous priority setting and efficiency of operations. The department undertakes this through a risk-based process where research priorities are driven by government and legislative obligations and strategic planning processes. The immediate, broader and longer-term benefits of investment in science are maximised through a departmental Science Policy.


In 2007 the Science Division prepared A Strategic Plan for Biodiversity Conservation Research for the 10-year period 2008-2017. In early 2013 the Science Management Team reviewed progress on the first five years of the strategic plan. It is very pleasing to see major progress against actions identified under the five key strategic goals that cover (1) understanding the composition of and patterning in biodiversity, (2) understanding the threats to biodiversity and developing options for amelioration, (3) monitoring condition and trends in ecosystems, (4) providing scientific concepts and tools for best practice management, and (5) improving knowledge of how people respond to and interact with the natural environment.


I can only highlight here a few of the many achievements completed over the past five years. We have undertaken planning processes to guide terrestrial and marine biological survey, plant taxonomy, and plant molecular genetics research and survey to provide an informed basis for management decisions. We developed major understanding of the biology of the Pilbara region and Kimberley Islands through biological survey. We have undertaken a large number of plant and animal translocations to improve the conservation status of threatened species, including doubling the number of individuals of Gilbert's potoroo--Australia's rarest mammal--and establishing five mammals at the Lorna Glen Rangelands Restoration site. We have exceeded the global conservation targets for translocations of threatened plants and germplasm collections.


The FORESTCHECK monitoring program completed two rounds of five-yearly monitoring and a full analysis of the first round provides a baseline on which to assess change. We have completed investigations into the effects of timber harvesting activities on forest ecosystems as required under the Forest Management Plan 2004-2013.


We developed a departmental Science Policy that was endorsed by Corporate Executive in 2011 and guides science activities across the department. We established the Western Australian Conservation Science Centre and further developed our partnerships and collaborations with academia, industry and community groups. We are a foundation partner in the Western Australian Biodiversity Science Institute and have engagement with the Western Australian Marine Science Institution. A review of publications showed an 86\% increase over the five year period 2006-2011, demonstrating increased effectiveness in communication.


The main actions to achieve further progress in implementing the key strategic goals for the next five years from 2013-2017 have been articulated in a revision of the strategic plan.


The scientific work of the Division that has led to the achievements made over the past five years is a credit to all staff, including scientists and technical officers, and would not be possible without the efficient support of the administrative staff, the work centre managers and library staff. I thank all staff for their ongoing dedication to undertaking and supporting science and research to underpin conservation of Western Australia's unique natural assets. I also take this opportunity to honour the memory of past Director General, Keiran McNamara, who provided great leadership in the application of science to conservation policy and operation.


At the end of June 2013, the Department of Environment and Conservation ceased and the conservation aspects of our business will now continue with renewed focus under the strategic directions of the Department of Parks and Wildlife. I look forward to continuing to provide excellent world-class science and research to support the work of the new department in protecting, managing and enhancing Western Australia's natural assets.


Dr Margaret Byrne
Director Science
June 2013




\clearpage

    

\section{Vision}



We envisage a society where scientific enquiry is highly respected and forms an objective basis for environmental decision making and policy development.




\clearpage

    

\section{Focus and Purpose}



Provision of up-to-date and scientifically sound information to uphold effective conservation of biodiversity and sustainable natural resource management in Western Australia. We strive to provide excellence in science and technology based on internationally recognised best practice. We operate research centres that foster, promote and reward creativity and innovation.




\clearpage

    

\section{Role}



\begin{itemize}

  \item To provide a scientifically objective and independent source of reliable knowledge and
understanding about conserving species and ecological communities in Western Australia, managing the public lands and waters entrusted to the Department of Parks and Wildlife, and carrying out the Department's other legislative responsibilities.

  \item To ensure that Science and Conservation Division is responsive to the needs of policy makers and end users in the Department by bringing science to bear on the solution of the state's most pressing problems relating to conservation and land management.

  \item To work in partnership with the department's managers, research institutions and the broader community to increase knowledge underpinning conservation and land management in Western Australia.

  \item To advise the Department on sustainable resource development opportunities and to promote the conservation of biological resources through their sustainable utilisation.

  \item To communicate and transfer to managers in the Department, and to the broader community, knowledge, information and other insights obtained through scientific investigation in Western Australia and elsewhere.

  \item To attain a worldwide reputation for excellence in science by publishing knowledge obtained through scientific research in premier national and international scientific journals and through electronic means.

  \item To contribute, as an integrated part of the Department of Parks and Wildlife, to meeting the need for knowledge on conservation and land management matters by the public of Western Australia.

  \item To undertake and participate in astronomy research, information and education for the benefit of local, national and international communities.

\end{itemize}




\clearpage

    

\section{Publications and Reports}



Abbott I (2013) Mutual help: agricultural journalist William Catton Grasby of the Western Mail and his influence in Western Australia.\emph{Early Days}\textbf{14}, 107-126.


Abbott I (2014) Bird species in Kings Park: going, going?\emph{Western Australian Bird Notes} 26-27.


Algar D, Hamilton N, Nickels D, Flematti G (2013) Science, a weed and cat control.\emph{Landscope}\textbf{29(2)}, 59-61.


Algar D, Onus M, Hamilton N (2013) Feral cat control as part of rangelands restoration at Lorna Glen (Matuwa), Western Australia: the first seven years.\emph{Conservation Science, Western Australia}\textbf{8}, 367-381.


Andersen AN, Bocciarelli D, Fairman R, Radford IJ (2014) Conservation status of ants in an iconic region of monsoonal Australia: levels of endemism and responses to fire in the eastern Kimberley.\emph{Journal of Insect Conservation}\textbf{18}, 137-146.


Anderson BM, Barrett MD, Thiele K, Grierson P, Krauss S (2013) Taxon delimitation in the \emph{Triodia} \emph{basedowii} E.Pritz. species complex (ABSTRACT).In_ Systematics without Borders: 1-6 December, 2013, Sydney, Australia_ p. 8.


Ashworth EC, Depczynski M, Holmes TH, Wilson SK (2014) Quantitative diet analysis of four mesopredators from a coral reef.\emph{Journal of Fish Biology}\textbf{84}, 1031-1045.


Bain K, Wayne A, Bencini R (2014) Overcoming the challenges of measuring the abundance of a cryptic macropod: is a qualitative approach good enough?\emph{Wildlife Research}\textbf{41}, 84-93.


Baldock J, Bancroft KP, Williams M, Shedrawi G, Field S (2014) Accurately estimating local water temperature from remotely sensed satellite sea surface temperature: a near real-time monitoring tool for marine protected areas.\emph{Ocean and Coastal Management}\textbf{96}, 73-81.


Bancroft KP (ed) (2013) Western Australian Marine Monitoring Program: annual marine protected area biodiversity assets and social values report: Jurien Bay Marine Park 2012-2013. Department of Parks and Wildlife, Kensington, WA. 150 p.


Bancroft KP (ed) (2013) Western Australian Marine Monitoring Program: annual marine protected area biodiversity assets and social values report: Marmion Marine Park 2012-2013. Department of Parks and Wildlife, Kensington, WA. 171 p.


Bancroft KP (2013) Western Australian Marine Monitoring Program: annual marine protected area biodiversity assets and social values report: Montebello Islands Marine Park, Barrow Island Marine Park and Barrow Island Marine Management Area, 2012-2013. Department of Parks and Wildlife, Kensington, WA. 218 p.


Bancroft KP (ed) (2013) Western Australian Marine Monitoring Program: annual marine protected area biodiversity assets and social values report: Ningaloo Marine Park and Muiron Islands Marine Management Area annual report, 2012-2013. Department of Parks and Wildlife, Kensington, WA. 278 p.


Bancroft KP (ed) (2013) Western Australian Marine Monitoring Program: annual marine protected area biodiversity assets and social values report: Rowley Shoals Marine Park, 2012-2013. Department of Parks and Wildlife, Kensington, WA. 136 p.


Bancroft KP (ed) (2013) Western Australian Marine Monitoring Program: annual marine protected area biodiversity assets and social values report: Shark Bay Marine Park and Hamelin Pool Marine Nature Reserve, 2012-2013. Department of Parks and Wildlife, Kensington, WA. 228 p.


Bancroft KP (ed) (2013) Western Australian Marine Monitoring Program: annual marine protected area biodiversity assets and social values report: Shoalwater Islands Marine Park, 2012-2013. Department of Parks and Wildlife, Kensington, WA. 175 p.


Bancroft KP (ed) (2013) Western Australian Marine Monitoring Program: annual marine protected area biodiversity assets and social values report: Swan Estuary Marine Park, 2012-2013. Department of Parks and Wildlife, Kensington, WA. 44 p.


Bancroft KP (ed) (2013) Western Australian Marine Monitoring Program: annual marine protected area biodiversity assets and social values report: Walpole and Nornalup Inlet Marine Park, 2012-2013. Department of Parks and Wildlife, Kensington, WA. 139 p.


Barker WR, Breitwieser I, Thiele KR (2013) Plant systematics in Australasia: is it time (again) for a good, hard look? (ABSTRACT).In_ Systematics without Borders: 1-6 December, 2013, Sydney, Australia_ p. 105.


Barrett RL (2014) \emph{Ammannia} \emph{fitzgeraldii}, a \emph{nom. nov.} for \emph{Nesaea} \emph{repens} (Lythraceae).\emph{Nuytsia}\textbf{24}, 101-102.


Barrett RL, Lowrie A (2013) Typification and application of names in \emph{Drosera} section \emph{Arachnopus} (Droseraceae).\emph{Nuytsia}\textbf{23}, 527-541.


Belton GS, Prud'Homme van Reine WH, Huisman JM, Draisma SGA, Gurgel CFD (2014) Resolving phenotypic plasticity and species designation in the morphologically challenging \emph{Caulerpa} \emph{racemosa}-\emph{peltata} complex (Caulerpaceae: Chlorophyta).\emph{Journal of Phycology}\textbf{50}, 32-54.


Berry O, Tatler J, Hamilton N, Hilmer S, Hitchen Y, Algar D (2013) Slow recruitment in a red fox population following poison baiting: a non-invasive mark-recapture analysis.\emph{Wildlife Research}\textbf{40}, 615-623.


Biggs LJ, Parker CL (2013) Updates to Western Australia's vascular plant census for 2012.\emph{Nuytsia}\textbf{23}, 503-526.


Binks RM, Gardner MG, Millar MA, Byrne M (2014) Characterization and cross-amplification of novel microsatellite markers for two Australian sedges, \emph{Lepidosperma} sp. Mt Caudan and \emph{L}. sp. Parker Range (Cyperaceae).\emph{Conservation Genetics Resources}\textbf{6}, 333-336.


Blythman M (2013) Lorna Glen bird banding expedition, April 2013.\emph{Western Australian Bird Notes}\textbf{147}, p. 20.


Bode M, Brennan KEC, Helmstedt K, Desmond A, Smia R, Algar D (2013) Interior fences can reduce cost and uncertainty when eradicating invasive species from large islands.\emph{Methods in Ecology and Evolution}\textbf{4}, 819-827.


Braaten CC, Matheny PB, Viess DL, Wood MG, Williams JH, Bougher NL (2014) Two new species of \emph{Inocybe} from Australia and North America that include novel secotioid forms.\emph{Botany}\textbf{92}, 9-22.


Brown A, Keighery G, Thomson-Dans C (2013) \emph{Common wildflowers of the south-west forests}. Department of Parks and Wildlife, Western Australia, Bush Books Department of Parks and Wildlife, Kensington, WA. 72 p.


Burbidge A, Joseph L (2013) The world's most mysterious bird: what's going on?\emph{Western Australian Bird Notes}\textbf{147}, p. 24.


Burrows ND (2013) Measuring mulga biomass at Lorna Glen: a brief report on results of a field trip 1-4 October 2014. Department of Parks and Wildlife, Kensington, WA. 9 p.


Burrows N (2014) Wild cat breach, Lorna Glen compound, chronology. Department of Parks and Wildlife, Kensington, WA. 8 p.


Burrows N, Liddelow G (2013) Breach of the Lorna Glen fauna refuge compound by a feral cat: chronology and lessons learnt. Department of Parks and Wildlife, Kensington, WA. 8 p.


Burrows N, Liddelow G (2013) Lorna Glen introduced predator control: post-bait survey report 7-10 August 2013. Department of Parks and Wildlife, Kensington, WA. 7 p.


Burrows ND, Liddelow GL, Ward B (2014) A guide to estimating fire rate of spread in spinifex grasslands (Mk2). Department of Parks and Wildlife, Kensington, WA. 6 p.


Burrows N, McCaw L (2013) Prescribed burning in southwestern Australian forests.\emph{Frontiers in Ecology and the Environment}\textbf{11}, e25-e34.


Burrows N, Pawero M (2013) Measuring mulga biomass at Lorna Glen.\emph{Kantri Laif: North Australian Indigenous Land and Sea Management News}\textbf{8}, 8-9.


Butcher R, Thiele K, Tapper S-L, Byrne M (2013) The more you look, the more you find: morphology and molecules identify rare and range-restricted \emph{Synaphea} R.Br. (Proteaceae) taxa from Western Australia's Swan Coastal Plain (ABSTRACT).\emph{In: Systematics without Borders: 1-6 December, 2013, Sydney, Australia} p. 84.


Cale DJ (2013) Wetland Biodiversity Monitoring Program: Towerrining Lake, fauna and water chemistry datasets. Department of Parks and Wildlife, Woodvale, WA. 8 p.

Carpenter F, Dziminski M (2014) A portal to the Pilbara's threatened fauna. Department of Parks and Wildlife, Western Australia, Information Sheet\textbf{75/2014}, DPaW, Kensington, WA. 2 p. Available at: \href{http://www.dpaw.wa.gov.au/about-us/science-and-research/publications-resources/111-science-division-information-sheets}{http://www.dpaw.wa.gov.au/about-us/science-and-research/publications-resources/111-science-division-information-sheets}.


Carwardine J, Nicol S, van Leeuwen S, Walters B, Firn J, Reeson A \emph{et al.} (2014) \emph{Priority threat management for Pilbara species of conservation significance}. CSIRO Ecosystems Sciences, Brisbane. 72 p.


Chapman TF (2013) Relic bilby (\emph{Macrotis} \emph{lagotis}) refuge burrows: assessment of potential contribution to a rangeland restoration program.\emph{Rangeland Journal}\textbf{35}, 167-180.


Chapman T, King R (2014) Going nuts for boodies.\emph{Landscope}\textbf{29(3)}, 37-39.


Cherriman S, Morris K (2013) Nest sites, breeding, satellite telemetry and diet of the wedge-tailed eagle \emph{Aquila} \emph{audax} at Lorna Glen, Western Australia. Department of Parks and Wildlife, Woodvale, WA. 26 p.


Clarke J, Warren K, Calver M, de Tores P, Mills J, Robertson I (2013) Hematologic and serum biochemical reference ranges and an assessment of exposure to infectious diseases prior to translocation of the threatened western ringtail possum (\emph{Pseudocheirus} \emph{occidentalis}).\emph{Journal of Wildlife Diseases}\textbf{49}, 831-840.


Cochrane A, Crawford A, Coates D (2014) Addressing Target 8 of the Global Strategy for Plant Conservation in Western Australia.\emph{Australasian Plant Conservation}\textbf{22(4)}, 13-15.


Coker DJ, Wilson SK, Pratchett MS (2014) Importance of live coral habitat for reef fishes.\emph{Reviews in Fish Biology and Fisheries}\textbf{24}, 89-126.


Comer S, Burbidge A, Algar D, Berryman A, Bondin A (2013) Kyloring, cats and conservation: the race to save the western ground parrot.\emph{Landscope}\textbf{29(2)}, 40-45.


Cook A, Lees J (2013) Spotting quolls in the Pilbara.\emph{Landscope}\textbf{29(1)}, 20-26.


Cook A, Morris K (2013) Northern quoll survey and monitoring project: Pilbara region of Western Australia. Department of Parks and Wildlife, Kensington, WA. 7 p.


Coughran DK, Gales NJ, Smith HC (2013) A spike in recorded mortality of humpback whales (\emph{Megaptera} \emph{novaeangliae}) in Western Australia.\emph{Journal of Cetacean Research and Management}\textbf{13}, 105-108.


Cowen S, Burbidge AH, Comer SJ (2014) From refuge to resource: Bald Island and the noisy scrub-bird (ABSTRACT).In \emph{Island Arks Symposium III: Program \& Guide} 25-26.


Cruz J, Sutherland DR, Anderson DP, Glen AS, de Tores P, Leung LK-P. (2013) Antipredator responses of koomal (\emph{Trichosurus} \emph{vulpecula} \emph{hypoleucus}) against introduced and native predators.\emph{Behavioral Ecology and Sociobiology}\textbf{67}, 1329-1338.


Cruz MG, Sullivan AL, Leonard R, Malkin S, Matthews S, Gould JS et al. [McCaw WL] (2014) Fire behaviour knowledge in Australia: a synthesis of disciplinary and stakeholder knowledge on fire spread prediction capability and application. CSIRO, Canberra. 182 p.


Davis J, Sim L, Pinder A, Murphy N, Brim Box J, Sheldon F \emph{et al.} (2014) Landscape-scale patterns in the diversity and distribution of invertebrate communities of temporary aquatic habitats across arid Australia. Paper presented at the Joint Aquatic Sciences Meeting, Portland, Oregon, 18-23 May 2014 (ABSTRACT). p. 1.


Davis J, Sunnucks P, Thompson R, Sim L, Pavlova A, Moran-Ordoñez A \emph{et al.} [Pinder A] (2013) \emph{Climate change adaptation guidelines for arid zone aquatic ecosystems and freshwater biodiversity}. National Climate Change Adaptation Research Facility, Gold Coast. 60 p.


Davis RW, Hammer TA, Thiele KA (2014) Two new and rare species of \emph{Ptilotus} (Amaranthaceae) from the Eneabba sandplains, Western Australia.\emph{Nuytsia}\textbf{24}, 123-129.


Davis R, Wege J (2013) A targeted flora survey of the Naturebank envelope in Millstream Chichester National Park. Department of Parks and Wildlife, Kensington, WA. 10 p.


Davison EM, Davison PJN, Barrett MD, Barrett RL (2013) \emph{Licea} \emph{xanthospora} E.M. Davison, P.J.N. Davison, M.D. Barrett \& R.L. Barrett, \emph{sp. nov. Persoonia}\textbf{31}, 278-279.


Davison EM, McGurk LE, Bougher NL, Syme K, Watkin ELJ (2013) \emph{Amanita} \emph{lesueuri} and \emph{Amanita} \emph{wadjukorum} (Basidiomycota), two new species from Western Australia, and an expanded description of \emph{Amanita} \emph{fibrillopes}.\emph{Nuytsia}\textbf{23}, 589-606.


Dillon SJ (2014) \emph{Grevillea} \emph{saxicola} (Proteaceae), a new species from the Pilbara of Western Australia.\emph{Nuytsia}\textbf{24}, 103-108.


Dixon RRM, Mattio L, Huisman JM, Payri CE, Bolton JJ, Gurgel CFD (2014) North meets south: taxonomic and biogeographic implications of a phylogenetic assessment of \emph{Sargassum} subgenera \emph{Arthrophycus} and \emph{Bactrophycus} (Fucales: Phaeophyceae).\emph{Phycologia}\textbf{53}, 15-22.


Doherty T, Davis R, van Etten E, Algar D, Collier N, Dickman CR \emph{et al.} [Palmer R] (2014) What's for dinner?: a continental-scale analysis of feral cat diet in Australia (ABSTRACT).In \emph{Australian Mammal Society, 60th Scientific Meeting Conference Handbook: 7th-10th July 2014, Melbourne, Melbourne Zoo} p. 111.


Drake PL, McCormick CA, Smith MJ (2014) Controls of soil respiration in a salinity-affected ephemeral wetland.\emph{Geoderma}\textbf{221/222}, 96-102.


Dunlop J, Lees J (2014) Northern quoll, \emph{Dasyurus} \emph{hallucatus} (POSTER). Department of Parks and Wildlife, Kensington, WA.


Dziminski M (2013) Progress report 2013: the conservation and management of the bilby (\emph{Macrotis} \emph{lagotis}) in the Pilbara. Department of Parks and Wildlife, Woodvale, WA. 7 p.


Dziminski M, Cowan M (2014) Bilby, \emph{Macrotis} \emph{lagotis} (POSTER). Department of Parks and Wildlife, Kensington, WA. 1 poster.


Elliott TF, Bougher NL, O'Donnell K, Trappe JM (2014) \emph{Morchella} \emph{australiana} \emph{sp. nov.,} an apparent Australian endemic from New South Wales and Victoria.\emph{Mycologia}\textbf{106}, 115-118.


Evans RD, Wilson SK, Field SN, Moore JAY (2014) Importance of macroalgal fields as coral reef fish nursery habitat in north-west Australia.\emph{Marine Biology}\textbf{161}, 599-607.


Farr J (2013) Notes for \emph{Cardiaspina} \emph{fiscella} (the brown lace lerp psyllid, Hemiptera: Psyllidae) on eucalypts in Western Australia. Department of Parks and Wildlife, Kensington, WA. 4 p.


Farr JD, Wills AJ (2014) Common insect pests of Western Australia's southern jarrah forests (POSTER). Department of Parks and Wildlife, Manjimup. 1 p.


Frank A, Johnson C, Legge S, Collis M-A, Fisher A, Lawes M \emph{et al.} [Radford I] (2014) Repeated evidence: feral cats prohibit establishment of reintroduced small native mammals in northern Australia (ABSTRACT).In \emph{Australian Mammal Society, 60th Scientific Meeting Conference Handbook: 7th-10th July 2014, Melbourne, Melbourne Zoo} p. 91.


Friend T (2014) Eradicating the house mouse from Boullanger and Whitlock Islands, Western Australia: challenges and solutions (ABSTRACT).In \emph{Island Arks Symposium III: Program \& Guide} p. 34.


Friend T, Bougher N, Button T, Hill S (2013) Use of pioneer animals to assess reintroduction sites in the recovery of the world's rarest marsupial, Gilbert's potoroo (ABSTRACT).In \emph{11th International Mammalogical Congress: Queen's University of Belfast, 11th-16th August 2013: Programme and Abstracts} p. 57.


Friend JA, Hill RF, Mosen CC (2013) Baiting feral cats in conservation areas within an agricultural landscape: challenges and possible solutions (ABSTRACT).\emph{Australian Mammal Society Newsletter}\textbf{Nov}, 28-29.


Friend JA, Mosen C, Button TA (2014) Can manipulation of fox control reveal fox-feral cat interactions? (ABSTRACT).In \emph{Australian Mammal Society, 60th Scientific Meeting Conference Handbook: 7th-10th July 2014, Melbourne, Melbourne Zoo} p. 71.


Fulton C, Depczynski M, Radford B, Wilson S, Holmes T (2014) Seasonal cycles in Ningaloo seaweed meadows. Department of Parks and Wildlife, Western Australia, Information Sheet\textbf{73/2014}, DPaW, Kensington, WA. 2 p. Available at: http://www.dpaw.wa.gov.au/about-us/science-and-research/publications-resources/111-science-division-information-sheets.


Gardner AG, Shepherd KA, Howarth DG, Jabaily RS (2014) The Australian plant family Goodeniaceae as a new model system for floral symmetry evolution (ABSTRACT).In \emph{Association of Southeastern Biologists, 75th Annual Meeting, April 2-5, 2014: Abstracts for Presentations} p. 17.


Gari NM, Conran JG, Macfarlane TD, Waycott M (2013) Leaf micromorphological variations and phylogeny in the Australasian genus \emph{Patersonia} (Iridaceae) (ABSTRACT).In_ Systematics without Borders: 1-6 December, 2013, Sydney, Australia_ p. 120


Gibson LA (2014) Biogeographic patterns on Kimberley islands, Western Australia.\emph{Records of the Western Australian Museum, Supplement}\textbf{81}, 245-280.


Gibson L (2014) Biogeographic patterns on Kimberley islands, Western Australia (ABSTRACT).In \emph{Island Arks Symposium III: Program \& Guide} p. 38.


Gibson N (2014) An amphibious \emph{Goodenia} (Goodeniaceae) from an ephemeral arid zone wetland.\emph{Nuytsia}\textbf{24}, 23-28.


Gosper CR, Prober SM, Yates CJ (2013) Multi-century changes in vegetation structure and fuel availability in fire-sensitive eucalypt woodlands.\emph{Forest Ecology and Management}\textbf{310}, 102-109.


Gosper C, Prober S, Yates C (2013) The role of fire in plant conservation in wheatbelt remnants.\emph{Australasian Plant Conservation}\textbf{22}, 21-22.


Gosper CR, Prober SM, Yates CJ (2014) Multi-century changes in vegetation structure and fuel availability in fire-sensitive eucalypt woodlands. Department of Parks and Wildlife, Western Australia, Information Sheet\textbf{72/2014}, DPaW, Kensington, WA. 2 p. Available at: http://www.dpaw.wa.gov.au/about-us/science-and-research/publications-resources/111-science-division-information-sheets.


Gosper CR, Yates CJ, Prober SM (2013) Floristic diversity in fire-sensitive eucalypt woodlands show a U-shaped relationship with time since fire.\emph{Journal of Applied Ecology}\textbf{50}, 1187-1196.


Gosper CR, Yates CJ, Prober SM, Wiehl G (2014) Application and validation of visual fuel hazard assessments in dry Mediterranean-climate woodlands.\emph{International Journal of Wildland Fire}\textbf{23}, 385-393.


Groom CJ, Coughran D, Smith HC (2014) Records of beaked whales (family Ziphiidae) in Western Australian waters.\emph{Marine Biodiversity Records}\textbf{7}, 1-13, e50.


Hammer T, Thiele K, Davis R, Motley T (2013) Preliminary ITS phylogeny of the Australian genus \emph{Ptilotus} (Amaranthaceae). Poster presented at Botany 2013: Celebrating Diversity, July 27-31, New Orleans.


Hassell NS, Williamson DH, Evans RD, Russ GR (2013) Reliability of non-expert observer estimates of the magnitude of marine reserve effects.\emph{Coastal Management}\textbf{41}, 361-380.


Hislop M (2013) A taxonomic update of \emph{Conostephium} (Ericaceae: Styphelioideae: Styphelieae).\emph{Nuytsia}\textbf{23}, 313-335.


Hislop M (2014) New species from the \emph{Leucopogon} \emph{pulchellus} group (Ericaceae: Styphelioideae: Styphelieae).\emph{Nuytsia}\textbf{24}, 71-93.


Hislop M, Wege JA, Webb AD (2014) Description of \emph{Gastrolobium} \emph{argyrotrichum} (Fabaceae: Mirbelieae), with taxonomic notes on some other species with bicoloured calyx hairs.\emph{Nuytsia}\textbf{24}, 113-122.


Hitchcock MJ, Radford IJ, Wintle BA, Murphy BP (2014) Arresting the declines of arboreal mammals in the Kimberley (ABSTRACT).In \emph{Australian Mammal Society, 60th Scientific Meeting Conference Handbook: 7th-10th July 2014, Melbourne, Melbourne Zoo} p. 146.


Holley D, Friedman K, Reinhold L, Cowell T (2013) Implementation of a camera monitoring program for the loggerhead turtle (\emph{Caretta} \emph{caretta}) rookery at Turtle Bay, Dirk Hartog Island, Shark Bay (ABSTRACT).In \emph{Proceedings of the First Western Australian Marine Turtle Symposium, 28-29th August 2012} (eds B Prince, S Whiting, H Raudino \emph{et al.}), p. 11.


Holmes TH, Wilson SK, Travers MJ, Langlois TJ, Evans RD, Moore G \emph{et al.} [Douglas RA, Shedrawi G] (2013) A comparison of visual- and stereo-video based fish community assessment methods in tropical and temperate marine waters of Western Australia.\emph{Limnology and Oceanography: Methods}\textbf{11}, 337-350.


Huisman J (2014) An illustrated guide to the world's southern temperate seagrasses (BOOK REVIEW).\emph{Australian Systematic Botany Newsletter}\textbf{158}, 23-24.


Huisman J (2014) A standard text on the ecology of temperate Australian reefs (BOOK REVIEW).\emph{Australian Systematic Botany Newsletter}\textbf{158}, 24-26.


Huisman J, Prideaux C (2014) Heaven sent?\emph{Landscope}\textbf{29(3)}, 46-49.


Iles WLD, Lee C, Sokoloff DD, Remizowa MV, Yadav SR, Barrett MD \emph{et al.} [Barrett RL, Macfarlane TD] (2014) Reconstructing the age and historical biogeography of the ancient flowering-plant family Hydatellaceae (Nymphaeales).\emph{BMC Evolutionary Biology}\textbf{14}, 1-10, e102.


Johnson E, Shepherd K, Jabaily R (2013) Phylogenetic analysis and species delimitation of \emph{Anthotium} (Goodeniaceae), a charismatic clade of Australian wildflowers (POSTER ABSTRACT).In \emph{Botany 2013: Celebrating Diversity, July 27-31, New Orleans} 1-2.


Keighery G (2013) Tootanellup Nature Reserve: flora and vegetation. Department of Parks and Wildlife, Kensington, WA. 16 p.


Keighery G (2013) The urban forest: trees in our backyard and beyond: forum reports.\emph{Australian Garden History Journal}\textbf{25(1)}, 34-35.


Keighery G (2014) Protecting the Kimberley's unique flora.\emph{Landscope}\textbf{29(4)}, 8-13.


Keighery G, Keighery B (2013) Elachbutting Reserve vegetation and flora 2013: a preliminary report, Shire of Westonia, Western Australia. Department of Parks and Wildlife, Kensington, WA. 13 p.


Keighery G, Keighery B (2013) Native plants of the Jolimont Primary School bushland: a report for the City of Subiaco. Department of Parks and Wildlife, Kensington, WA. 4 p.


Keighery G, Keighery B (2013) Vegetation and flora of Blackboy Ridge Reserve, Shire of Chittering, Western Australia. Department of Parks and Wildlife, Kensington, WA. 36 p.


Kendrick A, Rule M (2014) The last lighthouse.\emph{Landscope}\textbf{29(4)}, 46-48.


Kendrick A, Rule M, Huisman J (2013) Cities in the sand: benthic invertebrates of Walpole and Nornalup Inlets Marine Park.\emph{Landscope}\textbf{29(2)}, 16-23.


Kenneally KF, Lowrie A, Keighery G (2013) Tropical triggerplants.\emph{Australian Plants}\textbf{27}, 64-75.


Krzyszczyk E, Kopps AM, Bacher K, Smith H, Stephens N, Meighan NA \emph{et al.} (2013) A report on six cases of seagrass-associated gastric impaction in bottlenose dolphins (\emph{Tursiops} sp.).\emph{Marine Mammal Science}\textbf{29}, 548-554.


Lane JAK, Clarke AG, Winchcombe YC (2013) South west wetlands monitoring program report, 1977-2012. Department of Parks and Wildlife, Busselton. 172 p.


Lewington M (2013) Clarification of the locality of William Blackall's collection of the threatened species \emph{Grevillea} \emph{phanerophlebia} (Proteaceae).\emph{Nuytsia}\textbf{23}, p. 477.


Lin S-M, Huisman JM, Ballantine DL (2014) Revisiting the systematics of \emph{Ganonema} (Liagoraceae: Rhodophyta) with emphasis on species from the northwest Pacific Ocean.\emph{Phycologia}\textbf{53}, 37-51.


Lizamore J, Simons J, Davies S, Pinder A, Vogwill R (2013) Managing and adapting to secondary salinity and altered hydrology in a Ramsar listed lake suite: Lake Warden wetland system case study. Paper presented at the WA State Coastal Conference, Balancing communities and coasts, Esperance, 2013 (ABSTRACT). 1-2.


Llorens T (2014) Shape matters too!\emph{Western Wildlife: Newsletter of the Land for Wildlife Scheme}\textbf{18(2)}, 1-2.


Lohr C (2013) 100 years of biodiversity data put to work.\emph{Landscope}\textbf{29(1)}, 6-8.


Lohr C, van Dongen R, Huntley B, Gibson L, Morris K (2014) Vegetation change on the Montebello islands before and after rodent eradications and native fauna reintroductions (ABSTRACT).In \emph{Island Arks Symposium III: Program \& Guide} 41-42.


Lyons MN, Keighery GJ, Gibson LA, Handasyde T (2014) Flora and vegetation communities of selected islands off the Kimberley coast of Western Australia.\emph{Records of the Western Australian Museum, Supplement}\textbf{81}, 205-243.


Maslin BR (2013) \emph{Acacia} \emph{gibsonii}, a distinctive, rare new species of Acacia sect. Juliflorae (Fabaceae: Mimosoideae) from south-west Western Australia.\emph{Nuytsia}\textbf{23}, 277-281.


Maslin B (2013) Wattle key updated.\emph{Australian Systematic Botany Society Newsletter}\textbf{155}, 33-34.


Maslin BR (2014) Miscellaneous new species of \emph{Acacia} (Fabaceae: Mimosoideae) from south-west Western Australia.\emph{Nuytsia}\textbf{24}, 139-159.


Maslin B (2014) Two new species of \emph{Acacia} (Fabaceae: Mimosoideae) with conservation significance from banded iron formation ranges in the vicinity of Koolyanobbing, Western Australia.\emph{Nuytsia}\textbf{24}, 131-138.


Maslin BR (2014) \emph{Vachellia} \emph{bolei} (Fabaceae: Mimosoideae), the correct name for a species from India.\emph{Nuytsia}\textbf{24}, p. 21.


Maslin BR, Barrett MD, Barrett RL (2013) A baker's dozen of new wattles highlights significant \emph{Acacia} (Fabaceae: Mimosoideae) diversity and endemism in the north-west Kimberley region of Western Australia.\emph{Nuytsia}\textbf{23}, 543-587.


Maslin BR, Cowie ID (2014) \emph{Acacia} \emph{equisetifolia}, a rare, new species of \emph{Acacia} sect. \emph{Lycopodiifoliae} (Fabaceae: Mimosoideae) from the top end of Northern Territory.\emph{Nuytsia}\textbf{24}, 1-5.


May T, Stefani F, Robinson R (2012) Comparing morphological species as used in ecological surveys against molecular barcoding with the ITS region: a case study of \emph{Cortinarius} in the south-west of Western Australia. Poster presented at the Australian Mycological Society Conference, 26-28 September 2012, Cairns, Qld.


McLean EH, Prober SM, Stock WD, Steane DA, Potts BM, Vaillancourt RE \emph{et al.} [Byrne M] (2014) Plasticity of functional traits varies clinally along a rainfall gradient in \emph{Eucalyptus} \emph{tricarpa}.\emph{Plant, Cell and Environment}\textbf{37}, 1440-1451.


Meissner RA, Coppen R (2013) Flora and vegetation of the greenstone ranges of the Yilgarn Craton: Credo Station.\emph{Conservation Science Western Australia}\textbf{8}, 333-343.


Meissner R, Rathbone D, Wilkins CF (2014) \emph{Lasiopetalum} \emph{adenotrichum} (Malvaceae s.l.), a new species from Fitzgerald River National Park.\emph{Nuytsia}\textbf{24}, 65-69.


Menz JF, Gardner AG, Shepherd KA, Willis S, Jabaily RS (2014) Getting into shape: morphometric analysis of floral symmetry variation in Goodeniaceae (POSTER ABSTRACT).In \emph{Association of Southeastern Biologists, 75th Annual Meeting, April 2-5, 2014: Abstracts for Presentations} p. 99.


Millar MA, Byrne M, Coates DJ, Roberts JD (2014) Characterisation of microsatellite DNA markers for \emph{Grevillea} \emph{paradoxa} (F. Muell.).\emph{Conservation Genetics Resources}\textbf{6}, 139-141.


Millar MA, Byrne M, Coates DJ, Roberts JD (2014) Characterisation of microsatellite DNA markers for the wiry honey myrtle, \emph{Melaleuca} \emph{nematophylla} Craven.\emph{Conservation Genetics Resources}\textbf{6}, 439-441.


Millar MA, Coates DJ, Byrne M (2013) Genetic connectivity and diversity in inselberg populations of \emph{Acacia} \emph{woodmaniorum}, a rare endemic of the Yilgarn Craton banded iron formations.\emph{Heredity}\textbf{111}, 437-444.


Minton C, Connor M, Price D, Jessop R, Collins P, Sitters H \emph{et al.} [Pearson G] (2013) Wader numbers and distribution on Eighty Mile Beach, north-west Australia: baseline counts for the period 1981-2003.\emph{Conservation Science Western Australia}\textbf{8}, 345-366.


Mohring MB, Kendrick GA, Wemberg T, Rule MJ, Vanderklift MA (2013) Environmental influences on kelp performance across the reproductive period: an ecological trade-off between gametophyte survival and growth?\emph{PLoS One}\textbf{8}, 1-10, e65310.


Mohring M, Rule M (2013) Long-term trends in the condition of seagrass meadows in Cockburn and Warnbro sounds: technical report to the Cockburn Sound Management Council. Department of Parks and Wildlife, Kensington, WA. 94 p.


Mohring MB, Rule MJ (2014) A survey of selected seagrass meadows in Cockburn Sound, Owen Anchorage and Warnbro Sound. Department of Parks and Wildlife, Kensington, WA. 65 p.


Morgan BR, Shearer BL (2013) Soil type and season mediated \emph{Phytophthora} \emph{cinnamomi} sporangium formation and zoospore release.\emph{Australasian Plant Pathology}\textbf{42}, 477-483.


Morris K (2014) The ecological restoration of Dirk Hartog Island (ABSTRACT).In \emph{Island Arks Symposium III: Program \& Guide} 32-33.


Morris K,. Page M, Renwick J, Desmond A (2013) Forty years of fauna translocations in Western Australia: lessons learned (ABSTRACT).In \emph{11th International Mammalogical Congress: Queen's University of Belfast, 11th-16th August 2013: Programme and Abstracts} p. 56.


Moskovitz N, Endicott T, Lister T, Ryan B, Ryan E, Willman M \emph{et al.} [Verveer A] (2013) The near-Earth flyby of asteroid 2012 DA14 (ABSTRACT).In \emph{Abstract Book 2013: correlating to the online Bulletin of the American Astronomical Society, vol. 45, no. 9: 45th Annual Meeting, Division of Planetary Sciences, Denver Colorado, 6-11 October 2013} p. 4.


Moskovitz N, Endicott T, Lister T, Ryan B, Ryan E, Willman M \emph{et al.} [Verveer A] (2013) The near-Earth flyby of asteroid 2012 DA14 (invited) (ABSTRACT).In \emph{AGU Abstract Browser. http://abstractsearch.agu.org/meetings/2013/FM.html} 1-2.


Myburg AA, Grattapaglia D, Tuskan GA, Hellsten U, Hayes RD, Grimwood J \emph{et al.} [Byrne, M.] (2014) The genome of \emph{Eucalyptus} \emph{grandis}.\emph{Nature}\textbf{510}, 356-362.


Nistelberger H, Byrne M, Coates D, Roberts JD (2014) Strong phylogeographic structure in a millipede indicates Pleistocene vicariance between populations on banded iron formations in semi-arid Australia.\emph{PLoS One}\textbf{9(3)}, 1-9, e93038.


Nistelberger H, Coates D, Byrne M, Roberts JD (2014) Isolation and characterization of 11 microsatellite loci in the short-range endemic shrub \emph{Grevillea} \emph{georgeana} McGill (Proteaceae).\emph{Conservation Genetics Resources}\textbf{6}, 221-222.


Nistelberger H, Roberts JD, Coates D, Byrne M (2013) Isolation and characterisation of 14 microsatellite loci from a short-range endemic, Western Australian tree, \emph{Banksia} \emph{arborea} (C.A.Gardner).\emph{Conservation Genetics Resources}\textbf{5}, 1143-1145.


Obbens F (2014) \emph{Calandrinia} sp. Mt Bruce (M.E.Trudgen 1544) cannot be maintained as distinct from \emph{C. pumila}.\emph{Nuytsia}\textbf{24}, 109-111.


Obbens FJ (2014) Two new species of \emph{Calandrinia} (Portulacaceae) from southern Western Australia.\emph{Nuytsia}\textbf{24}, 37-43.


Ottewell K (2013) it's all in the genes: quenda genetic diversity in the spotlight.\emph{Watsnu}\textbf{19(2)}, 2-3.


Ottewell K, Dunlop J, Thomas N, Morris K, Coates D, Byrne M (2014) Evaluating success of translocations in maintaining genetic diversity in a threatened mammal.\emph{Biological Conservation}\textbf{171}, 209-219.


Pacioni C, Johansen CA, Mahony TJ, O'Dea MA, Robertson ID, Wayne AF \emph{et al.} (2013) A virological investigation into declining woylie populations.\emph{Australian Journal of Zoology}\textbf{61}, 446-453.


Pacioni C, Robertson ID, Maxwell M, van Weenen J, Wayne AF (2013) Hematologic characteristics of the woylie (\emph{Bettongia} \emph{penicillata} \emph{ogilbyi}).\emph{Journal of Wildlife Disease}\textbf{49}, 816-830.


Pakoa K, Friedman K, Moore B, Tardy E, Bertram I (2014) \emph{Assessing tropical marine invertebrates: a manual for Pacific Island resource managers}. Secretariat of the Pacific Community, Noumea. 119 p.


Palmer R, Pyke D, Meek P, Bardi Jawi Rangers, Morris K (2014) Native mosaic-tailed rats muscle-up on Iwany (Sunday Island, Kimberley) displacing invasive black rats. Paper presented at the Island Arks Symposium III, Hobart (ABSTRACT). p. 1.


Parker CM, Biggs L (2014) Updates to Western Australia's vascular plant census for 2013.\emph{Nuytsia}\textbf{24}, 45-63.


Peacock D, Abbott I (2013) The role of quoll (\emph{Dasyurus}) predation in the outcome of pre-1900 introductions of rabbits (\emph{Oryctolagus} \emph{cuniculus}) to the mainland and islands of Australia.\emph{Australian Journal of Zoology}\textbf{61}, 206-280.


Pearson DJ, Webb JK, Greenlees MJ, Phillips BL, Bedford GS, Brown GP \emph{et al.} (2014) Behavioural responses of reptile predators to invasive cane toads in tropical Australia.\emph{Austral Ecology}\textbf{39}, 448-454.


Pennifold M (2013) Saving streams of the south-west forests.\emph{Landscope}\textbf{29(1)}, 49-51.


Pfeifer M, Lefebvre V, Gardner TA, Arroyo-Rodriguez V, Baeten L, Banks-Leite C \emph{et al.} [Robinson R] (2014) BIOFRAG: a new database for analyzing BIOdiversity responses to forest FRAGmentation.\emph{Ecology and Evolution}\textbf{4}, 1524-1537.


Pinder A, Arslan N, Wetzel M (2014) \emph{Proceedings of the 12th International Symposium on Aquatic Oligochaeta}. Magnolia Press, Auckland. 102 p.


Pinder A, Cale D, Lizamore J (2013) Restore it and they will come? Hydrology and waterbirds in the Lake Warden wetlands. Paper presented at the WA State Coastal Conference, Balancing communities and coasts, Esperance, 2013 (ABSTRACT). p. 1.


Pinder AM, James A (2014) A new species of \emph{Macquaridrilus} (Annelida: Clitellata: Naididae) from subantarctic Campbell Island.\emph{New Zealand Journal of Zoology}\textbf{41}, 114-123.


Pinder AM, Quinlan KL (2013) Aquatic Invertebrates of Katjarra in the Birriliburu Indigenous Protected Area: report to the Birriliburu Native Title Holders and Central Desert Native Title Services. Department of Parks and Wildlife, Kensington, WA. 20 p.


Pinder A, Quinlan K, Cale D, Shiel R (2013) Invertebrate communities and hydrological persistence in seasonal claypans of Drummond Nature Reserve, Western Australia. Department of Parks and Wildlife, Kensington, WA. 30 p.


Plucinski MP, McCaw WL, Gould JS, Wotton BM (2014) Predicting the number of daily human-caused bushfires to assist suppression planning in south-west Western Australia.\emph{International Journal of Wildland Fire}\textbf{23}, 520-531.


Pratchett MS, Hoey AS, Wilson SK (2014) Reef degradation and the loss of critical ecosystem goods and services provided by coral reef fishes.\emph{Current Opinion in Environmental Sustainability}\textbf{7}, 37-43.


Pratt R, Bourke G, Laver R, Potter S, Doughty P, Donnellan S \emph{et al.} [Palmer R] (2014) A nana-rama? An exploration of genetic divergence and cryptic morphology in the Kimberley \emph{Gehyra} \emph{nana}. Poster presented at the Australian Society of Herpetologists conference, 29 January-1 February, 2014, Greenhills Conference Centre, ACT. Australian Society of Herpetologists, Canberra. 1 poster.


Pressey B, Brotankova J, Craigie I, Gibson L, Hall S, Hicks J \emph{et al.} [Lohr C, Morris K] (2014) Conservation in paradise: prioritising management actions on islands in tropical Australia (ABSTRACT).In \emph{Island Arks Symposium III: Program \& Guide} p. 27.


Prince RI (2013) Turtle connections: turtle identification, flipper tags, post-cards to the unknown and third party engagement (ABSTRACT).In \emph{Proceedings of the First Western Australian Marine Turtle Symposium, 28-29th August 2012} (eds B Prince, S Whiting, H Raudino \emph{et al.}), p. 25.


Prince B, Whiting S, Raudino H, Vitenbergs A, Pendoley K (eds) (2013) Proceedings of the First Western Australian Marine Turtle Symposium, 28-29th August 2012. Department of Parks and Wildlife, Kensington, WA. 57 p.


Prober SM, Thiele KR, Speijers J (2013) Management legacies shape decadal-scale responses of plant diversity to experimental disturbance regimes in fragmented grassy woodlands.\emph{Journal of Applied Ecology}\textbf{50}, 376-386.


Radford IJ (2013) Fluctuating resources, disturbance and plant strategies: diverse mechanisms underlying plant invasions.\emph{Journal of Arid Land}\textbf{5}, 284-297.


Radford I (2013) Responses of savanna animals to fire. Poster displayed at Parish's art exhibition at Artopia Gallery, Kununurra July 2013. Department of Environment and Conservation, Kensington, WA. 1 poster.


Radford IJ, Andersen AN, Graham G, Trauernicht C (2013) The fire refuge value of patches of a fire-sensitive tree in fire-prone savannas: \emph{Callitris} \emph{intratropica} in northern Australia.\emph{Biotropica}\textbf{45}, 594-601.


Radford I, Thomson-Dans, Fairman R, Hatherley E (2013) Kimberley mammals bouncing back.\emph{Landscope}\textbf{28(3),} 32-38


Radford IJ, Dickman CR, Start AN, Palmer C, Carnes K, Everitt C, \emph{et al.} [Fairman R] (2014) Mammals of Australia's tropical savannas: a conceptual model of assemblage structure and regulatory factors in the Kimberley Region.\emph{PLoS One}\textbf{9}, 1-13, e92341.


Ragupathy S, Seigler DS, Ebinger JE, Maslin BR (2014) New combinations in \emph{Vachellia} and \emph{Senegalia} (Leguminosae: Mimosoideae) for south and west Asia.\emph{Phytotaxa}\textbf{162}, 174-180.


Reardon TB, McKenzie NL, Cooper SJB, Appleton B, Carthew S, Adams M (2014) A molecular and morphological investigation of species boundaries and phylogenetic relationships in Australian free-tailed bats \emph{Mormopterus} (Chiroptera: Molossidae).\emph{Australian Journal of Zoology}\textbf{62}, 109-136.


Richards ZT, Huisman JM (2014) Coral-mimicking alga \emph{Eucheuma} \emph{arnoldii} found at Ashmore Reef, north-western Australia.\emph{Coral Reefs}\textbf{33}, p. 441.


Robinson R (2013) First record of \emph{Favolaschia} \emph{calocera} in Western Australia.\emph{Australasian Mycologist}\textbf{31}, 41-43.


Robinson R (2014) The response of fungal communities to disturbance associated with management of southern Australian eucalypt forests.In \emph{Why Mushrooms and Moulds Matter} p. 4. Fungimap, Brisbane.


Robinson R, Dumbrell I (comps) (2013) Western Australia.In \emph{Annual pest, disease \& quarantine status report for Australia and New Zealand, 2011-2012} (by the Sub-Committee on National Forest Health), pp. 58-75. Primary Industries Standing Committee, Canberra.


Robinson R, Dumbrell I (2014) Annual forest health and biosecurity status report of Western Australia.In \emph{Annual pest, disease \& quarantine status report for 2012-2013} (compiled by the Sub-Committee on National Forest Health), pp. 75-100. Primary Industries Standing Committee, Canberra.


Robinson RM, May TW, Stefani FOP (2014) Comparing morphological and molecular species concepts for \emph{Cortinarius} from long-term monitoring in Western Australia (ABSTRACT).In \emph{Scientific meeting of the Australasian Mycological Society, Brisbane 21-23 April 2014: Program and Abstract Book} p. 32.


Robinson R, Stefani F, May T (2013) Comparing morphological species as used in ecological surveys against molecular barcoding with the ITS region: a case study of \emph{Cortinarius} in the south-west of Western Australia. II, an illustrated phylogenetic tree. Poster presented at Fungimap VII, Rawson, Victoria, 23-27 May 2013.


Rye BL (2013) Postcards from the kwongan.\emph{Kwongan Matters}\textbf{3}, 9-14.


Rye BL (2013) An update to the taxonomy of some Western Australian genera of the Myrtaceae tribe Chamelaucieae. 1, Calytrix.\emph{Nuytsia}\textbf{23}, 483-501.


Rye BL, Wilson PG, Keighery GJ (2013) A revision of the species of \emph{Hypocalymma} (Myrtaceae: Chamelaucieae) with smooth or colliculate seeds.\emph{Nuytsia}\textbf{23}, 283-312.


Sampson JF, Byrne M, Yates CJ, Gibson N, Thavornkanlapachai R, Stankowski S \emph{et al.} [MacDonald B] (2014) Contemporary pollen-mediated gene immigration reflects the historical isolation of a rare, animal-pollinated shrub in a fragmented landscape.\emph{Heredity}\textbf{112}, 172-181.


Schmidt DJ, Grund R, Williams MR, Hughes JM (2014) Australian parasitic \emph{Ogyris} butterflies: east-west divergence of highly-specialized relicts.\emph{Biological Journal of the Linnean Society}\textbf{111}, 473-484.


Schut AGT, Wardell-Johnson GW, Yates CJ, Keppel G, Baran I, Franklin SE \emph{et al.} [Byrne M] (2014) Rapid characterisation of vegetation structure to predict refugia and climate change impacts across a global biodiversity hotspot.\emph{PLoS One}\textbf{9}, 1-15, e82778.


Shearer BL, Crane CE, Cochrane JA, Dunne CP (2013) Variation in susceptibility of threatened flora to \emph{Phytophthora} \emph{cinnamomi}.\emph{Australasian Plant Pathology}\textbf{42}, 491-502.


Shepherd KA (2013) \emph{The Australian Biological Resources Study Churchill Fellowship to investigate species diversity among fan flowers and other unique Australian plants}. Winston Churchill Memorial Trust of Australia, Subiaco. 24 p.


Shepherd KA, Hislop M (2014) \emph{Scaevola} \emph{xanthina} (Goodeniaceae), a yellow-flowered species from the south coast of Western Australia.\emph{Nuytsia}\textbf{24}, 95-99.


Smith A, Pinder A, McBurnie G, Brim-Box J, Thompson R, Harrisson K \emph{et al.} (2013) Species hiding in plain sight: using population genetics to infer cryptic species and dispersal in Australian arid-zone freshwater insects (ABSTRACT).In \emph{Australian Society for Limnology Congress, 52nd Annual Congress, Canberra, ACT, 2-5th December 2013} p. 93.


Smith HC, Pollock K, Waples K, Bradley S, Bejder L (2013) Use of the robust design to estimate seasonal abundance and demographic parameters of a coastal bottlenose dolphin (\emph{Tursiops} \emph{aduncus}) population.\emph{PLoS One}\textbf{8}, 1-10, e76574.


Smith LA, McKenzie NL (2013) An early record of Hutton's shearwater (\emph{Puffinus} \emph{huttoni}) from Augustus Island, Kimberley, Western Australia.\emph{Western Australian Naturalist}\textbf{29}, 8-10.


Sokoloff DD, Remizowa MV, Conran JG, Macfarlane TD, Ramsay MM, Rudall PJ (2014) Embryo and seedling morphology in \emph{Trithuria} \emph{lanterna} (Hydatellaceae: Nymphaeales): new data for infrafamilial systematics and a novel type of syncotyly.\emph{Botanical Journal of the Linnean Society}\textbf{174}, 551-573.


Speed CW, Babcock RC, Bancroft KP, Beckley LE, Bellchambers LM, Depczynski M \emph{et al.} [Field SN, Friedman KJ, Moore JAY, Nutt CD, Shedrawi G, Simpson CJ, Wilson SK] (2013) Dynamic stability of coral reefs on the west Australian coast.\emph{PLoS One}\textbf{8(7)}, 1-12.


Spencer P, Friend T, Hillyer M, Thomas N, Branch K, Reinhold L (2013) Final report: genetic diversity and profiling of island and translocated populations of the banded hare-wallaby, \emph{Lagostrophus} \emph{fasciatus}. Murdoch University, School of Veterinary \& Life Sciences, Murdoch. 39 p.


Spiers M, Vitenbergs A, Prince R (2013) Rosemary Island hawksbill turtle tagging program (ABSTRACT).In \emph{Proceedings of the First Western Australian Marine Turtle Symposium, 28-29th August 2012} (eds B Prince, S Whiting, H Raudino \emph{et al.}), p. 31.


Start AN (2013) A note on anthesis in a Thai mistletoe and its relationship with a flowerpecker.\emph{Natural History Bulletin of the Siam Society}\textbf{59}, 49-51.


Steane DA, Potts BM, McLean E, Prober SM, Stock WD, Vaillancourt RE, Byrne M (2014) Genome-wide scans detect adaptation to aridity in a widespread forest tree species.\emph{Molecular Ecology}\textbf{23}, 2500-2513.


Stephen SL, Burrows N, Buyantuyev A, Gray RW, Keane RE, Kubian R \emph{et al.} (2014) Temperate and boreal forest mega-fires: characteristics and challenges.\emph{Frontiers in Ecology and the Environment}\textbf{12}, 115-122.


Stukely M, Webster J, Ciampini J (2013) Vegetation Health Service: annual report, 2012-2013, \emph{Phytophthora} detection. Department of Parks and Wildlife, Kensington, WA. 14 p.


Thiele K (2013) Botanical blitz.\emph{Landscope}\textbf{29(1)}, 9-13.


Thiele K (2013) CHAH report.\emph{Australasian Systematic Botany Society Newsletter}\textbf{157}, p. 55.


Thiele KR (2013) \emph{Hibbertia} \emph{sericosepala} (Dilleniaceae), a new species from Western Australia.\emph{Nuytsia}\textbf{23}, 479-482.


Thiele K (2013) \emph{Hibbertia} sp. Mt Lesueur (M. Hislop 174) cannot be maintained as distinct from \emph{H. crassifolia}.\emph{Nuytsia}\textbf{23}, 475-476.


Thiele K (2014) \emph{Cochlospermum} \emph{macnamarae}.\emph{Landscope}\textbf{29(3)}, p. 27.


Thiele K (2014) Orchid taxonomy in Australia needs to lift its game.\emph{Australasian Systematic Botany Society Newsletter}\textbf{158}, 13-15.


Thiele KR, Coffey SC (2014) The taxonomic status of \emph{Aldrovanda} \emph{vesiculosa} var. \emph{rubescens}.\emph{Nuytsia}\textbf{24}, 19-20.


Thiele K, Klazenga N (2013) KeyBase update: still teaching them new tricks.\emph{Australasian Systematic Botany Society Newsletter}\textbf{155}, 30-31.


Thiele KR, Walsh NG (2014) \emph{Rorippa} \emph{dictyosperma} and \emph{R. cygnorum} have mucous seeds.\emph{Nuytsia}\textbf{24}, 17-18.


Thomas N, Burbidge A, Garretson S (2014) Hermite Island fauna reconstruction. Department of Parks and Wildlife, Western Australia, Information Sheet\textbf{76/2014}, DPaW, Kensington, WA. 2 p. Available at: http://www.dpaw.wa.gov.au/about-us/science-and-research/publications-resources/111-science-division-information-sheets.


Thompson CK, Wayne AF, Godfrey SS, Thompson RCA (2014) Temporal and spatial dynamics of trypanosomes infecting the brush-tailed bettong (\emph{Bettongia} \emph{penicillata}): a cautionary note of disease-induced population decline.\emph{Parasites and Vectors}\textbf{7}, 1-11, e169.


Thums M, Meekan M, Whiting S, Reisser J, Pendoley K, Harcourt R (2013) Understanding the early offshore migration patterns of turtle hatchlings and the effects of anthropogenic light: a pilot study using acoustic tracking (ABSTRACT).In \emph{Proceedings of the First Western Australian Marine Turtle Symposium, 28-29th August 2012} (eds B Prince, S Whiting, H Raudino \emph{et al.}), p. 32.


Toon A, Joseph L, Burbidge AH (2013) Genetic analysis of the Australian whipbirds and wedgebills illuminates the evolution of their plumage and vocal diversity.\emph{Emu}\textbf{113}, 359-366.


Trotter A, Haislip J, LaCluyze A, Reichart D, Verveer A, Spuck T \emph{et al.} (2014) GRB 140301A: Skynet R-COP observations.\emph{GCN Circular}\textbf{15897}, p. 1.


Trotter A, Haislip J, Reichart D, LaCluyze A, Verveer A, Spuck T \emph{et al.} (2014) GRB 140213A: continued Skynet R-COP/PROMPT detections of a rebrightening optical afterglow.\emph{GCN Circular}\textbf{15859}, p. 1.


Trotter A, Haislip J, Reichart D, LaCluyze A, Verveer A, Spuck T \emph{et al.} (2014) GRB 140213A: continued Skynet R-COP/PROMPT observations of the optical afterglow.\emph{GCN Circular}\textbf{15862}, p. 1.


Trotter A, Haislip J, Reichart D, LaCluyze A, Verveer A, Spuck T \emph{et al.} (2014) GRB 140213A: Skynet R-COP detection of optical afterglow.\emph{GCN Circular}\textbf{15828}, p. 1.


Trotter A, Reichart D, Verveer A, Spuck T, LaCluyze A, Haislip J \emph{et al.} (2014) Correction: MAXI J1421-612: Skynet PROMPT/R-COP observations.\emph{GCN Circular}\textbf{15755}, p. 1.


Trotter A, Reichart D, Verveer A, Spuck T, LaCluyze A, Haislip J \emph{et al.} (2014) GRB 140118A/MAXI J1421-613: Skynet PROMPT/R-COP observations.\emph{GCN Circular}\textbf{15752}, p. 1.


Trudgen ME, Rye BL (2014) An update to the taxonomy of some Western Australian genera of the Myrtaceae tribe Chamelaucieae. 1, \emph{Cyathostemon}.\emph{Nuytsia}\textbf{24}, 7-16.


Tucker AD, MacDonald BD, Seminoff JA (2014) Foraging site fidelity and stable isotope values of loggerhead turtles tracked in the Gulf of Mexico and northwest Caribbean.\emph{Marine Ecology Progress Series}\textbf{502}, 267-279.


Tucker T, Whiting S, Corey B (2013) FAQ: Cape Domett flatback turtle rookery: monitoring in 2013 (POSTER). Department of Parks and Wildlife, Kensington, WA.


Ward BG, Bragg TB, Hayes BA (2014) Relationship between fire-return interval and mulga (\emph{Acacia} \emph{aneura}) regeneration in the Gibson Desert and Gascoyne-Murchison regions of Western Australia.\emph{International Journal of Wildland Fire}\textbf{23}, 394-402.


Wayne A (2014) Perup Sanctuary: insuring woylies against extinction.\emph{Landscope}\textbf{29(3)}, 12-17.


Wayne A (2014) Woylie translocation activities 2013.\emph{Bushland News}\textbf{89}, p. 1.


Wayne AF, Maxwell MA, Ward CG, Vellios CV, Wilson IJ, Dawson KE (2013) Woylie Conservation and Research Project: progress report, 2010-2013. Department of Parks and Wildlife, Kensington, WA. 291 p.


Wege J (2013) Growing (Western) Australian native plants: an experience shared (BOOK REVIEW).\emph{Australasian Systematic Botany Society Newsletter}\textbf{157}, 68-70.


Wege J (2013) Untangling the taxonomy of the triggerplants.\emph{Australian Plants}\textbf{27}, 53-63.


Wege JA (2014) \emph{Stylidium} \emph{lithophilum} and \emph{S. oreophilum} (Stylidiaceae), two new species of conservation significance from Stirling Range National Park.\emph{Nuytsia}\textbf{24}, 29-35.


Wen CKC, Almany GR, Williamson DH, Pratchett MS, Mannering TD, Evans RD \emph{et al.} (2013) Recruitment hotspots boost the effectiveness of no-take marine reserves.\emph{Biological Conservation}\textbf{166}, 124-131.


Whiting AU, Whiting SD, Thomson A (2013) High density flatback turtle nesting at a winter rookery (ABSTRACT).In \emph{Proceedings of the First Western Australian Marine Turtle Symposium, 28-29th August 2012} (eds B Prince, S Whiting, H Raudino \emph{et al.}), p. 38.


Whiting SD, Guinea ML, Fomiatti K, Flint M, Limpus CJ (2014) Plasma biochemical and PCV ranges for healthy, wild, immature hawksbill (\emph{Eretmochelys} \emph{imbricata}) sea turtles.\emph{Veterinary Record}\textbf{174}, 1-5, 608.


Whiting S, Raudino H, Crane K (2013) The North West Shelf Flatback Turtle Conservation Program (NWSFTCP) (ABSTRACT).In \emph{Proceedings of the First Western Australian Marine Turtle Symposium, 28-29th August 2012} (eds B Prince, S Whiting, H Raudino \emph{et al.}), p. 39.


Williamson DH, Ceccarelli DM, Evans RD, Jones GP, Russ GR (2014) Habitat dynamics, marine reserve status and the decline and recovery of coral reef fish communities.\emph{Ecology and Evolution}\textbf{4}, 337-354.


Wills AJ, van Heurck PF, Farr JD (2013) Persistence of velvet worms (Onychophora: Peripatopsidae): effects of fire and climate in forests of south-west Western Australia. Poster presented at Australian Entomological Society 44th AGM \& Scientific Conference, 29 September-2nd October 2013, Invertebrates in extreme environments. Department of Environment and Conservation, Kensington, WA.


Wilson SK, Fisher R, Pratchett MS (2014) Differential use of shelter holes by sympatric species of blennies (Blennidae).\emph{Marine Biology}\textbf{160}, 2405-2411.


Wilson SK, Graham NAJ, Pratchett MS (2013) Susceptibility of butterflyfish to habitat disturbance: do `chaets' ever prosper?In \emph{The Biology of Butterflyfishes} (eds MS Pratchett, ML Berumen, BG Kapoor), pp. 226-245. CRC Press, Boca Raton.


Wood S, Bowman D, Prior L, Lindenmayer D, Wardlaw T, Robinson R (2014) Tall eucalypt forests.In \emph{Biodiversity and Environmental Change: Monitoring, Challenges and Direction} (ed D Lindenmayer, E Burns, N Thurgate \emph{et al.}), pp. 519-569. CSIRO Publishing, Collingwood.


Wysong M, Hobbs R, Burrows N, Buckley Y, Valentine L, Ritchie EG (2013) The effects of predator control on feral cat and dingo abundances and interactions in arid ecosystems (ABSTRACT).\emph{Australian Mammal Society Newsletter}\textbf{Nov}, p. 46.


Wysong M, Hobbs R, Burrows N, Valentine L, Morris K, Ritchie EG (2014) The truth about cats and dogs: investigating fine scale habitat use by dingoes and feral cats at Lorna Glen Station, WA (ABSTRACT).In \emph{Australian Mammal Society, 60th Scientific Meeting Conference Handbook: 7th-10th July 2014, Melbourne, Melbourne Zoo} p. 115.


Yeatman GJ, Wayne AJ, Mills H (2013) Terrestrial vertebrate assemblage and species level patterns between major habitat types inside and outside a fenced exclosure in south-western Australia. University of Western Australia, Crawley. 22 p.


Zich F, Addicott E, Cowie I, Crayn D, Croft J, Doherty P et al. [Thiele K] (2013) Australian Savanna Plant Identification System: a collaborative initiative toward an interactive identification and information system for all Australian savanna plants (ABSTRACT).In_ Systematics without Borders: 1-6 December, 2013, Sydney, Australia_ p. 136.




\clearpage

    
\section{Current Collaboration with Academia (Student Projects)}

\begin{longtabu} to \linewidth { | X |}
\hline
\rowcolor{infobg}
Project name\\
\hline
\endhead

  No projects. \\ \hline

\end{longtabu}

\clearpage

    
\section{External Partnerships}

\begin{longtabu} to \linewidth { | X |}
\hline
\rowcolor{infobg}
Project name\\
\hline
\endhead

  Collaboration project 2014-15 WAMSI Project 1.2.2. Key biological indices required to understand and manage nesting sea turtles along the Kimbelrey coast \newline  \\ \hline

  Collaboration project 2013-51 Fast track critically endangered flora recovery \newline  \\ \hline

  Collaboration project 2013-50 Age structure of \_Callitris\_ in the Carnarvon Range \newline  \\ \hline

  Collaboration project 2013-49 Species distribution modelling in the Pilbara \newline  \\ \hline

  Collaboration project 2013-48 Great Western Woodland vegetation map reconciliation project \newline  \\ \hline

  Collaboration project 2013-47 Jartaku bilby enclosure proposal \newline  \\ \hline

  Collaboration project 2013-46 Biodiversity assets and landscape-scale management of the Fortescue River catchment \newline  \\ \hline

  Collaboration project 2013-45 Genetic studies of Pilbara EPBC Act listed vertebrate fauna \newline  \\ \hline

  Collaboration project 2013-44 A review of subterranean fauna assessment in Western Australia \newline  \\ \hline

  Collaboration project 2013-43 Ecology and management of northern quoll in the Pilbara \newline  \\ \hline

  Collaboration project 2013-41 Landscape scale management in the central Pilbara \newline  \\ \hline

  Collaboration project 2013-39 Ecology and management of bilby in the Pilbara \newline  \\ \hline

  Collaboration project 2013-37 Western Australian black spot biological survey campaign \newline  \\ \hline

  Collaboration project 2013-36 Pilbara groundwater dependant ecosystem study \newline  \\ \hline

  Collaboration project 2013-35 Cost-effective conservation decisions to mitigate threats to Pilbara biodiversity \newline  \\ \hline

  Collaboration project 2013-33 Biodiversity modelling for BHP Billiton Iron Ore's Strategic Environmental Assessment in the Pilbara \newline  \\ \hline

  Collaboration project 2013-32 Invasive \_Passiflora foetida\_ in the Kimberley and Pilbara: understanding the threat and exploring solutions \newline  \\ \hline

  Collaboration project 2013-31 Biological survey of the Birriliburru Indigenous Protected Areas: phase 1 - Carnarvon Range \newline  \\ \hline

  Collaboration project 2013-30 Floristic survey of the Fortescue Marsh \newline  \\ \hline

  Collaboration project 2013-29 Ecology and management of Pilbara olive python in the Pilbara \newline  \\ \hline

  Collaboration project 2013-27 Ecology and management of the Pilbara leaf-nosed bat \newline  \\ \hline

  Collaboration project 2013-26 Strategic weed assessment for the Chichester subregion of the Pilbara \newline  \\ \hline

  Collaboration project 2013-25 Investigating the interactions between feral predators in the Pilbara \newline  \\ \hline

  Collaboration project 2013-24 Strategic weed risk assessment and implementation plan for the Chichester and Fortescue subregions of the Pilbara \newline  \\ \hline

  Collaboration project 2013-23 Ecology and management of the northern quoll in the Pilbara \newline  \\ \hline

  Collaboration project 2013-22 Sponsorship of the 'Research directions for Pilbara leaf-nose bat' workshop \newline  \\ \hline

  Collaboration project 2013-17 TERN: ecoinformatics facility and development of ecological databases and portals \newline  \\ \hline

  Collaboration project 2013-14 Phylogenetics and floral symmetry development of the core Goodeniaceae \newline  \\ \hline

  Collaboration project 2013-11 Phylogenomic assessment of conservation priorities in two biodiversity hotspots: the Pilbara and the Kimberley \newline  \\ \hline

  Collaboration project 2013-10 Restoring natural riparian vegetation systems previously infested by blackberry along the Warren and Donnelly Rivers \newline  \\ \hline

  Collaboration project 2013-9 Identifying threats to marine biodiversity of the Ningaloo World Heritage Area:deeper water fish community surveys within the Ningaloo Marine Park \newline  \\ \hline

  Collaboration project 2012-495 Taxonomic studies on Burrup flora \newline  \\ \hline

  Collaboration project 2012-494 Pilbara biological survey \newline  \\ \hline

  Collaboration project 2012-493 NatureMap: data sharing and joint custodianship \newline  \\ \hline

  Collaboration project 2012-492 Kimberley island biodiversity asset identification \newline  \\ \hline

  Collaboration project 2012-491 Using well managed habitat to rescue woylies from the brink of extinction \newline  \\ \hline

  Collaboration project 2012-486 Contemporary ecological factors and historical evolutionary factors influencing the distribution and abundance of arid-zone reptile species in space and time \newline  \\ \hline

  Collaboration project 2012-485 Hands-On Universe: internet telescope \newline  \\ \hline

  Collaboration project 2012-484 Wetland monitoring program: rotifer and cladoceran identifications \newline  \\ \hline

  Collaboration project 2012-483 Susceptibility of frogs to declining rainfall in a biodiversity hotspot \newline  \\ \hline

  Collaboration project 2012-482 Assessing the vulnerability of honey possums to climate change and habitat disturbances in south-western Australia \newline  \\ \hline

  Collaboration project 2012-481 Understanding the early offshore migration patterns of turtle hatchlings and the effects of anthropogenic light: a pilot study \newline  \\ \hline

  Collaboration project 2012-479 Variable star observations \newline  \\ \hline

  Collaboration project 2012-478 Genetic diversity of corals in the Montebello and Barrow Islands MPAs \newline  \\ \hline

  Collaboration project 2012-477 Assessing fish communities in Shoalwater Islands Marine Park \newline  \\ \hline

  Collaboration project 2012-476 Assessing fish communities in Marmion Marine Park \newline  \\ \hline

  Collaboration project 2012-475 TERN multiscale plot network: AusPlot Rangelands and SWATT \newline  \\ \hline

  Collaboration project 2012-472 PLANET: Monitoring gravitational microlenses \newline  \\ \hline

  Collaboration project 2012-470 Increasing native habitat through protection of EPBC species and ecological communities (dibbler recovery) \newline  \\ \hline

  Collaboration project 2012-469 Astronomy education \newline  \\ \hline

  Collaboration project 2012-468 Factors associated with western ringtail possum (\_Pseudocheirus occidentalis\_) persistence within retained habitat at development sites \newline  \\ \hline

  Collaboration project 2012-467 Molecular assessment of morphological species of \_Cortinarius\_ (Fungi) as used in field surveys by analysis of the ITS barcode region \newline  \\ \hline

  Collaboration project 2012-466 Bilby conservation and management in the Pilbara \newline  \\ \hline

  Collaboration project 2012-465 Fire-mulga study: post-burn monitoring and tussock grassland survey of the Hamersley Range \newline  \\ \hline

  Collaboration project 2012-464 Resolving the systematics and taxonomy of \_Tephrosia\_ in Western Australia \newline  \\ \hline

  Collaboration project 2012-463 Seed collection zones for the Pilbara \newline  \\ \hline

  Collaboration project 2012-462 Identification Botanist position at the Western Australian Herbarium \newline  \\ \hline

  Collaboration project 2012-461 Systematics and Biogeography of the Inocybaceae \newline  \\ \hline

  Collaboration project 2012-459 Building the climate resilience of arid zone freshwater biota: identifying and prioritising processes and scales for management \newline  \\ \hline

  Collaboration project 2012-458 A risk assessment and decision framework for managing groundwater dependent ecosystems with declining water levels \newline  \\ \hline

  Collaboration project 2012-454 Woylie conservation research project \newline  \\ \hline

  Collaboration project 2012-453 Taxonomic studies of Western Australian marine plants \newline  \\ \hline

  Collaboration project 2012-451 Fishes and invertebrates of the Vasse-Wonnerup Ramsar Site \newline  \\ \hline

  Collaboration project 2012-450 Fish populations and invasive species of Vasse-Wonnerup Ramsar Site \newline  \\ \hline

  Collaboration project 2012-448 Monitoring gravitational microlenses \newline  \\ \hline

  Collaboration project 2012-447 Seed collection, storage and biology \newline  \\ \hline

  Collaboration project 2012-446 Imaging and spectrophotometry of comets \newline  \\ \hline

  Collaboration project 2012-441 Imaging and CCD photometry of transient objects \newline  \\ \hline

  Collaboration project 2012-440 Cat eradication on Dirk Hartog Island \newline  \\ \hline

  Collaboration project 2012-439 Management of weed and genetic risk in perennial landuse systems \newline  \\ \hline

  Collaboration project 2012-438 Baiting feral cats on the Fortescue Marsh \newline  \\ \hline

  Collaboration project 2012-437 Spatial and temporal patterns in benthic invertebrate communities of the Walpole and Nornalup Inlets Marine Park \newline  \\ \hline

  Collaboration project 2012-436 Monitoring movement patterns of marine fauna using Vemco VRAP Acoustic tracking system \newline  \\ \hline

  Collaboration project 2012-435 PAPP toxicosis and cat bait pellet development \newline  \\ \hline

  Collaboration project 2012-434 Christmas Island cat and rat management plan (stage 2B) \newline  \\ \hline

  Collaboration project 2012-432 Explaining and predicting the occurrence of night parrots (\_Pezoporus occidentalis\_) using GIS and ecological modelling \newline  \\ \hline

  Collaboration project 2012-431 What is the role of predators at Ningaloo and how are they impacted by human use? \newline  \\ \hline

  Collaboration project 2012-430 Climate-resilient vegetation of multi-use landscapes: exploiting genetic variability in widespread species \newline  \\ \hline

  Collaboration project 2012-429 Pilbara biological survey biodiversity GDM modelling/gap analysis: terrestrial fauna and wetland flora and fauna \newline  \\ \hline

  Collaboration project 2012-427 Translocations of mammals from Barrow Island: offset program \newline  \\ \hline

  Collaboration project 2012-426 Assessing spatial and demographic structure of anthropogenic mortality on Australasian marine turtles \newline  \\ \hline

  Collaboration project 2012-425 Assessment of genetic processes in \_Lepidosperma\_ sp. Parker Range and \_Lepidosperma\_ sp. Mt Caudan \newline  \\ \hline

  Collaboration project 2012-424 Western Desert fire project \newline  \\ \hline

  Collaboration project 2012-423 Eradication of exotic rodents from six islands of high conservation value \newline  \\ \hline

  Collaboration project 2012-422 Bushfire occurrence and fire growth modelling \newline  \\ \hline

  Collaboration project 2012-418 Identification Botanist position at the Western Australian Herbarium \newline  \\ \hline

  Collaboration project 2012-416 Establishment of translocated populations of critically endangered \_Acacia imitans\_ and \_A. unguicula\_ \newline  \\ \hline

  Collaboration project 2012-413 Temporal and spatial variation in coral cover on Western Australian reefs \newline  \\ \hline

  Collaboration project 2012-411 Ningaloo seasonal seaweeds \newline  \\ \hline

  Collaboration project 2012-410 Coral reef fish recruitment study \newline  \\ \hline

  Collaboration project 2012-408 Bush Blitz: ex-Credo Station survey \newline  \\ \hline

  Collaboration project 2012-407 Bush Blitz: Cane River Conservation Park survey \newline  \\ \hline

  Collaboration project 2012-405 Automation of species recognition and size measurement of fish from underwater stereo-video imagery \newline  \\ \hline

  Collaboration project 2012-404 IdentifyLife, a new platform for collaborative development of identification tools \newline  \\ \hline

  Collaboration project 2012-403 Defining biologically significant units in spinifex (\_Triodia\_ spp.) for improved ecological restoration in arid Australia \newline  \\ \hline

  Collaboration project 2012-402 Managing genetic diversity and evolutionary processes in foundation species for landscape restoration in the midwest of Western Australia \newline  \\ \hline

  Collaboration project 2012-401 Predicting the ecological impact of cane toads on native fauna of north western Australia \newline  \\ \hline

  Collaboration project 2012-400 Protecting the safe havens: will granite outcrop environments serve as refuges for flora threatened by anthropogenic climate change? \newline  \\ \hline

\end{longtabu}

\clearpage

    
\section{Summary of Research Projects}

\begin{longtabu} to \linewidth { | X |}
\hline
\rowcolor{infobg}
Project name\\
\hline
\endhead

  No projects. \\ \hline

\end{longtabu}

\clearpage

    
\section{Research Activities}



\clearpage

    
\subsection{Student projects - Progress Reports}



    


\end{document}

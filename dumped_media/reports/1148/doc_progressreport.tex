\documentclass[version=last, paper=a4, DIV=18, usenames, dvipsnames]{scrartcl}
\usepackage{txfonts}
\usepackage{pdflscape}
\usepackage{pdfpages}
\usepackage[english]{babel} % English language/hyphenation
%%% Bootstrap colors
\definecolor{RedFire}{RGB}{146,25,28}
\definecolor{ParksWildlife}{RGB}{0,85,144}
\definecolor{successbg}{RGB}{223,240,216}
\definecolor{errorbg}{RGB}{242,222,222}
\definecolor{warningbg}{RGB}{252,248,227}
\definecolor{infobg}{RGB}{217,237,247}
\definecolor{muted}{RGB}{153,153,153}
\definecolor{success}{RGB}{70,136,71}
\definecolor{error}{RGB}{185,74,72}
\definecolor{warning}{RGB}{192,152,83}
\definecolor{info}{RGB}{58,135,173}
\usepackage[colorlinks=true,pdftitle=doc\_progressreport.pdf,linktoc=all,linkcolor=RedFire,urlcolor=ParksWildlife]{hyperref}
\usepackage{colortbl}
\usepackage{longtable}
\usepackage{tabu}
\setlength{\tabulinesep}{1.5mm}
\usepackage{enumerate}
\usepackage{enumitem}
\usepackage{fancyhdr}
\usepackage{lastpage}
\usepackage{graphicx}
\usepackage{eso-pic}
\usepackage{hyphenat}



%%% Custom headers/footers (fancyhdr package)
\fancypagestyle{plain}{
\fancyhf{}
\setlength\headheight{40pt}
\renewcommand{\headrulewidth}{0.1pt}
\renewcommand{\footrulewidth}{0.1pt}



    \fancyhead[L]{ \href{http://sdis.dpaw.wa.gov.au/documents/progressreport/1148/download/}{} \newline }
\fancyhead[R]{ \hfill\href{http://www.dpaw.wa.gov.au}{Department of Parks and Wildlife}\newline\href{http://sdis.dpaw.wa.gov.au}{Pythia}}




\fancyfoot[L]{ \leftmark\newline\textbf{Last Modified}\textit{ }\quad\textbf{Printed}\textit{ June 19, 2014, 11:46 a.m. } }
\fancyfoot[R]{  \, \newline Page \thepage\ of \pageref{LastPage} } % Pagenumbering


}
\pagestyle{plain}


\newcommand{\HRule}{\rule{\linewidth}{0.1pt}}

\newcommand{\placetextbox}[3]{% \placetextbox{<horizontal pos>}{<vertical pos>}{<stuff>}
  \setbox0=\hbox{#3}% Put <stuff> in a box
  \AddToShipoutPictureFG*{% Add <stuff> to current page foreground
    \put(\LenToUnit{#1\paperwidth},\LenToUnit{#2\paperheight}){\vtop{{\null}\makebox[0pt][c]{#3}}}%
  }%
}%

\begin{document}

\setcounter{secnumdepth}{-1}


\begin{titlepage}
\begin{center}
% Upper part of the page
\begin{minipage}[t]{0.28\textwidth}
\begin{flushleft}
\href{http://www.dpaw.wa.gov.au}{\includegraphics[scale=0.6]{/var/www/sdis_8271/staticfiles/img/logo-dpaw.png}}
\end{flushleft}
\end{minipage}
\begin{minipage}[b]{0.7\textwidth}
\begin{flushright}
    \href{http://sdis.dpaw.wa.gov.au/documents/progressreport/1148/download/}{}) \\
\end{flushright}
\end{minipage}
\HRule \\[0.4cm]
\vfill
\textsc{\Huge Science project 2012-34 Genetic assessment for conservation of rare and threatened fauna \newline }
\vfill
\textsc{\Huge Progress Report}

\vfill\vfill\vfill\vfill
title and summary

\vfill\vfill\vfill\vfill\vfill\vfill\vfill\vfill

\textbf{Version created on} June 19, 2014, 11:46 a.m.
\vfill
\textbf{Last Modified on}  by 
\vfill\vfill
\textbf{Report Status}\\\,
\begin{tabu} to \linewidth { | X[l] | X | }
\hline
\rowcolor{infobg}
Status & Last Updated \\
\hline
\textbf{Planning - } \\
\hline
\end{tabu}
\vfill
\textbf{Science Project Overview}\\\,
\begin{tabu} to \linewidth { | X[l] | X | }
\hline
\rowcolor{infobg}
Part & Checklist Last Updated \\
\hline
\textbf{Part A - Summary \& Approval} & bla \\
\hline
\end{tabu}

\end{center}
\end{titlepage}

\setcounter{tocdepth}{2}
\tableofcontents
\clearpage






\section{Context Summary}



Genetic analysis of threatened species can provide important information to support and guide conservation management. In particular, genetic tools can be used to aid resolution of the taxonomic identity of species to determine whether they have appropriate conservation listing. At a population level, analysis of the genetic diversity present in extant populations informs us of the genetic `health' of threatened species and how this may be maintained or improved through management actions, leading to long-term positive conservation outcomes.






\section{Aims Summary}



\begin{itemize}

  \item Resolve taxonomic boundaries of Western Australian bandicoots (\emph{Isoodon} sp.), particularly \emph{I. auratus} and \emph{I. obesulus} and their subspecies, to determine appropriate conservation rankings.

  \item Assess genetic diversity and effective population size of source and translocated populations of golden bandicoot (\emph{I. auratus}) and perform population viability analysis to predict the long-term trajectory of translocated populations.

\end{itemize}






\section{Progress}



\begin{itemize}

  \item Genetic and statistical analysis of golden bandicoot translocations has been completed and published in the journal Biological Conservation (Ottewell, K., Dunlop, J., Thomas, N., Morris, K., Coates, D., \& Byrne, M. (2014). Evaluating success of translocations in maintaining genetic diversity in a threatened mammal. \emph{Biological Conservation}, \emph{171}, 209-219). Results show genetic diversity has been maintained in translocated populations of golden bandicoots at Lorna Glen, Hermite and Doole Islands, but population viability analysis showed that if population sizes are kept lower than ~1000 animals genetic diversity will be eroded over time and augmentation will be required.

  \item Tissue samples of \emph{I. obesulus}, \emph{I. auratus}, \emph{I. macrourus} and their subspecies have been sourced for taxonomic analysis and mitochondrial and nuclear sequencing of these is nearing completion. Data will be incorporated into a broader phylogeny of the genus \emph{Isoodon} (including eastern states species) in collaboration with Steve Cooper of the SA Museum. Preliminary analysis suggests revision of \emph{I. obesulus} sub-species designations will be required.

\end{itemize}






\section{Management implications}



\begin{itemize}

  \item Population viability analysis in conjunction with genetic data provides a means of determining the effective population sizes required to maintain translocated populations of animals. The fenced enclosure at Lorna Glen currently supports ~300 golden bandicoots and periodic augmentation of the population will be required to maintain genetic diversity over time. Expansion of the enclosure would enable a larger population to be maintained and will benefit the long-term viability of the translocated bandicoot population at Lorna Glen.

  \item Resolution of taxonomic boundaries between \emph{I. obesulus} and \emph{I. auratus} and their broader relationships with eastern states bandicoots should enable revision of current threatened species status at state and federal levels.

\end{itemize}






\section{Future directions}



\begin{itemize}

  \item Complete sequencing of mitochondrial and nuclear markers for taxonomic analysis of \emph{Isoodon} spp.

  \item Prepare manuscript on the taxonomy and phylogeny of \emph{I. auratus}, \emph{I. obesulus} and \emph{I. macrourus}, and contribute to manuscript on the broader phylogeny of \emph{Isoodon}. Investigate use of coalescent models to infer the evolutionary history of the genus.

\end{itemize}






\clearpage



\end{document}

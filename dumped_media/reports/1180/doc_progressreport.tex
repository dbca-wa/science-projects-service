\documentclass[version=last, paper=a4, DIV=18, usenames, dvipsnames]{scrartcl}
\usepackage{txfonts}
\usepackage{pdflscape}
\usepackage{pdfpages}
\usepackage[english]{babel} % English language/hyphenation
%%% Bootstrap colors
\definecolor{RedFire}{RGB}{146,25,28}
\definecolor{ParksWildlife}{RGB}{0,85,144}
\definecolor{successbg}{RGB}{223,240,216}
\definecolor{errorbg}{RGB}{242,222,222}
\definecolor{warningbg}{RGB}{252,248,227}
\definecolor{infobg}{RGB}{217,237,247}
\definecolor{muted}{RGB}{153,153,153}
\definecolor{success}{RGB}{70,136,71}
\definecolor{error}{RGB}{185,74,72}
\definecolor{warning}{RGB}{192,152,83}
\definecolor{info}{RGB}{58,135,173}
\usepackage[colorlinks=true,pdftitle=doc\_progressreport.pdf,linktoc=all,linkcolor=RedFire,urlcolor=ParksWildlife]{hyperref}
\usepackage{colortbl}
\usepackage{longtable}
\usepackage{tabu}
\setlength{\tabulinesep}{1.5mm}
\usepackage{enumerate}
\usepackage{enumitem}
\usepackage{fancyhdr}
\usepackage{lastpage}
\usepackage{graphicx}
\usepackage{eso-pic}
\usepackage{hyphenat}



%%% Custom headers/footers (fancyhdr package)
\fancypagestyle{plain}{
\fancyhf{}
\setlength\headheight{40pt}
\renewcommand{\headrulewidth}{0.1pt}
\renewcommand{\footrulewidth}{0.1pt}



    \fancyhead[L]{ \href{http://sdis.dpaw.wa.gov.au/documents/progressreport/1180/download/}{} \newline }
\fancyhead[R]{ \hfill\href{http://www.dpaw.wa.gov.au}{Department of Parks and Wildlife}\newline\href{http://sdis.dpaw.wa.gov.au}{Pythia}}




\fancyfoot[L]{ \leftmark\newline\textbf{Last Modified}\textit{ }\quad\textbf{Printed}\textit{ June 12, 2014, 12:13 p.m. } }
\fancyfoot[R]{  \, \newline Page \thepage\ of \pageref{LastPage} } % Pagenumbering


}
\pagestyle{plain}


\newcommand{\HRule}{\rule{\linewidth}{0.1pt}}

\newcommand{\placetextbox}[3]{% \placetextbox{<horizontal pos>}{<vertical pos>}{<stuff>}
  \setbox0=\hbox{#3}% Put <stuff> in a box
  \AddToShipoutPictureFG*{% Add <stuff> to current page foreground
    \put(\LenToUnit{#1\paperwidth},\LenToUnit{#2\paperheight}){\vtop{{\null}\makebox[0pt][c]{#3}}}%
  }%
}%

\begin{document}

\setcounter{secnumdepth}{-1}


\begin{titlepage}
\begin{center}
% Upper part of the page
\begin{minipage}[t]{0.28\textwidth}
\begin{flushleft}
\href{http://www.dpaw.wa.gov.au}{\includegraphics[scale=0.6]{/var/www/sdis_8271/staticfiles/img/logo-dpaw.png}}
\end{flushleft}
\end{minipage}
\begin{minipage}[b]{0.7\textwidth}
\begin{flushright}
    \href{http://sdis.dpaw.wa.gov.au/documents/progressreport/1180/download/}{}) \\
\end{flushright}
\end{minipage}
\HRule \\[0.4cm]
\vfill
\textsc{\Huge Science project 2010-10 Comparison of underwater visual census and diver-operated video methods for assessing fish community condition in tropical and temperate coastal waters of Western Australia \newline }
\vfill
\textsc{\Huge Progress Report}

\vfill\vfill\vfill\vfill
title and summary

\vfill\vfill\vfill\vfill\vfill\vfill\vfill\vfill

\textbf{Version created on} June 12, 2014, 12:13 p.m.
\vfill
\textbf{Last Modified on}  by 
\vfill\vfill
\textbf{Report Status}\\\,
\begin{tabu} to \linewidth { | X[l] | X | }
\hline
\rowcolor{infobg}
Status & Last Updated \\
\hline
\textbf{Planning - } \\
\hline
\end{tabu}
\vfill
\textbf{Science Project Overview}\\\,
\begin{tabu} to \linewidth { | X[l] | X | }
\hline
\rowcolor{infobg}
Part & Checklist Last Updated \\
\hline
\textbf{Part A - Summary \& Approval} & bla \\
\hline
\end{tabu}

\end{center}
\end{titlepage}

\setcounter{tocdepth}{2}
\tableofcontents
\clearpage






\section{Context Summary}



In shallow coastal waters, the condition of fish communities has traditionally been monitored using a technique known as underwater visual census (UVC). More recently, a technique for assessing finfish community condition has been developed that uses stereo-video to capture imagery of fish communities that is later analysed in the laboratory. Known as diver-operated video (DOV), this technique has potential advantages over UVC because sampling theoretically requires less scientific expertise, takes less time in the field, and data analysts have access to reference material to help identify fish. In addition, it provides a permanent record of the survey that can be checked or revisited at a later point if required. However, DOV datasets may be more costly to process and it is unclear if video imagery captures the same level of diversity and abundance as skilled divers conducting UVC.


Despite the significant amount of DOV surveys that have been completed on fish communities in Western Australia over recent years, no thorough investigation comparing the overall utility, results and cost-effectiveness of this technique with conventional UVC methods has been conducted. As such, comparable assessments need to be undertaken to assess the relative utility of these techniques in both temperate and tropical waters where the Western Australian Marine Monitoring Program (WAMMP) has monitoring responsibilities. This study will improve our understanding of historical UVC and DOV dataset compatibility and the relative costs of each method.






\section{Aims Summary}



\begin{itemize}

  \item Examine the comparability of the fish community dataset (diversity, abundance and size measures) resulting from collection using UVC and DOV survey techniques.

  \item Examine the effect of varying levels of diversity and abundance on the resulting fish community datasets when data is collected using the two techniques.

  \item Examine the effect of habitat complexity related to tropical and temperate marine ecosystems on the resulting fish datasets when data is collected using the two techniques.

  \item Assess the relative cost and practicality of both UVC and DOV techniques in the context of long-term monitoring programs in both tropical and temperate remote locations.

\end{itemize}






\section{Progress}



\begin{itemize}

  \item The study found that UVC consistently recorded higher measures of species richness and these differences were most pronounced at tropical locations where diversity was high. Differences in the characteristics of fish assemblages were primarily driven by UVC detecting more cryptic species. When examined at higher taxonomic or functional levels there was greater comparability between the assemblages recorded by each method, particularly in temperate locations. Data collected using stereo-DOV took 2–3 times longer to obtain than with UVC due to extensive post-processing time required by the stereo-DOV method. Overall, data collected by the two methods are most comparable in temperate locations, or when examined at higher taxonomic/functional levels. However, comparisons should  be approached more cautiously in higher diversity locations, or when assessing at finer taxonomic resolution.

  \item Results from the Rowley Shoals were presented at the Australian Coral Reef Symposium.

  \item Results from the entire study were presented at the MSP science meeting 2012.

  \item A paper on the project has been accepted for publication in \emph{Limnology and Oceanography} \emph{Methods}.

\end{itemize}






\section{Management implications}



\begin{itemize}

  \item Evaluation of each methodology, including cost and time analysis, will provide advice to monitoring programs on which methods are most appropriate for assessing diversity and abundance of fish in marine protected areas.

  \item Time series analyses allow managers to detect trends in the condition of assests within marine parks. Information obtained from this study will allow us to assess if comparisons between historical datasets collected using UVC are comparable with more recent data collected with DOV.

\end{itemize}






\section{Future directions}



This project is complete.






\clearpage



\end{document}


\documentclass[version=last,
    paper=a4, % paper size
    10pt, % default font size
    usenames,
    dvipsnames,
    oneside, % ONLINE
    headings=openany, % open chapters on odd and even pages
    %toc=chapterentrywithdots, % Table of Contents style
    %BCOR=7mm, % PRINT Binding Correction
    %DIV=13, % typearea 161.54 mm x 228.46 mm, top margin 22.85 mm, inner margin 16.15 mm
    %DIV=14, % 165.00 233.36 21.21 15.00
    DIV=15 % 168.00 237.60 19.80 14.00
]{scrbook}
\usepackage{typearea}
\usepackage[automark,headsepline,footsepline]{scrlayer-scrpage} % Headers and footers

%%
%% Fonts, encoding, spacing, indentation
%%
\usepackage{txfonts}
\renewcommand{\familydefault}{\sfdefault} % Default to Sans Serif font
\usepackage[english]{babel}
\usepackage[T1]{fontenc}
\usepackage[utf8]{inputenc}

% Paragraph spacing
%\usepackage{parskip}    % Paragraph spacing
%\setlength{\parindent}{0em} % Don't indent paragraphs - ONLINE
%\setlength{\parskip}{1.3 ex plus 0.5ex minus 0.3ex} % 1-1.8 ex vertical space between paragraphs - ONLINE

% Spacing of headings
%\RedeclareSectionCommand[afterskip=3pt]{section} % less space after section
%\RedeclareSectionCommand[beforeskip=0cm]{subsection} % less space between HRule and project name
%\RedeclareSectionCommand[afterskip=0.1\baselineskip]{subsubsection} % less space after progressreport subheadings

% Table font size
\usepackage{etoolbox}
\AtBeginEnvironment{longtabu}{\footnotesize}{}{}

%%
%% Tables, columns, layout
%%
\usepackage{multicol}   % 2 col publications
\usepackage{pdflscape}  % Landscape pages
\usepackage{pdfpages}   % Include PDFs
\usepackage{hanging}    % hanging paragraphs for publications
%\usepackage{titletoc}   % Required for manipulating the table of contents
\setcounter{tocdepth}{2} % TOC list down to section
\usepackage{enumerate}  % Enumerations
\usepackage{enumitem}   % Enumerations
\usepackage{longtable}  % Multipage table
\usepackage{tabu}       %
\setlength{\tabulinesep}{1.5mm} % Consistent vertical spacing in tabu

%%
%% Graphics, images, colours
%%
\usepackage{graphicx} % embedded images
\usepackage{eso-pic} %
\usepackage{colortbl} % define custom named colours
\definecolor{RedFire}{RGB}{146,25,28}
\definecolor{ParksWildlife}{RGB}{0,85,144}
\definecolor{successbg}{RGB}{223,240,216}
\definecolor{errorbg}{RGB}{242,222,222}
\definecolor{warningbg}{RGB}{252,248,227}
\definecolor{infobg}{RGB}{217,237,247}
\definecolor{muted}{RGB}{153,153,153}
\definecolor{success}{RGB}{70,136,71}
\definecolor{error}{RGB}{185,74,72}
\definecolor{warning}{RGB}{192,152,83}
\definecolor{info}{RGB}{58,135,173}

\definecolor{required}{RGB}{192,152,83}
\definecolor{requiredbg}{RGB}{252,248,227}
\definecolor{denied}{RGB}{185,74,72}
\definecolor{deniedbg}{RGB}{242,222,222}
\definecolor{granted}{RGB}{70,136,71}
\definecolor{grantedbg}{RGB}{223,240,216}
\definecolor{not reqiured}{RGB}{153,153,153}
\definecolor{not requiredbg}{RGB}{255,255,255}

\usepackage{tikz} % Drawing
\usetikzlibrary{arrows,shapes,positioning,shadows,trees}

%%
%% Links, URLs
%%
\usepackage[
    linktoc=all,
    %colorlinks=false,  %PRINT
    colorlinks=true, % ONLINE
    linkcolor=RedFire, % ONLINE
    urlcolor=ParksWildlife, % ONLINE
    pdftitle=Progress Report SP 2010-003 (FY 2015-2016)
]{hyperref}

% Black magic to linebreak URLs
\usepackage{url}
\makeatletter
\g@addto@macro{\UrlBreaks}{\UrlOrds}
\makeatother

%%
%% Custom macros
%%
% Thick Horizontal rule
\newcommand{\HRule}{\vspace{8mm}\\\noindent\rule{\linewidth}{0.1pt}}

% Custom Tikz node for SDS diagram
\newcommand\mynode[6][]{
    \node[#1] (#2){
        \parbox{#3\relax}{
            \begin{center}
            \textbf{#4}\\
            #5\\
            \footnotesize{#6}
            \end{center}}};}



%-----------------------------------------------------------------------------%
% Headers and Footers
\automark{section}
\ohead{\href{http://sdis.dpaw.wa.gov.au/documents/progressreport/1647/}{Progress Report SP 2010-003
}}
\chead{\href{http://sdis.dpaw.wa.gov.au}{SDIS}} % center header ONLINE
\ihead{\href{http://sdis.dpaw.wa.gov.au}{
        \includegraphics[scale=0.4]{/mnt/projects/sdis/staticfiles/img/logo-dpaw.png}}}
\ifoot{\textbf{Printed}~Mon, 4 Jul 2016 16:16:14 +0800} % inner/left footer
\cfoot{} % center footer
\ofoot{\pagemark} % outer/right footer
\pagestyle{scrheadings}
\setkomafont{pageheadfoot}{\normalfont}

%-----------------------------------------------------------------------------%
\begin{document}
\raggedbottom

%-----------------------------------------------------------------------------%
% Title page
\subject{Progress Report SP 2010-003
}
\title{Temperature thresholds for recruitment in south-west Western Australian
flora
}
\subtitle{Plant Science and Herbarium
}
\author{}
\publishers{\small
    \subsection*{Project Core Team}
\begin{tabu} {X X}
\textbf{Supervising Scientist} & Anne Cochrane
\\
\textbf{Data Custodian} & 
\\
\textbf{Site Custodian} & 
\\
\end{tabu}


    \subsection*{Project status as of July 4, 2016, 4:16 p.m.}
\begin{tabu} {X X}
& Approved and active
\\
\end{tabu}

    
\subsection*{Document endorsements and approvals as of July 4, 2016, 4:16 p.m.}
\begin{tabu} {X X}

%\rowcolor{grantedbg}
    \textbf{Project Team} & 
    \textcolor{granted}{ granted}\\

%\rowcolor{grantedbg}
    \textbf{Program Leader} & 
    \textcolor{granted}{ granted}\\

%\rowcolor{grantedbg}
    \textbf{Directorate} & 
    \textcolor{granted}{ granted}\\

\end{tabu}



}
\uppertitleback{}
\lowertitleback{}
\date{}

%-----------------------------------------------------------------------------%
% Front matter
\frontmatter
\maketitle
%-----------------------------------------------------------------------------%
% Main matter
\mainmatter

\section*{Temperature thresholds for recruitment in south-west Western Australian
flora
}

A Cochrane


\section*{Context}
Germination is one of the fundamental biological activities vital to
persistence in obligate-seeding species. Climate directly influences
germination and seedling growth, with temperature arguably the most
important climatic variable after moisture, since it synchronises
germination to environmental conditions most suitable for seedling
establishment. Although species have climate preferences, knowledge of
basic physiological tolerances is lacking for most native species.
Assessing direct physiological constraints on recruitment (e.g. upper
and lower temperature limits for germination) and early seedling growth
may assist our understanding of the impact of warming temperatures on
the persistence of plant species at the limits of their geographic
range.



\section*{Aims}
\begin{itemize}
\itemsep1pt\parskip0pt\parsep0pt
\item
  Determine temperature thresholds for germination in south-west Western
  Australia flora.
\item
  Identify potentially `at risk' plant species and incorporate this data
  into the modelling of impacts of climate change on `at risk' species.
\item
  Utilise this data as a basis for developing management response
  options, including fire management and flora translocations.
\end{itemize}



\section*{Progress}
\begin{itemize}
\item
  Paper published in \emph{Seed Science Research} describing effects of
  temperature on germination of Western Australian obligate seeding
  \emph{Banksia} species.
\item
  Paper in press in \emph{Austral Ecology} describing variation in plant
  functional traits across and within four species of Western Australian
  \emph{Banksia} (Proteaceae) along a natural climate gradient.
\item
  Project assessing the impact of long duration, temperature
  fluctuations on overcoming seed dormancy in common \emph{Acacia}
  species nearing completion.
\item
  Continuing assessment of the temperature niche for germination in
  species from the south west of Western Australia focussing on a range
  of \emph{Eucalyptus} species.
\end{itemize}



\section*{Management implications}
\begin{itemize}
\itemsep1pt\parskip0pt\parsep0pt
\item
  Developing a framework to assess seed viability under environmental
  change will assist in restoration and land management programs.
\item
  Incorporation of seed biology knowledge~into threatened species
  translocations will increase the opportunities for recovery success.
\end{itemize}



\section*{Future directions}
\begin{itemize}
\itemsep1pt\parskip0pt\parsep0pt
\item
  Draft and submit a paper on the influence of temperature on seed
  germination in a range of Western Australian\emph{Eucalyptus} species.
\item
  Draft and submit a paper on the influence of long duration temperature
  fluctuations on overcoming dormancy in a range of Western
  Australian\emph{Acacia} species.
\item
  Ongoing studies profiling the temperature niche for germination of a
  range of Western Australian species.
\end{itemize}



%-----------------------------------------------------------------------------%
% Back matter
%\backmatter
\end{document}
%-----------------------------------------------------------------------------%



\documentclass[version=last, 
    paper=a4, % paper size
    10pt, % default font size
    usenames,
    dvipsnames, 
    oneside, % ONLINE
    headings=openany, % open chapters on odd and even pages
    %toc=chapterentrywithdots, % Table of Contents style
    %BCOR=7mm, % PRINT Binding Correction
    %DIV=13, % typearea 161.54 mm x 228.46 mm, top margin 22.85 mm, inner margin 16.15 mm
    %DIV=14, % 165.00 233.36 21.21 15.00
    DIV=15 % 168.00 237.60 19.80 14.00
]{scrbook}
\usepackage{typearea}
\usepackage[automark,headsepline,footsepline]{scrlayer-scrpage} % Headers and footers

%%
%% Fonts, encoding, spacing, indentation
%%
\usepackage{txfonts}
\renewcommand{\familydefault}{\sfdefault} % Default to Sans Serif font
\usepackage[english]{babel}
\usepackage[T1]{fontenc}
\usepackage[utf8]{inputenc}

% Paragraph spacing
%\usepackage{parskip}    % Paragraph spacing
%\setlength{\parindent}{0em} % Don't indent paragraphs - ONLINE
%\setlength{\parskip}{1.3 ex plus 0.5ex minus 0.3ex} % 1-1.8 ex vertical space between paragraphs - ONLINE

% Spacing of headings
%\RedeclareSectionCommand[afterskip=3pt]{section} % less space after section
%\RedeclareSectionCommand[beforeskip=0cm]{subsection} % less space between HRule and project name
%\RedeclareSectionCommand[afterskip=0.1\baselineskip]{subsubsection} % less space after progressreport subheadings

% Table font size
\usepackage{etoolbox}
\AtBeginEnvironment{longtabu}{\footnotesize}{}{}

%%
%% Tables, columns, layout
%%
\usepackage{multicol}   % 2 col publications
\usepackage{pdflscape}  % Landscape pages
\usepackage{pdfpages}   % Include PDFs
\usepackage{hanging}    % hanging paragraphs for publications
%\usepackage{titletoc}   % Required for manipulating the table of contents
\setcounter{tocdepth}{2} % TOC list down to section
\usepackage{enumerate}  % Enumerations
\usepackage{enumitem}   % Enumerations
\usepackage{longtable}  % Multipage table
\usepackage{tabu}       % 
\setlength{\tabulinesep}{1.5mm} % Consistent vertical spacing in tabu

%%
%% Graphics, images, colours
%%
\usepackage{graphicx} % embedded images
\usepackage{eso-pic} % 
\usepackage{colortbl} % define custom named colours
\definecolor{RedFire}{RGB}{146,25,28}
\definecolor{ParksWildlife}{RGB}{0,85,144}
\definecolor{successbg}{RGB}{223,240,216}
\definecolor{errorbg}{RGB}{242,222,222}
\definecolor{warningbg}{RGB}{252,248,227}
\definecolor{infobg}{RGB}{217,237,247}
\definecolor{muted}{RGB}{153,153,153}
\definecolor{success}{RGB}{70,136,71}
\definecolor{error}{RGB}{185,74,72}
\definecolor{warning}{RGB}{192,152,83}
\definecolor{info}{RGB}{58,135,173}

\definecolor{required}{RGB}{192,152,83}
\definecolor{requiredbg}{RGB}{252,248,227}
\definecolor{denied}{RGB}{185,74,72}
\definecolor{deniedbg}{RGB}{242,222,222}
\definecolor{granted}{RGB}{70,136,71}
\definecolor{grantedbg}{RGB}{223,240,216}
\definecolor{not reqiured}{RGB}{153,153,153}
\definecolor{not requiredbg}{RGB}{255,255,255}

\usepackage{tikz} % Drawing
\usetikzlibrary{arrows,shapes,positioning,shadows,trees}

%%
%% Links, URLs
%%
\usepackage[
    linktoc=all,
    %colorlinks=false,  %PRINT
    colorlinks=true, % ONLINE
    linkcolor=RedFire, % ONLINE
    urlcolor=ParksWildlife, % ONLINE
    pdftitle=doc\_studentreport.pdf
]{hyperref}

% Black magic to linebreak URLs
\usepackage{url}
\makeatletter
\g@addto@macro{\UrlBreaks}{\UrlOrds}
\makeatother

%%
%% Custom macros
%%
% Thick Horizontal rule
\newcommand{\HRule}{\vspace{8mm}\\\noindent\rule{\linewidth}{0.1pt}}

% Custom Tikz node for SDS diagram
\newcommand\mynode[6][]{\node[#1] (#2){\parbox{#3\relax}{\begin{center}\textbf{#4}\\#5\\\footnotesize{#6}\end{center}}};}




%-----------------------------------------------------------------------------%
% Headers and Footers
\automark{section}
\ohead{\href{http://sdis.dpaw.wa.gov.au/documents/studentreport/1724/}{Progress Report STP 2013-016 (FY 2015-2016)
}}
\chead{\href{http://sdis.dpaw.wa.gov.au}{SDIS}} % center header ONLINE
\ihead{\href{http://sdis.dpaw.wa.gov.au}{
        \includegraphics[scale=0.4]{/mnt/projects/sdis/staticfiles/img/logo-dpaw.png}}}
\ifoot{\textbf{Printed}~Tue, 14 Jun 2016 14:14:17 +0800} % inner/left footer
\cfoot{} % center footer
\ofoot{\pagemark} % outer/right footer
\pagestyle{scrheadings}
\setkomafont{pageheadfoot}{\normalfont}

%-----------------------------------------------------------------------------%
\begin{document}
\raggedbottom

%-----------------------------------------------------------------------------%
% Title page
\subject{Progress Report STP 2013-016 (FY 2015-2016)
}
\title{A comparative health and disease investigation in the woylie: captive vs
free-range enclosure vs wild
}
\subtitle{Animal Science
}
\author{}
\publishers{\small
    \subsection*{Project Core Team}
\begin{tabu} {X X}
\textbf{Supervising Scientist} & Adrian Wayne
\\
\textbf{Data Custodian} & 
\\
\textbf{Site Custodian} & 
\\
\end{tabu}


    \subsection*{Project status as of June 14, 2016, 2:14 p.m.}
\begin{tabu} {X X}
& Update requested
\\
\end{tabu}

    
\subsection*{Document endorsements and approvals as of June 14, 2016, 2:14 p.m.}
\begin{tabu} {X X}

%\rowcolor{requiredbg}
    \textbf{Project Team} & 
    \textcolor{required}{ required}\\

%\rowcolor{requiredbg}
    \textbf{Program Leader} & 
    \textcolor{required}{ required}\\

%\rowcolor{requiredbg}
    \textbf{Directorate} & 
    \textcolor{required}{ required}\\

\end{tabu}



}
\uppertitleback{}
\lowertitleback{}
\date{}

%-----------------------------------------------------------------------------%
% Front matter
\frontmatter
\maketitle
%-----------------------------------------------------------------------------%
% Main matter
\mainmatter

\section*{A comparative health and disease investigation in the woylie: captive vs
free-range enclosure vs wild
}

A Wayne, K Skogvold, Dr K Warren, Dr S Vitali, Dr C Holyoake, Dr C
Monaghan


\section*{Progress Report}
This project investigates if disease is a significant factor in the
declines and lack of recovery of the woylie. Comparisons are made over
time of the health and disease of woylies in three varying population
management systems - wild, predator-free sanctuary and captive. Sampling
from the Perup Sanctuary gives the opportunity to investigate and focus
on the role of disease in the absence of introduced predators.

Health testing at these sites has been completed and included
haematology, biochemistry, gastrointestinal parasite load, anti-oxidant
and vitamin levels, and determination of stress levels using hair,
faecal and serum cortisol. Screening for significant marsupial pathogens
and diseases has included haemoparasites, toxoplasmosis and selected
viruses (Wallal \& Warrego orbiviruses and macropodid herpesviruses).
Significant progress has been made in the data analysis and writing up
phases. Reference ranges for haematology and serum biochemistry are
being established, and a journal paper on the findings of herpesvirus
screening is underway.



%-----------------------------------------------------------------------------%
% Back matter
%\backmatter
\end{document}
%-----------------------------------------------------------------------------%


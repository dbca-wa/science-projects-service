
\documentclass[version=last,
    paper=a4, % paper size
    10pt, % default font size
    usenames,
    dvipsnames,
    oneside, % ONLINE
    headings=openany, % open chapters on odd and even pages
    %toc=chapterentrywithdots, % Table of Contents style
    %BCOR=7mm, % PRINT Binding Correction
    %DIV=13, % typearea 161.54 mm x 228.46 mm, top margin 22.85 mm, inner margin 16.15 mm
    %DIV=14, % 165.00 233.36 21.21 15.00
    DIV=15 % 168.00 237.60 19.80 14.00
]{scrbook}
\usepackage{typearea}
\usepackage[automark,headsepline,footsepline]{scrlayer-scrpage} % Headers and footers

%%
%% Fonts, encoding, spacing, indentation
%%
\usepackage{txfonts}
\renewcommand{\familydefault}{\sfdefault} % Default to Sans Serif font
\usepackage[english]{babel}
\usepackage[T1]{fontenc}
\usepackage[utf8]{inputenc}

% Paragraph spacing
%\usepackage{parskip}    % Paragraph spacing
%\setlength{\parindent}{0em} % Don't indent paragraphs - ONLINE
%\setlength{\parskip}{1.3 ex plus 0.5ex minus 0.3ex} % 1-1.8 ex vertical space between paragraphs - ONLINE

% Spacing of headings
%\RedeclareSectionCommand[afterskip=3pt]{section} % less space after section
%\RedeclareSectionCommand[beforeskip=0cm]{subsection} % less space between HRule and project name
%\RedeclareSectionCommand[afterskip=0.1\baselineskip]{subsubsection} % less space after progressreport subheadings

% Table font size
\usepackage{etoolbox}
\AtBeginEnvironment{longtabu}{\footnotesize}{}{}

%%
%% Tables, columns, layout
%%
\usepackage{multicol}   % 2 col publications
\usepackage{pdflscape}  % Landscape pages
\usepackage{pdfpages}   % Include PDFs
\usepackage{hanging}    % hanging paragraphs for publications
%\usepackage{titletoc}   % Required for manipulating the table of contents
\setcounter{tocdepth}{2} % TOC list down to section
\usepackage{enumerate}  % Enumerations
\usepackage{enumitem}   % Enumerations
\usepackage{longtable}  % Multipage table
\usepackage{tabu}       %
\setlength{\tabulinesep}{1.5mm} % Consistent vertical spacing in tabu

%%
%% Graphics, images, colours
%%
\usepackage{graphicx} % embedded images
\usepackage{eso-pic} %
\usepackage{colortbl} % define custom named colours
\definecolor{RedFire}{RGB}{146,25,28}
\definecolor{ParksWildlife}{RGB}{0,85,144}
\definecolor{successbg}{RGB}{223,240,216}
\definecolor{errorbg}{RGB}{242,222,222}
\definecolor{warningbg}{RGB}{252,248,227}
\definecolor{infobg}{RGB}{217,237,247}
\definecolor{muted}{RGB}{153,153,153}
\definecolor{success}{RGB}{70,136,71}
\definecolor{error}{RGB}{185,74,72}
\definecolor{warning}{RGB}{192,152,83}
\definecolor{info}{RGB}{58,135,173}

\definecolor{required}{RGB}{192,152,83}
\definecolor{requiredbg}{RGB}{252,248,227}
\definecolor{denied}{RGB}{185,74,72}
\definecolor{deniedbg}{RGB}{242,222,222}
\definecolor{granted}{RGB}{70,136,71}
\definecolor{grantedbg}{RGB}{223,240,216}
\definecolor{not reqiured}{RGB}{153,153,153}
\definecolor{not requiredbg}{RGB}{255,255,255}

\usepackage{tikz} % Drawing
\usetikzlibrary{arrows,shapes,positioning,shadows,trees}

%%
%% Links, URLs
%%
\usepackage[
    linktoc=all,
    %colorlinks=false,  %PRINT
    colorlinks=true, % ONLINE
    linkcolor=RedFire, % ONLINE
    urlcolor=ParksWildlife, % ONLINE
    pdftitle=Progress Report SP 2003-004 (FY 2014-2015)
]{hyperref}

% Black magic to linebreak URLs
\usepackage{url}
\makeatletter
\g@addto@macro{\UrlBreaks}{\UrlOrds}
\makeatother

%%
%% Custom macros
%%
% Thick Horizontal rule
\newcommand{\HRule}{\vspace{8mm}\\\noindent\rule{\linewidth}{0.1pt}}

% Custom Tikz node for SDS diagram
\newcommand\mynode[6][]{
    \node[#1] (#2){
        \parbox{#3\relax}{
            \begin{center}
            \textbf{#4}\\
            #5\\
            \footnotesize{#6}
            \end{center}}};}



\usepackage[automark,headsepline,footsepline,plainfootsepline]{scrlayer-scrpage}
\automark*[section]{}
\addtokomafont{pageheadfoot}{\normalfont\footnotesize\sffamily} % Don't italicise
\renewcommand{\chaptermark}[1]{\markleft{#1}{}}     % Chapter: suppress numbering
\renewcommand{\sectionmark}[1]{\markright{#1}{}}    % Section: suppress numbering

% Header (inner, center, outer)
\ihead{\href{http://sdis.dpaw.wa.gov.au/documents/progressreport/1402/}{Progress Report SP 2003-004 (FY 2014-2015)}}
%\chead{\href{http://sdis.dpaw.wa.gov.au}{Science Directorate Information System}}
\ohead{\href{https://www.dpaw.wa.gov.au/about-us/science-and-research}{\includegraphics[height=6mm, keepaspectratio]{/mnt/projects/sdis/staticfiles/img/logo-dpaw.png}}}

% Footer (inner, center, outer)
\ifoot{\textbf{Printed}~Wed, 27 Sep 2017 09:23:07 +0800} % inner/left footer
\cfoot{}
\ofoot[\bfseries\thepage]{\bfseries\thepage}        % Page number (also [plain])


\pagestyle{scrheadings}
\setkomafont{pageheadfoot}{\normalfont}

%-----------------------------------------------------------------------------%
\begin{document}
\raggedbottom

%-----------------------------------------------------------------------------%
% Title page
\subject{Progress Report SP 2003-004
}
\title{Project Rangelands Restoration: developing sustainable management
systems for the conservation of biodiversity at the landscape scale in
rangelands of the Murchison and Gascoyne bioregions--managing fire and
introduced predators
}
\subtitle{Ecosystem Science
}
\author{}
\publishers{\small
    \subsection*{Project Core Team}
\begin{tabu} {X X}
\textbf{Supervising Scientist} & Neil Burrows
\\
\textbf{Data Custodian} & 
\\
\textbf{Site Custodian} & 
\\
\end{tabu}


    \subsection*{Project status as of Sept. 27, 2017, 9:23 a.m.}
\begin{tabu} {X X}
& Approved and active
\\
\end{tabu}

    
\subsection*{Document endorsements and approvals as of Sept. 27, 2017, 9:23 a.m.}
\begin{tabu} {X X}

%\rowcolor{grantedbg}
    \textbf{Project Team} & 
    \textcolor{granted}{ granted}\\

%\rowcolor{grantedbg}
    \textbf{Program Leader} & 
    \textcolor{granted}{ granted}\\

%\rowcolor{grantedbg}
    \textbf{Directorate} & 
    \textcolor{granted}{ granted}\\

\end{tabu}



}
\uppertitleback{}
\lowertitleback{}
\date{}

%-----------------------------------------------------------------------------%
% Front matter
\frontmatter
\maketitle
%-----------------------------------------------------------------------------%
% Main matter
\mainmatter

\section*{Project Rangelands Restoration: developing sustainable management
systems for the conservation of biodiversity at the landscape scale in
rangelands of the Murchison and Gascoyne bioregions--managing fire and
introduced predators
}

N Burrows, G Liddelow



\section*{Context}

Despite the relatively pristine nature of most of the arid interior
(desert bioregions) and rangelands (beyond the pastoral zone), there has
been an alarming and recent loss of mammal fauna, with about 90\% of
medium-size mammals and 33\% of all mammals either becoming extinct or
suffering massive range contractions. There is also evidence of
degradation of some floristic communities due to altered fires regimes.
The extent and nature of change in other components of the biodiversity,
including extant mammals, birds, reptiles and invertebrates is unknown.
The most likely causes of the decline and degradation in biodiversity
are introduced predators, especially the fox (\emph{Vulpes vulpes}) and
the feral cat (\emph{Felis catus}), and altered fire regimes since the
departure from traditional Aboriginal burning practices over much of the
region. Taking an adaptive experimental management approach in
partnership with Goldfields Region, this project aims to reconstruct
some assemblages of the original native mammal fauna on Lorna Glen, a
pastoral lease acquired by the Department. This will be achieved by an
integrated approach to controlling introduced predators and herbivores,
ecologically appropriate fire management, and fauna translocations.




\section*{Aims}

\begin{itemize}
\itemsep1pt\parskip0pt\parsep0pt
\item
  Develop efficient, effective and safe introduced predator (fox and
  feral cat) control technologies for the interior rangelands and the
  arid region.
\item
  Reconstruct the original suite of native mammal fauna through
  translocation once sustainable feral cat control can be demonstrated.
\item
  Implement a patch-burn strategy to create a fine-grained, fire-induced
  habitat mosaic to protect biodiversity and other values.
\item
  Describe and predict pyric (post-fire) plant succession and describe
  the life histories of key plant species.
\item
  Monitor the long-term trends in species assemblages and abundance of
  small mammals and reptiles in an area where introduced predators are
  not controlled compared with an area where they are controlled.
\item
  Model the relationship between seasons (rainfall) and the frequency
  and size of wildfires.
\end{itemize}




\section*{Progress}

\begin{itemize}
\itemsep1pt\parskip0pt\parsep0pt
\item
  Cat, fox and wild dog aerial baiting carried out on Lorna Glen in July
  2014 as part of the Western Shield program was partially effective
  with the feral cat population reduced by \textasciitilde{}30\% from an
  activity index high of 22.4.~Radio tracking however, suggested a cat
  reduction of 60\%. Deterioration of track count survey lines due to
  heavy rain prior to the survey may have contributed to this
  discrepancy.~~
\item
  This year for the first time, a survey was also carried out on
  Earaheedy, which has never been baited. The cat density was about 50\%
  higher than on Lorna Glen.~
\item
  The field work component of~a PhD project to investigate interactions
  between wild dogs/dingoes and wild cats~is complete and data analysis
  and write-up are underway. The study is testing the hypothesis that
  there is an inverse relationship between dog and cat density. The
  management implication is that retaining dingoes could result in a
  reduction in cats.
\item
  A report on 10 years of monitoring vertebrate fauna has been completed
  and shows that some taxa have increased in abundance, possibly in
  response to management actions. This work is being prepared for
  publication.
\item
  Mulgara (\emph{Dasycercus cristicauda}) population has declined on
  Lorna Glen, but is still significantly higher than before baiting
  commenced in 2003 and is~about double~the population on Earaheedy.
\item
  The fire management plan continued to be implemented, including
  further installation of fuel-reduced buffers around some fire
  management cells and some core ignition using aircraft. A wildfire
  started by lightning in late September 2014, and which had the
  potential to threaten the predator exclusion compound containing
  threatened fauna,~was stopped by the~buffer burning.
\end{itemize}




\section*{Management implications}

\begin{itemize}
\itemsep1pt\parskip0pt\parsep0pt
\item
  This project is providing insurance populations of threatened arid
  zone mammals.
\item
  Information will inform guidelines for the proactive management of
  fire in the arid zone rangelands to reduce the severity (scale and
  intensity) of wildfires and to provide habitat choice through mosaic
  burning.
\item
  Guidelines for controlling introduced predators in the arid zone
  rangelands will reduce this threat to native fauna. Reintroduction and
  protection of mammals of the arid zone rangelands, other extant fauna,
  vegetation and other elements of the biota will provide reconstruction
  of animal and plant assemblages in an arid zone ecosystem.
\item
  A framework and protocol for assessing and reporting trends in
  ecosystem condition in arid zone rangelands will provide a basis for
  ecosystem condition monitoring.
\end{itemize}




\section*{Future directions}

\begin{itemize}
\itemsep1pt\parskip0pt\parsep0pt
\item
  Assess and report on the effectiveness of wild cat and dog baiting to
  be undertaken in July 2015. Trail cameras will be evaluated for their
  utility for assessing predator density before and after baiting.
\item
  Prepare a paper for publication reporting on 10 years of biodiversity
  monitoring on Lorna Glen.
\item
  Carry out a biological survey of Earaheedy.
\item
  Survey wild dogs, cats and mulgara on Earaheedy where there has been
  no introduced predator control, and compare results with Lorna Glen.
\item
  Continue to implement the fire management plan including buffer
  burning and aerial patch burning. Carry out patch-burning in the
  predator-proof compound.
\end{itemize}



%-----------------------------------------------------------------------------%
% Back matter
%\backmatter
\end{document}
%-----------------------------------------------------------------------------%

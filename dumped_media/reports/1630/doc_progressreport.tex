
\documentclass[version=last,
    paper=a4, % paper size
    10pt, % default font size
    usenames,
    dvipsnames,
    oneside, % ONLINE
    headings=openany, % open chapters on odd and even pages
    %toc=chapterentrywithdots, % Table of Contents style
    %BCOR=7mm, % PRINT Binding Correction
    %DIV=13, % typearea 161.54 mm x 228.46 mm, top margin 22.85 mm, inner margin 16.15 mm
    %DIV=14, % 165.00 233.36 21.21 15.00
    DIV=15 % 168.00 237.60 19.80 14.00
]{scrbook}
\usepackage{typearea}
\usepackage[automark,headsepline,footsepline]{scrlayer-scrpage} % Headers and footers

%%
%% Fonts, encoding, spacing, indentation
%%
\usepackage{txfonts}
\renewcommand{\familydefault}{\sfdefault} % Default to Sans Serif font
\usepackage[english]{babel}
\usepackage[T1]{fontenc}
\usepackage[utf8]{inputenc}

% Paragraph spacing
%\usepackage{parskip}    % Paragraph spacing
%\setlength{\parindent}{0em} % Don't indent paragraphs - ONLINE
%\setlength{\parskip}{1.3 ex plus 0.5ex minus 0.3ex} % 1-1.8 ex vertical space between paragraphs - ONLINE

% Spacing of headings
%\RedeclareSectionCommand[afterskip=3pt]{section} % less space after section
%\RedeclareSectionCommand[beforeskip=0cm]{subsection} % less space between HRule and project name
%\RedeclareSectionCommand[afterskip=0.1\baselineskip]{subsubsection} % less space after progressreport subheadings

% Table font size
\usepackage{etoolbox}
\AtBeginEnvironment{longtabu}{\footnotesize}{}{}

%%
%% Tables, columns, layout
%%
\usepackage{multicol}   % 2 col publications
\usepackage{pdflscape}  % Landscape pages
\usepackage{pdfpages}   % Include PDFs
\usepackage{hanging}    % hanging paragraphs for publications
%\usepackage{titletoc}   % Required for manipulating the table of contents
\setcounter{tocdepth}{2} % TOC list down to section
\usepackage{enumerate}  % Enumerations
\usepackage{enumitem}   % Enumerations
\usepackage{longtable}  % Multipage table
\usepackage{tabu}       %
\setlength{\tabulinesep}{1.5mm} % Consistent vertical spacing in tabu

%%
%% Graphics, images, colours
%%
\usepackage{graphicx} % embedded images
\usepackage{eso-pic} %
\usepackage{colortbl} % define custom named colours
\definecolor{RedFire}{RGB}{146,25,28}
\definecolor{ParksWildlife}{RGB}{0,85,144}
\definecolor{successbg}{RGB}{223,240,216}
\definecolor{errorbg}{RGB}{242,222,222}
\definecolor{warningbg}{RGB}{252,248,227}
\definecolor{infobg}{RGB}{217,237,247}
\definecolor{muted}{RGB}{153,153,153}
\definecolor{success}{RGB}{70,136,71}
\definecolor{error}{RGB}{185,74,72}
\definecolor{warning}{RGB}{192,152,83}
\definecolor{info}{RGB}{58,135,173}

\definecolor{required}{RGB}{192,152,83}
\definecolor{requiredbg}{RGB}{252,248,227}
\definecolor{denied}{RGB}{185,74,72}
\definecolor{deniedbg}{RGB}{242,222,222}
\definecolor{granted}{RGB}{70,136,71}
\definecolor{grantedbg}{RGB}{223,240,216}
\definecolor{not reqiured}{RGB}{153,153,153}
\definecolor{not requiredbg}{RGB}{255,255,255}

\usepackage{tikz} % Drawing
\usetikzlibrary{arrows,shapes,positioning,shadows,trees}

%%
%% Links, URLs
%%
\usepackage[
    linktoc=all,
    %colorlinks=false,  %PRINT
    colorlinks=true, % ONLINE
    linkcolor=RedFire, % ONLINE
    urlcolor=ParksWildlife, % ONLINE
    pdftitle=Progress Report SP 2012-002 (FY 2015-2016)
]{hyperref}

% Black magic to linebreak URLs
\usepackage{url}
\makeatletter
\g@addto@macro{\UrlBreaks}{\UrlOrds}
\makeatother

%%
%% Custom macros
%%
% Thick Horizontal rule
\newcommand{\HRule}{\vspace{8mm}\\\noindent\rule{\linewidth}{0.1pt}}

% Custom Tikz node for SDS diagram
\newcommand\mynode[6][]{
    \node[#1] (#2){
        \parbox{#3\relax}{
            \begin{center}
            \textbf{#4}\\
            #5\\
            \footnotesize{#6}
            \end{center}}};}



%-----------------------------------------------------------------------------%
% Headers and Footers
\automark{section}
\ohead{\href{http://sdis.dpaw.wa.gov.au/documents/progressreport/1630/}{Progress Report SP 2012-002
}}
\chead{\href{http://sdis.dpaw.wa.gov.au}{SDIS}} % center header ONLINE
\ihead{\href{http://sdis.dpaw.wa.gov.au}{
        \includegraphics[scale=0.4]{/mnt/projects/sdis/staticfiles/img/logo-dpaw.png}}}
\ifoot{\textbf{Printed}~Mon, 11 Jul 2016 10:11:30 +0800} % inner/left footer
\cfoot{} % center footer
\ofoot{\pagemark} % outer/right footer
\pagestyle{scrheadings}
\setkomafont{pageheadfoot}{\normalfont}

%-----------------------------------------------------------------------------%
\begin{document}
\raggedbottom

%-----------------------------------------------------------------------------%
% Title page
\subject{Progress Report SP 2012-002
}
\title{Climate-resilient vegetation of multi-use landscapes: exploiting genetic
variability in widespread species
}
\subtitle{Ecosystem Science
}
\author{}
\publishers{\small
    \subsection*{Project Core Team}
\begin{tabu} {X X}
\textbf{Supervising Scientist} & Margaret Byrne
\\
\textbf{Data Custodian} & 
\\
\textbf{Site Custodian} & 
\\
\end{tabu}


    \subsection*{Project status as of July 11, 2016, 10:11 a.m.}
\begin{tabu} {X X}
& Approved and active
\\
\end{tabu}

    
\subsection*{Document endorsements and approvals as of July 11, 2016, 10:11 a.m.}
\begin{tabu} {X X}

%\rowcolor{grantedbg}
    \textbf{Project Team} & 
    \textcolor{granted}{ granted}\\

%\rowcolor{grantedbg}
    \textbf{Program Leader} & 
    \textcolor{granted}{ granted}\\

%\rowcolor{grantedbg}
    \textbf{Directorate} & 
    \textcolor{granted}{ granted}\\

\end{tabu}



}
\uppertitleback{}
\lowertitleback{}
\date{}

%-----------------------------------------------------------------------------%
% Front matter
\frontmatter
\maketitle
%-----------------------------------------------------------------------------%
% Main matter
\mainmatter

\section*{Climate-resilient vegetation of multi-use landscapes: exploiting genetic
variability in widespread species
}

M Byrne, C Yates, B Macdonald, R Binks, L McLean, Dr S Prober (CSIRO),
Prof W Stock (Edith Cowan University), Prof B Potts (University of
Tasmania), A/Prof R Vaillancourt (University of Tasmania), Dr D Steane
(University of Tasmania)


\section*{Context}
Multi-million dollar investments in the restoration of Australia's
degraded and fragmented multi-use landscapes currently take little
account of climate change. Until recently there has been a strong focus
on maintaining local genetic patterns for optimal restoration. In a
changing climate this paradigm may no longer be relevant and a new
framework is urgently needed. The proposed project will deliver such a
framework by undertaking pioneering research and development at the
interface between molecular genetics, plant physiology and climate
adaptation. Specifically, it will test hypotheses of adaptation in
widespread eucalypt species, by investigating correlations between key
physiological traits and genetic signatures of adaptation across
climatic gradients utilising recent advances in eucalypt genomics.
Addressing this question will ensure optimal, climate-resilient outcomes
for Australia-wide investment in ecological restoration, offering a
novel solution to ecosystem adaptation in changing environments.



\section*{Aims}
The project will test the following alternative hypotheses:

\begin{itemize}
\itemsep1pt\parskip0pt\parsep0pt
\item
  Widespread species, having evolved under highly variable environments,
  retain high potential for adaptability to environmental change within
  the gene pool of local populations or individuals (implying that
  genetic material sourced from local populations will have tolerance to
  changing climatic conditions).
\item
  Widespread species, having evolved across wide ecological gradients,
  comprise a suite of locally adapted sub-populations (implying that
  genetic material should be sourced not from local populations but from
  distant and potentially resilient populations that are pre-adapted to
  the future climate).
\end{itemize}



\section*{Progress}
\begin{itemize}
\itemsep1pt\parskip0pt\parsep0pt
\item
  Analysis of genomic variation and environmental traits in nine
  populations across a climate gradient has been completed in
  \emph{Eucalyptus loxophleba} with evidence of 50 outlier markers
  associated with climate adaptation.
\item
  A paper describing the genomic architecture of climate adaptation in
  three species, \emph{E. salubris}and\emph{E. loxophleba}from Western
  Australia and \emph{E. tricarpa} from Victoria has been written and is
  in review.
\item
  To further explore geographic patterns in the two genetic lineages
  detected in~\emph{E. salubris}, 13 additional populations have
  been~collected across the species' distribution and genomic sequencing
  completed. The raw genomic data is currently under analysis.~
\end{itemize}



\section*{Management implications}
The findings of both genetic adaption to local conditions and capacity
for plastic responses highlight the complex nature of climate
adaptation. Widespread eucalypts are therefore likely to be able to
adjust to a changing climate to some extent, but selection of seed
sources to match projected climate changes may confer greater climate
resilience in environmental plantings. A strategy of `climate-adjusted
provenancing' with seed sources biased toward the direction of predicted
climatic change is recommended for restoration and revegetation.



\section*{Future directions}
\begin{itemize}
\itemsep1pt\parskip0pt\parsep0pt
\item
  Complete final scientific paper on the genomic architecture of
  adaptation.~
\item
  Complete analysis of lineage divergence in \emph{E. salubris}.
\end{itemize}

~



%-----------------------------------------------------------------------------%
% Back matter
%\backmatter
\end{document}
%-----------------------------------------------------------------------------%



\documentclass[version=last,
    paper=a4, % paper size
    10pt, % default font size
    usenames,
    dvipsnames,
    oneside, % ONLINE
    headings=openany, % open chapters on odd and even pages
    %toc=chapterentrywithdots, % Table of Contents style
    %BCOR=7mm, % PRINT Binding Correction
    %DIV=13, % typearea 161.54 mm x 228.46 mm, top margin 22.85 mm, inner margin 16.15 mm
    %DIV=14, % 165.00 233.36 21.21 15.00
    DIV=15 % 168.00 237.60 19.80 14.00
]{scrbook}
\usepackage{typearea}
\usepackage[automark,headsepline,footsepline]{scrlayer-scrpage} % Headers and footers

%%
%% Fonts, encoding, spacing, indentation
%%
\usepackage{txfonts}
\renewcommand{\familydefault}{\sfdefault} % Default to Sans Serif font
\usepackage[english]{babel}
\usepackage[T1]{fontenc}
\usepackage[utf8]{inputenc}

% Paragraph spacing
%\usepackage{parskip}    % Paragraph spacing
%\setlength{\parindent}{0em} % Don't indent paragraphs - ONLINE
%\setlength{\parskip}{1.3 ex plus 0.5ex minus 0.3ex} % 1-1.8 ex vertical space between paragraphs - ONLINE

% Spacing of headings
%\RedeclareSectionCommand[afterskip=3pt]{section} % less space after section
%\RedeclareSectionCommand[beforeskip=0cm]{subsection} % less space between HRule and project name
%\RedeclareSectionCommand[afterskip=0.1\baselineskip]{subsubsection} % less space after progressreport subheadings

% Table font size
\usepackage{etoolbox}
\AtBeginEnvironment{longtabu}{\footnotesize}{}{}

%%
%% Tables, columns, layout
%%
\usepackage{multicol}   % 2 col publications
\usepackage{pdflscape}  % Landscape pages
\usepackage{pdfpages}   % Include PDFs
\usepackage{hanging}    % hanging paragraphs for publications
%\usepackage{titletoc}   % Required for manipulating the table of contents
\setcounter{tocdepth}{2} % TOC list down to section
\usepackage{enumerate}  % Enumerations
\usepackage{enumitem}   % Enumerations
\usepackage{longtable}  % Multipage table
\usepackage{tabu}       %
\setlength{\tabulinesep}{1.5mm} % Consistent vertical spacing in tabu

%%
%% Graphics, images, colours
%%
\usepackage{graphicx} % embedded images
\usepackage{eso-pic} %
\usepackage{colortbl} % define custom named colours
\definecolor{RedFire}{RGB}{146,25,28}
\definecolor{ParksWildlife}{RGB}{0,85,144}
\definecolor{successbg}{RGB}{223,240,216}
\definecolor{errorbg}{RGB}{242,222,222}
\definecolor{warningbg}{RGB}{252,248,227}
\definecolor{infobg}{RGB}{217,237,247}
\definecolor{muted}{RGB}{153,153,153}
\definecolor{success}{RGB}{70,136,71}
\definecolor{error}{RGB}{185,74,72}
\definecolor{warning}{RGB}{192,152,83}
\definecolor{info}{RGB}{58,135,173}

\definecolor{required}{RGB}{192,152,83}
\definecolor{requiredbg}{RGB}{252,248,227}
\definecolor{denied}{RGB}{185,74,72}
\definecolor{deniedbg}{RGB}{242,222,222}
\definecolor{granted}{RGB}{70,136,71}
\definecolor{grantedbg}{RGB}{223,240,216}
\definecolor{not reqiured}{RGB}{153,153,153}
\definecolor{not requiredbg}{RGB}{255,255,255}

\usepackage{tikz} % Drawing
\usetikzlibrary{arrows,shapes,positioning,shadows,trees}

%%
%% Links, URLs
%%
\usepackage[
    linktoc=all,
    %colorlinks=false,  %PRINT
    colorlinks=true, % ONLINE
    linkcolor=RedFire, % ONLINE
    urlcolor=ParksWildlife, % ONLINE
    pdftitle=Progress Report SP 2014-024 (FY 2015-2016)
]{hyperref}

% Black magic to linebreak URLs
\usepackage{url}
\makeatletter
\g@addto@macro{\UrlBreaks}{\UrlOrds}
\makeatother

%%
%% Custom macros
%%
% Thick Horizontal rule
\newcommand{\HRule}{\vspace{8mm}\\\noindent\rule{\linewidth}{0.1pt}}

% Custom Tikz node for SDS diagram
\newcommand\mynode[6][]{
    \node[#1] (#2){
        \parbox{#3\relax}{
            \begin{center}
            \textbf{#4}\\
            #5\\
            \footnotesize{#6}
            \end{center}}};}



%-----------------------------------------------------------------------------%
% Headers and Footers
\automark{section}
\ohead{\href{http://sdis.dpaw.wa.gov.au/documents/progressreport/1596/}{Progress Report SP 2014-024
}}
\chead{\href{http://sdis.dpaw.wa.gov.au}{SDIS}} % center header ONLINE
\ihead{\href{http://sdis.dpaw.wa.gov.au}{
        \includegraphics[scale=0.4]{/mnt/projects/sdis/staticfiles/img/logo-dpaw.png}}}
\ifoot{\textbf{Printed}~Tue, 5 Jul 2016 15:31:03 +0800} % inner/left footer
\cfoot{} % center footer
\ofoot{\pagemark} % outer/right footer
\pagestyle{scrheadings}
\setkomafont{pageheadfoot}{\normalfont}

%-----------------------------------------------------------------------------%
\begin{document}
\raggedbottom

%-----------------------------------------------------------------------------%
% Title page
\subject{Progress Report SP 2014-024
}
\title{Understanding peat wetland resilience: evaluating the impact of climate
and landuse change on the hydrodynamics and hydrogeochemistry of peat
wetlands in the Warren (Muir-Byenup) District
}
\subtitle{Wetlands Conservation
}
\author{}
\publishers{\small
    \subsection*{Project Core Team}
\begin{tabu} {X X}
\textbf{Supervising Scientist} & Adrian Pinder
\\
\textbf{Data Custodian} & Adrian Pinder
\\
\textbf{Site Custodian} & Adrian Pinder
\\
\end{tabu}


    \subsection*{Project status as of July 5, 2016, 3:31 p.m.}
\begin{tabu} {X X}
& Approved and active
\\
\end{tabu}

    
\subsection*{Document endorsements and approvals as of July 5, 2016, 3:31 p.m.}
\begin{tabu} {X X}

%\rowcolor{grantedbg}
    \textbf{Project Team} & 
    \textcolor{granted}{ granted}\\

%\rowcolor{grantedbg}
    \textbf{Program Leader} & 
    \textcolor{granted}{ granted}\\

%\rowcolor{grantedbg}
    \textbf{Directorate} & 
    \textcolor{granted}{ granted}\\

\end{tabu}



}
\uppertitleback{}
\lowertitleback{}
\date{}

%-----------------------------------------------------------------------------%
% Front matter
\frontmatter
\maketitle
%-----------------------------------------------------------------------------%
% Main matter
\mainmatter

\section*{Understanding peat wetland resilience: evaluating the impact of climate
and landuse change on the hydrodynamics and hydrogeochemistry of peat
wetlands in the Warren (Muir-Byenup) District
}

J Rutherford


\section*{Context}
Peat wetlands are relatively rare in Western Australia but constitute an
important habitat for biodiversity where they occur, especially in the
far south-west of the State, providing refugia from seasonal and
long-term drying for a range of restricted flora and fauna. Some peat
wetlands in the Muir-Byenup Ramsar wetland suite are threatened by
acidification and some have already acidified as a result of declining
groundwater levels. Drying is also making these organic wetlands much
more prone to catastrophic fires. The peat also stores a range of toxic
metals and metalloids which are released to the environment as they dry.
The major aim of this project is to undertake a risk assessment of fire
susceptibility and release of acidity and other contaminants. The
project will deliver a map of the distribution of at-risk peat wetlands,
combined with recommendations for fire management and maintaining water
balance.



\section*{Aims}
\begin{itemize}
\itemsep1pt\parskip0pt\parsep0pt
\item
  To determine current hydrogeological and hydrochemical conditions of
  four representative peat wetlands (eg water and chemical conditions
  and gradients)
\item
  To map and quantify peat wetland carbon and acid stores
\item
  To identify and assess the transient behaviour of major threats to the
  health of the peat wetlands studied (eg role of drying climate in
  changing water retention in peat sediments and the source and mobility
  of acidity and salinity)
\end{itemize}



\section*{Progress}
\begin{itemize}
\itemsep1pt\parskip0pt\parsep0pt
\item
  High resolution groundwater monitoring on shorelines of three wetland
  types showed that resilience was higher where wetlands were discrete
  in size, rainfall-runoff response was high and aquifers beneath the
  wetlands were less dependent on local recharge to maintain high (near
  surface) water levels.
\item
  All wetlands studied had high carbon and acid stores, which were
  released during sequential wetting and drying of lake sediments, in
  both summer and winter months,~
\item
  Laboratory analyses of peat sediments confirmed they have sufficient
  buffering capacity to neutralise acidity, but acidity prevailed where
  changes in surface and groundwater hydrology resulted in lower water
  retention in the peat sediments.~
\end{itemize}



\section*{Management implications}
\begin{itemize}
\itemsep1pt\parskip0pt\parsep0pt
\item
  It is likely that small changes in water balance make a significant
  difference to peat wetland health.~~Management of surrounding
  vegetation will impact water balance and consequently peat wetland
  health.
\item
  Stores of organic carbon are significant to depths of
  \textasciitilde{}1.5m and persist as wetland substrates dry.
  Understanding the spatial variability of carbon stores and rate of
  decline will assist in predicting fire risk.
\end{itemize}



\section*{Future directions}
\begin{itemize}
\itemsep1pt\parskip0pt\parsep0pt
\item
  Undertake mineralogical assessment of acidity to incorporate into
  hydrogeochemical models to assess affects of wetting and drying under
  the current climate and make predictions under different conditions.
\item
  Develop a three dimensional understanding of the water, salt and acid
  stores in order to numerically model the spatial variability and
  identify wetlands (or areas within wetlands) where interventions are
  likely to be effective and the risk of peat fires reduced.
\end{itemize}



%-----------------------------------------------------------------------------%
% Back matter
%\backmatter
\end{document}
%-----------------------------------------------------------------------------%


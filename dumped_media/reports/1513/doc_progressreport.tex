
\documentclass[version=last, 
    paper=a4, % paper size
    10pt, % default font size
    usenames,
    dvipsnames, 
    oneside, % ONLINE
    headings=openany, % open chapters on odd and even pages
    %toc=chapterentrywithdots, % Table of Contents style
    %BCOR=7mm, % PRINT Binding Correction
    %DIV=13, % typearea 161.54 mm x 228.46 mm, top margin 22.85 mm, inner margin 16.15 mm
    %DIV=14, % 165.00 233.36 21.21 15.00
    DIV=15 % 168.00 237.60 19.80 14.00
]{scrbook}
\usepackage{typearea}
\usepackage[automark,headsepline,footsepline]{scrlayer-scrpage} % Headers and footers

%%
%% Fonts, encoding, spacing, indentation
%%
\usepackage{txfonts}
\renewcommand{\familydefault}{\sfdefault} % Default to Sans Serif font
\usepackage[english]{babel}
\usepackage[T1]{fontenc}
\usepackage[utf8]{inputenc}

% Paragraph spacing
%\usepackage{parskip}    % Paragraph spacing
%\setlength{\parindent}{0em} % Don't indent paragraphs - ONLINE
%\setlength{\parskip}{1.3 ex plus 0.5ex minus 0.3ex} % 1-1.8 ex vertical space between paragraphs - ONLINE

% Spacing of headings
%\RedeclareSectionCommand[afterskip=3pt]{section} % less space after section
%\RedeclareSectionCommand[beforeskip=0cm]{subsection} % less space between HRule and project name
%\RedeclareSectionCommand[afterskip=0.1\baselineskip]{subsubsection} % less space after progressreport subheadings

% Table font size
\usepackage{etoolbox}
\AtBeginEnvironment{longtabu}{\footnotesize}{}{}

%%
%% Tables, columns, layout
%%
\usepackage{multicol}   % 2 col publications
\usepackage{pdflscape}  % Landscape pages
\usepackage{pdfpages}   % Include PDFs
\usepackage{hanging}    % hanging paragraphs for publications
%\usepackage{titletoc}   % Required for manipulating the table of contents
\setcounter{tocdepth}{2} % TOC list down to section
\usepackage{enumerate}  % Enumerations
\usepackage{enumitem}   % Enumerations
\usepackage{longtable}  % Multipage table
\usepackage{tabu}       % 
\setlength{\tabulinesep}{1.5mm} % Consistent vertical spacing in tabu

%%
%% Graphics, images, colours
%%
\usepackage{graphicx} % embedded images
\usepackage{eso-pic} % 
\usepackage{colortbl} % define custom named colours
\definecolor{RedFire}{RGB}{146,25,28}
\definecolor{ParksWildlife}{RGB}{0,85,144}
\definecolor{successbg}{RGB}{223,240,216}
\definecolor{errorbg}{RGB}{242,222,222}
\definecolor{warningbg}{RGB}{252,248,227}
\definecolor{infobg}{RGB}{217,237,247}
\definecolor{muted}{RGB}{153,153,153}
\definecolor{success}{RGB}{70,136,71}
\definecolor{error}{RGB}{185,74,72}
\definecolor{warning}{RGB}{192,152,83}
\definecolor{info}{RGB}{58,135,173}
\definecolor{required}{HTML}{D9534F}
\definecolor{denied}{HTML}{D9534F}
\definecolor{granted}{HTML}{47A447}
\definecolor{not required}{RGB}{200, 200, 200}

\usepackage{tikz} % Drawing
\usetikzlibrary{arrows,shapes,positioning,shadows,trees}

%%
%% Links, URLs
%%
\usepackage[
    linktoc=all,
    %colorlinks=false,  %PRINT
    colorlinks=true, % ONLINE
    linkcolor=RedFire, % ONLINE
    urlcolor=ParksWildlife, % ONLINE
    pdftitle=doc\_progressreport.pdf
]{hyperref}

% Black magic to linebreak URLs
\usepackage{url}
\makeatletter
\g@addto@macro{\UrlBreaks}{\UrlOrds}
\makeatother

%%
%% Custom macros
%%
% Thick Horizontal rule
\newcommand{\HRule}{\vspace{8mm}\\\noindent\rule{\linewidth}{0.1pt}}

% Custom Tikz node for SDS diagram
\newcommand\mynode[6][]{\node[#1] (#2){\parbox{#3\relax}{\begin{center}\textbf{#4}\\#5\\\footnotesize{#6}\end{center}}};}




%-----------------------------------------------------------------------------%
% Headers and Footers
\automark{section}
\ohead{\href{http://sdis.dpaw.wa.gov.au/documents/progressreport/1513/}{Progress Report 2015-17 (FY 2014-2015)}}
\chead{\href{http://sdis.dpaw.wa.gov.au}{SDIS}} % center header ONLINE
\ihead{\href{http://sdis.dpaw.wa.gov.au}{
        \includegraphics[scale=0.4]{/mnt/projects/sdis/staticfiles/img/logo-dpaw.png}}}
\ifoot{\textbf{Printed}~Thu, 21 Jan 2016 10:05:51 +0800} % inner/left footer
\cfoot{} % center footer
\ofoot{\pagemark} % outer/right footer
\pagestyle{scrheadings}
\setkomafont{pageheadfoot}{\normalfont}

%-----------------------------------------------------------------------------%
\begin{document}
\raggedbottom

%-----------------------------------------------------------------------------%
% Title page
\subject{Progress Report 2015-17 (FY 2014-2015)}
\title{Science project 2015-17 Responses of aquatic invertebrate communities to
changing hydrology and water quality in streams and significant wetlands
of the south-west forests of Western Australia.
}
\subtitle{Wetland Conservation
}
\author{}
\publishers{
\subsection*{Endorsements and approvals as of Jan. 21, 2016, 10:05 a.m.}
\begin{tabu} {X X}
\hline

\rowcolor{granted}
Project Team & granted\\
\hline

\rowcolor{granted}
Program Leader & granted\\
\hline

\rowcolor{required}
Directorate & required\\
\hline

\end{tabu}

}
    \uppertitleback{}
\lowertitleback{\begin{center}\textbf{Printed}~Thu, 21 Jan 2016 10:05:51 +0800\end{center}}
\date{}

%-----------------------------------------------------------------------------%
% Front matter
\frontmatter
\maketitle
%-----------------------------------------------------------------------------%
% Main matter
\mainmatter

\section*{Science project 2015-17 Responses of aquatic invertebrate communities to
changing hydrology and water quality in streams and significant wetlands
of the south-west forests of Western Australia.
}

\section*{Context}
Aquatic habitats in the south-west of WA are under increasing threat
from changes in hydrology, water quality and fire as a result of the
drying climate and historical and current land use. The south west of
Western Australia has had a significant reduction in rainfall since the
1970s and~it is predicted that by 2050 there will be little stream
inflow into water supply dams. At present, there is an inadequate
understanding of the responses of aquatic communities to these threats
to inform the management of many aquatic systems in the Forest
Management Plan (FMP) area, including the Muir-Byenup Ramsar wetlands.~

This project has two components: 1) Re-surveys of aquatic invertebrates
in Muir-Byenup Ramsar wetlands sampled in 1994 and 2004 and suites of
wetlands further south sampled in 1993 addressing KPI3 of the 2014-23
FMP and, 2) Continued monitoring of high condition streams, with a focus
on effects of the drying climate and forest management, addressing KPI1
of the 2014-23 FMP.

~



\section*{Aims}
\begin{itemize}
\item
  To address KPI1 of the 2014-2023 FMP by monitoring the condition of
  currently healthy streams in relation to reduced rainfall and forest
  management practices.
\item
  To address KPI3 of the 2014-2023 FMP by determining responses of
  faunas of high value Warren region wetlands to changes in hydrology,
  water chemistry and fire over the last 10 to 20 years.
\item
  Provide baseline data for some internationally significant wetlands,
  e.g. Lake Muir.
\item
  Use the above information to report on the current conservation
  significance of key DPaW managed wetlands and their response and
  vulnerability to threats.
\end{itemize}



\section*{Progress}
\begin{itemize}
\itemsep1pt\parskip0pt\parsep0pt
\item
  A journal article is being prepared in collaboration with scientists
  from CSIRO in Canberra: ``Whole of landscape modelling of
  compositional turnover in aquatic macro-invertebrates informs
  conservation gap analysis: an example from south-west Western
  Australia.''
\item
  Conducted spring 2014 and summer 2015 sampling of aquatic
  invertebrates in Muir-Byenup Ramsar wetlands.
\item
  Commenced processing Muir-Byenup invertebrate samples.
\end{itemize}



\section*{Management implications}
\begin{itemize}
\itemsep1pt\parskip0pt\parsep0pt
\item
  Re-surveying the Muir-Byenup Ramsar~and other high value wetlands will
  provide the region with knowledge of how these wetlands have responded
  to threats over the last 20 years. This, in conjunction with results
  from the peat wetlands project (SPP2014-24), will help the Warren
  Region to make decisions about protecting remaining high conservation
  value wetlands versus taking remedial action at those where condition
  is declining.
\item
  Forest Management Plan commitments will be met with regard to
  measuring and assessing change in condition of 1) currently healthy
  (reference condition) stream ecosystems (KPI1) and 2) Ramsar and
  nationally listed wetlands (KPI3). Results of these will inform future
  forest management practices.
\end{itemize}



\section*{Future directions}
\begin{itemize}
\itemsep1pt\parskip0pt\parsep0pt
\item
  Identify Muir-Byenup invertebrates collected~in 2014/2015.
\item
  Publish report with summaries of 10 year trends (2005 to 2015) for all
  stream monitoring sites.
\item
  Re-sample streams in 2016, with a focus on those considered to be in
  reference condition or in minimally disturbed catchments, to provide
  long-term data on the response of aquatic invertebrate communities to
  declining rainfall and forest management.
\item
  Continue to up date fire and logging history for catchment areas.
\item
  Publish further papers examining impacts of declining rainfall and
  forest management practices on macroinvertebrate diversity in forest
  streams.
\item
  Re-survey nationally important Warren Region wetlands previously
  sampled by Horwtiz in 1997 (e.g. Owingup, Lake Jasper, Doggerup,
  Marringup, Mt.Soho Swamp) and identified as priorities in the Warren
  Region Nature Conservation Plan.
\end{itemize}



%-----------------------------------------------------------------------------%
% Back matter
%\backmatter
\end{document}
%-----------------------------------------------------------------------------%



\documentclass[version=last, paper=a4, DIV=18, usenames, dvipsnames]{scrartcl}
\usepackage{txfonts}
\usepackage{pdflscape}
\usepackage{pdfpages}
\usepackage[english]{babel} % English language/hyphenation
%%% Bootstrap colors
\definecolor{RedFire}{RGB}{146,25,28}
\definecolor{ParksWildlife}{RGB}{0,85,144}
\definecolor{successbg}{RGB}{223,240,216}
\definecolor{errorbg}{RGB}{242,222,222}
\definecolor{warningbg}{RGB}{252,248,227}
\definecolor{infobg}{RGB}{217,237,247}
\definecolor{muted}{RGB}{153,153,153}
\definecolor{success}{RGB}{70,136,71}
\definecolor{error}{RGB}{185,74,72}
\definecolor{warning}{RGB}{192,152,83}
\definecolor{info}{RGB}{58,135,173}

\definecolor{required}{HTML}{D9534F}
\definecolor{denied}{HTML}{D9534F}
\definecolor{granted}{HTML}{47A447}
\definecolor{not required}{RGB}{200, 200, 200}

\usepackage[colorlinks=true,pdftitle=doc\_studentreport.pdf
,linktoc=all,linkcolor=RedFire,urlcolor=ParksWildlife]{hyperref}
\usepackage{colortbl}
\usepackage{longtable}
\usepackage{tabu}
\setlength{\tabulinesep}{1.5mm}
\usepackage{enumerate}
\usepackage{enumitem}
\usepackage{fancyhdr}
\usepackage{lastpage}
\usepackage{graphicx}
\usepackage{eso-pic}
\usepackage{hyphenat}
\renewcommand{\familydefault}{\sfdefault}



\newcommand{\HRule}{\rule{\linewidth}{0.1pt}}

\newcommand{\placetextbox}[3]{% \placetextbox{<horizontal pos>}{<vertical pos>}{<stuff>}
  \setbox0=\hbox{#3}% Put <stuff> in a box
  \AddToShipoutPictureFG*{% Add <stuff> to current page foreground
    \put(\LenToUnit{#1\paperwidth},\LenToUnit{#2\paperheight}){\vtop{{\null}\makebox[0pt][c]{#3}}}%
  }%
}%




%-----------------------------------------------------------------------------%
% Headers and footers
%
\fancypagestyle{plain}{
  \fancyhf{}
  \setlength\headheight{60pt} % push page content below header
  \renewcommand{\headrulewidth}{0.1pt}
  \renewcommand{\footrulewidth}{0.1pt}
  
  
  \fancyhead[L]{ 
    \href{http://sdis.dpaw.wa.gov.au}{
    \includegraphics[scale=0.6]{/mnt/projects/sdis/staticfiles/img/logo-dpaw.png}}
  }
  \fancyhead[R]{ 
      \hfill
      \href{http://sdis.dpaw.wa.gov.au}{Science Directorate Information System} 
      \newline 
      \href{http://sdis.dpaw.wa.gov.au/documents/studentreport/1434/}{Progress Report 2014-10 (FY 2014-2015)} 
  }
  
  
  
  
  \fancyfoot[L]{ \leftmark\newline\textbf{Printed}\textit{ June 24, 2015, 3:51 p.m. }}
  \fancyfoot[R]{  \, \newline Page \thepage\ of \pageref{LastPage} }
  
  
}
\pagestyle{plain}
%
% end Headers
%-----------------------------------------------------------------------------%

\begin{document}

%-----------------------------------------------------------------------------%
% Title page
%

%
% end title page
%-----------------------------------------------------------------------------%




\section*{Progress Report}
Infectious disease has been suggested as a factor contributing to the
recent 90\% decline of the woylie, now critically endangered. The
effects of infectious disease on woylies may be exacerbated by as yet
unknown factors such as stress. This project aims to investigate how
stress affects immune function and patterns of infection in the context
of endangered species conservation. The hypothesis is: if stress affects
immune function and patterns of infection in woylies, we expect changes
in immunological variables and patterns of parasite infection with
varying exposure to conservation relevant stressors. Endangered species
face numerous threats that can constitute stressors that challenge an
animals' physiological balance. Stressors to be investigated in this
study include predators, resource availability, social interactions,
population density and translocation.

Extensive field and laboratory work will be performed to investigate
links between stress and disease expression in woylies in the context of
their decline. Study populations include captive and free-ranging
woylies at Native Animal Rescue, Karakamia Sanctuary, Whiteman Park and
the Upper Warren region. Diagnostic and laboratory methods will be
applied to conduct parallel evaluation of stress hormones, immune
function and parasites in woylies. Data will be used to develop models
to improve our understanding of how stress affects the health of
endangered wildlife and potential ramifications for species conservation
and management.

In the project's first year, intensive sample collection and preliminary
analyses has been undertaken and the first paper accepted for
publication. Fieldwork, sample collection, laboratory analyses and
dissemination of information will continue with support from the
Australian Academy of Science Margaret Middleton Fund, Foundation for
National Parks and Wildlife and Holsworth Research Endowment.




\clearpage



\end{document}


\documentclass[version=last,
    paper=a4,                               % paper size
    10pt,                                   % default font size
    dvipsnames,
    % twoside,                                % PRINT Binding Correction
    oneside,                              % ONLINE
    headings=openany,                       % open chapters on odd and even pages
    open=any,
    BCOR=7mm,                               % PRINT Binding Correction
    %DIV=13,    % typearea 161.54mm x 228.46mm, top 22.85mm, inner 16.15mm
    %DIV=14,    % 165.00 233.36 21.21 15.00
    DIV=15,     % 168.00 237.60 19.80 14.00
    % toc=chapterentrywithdots              % Table of Contents style
]{scrbook}
\usepackage{typearea}


%------------------------------------------------------------------------------%
% Headers and footers
%------------------------------------------------------------------------------%
\usepackage[automark,headsepline,footsepline,plainfootsepline]{scrlayer-scrpage}
\automark*[section]{}
\addtokomafont{pageheadfoot}{\normalfont\footnotesize\sffamily} % Don't italicise
\renewcommand{\chaptermark}[1]{\markleft{#1}{}}     % Chapter: suppress numbering
\renewcommand{\sectionmark}[1]{\markright{#1}{}}    % Section: suppress numbering

% Header (inner, center, outer)
% \ihead{\href{http://sdis.dbca.wa.gov.au}{\textbf{Concept Plan SP 2015-022}}}
%\chead{\href{http://sdis.dbca.wa.gov.au}{Science Directorate Information System}}
% \ohead{\href{https://www.dbca.wa.gov.au/science/10-biodiversity-and-conservation-science}{
% \includegraphics[height=8mm, keepaspectratio]{/usr/src/app/staticfiles/img/logo-dbca-bcs.jpg}}}

% Footer (inner, center, outer)
% \ifoot{\RaggedRight\leftmark}                       % Chapter
% \cfoot{\RaggedLeft\rightmark}                       % Section
% \ofoot[\bfseries\thepage]{\bfseries\thepage}        % Page number (also [plain])


%------------------------------------------------------------------------------%
% Fonts, encoding
%------------------------------------------------------------------------------%
%\usepackage{avant}             % Use the Avantgarde font for headings
\usepackage{txfonts}
\usepackage{mathptmx}
\usepackage{gensymb}            % provides \textdegree
\renewcommand{\familydefault}{\sfdefault} % Default to Sans Serif font
\usepackage{microtype}          % Slightly tweak font spacing for aesthetics
\usepackage[english]{babel}
\usepackage[utf8]{inputenc}  % Allow letters with accents
\usepackage[utf8]{luainputenc}  % Allow letters with accents
\usepackage[T1]{fontenc}        % Use 8-bit encoding that has 256 glyphs
\usepackage{textcomp}
\usepackage[explicit]{titlesec}           % Customise of titles
%\DeclareUnicodeCharacter{0080}{\textregistered}
\DeclareUnicodeCharacter{00B0}{\textdegree}

%------------------------------------------------------------------------------%
% Tables, columns, layout
%------------------------------------------------------------------------------%
\usepackage{etoolbox}
\AtBeginEnvironment{longtabu}{\footnotesize}{}{}  % Table font size
\usepackage{booktabs}           % Required for nicer horizontal rules in tables
\usepackage{multicol}           % 2 col publications
\usepackage{pdflscape}          % Landscape pages
\usepackage{pdfpages}           % Include PDFs
\usepackage{hanging}            % hanging paragraphs for publications
%\usepackage{titletoc}          % Manipulate the table of contents
\setcounter{tocdepth}{2}        % TOC list down to section
\usepackage{enumerate}          % Enumerations
\usepackage{enumitem}           % Enumerations
\usepackage{longtable}          % Multipage table
\usepackage{tabu}               %
\setlength{\tabulinesep}{1.5mm} % Consistent vertical spacing in tabu
\newcommand{\HRule}{\vspace{8mm}\noindent\rule{\linewidth}{0.1pt}}
\usepackage[export]{adjustbox}  % minipage, image frame


%------------------------------------------------------------------------------%
% Graphics, images, colours
%------------------------------------------------------------------------------%
\usepackage{graphicx} % embedded images
\usepackage{wrapfig}  % wrap figures in text
\usepackage{caption}  % allow unnumbered captions
\usepackage{eso-pic} % Required for specifying an image background in the title page
\usepackage{colortbl} % define custom named colours
\usepackage{xstring} % Conditionals
\usepackage{transparent} % Allow transparent images

\definecolor{RedFire}{RGB}{146,25,28}
% Following PICA branding guidelines
% https://dpaw.sharepoint.com/Divisions/pica/Documents/Branding%20guidelines.pdf
\definecolor{dpawblue}{RGB}{35,97,146}          % Pantone 647
\definecolor{dpaworange}{RGB}{237,139,0}        % Pantone 144
\definecolor{dpawgreen}{RGB}{116,170,80}        % Pantone 7489
\definecolor{dpawred}{RGB}{124,46,44}           % Paul's suggestion

% bootstrap colours
\definecolor{successbg}{RGB}{223,240,216}
\definecolor{errorbg}{RGB}{242,222,222}
\definecolor{warningbg}{RGB}{252,248,227}
\definecolor{infobg}{RGB}{217,237,247}
\definecolor{muted}{RGB}{153,153,153}
\definecolor{success}{RGB}{70,136,71}
\definecolor{error}{RGB}{185,74,72}
\definecolor{warning}{RGB}{192,152,83}
\definecolor{info}{RGB}{58,135,173}

% SDIS approval colours
\definecolor{required}{RGB}{192,152,83}
\definecolor{requiredbg}{RGB}{252,248,227}
\definecolor{denied}{RGB}{185,74,72}
\definecolor{deniedbg}{RGB}{242,222,222}
\definecolor{granted}{RGB}{70,136,71}
\definecolor{grantedbg}{RGB}{223,240,216}
\definecolor{notrequired}{RGB}{153,153,153}
\definecolor{notrequiredbg}{RGB}{255,255,255}

\usepackage{tikz} % Drawing
\usetikzlibrary{arrows,shapes,positioning,shadows,trees}


%------------------------------------------------------------------------------%
% Hyperlinks
%------------------------------------------------------------------------------%
\usepackage[open=true]{bookmark}
\usepackage{nameref}
\usepackage{ifxetex,ifluatex}
\ifxetex
  \usepackage[
    setpagesize=false,        % page size defined by xetex
    unicode=false,            % unicode breaks when used with xetex
    xetex]{hyperref}
\else
  \usepackage[unicode=true]{hyperref}
\fi

\hypersetup{
  backref=true,
  pagebackref=true,
  hyperindex=true,
  breaklinks=true,
  urlcolor=dpawblue,
  bookmarks=true,
  bookmarksopen=false,
  pdfauthor={Biodiversity and Conservation Science, Department of Biodiversity, Conservation and Attractions, WA},
  pdftitle=Concept Plan SP 2015-022
,
  colorlinks=true,
  linkcolor=dpawblue,
  pdfborder={0 0 0}}

\urlstyle{same}                         % don't use monospace font for urlstyle


%------------------------------------------------------------------------------%
% Black magic to linebreak URLs
%------------------------------------------------------------------------------%
\usepackage{url}
\makeatletter\g@addto@macro{\UrlBreaks}{\UrlOrds}\makeatother
\Urlmuskip=0mu plus 1mu


%------------------------------------------------------------------------------%
% Fix latex errors
%------------------------------------------------------------------------------%
\providecommand{\tightlist}{\setlength{\itemsep}{0pt}\setlength{\parskip}{0pt}}

% copy-pasted HTML <span> in SDIS fields becomes \text{} in tex source
\providecommand{\text}{}


%------------------------------------------------------------------------------%
% Custom Tikz node for SDS diagram
%------------------------------------------------------------------------------%
\newcommand\mynode[6][]{
  \node[#1] (#2){
    \parbox{#3\relax}{
      \begin{center}
      \textbf{#4}\\
      #5\\
      \footnotesize{#6}
      \end{center}
    }};}


%------------------------------------------------------------------------------%
% Custom no-pagebreaks-environment
%------------------------------------------------------------------------------%
\newenvironment{absolutelynopagebreak}
  {\par\nobreak\vfil\penalty0\vfilneg\vtop\bgroup}
  {\par\xdef\tpd{\the\prevdepth}\egroup\prevdepth=\tpd}


%------------------------------------------------------------------------------%
% Remove the header from odd empty pages at the end of chapters
%------------------------------------------------------------------------------%
\makeatletter
\renewcommand{\cleardoublepage}{
\clearpage\ifodd\c@page\else
\hbox{}
\vspace*{\fill}
\thispagestyle{empty}
\newpage
\fi}


%----------------------------------------------------------------------------------------
%  Page flow control
%----------------------------------------------------------------------------------------
%\widowpenalty=10000
%\clubpenalty=10000
%\vbadness=1200
%\hbadness=11000


%----------------------------------------------------------------------------------------
%   CHAPTER HEADINGS
%----------------------------------------------------------------------------------------
\newcommand{\thechapterimage}{}
\newcommand{\chapterimage}[1]{\renewcommand{\thechapterimage}{#1}}

% Numbered chapters with mini tableofcontents
\def\thechapter{\arabic{chapter}}
\def\@makechapterhead#1{
%\thispagestyle{plain}
{\centering \normalfont\sffamily
\ifnum \c@secnumdepth >\m@ne
\if@mainmatter
\startcontents
\begin{tikzpicture}[remember picture,overlay]
\node at (current page.north west)
{\begin{tikzpicture}[remember picture,overlay]
\node[anchor=north west,inner sep=0pt] at (0,0) {
\includegraphics[width=\paperwidth,height=0.5\paperwidth]{\thechapterimage}};
%------------------------------------------------------------------------------%
% Small contents box in the chapter heading
% Mini TOC background box
%\fill[color=dpawblue!10!white,opacity=.2] (1cm,0) rectangle (
%  3.5cm, % Mini TOC box width
%  -3.5cm % Mini TOC box height
%);
% Mini TOC text content
%\node[anchor=north west] at (1.1cm,.35cm) {
%  \parbox[t][8cm][t]{6.5cm}{
%    \huge\bfseries\flushleft
%    \printcontents{l}{1}{
%    \setcounter{tocdepth}{1}                   % Mini TOC level depth
%    }
% }
%};
%------------------------------------------------------------------------------%
% Chapter heading
\draw[anchor=west] (5cm,-9cm) node [
rounded corners=20pt,
fill=dpawblue!10!white,
text opacity=1,
draw=dpawblue,
draw opacity=1,
line width=1.5pt,
fill opacity=.2,
inner sep=12pt]{
    \huge\sffamily\bfseries\textcolor{black}{
      \thechapter. #1\strut\makebox[22cm]{}
    }
};
\end{tikzpicture}};
\end{tikzpicture}}
\par\vspace*{240\p@}                            % Push text below chapter image
\fi
\fi}

%------------------------------------------------------------------------------%
% Unnumbered chapters without mini tableofcontents
%------------------------------------------------------------------------------%
\def\@makeschapterhead#1{
%\thispagestyle{plain}
{\centering \normalfont\sffamily
\ifnum \c@secnumdepth >\m@ne
\if@mainmatter
\begin{tikzpicture}[remember picture,overlay]
\node at (current page.north west)
{\begin{tikzpicture}[remember picture,overlay]
\node[anchor=north west,inner sep=0pt] at (0,0) {
  \includegraphics[width=\paperwidth,height=0.5\paperwidth]{\thechapterimage}};
% Mini TOC background box
%\fill[color=dpawblue!10!white,opacity=.2] (1cm,0) rectangle (
%  3.5cm,                                       % Mini TOC box width
%  -3.5cm                                       % Mini TOC box height
%);
% Mini TOC text content
%\node[anchor=north west] at (1.1cm,.35cm) {
%  \parbox[t][8cm][t]{6.5cm}{
%    \huge\bfseries\flushleft
%    \printcontents{l}{1}{
%    \setcounter{tocdepth}{1} % Mini TOC level depth
%    }
%}
%};
\draw[anchor=west] (5cm,-9cm) node [rounded corners=20pt,
  fill=dpawblue!10!white,fill opacity=.6,inner sep=12pt,text opacity=1,
  draw=dpawblue,draw opacity=1,line width=1.5pt]{
  \huge\sffamily\bfseries\textcolor{black}{#1\strut\makebox[22cm]{}}};
\end{tikzpicture}};
\end{tikzpicture}}
\par\vspace*{240\p@}
\fi
\fi
}
\makeatother



\usepackage[automark,headsepline,footsepline,plainfootsepline]{scrlayer-scrpage}
\automark*[section]{}
\addtokomafont{pageheadfoot}{\normalfont\footnotesize\sffamily} % Don't italicise
\renewcommand{\chaptermark}[1]{\markleft{#1}{}}     % Chapter: suppress numbering
\renewcommand{\sectionmark}[1]{\markright{#1}{}}    % Section: suppress numbering

% Header (inner, center, outer)
\ihead{\href{http://sdis.dbca.wa.gov.au/documents/conceptplan/1518/}{Concept Plan SP 2015-022}}
%\chead{\href{http://sdis.dbca.wa.gov.au}{Science Directorate Information System}}
\ohead{\href{https://www.dbca.wa.gov.au/science/10-biodiversity-and-conservation-science}{
\includegraphics[height=6mm, keepaspectratio]{/usr/src/app/staticfiles/img/logo-dbca-bcs.jpg}}}
% Footer (inner, center, outer)
\ifoot{\textbf{Printed}~Tue, 6 Oct 2020 12:44:09 +0800} % inner/left footer
\cfoot{}
\ofoot[\bfseries\thepage]{\bfseries\thepage}        % Page number (also [plain])


\pagestyle{scrheadings}
\setkomafont{pageheadfoot}{\normalfont}

%-----------------------------------------------------------------------------%
\begin{document}
\raggedbottom

%-----------------------------------------------------------------------------%
% Title page
\subject{Concept Plan SP 2015-022
}
\title{Recovery of Gilbert's potoroo \emph{Potorous gilbertii}
}
\subtitle{Animal Science
}
\author{}
\publishers{\small
    \subsection*{Project Core Team}
\begin{tabu} {X X}
\textbf{Supervising Scientist} & Tony Friend
\\
\textbf{Data Custodian} & Tony Friend
\\
\textbf{Site Custodian} & 
\\
\end{tabu}


    \subsection*{Project status as of Oct. 6, 2020, 12:44 p.m.}
\begin{tabu} {X X}
& Pending project plan approval
\\
\end{tabu}

    
\subsection*{Document endorsements and approvals as of Oct. 6, 2020, 12:44 p.m.}
\begin{tabu} {X X}

%\rowcolor{grantedbg}
    \textbf{Project Team} & 
    \textcolor{granted}{ granted}\\

%\rowcolor{grantedbg}
    \textbf{Program Leader} & 
    \textcolor{granted}{ granted}\\

%\rowcolor{grantedbg}
    \textbf{Directorate} & 
    \textcolor{granted}{ granted}\\

\end{tabu}



}
\uppertitleback{}
\lowertitleback{}
\date{}

%-----------------------------------------------------------------------------%
% Front matter
\frontmatter
\maketitle
%-----------------------------------------------------------------------------%
% Main matter
\mainmatter


\section*{Recovery of Gilbert's potoroo \emph{Potorous gilbertii}
}



\subsection*{Biodiversity and Conservation Science Program}

Animal Science




\subsection*{Departmental Service}

Service 6: Conserving Habitats, Species and Communities




\subsection*{Aims}

To carry out research projects to support the objectives listed in the
current Gilbert's potoroo recovery plan:

\begin{itemize}
\tightlist
\item
  To ensure existing populations of Gilbert's potoroo are returned to
  and maintained at sustainable levels and that genetic diversity is
  maximised.
\item
  To increase the number of populations and individuals of Gilbert's
  potoroo.
\end{itemize}

To achieve these aims, recommendations for management are required and
in some cases, management actions will be undertaken or led by research
personnel.~ The main areas of research will be:

\begin{itemize}
\tightlist
\item
  Evaluation of the importance of predation by carpet pythons in the
  Waychinicup NP enclosure in regulating potoroo numbers. This aspect is
  largely complete, and a paper is in preparation.
\item
  If python predation~is significant, evaluate methods for reducing
  python predation at this site, for implementation.~This research
  aspect is covered by a separate SPP and is being led by Dr David
  Pearson with input from Albany-based SCD and FRSD staff.
\item
  Evaluate potential translocation sites, including Middle Island in the
  Recherche Archipelago and a mainland site, including resource
  availability and predation risk.
\item
  Implement a translocation to at least one of these sites.
\item
  Undertake a survey of the genetic variability of Gilbert's potoroo
  across all populations and develop a population management strategy.
\item
  Publication of 20 + years of research and monitoring data to ensure
  appropriate management into the future.
\end{itemize}

~

~




\subsection*{Expected outcome}

The expected outcome of this project will be a series of
recommendations, based on published information for management of
Gilbert's potoroo to achieve the greater security of the species.
Ideally, management of the populations will achieve population stability
and the highest genetic variability possible given the low number of
animals.~ This will require a greater understanding of the current
status of populations and the ecological requirements of the species at
its various sites.

~Management Implications

The recommendations arising from this project will include management
actions necessary to increase the likelihood that this species will
persist. These are expected to include continued and enhanced control of
foxes and feral cats at all existing and new mainland population sites
as well as fire management at all sites to protect long-unburnt habitat
areas. Action to reduce predation by carpet pythons may also be
recommended. Management of potoroo population genetics will involve
movement of animals between the isolated population sites. Some or all
of these actions will involve District and Regional staff.




\subsection*{Strategic context}

This project will further the aims of and operate within the context of
the following plans and documents:

\begin{itemize}
\tightlist
\item
  Department of Parks and Wildlife Gilbert's Potoroo Recovery Plan
  (revised plan currently under review)
\item
  Department of Parks and Wildlife Strategic Direction 2014-2017 (DPaW
  2014a)
\item
  Department of Parks and Wildlife Corporate Policy Statement No. 35
  Conserving threatened species and ecological communities (DPaW 2015a)
\item
  Department of Parks and Wildlife Corporate Policy Statement No. 12
  Management of pest animals (DPaW 2015c)
\item
  Department of Parks and Wildlife Corporate Policy Statement No. 19
  Fire management (DPaW 2015d)
\item
  Department of Parks and Wildlife Corporate Policy Statement No. 88
  Prescribed burning (DPaW 2016a)
\item
  Department of Parks and Wildlife Corporate Guidelines No. 35 Listing
  and recovery of threatened species and ecological communities (DPaW
  2015b)
\item
  Department of Parks and Wildlife Corporate Guidelines No. 36
  Recovering threatened species through translocations and captive
  breeding (DPaW 2015f)
\item
  Two Peoples Bay Nature Reserve Management Plan (CALM 1995a)
\item
  Esperance and Recherche Parks and Reserves Management Plan 84 (DPaW
  2016b)
\item
  Albany Coast Draft Management Plan (DPaW 2016c)
\item
  South Coast Regional Fire Management Plan 2009-2014 (DEC 2009)
\item
  South Coast Threatened Birds Recovery Plan (DPaW 2014f)
\item
  Quokka Recovery Plan (DEC 2013b)
\item
  Western Ringtail Possum Recovery Plan (DPaW 2014d)
\item
  Threatened Species and Ecological Communities Strategic Management
  Plan, South Coast Region, WA (Gilfillan \emph{et al.} 2009)
\item
  Western Shield Fauna Recovery Program Interim Strategic Plan 2009/10
  -- 2012/13 (DEC 2008)
\item
  Threat abatement plan for predation by European red fox (DEWHA 2008)
\item
  Threat abatement plan for predation by feral cats (Commonwealth of
  Australia 2015)
\item
  Survey guidelines for Australia's threatened mammals (DSEWPaC 2011)
\item
  Threatened Species Strategy (Commonwealth of Australia 2016
\end{itemize}




\subsection*{Expected collaborations}

Existing collaborations with Parks and Wildlife Albany District and
South Coast Region staff are vital for the successful management and
recovery of Gilbert's potoroo. This involves working closely with
management staff, particularly on predator control, fire management and
management of infrastructure such as the captive holding facility at Two
Peoples Bay and the Waychinicup NP enclosure.

Collaborations with university academics and students from ECU, Murdoch
University and UWA have already produced valuable information on the
biology and ecology of Gilbert's potoroo and it is expected that such
collaborations will continue and increase. In particular, students at
the UWA Albany campus are ideally situated to be involved in such
projects.

Funding collaboration, most importantly with South Coast NRM and local
community group the Gilbert's Potoroo Action Group (GPAG) will continue
and will involve close cooperation with these groups to foster and grow
support for potoroo recovery. GPAG has been awarded a grant of \$250,000
over two years to support the investigation of potential~translocation
sites and an initial translocation.


\subsection*{Proposed period of the project}
Oct. 8, 2015 -- June 30, 2019



\subsection*{Staff time allocation }



\begin{longtabu} to \linewidth { |  X | X | X | X | }
\hline
\rowcolor{infobg}
Role & Year 1 & Year 2 & Year 3\\
\hline
\endhead



Scientist & 0.4 & 0.4 & 0.4\\



Technical & 1.5 & 1.5 & 1.5\\



Volunteer & 0.2 & 0.2 & 0.2\\



Collaborator & 0.1 & 0.1 & 0.1\\


\hline
\end{longtabu}



\subsection*{Indicative operating budget }



\begin{longtabu} to \linewidth { |  X | X | X | X | }
\hline
\rowcolor{infobg}
Source & Year 1 & Year 2 & Year 3\\
\hline
\endhead



Consolidated Funds (DPaW) & 20,000 & 20,000 & 20,000\\



External Funding & 100,500 & 100,500 & 50,000\\


\hline
\end{longtabu}






%-----------------------------------------------------------------------------%
% Back matter
%\backmatter
\end{document}
%-----------------------------------------------------------------------------%

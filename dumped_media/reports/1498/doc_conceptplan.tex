
\documentclass[version=last,
    paper=a4, % paper size
    10pt, % default font size
    usenames,
    dvipsnames,
    twoside, % PRINT Binding Correction
    % oneside, % ONLINE
    headings=openany, % open chapters on odd and even pages
    BCOR=7mm, % PRINT Binding Correction
    %DIV=13, % typearea 161.54 mm x 228.46 mm, top margin 22.85 mm, inner margin 16.15 mm
    %DIV=14, % 165.00 233.36 21.21 15.00
    DIV=15, % 168.00 237.60 19.80 14.00
    toc=chapterentrywithdots % Table of Contents style
]{scrbook}
\usepackage{typearea}
\usepackage[automark,headsepline%,footsepline
]{scrlayer-scrpage} % Headers and footers

%%
%% Fonts, encoding, spacing, indentation
%%
\usepackage{txfonts}
\renewcommand{\familydefault}{\sfdefault} % Default to Sans Serif font
\usepackage[english]{babel}
\usepackage[T1]{fontenc}
\usepackage[utf8]{inputenc}

% Chapter and section headings
%\renewcommand{\raggedchapter}{\centering}
%\addtokomafont{chapter}{\Huge\scshape} % Chapter headings in smallcaps
%\addtokomafont{chapterentry}{\scshape} % Chapter headings in smallcaps
%\addtokomafont{section}{\centering\Large\scshape} % Chapter headings in smallcaps
%\addtokomafont{sectionentry}{\scshape} % Chapter headings in smallcaps

% Paragraph spacing
%\usepackage{parskip}    % Paragraph spacing
%\setlength{\parindent}{0em} % Don't indent paragraphs - ONLINE
%\setlength{\parskip}{1.3 ex plus 0.5ex minus 0.3ex} % 1-1.8 ex vertical space between paragraphs - ONLINE

% Spacing of headings
\RedeclareSectionCommand[afterskip=3pt]{section} % less space after section
\RedeclareSectionCommand[beforeskip=0cm]{subsection} % less space between HRule and project name
\RedeclareSectionCommand[afterskip=0.1\baselineskip]{subsubsection} % less space after progressreport subheadings

% Table font size
\usepackage{etoolbox}
\AtBeginEnvironment{longtabu}{\footnotesize}{}{}

%%
%% Tables, columns, layout
%%
\usepackage{multicol}   % 2 col publications
\usepackage{pdflscape}  % Landscape pages
\usepackage{pdfpages}   % Include PDFs
\usepackage{hanging}    % hanging paragraphs for publications
%\usepackage{titletoc}   % Required for manipulating the table of contents
\setcounter{tocdepth}{2} % TOC list down to section
\usepackage{enumerate}  % Enumerations
\usepackage{enumitem}   % Enumerations
\usepackage{longtable}  % Multipage table
\usepackage{tabu}       %
\setlength{\tabulinesep}{1.5mm} % Consistent vertical spacing in tabu

%%
%% Graphics, images, colours
%%
\usepackage{graphicx} % embedded images
\usepackage{eso-pic} %
\usepackage{colortbl} % define custom named colours

\definecolor{RedFire}{RGB}{146,25,28}
\definecolor{ParksWildlife}{RGB}{0,85,144}
\definecolor{successbg}{RGB}{223,240,216}
\definecolor{errorbg}{RGB}{242,222,222}
\definecolor{warningbg}{RGB}{252,248,227}
\definecolor{infobg}{RGB}{217,237,247}
\definecolor{muted}{RGB}{153,153,153}
\definecolor{success}{RGB}{70,136,71}
\definecolor{error}{RGB}{185,74,72}
\definecolor{warning}{RGB}{192,152,83}
\definecolor{info}{RGB}{58,135,173}

\definecolor{required}{HTML}{D9534F}
\definecolor{denied}{HTML}{D9534F}
\definecolor{granted}{HTML}{47A447}
\definecolor{not required}{RGB}{200, 200, 200}

\usepackage{tikz} % Drawing
\usetikzlibrary{arrows,shapes,positioning,shadows,trees}

%%
%% Links, URLs
%%
\usepackage[
    colorlinks=true,
    linktoc=all,
    linkcolor=RedFire,
    urlcolor=ParksWildlife,
    pdftitle=doc\_conceptplan.pdf
]{hyperref}

% Black magic to linebreak URLs
\usepackage{url}
\makeatletter
\g@addto@macro{\UrlBreaks}{\UrlOrds}
\makeatother

%%
%% Custom macros
%%
% Thick Horizontal rule
\newcommand{\HRule}{\vspace{8mm}\\\noindent\rule{\linewidth}{0.1pt}}

% Custom Tikz node for SDS diagram
\newcommand\mynode[6][]{\node[#1] (#2){\parbox{#3\relax}{\begin{center}\textbf{#4}\\#5\\\footnotesize{#6}\end{center}}};}




%-----------------------------------------------------------------------------%
% Headers and footers
%
\fancypagestyle{plain}{
  \fancyhf{}
  \setlength\headheight{60pt} % push page content below header
  \renewcommand{\headrulewidth}{0.1pt}
  \renewcommand{\footrulewidth}{0.1pt}
  
  
  \fancyhead[L]{ 
    \href{http://sdis.dpaw.wa.gov.au}{
    \includegraphics[scale=0.6]{/mnt/projects/sdis/staticfiles/img/logo-dpaw.png}}
  }
  \fancyhead[R]{ 
      \hfill
      \href{http://sdis.dpaw.wa.gov.au}{Science Directorate Information System} 
      \newline 
      \href{http://sdis.dpaw.wa.gov.au/documents/conceptplan/1498/}{Concept Plan 2015-015} 
  }
  
  
  
  
  \fancyfoot[L]{ \leftmark\newline\textbf{Printed}\textit{ Jan. 19, 2016, 9:59 a.m. }}
  \fancyfoot[R]{  \, \newline Page \thepage\ of \pageref{LastPage} }
  
  
}
\pagestyle{plain}
%
% end Headers
%-----------------------------------------------------------------------------%

\begin{document}

%-----------------------------------------------------------------------------%
% Title page
%

%
% end title page
%-----------------------------------------------------------------------------%



\section*{Concept Plan 2015-015}

\subsection*{Project title}
Establishing long-term monitoring in the proposed Dampier Archipelago
marine reserves



\subsection*{Science and Conservation Division Program}
Marine Science



\subsection*{Parks and Wildlife Service}
Service 2: Conserving Habitats, Species and Ecological Communities



\subsection*{Background and Aims}
The Dampier Archipelago is situated adjacent to Karratha, 1,650 km north
of Perth in the Pilbara Nearshore marine bioregion. The marine and
coastal environment of the Dampier Archipelago/Cape Preston Region has a
complex structure of islands and supports diverse intertidal and
subtidal habitats that include rocky reefs, corals, mangroves,
macro-algae and soft sediments. The area is subject to increasing human
pressures, including the development and expansion of several heavy
industries that have required port infrastructure and dredging. The
archipelago waters also support pearling and aquaculture and commercial
and recreational fishing. High levels of private boat ownership in
adjacent communities suggest that visitor use of this area is high. Part
of the archipelago waters have been identified as a proposed marine
conservation reserve and an Indicative Management Plan was released in
2005 outlining how the outstanding marine ecological assets and social
values of the area can be managed and protected.

\emph{~}This project will implement long-term monitoring within the
proposed Dampier Archipelago marine reserves in a manner consistent with
the 2005 Indicative Management Plan.~



\subsection*{Expected outcome}
This project will ensure that a prioritised, long-term monitoring
program focused on key ecological assets and selected social values will
be in place when the proposed Dampier Archipelago marine reserves are
gazetted. The monitoring program will record the condition of ecological
assets and social values and the pressures acting on them.

This will:

\begin{itemize}
\itemsep1pt\parskip0pt\parsep0pt
\item
  Provide an understanding of the baseline condition of key ecological
  assets and selected social values of the proposed Dampier Archipelago
  marine reserves and the pressures acting on them;
\item
  Identify and trial new monitoring indicators, methods and operating
  procedures for ecological assets for which these are not currently in
  place;
\item
  Develop collaborations with other Agencies, industry, academic
  institutions and traditional owners;
\item
  Create awareness in the broader community of the ecological assets and
  social values of the proposed Dampier Archipelago marine reserves, the
  issues that affect these assets, and the role of conservation
  management in protecting them.
\end{itemize}



\subsection*{Strategic context}
This project aligns with the following Departmental strategic
priorities:

\emph{Department of Parks and Wildlife Strategic Directions 2014-2017.}

\begin{itemize}
\itemsep1pt\parskip0pt\parsep0pt
\item
  Focus conservation science on management priorities.
\item
  Ensure conservation management is based on best practice science.
\end{itemize}

\emph{Science and Conservation Division Strategic Plan 2014-2017.}

\begin{itemize}
\itemsep1pt\parskip0pt\parsep0pt
\item
  Deliver and/or support the marine monitoring program in the State's
  marine parks and reserves.
\item
  Provide information and advice for effective maintenance, protection
  and management of park biodiversity values.
\item
  Communicate monitoring knowledge to marine park managers.
\item
  Partner with Traditional Owners in development of conservation actions
  and knowledge of Country.
\item
  Engage with the community and communicate the value of wildlife and
  its conservation requirements, and the positive contribution that
  wildlife makes to people's lives.
\end{itemize}

\emph{Indicative Management Plan for the proposed Dampier Archipelago
Marine Park and Cape Preston Marine Management Area 2005.}

\begin{itemize}
\itemsep1pt\parskip0pt\parsep0pt
\item
  Develop and progressively implement a coordinated and prioritised
  monitoring program of key values and processes in the proposed
  reserves.
\end{itemize}

~



\subsection*{Expected collaborations}
Collaborations will be developed with other State and Commonwealth
Government agencies (such as AIMS, CSIRO and the Western Australian
Museum), industry partners (such as Woodside and the Pilbara Ports
Authority), WA universities, local stakeholder groups, regional Parks
and Wildlife staff and Traditional Owners.


\subsection*{Proposed period of the project}
July 15, 2014 -- June 30, 2019

\subsection*{Staff time allocation }



\begin{longtabu} to \linewidth { |  X | X | X | X | }
\hline
\rowcolor{infobg}
Role & Year 1 & Year 2 & Year 5\\
\hline
\endhead



Scientist & 1.25 & 1.25 & 1.25\\



Technical & 0.25 & 0.25 & 0.25\\



Volunteer &  &  & \\



Collaborator &  &  & \\


\hline
\end{longtabu}



\subsection*{Indicative operating budget }



\begin{longtabu} to \linewidth { |  X | X | X | X | }
\hline
\rowcolor{infobg}
Source & Year 1 & Year 2 & Year 5\\
\hline
\endhead



Consolidated Funds (DPaW) &  &  & \\



External Funding & $200,000 & $200,000 & $200,000\\


\hline
\end{longtabu}





\subsection*{Endorsements}
Endorsements and approvals as of Jan. 19, 2016, 9:59 a.m.:\\
\begin{tabu} {| X | X |}
\hline

\rowcolor{granted}
Project Team & granted\\
\hline

\rowcolor{granted}
Program Leader & granted\\
\hline

\rowcolor{granted}
Directorate & granted\\
\hline

\end{tabu}





\end{document}

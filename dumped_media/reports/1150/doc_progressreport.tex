\documentclass[version=last, paper=a4, DIV=18, usenames, dvipsnames]{scrartcl}
\usepackage{txfonts}
\usepackage{pdflscape}
\usepackage{pdfpages}
\usepackage[english]{babel} % English language/hyphenation
%%% Bootstrap colors
\definecolor{RedFire}{RGB}{146,25,28}
\definecolor{ParksWildlife}{RGB}{0,85,144}
\definecolor{successbg}{RGB}{223,240,216}
\definecolor{errorbg}{RGB}{242,222,222}
\definecolor{warningbg}{RGB}{252,248,227}
\definecolor{infobg}{RGB}{217,237,247}
\definecolor{muted}{RGB}{153,153,153}
\definecolor{success}{RGB}{70,136,71}
\definecolor{error}{RGB}{185,74,72}
\definecolor{warning}{RGB}{192,152,83}
\definecolor{info}{RGB}{58,135,173}
\usepackage[colorlinks=true,pdftitle=doc\_progressreport.pdf,linktoc=all,linkcolor=RedFire,urlcolor=ParksWildlife]{hyperref}
\usepackage{colortbl}
\usepackage{longtable}
\usepackage{tabu}
\setlength{\tabulinesep}{1.5mm}
\usepackage{enumerate}
\usepackage{enumitem}
\usepackage{fancyhdr}
\usepackage{lastpage}
\usepackage{graphicx}
\usepackage{eso-pic}
\usepackage{hyphenat}



%%% Custom headers/footers (fancyhdr package)
\fancypagestyle{plain}{
\fancyhf{}
\setlength\headheight{40pt}
\renewcommand{\headrulewidth}{0.1pt}
\renewcommand{\footrulewidth}{0.1pt}



    \fancyhead[L]{ \href{http://sdis.dpaw.wa.gov.au/documents/progressreport/1150/download/}{} \newline }
\fancyhead[R]{ \hfill\href{http://www.dpaw.wa.gov.au}{Department of Parks and Wildlife}\newline\href{http://sdis.dpaw.wa.gov.au}{Pythia}}




\fancyfoot[L]{ \leftmark\newline\textbf{Last Modified}\textit{ }\quad\textbf{Printed}\textit{ July 6, 2014, 1:07 p.m. } }
\fancyfoot[R]{  \, \newline Page \thepage\ of \pageref{LastPage} } % Pagenumbering


}
\pagestyle{plain}


\newcommand{\HRule}{\rule{\linewidth}{0.1pt}}

\newcommand{\placetextbox}[3]{% \placetextbox{<horizontal pos>}{<vertical pos>}{<stuff>}
  \setbox0=\hbox{#3}% Put <stuff> in a box
  \AddToShipoutPictureFG*{% Add <stuff> to current page foreground
    \put(\LenToUnit{#1\paperwidth},\LenToUnit{#2\paperheight}){\vtop{{\null}\makebox[0pt][c]{#3}}}%
  }%
}%

\begin{document}

\setcounter{secnumdepth}{-1}


\begin{titlepage}
\begin{center}
% Upper part of the page
\begin{minipage}[t]{0.28\textwidth}
\begin{flushleft}
\href{http://www.dpaw.wa.gov.au}{\includegraphics[scale=0.6]{/var/www/sdis_8271/staticfiles/img/logo-dpaw.png}}
\end{flushleft}
\end{minipage}
\begin{minipage}[b]{0.7\textwidth}
\begin{flushright}
    \href{http://sdis.dpaw.wa.gov.au/documents/progressreport/1150/download/}{}) \\
\end{flushright}
\end{minipage}
\HRule \\[0.4cm]
\vfill
\textsc{\Huge Science project 2012-31 Biosystematics of fungi for conservation and restoration of Western Australia's biota \newline }
\vfill
\textsc{\Huge Progress Report}

\vfill\vfill\vfill\vfill
title and summary

\vfill\vfill\vfill\vfill\vfill\vfill\vfill\vfill

\textbf{Version created on} July 6, 2014, 1:07 p.m.
\vfill
\textbf{Last Modified on}  by 
\vfill\vfill
\textbf{Report Status}\\\,
\begin{tabu} to \linewidth { | X[l] | X | }
\hline
\rowcolor{infobg}
Status & Last Updated \\
\hline
\textbf{Planning - } \\
\hline
\end{tabu}
\vfill
\textbf{Science Project Overview}\\\,
\begin{tabu} to \linewidth { | X[l] | X | }
\hline
\rowcolor{infobg}
Part & Checklist Last Updated \\
\hline
\textbf{Part A - Summary \& Approval} & bla \\
\hline
\end{tabu}

\end{center}
\end{titlepage}

\setcounter{tocdepth}{2}
\tableofcontents
\clearpage






\section{Context Summary}



This project represents a new and timely effort to build the state's biodiversity knowledge base, and create and apply more comprehensive and accurate fungal scientific knowledge for conservation and management of the state's biodiversity.






\section{Aims Summary}



\begin{itemize}

  \item Generate and provide scientifically accurate and comprehensive taxonomic data for fungal taxa in Western Australia that are previously unrecorded, unidentified, misidentified, or ill-defined, particularly taxa of relevance to specific, current DEC conservation initiatives.

  \item Make available descriptive information about fungi taxa in published form and in online information systems.

  \item Improve access and uptake of scientific knowledge about fungi and thereby promote better awareness and understanding by scientists and community of the significance of fungal diversity and function in bushlands.

  \item Achieve greater taxonomic and geographic representation of representation of Western Australian fungi in datasets and as permanent vouchers at the Western Australian Herbarium.

\end{itemize}






\section{Progress}



\begin{itemize}

  \item Defined and illustrated morphological and molecular characteristics of a further 88 species of Australian Inocybaceae species, in readiness for upcoming monograph book.

  \item As a core part of this work, built global DNA nLSU-rRNA dataset including 474 sequences from Australasia, and rpb2 gene dataset including 367 Australasian sequences. Released 724 DNA sequences to GenBank.

  \item Released over100 submissions for resolved new species of Inocybaceae to MycoBank.

  \item Published journal paper highlighting parallel agaricoid-secotioid species pairings of \emph{Inocybe} from WA wheatbelt and North America.

\end{itemize}






\section{Management implications}



The availability of scientifically accurate and comprehensive information about taxa of fungi in Western Australia will encourage and allow DEC and the community to incorporate knowledge of fungi into management practices. This includes regional biological surveys, managing the interdependent linkages between fungi and plants and animals, and a providing a better basis for assessment of the conservation status of fungi taxa.






\section{Future directions}



\begin{itemize}

  \item Continued taxonomic research defining and documenting species of the ectomycorrhizal fungi family Inocybaceae in Australia.

  \item Morphological and molecular phylogenetic data of individual collections of Inocybaceae will be generated and assessed then used to define species concepts.

  \item Construction and compilation of text and illustrative material to produce a complete manuscript that will be submitted to the external client (ABRS) for the target product, a monographic book on the Australian Inocybaceae for the \emph{Fungi of Australia} series.

\end{itemize}






\clearpage



\end{document}

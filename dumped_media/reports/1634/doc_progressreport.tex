
\documentclass[version=last,
    paper=a4, % paper size
    10pt, % default font size
    usenames,
    dvipsnames,
    oneside, % ONLINE
    headings=openany, % open chapters on odd and even pages
    %toc=chapterentrywithdots, % Table of Contents style
    %BCOR=7mm, % PRINT Binding Correction
    %DIV=13, % typearea 161.54 mm x 228.46 mm, top margin 22.85 mm, inner margin 16.15 mm
    %DIV=14, % 165.00 233.36 21.21 15.00
    DIV=15 % 168.00 237.60 19.80 14.00
]{scrbook}
\usepackage{typearea}
\usepackage[automark,headsepline,footsepline]{scrlayer-scrpage} % Headers and footers

%%
%% Fonts, encoding, spacing, indentation
%%
\usepackage{txfonts}
\renewcommand{\familydefault}{\sfdefault} % Default to Sans Serif font
\usepackage[english]{babel}
\usepackage[T1]{fontenc}
\usepackage[utf8]{inputenc}

% Paragraph spacing
%\usepackage{parskip}    % Paragraph spacing
%\setlength{\parindent}{0em} % Don't indent paragraphs - ONLINE
%\setlength{\parskip}{1.3 ex plus 0.5ex minus 0.3ex} % 1-1.8 ex vertical space between paragraphs - ONLINE

% Spacing of headings
%\RedeclareSectionCommand[afterskip=3pt]{section} % less space after section
%\RedeclareSectionCommand[beforeskip=0cm]{subsection} % less space between HRule and project name
%\RedeclareSectionCommand[afterskip=0.1\baselineskip]{subsubsection} % less space after progressreport subheadings

% Table font size
\usepackage{etoolbox}
\AtBeginEnvironment{longtabu}{\footnotesize}{}{}

%%
%% Tables, columns, layout
%%
\usepackage{multicol}   % 2 col publications
\usepackage{pdflscape}  % Landscape pages
\usepackage{pdfpages}   % Include PDFs
\usepackage{hanging}    % hanging paragraphs for publications
%\usepackage{titletoc}   % Required for manipulating the table of contents
\setcounter{tocdepth}{2} % TOC list down to section
\usepackage{enumerate}  % Enumerations
\usepackage{enumitem}   % Enumerations
\usepackage{longtable}  % Multipage table
\usepackage{tabu}       %
\setlength{\tabulinesep}{1.5mm} % Consistent vertical spacing in tabu

%%
%% Graphics, images, colours
%%
\usepackage{graphicx} % embedded images
\usepackage{eso-pic} %
\usepackage{colortbl} % define custom named colours
\definecolor{RedFire}{RGB}{146,25,28}
\definecolor{ParksWildlife}{RGB}{0,85,144}
\definecolor{successbg}{RGB}{223,240,216}
\definecolor{errorbg}{RGB}{242,222,222}
\definecolor{warningbg}{RGB}{252,248,227}
\definecolor{infobg}{RGB}{217,237,247}
\definecolor{muted}{RGB}{153,153,153}
\definecolor{success}{RGB}{70,136,71}
\definecolor{error}{RGB}{185,74,72}
\definecolor{warning}{RGB}{192,152,83}
\definecolor{info}{RGB}{58,135,173}

\definecolor{required}{RGB}{192,152,83}
\definecolor{requiredbg}{RGB}{252,248,227}
\definecolor{denied}{RGB}{185,74,72}
\definecolor{deniedbg}{RGB}{242,222,222}
\definecolor{granted}{RGB}{70,136,71}
\definecolor{grantedbg}{RGB}{223,240,216}
\definecolor{not reqiured}{RGB}{153,153,153}
\definecolor{not requiredbg}{RGB}{255,255,255}

\usepackage{tikz} % Drawing
\usetikzlibrary{arrows,shapes,positioning,shadows,trees}

%%
%% Links, URLs
%%
\usepackage[
    linktoc=all,
    %colorlinks=false,  %PRINT
    colorlinks=true, % ONLINE
    linkcolor=RedFire, % ONLINE
    urlcolor=ParksWildlife, % ONLINE
    pdftitle=Progress Report SP 2011-018 (FY 2015-2016)
]{hyperref}

% Black magic to linebreak URLs
\usepackage{url}
\makeatletter
\g@addto@macro{\UrlBreaks}{\UrlOrds}
\makeatother

%%
%% Custom macros
%%
% Thick Horizontal rule
\newcommand{\HRule}{\vspace{8mm}\\\noindent\rule{\linewidth}{0.1pt}}

% Custom Tikz node for SDS diagram
\newcommand\mynode[6][]{
    \node[#1] (#2){
        \parbox{#3\relax}{
            \begin{center}
            \textbf{#4}\\
            #5\\
            \footnotesize{#6}
            \end{center}}};}



%-----------------------------------------------------------------------------%
% Headers and Footers
\automark{section}
\ohead{\href{http://sdis.dpaw.wa.gov.au/documents/progressreport/1634/}{Progress Report SP 2011-018
}}
\chead{\href{http://sdis.dpaw.wa.gov.au}{SDIS}} % center header ONLINE
\ihead{\href{http://sdis.dpaw.wa.gov.au}{
        \includegraphics[scale=0.4]{/mnt/projects/sdis/staticfiles/img/logo-dpaw.png}}}
\ifoot{\textbf{Printed}~Tue, 5 Jul 2016 16:02:03 +0800} % inner/left footer
\cfoot{} % center footer
\ofoot{\pagemark} % outer/right footer
\pagestyle{scrheadings}
\setkomafont{pageheadfoot}{\normalfont}

%-----------------------------------------------------------------------------%
\begin{document}
\raggedbottom

%-----------------------------------------------------------------------------%
% Title page
\subject{Progress Report SP 2011-018
}
\title{Western Australian wetland fauna surveys
}
\subtitle{Wetlands Conservation
}
\author{}
\publishers{\small
    \subsection*{Project Core Team}
\begin{tabu} {X X}
\textbf{Supervising Scientist} & Adrian Pinder
\\
\textbf{Data Custodian} & Adrian Pinder
\\
\textbf{Site Custodian} & Adrian Pinder
\\
\end{tabu}


    \subsection*{Project status as of July 5, 2016, 4:02 p.m.}
\begin{tabu} {X X}
& Approved and active
\\
\end{tabu}

    
\subsection*{Document endorsements and approvals as of July 5, 2016, 4:02 p.m.}
\begin{tabu} {X X}

%\rowcolor{grantedbg}
    \textbf{Project Team} & 
    \textcolor{granted}{ granted}\\

%\rowcolor{grantedbg}
    \textbf{Program Leader} & 
    \textcolor{granted}{ granted}\\

%\rowcolor{grantedbg}
    \textbf{Directorate} & 
    \textcolor{granted}{ granted}\\

\end{tabu}



}
\uppertitleback{}
\lowertitleback{}
\date{}

%-----------------------------------------------------------------------------%
% Front matter
\frontmatter
\maketitle
%-----------------------------------------------------------------------------%
% Main matter
\mainmatter

\section*{Western Australian wetland fauna surveys
}

A Pinder, K Quinlan, R Coppen, L Lewis, Dr RJ Shiel (University of
Adelaide)


\section*{Context}
Regional biological surveys provide analyses of biodiversity patterning
for conservation planning at broader scales, but sites in these projects
are usually too sparse for use at a more local scale, such as individual
reserves, catchments or wetland complexes. This umbrella project is
designed to fill gaps within and between the regional surveys by
providing aquatic invertebrate biodiversity data and analyses at finer
scales. Past examples of such projects are surveys of wetlands in the
Drummond, Warden and Bryde Natural Diversity Recovery Catchments, the
Hutt River/Hutt Lagoon catchments and the mound springs near Three
Springs. This project runs on an `as-needed' basis.



\section*{Aims}
\begin{itemize}
\itemsep1pt\parskip0pt\parsep0pt
\item
  Provide understanding of aquatic biodiversity patterning at the scale
  of individual wetlands to wetland complexes and catchments to inform
  local conservation planning and as baselines for future monitoring.
\item
  Provide better data on the distribution, ecological tolerances and
  conservation status of aquatic fauna species and communities.
\end{itemize}



\section*{Progress}
\begin{itemize}
\itemsep1pt\parskip0pt\parsep0pt
\item
  Completed a brief report on aquatic invertebrates from the Brixton
  Street Threatened Ecological Community wetland for Swan Region.
\item
  Published a paper on aquatic invertebrates of Goldfields wetlands
  sampled in 2014 following a rare summer rainfall event.
\item
  Published paper on aquatic invertebrates of Cervantes to Coolimba area
  wetlands as part of an offset provided by CSR Gyprock.
\item
  Re-scored fringing flora quadrats of wetlands in the Cervantes to
  Coolimba area and commenced work on a paper. Lodged voucher specimens
  in Western Australian Herbarium.
\item
  Surveyed aquatic biodiversity of the springs and wetlands around
  Mandora Marsh (Walyarta), funded by BHP Billiton~via the Kimberley
  Region, in Aug/Sept 2015 and identified \textasciitilde{}85\% of
  specimens.
\item
  Continued to survey aquatic invertebrates and flora from wetlands in
  the middle to upper Fortescue Valley as part of a Pilbara Corridors -
  Rangelands NRM project. Work included identifying 95\% of the
  invertebrate specimens and all of the flora specimens collected from
  wetlands on Mulga Downs station in 2014/15, surveying wetlands on
  Ethel Creek and Roy Hill stations and commencing identification of the
  resulting flora collection.
\end{itemize}



\section*{Management implications}
\begin{itemize}
\itemsep1pt\parskip0pt\parsep0pt
\item
  In the Cervantes to Coolimba wetland system the survey of wetland
  flora will assist with assessment of proposals to expand gypsum mining
  and determining the ecological water requirements of groundwater
  dependent ecosystems.
\item
  The survey of aquatic invertebrates in Goldfields wetlands fills a gap
  in knowledge of biodiversity in an area still subject to intensive
  mining, thus allowing more informed assessment and approvals decision
  making.
\item
  On the Swan Coastal Plain,~fill a data gap (aquatic invertebrates in
  seasonal vegetated claypans) that will allow more informed land use
  planning.
\item
  Information~from the 2015~survey work on Mandora Marsh (Walyarta),
  combined with the baselining work completed there in 1999, will enable
  more informed conservation planning~for the new Walyarta Conservation
  Park, including assessment of management actions taken to protect
  mound springs. Survey results will also be used to report on the
  ecological character of the springs when reporting on this Ramsar
  wetlands to the Commonwealth.
\item
  In the Fortescue Valley, surveys of wetland biota will inform
  efficient wetland conservation planning in an important area for
  wetland biodiversity in the Pilbara.
\end{itemize}



\section*{Future directions}
\begin{itemize}
\itemsep1pt\parskip0pt\parsep0pt
\item
  Publish paper on invertebrate diversity in vegetated claypans of
  south-west Western Australia.
\item
  Publish meta-analyses of Western Australia arid zone invertebrate
  surveys.
\item
  Complete paper on Cervantes to Coolimba area wetland flora.
\item
  Complete identifications and supplementary~molecular work for~Mandora
  Marsh (Walyarta), analyse data and provide a report to the~Kimberley
  Region.
\item
  Complete identifications of already collected flora and fauna from the
  Pilbara Corridors project, survey additional wetlands on Roy Hill and
  Mount Florence stations and analyse data.
\end{itemize}



%-----------------------------------------------------------------------------%
% Back matter
%\backmatter
\end{document}
%-----------------------------------------------------------------------------%


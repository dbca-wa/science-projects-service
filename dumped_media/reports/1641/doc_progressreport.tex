
\documentclass[version=last,
    paper=a4, % paper size
    10pt, % default font size
    usenames,
    dvipsnames,
    oneside, % ONLINE
    headings=openany, % open chapters on odd and even pages
    %toc=chapterentrywithdots, % Table of Contents style
    %BCOR=7mm, % PRINT Binding Correction
    %DIV=13, % typearea 161.54 mm x 228.46 mm, top margin 22.85 mm, inner margin 16.15 mm
    %DIV=14, % 165.00 233.36 21.21 15.00
    DIV=15 % 168.00 237.60 19.80 14.00
]{scrbook}
\usepackage{typearea}
\usepackage[automark,headsepline,footsepline]{scrlayer-scrpage} % Headers and footers

%%
%% Fonts, encoding, spacing, indentation
%%
\usepackage{txfonts}
\renewcommand{\familydefault}{\sfdefault} % Default to Sans Serif font
\usepackage[english]{babel}
\usepackage[T1]{fontenc}
\usepackage[utf8]{inputenc}

% Paragraph spacing
%\usepackage{parskip}    % Paragraph spacing
%\setlength{\parindent}{0em} % Don't indent paragraphs - ONLINE
%\setlength{\parskip}{1.3 ex plus 0.5ex minus 0.3ex} % 1-1.8 ex vertical space between paragraphs - ONLINE

% Spacing of headings
%\RedeclareSectionCommand[afterskip=3pt]{section} % less space after section
%\RedeclareSectionCommand[beforeskip=0cm]{subsection} % less space between HRule and project name
%\RedeclareSectionCommand[afterskip=0.1\baselineskip]{subsubsection} % less space after progressreport subheadings

% Table font size
\usepackage{etoolbox}
\AtBeginEnvironment{longtabu}{\footnotesize}{}{}

%%
%% Tables, columns, layout
%%
\usepackage{multicol}   % 2 col publications
\usepackage{pdflscape}  % Landscape pages
\usepackage{pdfpages}   % Include PDFs
\usepackage{hanging}    % hanging paragraphs for publications
%\usepackage{titletoc}   % Required for manipulating the table of contents
\setcounter{tocdepth}{2} % TOC list down to section
\usepackage{enumerate}  % Enumerations
\usepackage{enumitem}   % Enumerations
\usepackage{longtable}  % Multipage table
\usepackage{tabu}       %
\setlength{\tabulinesep}{1.5mm} % Consistent vertical spacing in tabu

%%
%% Graphics, images, colours
%%
\usepackage{graphicx} % embedded images
\usepackage{eso-pic} %
\usepackage{colortbl} % define custom named colours
\definecolor{RedFire}{RGB}{146,25,28}
\definecolor{ParksWildlife}{RGB}{0,85,144}
\definecolor{successbg}{RGB}{223,240,216}
\definecolor{errorbg}{RGB}{242,222,222}
\definecolor{warningbg}{RGB}{252,248,227}
\definecolor{infobg}{RGB}{217,237,247}
\definecolor{muted}{RGB}{153,153,153}
\definecolor{success}{RGB}{70,136,71}
\definecolor{error}{RGB}{185,74,72}
\definecolor{warning}{RGB}{192,152,83}
\definecolor{info}{RGB}{58,135,173}

\definecolor{required}{RGB}{192,152,83}
\definecolor{requiredbg}{RGB}{252,248,227}
\definecolor{denied}{RGB}{185,74,72}
\definecolor{deniedbg}{RGB}{242,222,222}
\definecolor{granted}{RGB}{70,136,71}
\definecolor{grantedbg}{RGB}{223,240,216}
\definecolor{not reqiured}{RGB}{153,153,153}
\definecolor{not requiredbg}{RGB}{255,255,255}

\usepackage{tikz} % Drawing
\usetikzlibrary{arrows,shapes,positioning,shadows,trees}

%%
%% Links, URLs
%%
\usepackage[
    linktoc=all,
    %colorlinks=false,  %PRINT
    colorlinks=true, % ONLINE
    linkcolor=RedFire, % ONLINE
    urlcolor=ParksWildlife, % ONLINE
    pdftitle=Progress Report SP 2011-002 (FY 2015-2016)
]{hyperref}

% Black magic to linebreak URLs
\usepackage{url}
\makeatletter
\g@addto@macro{\UrlBreaks}{\UrlOrds}
\makeatother

%%
%% Custom macros
%%
% Thick Horizontal rule
\newcommand{\HRule}{\vspace{8mm}\\\noindent\rule{\linewidth}{0.1pt}}

% Custom Tikz node for SDS diagram
\newcommand\mynode[6][]{
    \node[#1] (#2){
        \parbox{#3\relax}{
            \begin{center}
            \textbf{#4}\\
            #5\\
            \footnotesize{#6}
            \end{center}}};}



%-----------------------------------------------------------------------------%
% Headers and Footers
\automark{section}
\ohead{\href{http://sdis.dpaw.wa.gov.au/documents/progressreport/1641/}{Progress Report SP 2011-002
}}
\chead{\href{http://sdis.dpaw.wa.gov.au}{SDIS}} % center header ONLINE
\ihead{\href{http://sdis.dpaw.wa.gov.au}{
        \includegraphics[scale=0.4]{/mnt/projects/sdis/staticfiles/img/logo-dpaw.png}}}
\ifoot{\textbf{Printed}~Mon, 4 Jul 2016 16:15:26 +0800} % inner/left footer
\cfoot{} % center footer
\ofoot{\pagemark} % outer/right footer
\pagestyle{scrheadings}
\setkomafont{pageheadfoot}{\normalfont}

%-----------------------------------------------------------------------------%
\begin{document}
\raggedbottom

%-----------------------------------------------------------------------------%
% Title page
\subject{Progress Report SP 2011-002
}
\title{Resolving the systematics and taxonomy of \emph{Tephrosia} in Western
Australia
}
\subtitle{Plant Science and Herbarium
}
\author{}
\publishers{\small
    \subsection*{Project Core Team}
\begin{tabu} {X X}
\textbf{Supervising Scientist} & Ryonen Butcher
\\
\textbf{Data Custodian} & Ryonen Butcher
\\
\textbf{Site Custodian} & Ryonen Butcher
\\
\end{tabu}


    \subsection*{Project status as of July 4, 2016, 4:15 p.m.}
\begin{tabu} {X X}
& Approved and active
\\
\end{tabu}

    
\subsection*{Document endorsements and approvals as of July 4, 2016, 4:15 p.m.}
\begin{tabu} {X X}

%\rowcolor{grantedbg}
    \textbf{Project Team} & 
    \textcolor{granted}{ granted}\\

%\rowcolor{grantedbg}
    \textbf{Program Leader} & 
    \textcolor{granted}{ granted}\\

%\rowcolor{grantedbg}
    \textbf{Directorate} & 
    \textcolor{granted}{ granted}\\

\end{tabu}



}
\uppertitleback{}
\lowertitleback{}
\date{}

%-----------------------------------------------------------------------------%
% Front matter
\frontmatter
\maketitle
%-----------------------------------------------------------------------------%
% Main matter
\mainmatter

\section*{Resolving the systematics and taxonomy of \emph{Tephrosia} in Western
Australia
}

R Butcher


\section*{Context}
\emph{Tephrosia} is a large, pantropical legume genus comprising
\emph{c.} 400 species of herbs and shrubs. Sixty-two taxa are currently
recognised in the Eremaean and Northern Botanical Provinces of Western
Australia; including 29 phrase-named taxa, with a number of species
complexes requiring further study. \emph{Tephrosia} specimens are
frequently collected during vegetation surveys for proposed mining
developments in northern Western Australia; however, many of them cannot
be adequately identified as they belong to poorly-known, undescribed
taxa or to species complexes. Their identification is further hindered
by the absence of up-to-date taxonomic keys and of comparable specimens,
as many species of \emph{Tephrosia} grow in remote areas and are poorly
collected. Identification difficulties inhibit the accurate assessment
of each taxon's distribution and hence its conservation status.



\section*{Aims}
\begin{itemize}
\itemsep1pt\parskip0pt\parsep0pt
\item
  Resolve the taxonomy of \emph{Tephrosia} in Western Australia using
  morphological and molecular approaches.
\item
  Assess the conservation status of all Western Australian taxa.
\item
  Prepare identification tools, including an electronic key to the
  genus.
\end{itemize}



\section*{Progress}
\begin{itemize}
\itemsep1pt\parskip0pt\parsep0pt
\item
  A new species was recognised from North West Cape and was added to
  WACensus as a phrase name.
\item
  A paper providing a conspectus of \emph{Tephrosia} in the Eremaean
  Botanical Province, including descriptions for 15 to 20 undescribed
  taxa, is in preparation.
\item
  All~\emph{Tephrosia} specimens submitted to the Western Australian
  Herbarium by external stakeholders were examined and had their
  identifications confirmed or corrected, thus maintaining the accuracy
  of~\emph{FloraBase}.
\item
  A final report is in preparation that provides a taxonomic key to
  all~\emph{Tephrosia} in WA as well as diagnostic descriptions for all
  informally named taxa, and~taxa reinstated or re-circumscribed as a
  result of this research for which descriptions are currently
  unavailable or incorrect.
\end{itemize}



\section*{Management implications}
Providing names, scientific descriptions, illustrations and
identification tools for the various \emph{Tephrosia} in Western
Australia will enable industry and conservation practioners to
accurately identify taxa, thereby improving their management and the
assessment of their conservation status. If it is found that the
individual \emph{Tephrosia} taxa can be identified through DNA
barcoding, this method will enable sterile or poor specimens, often
collected during botanical surveys, to be accurately identified.



\section*{Future directions}
\begin{itemize}
\itemsep1pt\parskip0pt\parsep0pt
\item
  Conduct further studies on poorly collected and taxonomically
  difficult species groups.
\item
  Analyse \emph{Tephrosia} DNA barcoding sequences in conjunction with
  researchers at the University of Guelph, to assess intra- and
  inter-specific variation and taxon relationships.
\item
  Continue with the construction of written and electronic
  identification tools.
\item
  Publish taxonomic papers describing new species endemic to~Western
  Australia's Northern Botanical Province (Kimberley region).
\item
  Collaborate with specialists in the Northern Territory and Queensland
  to resolve and describe new taxa occurring across Australia's monsoon
  tropics.
\end{itemize}



%-----------------------------------------------------------------------------%
% Back matter
%\backmatter
\end{document}
%-----------------------------------------------------------------------------%


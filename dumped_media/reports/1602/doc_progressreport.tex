
\documentclass[version=last,
    paper=a4, % paper size
    10pt, % default font size
    usenames,
    dvipsnames,
    oneside, % ONLINE
    headings=openany, % open chapters on odd and even pages
    %toc=chapterentrywithdots, % Table of Contents style
    %BCOR=7mm, % PRINT Binding Correction
    %DIV=13, % typearea 161.54 mm x 228.46 mm, top margin 22.85 mm, inner margin 16.15 mm
    %DIV=14, % 165.00 233.36 21.21 15.00
    DIV=15 % 168.00 237.60 19.80 14.00
]{scrbook}
\usepackage{typearea}
\usepackage[automark,headsepline,footsepline]{scrlayer-scrpage} % Headers and footers

%%
%% Fonts, encoding, spacing, indentation
%%
\usepackage{txfonts}
\renewcommand{\familydefault}{\sfdefault} % Default to Sans Serif font
\usepackage[english]{babel}
\usepackage[T1]{fontenc}
\usepackage[utf8]{inputenc}

% Paragraph spacing
%\usepackage{parskip}    % Paragraph spacing
%\setlength{\parindent}{0em} % Don't indent paragraphs - ONLINE
%\setlength{\parskip}{1.3 ex plus 0.5ex minus 0.3ex} % 1-1.8 ex vertical space between paragraphs - ONLINE

% Spacing of headings
%\RedeclareSectionCommand[afterskip=3pt]{section} % less space after section
%\RedeclareSectionCommand[beforeskip=0cm]{subsection} % less space between HRule and project name
%\RedeclareSectionCommand[afterskip=0.1\baselineskip]{subsubsection} % less space after progressreport subheadings

% Table font size
\usepackage{etoolbox}
\AtBeginEnvironment{longtabu}{\footnotesize}{}{}

%%
%% Tables, columns, layout
%%
\usepackage{multicol}   % 2 col publications
\usepackage{pdflscape}  % Landscape pages
\usepackage{pdfpages}   % Include PDFs
\usepackage{hanging}    % hanging paragraphs for publications
%\usepackage{titletoc}   % Required for manipulating the table of contents
\setcounter{tocdepth}{2} % TOC list down to section
\usepackage{enumerate}  % Enumerations
\usepackage{enumitem}   % Enumerations
\usepackage{longtable}  % Multipage table
\usepackage{tabu}       %
\setlength{\tabulinesep}{1.5mm} % Consistent vertical spacing in tabu

%%
%% Graphics, images, colours
%%
\usepackage{graphicx} % embedded images
\usepackage{eso-pic} %
\usepackage{colortbl} % define custom named colours
\definecolor{RedFire}{RGB}{146,25,28}
\definecolor{ParksWildlife}{RGB}{0,85,144}
\definecolor{successbg}{RGB}{223,240,216}
\definecolor{errorbg}{RGB}{242,222,222}
\definecolor{warningbg}{RGB}{252,248,227}
\definecolor{infobg}{RGB}{217,237,247}
\definecolor{muted}{RGB}{153,153,153}
\definecolor{success}{RGB}{70,136,71}
\definecolor{error}{RGB}{185,74,72}
\definecolor{warning}{RGB}{192,152,83}
\definecolor{info}{RGB}{58,135,173}

\definecolor{required}{RGB}{192,152,83}
\definecolor{requiredbg}{RGB}{252,248,227}
\definecolor{denied}{RGB}{185,74,72}
\definecolor{deniedbg}{RGB}{242,222,222}
\definecolor{granted}{RGB}{70,136,71}
\definecolor{grantedbg}{RGB}{223,240,216}
\definecolor{not reqiured}{RGB}{153,153,153}
\definecolor{not requiredbg}{RGB}{255,255,255}

\usepackage{tikz} % Drawing
\usetikzlibrary{arrows,shapes,positioning,shadows,trees}

%%
%% Links, URLs
%%
\usepackage[
    linktoc=all,
    %colorlinks=false,  %PRINT
    colorlinks=true, % ONLINE
    linkcolor=RedFire, % ONLINE
    urlcolor=ParksWildlife, % ONLINE
    pdftitle=Progress Report SP 2014-003 (FY 2015-2016)
]{hyperref}

% Black magic to linebreak URLs
\usepackage{url}
\makeatletter
\g@addto@macro{\UrlBreaks}{\UrlOrds}
\makeatother

%%
%% Custom macros
%%
% Thick Horizontal rule
\newcommand{\HRule}{\vspace{8mm}\\\noindent\rule{\linewidth}{0.1pt}}

% Custom Tikz node for SDS diagram
\newcommand\mynode[6][]{
    \node[#1] (#2){
        \parbox{#3\relax}{
            \begin{center}
            \textbf{#4}\\
            #5\\
            \footnotesize{#6}
            \end{center}}};}



%-----------------------------------------------------------------------------%
% Headers and Footers
\automark{section}
\ohead{\href{http://sdis.dpaw.wa.gov.au/documents/progressreport/1602/}{Progress Report SP 2014-003
}}
\chead{\href{http://sdis.dpaw.wa.gov.au}{SDIS}} % center header ONLINE
\ihead{\href{http://sdis.dpaw.wa.gov.au}{
        \includegraphics[scale=0.4]{/mnt/projects/sdis/staticfiles/img/logo-dpaw.png}}}
\ifoot{\textbf{Printed}~Tue, 5 Jul 2016 10:31:45 +0800} % inner/left footer
\cfoot{} % center footer
\ofoot{\pagemark} % outer/right footer
\pagestyle{scrheadings}
\setkomafont{pageheadfoot}{\normalfont}

%-----------------------------------------------------------------------------%
\begin{document}
\raggedbottom

%-----------------------------------------------------------------------------%
% Title page
\subject{Progress Report SP 2014-003
}
\title{Cat Eradication on Dirk Hartog Island
}
\subtitle{Animal Science
}
\author{}
\publishers{\small
    \subsection*{Project Core Team}
\begin{tabu} {X X}
\textbf{Supervising Scientist} & Dave Algar
\\
\textbf{Data Custodian} & M Johnston
\\
\textbf{Site Custodian} & Dave Algar
\\
\end{tabu}


    \subsection*{Project status as of July 5, 2016, 10:31 a.m.}
\begin{tabu} {X X}
& Update requested
\\
\end{tabu}

    
\subsection*{Document endorsements and approvals as of July 5, 2016, 10:31 a.m.}
\begin{tabu} {X X}

%\rowcolor{grantedbg}
    \textbf{Project Team} & 
    \textcolor{granted}{ granted}\\

%\rowcolor{grantedbg}
    \textbf{Program Leader} & 
    \textcolor{granted}{ granted}\\

%\rowcolor{requiredbg}
    \textbf{Directorate} & 
    \textcolor{required}{ required}\\

\end{tabu}



}
\uppertitleback{}
\lowertitleback{}
\date{}

%-----------------------------------------------------------------------------%
% Front matter
\frontmatter
\maketitle
%-----------------------------------------------------------------------------%
% Main matter
\mainmatter

\section*{Cat Eradication on Dirk Hartog Island
}

D Algar, G Desmond, J Fletcher, N Hamilton, M Johnston, M Onus, C Tiller


\section*{Context}
On Dirk Hartog Island, the largest island off the Western Australian
coast, 10 of the 13 species of native terrestrial mammals once present
are now locally extinct most likely due to predation by cats. The island
was established as a National Park in November 2009, which now provides
the opportunity to reconstruct the native mammal fauna. Dirk Hartog
Island could potentially support one of the most diverse mammal
assemblages in Australia and contribute significantly to the long-term
conservation of several threatened species. Eradication of feral cats
would be a necessary precursor to any mammal reintroductions. The
objective of this project is to develop and implement a successful
eradication campaign for feral cats on the island.



\section*{Aims}
The aims of this project are to facilitate native fauna reintroductions
to Dirk Hartog Island through researching feral cat behaviour and
susceptibility to baiting programs, implementing a cat eradication
program, and developing effective cat monitoring protocols that will
allow success of eradication programs to be assessed.



\section*{Progress}
~

\begin{itemize}
\itemsep1pt\parskip0pt\parsep0pt
\item
  Monitoring track access and camera traps for cats in both the southern
  and northern sections of Dirk Hartog Island continued.
\item
  Cat activity is absent south of the barrier fence.
\item
  Baiting of the northern section was undertaken in May~2015 and
  subsequent trapping has removed the majority of cats that remained
  post-baiting.~
\end{itemize}



\section*{Management implications}
The biodiversity outcome from this project will be a measurable decline
in the cat population on Dirk Hartog Island, eventually to zero when
eradication is confirmed. This is essential before fauna reconstruction
activities can commence. Cat eradication will also assist the
conservation of the extant fauna, including three threatened taxa. There
is global interest in the outcomes of this project and the techniques
used. Knowledge and technology transfer to other agencies contemplating
cat eradications on islands will be through presentations and
publication of manuscripts in scientific journals.



\section*{Future directions}
\begin{itemize}
\itemsep1pt\parskip0pt\parsep0pt
\item
  Verify absence of cats using independent techniques (detector dogs and
  sentinel cats).
\item
  Undertake seasonal surveillance monitoring for cat presence across the
  island and instigate immediate control effort if~detected.
\item
  It is anticipated that cats will have been eradicated by late~2016,
  with confirmation after extensive monitoring by late~2018.
\item
  Drafting a series of manuscripts for publication relevant to
  the~overall program and techniques developed. These manuscripts~are
  additional to those already published.
\end{itemize}



%-----------------------------------------------------------------------------%
% Back matter
%\backmatter
\end{document}
%-----------------------------------------------------------------------------%


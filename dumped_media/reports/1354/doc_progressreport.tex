
\documentclass[version=last, paper=a4, DIV=18, usenames, dvipsnames]{scrartcl}
\usepackage{txfonts}
\usepackage{pdflscape}
\usepackage{pdfpages}
\usepackage[english]{babel} % English language/hyphenation
%%% Bootstrap colors
\definecolor{RedFire}{RGB}{146,25,28}
\definecolor{ParksWildlife}{RGB}{0,85,144}
\definecolor{successbg}{RGB}{223,240,216}
\definecolor{errorbg}{RGB}{242,222,222}
\definecolor{warningbg}{RGB}{252,248,227}
\definecolor{infobg}{RGB}{217,237,247}
\definecolor{muted}{RGB}{153,153,153}
\definecolor{success}{RGB}{70,136,71}
\definecolor{error}{RGB}{185,74,72}
\definecolor{warning}{RGB}{192,152,83}
\definecolor{info}{RGB}{58,135,173}

\definecolor{required}{HTML}{D9534F}
\definecolor{denied}{HTML}{D9534F}
\definecolor{granted}{HTML}{47A447}
\definecolor{not required}{RGB}{200, 200, 200}

\usepackage[colorlinks=true,pdftitle=doc\_progressreport.pdf
,linktoc=all,linkcolor=RedFire,urlcolor=ParksWildlife]{hyperref}
\usepackage{colortbl}
\usepackage{longtable}
\usepackage{tabu}
\setlength{\tabulinesep}{1.5mm}
\usepackage{enumerate}
\usepackage{enumitem}
\usepackage{fancyhdr}
\usepackage{lastpage}
\usepackage{graphicx}
\usepackage{eso-pic}
\usepackage{hyphenat}
\renewcommand{\familydefault}{\sfdefault}



\newcommand{\HRule}{\rule{\linewidth}{0.1pt}}

\newcommand{\placetextbox}[3]{% \placetextbox{<horizontal pos>}{<vertical pos>}{<stuff>}
  \setbox0=\hbox{#3}% Put <stuff> in a box
  \AddToShipoutPictureFG*{% Add <stuff> to current page foreground
    \put(\LenToUnit{#1\paperwidth},\LenToUnit{#2\paperheight}){\vtop{{\null}\makebox[0pt][c]{#3}}}%
  }%
}%




%-----------------------------------------------------------------------------%
% Headers and footers
%
\fancypagestyle{plain}{
  \fancyhf{}
  \setlength\headheight{60pt} % push page content below header
  \renewcommand{\headrulewidth}{0.1pt}
  \renewcommand{\footrulewidth}{0.1pt}
  
  
  \fancyhead[L]{ 
    \href{http://sdis.dpaw.wa.gov.au}{
    \includegraphics[scale=0.6]{/mnt/projects/sdis/staticfiles/img/logo-dpaw.png}}
  }
  \fancyhead[R]{ 
      \hfill
      \href{http://sdis.dpaw.wa.gov.au}{Science Directorate Information System} 
      \newline 
      \href{http://sdis.dpaw.wa.gov.au/documents/progressreport/1354/}{Progress Report 2011-20 (FY 2014-2015)} 
  }
  
  
  
  
  \fancyfoot[L]{ \leftmark\newline\textbf{Printed}\textit{ June 15, 2015, 8:56 a.m. }}
  \fancyfoot[R]{  \, \newline Page \thepage\ of \pageref{LastPage} }
  
  
}
\pagestyle{plain}
%
% end Headers
%-----------------------------------------------------------------------------%

\begin{document}

%-----------------------------------------------------------------------------%
% Title page
%

%
% end title page
%-----------------------------------------------------------------------------%




\section*{Context Summary}
This project provides information to underpin the management of karri in
the immature stage of stand development (25-120 years old). Regenerated
karri stands have important values for future timber production,
biodiversity conservation and as a store of terrestrial carbon. Immature
stands regenerated following timber harvesting and bushfire comprise
more than 50,000 hectares and represent around one third of the area of
karri forest managed by the department. There are a number of
well-designed experiments that investigate the dynamics of naturally
regenerated and planted stands managed at a range of stand densities.
These experiments span a range of site productivity and climatic
gradients in the karri forest, and have been measured repeatedly over a
period of several decades, providing important information to support
and improve management practices. This project addresses emerging issues
likely to be of growing importance in the next decade, including climate
change and declining groundwater levels, interactions with pests and
pathogens, and increased recognition of the role of forests in
maintaining global carbon cycles.



\section*{Aims Summary}
To quantify the response of immature karri stands to management
practices that manipulate stand density at establishment or through
intervention by thinning. Responses will be measured by tree and stand
growth, tree health and other indicators as appropriate (e.g. leaf water
potential, leaf area index).



\section*{Progress}
\begin{itemize}
\itemsep1pt\parskip0pt\parsep0pt
\item
  Findings from the Warren block thinning experiment were reported in a
  book on long term ecological monitoring studies in Australia
  (\emph{Biodiversity and environmental change}: CSIRO Publishing,
  February 2014).
\end{itemize}



\section*{Management implications}
\begin{itemize}
\itemsep1pt\parskip0pt\parsep0pt
\item
  Thinning concentrates the growth potential of a site onto selected
  trees and provides forest managers with options to manage stands for
  particular structural characteristics.
\item
  Tree mortality associated with \emph{Armillaria} root disease appears
  to reduce in older stands, and small gaps created by dead trees become
  less obvious as stands mature. Localised tree mortality can be
  regarded as a natural process and is likely to contribute to
  patchiness in the mature forest. However, the extent of tree mortality
  in silviculturally managed stands should be monitored to ensure that
  stand productivity and other forest values remain within acceptable
  ranges.
\end{itemize}



\section*{Future directions}
\begin{itemize}
\itemsep1pt\parskip0pt\parsep0pt
\item
  Analyse and report on trends in tree and stand growth, with a focus on
  possible links between climate and growth.
\item
  Analyse trends in the incidence and severity of \emph{Armillaria} root
  disease at Warren block since 2000.
\item
  Develop a plan for a second thinning at Warren block.
\end{itemize}




\clearpage



\end{document}

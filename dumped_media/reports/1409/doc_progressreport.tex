
\documentclass[version=last, paper=a4, DIV=18, usenames, dvipsnames]{scrartcl}
\usepackage{txfonts}
\usepackage{pdflscape}
\usepackage{pdfpages}
\usepackage[english]{babel} % English language/hyphenation
%%% Bootstrap colors
\definecolor{RedFire}{RGB}{146,25,28}
\definecolor{ParksWildlife}{RGB}{0,85,144}
\definecolor{successbg}{RGB}{223,240,216}
\definecolor{errorbg}{RGB}{242,222,222}
\definecolor{warningbg}{RGB}{252,248,227}
\definecolor{infobg}{RGB}{217,237,247}
\definecolor{muted}{RGB}{153,153,153}
\definecolor{success}{RGB}{70,136,71}
\definecolor{error}{RGB}{185,74,72}
\definecolor{warning}{RGB}{192,152,83}
\definecolor{info}{RGB}{58,135,173}

\definecolor{required}{HTML}{D9534F}
\definecolor{denied}{HTML}{D9534F}
\definecolor{granted}{HTML}{47A447}
\definecolor{not required}{RGB}{200, 200, 200}

\usepackage[colorlinks=true,pdftitle=doc\_progressreport.pdf
,linktoc=all,linkcolor=RedFire,urlcolor=ParksWildlife]{hyperref}
\usepackage{colortbl}
\usepackage{longtable}
\usepackage{tabu}
\setlength{\tabulinesep}{1.5mm}
\usepackage{enumerate}
\usepackage{enumitem}
\usepackage{fancyhdr}
\usepackage{lastpage}
\usepackage{graphicx}
\usepackage{eso-pic}
\usepackage{hyphenat}
\renewcommand{\familydefault}{\sfdefault}



\newcommand{\HRule}{\rule{\linewidth}{0.1pt}}

\newcommand{\placetextbox}[3]{% \placetextbox{<horizontal pos>}{<vertical pos>}{<stuff>}
  \setbox0=\hbox{#3}% Put <stuff> in a box
  \AddToShipoutPictureFG*{% Add <stuff> to current page foreground
    \put(\LenToUnit{#1\paperwidth},\LenToUnit{#2\paperheight}){\vtop{{\null}\makebox[0pt][c]{#3}}}%
  }%
}%




%-----------------------------------------------------------------------------%
% Headers and footers
%
\fancypagestyle{plain}{
  \fancyhf{}
  \setlength\headheight{60pt} % push page content below header
  \renewcommand{\headrulewidth}{0.1pt}
  \renewcommand{\footrulewidth}{0.1pt}
  
  
  \fancyhead[L]{ 
    \href{http://sdis.dpaw.wa.gov.au}{
    \includegraphics[scale=0.6]{/mnt/projects/sdis/staticfiles/img/logo-dpaw.png}}
  }
  \fancyhead[R]{ 
      \hfill
      \href{http://sdis.dpaw.wa.gov.au}{Science Directorate Information System} 
      \newline 
      \href{http://sdis.dpaw.wa.gov.au/documents/progressreport/1409/}{Progress Report 2000-3 (FY 2014-2015)} 
  }
  
  
  
  
  \fancyfoot[L]{ \leftmark\newline\textbf{Printed}\textit{ June 30, 2015, 11:45 a.m. }}
  \fancyfoot[R]{  \, \newline Page \thepage\ of \pageref{LastPage} }
  
  
}
\pagestyle{plain}
%
% end Headers
%-----------------------------------------------------------------------------%

\begin{document}

%-----------------------------------------------------------------------------%
% Title page
%

%
% end title page
%-----------------------------------------------------------------------------%




\section*{Context Summary}
This is a long-term experiment established in 1999 to address part of
Ministerial Condition 12-3 attached to the \emph{Forest Management Plan
1994-2003}. Ministerial Condition 12-3 states that the Department shall
monitor and report on the status and effectiveness of silvicultural
measures in the intermediate rainfall zone (900-1100 mm/yr) of the
jarrah forest to protect water quality.



\section*{Aims Summary}
Investigate the hydrologic impacts of timber harvesting and associated
silvicultural treatments in the intermediate rainfall zone of the jarrah
forest.



\section*{Progress}
\begin{itemize}
\itemsep1pt\parskip0pt\parsep0pt
\item
  Monitoring of groundwater levels, streamflow, stream salinity and
  stream turbidity continued in Yarragil 6C (treated catchment) and
  Wuraming (control catchment).
\item
  Yarragil 4X (treated catchment) was not monitored during winter 2014
  because the corroded mild-steel V-notch weir plate was being replaced
  by a stainless steel weir plate.~
\item
  Monitoring of groundwater levels, streamflow, and stream salinity
  continued in Yarragil 4L, which was thinned in the mid 1980s, to
  examine the effect of thinning on stream water quality and quantity.
\item
  A paper reviewing the long-term hydrological response to thinning in
  Yarragil 4L is in preparation.
\end{itemize}



\section*{Management implications}
\begin{itemize}
\itemsep1pt\parskip0pt\parsep0pt
\item
  These catchments provide a unique long-term record of the hydrological
  response of the jarrah forest to climate change and forest management
  practices.
\item
  Monitoring in~these catchments contributes to reporting to KPI 10 for
  the \emph{Forest Management Plan 2014-23} which relates to stream
  condition and groundwater level within fully forested catchments.
\item
  Monitoring in these catchments helps inform understanding
  of~silviculture for water production.
\end{itemize}



\section*{Future directions}
\begin{itemize}
\itemsep1pt\parskip0pt\parsep0pt
\item
  Continue monitoring of groundwater levels, streamflow, stream salinity
  and turbidity and rainfall.
\item
  Re-measure forest density along fixed transects in Yarragil 4X and 6C
  to determine the forest regeneration response to the timber harvest
  and silvicultural treatments.
\item
  The mild-steel V-notch weir plate in Yarragil 4L is corroded and
  should be replaced by a stainless steel plate to extend the
  operational life of the weir for ongoing stream monitoring.
\item
  Re-measure tree growth in yarragil 4L to determine the long-term
  hydrological response to thinning, and write a paper.
\item
  Examine the feasibility of a second thinning in Yarragil 4L, 35 years
  after the previous thinning, with a view to informing silviculture for
  water production.
\end{itemize}




\clearpage



\end{document}


\documentclass[version=last,
    paper=a4, % paper size
    10pt, % default font size
    usenames,
    dvipsnames,
    oneside, % ONLINE
    headings=openany, % open chapters on odd and even pages
    %toc=chapterentrywithdots, % Table of Contents style
    %BCOR=7mm, % PRINT Binding Correction
    %DIV=13, % typearea 161.54 mm x 228.46 mm, top margin 22.85 mm, inner margin 16.15 mm
    %DIV=14, % 165.00 233.36 21.21 15.00
    DIV=15 % 168.00 237.60 19.80 14.00
]{scrbook}
\usepackage{typearea}
\usepackage[automark,headsepline,footsepline]{scrlayer-scrpage} % Headers and footers

%%
%% Fonts, encoding, spacing, indentation
%%
\usepackage{txfonts}
\renewcommand{\familydefault}{\sfdefault} % Default to Sans Serif font
\usepackage[english]{babel}
\usepackage[T1]{fontenc}
\usepackage[utf8]{inputenc}

% Paragraph spacing
%\usepackage{parskip}    % Paragraph spacing
%\setlength{\parindent}{0em} % Don't indent paragraphs - ONLINE
%\setlength{\parskip}{1.3 ex plus 0.5ex minus 0.3ex} % 1-1.8 ex vertical space between paragraphs - ONLINE

% Spacing of headings
%\RedeclareSectionCommand[afterskip=3pt]{section} % less space after section
%\RedeclareSectionCommand[beforeskip=0cm]{subsection} % less space between HRule and project name
%\RedeclareSectionCommand[afterskip=0.1\baselineskip]{subsubsection} % less space after progressreport subheadings

% Table font size
\usepackage{etoolbox}
\AtBeginEnvironment{longtabu}{\footnotesize}{}{}

%%
%% Tables, columns, layout
%%
\usepackage{multicol}   % 2 col publications
\usepackage{pdflscape}  % Landscape pages
\usepackage{pdfpages}   % Include PDFs
\usepackage{hanging}    % hanging paragraphs for publications
%\usepackage{titletoc}   % Required for manipulating the table of contents
\setcounter{tocdepth}{2} % TOC list down to section
\usepackage{enumerate}  % Enumerations
\usepackage{enumitem}   % Enumerations
\usepackage{longtable}  % Multipage table
\usepackage{tabu}       %
\setlength{\tabulinesep}{1.5mm} % Consistent vertical spacing in tabu

%%
%% Graphics, images, colours
%%
\usepackage{graphicx} % embedded images
\usepackage{eso-pic} %
\usepackage{colortbl} % define custom named colours
\definecolor{RedFire}{RGB}{146,25,28}
\definecolor{ParksWildlife}{RGB}{0,85,144}
\definecolor{successbg}{RGB}{223,240,216}
\definecolor{errorbg}{RGB}{242,222,222}
\definecolor{warningbg}{RGB}{252,248,227}
\definecolor{infobg}{RGB}{217,237,247}
\definecolor{muted}{RGB}{153,153,153}
\definecolor{success}{RGB}{70,136,71}
\definecolor{error}{RGB}{185,74,72}
\definecolor{warning}{RGB}{192,152,83}
\definecolor{info}{RGB}{58,135,173}

\definecolor{required}{RGB}{192,152,83}
\definecolor{requiredbg}{RGB}{252,248,227}
\definecolor{denied}{RGB}{185,74,72}
\definecolor{deniedbg}{RGB}{242,222,222}
\definecolor{granted}{RGB}{70,136,71}
\definecolor{grantedbg}{RGB}{223,240,216}
\definecolor{not reqiured}{RGB}{153,153,153}
\definecolor{not requiredbg}{RGB}{255,255,255}

\usepackage{tikz} % Drawing
\usetikzlibrary{arrows,shapes,positioning,shadows,trees}

%%
%% Links, URLs
%%
\usepackage[
    linktoc=all,
    %colorlinks=false,  %PRINT
    colorlinks=true, % ONLINE
    linkcolor=RedFire, % ONLINE
    urlcolor=ParksWildlife, % ONLINE
    pdftitle=Progress Report SP 2014-018 (FY 2015-2016)
]{hyperref}

% Black magic to linebreak URLs
\usepackage{url}
\makeatletter
\g@addto@macro{\UrlBreaks}{\UrlOrds}
\makeatother

%%
%% Custom macros
%%
% Thick Horizontal rule
\newcommand{\HRule}{\vspace{8mm}\\\noindent\rule{\linewidth}{0.1pt}}

% Custom Tikz node for SDS diagram
\newcommand\mynode[6][]{
    \node[#1] (#2){
        \parbox{#3\relax}{
            \begin{center}
            \textbf{#4}\\
            #5\\
            \footnotesize{#6}
            \end{center}}};}



%-----------------------------------------------------------------------------%
% Headers and Footers
\automark{section}
\ohead{\href{http://sdis.dpaw.wa.gov.au/documents/progressreport/1599/}{Progress Report SP 2014-018
}}
\chead{\href{http://sdis.dpaw.wa.gov.au}{SDIS}} % center header ONLINE
\ihead{\href{http://sdis.dpaw.wa.gov.au}{
        \includegraphics[scale=0.4]{/mnt/projects/sdis/staticfiles/img/logo-dpaw.png}}}
\ifoot{\textbf{Printed}~Mon, 11 Jul 2016 13:31:29 +0800} % inner/left footer
\cfoot{} % center footer
\ofoot{\pagemark} % outer/right footer
\pagestyle{scrheadings}
\setkomafont{pageheadfoot}{\normalfont}

%-----------------------------------------------------------------------------%
\begin{document}
\raggedbottom

%-----------------------------------------------------------------------------%
% Title page
\subject{Progress Report SP 2014-018
}
\title{Distribution and abundance estimate of Australian snubfin dolphins
(\emph{Orcaella heinsohni}) at a key site in the Kimberley region,
Western Australia
}
\subtitle{Marine Science
}
\author{}
\publishers{\small
    \subsection*{Project Core Team}
\begin{tabu} {X X}
\textbf{Supervising Scientist} & Kelly Waples
\\
\textbf{Data Custodian} & Holly Raudino
\\
\textbf{Site Custodian} & Holly Raudino
\\
\end{tabu}


    \subsection*{Project status as of July 11, 2016, 1:31 p.m.}
\begin{tabu} {X X}
& Approved and active
\\
\end{tabu}

    
\subsection*{Document endorsements and approvals as of July 11, 2016, 1:31 p.m.}
\begin{tabu} {X X}

%\rowcolor{grantedbg}
    \textbf{Project Team} & 
    \textcolor{granted}{ granted}\\

%\rowcolor{grantedbg}
    \textbf{Program Leader} & 
    \textcolor{granted}{ granted}\\

%\rowcolor{grantedbg}
    \textbf{Directorate} & 
    \textcolor{granted}{ granted}\\

\end{tabu}



}
\uppertitleback{}
\lowertitleback{}
\date{}

%-----------------------------------------------------------------------------%
% Front matter
\frontmatter
\maketitle
%-----------------------------------------------------------------------------%
% Main matter
\mainmatter

\section*{Distribution and abundance estimate of Australian snubfin dolphins
(\emph{Orcaella heinsohni}) at a key site in the Kimberley region,
Western Australia
}

K Waples, H Raudino


\section*{Context}
The current lack of knowledge of the Australian snubfin dolphin
(\emph{Orcaella heinsohni}) meant that its conservation status could not
be properly assessed in 2011 due to insufficent information on
population dynamics and distribution. This species is known from
tropical coastal waters of Australia and New Guinea, but tend to be shy,
evasive and difficult to study. Although they range southwards to the
the Pilbara region of WA, there has been little Western Australian-based
research on this species and much of this remains unpublished. This
project will compile existing data on snubfin dolphins across the
Kimberley to gain a better understanding of their habitat use and
distribution. The collation of data into a single database will also
facilitate the study of population structure and demographics based on
recognised individual animals. This project will assess dolphin
distribution across the Kimberley region between 2004-2012.

This project was funded by a grant from the Australian Marine Mammal
Centre and was undertaken in partnership with Dr Deborah Thiele (ANU)
who provided the dolphin survey data and Dr Philip Bouchet~(UWA)~who
provided data analysis expertise. A number of indigenous sea ranger
groups in the Kimberley participated in dolphin surveys and are
providing input to the associated broad-scale habitat use mapping.~



\section*{Aims}
This project will use existing data to:

\begin{enumerate}
\itemsep1pt\parskip0pt\parsep0pt
\item
  Provide a quantitative abundance estimate of snubfin dolphins for
  Roebuck Bay in Western Australia that will be used as a baseline for
  this population and will also enable comparison with abundance
  estimates of the species from sites at Cleveland Bay (Qld) and Port
  Essington (NT).
\item
  Compare methods for abundance estimation (mark-recapture versus
  distance sampling) and the suitability of these methods for abundance
  estimation of this species.
\item
  Map the extent of occurrence and area of occupancy of snubfin dolphins
  in the Kimberley by combining traditional knowledge and dolphin
  sightings from indigenous sea rangers and scientific survey sightings.
\item
  Refine and populate a purpose built and standardised database which
  will support long term data collection and curation~in WA and
  facilitate data-sharing between jurisdictions.
\end{enumerate}



\section*{Progress}
\begin{itemize}
\itemsep1pt\parskip0pt\parsep0pt
\item
  A manuscript titled~\emph{Cross-cultural knowledge informs the
  distribution of snubfin dolphins in the Kimberley, Western Australia}
  has been submitted to the journal \emph{Endangered Species.}The paper
  was accepted for publication but has subsequently been withdrawn at
  the request of co-authors.
\item
  A manuscript on snubfin dolphin abundance~contrasting distance
  sampling and mark-recapture survey techniques in Roebuck Bay has been
  prepared for submission.~
\end{itemize}



\section*{Management implications}
This research project has brought together scientific and traditional
knowledge of a poorly understand marine mammal species of high
conservation value. The implications and relevance for management are:

\begin{itemize}
\itemsep1pt\parskip0pt\parsep0pt
\item
  Managers now have baseline knowledge of the abundance of snubfin
  dolphins in the proposed Yawuru Nagulagun / Roebuck Bay Marine Park.~
\item
  A database has been established that will be maintained by Parks and
  Wildlife for all dolphin research and monitoring where survey and
  photo-identification data is collected. The database ensures~ that
  standardised data is available for assessing population abundance and
  distribution.~It also provides the capacity to develop~sighting
  histories for individual animals, thus providing a better
  understanding of population demographics and life history. This
  database can also be used for information sharing across jurisdictions
  and between research organisations.~
\item
  The broad-scale collation of information and modeling has provided
  relevant information on area of occupancy and extent of occurrence
  that can be used to more accurately assess the conservation status of
  snubfin dolphins.
\item
  Partnerships have been established with indigenous sea ranger groups
  to develop survey methodologies, data storage and reporting structures
  that are consistent with healthy country plans and reserve management
  plans.~
\end{itemize}



\section*{Future directions}
\begin{itemize}
\itemsep1pt\parskip0pt\parsep0pt
\item
  Submission of the manuscript \emph{Population estimate of the
  Australian snubfin dolphin}(Orcaella heinsohni)\emph{Roebuck Bay,
  Western Australia} to a peer-reviewed journal.
\item
  Re-submission of the manuscript \emph{Cross-cultural knowledge informs
  the distribution of the Australian snubfin dolphin}(Orcaella
  heinsohni)\emph{in the Kimberley, Western Australia} to the journal
  \emph{Biological Conservation.}
\item
  Support and advice will be provided to two Australian Government
  funded projects on snubfin dolphins through facilitating access to and
  training in the~use of the Department's DolFin Database.~
\end{itemize}



%-----------------------------------------------------------------------------%
% Back matter
%\backmatter
\end{document}
%-----------------------------------------------------------------------------%


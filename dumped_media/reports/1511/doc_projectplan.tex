
\documentclass[version=last,
    paper=a4, % paper size
    10pt, % default font size
    usenames,
    dvipsnames,
    oneside, % ONLINE
    headings=openany, % open chapters on odd and even pages
    %toc=chapterentrywithdots, % Table of Contents style
    %BCOR=7mm, % PRINT Binding Correction
    %DIV=13, % typearea 161.54 mm x 228.46 mm, top margin 22.85 mm, inner margin 16.15 mm
    %DIV=14, % 165.00 233.36 21.21 15.00
    DIV=15 % 168.00 237.60 19.80 14.00
]{scrbook}
\usepackage{typearea}
\usepackage[automark,headsepline,footsepline]{scrlayer-scrpage} % Headers and footers

%%
%% Fonts, encoding, spacing, indentation
%%
\usepackage{txfonts}
\renewcommand{\familydefault}{\sfdefault} % Default to Sans Serif font
\usepackage[english]{babel}
\usepackage[T1]{fontenc}
\usepackage[utf8]{inputenc}

% Paragraph spacing
%\usepackage{parskip}    % Paragraph spacing
%\setlength{\parindent}{0em} % Don't indent paragraphs - ONLINE
%\setlength{\parskip}{1.3 ex plus 0.5ex minus 0.3ex} % 1-1.8 ex vertical space between paragraphs - ONLINE

% Spacing of headings
%\RedeclareSectionCommand[afterskip=3pt]{section} % less space after section
%\RedeclareSectionCommand[beforeskip=0cm]{subsection} % less space between HRule and project name
%\RedeclareSectionCommand[afterskip=0.1\baselineskip]{subsubsection} % less space after progressreport subheadings

% Table font size
\usepackage{etoolbox}
\AtBeginEnvironment{longtabu}{\footnotesize}{}{}

%%
%% Tables, columns, layout
%%
\usepackage{multicol}   % 2 col publications
\usepackage{pdflscape}  % Landscape pages
\usepackage{pdfpages}   % Include PDFs
\usepackage{hanging}    % hanging paragraphs for publications
%\usepackage{titletoc}   % Required for manipulating the table of contents
\setcounter{tocdepth}{2} % TOC list down to section
\usepackage{enumerate}  % Enumerations
\usepackage{enumitem}   % Enumerations
\usepackage{longtable}  % Multipage table
\usepackage{tabu}       %
\setlength{\tabulinesep}{1.5mm} % Consistent vertical spacing in tabu

%%
%% Graphics, images, colours
%%
\usepackage{graphicx} % embedded images
\usepackage{eso-pic} %
\usepackage{colortbl} % define custom named colours
\definecolor{RedFire}{RGB}{146,25,28}
\definecolor{ParksWildlife}{RGB}{0,85,144}
\definecolor{successbg}{RGB}{223,240,216}
\definecolor{errorbg}{RGB}{242,222,222}
\definecolor{warningbg}{RGB}{252,248,227}
\definecolor{infobg}{RGB}{217,237,247}
\definecolor{muted}{RGB}{153,153,153}
\definecolor{success}{RGB}{70,136,71}
\definecolor{error}{RGB}{185,74,72}
\definecolor{warning}{RGB}{192,152,83}
\definecolor{info}{RGB}{58,135,173}

\definecolor{required}{RGB}{192,152,83}
\definecolor{requiredbg}{RGB}{252,248,227}
\definecolor{denied}{RGB}{185,74,72}
\definecolor{deniedbg}{RGB}{242,222,222}
\definecolor{granted}{RGB}{70,136,71}
\definecolor{grantedbg}{RGB}{223,240,216}
\definecolor{not reqiured}{RGB}{153,153,153}
\definecolor{not requiredbg}{RGB}{255,255,255}

\usepackage{tikz} % Drawing
\usetikzlibrary{arrows,shapes,positioning,shadows,trees}

%%
%% Links, URLs
%%
\usepackage[
    linktoc=all,
    %colorlinks=false,  %PRINT
    colorlinks=true, % ONLINE
    linkcolor=RedFire, % ONLINE
    urlcolor=ParksWildlife, % ONLINE
    pdftitle=Project Plan SP 2015-017
]{hyperref}

% Black magic to linebreak URLs
\usepackage{url}
\makeatletter
\g@addto@macro{\UrlBreaks}{\UrlOrds}
\makeatother

%%
%% Custom macros
%%
% Thick Horizontal rule
\newcommand{\HRule}{\vspace{8mm}\\\noindent\rule{\linewidth}{0.1pt}}

% Custom Tikz node for SDS diagram
\newcommand\mynode[6][]{
    \node[#1] (#2){
        \parbox{#3\relax}{
            \begin{center}
            \textbf{#4}\\
            #5\\
            \footnotesize{#6}
            \end{center}}};}



%-----------------------------------------------------------------------------%
% Headers and Footers
\automark{section}
\ohead{\href{http://sdis.dpaw.wa.gov.au/documents/projectplan/1511/}{Project Plan SP 2015-017
}}
\chead{\href{http://sdis.dpaw.wa.gov.au}{SDIS}} % center header ONLINE
\ihead{\href{http://sdis.dpaw.wa.gov.au}{
        \includegraphics[scale=0.4]{/mnt/projects/sdis/staticfiles/img/logo-dpaw.png}}}
\ifoot{\textbf{Printed}~Tue, 28 Feb 2017 10:55:42 +0800} % inner/left footer
\cfoot{} % center footer
\ofoot{\pagemark} % outer/right footer
\pagestyle{scrheadings}
\setkomafont{pageheadfoot}{\normalfont}

%-----------------------------------------------------------------------------%
\begin{document}
\raggedbottom

%-----------------------------------------------------------------------------%
% Title page
\subject{Project Plan SP 2015-017
}
\title{Responses of aquatic invertebrate communities to changing hydrology and
water quality in streams and significant wetlands of the south-west
forests of Western Australia
}
\subtitle{Wetlands Conservation
}
\author{}
\publishers{\small
    \subsection*{Project Core Team}
\begin{tabu} {X X}
\textbf{Supervising Scientist} & Melita Pennifold
\\
\textbf{Data Custodian} & Melita Pennifold
\\
\textbf{Site Custodian} & 
\\
\end{tabu}


    \subsection*{Project status as of Feb. 28, 2017, 10:55 a.m.}
\begin{tabu} {X X}
& Approved and active
\\
\end{tabu}

    
\subsection*{Document endorsements and approvals as of Feb. 28, 2017, 10:55 a.m.}
\begin{tabu} {X X}

%\rowcolor{grantedbg}
    \textbf{Project Team} & 
    \textcolor{granted}{ granted}\\

%\rowcolor{grantedbg}
    \textbf{Program Leader} & 
    \textcolor{granted}{ granted}\\

%\rowcolor{grantedbg}
    \textbf{Directorate} & 
    \textcolor{granted}{ granted}\\

%\rowcolor{grantedbg}
    \textbf{Biometrician} & 
    \textcolor{granted}{ granted}\\

%\rowcolor{not requiredbg}
    \textbf{Herbarium Curator} & 
    \textcolor{not required}{ not required}\\

%\rowcolor{not requiredbg}
    \textbf{Animal Ethics Committee} & 
    \textcolor{not required}{ not required}\\

\end{tabu}



}
\uppertitleback{}
\lowertitleback{}
\date{}

%-----------------------------------------------------------------------------%
% Front matter
\frontmatter
\maketitle
%-----------------------------------------------------------------------------%
% Main matter
\mainmatter



\section*{Responses of aquatic invertebrate communities to changing hydrology and
water quality in streams and significant wetlands of the south-west
forests of Western Australia
}


\subsection*{Science and Conservation Division Program}
Wetlands Conservation



\subsection*{Parks and Wildlife Service}
Service 4: Forest Management Plan Implementation


\subsection*{Project Staff}
\begin{tabu} {X X X}
%\rowcolor{infobg}
\textbf{Role} & \textbf{Person} & \textbf{Time allocation (FTE)}\\

Supervising Scientist & Melita  Pennifold & 0.5\\

Supervising Scientist & Adrian  Pinder & 0.1\\

\end{tabu}



\subsection*{Related Science Projects}
None


\subsection*{Proposed period of the project}
July 1, 2014 -- June 30, 2023



\section*{Relevance and Outcomes}

\subsection*{Background}
Aquatic habitats in the south-west of WA are under increasing threat
from changes in hydrology, water quality and fire as a result of the
drying climate and historical and current land use. At present, there is
an inadequate understanding of the responses of aquatic communities to
these threats to inform the management of many aquatic systems in the
Forest Management Plan area, including the Muir-Byenup Ramsar wetlands.

This project has two components. Priority will be given to component 1,
component 2 will be dependent on funding.

\emph{1) Re-surveys of aquatic invertebrates in Muir-Byenup Ramsar
wetlands sampled in 1994 and 2004 and suites of wetlands further south
sampled in 1993. This addresses KPI3 of the 2014-23 FMP.}

The FMP area has many high value wetlands, particularly in the Warren
region. Some of these are listed as nationally or internationally
significant and~many are priorities in regional nature conservation
plans. These support numerous priority flora species, priority
ecological communities, significant waterbirds, 6 of the 8 species of
south-west endemic fish and a very high diversity and endemicity of
invertebrates. Threats to many of these wetlands have intensified over
the last decade. The available biodiversity data is 10 to 20 year old
and up to date information is required to assess responses to threats
and inform the allocation of resources to management actions.

~

\emph{2) Continued monitoring of high condition streams, with a focus on
effects of the drying climate and forest management. This addresses KPI1
of the 2014-23 FMP.}

KPI20 of the previous FMP scored 24 of 51 monitored stream sites as
impaired. This was not clearly related to forestry activities but could,
in part, be related to reduced rainfall. This project would see
continued monitoring at `reference condition' streams and those that are
already affected by reduced rainfall. A focus on these streams aligns
with KPI1 of the current FMP which focuses on change in `currently
healthy ecosystems' and will allow us to track condition in relation to
the ongoing decline in rainfall combined with forest management. In a
region with high climatic variability long-term studies are essential to
understand ecosystem responses.



\subsection*{Aims}
\begin{itemize}
\item
  To address KPI3 of the 2014-2023 FMP by determining responses of fauna
  of high value Warren region wetlands to changes in hydrology, water
  chemistry and fire over the last 10 to 20 years.
\item
  To address KPI1 of the 2014-2023 FMP by monitoring the condition of
  currently healthy streams in relation to reduced rainfall and forest
  management practices.
\item
  Provide baseline data for some internationally significant wetlands,
  e.g. Lake Muir.
\item
  Use the above information to report on the current conservation
  significance of key DPaW managed wetlands and their response and
  vulnerability to threats.
\end{itemize}



\subsection*{Expected outcome}
1)~~~ FMP commitments met with regard to measuring and assessing change
in condition of: 1) currently healthy (reference condition) stream
ecosystems (KPI1); and 2) Ramsar and nationally listed wetlands (KPI3).
Results will provide information on the effectiveness of forest
management practices in protecting stream and wetland biodiversity.

2)~~~ Warren Region conservation managers will have the information
needed to address a priority identified in the 2009-14 Warren Region
Nature Conservation Plan: Target 5, candidate action 1, including the
milestones.

``\emph{Analyse condition trends} {[}of 7 nationally listed wetlands{]}
\emph{and update adaptive management targets on the basis of these
trends}'' and

``\emph{Establish and consolidate benchmark information for Broke Inlet,
Doggerup, Marringup, Mt Soho Swamp, and Byenup to determine condition,
identify threats and to determine interim management actions}\emph{.''}

Indications from Warren Region staff are that these wetlands will be a
priority in the new 2015-20 Regional Nature Conservation Plan.

Using the above information on knowledge of the biodiversity
trajectories of wetlands in the Muir-Byenup Ramsar system and other
wetlands Warren Region staff will be better placed to prioritise
conservation actions.

3)~~~ DPaW will be able to report on the condition of a significant
Ramsar site.



\subsection*{Knowledge transfer}
The main users of the knowledge gained in this project will be DPaW
Warren Region staff and other DPaW staff working in areas which require
them to understand and manage streams and wetlands, especially those in
the conservation estate.

Results will also be used by Forest and Ecosystems Management Division
and the Conservation Commission to assess compliance with the FMP
2014-2023.

Project results will be disseminated by:

\begin{itemize}
\itemsep1pt\parskip0pt\parsep0pt
\item
  Reports on the aquatic invertebrates of the Muir-Byenup and South
  Coast wetlands (KPI3 component)
\item
  Reports on condition of selected streams in the FMP area (KPI1
  component)
\item
  Science Information Sheets and scientific publications as appropriate.
\item
  Presentations at workshops etc.
\end{itemize}



\subsection*{Tasks and Milestones}
Below is a timeline for a five year project including sampling the
Muir-Byenup Ramsar wetlands and significant wetlands along the south
coast, plus an additional sampling of a selection of the KPI20 streams.
Further sampling of these streams would be dependant on continued
funding.

\begin{longtable}[c]{@{}ll@{}}
\toprule\addlinespace
\textbf{Time} & \textbf{~Task}
\\\addlinespace
Spring 2014 and Summer 2015: & ~Sample Muir-Byenup wetlands
\\\addlinespace
Autumn 2015 to Summer 2015/16 & ~Process Muir-Byenup samples
\\\addlinespace
Spring 2015 to Autumn 2016 & ~Identification and vouchering of
Muir-Byenup invertebrates
\\\addlinespace
Autumn/Winter 2016 & Data analysis and reporting on Muir-Byenup wetlands
\\\addlinespace
Spring 2016 & ~Sample KPI1 streams
\\\addlinespace
Spring 2016 to Summer 2016/17 & Identification of stream invertebrate
samples from 2013 and 2016
\\\addlinespace
Autumn 2017 to Winter 2017 & Analysis and reporting of KPI20 and KPI1
stream invertebrate data
\\\addlinespace
Spring 2017 and Summer 2017/18 & ~Sampling south coast wetlands
previously sampled by Pierre Horwitz
\\\addlinespace
Autumn 2018 to Summer 2018/19 & ~Process south coast samples
\\\addlinespace
Spring 2018 to Autumn 2019 & ~Identification and vouchering of south
coast invertebrates
\\\addlinespace
Autumn/Winter 2019 & Data analysis and reporting on south coast wetlands
\\\addlinespace
2019 onwards & Optional further stream sampling
\\\addlinespace
\bottomrule
\end{longtable}



\subsection*{References}
Aquatic Research Laboratory. (1992) Survey of the macroinvertebrate
fauna and water chemistry of permanent lakes of the south coast of
Western Australia. A report for the Department of Conservation and Land
Management. University of Western Australia The University of Western
Australia Aquatic Research Laboratory,,Perth. 22.

Davies P. \& Stewart B. (2013) Aquatic biodiversity in the Mediterranean
climate rivers of southwestern Australia. \emph{Hydrobiologia,}
\textbf{719,} 215-235.

Edward D.H.D., Gazey P. \& Davies P.M. (1994) Invertebrate community
structure related to physico-chemical parameters of permanent lakes of
the south coast of Western Australia. \emph{Journal of the Royal Society
of Western Australia,} \textbf{77,} 51-63.

Edward D.H.D., Storey A.W. \& Smith M.J.B. (2000) Assessing river health
in south-western Australia: A comparison of macroinvertebrates at family
level with Chironomidae at species level. \emph{Verhandlungen
Internationale Vereinigung für theoretische und angewandte Limnologie,}
\textbf{27,} 2326-2335.

Farrell C. \& Cook B. (2009) Ecological character description of the
Muir-Byenup System Ramsar site south-west Western Australia. Centre of
Excellence in Natural Resource Management;University of Western
Australia Report prepared for the Department of Environment and
Conservation.

Horwitz P. (1994) Patterns of endemism in the freshwater fauna of the
far southern peatlands and shrublands of southwestern Australia. Edith
Cowan University,Perth.

Horwitz P. (1997) Comparative endemism and richness of the aquatic
invertebrate fauna in peatlands and shrublands of far south-western
Australia. \emph{Memoirs of the Museum of Victoria,} \textbf{56,}
313-321.

Horwitz P. (1999) Wetlands in south-western Australia and their organic
material - To burn or not to burn? \emph{Western Wildlife,} \textbf{3}.

Horwitz P., Bradshaw D., Hopper S., Davies P., Froend R. \& Bradshaw F.
(2008) Hydrological change escalates risk of ecosystem stress in
Australia's threatened biodiversity hotspot. \emph{Journal of the Royal
Society of Western Australia,} \textbf{91,} 1-11.

Horwitz P., Jasinska E.J., Fairhurst E. \& Davis J.A. (1997) A review of
knowledge on the effect of key disturbances on aquatic invertebrates and
fish in the south-west forest region of Western Australia. Commonwealth
and Western Australian Governments for the Western Australian Regional
Forest Agreement,Perth.

Horwitz P., Pemberton M. \& Ryder D. (1996) Catastrophic loss of organic
carbon from a management fire in a peatland in southwestern Australia.
In: \emph{Wetlands for the Future}.

Kingsford R.T. (2011) Conservation management of rivers and wetlands
under climate change -- a synthesis. \emph{Marine and Freshwater
Research,} \textbf{62,} 217-222.

Storey A. (1998) Assessment of the nature conservation values of the
Byenup-Muir peat swamp system, southwestern Australia:
physico-chemistry, aquatic macroinvertebrates and fishes.

Trayler K.M., Davis J.A., Horwitz P. \& Morgan D. (1996) Aquatic Fauna
of the Warren bioregion, south-west Western Australia: Does reservation
guarantee preservation? \emph{Journal of the Royal Society of Western
Australia,} \textbf{79}.

Wetland Research and Management. (2005) Re-assessment of the nature
conservation values of the Byenup-Muir peat swamp system. Draft Report
Prepared for Department of Conservation and Land Management.



\section*{Study design}

\subsection*{Methodology}




\subsection*{Biometrician's Endorsement}
granted



\section*{Data management}

\subsection*{No. specimens}




\subsection*{Herbarium Curator's Endorsement}
not required



\subsection*{Animal Ethics Committee's Endorsement}
not required



\subsection*{Data management}
Data will be entered into the Wetlands Conservation Program aquatic
database.

Data will also be uploaded to Naturemap where biological and
physichemical~data will be~described in the catalogue.

Specimen vouchers will be lodged with the Western Australin Museum and
the Wetlands Conservation Program aquatic invertebrate voucher
collection.




\section*{Budget}

\section*{Consolidated Funds }



\begin{longtabu} to \linewidth { |  X | X | X | X | }
\hline
\rowcolor{infobg}
Source & Year 1 (14/15) & Year 2 & Year 3\\
\hline
\endhead



FTE Scientist & 70000 & 50000 & 50000\\



FTE Technical & 3500 & 32000 & 21000\\



Equipment & 10000 & 5000 & 5000\\



Vehicle & 5000 & 1000 & 5000\\



Travel & 3000 & 1000 & 4000\\



Other & 12500 & 11000 & 15000\\



Total & 104000 & 100000 & 100000\\


\hline
\end{longtabu}



\section*{External Funds }



\begin{longtabu} to \linewidth { |  X | X | X | X | }
\hline
\rowcolor{infobg}
Source & Year 1 & Year 2 & Year 3\\
\hline
\endhead



Salaries, Wages, Overtime &  &  & \\



Overheads &  &  & \\



Equipment &  &  & \\



Vehicle &  &  & \\



Travel &  &  & \\



Other &  &  & \\



Total &  &  & \\


\hline
\end{longtabu}





%-----------------------------------------------------------------------------%
% Back matter
%\backmatter
\end{document}
%-----------------------------------------------------------------------------%


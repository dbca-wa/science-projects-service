
\documentclass[version=last,
    paper=a4, % paper size
    10pt, % default font size
    usenames,
    dvipsnames,
    oneside, % ONLINE
    headings=openany, % open chapters on odd and even pages
    %toc=chapterentrywithdots, % Table of Contents style
    %BCOR=7mm, % PRINT Binding Correction
    %DIV=13, % typearea 161.54 mm x 228.46 mm, top margin 22.85 mm, inner margin 16.15 mm
    %DIV=14, % 165.00 233.36 21.21 15.00
    DIV=15 % 168.00 237.60 19.80 14.00
]{scrbook}
\usepackage{typearea}
\usepackage[automark,headsepline,footsepline]{scrlayer-scrpage} % Headers and footers

%%
%% Fonts, encoding, spacing, indentation
%%
\usepackage{txfonts}
\renewcommand{\familydefault}{\sfdefault} % Default to Sans Serif font
\usepackage[english]{babel}
\usepackage[T1]{fontenc}
\usepackage[utf8]{inputenc}

% Paragraph spacing
%\usepackage{parskip}    % Paragraph spacing
%\setlength{\parindent}{0em} % Don't indent paragraphs - ONLINE
%\setlength{\parskip}{1.3 ex plus 0.5ex minus 0.3ex} % 1-1.8 ex vertical space between paragraphs - ONLINE

% Spacing of headings
%\RedeclareSectionCommand[afterskip=3pt]{section} % less space after section
%\RedeclareSectionCommand[beforeskip=0cm]{subsection} % less space between HRule and project name
%\RedeclareSectionCommand[afterskip=0.1\baselineskip]{subsubsection} % less space after progressreport subheadings

% Table font size
\usepackage{etoolbox}
\AtBeginEnvironment{longtabu}{\footnotesize}{}{}

%%
%% Tables, columns, layout
%%
\usepackage{multicol}   % 2 col publications
\usepackage{pdflscape}  % Landscape pages
\usepackage{pdfpages}   % Include PDFs
\usepackage{hanging}    % hanging paragraphs for publications
%\usepackage{titletoc}   % Required for manipulating the table of contents
\setcounter{tocdepth}{2} % TOC list down to section
\usepackage{enumerate}  % Enumerations
\usepackage{enumitem}   % Enumerations
\usepackage{longtable}  % Multipage table
\usepackage{tabu}       %
\setlength{\tabulinesep}{1.5mm} % Consistent vertical spacing in tabu

%%
%% Graphics, images, colours
%%
\usepackage{graphicx} % embedded images
\usepackage{eso-pic} %
\usepackage{colortbl} % define custom named colours
\definecolor{RedFire}{RGB}{146,25,28}
\definecolor{ParksWildlife}{RGB}{0,85,144}
\definecolor{successbg}{RGB}{223,240,216}
\definecolor{errorbg}{RGB}{242,222,222}
\definecolor{warningbg}{RGB}{252,248,227}
\definecolor{infobg}{RGB}{217,237,247}
\definecolor{muted}{RGB}{153,153,153}
\definecolor{success}{RGB}{70,136,71}
\definecolor{error}{RGB}{185,74,72}
\definecolor{warning}{RGB}{192,152,83}
\definecolor{info}{RGB}{58,135,173}

\definecolor{required}{RGB}{192,152,83}
\definecolor{requiredbg}{RGB}{252,248,227}
\definecolor{denied}{RGB}{185,74,72}
\definecolor{deniedbg}{RGB}{242,222,222}
\definecolor{granted}{RGB}{70,136,71}
\definecolor{grantedbg}{RGB}{223,240,216}
\definecolor{not reqiured}{RGB}{153,153,153}
\definecolor{not requiredbg}{RGB}{255,255,255}

\usepackage{tikz} % Drawing
\usetikzlibrary{arrows,shapes,positioning,shadows,trees}

%%
%% Links, URLs
%%
\usepackage[
    linktoc=all,
    %colorlinks=false,  %PRINT
    colorlinks=true, % ONLINE
    linkcolor=RedFire, % ONLINE
    urlcolor=ParksWildlife, % ONLINE
    pdftitle=Progress Report SP 2010-011 (FY 2015-2016)
]{hyperref}

% Black magic to linebreak URLs
\usepackage{url}
\makeatletter
\g@addto@macro{\UrlBreaks}{\UrlOrds}
\makeatother

%%
%% Custom macros
%%
% Thick Horizontal rule
\newcommand{\HRule}{\vspace{8mm}\\\noindent\rule{\linewidth}{0.1pt}}

% Custom Tikz node for SDS diagram
\newcommand\mynode[6][]{
    \node[#1] (#2){
        \parbox{#3\relax}{
            \begin{center}
            \textbf{#4}\\
            #5\\
            \footnotesize{#6}
            \end{center}}};}



%-----------------------------------------------------------------------------%
% Headers and Footers
\automark{section}
\ohead{\href{http://sdis.dpaw.wa.gov.au/documents/progressreport/1643/}{Progress Report SP 2010-011
}}
\chead{\href{http://sdis.dpaw.wa.gov.au}{SDIS}} % center header ONLINE
\ihead{\href{http://sdis.dpaw.wa.gov.au}{
        \includegraphics[scale=0.4]{/mnt/projects/sdis/staticfiles/img/logo-dpaw.png}}}
\ifoot{\textbf{Printed}~Tue, 13 Sep 2016 17:33:25 +0800} % inner/left footer
\cfoot{} % center footer
\ofoot{\pagemark} % outer/right footer
\pagestyle{scrheadings}
\setkomafont{pageheadfoot}{\normalfont}

%-----------------------------------------------------------------------------%
\begin{document}
\raggedbottom

%-----------------------------------------------------------------------------%
% Title page
\subject{Progress Report SP 2010-011
}
\title{Fire regimes and impacts in transitional woodlands and shrublands
}
\subtitle{Ecosystem Science
}
\author{}
\publishers{\small
    \subsection*{Project Core Team}
\begin{tabu} {X X}
\textbf{Supervising Scientist} & Colin Yates
\\
\textbf{Data Custodian} & 
\\
\textbf{Site Custodian} & 
\\
\end{tabu}


    \subsection*{Project status as of Sept. 13, 2016, 5:33 p.m.}
\begin{tabu} {X X}
& Approved and active
\\
\end{tabu}

    
\subsection*{Document endorsements and approvals as of Sept. 13, 2016, 5:33 p.m.}
\begin{tabu} {X X}

%\rowcolor{grantedbg}
    \textbf{Project Team} & 
    \textcolor{granted}{ granted}\\

%\rowcolor{grantedbg}
    \textbf{Program Leader} & 
    \textcolor{granted}{ granted}\\

%\rowcolor{grantedbg}
    \textbf{Directorate} & 
    \textcolor{granted}{ granted}\\

\end{tabu}



}
\uppertitleback{}
\lowertitleback{}
\date{}

%-----------------------------------------------------------------------------%
% Front matter
\frontmatter
\maketitle
%-----------------------------------------------------------------------------%
% Main matter
\mainmatter

\section*{Fire regimes and impacts in transitional woodlands and shrublands
}

C Yates, C Gosper


\section*{Context}
The Great Western Woodlands (GWW) is an internationally significant area
with great biological and cultural richness. This 16 million hectare
region of south-western Australia arguably contains the world's largest
and most intact area of contiguous temperate woodland. The GWW
Conservation Strategy and a review conducted by a wide range of
scientists with expertise in the region each identified inappropriate
fire regimes as a threat to the woodlands and emphasised the need for a
science-based fire management regime for the area. Critical gaps in the
knowledge of fire ecology for GWW ecosystems are a major hindrance for
ecological fire management in the region. The GWW supports eucalypt
woodlands at very low mean annual rainfall (250-350 mm). Woodlands
require fire to establish but are very slow growing. In recent decades a
large part of the GWW has been burnt and concern has been expressed over
the ecological impacts of this. Fire ecology research already undertaken
in eastern wheatbelt nature reserves will help resolve ecological fire
management issues for mallee and mallee-heath communities in the GWW,
but similar information for the dominant eucalypt woodlands is urgently
needed.



\section*{Aims}
\begin{itemize}
\itemsep1pt\parskip0pt\parsep0pt
\item
  Develop a method to robustly estimate stand time since fire in gimlet
  (\emph{Eucalyptus salubris}) woodlands that have not been burnt during
  the period covered by remotely-sensed imagery.
\item
  Investigate the effects of time since fire on the assembly and
  recovery of gimlet woodlands, including on plant and animal community
  composition and development of ecosystem structure.
\item
  Measure fuel and carbon dynamics with time since fire in gimlet
  woodland.
\item
  Investigate pathways to weed invasion in the GWW.
\end{itemize}



\section*{Progress}
\begin{itemize}
\itemsep1pt\parskip0pt\parsep0pt
\item
  A multi-century time since fire chronosequence of 76 plots has been
  established in gimlet woodlands, with sampling of plant composition,
  vegetation structure, visual fuel assessment, ants and birds.
\item
  Plant data from the chronosequence have been used in national-scale
  synthesis of: (i) the spatial and vegetation type distribution of fire
  response traits in woody plants (published in \emph{Science of the
  Total Environment}); (ii) fire regime and environmental correlates of
  fire response traits (in preparation); (iii) how Western Australian
  eucalypt woodlands differ in fire ecology from eastern Australian
  analogues (\emph{Journal of Biogeography}); and (iv) the composition,
  biogeography, environmental correlates and ecology of Australia's
  temperate woodlands (in review for the book \emph{Australian
  Vegetation}).
\item
  Alien plant records from the GWW were used to identify environmental
  and disturbance predictors of weed occurrence, and priority weed
  species for preventative weed management in a climate change context
  (published in \emph{Biodiversity and Conservation})
\item
  Gimlet tree size was sampled at 100 plots systematically located
  across the GWW to estimate stand-class structure, and via linking with
  models estimating tree age from tree size, woodland age-class
  distribution.
\item
  Methodology to sample tree and shrub, woody debris and litter carbon
  pools was tested at pilot sites.
\end{itemize}



\section*{Management implications}
\begin{itemize}
\itemsep1pt\parskip0pt\parsep0pt
\item
  National-scale syntheses of temperate eucalypt woodland fire ecology
  revealed that Western Australian woodlands are uniquely dominated by
  taxa that are obligate seeding or recolonise after fire from unburnt
  populations, and have vegetation dynamics driven by rare,
  stand-replacing disturbances. These characteristics illustrative a
  putative vulnerability to decreases in intervals between fires.
\item
  Post-fire succession in plant composition and structure, which in turn
  determines successional patterns in animals, occurs over multi-century
  timescales, demonstrating the value of avoiding fire in mature
  woodlands to maximise future fire management options.
\item
  Changes in vegetation structure, cover and hazard indicate maximum
  gimlet woodland flammability at intermediate times since fire,
  supporting the revision of fire behaviour ratings.
\item
  Contemporary invasive plant spread in the GWW could be reduced via:
  (i) targeting abandoned and current settlements for removal of
  disjunct weed populations; (ii) minimising new settlement creation in
  locations currently remote from towns; and (iii) closing water points
  on conservation estate to reduce disturbance-induced weed recruitment.
  Considering future climate tolerance in weed species prioritisation
  results in a feasibly small selection of taxa for pre-emptive
  regional-scale eradication or containment.
\item
  GWW woodlands have little grass in comparison to other temperate
  woodlands. If perennial grass weeds (such as buffel grass) become
  widely established, potentially facilitated by climate change,
  substantial fire regime shifts and subsequent loss of mature woodlands
  are plausible.
\item
  Knowledge generated through this project has been incorporated into
  eucalypt woodland fire ecology training, to be delivered to Department
  of Parks and Wildlife staff.
\end{itemize}



\section*{Future directions}
\begin{itemize}
\itemsep1pt\parskip0pt\parsep0pt
\item
  Refine models estimating the time since fire of long-unburnt gimlet
  woodlands through the use of the growth ring increment data.
\item
  Use refined age-size models and stand structure data to generate a
  robust age-class distribution of gimlet woodland and hence assess
  whether recent extensive wildfires are unprecedented over the period
  in which existing gimlet stands developed.
\item
  Complete measurement of carbon pools across the gimlet chronosequence
  to determine the role of fire management in carbon sequestration.
\end{itemize}



%-----------------------------------------------------------------------------%
% Back matter
%\backmatter
\end{document}
%-----------------------------------------------------------------------------%


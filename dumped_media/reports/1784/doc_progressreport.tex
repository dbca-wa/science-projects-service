
\documentclass[version=last,
    paper=a4, % paper size
    10pt, % default font size
    usenames,
    dvipsnames,
    oneside, % ONLINE
    headings=openany, % open chapters on odd and even pages
    %toc=chapterentrywithdots, % Table of Contents style
    %BCOR=7mm, % PRINT Binding Correction
    %DIV=13, % typearea 161.54 mm x 228.46 mm, top margin 22.85 mm, inner margin 16.15 mm
    %DIV=14, % 165.00 233.36 21.21 15.00
    DIV=15 % 168.00 237.60 19.80 14.00
]{scrbook}
\usepackage{typearea}
\usepackage[automark,headsepline,footsepline]{scrlayer-scrpage} % Headers and footers

%%
%% Fonts, encoding, spacing, indentation
%%
\usepackage{txfonts}
\renewcommand{\familydefault}{\sfdefault} % Default to Sans Serif font
\usepackage[english]{babel}
\usepackage[T1]{fontenc}
\usepackage[utf8]{inputenc}

% Paragraph spacing
%\usepackage{parskip}    % Paragraph spacing
%\setlength{\parindent}{0em} % Don't indent paragraphs - ONLINE
%\setlength{\parskip}{1.3 ex plus 0.5ex minus 0.3ex} % 1-1.8 ex vertical space between paragraphs - ONLINE

% Spacing of headings
%\RedeclareSectionCommand[afterskip=3pt]{section} % less space after section
%\RedeclareSectionCommand[beforeskip=0cm]{subsection} % less space between HRule and project name
%\RedeclareSectionCommand[afterskip=0.1\baselineskip]{subsubsection} % less space after progressreport subheadings

% Table font size
\usepackage{etoolbox}
\AtBeginEnvironment{longtabu}{\footnotesize}{}{}

%%
%% Tables, columns, layout
%%
\usepackage{multicol}   % 2 col publications
\usepackage{pdflscape}  % Landscape pages
\usepackage{pdfpages}   % Include PDFs
\usepackage{hanging}    % hanging paragraphs for publications
%\usepackage{titletoc}   % Required for manipulating the table of contents
\setcounter{tocdepth}{2} % TOC list down to section
\usepackage{enumerate}  % Enumerations
\usepackage{enumitem}   % Enumerations
\usepackage{longtable}  % Multipage table
\usepackage{tabu}       %
\setlength{\tabulinesep}{1.5mm} % Consistent vertical spacing in tabu

%%
%% Graphics, images, colours
%%
\usepackage{graphicx} % embedded images
\usepackage{eso-pic} %
\usepackage{colortbl} % define custom named colours
\definecolor{RedFire}{RGB}{146,25,28}
\definecolor{ParksWildlife}{RGB}{0,85,144}
\definecolor{successbg}{RGB}{223,240,216}
\definecolor{errorbg}{RGB}{242,222,222}
\definecolor{warningbg}{RGB}{252,248,227}
\definecolor{infobg}{RGB}{217,237,247}
\definecolor{muted}{RGB}{153,153,153}
\definecolor{success}{RGB}{70,136,71}
\definecolor{error}{RGB}{185,74,72}
\definecolor{warning}{RGB}{192,152,83}
\definecolor{info}{RGB}{58,135,173}

\definecolor{required}{RGB}{192,152,83}
\definecolor{requiredbg}{RGB}{252,248,227}
\definecolor{denied}{RGB}{185,74,72}
\definecolor{deniedbg}{RGB}{242,222,222}
\definecolor{granted}{RGB}{70,136,71}
\definecolor{grantedbg}{RGB}{223,240,216}
\definecolor{not reqiured}{RGB}{153,153,153}
\definecolor{not requiredbg}{RGB}{255,255,255}

\usepackage{tikz} % Drawing
\usetikzlibrary{arrows,shapes,positioning,shadows,trees}

%%
%% Links, URLs
%%
\usepackage[
    linktoc=all,
    %colorlinks=false,  %PRINT
    colorlinks=true, % ONLINE
    linkcolor=RedFire, % ONLINE
    urlcolor=ParksWildlife, % ONLINE
    pdftitle=Progress Report SP 2016-015 (FY 2015-2016)
]{hyperref}

% Black magic to linebreak URLs
\usepackage{url}
\makeatletter
\g@addto@macro{\UrlBreaks}{\UrlOrds}
\makeatother

%%
%% Custom macros
%%
% Thick Horizontal rule
\newcommand{\HRule}{\vspace{8mm}\\\noindent\rule{\linewidth}{0.1pt}}

% Custom Tikz node for SDS diagram
\newcommand\mynode[6][]{
    \node[#1] (#2){
        \parbox{#3\relax}{
            \begin{center}
            \textbf{#4}\\
            #5\\
            \footnotesize{#6}
            \end{center}}};}



%-----------------------------------------------------------------------------%
% Headers and Footers
\automark{section}
\ohead{\href{http://sdis.dpaw.wa.gov.au/documents/progressreport/1784/}{Progress Report SP 2016-015
}}
\chead{\href{http://sdis.dpaw.wa.gov.au}{SDIS}} % center header ONLINE
\ihead{\href{http://sdis.dpaw.wa.gov.au}{
        \includegraphics[scale=0.4]{/mnt/projects/sdis/staticfiles/img/logo-dpaw.png}}}
\ifoot{\textbf{Printed}~Wed, 31 Aug 2016 18:24:15 +0800} % inner/left footer
\cfoot{} % center footer
\ofoot{\pagemark} % outer/right footer
\pagestyle{scrheadings}
\setkomafont{pageheadfoot}{\normalfont}

%-----------------------------------------------------------------------------%
\begin{document}
\raggedbottom

%-----------------------------------------------------------------------------%
% Title page
\subject{Progress Report SP 2016-015
}
\title{Is restoration working? An ecological assessment.
}
\subtitle{Plant Science and Herbarium
}
\author{}
\publishers{\small
    \subsection*{Project Core Team}
\begin{tabu} {X X}
\textbf{Supervising Scientist} & Dave Coates
\\
\textbf{Data Custodian} & Melissa A Millar
\\
\textbf{Site Custodian} & 
\\
\end{tabu}


    \subsection*{Project status as of Aug. 31, 2016, 6:24 p.m.}
\begin{tabu} {X X}
& Update requested
\\
\end{tabu}

    
\subsection*{Document endorsements and approvals as of Aug. 31, 2016, 6:24 p.m.}
\begin{tabu} {X X}

%\rowcolor{grantedbg}
    \textbf{Project Team} & 
    \textcolor{granted}{ granted}\\

%\rowcolor{grantedbg}
    \textbf{Program Leader} & 
    \textcolor{granted}{ granted}\\

%\rowcolor{requiredbg}
    \textbf{Directorate} & 
    \textcolor{required}{ required}\\

\end{tabu}



}
\uppertitleback{}
\lowertitleback{}
\date{}

%-----------------------------------------------------------------------------%
% Front matter
\frontmatter
\maketitle
%-----------------------------------------------------------------------------%
% Main matter
\mainmatter

\section*{Is restoration working? An ecological assessment.
}

D Coates, M Byrne, MA Millar, Prof SD Hopper (The University of Western
Australia), Dr S Krauss (Botanic Gardens and Parks Authority)


\section*{Context}
The recognition of poorly defined success criteria and a lack of long
term monitoring have highlighted the need for the development of post
implementation empirical evaluations of the quality of restoration
activities.~ This recognition has led to the hypothesis that the most
ecologically and genetically viable restored populations will be those
where reproductive outputs, plant pollinator interactions, levels of
genetic diversity, mating systems and patterns of pollen dispersal most
closely mimic those found in natural or undisturbed remnant vegetation.
These populations are more likely to persist in the long term and
contribute to effective ecosystem function through integration into the
broader landscape. This project aims to assess the success of
restoration in terms of ecological and genetic viability for plant
species in the Fitzgerald River-Stirling Range region of Western
Australia, where significant investment is being made in restoring
connectivity at a landscape scale.



\section*{Aims}
\begin{itemize}
\itemsep1pt\parskip0pt\parsep0pt
\item
  Evaluate levels of genetic diversity for each of six target species,
  at each of the restoration sites at which they occur and in equivalent
  remnant reference sites. 20 individuals will be sampled. This will
  include the 10 mother plants utilised in the second aim. DNA will be
  extracted from all samples and all individuals will be assessed for
  genetic variation at 12 SSR markers developed for these species.
\item
  Evaluate mating system parameters for each of six target species, at
  each of the restoration sites at which they occur and in equivalent
  remnant reference sites. 10 mother plants will be sampled and seed
  collected from each. DNA will be extracted from all samples and all
  mother plants will be genotyped at 6 SSR loci. 20 progeny from each
  mother plant will be genotyped. Mother and progeny data sets will be
  analysed.
\item
  Evaluate patterns of pollen mediated gene dispersal in two Proteaceous
  species. For each of two Proteaceous species all plants within a given
  area will be sampled as potential father plants (these will be
  inclusive of some individuals sampled for the first aim). Up to 200
  progeny per species will be collected from sampled mother plants. DNA
  will be extracted from all samples and potential fathers and progeny
  genotyped with 12 SSR loci. The spatial position of all mother and
  potential father individuals will be recorded via GPS. Paternity of
  all progeny with known mothers will be analysed.
\item
  Ecological and genetic processes in restoration populations will be
  benchmarked against reference populations of remnant natural
  vegetation.
\end{itemize}



\section*{Progress}
\begin{itemize}
\itemsep1pt\parskip0pt\parsep0pt
\item
  Microsatellite libraries have been constructed for six target species.
\item
  Leaf and seed material has been sampled from \emph{Hakea nitida,~}
  \emph{Hakea laurina,~Melaleuca acuminata,~Eucalyptus occidentalis
  ~}and\emph{~}\emph{Acacia cyclops~}from restored and
  remnant~populations
\item
  DNA has been extracted and primers developed or under development for
  \emph{Hakea nitida,~} \emph{Hakea laurina,~Melaleuca
  acuminata,~Eucalyptus occidentalis ~}and\emph{~}\emph{Acacia cyclops.}
\item
  Analysis of genetic diversity, mating systems and pollination biology
  is underway for \emph{Hakea laurina} in restored and remnant
  populations
\end{itemize}



\section*{Management implications}
This project will provide practical recommendations on how the
ecological and genetic viability of restored populations may be affected
by different establishment regimes.



\section*{Future directions}
\begin{itemize}
\itemsep1pt\parskip0pt\parsep0pt
\item
  Field work collecting leaf and seed material, and mapping individuals
  in four restoration sites will continue.
\item
  Genetic and mating system studies will continue on six target species.
\end{itemize}



%-----------------------------------------------------------------------------%
% Back matter
%\backmatter
\end{document}
%-----------------------------------------------------------------------------%

